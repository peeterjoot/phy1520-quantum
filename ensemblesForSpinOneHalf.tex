%
% Copyright � 2015 Peeter Joot.  All Rights Reserved.
% Licenced as described in the file LICENSE under the root directory of this GIT repository.
%
%\input{../blogpost.tex}
%\renewcommand{\basename}{ensemblesForSpinOneHalf}
%\renewcommand{\dirname}{notes/phy1520/}
%%\newcommand{\dateintitle}{}
%%\newcommand{\keywords}{}
%
%\input{../peeter_prologue_print2.tex}
%
%\usepackage{peeters_layout_exercise}
%\usepackage{peeters_braket}
%\usepackage{peeters_figures}
%
%\beginArtNoToc
%
%\generatetitle{Ensembles for spin one half}
%%\chapter{Ensembles for spin one half}
%%\label{chap:ensemblesForSpinOneHalf}

\makeoproblem{Ensembles for spin one half.}{problem:ensemblesForSpinOneHalf:1}{\citep{sakurai2014modern} pr. 3.10}{
\index{spin half!ensemble averages}

\makesubproblem{}{problem:ensemblesForSpinOneHalf:1:a}
Sakurai leaves it to the reader to verify that knowledge of the three ensemble averages [S_x], [S_y],[S_z] is sufficient to reconstruct the density operator for a spin one half system.  Show this.

\makesubproblem{}{problem:ensemblesForSpinOneHalf:1:b}
Show how the expectation values \( \expectation{S_x}, \expectation{S_y},\expectation{S_x} \) fully determine the spin orientation for a pure ensemble.
} % problem

\makeanswer{problem:ensemblesForSpinOneHalf:1}{
\makeSubAnswer{}{problem:ensemblesForSpinOneHalf:1:a}

I'll do this in two parts, the first using a spin-up/down ensemble to see what form this has, then the general case.  The general case is a bit messy algebraically.  After first attempting it the hard way, I did the grunt work portion of that calculation in Mathematica, but then realized it's not so bad to do it manually.

Consider first an ensemble with \textAndIndex{density operator}
%
\begin{dmath}\label{eqn:ensemblesForSpinOneHalf:20}
\rho =
w_{+} \ket{+}\bra{+} + w_{-} \ket{-}\bra{-},
\end{dmath}

where these are the \( \BS \cdot (\pm \zcap) \) eigenstates.  The traces are
%
\begin{dmath}\label{eqn:ensemblesForSpinOneHalf:40}
\tr( \rho \sigma_x )
=
\bra{+} \rho \sigma_x \ket{+}
+
\bra{-} \rho \sigma_x \ket{-}
=
\bra{+} \rho \PauliX \ket{+}
+
\bra{-} \rho \PauliX \ket{-}
=
\bra{+} \lr{ w_{+} \ket{+}\bra{+} + w_{-} \ket{-}\bra{-} } \ket{-}
+
\bra{-} \lr{ w_{+} \ket{+}\bra{+} + w_{-} \ket{-}\bra{-} } \ket{+}
=
\bra{+} w_{-} \ket{-}
+
\bra{-} w_{+} \ket{+}
=
0,
\end{dmath}
%
\begin{dmath}\label{eqn:ensemblesForSpinOneHalf:60}
\tr( \rho \sigma_y )
=
\bra{+} \rho \sigma_y \ket{+}
+
\bra{-} \rho \sigma_y \ket{-}
=
\bra{+} \rho \PauliY \ket{+}
+
\bra{-} \rho \PauliY \ket{-}
=
i \bra{+} \lr{ w_{+} \ket{+}\bra{+} + w_{-} \ket{-}\bra{-} } \ket{-}
-
i \bra{-} \lr{ w_{+} \ket{+}\bra{+} + w_{-} \ket{-}\bra{-} } \ket{+}
=
i \bra{+} w_{-} \ket{-}
-
i \bra{-} w_{+} \ket{+}
=
0,
\end{dmath}

and
\begin{dmath}\label{eqn:ensemblesForSpinOneHalf:100}
\tr( \rho \sigma_z )
=
\bra{+} \rho \sigma_z \ket{+}
+
\bra{-} \rho \sigma_z \ket{-}
=
\bra{+} \rho \ket{+}
-
\bra{-} \rho \ket{-}
=
 \bra{+} \lr{ w_{+} \ket{+}\bra{+} + w_{-} \ket{-}\bra{-} } \ket{+}
-
 \bra{-} \lr{ w_{+} \ket{+}\bra{+} + w_{-} \ket{-}\bra{-} } \ket{-}
=
 \bra{+} w_{+} \ket{+}
-
 \bra{-} w_{-} \ket{-}
=
w_{+} - w_{-}.
\end{dmath}

Since \( w_{+} + w_{-} = 1 \), this gives
%
\boxedEquation{eqn:ensemblesForSpinOneHalf:80}{
\begin{aligned}
w_{+} &= \frac{1 + \tr( \rho \sigma_z )}{2} \\
w_{-} &= \frac{1 - \tr( \rho \sigma_z )}{2}
\end{aligned}
}

Attempting to do a similar set of trace expansions this way for a more general spin basis turns out to be a really bad idea and horribly messy.  So much so that I resorted to
%\href{https://raw.githubusercontent.com/peeterjoot/mathematica/master/phy1520/spinOneHalfSymbolicManipulation.nb}{Mathematica to do this symbolic work}
\nbref{spinOneHalfSymbolicManipulation.nb} to do this symbolic work.
However, it's not so bad if the trace is done completely in matrix form.

Using the basis
%
\begin{dmath}\label{eqn:ensemblesForSpinOneHalf:120}
\begin{aligned}
\ket{\BS \cdot \ncap ; + } &=
\begin{bmatrix}
\cos(\theta/2) \\
\sin(\theta/2) e^{i \phi}
\end{bmatrix} \\
\ket{\BS \cdot \ncap ; - } &=
\begin{bmatrix}
\sin(\theta/2) e^{-i \phi} \\
-\cos(\theta/2) \\
\end{bmatrix},
\end{aligned}
\end{dmath}

\index{projection operator}
the projector matrices are
%
\begin{dmath}\label{eqn:ensemblesForSpinOneHalf:140}
\ket{\BS \cdot \ncap ; + } \bra{\BS \cdot \ncap ; + }
=
\begin{bmatrix}
\cos(\theta/2) \\
\sin(\theta/2) e^{i \phi}
\end{bmatrix}
\begin{bmatrix}
\cos(\theta/2) &
\sin(\theta/2) e^{-i \phi}
\end{bmatrix}
=
\begin{bmatrix}
\cos^2(\theta/2) & \cos(\theta/2) \sin(\theta/2) e^{-i \phi} \\
\sin(\theta/2) \cos(\theta/2) e^{i \phi} & \sin^2(\theta/2)
\end{bmatrix},
\end{dmath}
\begin{dmath}\label{eqn:ensemblesForSpinOneHalf:160}
\ket{\BS \cdot \ncap ; - } \bra{\BS \cdot \ncap ; - }
=
\begin{bmatrix}
\sin(\theta/2) e^{-i \phi} \\
-\cos(\theta/2) \\
\end{bmatrix}
\begin{bmatrix}
\sin(\theta/2) e^{i \phi} & -\cos(\theta/2) \\
\end{bmatrix}
=
\begin{bmatrix}
\sin^2(\theta/2)  & -\cos(\theta/2) \sin(\theta/2) e^{-i \phi} \\
-\cos(\theta/2) \sin(\theta/2) e^{i \phi} & \cos^2(\theta/2)
\end{bmatrix}
\end{dmath}

With \( C = \cos(\theta/2), S = \sin(\theta/2) \), a general density operator in this basis has the form
%
\begin{dmath}\label{eqn:ensemblesForSpinOneHalf:180}
\rho
=
w_{+}
\begin{bmatrix}
C^2 & C S e^{-i \phi} \\
S C e^{i \phi} & S^2
\end{bmatrix}
+
w_{-}
\begin{bmatrix}
S^2  & -C S e^{-i \phi} \\
-C S e^{i \phi} & C^2
\end{bmatrix}
=
\begin{bmatrix}
w_{+} C^2 + w_{-} S^2  & (w_{+} - w_{-})C S e^{-i \phi} \\
(w_{+} -w_{-} ) S C e^{i \phi} & w_{+} S^2  + w_{-} C^2
\end{bmatrix}.
\end{dmath}

The products with the Pauli matrices are
\index{Pauli matrix}
%
\begin{dmath}\label{eqn:ensemblesForSpinOneHalf:200}
\rho \sigma_x
=
\begin{bmatrix}
w_{+} C^2 + w_{-} S^2  & (w_{+} - w_{-})C S e^{-i \phi} \\
(w_{+} -w_{-} ) S C e^{i \phi} & w_{+} S^2  + w_{-} C^2
\end{bmatrix}
\PauliX
=
\begin{bmatrix}
(w_{+} - w_{-})C S e^{-i \phi}  & w_{+} C^2 + w_{-} S^2  \\
w_{+} S^2  + w_{-} C^2 & (w_{+} -w_{-} ) S C e^{i \phi} \\
\end{bmatrix}
\end{dmath}
%
\begin{dmath}\label{eqn:ensemblesForSpinOneHalf:220}
\rho \sigma_y
=
\begin{bmatrix}
w_{+} C^2 + w_{-} S^2  & (w_{+} - w_{-})C S e^{-i \phi} \\
(w_{+} -w_{-} ) S C e^{i \phi} & w_{+} S^2  + w_{-} C^2
\end{bmatrix}
\PauliY
=
i
\begin{bmatrix}
(w_{+} - w_{-})C S e^{-i \phi}  & -w_{+} C^2 - w_{-} S^2  \\
w_{+} S^2  + w_{-} C^2          & -(w_{+} -w_{-} ) S C e^{i \phi} \\
\end{bmatrix}
\end{dmath}
%
\begin{dmath}\label{eqn:ensemblesForSpinOneHalf:240}
\rho \sigma_z
=
\begin{bmatrix}
w_{+} C^2 + w_{-} S^2  & (w_{+} - w_{-})C S e^{-i \phi} \\
(w_{+} -w_{-} ) S C e^{i \phi} & w_{+} S^2  + w_{-} C^2
\end{bmatrix}
\PauliZ
=
\begin{bmatrix}
w_{+} C^2 + w_{-} S^2  & -(w_{+} - w_{-})C S e^{-i \phi} \\
(w_{+} -w_{-} ) S C e^{i \phi} & - (w_{+} S^2  + w_{-} C^2)
\end{bmatrix}
\end{dmath}

The respective traces can be read right off the matrices
\begin{dmath}\label{eqn:ensemblesForSpinOneHalf:260}
\begin{aligned}
\tr( \rho \sigma_x ) &= (w_{+} - w_{-}) \sin\theta \cos\phi \\
\tr( \rho \sigma_y ) &= (w_{+} - w_{-}) \sin\theta \sin\phi \\
\tr( \rho \sigma_z ) &= (w_{+} - w_{-}) \cos\theta \\
\end{aligned}.
\end{dmath}

This gives
%
\begin{dmath}\label{eqn:ensemblesForSpinOneHalf:280}
(w_{+} - w_{-}) \ncap = \lr{ \tr( \rho \sigma_x ), \tr( \rho \sigma_y ), \tr( \rho \sigma_z ) },
\end{dmath}

or
%
\boxedEquation{eqn:ensemblesForSpinOneHalf:320}{
w_{\pm} = \frac{1 \pm \sqrt{ \tr^2( \rho \sigma_x ) + \tr^2( \rho \sigma_y ) + \tr^2( \rho \sigma_z )} }{2} .
}

So, as claimed, it's possible to completely describe the ensemble weight factors using the ensemble averages of \( [S_x], [S_y], [S_z] \).  I used the Pauli matrices instead, but the difference is just an \( \Hbar/2 \) scaling adjustment.

\paragraph{Alternate approach}

Another easier and trig free way to look at this problem is assume the density operator's representation is given by a \( 2 \times 2 \) matrix with undetermined values
%
\begin{dmath}\label{eqn:ensemblesForSpinOneHalf:400}
\rho =
\begin{bmatrix}
a & b \\
c & d
\end{bmatrix}
\end{dmath}

For such a representation we have
%
\begin{equation}\label{eqn:ensemblesForSpinOneHalf:420}
\begin{aligned}
\rho \sigma_x
&=
\begin{bmatrix}
a & b \\
c & d
\end{bmatrix}
\PauliX
=
\begin{bmatrix}
b & a \\
d & c
\end{bmatrix} \\
\rho \sigma_y
&=
\begin{bmatrix}
a & b \\
c & d
\end{bmatrix}
\PauliY
=
i
\begin{bmatrix}
b & -a \\
d & -c
\end{bmatrix} \\
\rho \sigma_z
&=
\begin{bmatrix}
a & b \\
c & d
\end{bmatrix}
\PauliZ
=
\begin{bmatrix}
a & -b\\
c & -d
\end{bmatrix} \\
\end{aligned}
\end{equation}

The ensemble averages can be read by inspection
\begin{equation}\label{eqn:ensemblesForSpinOneHalf:440}
\begin{aligned}
[\sigma_x] &= b + c \\
[\sigma_y] &= i(b - c) \\
[\sigma_z] &= a - d \\
\end{aligned}
\end{equation}

Noting that \( \tr \lr{E A E^{-1}} = \tr \lr{A E^{-1} E} = \tr \lr{A} \), and that there must be a diagonal basis for which \( \braket{+}{-} = 0 \) and
%
\begin{dmath}\label{eqn:ensemblesForSpinOneHalf:460}
\rho = w_{+} \ket{+}\bra{+} + w_{-} \ket{-}\bra{-},
\end{dmath}

we must have
%
\begin{dmath}\label{eqn:ensemblesForSpinOneHalf:480}
\trace{\rho} = a + d = w_{+} + w_{-} = 1.
\end{dmath}

This provides one set of equations for each of \( b,c \) and \( a, d\)
%
\begin{equation}\label{eqn:ensemblesForSpinOneHalf:500}
\begin{aligned}
[\sigma_x] &= b + c \\
[\sigma_y] &= i(b - c),
\end{aligned}
\end{equation}

and
\begin{equation}\label{eqn:ensemblesForSpinOneHalf:520}
\begin{aligned}
[\sigma_z] &= a - d \\
1          &= a + d.
\end{aligned}
\end{equation}

These have solutions
\begin{equation}\label{eqn:ensemblesForSpinOneHalf:540}
\begin{aligned}
b &= \frac{[\sigma_x] - i [\sigma_y]}{2} \\
c &= \frac{[\sigma_x] + i [\sigma_y]}{2} \\
a &= \frac{1 + [\sigma_z]}{2} \\
d &= \frac{1 - [\sigma_z]}{2},
\end{aligned}
\end{equation}

or
%
\boxedEquation{eqn:ensemblesForSpinOneHalf:560}{
\rho = \inv{2}
\begin{bmatrix}
1 + [\sigma_z] & [\sigma_x] - i [\sigma_y] \\
[\sigma_x] + i [\sigma_y] & 1 - [\sigma_z]
\end{bmatrix}.
}

The characteristic equation for this operator is
%
\begin{equation}\label{eqn:ensemblesForSpinOneHalf:600}
0 =
\lr{ \lr{ \inv{2} - \lambda } + \frac{[\sigma_z] }{2} }
\lr{ \lr{ \inv{2} - \lambda } - \frac{[\sigma_z] }{2} }
- \inv{4} \lr{ [\sigma_x] + i [\sigma_y] } \lr{ [\sigma_x] - i [\sigma_y] },
\end{equation}

or
\boxedEquation{eqn:ensemblesForSpinOneHalf:580}{
\lambda = \frac{1 \pm \sqrt{ [\sigma_x]^2 + [\sigma_y]^2 + [\sigma_z]^2 }}{2},
}

as found above.

\makeSubAnswer{}{problem:ensemblesForSpinOneHalf:1:b}

Suppose that the system is in the state \( \ket{\BS \cdot \ncap ; + } \) as defined in \cref{eqn:ensemblesForSpinOneHalf:120}, then the expectation values of \( \sigma_x, \sigma_y, \sigma_z \) with respect to this state are
%
\begin{dmath}\label{eqn:ensemblesForSpinOneHalf:300}
\expectation{\sigma_x}
=
\begin{bmatrix}
\cos(\theta/2) &
\sin(\theta/2) e^{-i \phi}
\end{bmatrix}
\PauliX
\begin{bmatrix}
\cos(\theta/2) \\
\sin(\theta/2) e^{i \phi}
\end{bmatrix}
=
\begin{bmatrix}
\cos(\theta/2) &
\sin(\theta/2) e^{-i \phi}
\end{bmatrix}
\begin{bmatrix}
\sin(\theta/2) e^{i \phi} \\
\cos(\theta/2) \\
\end{bmatrix}
=
\sin\theta \cos\phi,
\end{dmath}
\begin{dmath}\label{eqn:ensemblesForSpinOneHalf:340}
\expectation{\sigma_y}
=
\begin{bmatrix}
\cos(\theta/2) &
\sin(\theta/2) e^{-i \phi}
\end{bmatrix}
\PauliY
\begin{bmatrix}
\cos(\theta/2) \\
\sin(\theta/2) e^{i \phi}
\end{bmatrix}
=
i
\begin{bmatrix}
\cos(\theta/2) &
\sin(\theta/2) e^{-i \phi}
\end{bmatrix}
\begin{bmatrix}
-\sin(\theta/2) e^{i \phi} \\
\cos(\theta/2) \\
\end{bmatrix}
=
\sin\theta \sin\phi,
\end{dmath}
\begin{dmath}\label{eqn:ensemblesForSpinOneHalf:360}
\expectation{\sigma_z}
=
\begin{bmatrix}
\cos(\theta/2) &
\sin(\theta/2) e^{-i \phi}
\end{bmatrix}
\PauliZ
\begin{bmatrix}
\cos(\theta/2) \\
\sin(\theta/2) e^{i \phi}
\end{bmatrix}
=
\begin{bmatrix}
\cos(\theta/2) &
\sin(\theta/2) e^{-i \phi}
\end{bmatrix}
\begin{bmatrix}
\cos(\theta/2) \\
-\sin(\theta/2) e^{i \phi}
\end{bmatrix}
=
\cos\theta.
\end{dmath}

So we have
\boxedEquation{eqn:ensemblesForSpinOneHalf:380}{
\ncap = \lr{ \expectation{\sigma_x}, \expectation{\sigma_y}, \expectation{\sigma_z} }.
}

\index{spin half!expectation}
The spin direction is completely determined by this vector of expectation values (or equivalently, the expectation values of \( S_x, S_y, S_z \)).
} % answer

%\EndArticle
