%
% Copyright � 2015 Peeter Joot.  All Rights Reserved.
% Licenced as described in the file LICENSE under the root directory of this GIT repository.
%%
%\input{../blogpost.tex}
%\renewcommand{\basename}{qmLecture11}
%\renewcommand{\dirname}{notes/phy1520/}
%\newcommand{\keywords}{PHY1520H}
%\input{../peeter_prologue_print2.tex}
%
%%\usepackage{phy1520}
%\usepackage{peeters_braket}
%\usepackage{peeters_layout_exercise}
%\usepackage{peeters_figures}
%\usepackage{mathtools}
%
%\beginArtNoToc
%\generatetitle{PHY1520H Graduate Quantum Mechanics.  Lecture 11: Symmetries in QM.  Taught by Prof.\ Arun Paramekanti}
%%\chapter{Symmetries in QM}
%\label{chap:qmLecture11}
%
%\paragraph{Disclaimer}
%
%Peeter's lecture notes from class.  These may be incoherent and rough.
%
%These are notes for the UofT course PHY1520, Graduate Quantum Mechanics, taught by Prof. Paramekanti, covering \textchapref{{4}} \citep{sakurai2014modern} content.
%
\section{Symmetry in classical mechanics.}
\index{symmetry!classical}

In a classical context considering a Hamiltonian
%
\begin{dmath}\label{eqn:qmLecture11:20}
H(q_i, p_i),
\end{dmath}
%
a symmetry means that certain \( q_i \) don't appear.  In that case the rate of change of one of the generalized momenta is zero
%
\begin{equation}\label{eqn:qmLecture11:40}
\ddt{p_k} = - \PD{q_k}{H} = 0,
\end{equation}
%
so \( p_k \) is a constant of motion.  This simplifies the problem by reducing the number of degrees of freedom.  Another aspect of such a symmetry is that it \underline{relates trajectories}.  For example, assuming a rotational symmetry as in \cref{fig:lecture11:lecture11Fig1}.

\imageFigure{../figures/phy1520-quantum/lecture11Fig1}{Trajectory under rotational symmetry.}{fig:lecture11:lecture11Fig1}{0.3}

the trajectory of a particle after rotation is related by rotation to the trajectory of the unrotated particle.

\section{Symmetry in quantum mechanics.}

Suppose that we have a symmetry operation that takes states from
%
\begin{dmath}\label{eqn:qmLecture11:60}
\ket{\psi} \rightarrow \ket{U \psi}
\end{dmath}
\begin{dmath}\label{eqn:qmLecture11:80}
\ket{\phi} \rightarrow \ket{U \phi},
\end{dmath}
%
we expect that
%
\begin{dmath}\label{eqn:qmLecture11:100}
\Abs{\braket{ \psi}{\phi} }^2 = \Abs{\braket{ U\psi}{ U\phi} }^2.
\end{dmath}
%
This won't hold true for a general operator.   Two cases where this does hold true is when

\begin{itemize}
\item \( \braket{\psi}{\phi} = \braket{ U\psi}{ U\phi} \).  Here \( U \) is \textAndIndex{unitary}, and the equivalence follows from
%
\begin{dmath}\label{eqn:qmLecture11:120}
\braket{ U\psi}{ U\phi} = \bra{ \psi} U^\dagger U { \phi} = \bra{ \psi} 1 { \phi} = \braket{\psi}{\phi}.
\end{dmath}
%
\item \( \braket{\psi}{\phi} = \braket{ U\psi}{ U\phi}^\conj \).  Here \( U \) is \textAndIndex{anti-unitary}.

% FIXME: This second sort of symmetry will be useful in translation operations?
\end{itemize}

\paragraph{Unitary case}
\index{symmetry!unitary}

If an ``observable'' is not changed by a unitary operation representing a symmetry we must have
%
\begin{dmath}\label{eqn:qmLecture11:140}
\bra{\psi} \hatA \ket{\psi}
\rightarrow
\bra{U \psi} \hatA \ket{U \psi}
=
\bra{\psi} U^\dagger \hatA U \ket{\psi},
\end{dmath}
%
so
\begin{dmath}\label{eqn:qmLecture11:160}
U^\dagger \hatA U = \hatA,
\end{dmath}
%
or
\boxedEquation{eqn:qmLecture11:180}{
\hatA U  = U \hatA.
}

An observable that is unchanged by a unitary symmetry commutes \( \antisymmetric{\hatA}{U} \) with the operator \( U \) for that transformation.

\paragraph{Symmetries of the Hamiltonian}

Given
\begin{dmath}\label{eqn:qmLecture11:200}
\antisymmetric{H}{U} = 0,
\end{dmath}
%
\( H \) is invariant.

Given
%
\begin{dmath}\label{eqn:qmLecture11:220}
H \ket{\phi_n} = \epsilon_n \ket{\phi_n} .
\end{dmath}
%
\begin{dmath}\label{eqn:qmLecture11:240}
U H \ket{\phi_n}
= H U \ket{\phi_n}
= \epsilon_n U \ket{\phi_n}
\end{dmath}

Such a state
%
\begin{dmath}\label{eqn:qmLecture11:260}
\ket{\psi_n} = U \ket{\phi_n}
\end{dmath}

is also an eigenstate with the \underline{same} energy.

Suppose this process is repeated, finding other states
%
\begin{dmath}\label{eqn:qmLecture11:280}
U \ket{\psi_n} = \ket{\chi_n}
\end{dmath}
\begin{dmath}\label{eqn:qmLecture11:300}
U \ket{\chi_n} = \ket{\alpha_n}
\end{dmath}

Because such a transformation only generates states with the initial energy, this process cannot continue forever.  At some point this process will enumerate a fixed size set of states.  These states can be orthonormalized.

We can say that symmetry operations are generators of a \underlineAndIndex{group}.  For a set of symmetry operations we can

\index{symmetry!group}
\begin{itemize}
\item
Form products that lie in a closed set
%
\begin{dmath}\label{eqn:qmLecture11:320}
U_1 U_2 = U_3
\end{dmath}
\item can define an inverse
\begin{dmath}\label{eqn:qmLecture11:340}
U \leftrightarrow U^{-1}.
\end{dmath}
\item obeys associative rules for multiplication
\begin{dmath}\label{eqn:qmLecture11:360}
U_1 ( U_2 U_3 ) = (U_1 U_2) U_3.
\end{dmath}
\item has an identity operation.
\end{itemize}

When \( H \) has a symmetry, then degenerate eigenstates form \underlineAndIndex{irreducible} representations (which cannot be further block diagonalized).

\paragraph{Parity symmetry}
\index{parity}

\makeexample{Inversion.}{example:qmLecture11:1}{

Given a state and a parity operation \( \hat\Pi \), with the transformation
%
\begin{dmath}\label{eqn:qmLecture11:380}
\ket{\psi} \rightarrow \hat\Pi \ket{\psi}
\end{dmath}

In one dimension, the parity operation is just inversion.  In two dimensions, this is a set of flipping operations on two axes \cref{fig:lecture11:lecture11Fig2}.

\imageFigure{../figures/phy1520-quantum/lecture11Fig2}{2D parity operation.}{fig:lecture11:lecture11Fig2}{0.3}

The operational effects of this operator are
%
\begin{equation}\label{eqn:qmLecture11:400}
\begin{aligned}
\hatx &\rightarrow - \hatx \\
\hatp &\rightarrow - \hatp.
\end{aligned}
\end{equation}
%
Acting again with the \textAndIndex{parity operator} produces the original value, so it is its own inverse, and \( \hat\Pi^\dagger = \hat\Pi = \hat\Pi^{-1} \).  In an expectation value
%
\begin{dmath}\label{eqn:qmLecture11:420}
\bra{ \hat\Pi \psi } \hatx \ket{ \hat\Pi \psi } = - \bra{\psi} \hatx \ket{\psi}.
\end{dmath}
%
This means that
%
\begin{dmath}\label{eqn:qmLecture11:440}
\hat\Pi^\dagger \hatx \hat\Pi = - \hatx,
\end{dmath}
%
or
\begin{dmath}\label{eqn:qmLecture11:460}
\hatx \hat\Pi = - \hat\Pi \hatx,
\end{dmath}
%
%FIXME: show that \( \hat\Pi^\dagger \hatp \hat\Pi = - \hatp \).
\begin{dmath}\label{eqn:qmLecture11:480}
\hatx \hat\Pi \ket{x_0}
= - \hat\Pi \hatx  \ket{x_0}
= - \hat\Pi x_0 \ket{x_0}
= - x_0 \hat\Pi \ket{x_0}
\end{dmath}

so
%
\begin{dmath}\label{eqn:qmLecture11:500}
\hat\Pi \ket{x_0} = \ket{-x_0}.
\end{dmath}
%
Acting on a wave function
%
\begin{dmath}\label{eqn:qmLecture11:520}
\bra{x} \hat\Pi \ket{\psi}
=
\braket{-x}{\psi}
= \psi(-x).
\end{dmath}
%
What does this mean for eigenfunctions.  Eigenfunctions are supposed to form irreducible representations of the group.  The group has just two elements
%
\begin{dmath}\label{eqn:qmLecture11:540}
\setlr{ 1, \hat\Pi },
\end{dmath}
%
where \( \hat\Pi^2 = 1 \).

Suppose we have a Hamiltonian
%
\begin{dmath}\label{eqn:qmLecture11:560}
H = \frac{\hatp^2}{2m} + V(\hatx),
\end{dmath}
%
where \( V(\hatx) \) is even, or \( \antisymmetric{V(\hatx)}{\hat\Pi } = 0 \).  The squared momentum commutes with the parity operator
%
\begin{dmath}\label{eqn:qmLecture11:580}
\antisymmetric{\hatp^2}{\hat\Pi}
=
\hatp^2 \hat\Pi
- \hat\Pi \hatp^2
=
\hatp^2 \hat\Pi
- (\hat\Pi \hatp) \hatp
=
\hatp^2 \hat\Pi
-(- \hatp \hat\Pi) \hatp
=
\hatp^2 \hat\Pi
+ \hatp (-\hatp \hat\Pi)
=
0.
\end{dmath}
%
Only two functions are possible in the symmetry set \( \setlr{ \Psi(x), \hat\Pi \Psi(x) } \), since
%
\begin{dmath}\label{eqn:qmLecture11:600}
\hat\Pi^2 \Psi(x)
= \hat\Pi \Psi(-x)
= \Psi(x).
\end{dmath}
%
This symmetry severely restricts the possible solutions, making it so there can be only one dimensional forms of this problem with solutions that are either even or odd respectively
%
\begin{dmath}\label{eqn:qmLecture11:620}
\begin{aligned}
\phi_e(x) &= \psi(x ) + \psi(-x) \\
\phi_o(x) &= \psi(x ) - \psi(-x).
\end{aligned}
\end{dmath}
%
} % example

%\EndArticle
