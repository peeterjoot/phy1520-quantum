%
% Copyright � 2015 Peeter Joot.  All Rights Reserved.
% Licenced as described in the file LICENSE under the root directory of this GIT repository.
%
%\input{../blogpost.tex}
%\renewcommand{\basename}{qmLecture19}
%\renewcommand{\dirname}{notes/phy1520/}
%\newcommand{\keywords}{PHY1520H}
%\input{../peeter_prologue_print2.tex}
%
%%\usepackage{phy1520}
%\usepackage{peeters_braket}
%%\usepackage{peeters_layout_exercise}
%\usepackage{peeters_figures}
%\usepackage{mathtools}
%
%\beginArtNoToc
%\generatetitle{PHY1520H Graduate Quantum Mechanics.  Lecture 19: Variational method.  Taught by Prof.\ Arun Paramekanti}
%%\chapter{Variational method}
%\label{chap:qmLecture19}
%
%\paragraph{Disclaimer}
%
%Peeter's lecture notes from class.  These may be incoherent and rough.
%
%These are notes for the UofT course PHY1520, Graduate Quantum Mechanics, taught by Prof. Paramekanti, covering \textchapref{{5}} \citep{sakurai2014modern} content.
%
\section{Variational method.}
\index{variational method}

Today we want to use the variational degree of freedom to try to solve some problems that we don't have analytic solutions for.

\paragraph{Anharmonic oscillator}
\index{anharmonic oscillator}
%
\begin{equation}\label{eqn:qmLecture19:20}
V(x) = \inv{2} m \omega^2 x^2 + \lambda x^4, \qquad \lambda \ge 0.
\end{equation}
%
With the potential growing faster than the harmonic oscillator, which had a ground state solution
%
\begin{equation}\label{eqn:qmLecture19:40}
\psi(x) = \inv{\pi^{1/4}} \inv{a_0^{1/2} } e^{- x^2/2 a_0^2},
\end{equation}
%
where
\begin{equation}\label{eqn:qmLecture19:60}
a_0 = \sqrt{\frac{\Hbar}{m \omega}}.
\end{equation}
%
Let's try allowing \( a_0 \rightarrow a \), to be a variational degree of freedom
%
\begin{equation}\label{eqn:qmLecture19:80}
\psi_a(x) = \inv{\pi^{1/4}} \inv{a^{1/2} } e^{- x^2/2 a^2},
\end{equation}
%
\begin{equation}\label{eqn:qmLecture19:100}
\bra{\psi_a} H \ket{\psi_a}
=
\bra{\psi_a} \frac{p^2}{2m} + \inv{2} m \omega^2 x^2 + \lambda x^4 \ket{\psi_a}.
\end{equation}
We can find
\begin{equation}\label{eqn:qmLecture19:120}
\expectation{x^2} = \inv{2} a^2,
\end{equation}
\begin{equation}\label{eqn:qmLecture19:140}
\expectation{x^4} = \frac{3}{4} a^4.
\end{equation}
Define
%
\begin{equation}\label{eqn:qmLecture19:160}
\tilde{\omega} = \frac{\Hbar}{m a^2},
\end{equation}
%
so that
%
\begin{equation}\label{eqn:qmLecture19:180}
\begin{aligned}
\overbar{E}_a
&=
\bra{\psi_a} \lr{ \frac{p^2}{2m} + \inv{2} m \tilde{\omega}^2 x^2 }
+ \lr{
\inv{2} m \lr{ \omega^2 - \tilde{\omega}^2 } x^2
+
\lambda x^4 }
\ket{\psi_a}
\\ &=
\inv{2} \Hbar \tilde{\omega} + \inv{2} m  \lr{ \omega^2 - \tilde{\omega}^2 } \inv{2} a^2 + \frac{3}{4} \lambda a^4.
\end{aligned}
\end{equation}
%
Write this as
\begin{equation}\label{eqn:qmLecture19:200}
\overbar{E}_{\tilde{\omega}}
=
\inv{2} \Hbar \tilde{\omega} + \inv{4} \frac{\Hbar}{\tilde{\omega}} \lr{ \omega^2 - \tilde{\omega}^2 } + \frac{3}{4} \lambda \frac{\Hbar^2}{m^2 \tilde{\omega}^2 }.
\end{equation}
%
This might look something like \cref{fig:lecture19:lecture19Fig1}.
\imageFigure{../figures/phy1520-quantum/lecture19Fig1a}{Energy after perturbation.}{fig:lecture19:lecture19Fig1}{0.2}
Demand that
%
\begin{equation}\label{eqn:qmLecture19:220}
\begin{aligned}
0
&= \PD{\tilde{\omega}}{ \overbar{E}_{\tilde{\omega}}}
\\ &=
\frac{\Hbar}{2} - \frac{\Hbar}{4} \frac{\omega^2}{\tilde{\omega}^2}
- \frac{\Hbar}{4}
+ \frac{3}{4} (-2) \frac{\lambda \Hbar^2}{m^2 \tilde{\omega}^3}
\\ &=
\frac{\Hbar}{4}
\lr{
1 - \frac{\omega^2}{\tilde{\omega}^2}
- 6 \frac{\lambda \Hbar}{m^2 \tilde{\omega}^3}
},
\end{aligned}
\end{equation}
or
\begin{equation}\label{eqn:qmLecture19:260}
\tilde{\omega}^3 - \omega^2 \tilde{\omega} - \frac{6 \lambda \Hbar}{m^2} = 0.
\end{equation}
%
for \( \lambda a_0^4 \ll \Hbar \omega \), we have something like \( \tilde{\omega} = \omega + \epsilon \).  Expanding \cref{eqn:qmLecture19:260} to first order in \( \epsilon \), this gives
%
\begin{equation}\label{eqn:qmLecture19:280}
\omega^3 + 3 \omega^2 \epsilon - \omega^2 \lr{ \omega + \epsilon } - \frac{6 \lambda \Hbar}{m^2} = 0,
\end{equation}
%
so that
%
\begin{equation}\label{eqn:qmLecture19:300}
2 \omega^2 \epsilon = \frac{6 \lambda \Hbar}{m^2},
\end{equation}
%
and
%
\begin{equation}\label{eqn:qmLecture19:320}
\Hbar \epsilon = \frac{ 3 \lambda \Hbar^2}{m^2 \omega^2 } = 3 \lambda a_0^4.
\end{equation}
%
Plugging into
%
\begin{equation}\label{eqn:qmLecture19:340}
\begin{aligned}
\overbar{E}_{\omega + \epsilon}
&=
\inv{2} \Hbar \lr{ \omega + \epsilon }
+ \inv{4} \frac{\Hbar}{\omega} \lr{ -2 \omega \epsilon + \epsilon^2 } + \frac{3}{4} \lambda \frac{\Hbar^2}{m^2 \omega^2 }
\approx
\inv{2} \Hbar \lr{ \omega + \epsilon }
- \inv{2} \Hbar \epsilon
+ \frac{3}{4} \lambda \frac{\Hbar^2}{m^2 \omega^2 }
\\ &=
\inv{2} \Hbar \omega + \frac{3}{4} \lambda a_0^4.
\end{aligned}
\end{equation}
%
With \cref{eqn:qmLecture19:320}, that is
%
\begin{equation}\label{eqn:qmLecture19:540}
\overbar{E}_{\tilde{\omega} = \omega + \epsilon} \approx \inv{2} \Hbar \lr{ \omega + \frac{\epsilon}{2} }.
\end{equation}
%
The energy levels are shifted slightly for each shift in the Hamiltonian frequency.

What do we have in the extreme anharmonic limit, where \( \lambda a_0^4 \gg \Hbar \omega \)?  Now we get
%
\begin{equation}\label{eqn:qmLecture19:360}
\tilde{\omega}^\conj = \lr{ \frac{ 6 \Hbar \lambda }{m^2} }^{1/3},
\end{equation}
%
and
\begin{equation}\label{eqn:qmLecture19:380}
\overbar{E}_{\tilde{\omega}^\conj} = \frac{\Hbar^{4/3} \lambda^{1/3}}{m^{2/3}} \frac{3}{8} 6^{1/3}.
\end{equation}
%
(this last result is pulled from a web treatment somewhere of the anharmonic oscillator).  Note that the first factor in this energy, with \( \Hbar^4 \lambda/m^2 \) travelling together could have been worked out on dimensional grounds.

This variational method tends to work quite well in these limits.  For a system where \( m = \omega = \Hbar = 1 \), for this problem, we have

\captionedTable{Comparing numeric and variational solutions}{tab:1}{
\begin{tabular}{|l|l|l|}
\hline
\(\Hbar/\omega\) & numeric & variational \\ \hline
100 & 3.13 & 3.16 \\ \hline
1000 & 6.69  & 6.81 \\ \hline
\end{tabular}
}

\paragraph{Example: (sketch) double well potential}
\index{double well potential}

%\cref{fig:lecture19:lecture19Fig2}.
\imageFigure{../figures/phy1520-quantum/lecture19Fig2}{Double well potential.}{fig:lecture19:lecture19Fig2}{0.15}
%
\begin{equation}\label{eqn:qmLecture19:400}
V(x) = \frac{m \omega^2}{8 a^2} \lr{ x - a }^2\lr{ x + a}^2.
\end{equation}
%
Note that this potential, and the Hamiltonian, both commute with parity.
We are interested in the regime where \( a_0^2 = \frac{\Hbar}{m \omega} \ll a^2 \).
Near \( x = \pm a \), this will be approximately
%
\begin{equation}\label{eqn:qmLecture19:420}
V(x) = \inv{2} m \omega^2 \lr{ x \pm a }^2.
\end{equation}
%
Guessing a wave function that is an eigenstate of parity
%
\begin{equation}\label{eqn:qmLecture19:440}
\Psi_{\pm} = g_{\pm} \lr{ \phi_\txtR(x) \pm \phi_\txtL(x) }.
\end{equation}
%
Perhaps this looks like the even and odd functions sketched in \cref{fig:lecture19:lecture19Fig3}, and \cref{fig:lecture19:lecture19Fig4}.
\imageFigure{../figures/phy1520-quantum/lecture19Fig3}{Even double well function.}{fig:lecture19:lecture19Fig3}{0.2}
\imageFigure{../figures/phy1520-quantum/lecture19Fig4}{Odd double well function.}{fig:lecture19:lecture19Fig4}{0.1}
Using harmonic oscillator functions
%
\begin{equation}\label{eqn:qmLecture19:460}
\begin{aligned}
\phi_\txtL(x) &= \Psi_{\txtH.\txtO.}(x + a), \\
\phi_\txtR(x) &= \Psi_{\txtH.\txtO.}(x - a).
\end{aligned}
\end{equation}
After doing a lot of integral (i.e. in the problem set), we will see a splitting of the variational energy levels as sketched in \cref{fig:lecture19:lecture19Fig5}.
\imageFigure{../figures/phy1520-quantum/lecture19Fig5}{Splitting for double well potential.}{fig:lecture19:lecture19Fig5}{0.1}
This sort of level splitting was what was used in the very first mazers.
\section{Perturbation theory (outline).}
\index{perturbation}
Given
%
\begin{equation}\label{eqn:qmLecture19:480}
H = H_0 + \lambda V,
\end{equation}
%
where \( \lambda V \) is ``small''.  We want to figure out the eigenvalues and eigenstates of this Hamiltonian
%
\begin{equation}\label{eqn:qmLecture19:500}
H \ket{n} = E_n \ket{n}.
\end{equation}
%
We don't know what these are, but do know that
%
\begin{equation}\label{eqn:qmLecture19:520}
H_0 \ket{n^{(0)}} = E_n^{(0)} \ket{n^{(0)}}.
\end{equation}
%
We are hoping that the level transitions have adiabatic transitions between the original and perturbed levels as sketched in \cref{fig:lecture19:lecture19Fig6}.
\imageFigure{../figures/phy1520-quantum/lecture19Fig6}{Adiabatic transitions.}{fig:lecture19:lecture19Fig6}{0.1}
and not crossed level transitions as sketched in \cref{fig:lecture19:lecture19Fig7}.
\imageFigure{../figures/phy1520-quantum/lecture19Fig7}{Crossed level transitions.}{fig:lecture19:lecture19Fig7}{0.1}
If we have level crossings (which can in general occur), as opposed to adiabatic transitions, then we have no hope of using perturbation theory.
%\EndArticle
