%
% Copyright � 2015 Peeter Joot.  All Rights Reserved.
% Licenced as described in the file LICENSE under the root directory of this GIT repository.
%
%\input{../blogpost.tex}
%\renewcommand{\basename}{symmetricPotentialDerivativeExpectation}
%\renewcommand{\dirname}{notes/phy1520/}
%%\newcommand{\dateintitle}{}
%%\newcommand{\keywords}{}
%
%\input{../peeter_prologue_print2.tex}
%
%\usepackage{peeters_layout_exercise}
%\usepackage{peeters_braket}
%\usepackage{peeters_figures}
%
%\beginArtNoToc
%
%\generatetitle{Expectation of spherically symmetric 3D potential derivative}
%%\chapter{Expectation of spherically symmetric 3D potential derivative}
%%\label{chap:symmetricPotentialDerivativeExpectation}

\makeoproblem{Expectation of spherically symmetric 3D potential derivative.}{problem:symmetricPotentialDerivativeExpectation:1}{\citep{sakurai2014modern} pr. 5.16}{
\index{spherically symmetric potential!derivative}

\makesubproblem{}{problem:symmetricPotentialDerivativeExpectation:1:a}
For a particle in a spherically symmetric potential \( V(r) \) show that
%
\begin{dmath}\label{eqn:symmetricPotentialDerivativeExpectation:20}
\Abs{\psi(0)}^2 = \frac{m}{2 \pi \Hbar^2} \expectation{ \frac{dV}{dr} },
\end{dmath}
%
for all s-states, ground or excited.

\makesubproblem{}{problem:symmetricPotentialDerivativeExpectation:1:b}

Show this is the case for the 3D SHO and hydrogen wave functions.

} % problem

\makeanswer{problem:symmetricPotentialDerivativeExpectation:1}{

\makeSubAnswer{}{problem:symmetricPotentialDerivativeExpectation:1:a}

The text works a problem that looks similar to this by considering the commutator of an operator \( A \), later set to \( A = p_r = -i \Hbar \PDi{r}{} \) the radial momentum operator.  First it is noted that
%
\begin{dmath}\label{eqn:symmetricPotentialDerivativeExpectation:40}
0 = \bra{nlm} \antisymmetric{H}{A} \ket{nlm},
\end{dmath}
%
since \( H \) operating to either the right or the left is the energy eigenvalue \( E_n \).  Next it appears the author uses an angular momentum factoring of the squared momentum operator.  Looking earlier in the text that factoring is found to be
%
\begin{dmath}\label{eqn:symmetricPotentialDerivativeExpectation:60}
\frac{\Bp^2}{2m}
= \inv{2 m r^2} \BL^2 - \frac{\Hbar^2}{2m} \lr{ \PDSq{r}{} + \frac{2}{r} \PD{r}{} }.
\end{dmath}

With
\begin{dmath}\label{eqn:symmetricPotentialDerivativeExpectation:80}
R = - \frac{\Hbar^2}{2m} \lr{ \PDSq{r}{} + \frac{2}{r} \PD{r}{} }.
\end{dmath}

we have
%
\begin{dmath}\label{eqn:symmetricPotentialDerivativeExpectation:100}
0
= \bra{nlm} \antisymmetric{H}{p_r} \ket{nlm}
= \bra{nlm} \antisymmetric{\frac{\Bp^2}{2m} + V(r)}{p_r} \ket{nlm}
= \bra{nlm} \antisymmetric{\inv{2 m r^2} \BL^2 + R + V(r)}{p_r} \ket{nlm}
= \bra{nlm} \antisymmetric{\frac{-\Hbar^2 l (l+1)}{2 m r^2} + R + V(r)}{p_r} \ket{nlm}.
\end{dmath}

Let's consider the commutator of each term separately.  First
%
\begin{dmath}\label{eqn:symmetricPotentialDerivativeExpectation:120}
\antisymmetric{V}{p_r} \psi
=
V p_r \psi
-
p_r V \psi
=
V p_r \psi
-
(p_r V) \psi
-
V p_r \psi
=
-
(p_r V) \psi
=
i \Hbar \PD{r}{V} \psi.
\end{dmath}

Setting \( V(r) = 1/r^2 \), we also have
%
\begin{dmath}\label{eqn:symmetricPotentialDerivativeExpectation:160}
\antisymmetric{\inv{r^2}}{p_r} \psi
=
-\frac{2 i \Hbar}{r^3} \psi.
\end{dmath}

Finally
\begin{dmath}\label{eqn:symmetricPotentialDerivativeExpectation:180}
\antisymmetric{\PDSq{r}{} + \frac{2}{r} \PD{r}{} }{ \PD{r}{}}
=
\lr{ \partial_{rr} + \frac{2}{r} \partial_r } \partial_r
-
\partial_r \lr{ \partial_{rr} + \frac{2}{r} \partial_r }
=
\partial_{rrr} + \frac{2}{r} \partial_{rr}
-
\lr{
\partial_{rrr} -\frac{2}{r^2} \partial_r + \frac{2}{r} \partial_{rr}
}
=
-\frac{2}{r^2} \partial_r,
\end{dmath}
%
so
\begin{dmath}\label{eqn:symmetricPotentialDerivativeExpectation:200}
\antisymmetric{R}{p_r}
=-\frac{2}{r^2} \frac{-\Hbar^2}{2m} p_r
=\frac{\Hbar^2}{m r^2} p_r.
\end{dmath}

Putting all the pieces back together, we've got
\begin{dmath}\label{eqn:symmetricPotentialDerivativeExpectation:220}
0
= \bra{nlm} \antisymmetric{\frac{-\Hbar^2 l (l+1)}{2 m r^2} + R + V(r)}{p_r} \ket{nlm}
=
i \Hbar
\bra{nlm} \lr{
\frac{\Hbar^2 l (l+1)}{m r^3} - \frac{i\Hbar}{m r^2} p_r +
\PD{r}{V}
}
\ket{nlm}.
\end{dmath}

Since s-states are those for which \( l = 0 \), this means
%
\begin{dmath}\label{eqn:symmetricPotentialDerivativeExpectation:240}
\expectation{\PD{r}{V}}
= \frac{i\Hbar}{m } \expectation{ \inv{r^2} p_r }
= \frac{\Hbar^2}{m } \expectation{ \inv{r^2} \PD{r}{} }
= \frac{\Hbar^2}{m } \int_0^\infty dr \int_0^\pi d\theta \int_0^{2 \pi} d\phi r^2 \sin\theta \psi^\conj(r,\theta, \phi) \inv{r^2} \PD{r}{\psi(r,\theta,\phi)}.
\end{dmath}
%
Since s-states are spherically symmetric, this is
\begin{dmath}\label{eqn:symmetricPotentialDerivativeExpectation:260}
\expectation{\PD{r}{V}}
= \frac{4 \pi \Hbar^2}{m } \int_0^\infty dr \psi^\conj \PD{r}{\psi}.
\end{dmath}

That integral is
%
\begin{dmath}\label{eqn:symmetricPotentialDerivativeExpectation:280}
\int_0^\infty dr \psi^\conj \PD{r}{\psi}
=
\evalrange{\Abs{\psi}^2}{0}{\infty} - \int_0^\infty dr \PD{r}{\psi^\conj} \psi.
\end{dmath}

With the hydrogen atom, our radial wave functions are real valued.  It's reasonable to assume that we can do the same for other real-valued spherical potentials.  If that is the case, we have
%
\begin{dmath}\label{eqn:symmetricPotentialDerivativeExpectation:300}
2 \int_0^\infty dr \psi^\conj \PD{r}{\psi}
=
\Abs{\psi(0)}^2,
\end{dmath}
%
and
%
%\begin{equation}\label{eqn:symmetricPotentialDerivativeExpectation:320}
\boxedEquation{eqn:symmetricPotentialDerivativeExpectation:340}{
\expectation{\PD{r}{V}}
= \frac{2 \pi \Hbar^2}{m } \Abs{\psi(0)}^2,
}
%\end{equation}

which completes this part of the problem.

\makeSubAnswer{}{problem:symmetricPotentialDerivativeExpectation:1:b}

For a hydrogen like atom, in atomic units, we have
%
\begin{dmath}\label{eqn:symmetricPotentialDerivativeExpectation:360}
\expectation{
\PD{r}{V}
}
=
\expectation{
\PD{r}{} \lr{ -\frac{Z e^2}{r} }
}
=
Z e^2
\expectation
{
\inv{r^2}
}
=
Z e^2 \frac{Z^2}{n^3 a_0^2 \lr{ l + 1/2 }}
=
\frac{\Hbar^2}{m a_0} \frac{2 Z^3}{n^3 a_0^2}
=
\frac{2 \Hbar^2 Z^3}{m n^3 a_0^3}.
\end{dmath}

On the other hand for \( n = 1 \), we have
%
\begin{dmath}\label{eqn:symmetricPotentialDerivativeExpectation:380}
\frac{2 \pi \Hbar^2}{m} \Abs{R_{10}(0)}^2 \Abs{Y_{00}}^2
=
\frac{2 \pi \Hbar^2}{m} \frac{Z^3}{a_0^3} 4 \inv{4 \pi}
=
\frac{2 \Hbar^2 Z^3}{m a_0^3},
\end{dmath}
%
and for \( n = 2 \), we have
%
\begin{dmath}\label{eqn:symmetricPotentialDerivativeExpectation:400}
\frac{2 \pi \Hbar^2}{m} \Abs{R_{20}(0)}^2 \Abs{Y_{00}}^2
=
\frac{2 \pi \Hbar^2}{m} \frac{Z^3}{8 a_0^3} 4 \inv{4 \pi}
=
\frac{\Hbar^2 Z^3}{4 m a_0^3}.
\end{dmath}

These both match the potential derivative expectation when evaluated for the s-orbital (\( l = 0 \)).

In \nbref{sakuraiProblem5.16bSHO.nb} is a verification for the 3D SHO ground state.  There it was found that
%
\begin{dmath}\label{eqn:symmetricPotentialDerivativeExpectation:420}
\expectation{\PD{r}{V}}
= \frac{2 \pi \Hbar^2}{m } \Abs{\psi(0)}^2
= 2 \sqrt{\frac{m \omega ^3 \Hbar}{ \pi }}
\end{dmath}

} % answer

%\EndArticle
