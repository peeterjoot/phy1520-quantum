%
% Copyright � 2015 Peeter Joot.  All Rights Reserved.
% Licenced as described in the file LICENSE under the root directory of this GIT repository.
%
\input{../assignment.tex}
\renewcommand{\basename}{gradQuantumProblemSet7}
\renewcommand{\dirname}{notes/phy1520-quantum/}
\newcommand{\keywords}{Graduate Quantum Mechanics, PHY1520H}
\newcommand{\dateintitle}{}
\input{../peeter_prologue_print2.tex}

\usepackage{peeters_layout_exercise}
\usepackage{peeters_braket}
%\usepackage{phy1520}
\usepackage{siunitx}
%\usepackage{esint} % \oiint

\renewcommand{\QuestionNB}{\alph{Question}.\ }
\renewcommand{\theQuestion}{\alph{Question}}

\newcommand{\nbref}[1]{%
\itemRef{phy1520}{#1}%
}

\beginArtNoToc
\generatetitle{PHY1520H Graduate Quantum Mechanics.  Problem Set 7: Variational problems}
%\chapter{Variational problems}
\label{chap:gradQuantumProblemSet7}

%
% Copyright � 2015 Peeter Joot.  All Rights Reserved.
% Licenced as described in the file LICENSE under the root directory of this GIT repository.
%
\makeoproblem{Double well potential.}{gradQuantum:problemSet7:1}{2015 ps7 p1}{
\index{double well potential}

Consider a particle in the double well potential
%
\begin{dmath}\label{eqn:gradQuantumProblemSet7Problem1:20}
V (x) =
\frac{m \omega^2}{ 8 a^2 }
\lr{ x + a }^2 \lr{x - a}^2.
\end{dmath}
%
Expanding \( V(x) \) around \( x = \pm a \) leads to a harmonic potential with frequency \( \omega \).
Construct variational states with even/odd parity as \( \psi_\pm(x) = g_\pm \lr{ \phi(x - a) \pm \phi(x + a) } \) where \( \phi(x) \) is the normalized ground state of the usual harmonic oscillator with frequency \( \omega \), i.e.,
%
\begin{equation}\label{eqn:gradQuantumProblemSet7Problem1:40}
\phi(x)
=
\lr{ \inv{ \pi a_0^2 }}^{1/4}
e^{
- \frac{x^2}{ 2 a_0^2 }
}
;
\qquad a_0 = \sqrt{ \frac{\Hbar}{m \omega} }.
\end{equation}
%
\makesubproblem{}{gradQuantum:problemSet7:1a}
Determine the normalization constants \( g_\pm \).
Next using these wavefunctions, determine the variational energies of these two states.
Hence determine the `tunnel splitting' between the two states, induced by the tunneling through
the barrier region.
In your calculations, you can assume \( a \gg a_0 \), so retain only the leading terms in any polynomials you might encounter when you do the integrals.
%
\makesubproblem{}{gradQuantum:problemSet7:1b}
If we pay attention to these lowest two states (left well and right well) in the full Hilbert space, we can write a phenomenological \( 2 \times 2 \) Hamiltonian
%
\begin{equation}\label{eqn:gradQuantumProblemSet7Problem1:60}
H =
\begin{bmatrix}
\epsilon_0 & -\gamma \\
-\gamma & \epsilon_0
\end{bmatrix},
\end{equation}
%
where \( \epsilon_0 \) is the energy on each side, and \( \gamma \) leads to tunneling, so if we start off in the left well, \( t \) leads to a nonzero amplitude to find it in the right well at a later time.
Find its eigenvalues and eigenvectors.
Comparing with your variational result for the energy splitting, determine the `tunnel coupling' \( \gamma \).
%
} % makeproblem
%
\makeanswer{gradQuantum:problemSet7:1}{
\withproblemsetsParagraph{
\makeSubAnswer{}{gradQuantum:problemSet7:1a}
%
The integration grunt work for this problem can be found in \nbref{ps7:doubleWellPotential.nb}.  This yields an energy difference of
%
\begin{equation}\label{eqn:gradQuantumProblemSet7Problem1:80}
\overbar{E}_{+} - \overbar{E}_{-}
=
\frac{\Hbar \omega}{8}
\lr{
\lr{
\frac{4 a^2}{a_0^2}
+
\lr{\frac{a^2}{a_0^2}-5 } \exp\lr{\frac{a^2}{a_0^2}} - 2
}
\lr{ \coth\lr{ \frac{a^2}{a_0^2} } -1 }
}.
\end{equation}
%
With \( u = a^2/a_0^2 \), in the \( a \gg a_0 \) limit, the almost zero \( \coth u - 1 \) difference can be approximated as an exponential
%
\begin{dmath}\label{eqn:gradQuantumProblemSet7Problem1:100}
\coth u -1
=
\frac{e^{2u} + 1}{e^{2u} - 1} - 1
=
\frac{e^{2u} + 1 - e^{2u} + 1 }{e^{2u} - 1}
=
\frac{ 2 }{e^{2u} - 1}
\approx
2 e^{-2u},
\end{dmath}
%
so the energy difference is approximately
%
\begin{dmath}\label{eqn:gradQuantumProblemSet7Problem1:120}
\overbar{E}_{+} - \overbar{E}_{-}
\approx
\frac{\Hbar \omega}{4} \frac{a^2}{a_0^2} \exp\lr{-\frac{a^2}{a_0^2}}.
\end{dmath}
%
\makeSubAnswer{}{gradQuantum:problemSet7:1b}
Now lets compare to the energy levels of the phenomenological Hamiltonian, which are given by
%
\begin{dmath}\label{eqn:gradQuantumProblemSet7Problem1:140}
0 = \lr{ \epsilon_0 - E }^2 - \gamma^2,
\end{dmath}
%
with eigenvalues
%
\begin{dmath}\label{eqn:gradQuantumProblemSet7Problem1:160}
E_{\pm} = \epsilon_0 \pm \gamma.
\end{dmath}
%
If the eigenvectors are proportional to the column vector given by
%
\begin{dmath}\label{eqn:gradQuantumProblemSet7Problem1:180}
\ket{\pm} =
\begin{bmatrix}
a \\
b
\end{bmatrix},
\end{dmath}
%
then we must have
%
\begin{dmath}\label{eqn:gradQuantumProblemSet7Problem1:200}
0
= \lr{ \epsilon_0 - (\epsilon_0 \pm \gamma) } a - \gamma b
= \gamma \lr{ \mp a - b },
\end{dmath}
%
or
%
\begin{dmath}\label{eqn:gradQuantumProblemSet7Problem1:220}
\ket{\pm}
=
\inv{\sqrt{2}}
\begin{bmatrix}
1 \\
\mp 1
\end{bmatrix}.
\end{dmath}
%
The energy level difference for this Hamiltonian is
%
\begin{dmath}\label{eqn:gradQuantumProblemSet7Problem1:240}
\Delta E
= E_{+} - E_{-}
= \epsilon_0 + \gamma - \lr{ \epsilon_0 - \gamma }
= 2 \gamma.
\end{dmath}
%
Equating this difference with \cref{eqn:gradQuantumProblemSet7Problem1:120}, we have
%
\begin{dmath}\label{eqn:gradQuantumProblemSet7Problem1:260}
\gamma
=
\frac{\Hbar \omega}{8} \frac{a^2}{a_0^2} \exp\lr{-\frac{a^2}{a_0^2}}.
\end{dmath}
}
}

%
% Copyright � 2015 Peeter Joot.  All Rights Reserved.
% Licenced as described in the file LICENSE under the root directory of this GIT repository.
%
\makeoproblem{Helium-4 atom.}{gradQuantum:problemSet7:2}{2015 ps7 p2}{
\index{helium-4}

Consider the Helium atom with atomic number \( Z=2 \), which leads to the nuclear charge \( Z=2e \), and two electrons with charge \( -e \) each.

\makesubproblem{}{gradQuantum:problemSet7:2a}
Show that ignoring electron-electron interactions leads to a ground state energy \( E_{\textrm{He}} = 4 E_\txtH \)
where \( E_\txtH \) is the ground state energy of the hydrogen atom.

\makesubproblem{}{gradQuantum:problemSet7:2b}
Consider the full problem which retains the Coulomb interaction between the electrons, i.e.

\begin{dmath}\label{eqn:gradQuantumProblemSet7Problem2:20}
H
=
\inv{2m} \lr{ \Bp_1^2 + \Bp_2^2 }
-
2 e^2
\inv{4 \pi \epsilon_0 }
\lr{ \inv{r_1} + \inv{r_2} }
+
e^2
\inv{4 \pi \epsilon_0 }
\inv{ \Abs{\Br_1 - \Br_2} }.
\end{dmath}

and consider the variational wavefunction
\begin{dmath}\label{eqn:gradQuantumProblemSet7Problem2:40}
\psi(\Br_1, \Br_2)
=
N
e^{- \inv{a} \lr{ r_1 + r_2 } }.
\end{dmath}

where \( N \) is the normalization constant, and \( a \) is a variational parameter. Determine the variational ground state energy, and minimize with respect to a to find the best estimate for the ground state energy of Helium.
Compare with numerical estimates of the energy.
} % makeproblem

\makeanswer{gradQuantum:problemSet7:2}{
\withproblemsetsParagraph{
\makeSubAnswer{}{gradQuantum:problemSet7:2a}

%\paragraph{Comparing the Hydrogen and Helium ground state energies.}

%Having initially thought that I had to show that \( E_{\textrm{He}} = 4 E_\txtH\), and geting an 8 times difference above, I went looking for mistakes and tried the Helium ground state a different way.
Without the electron-electron interaction term, the Helium Hamiltonian is separable.  Assuming a wave function of the form

\begin{dmath}\label{eqn:gradQuantumProblemSet7Problem2:660}
\psi(r_1, r_2) = \psi_1(r_1) \psi_2(r_2),
\end{dmath}

The Hamiltonian action on this wave function is

\begin{dmath}\label{eqn:gradQuantumProblemSet7Problem2:680}
E \psi_1(r_1) \psi_2(r_2)
=
\lr{ \inv{2m} \Bp_1^2 \psi_1(r_1) } \psi_2(r_2) + \lr{ \inv{2m} \Bp_2^2 \psi_2(r_1) } \psi_1(r_2)
-
2 e^2
\inv{4 \pi \epsilon_0 } \inv{r_1} \psi_1(r_1) \psi_2(r_2)
-
2 e^2
\inv{4 \pi \epsilon_0 } \inv{r_1} \psi_1(r_1) \psi_2(r_2),
\end{dmath}

or
\begin{dmath}\label{eqn:gradQuantumProblemSet7Problem2:700}
E
=
\lr{\inv{\psi_1(r)} \lr{ \inv{2m} \Bp_1^2 \psi_1(r_1) }
-
2 e^2
\inv{4 \pi \epsilon_0 } \inv{r_1} }
+
\lr{\inv{\psi_2(r)} \lr{ \inv{2m} \Bp_2^2 \psi_2(r_2) }
-
2 e^2
\inv{4 \pi \epsilon_0 } \inv{r_2}}.
\end{dmath}

This can be written in separated form as

\begin{equation}\label{eqn:gradQuantumProblemSet7Problem2:720}
\begin{aligned}
E_1 \psi_1(r_1) &= \inv{2m} \Bp_1^2 \psi_1(r_1) - 2 e^2 \inv{4 \pi \epsilon_0 } \inv{r_1} \psi_1(r_1) \\
E_2 \psi_2(r_2) &= \inv{2m} \Bp_2^2 \psi_2(r_2) - 2 e^2 \inv{4 \pi \epsilon_0 } \inv{r_2} \psi_2(r_2) \\
E &= E_1 + E_2.
\end{aligned}
\end{equation}

Observe that each of these separated Hamiltonians have (with \( Z = 2 \) ) the form

\begin{equation}\label{eqn:gradQuantumProblemSet7Problem2:740}
E \psi(r) = \inv{2m} \Bp^2 \psi(r) - Z e^2 \inv{4 \pi \epsilon_0 } \inv{r} \psi(r).
\end{equation}

With \( Z = 1 \) that is precisely the Hamiltonian for the Hydrogen atom.  If the wavefunction for this Hamiltonian is assumed to be \( \psi(r) = e^{-r/a} \), we find

\begin{equation}\label{eqn:gradQuantumProblemSet7Problem2:760}
\frac{\bra{\psi} H \ket{\psi} }{\braket{\psi}{\psi}}
=
\frac{\Hbar^2}{2 m a^2} - \frac{Z e^2}{4 \pi \epsilon_0 a},
\end{equation}

which has its minimum at

\begin{equation}\label{eqn:gradQuantumProblemSet7Problem2:780}
a_{\mathrm{min}} = \frac{a_0}{Z},
\end{equation}

where
\begin{dmath}\label{eqn:gradQuantumProblemSet7Problem2:880}
a_0 = \frac{4 \pi \epsilon_0 \Hbar^2}{m e^2}.
\end{dmath}

The minimum energy is found to be

\begin{equation}\label{eqn:gradQuantumProblemSet7Problem2:800}
E_{\mathrm{min}} = -\inv{2} \frac{e^2 Z^2}{ 4 \pi \epsilon_0 a_0 }.
\end{equation}

With \( Z = 1 \), the Hydrogen ground state energy is

\begin{equation}\label{eqn:gradQuantumProblemSet7Problem2:820}
E_\txtH
= -\inv{2} \frac{e^2}{ 4 \pi \epsilon_0 a_0 },
\end{equation}

a value of about \( -13.6 \si{eV} \).  The Helium ground state energy is

\begin{equation}\label{eqn:gradQuantumProblemSet7Problem2:840}
\begin{aligned}
E_{\textrm{He}}
&= \evalbar{E_1}{Z=2} + \evalbar{E_2}{Z=2} \\
&= -\inv{2} \lr{ 2^2 + 2^2 } \frac{e^2 }{ 4 \pi \epsilon_0 a_0 } \\
&= - 4 \frac{e^2 }{ 4 \pi \epsilon_0 a_0 }.
\end{aligned}
\end{equation}

This is \( E_{\textrm{He}} = 8 E_\txtH\), a value of about \( -109 eV \).

The computations above can be found in \nbref{ps7:heliumAtomGroundStateWithInteraction.nb}.

\makeSubAnswer{}{gradQuantum:problemSet7:2b}

The Laplacian of an exponentially decreasing trial function \( e^{-r/a} \) is

\begin{dmath}\label{eqn:gradQuantumProblemSet7Problem2:340}
\begin{aligned}
\spacegrad^2 e^{-r/a}
&=
\inv{r^2} \PD{r}{} \lr{ r^2 \PD{r}{e^{-r/a}} } \\
&=
\inv{r^2} \PD{r}{} \lr{ -\frac{r^2}{a} e^{-r/a} } \\
&=
-\inv{r^2 a} \lr{ 2 r - \frac{r^2}{a} } e^{-r/a},
\end{aligned}
\end{dmath}

%To calculate \( \Bp_j^2 \psi \) first compute the Laplacian of an exponential
%
%\begin{dmath}\label{eqn:gradQuantumProblemSet7Problem2:60}
%\spacegrad^2 e^{\phi}
%=
%\spacegrad \cdot \spacegrad e^\phi
%=
%\spacegrad \cdot \lr{ e^\phi \spacegrad \phi }
%=
%e^\phi \spacegrad^2 \phi + \lr{ \spacegrad \phi }^2 e^\phi
%=
%\lr{ \spacegrad^2 \phi + \lr{ \spacegrad \phi }^2  } e^\phi.
%\end{dmath}
%
%For \( \phi = -r/a \), we have
%
%\begin{dmath}\label{eqn:gradQuantumProblemSet7Problem2:80}
%\spacegrad \phi
%=
%- \inv{a} \spacegrad \sqrt{ \Bx^2 }
%=
%- \inv{a} \spacegrad \sqrt{ x_k x_k }
%=
%- \inv{a} \inv{2 r} \Be_j ( 2 \partial_j x_k ) x_k
%=
%- \frac{\Bx}{a r},
%\end{dmath}
%
%and
%\begin{dmath}\label{eqn:gradQuantumProblemSet7Problem2:100}
%\spacegrad^2 \phi
%=
%-\inv{a} \spacegrad \cdot \frac{ \Bx}{ r}
%=
%-\inv{a} \lr{
%\inv{r} \spacegrad \cdot \Bx
%+
%\Bx \cdot \spacegrad \inv{ r }
%}
%=
%-\inv{a} \lr{
%\frac{3}{r}
%+
%\Bx \cdot \lr{ -\inv{r^3} \Bx }
%}
%=
%-\inv{a} \lr{
%\frac{3}{r}
%-
%\frac{1}{r}
%}
%=
%-\frac{2}{a r}.
%\end{dmath}
%
or

\begin{dmath}\label{eqn:gradQuantumProblemSet7Problem2:120}
\spacegrad^2 e^{-r/a} = \inv{a} \lr{ \inv{a} -\frac{2}{r} } e^{-r/a}.
\end{dmath}

%%%For the Hydrogen atom (with \( a = a_0 \)), the Hamiltonian action on the unnormalized ground state wavefunction \( \psi = e^{-r/a} \) is
%%%
%%%\begin{dmath}\label{eqn:gradQuantumProblemSet7Problem2:380}
%%%H \psi(r)
%%%=
%%%\frac{\Bp^2}{2m} \psi
%%%- e^2 \inv{4 \pi \epsilon_0 } \inv{r} \psi
%%%=
%%%-\frac{\Hbar^2}{2m} \inv{a} \lr{ \inv{a} -\frac{2}{r} } \psi
%%%- e^2 \inv{4 \pi \epsilon_0 } \inv{r} \psi
%%%=
%%%\lr{ -\frac{\Hbar^2}{2m} \inv{a^2} + \lr{ \frac{\Hbar^2}{m a} - e^2 \inv{4 \pi \epsilon_0 } \inv{r} } } e^{-r/a}.
%%%\end{dmath}
%%%
%%%The hydrogen ground state energy is
%%%\begin{dmath}\label{eqn:gradQuantumProblemSet7Problem2:160}
%%%E_\txtH
%%%%=
%%%%\frac{
%%%%   \bra{ \psi }
%%%%   \lr{ -\frac{\Hbar^2}{2m a} \lr{ \inv{a} -\frac{2}{ r} } - e^2 \inv{4 \pi \epsilon_0 r} }
%%%%   \ket{ \psi }
%%%%}
%%%%{ \braket{ \psi }{ \psi } }
%%%=
%%%   \lr{ \frac{\Hbar^2}{m a} - e^2 \inv{4 \pi \epsilon_0 } }
%%%\frac{
%%%\bra{ \psi } \inv{r} \ket{ \psi }
%%%}
%%%{
%%%   \braket{ \psi }{ \psi }
%%%}
%%%   - \frac{\Hbar^2}{2m a^2} .
%%%\end{dmath}

For Helium without electron-electron interaction the kinetic portion of the Hamiltonian action on this trial function \( \psi = e^{-(r_1 + r_2)/a} \) is

\begin{dmath}\label{eqn:gradQuantumProblemSet7Problem2:140}
H \psi(r_1, r_2)
=
\frac{\Bp_1^2}{2m} \psi
+
\frac{\Bp_2^2}{2m} \psi
- 2 e^2 \inv{4 \pi \epsilon_0 } \lr{ \inv{r_1} + \inv{r_2} } \psi
=
-\frac{\Hbar^2}{2m a} \lr{ \frac{2}{a} -\frac{2}{ r_1}  -\frac{2}{ r_2} } \psi
- 2 e^2 \inv{4 \pi \epsilon_0 } \lr{ \inv{r_1} + \inv{r_2} } \psi
=
\lr{ -\frac{\Hbar^2}{m a^2}
+
\lr{ \frac{\Hbar^2}{m a} - \frac{e^2}{2 \pi \epsilon_0} } \lr{ \inv{r_1} + \inv{r_2} }
}
e^{-(r_1 + r_2)/a}.
\end{dmath}

Now, assuming that \( \psi = e^{-(r_1 + r_2)/a} \) is the unnormalized ground state wavefunction for the Helium atom without electron-electron interaction, that ground state energy is given by

\begin{dmath}\label{eqn:gradQuantumProblemSet7Problem2:260}
E_{\textrm{He}}
%=
%\frac{
%   \bra{ \psi }
%   \lr{
%      -\frac{\Hbar^2}{m a} \lr{ \inv{a} -\frac{1}{ r_1}  -\frac{1}{ r_2} }
%      - 2 e^2 \inv{4 \pi \epsilon_0 } \lr{ \inv{r_1} + \inv{r_2} }
%   }
%   \ket{ \psi }
%}
%{ \braket{ \psi }{ \psi } }
=
   \lr{ \frac{\Hbar^2}{m a} - e^2 \inv{2 \pi \epsilon_0 } }
\frac{
\bra{ \psi } \inv{r_1} + \inv{r_2} \ket{ \psi }
}
{
   \braket{ \psi }{ \psi }
}
   - \frac{\Hbar^2}{m a^2} .
\end{dmath}

%%%\paragraph{Calculating Hydrogen ground state energy}

We'll need a couple helper integrals
%A couple helper integrals
%For the normalization factor we have

\begin{dmath}\label{eqn:gradQuantumProblemSet7Problem2:180}
%\begin{aligned}
%\braket{ \psi }{ \psi }
%&=
4 \pi \int_0^\infty r^2 dr e^{-2 r/a}
=
%&=
%{4 \pi}\frac{a^3}{2^3} \int_0^\infty r^2 dr e^{-r} \\
%&=
%\inv{2} { \pi}{a^3} \int_0^\infty 2r dr e^{-r} \\
%&=
%{\pi}{a^3} \int_0^\infty dr e^{-r} \\
%&=
{\pi}{a^3},
%\end{aligned}
\end{dmath}

and %for the inverse radial expectation we have

\begin{dmath}\label{eqn:gradQuantumProblemSet7Problem2:200}
%\bra{ \psi } \inv{r} \ket{ \psi }
%=
4 \pi \int_0^\infty r dr e^{-2 r/a}
%=
%{4 \pi}\frac{a^2}{2^2} \int_0^\infty r dr e^{-r}
=
{\pi}{a^2}.
\end{dmath}

%so
%
%%%\begin{dmath}\label{eqn:gradQuantumProblemSet7Problem2:220}
%%%E_\txtH
%%%=
%%%\lr{ \frac{\Hbar^2}{m a} - e^2 \inv{4 \pi \epsilon_0 } } \frac{\pi a^2}{\pi a^3}
%%%   - \frac{\Hbar^2}{2m a^2} ,
%%%=
%%%\frac{\Hbar^2}{2 m a^2} - \frac{e^2}{4 \pi \epsilon_0 a }.
%%%\end{dmath}
%%%
%%%A test minimization of this energy using \( a \) as a variational parameter finds
%%%
%%%\begin{dmath}\label{eqn:gradQuantumProblemSet7Problem2:620}
%%%a = \frac{4 \pi \epsilon_0 \Hbar^2}{m e^2},
%%%\end{dmath}
%%%
%%%which is the Bohr-radius as expected.  Substituting that gives
%%%
%%%%\begin{dmath}\label{eqn:gradQuantumProblemSet7Problem2:240}
%%%\boxedEquation{eqn:gradQuantumProblemSet7Problem2:240}{
%%%E_\txtH
%%%=
%%%-
%%%\frac{ m e^4 }{ 32 \pi^2 \epsilon_0^2 \Hbar^2 }
%%%=
%%%-
%%%\inv{2} \frac{e^2}{4 \pi \epsilon_0 a_0}.
%%%}
%%%%\end{dmath}
%%%
%%%Numerically this is about \( -13.6 \si{eV} \).
%
%\paragraph{Calculating Helium ground state energy}
%
To normalize the wavefunction, we need a six-fold integral over both the spatial domains.  With only radial dependence that is

\begin{dmath}\label{eqn:gradQuantumProblemSet7Problem2:280}
%\begin{aligned}
\braket{ \psi }{ \psi }
=
\lr{ 4 \pi}^2
\int_0^\infty r_1^2 dr_1 e^{-2 r_1/a}
\int_0^\infty r_2^2 dr_2 e^{-2 r_2/a}
%&=
%\lr{ 4 \pi}^2 \lr{ \frac{a}{2} }^6
%\lr{ \int_0^\infty r^2 dr e^{-r} }^2 \\
%&=
%2^{4 - 6 + 2}
%\pi^2 a^6 \\
= \pi^2 a^6.
%\end{aligned}
\end{dmath}

We also need the inverse radial expectations.  Calculating the expectation of \( 1/r_1 \) is sufficient, and is

\begin{dmath}\label{eqn:gradQuantumProblemSet7Problem2:300}
\bra{ \psi } \inv{r_1} \ket{ \psi }
=
\lr{ 4 \pi}^2
\int_0^\infty r_1 dr_1 e^{-2 r_1/a}
\int_0^\infty r_2^2 dr_2 e^{-2 r_2/a}
%=
%\lr{ 4 \pi}^2 \lr{ \frac{a}{2} }^5
%\lr{ \int_0^\infty r dr e^{-r} }
%\lr{ \int_0^\infty r^2 dr e^{-r} }
%=
%2^{4 - 5 + 1}
%\pi^2 a^5
= \pi^2 a^5.
\end{dmath}

So, without the electron-electron interaction, the ground state energy is

\begin{dmath}\label{eqn:gradQuantumProblemSet7Problem2:320}
E_{\textrm{He}}
=
   \lr{ \frac{\Hbar^2}{m a} - e^2 \inv{2 \pi \epsilon_0 } }
\frac{ 2 \pi^2 a^5 }
{
   \pi^2 a^6
}
   - \frac{\Hbar^2}{m a^2}
=
   \frac{\Hbar^2}{m a^2} - e^2 \inv{\pi \epsilon_0 a }.
\end{dmath}

%%%It appears that the value of \( a \) that minimizes this energy is not the bohr radius for this wave function, so the assumption that \( \psi(r_1, r_2) = e^{-(r_1 + r_2)/a_0} \) was an eigenfunction for the Hamiltonian was incorrect.  Performing the variation, we find that the minimum energy is found at
%%%
%%%\begin{dmath}\label{eqn:gradQuantumProblemSet7Problem2:640}
%%%a = \inv{2} \frac{4 \pi \epsilon_0 \Hbar^2}{m e^2},
%%%\end{dmath}
%%%
%%%which is half the Bohr-radius.  Substituting that into the energy above gives
%%%
%%%\boxedEquation{eqn:gradQuantumProblemSet7Problem2:400}{
%%%E_{\textrm{He}}
%%%=
%%%-\frac{e^2 m}{4 \pi^2 \epsilon_0^2 \Hbar^2}
%%%=
%%%-
%%%\frac{e^2}{\pi \epsilon_0 a_0}.
%%%}
%%%
%This is \( E_{\textrm{He}} = 8 E_\txtH\), a value of about \( -110 eV \).

To evaluate the interaction term, a Fourier transform representation of that inverse radial distance can be employed

\begin{dmath}\label{eqn:gradQuantumProblemSet7Problem2:900}
\inv{\Abs{\Br}}
= \inv{2 \pi^2} \int d^3 k \frac{e^{i \Bk \cdot \Br}}{\Bk^2},
\end{dmath}

where this is understood to be the \( \epsilon \rightarrow 0 \) limit of

\begin{dmath}\label{eqn:gradQuantumProblemSet7Problem2:420}
\inv{\Abs{\Br}} e^{-\epsilon \Abs{\Br}}
= \inv{2 \pi^2} \int d^3 k \frac{e^{i \Bk \cdot \Br}}{\Bk^2 + \epsilon^2}.
\end{dmath}

See \citep{byron1992mca} for a demonstration of this identity, and the contour used to evaluate the RHS of \cref{eqn:gradQuantumProblemSet7Problem2:420}.  Employing this inverse radial representation, the \( r_1 \) and \( r_2 \) contributions to the interaction can be decoupled

\begin{dmath}\label{eqn:gradQuantumProblemSet7Problem2:440}
\frac{e^2}{4 \pi \epsilon_0} \bra{\psi} \inv{\Abs{\Br_1 - \Br_2}} \ket{\psi}
=
\frac{e^2}{4 \pi \epsilon_0}
\int 2 \pi dr_1 d\theta_1 r_1^2 \sin(\theta_1)
\int 2 \pi dr_2 d\theta_2 r_2^2 \sin(\theta_2)
e^{ -2(r_1 + r_2)/a}
\inv{2 \pi^2}
\int d^3 k \frac{e^{i \Bk \cdot \lr{ \Br_1 - \Br_2} }}{\Bk^2}
=
\frac{e^2}{4 \pi \epsilon_0}
\inv{2 \pi^2}
\int d^3 k \inv{\Bk^2}
\int 2 \pi dr_1 d\theta_1 r_1^2 \sin(\theta_1) e^{-2r_1/a + i \Bk \cdot \Br_1}
\int 2 \pi dr_2 d\theta_2 r_2^2 \sin(\theta_2) e^{-2r_2/a - i \Bk \cdot \Br_2}.
\end{dmath}

The spatial domain integrals can now be evaluated separately.  With a coordinate system picked so that \( \Bk = \pm k \zcap \), that gives

\begin{dmath}\label{eqn:gradQuantumProblemSet7Problem2:460}
\begin{aligned}
2 \pi \int dr d\theta r^2 \sin(\theta) e^{-2r/a + i \Bk \cdot \Br}
&=
2 \pi \int_0^\infty dr r^2 e^{-2r/a}
\int_0^\pi d\theta
\frac{d}{d\theta} (-\cos(\theta))
e^{\pm i k r \cos\theta} \\
&=
2 \pi \int_0^\infty dr r^2 e^{-2r/a}
\int_{-1}^1 du
e^{\mp i k r u} \\
&=
2 \pi \int_0^\infty dr r^2 e^{-2r/a}
\frac{e^{\mp i k r} - e^{\pm i k r} }{ \mp i k r } \\
&=
2 \pi \frac{2}{k} \int_0^\infty dr r e^{-2r/a} \sin( k r ) \\
&=
2 \pi \frac{2}{k} \int_0^\infty dr r e^{-2r/a} \sin( k r ) \\
&=
\frac{16 \pi a^3}{\lr{1 + a^2 k^2 }^2}.
\end{aligned}
\end{dmath}

We see that the specific orientation used to evaluate the integral does not matter, so we have

\begin{dmath}\label{eqn:gradQuantumProblemSet7Problem2:480}
\frac{e^2}{4 \pi \epsilon_0} \bra{\psi} \inv{\Abs{\Br_1 - \Br_2}} \ket{\psi}
=
\frac{e^2}{4 \pi \epsilon_0}
\inv{2 \pi^2}
\int d^3 k \inv{\Bk^2}
\frac{(16 \pi a^3)^2}{\lr{1 + a^2 k^2 }^4}
=
\frac{e^2}{4 \pi \epsilon_0}
\inv{2 \pi^2}
(4\pi)
16^2 \pi^2 a^6
\int dk k^2 \inv{\Bk^2}
\inv{\lr{1 + a^2 k^2 }^4}
%=
%\frac{32 e^2 a^6}{\epsilon_0}
%\int dk
%\inv{\lr{1 + a^2 k^2 }^4}
%=
%\frac{32 e^2 a^6}{\epsilon_0}
%\frac{5 \pi}{32 a}
=
\frac{5 \pi e^2 a^5}{32 \epsilon_0}.
\end{dmath}

Rescaling with the normalization factor gives

\begin{dmath}\label{eqn:gradQuantumProblemSet7Problem2:500}
\frac{e^2}{4 \pi \epsilon_0} \bra{\psi} \inv{\Abs{\Br_1 - \Br_2}} \ket{\psi}/\braket{\psi}{\psi}
=
\frac{5 \pi e^2 a^5}{32 \epsilon_0 } \inv{\pi^2 a^6}
=
\frac{5 e^2 }{32 \pi \epsilon_0 a}.
\end{dmath}

Adding this electron-electron interaction to the Helium ground energy calculated in
\cref{eqn:gradQuantumProblemSet7Problem2:320} gives

\begin{dmath}\label{eqn:gradQuantumProblemSet7Problem2:520}
E_{\textrm{He}}
=
\frac{\Hbar^2}{m a^2} -\frac{27}{32} e^2 \inv{\pi \epsilon_0 a }.
\end{dmath}

For the minimum we want to solve

\begin{dmath}\label{eqn:gradQuantumProblemSet7Problem2:540}
0
=
\PD{a}{E}
=
-2 \frac{\Hbar^2}{m a^3} + \frac{27}{32} e^2 \inv{\pi \epsilon_0 a^2 },
\end{dmath}

which has the minimum at

\begin{dmath}\label{eqn:gradQuantumProblemSet7Problem2:560}
a = - \frac{64 \Hbar^2 \pi \epsilon_0}{27 m e^2}.
\end{dmath}

Note that \( m \) should really be treated as the reduced mass of the electron, but doing so isn't numerically significant.  The final result for the variational ground state energy is

%Noting that \( a_0 = \Hbar^2/m e^2 \),
%the ground state energy, after substituting this value of \( a \) is

%\begin{dmath}\label{eqn:gradQuantumProblemSet7Problem2:580}
\boxedEquation{eqn:gradQuantumProblemSet7Problem2:600}{
E_{\textrm{He}}
=
- \lr{\frac{27}{16}}^2 \frac{e^2}{4 \pi \epsilon_0 a_0} \approx -77.5 \si{eV}
}
%\end{dmath}

In atomic units this is

\begin{dmath}\label{eqn:gradQuantumProblemSet7Problem2:860}
E_{\textrm{He}}
=
- \lr{\frac{27}{16}}^2 \frac{e^2}{a_0} \approx -2.848 \frac{e^2}{a_0}.
\end{dmath}

In \citep{desai2009quantum} the measured value is stated as \( -2.90 \,\ifrac{e^2}{a_0} \).  Table 1 of \citep{aznabayev2015energy}, which lists high precision calculations of all the energy levels, has \( -2.903724 \,\ifrac{e^2}{a_0} \) for the 1s energy level.  The calculated value of \cref{eqn:gradQuantumProblemSet7Problem2:600} is about 2 \% off the mark.

See \nbref{ps7:heliumAtomGroundStateWithInteraction.nb}, for a complete end to end verification of the calculations above.
}
}

%
% Copyright � 2015 Peeter Joot.  All Rights Reserved.
% Licenced as described in the file LICENSE under the root directory of this GIT repository.
%
% desai 24.3
\makeoproblem{Harmonic oscillator variation.}{gradQuantum:problemSet7:3}{\citep{desai2009quantum} pr. 24.3}
%\makeproblem{Harmonic oscillator variation}{gradQuantum:problemSet7:3}
{
\index{harmonic oscillator!variational method}
%\makesubproblem{}{gradQuantum:problemSet7:3a}
%
Consider a 1D harmonic oscillator with an unnormalized trial wavefunction \( \psi_v(x) = e^{-\beta \Abs{x}} \).
Minimize the ground state energy with respect to \( \beta \), thus obtaining the optimal \( \beta \) as well as the variational ground state energy.
Compare with the exact result.
Note that you need to be careful evaluating derivatives since the wavefunction has a `cusp' at \( x = 0\).
} % makeproblem
%
\makeanswer{gradQuantum:problemSet7:3}{
\withproblemsetsParagraph{
%\makeSubAnswer{}{gradQuantum:problemSet7:3a}
%
In order to make the derivatives of the trial function better behaved at the origin, we can treat it as a distribution, writing
%
\begin{equation}\label{eqn:gradQuantumProblemSet7Problem3:20}
\psi(x) = \Theta(x) e^{-\beta x} + \Theta(-x) e^{\beta x},
\end{dmath}
%
This has a derivative
%
\begin{dmath}\label{eqn:gradQuantumProblemSet7Problem3:40}
\psi'(x)
=
\delta(x) e^{-\beta x} - \delta(-x) e^{\beta x}
+ \beta \lr{ -\Theta(x) e^{-\beta x} + \Theta(-x) e^{\beta x} }
=
-2 \delta(x) \sinh( \beta x )
+ \beta \lr{ -\Theta(x) e^{-\beta x} + \Theta(-x) e^{\beta x} }.
\end{dmath}
%
Using \( \delta'(x) = - \delta(x)/x \), the second derivative is
%
\begin{dmath}\label{eqn:gradQuantumProblemSet7Problem3:60}
\psi''(x)
=
+2 \delta(x) \sinh( \beta x )/x
-2 \beta \delta(x) \cosh( \beta x )
+ \beta \lr{ -\delta(x) e^{-\beta x} - \delta(-x) e^{\beta x} }
+ \beta^2 \lr{ \Theta(x) e^{-\beta x} + \Theta(-x) e^{\beta x} }
=
+2 \delta(x) \beta \lr{ \frac{\sinh( \beta x )}{\beta x} - \cosh( \beta x ) }
- 2 \beta \delta(x) \cosh( \beta x )
+ \beta^2 e^{-\beta \Abs{x} }.
\end{dmath}
%
Because
%
\begin{equation}\label{eqn:gradQuantumProblemSet7Problem3:80}
\int \delta(x) \cosh( \beta x ) f(x) dx = f(0),
\end{dmath}
%
and
%
\begin{dmath}\label{eqn:gradQuantumProblemSet7Problem3:100}
\int \delta(x) \lr{ \frac{\sinh( \beta x )}{\beta x} - \cosh( \beta x ) } f(x) dx
= \lr{ 1 - 1 } f(0)
= 0,
\end{dmath}
%
this second derivative can be simplified to
\begin{dmath}\label{eqn:gradQuantumProblemSet7Problem3:120}
\psi''(x)
=
- 2 \beta \delta(x) + \beta^2 e^{-\beta \Abs{x} }.
\end{dmath}
%
This has the \( \beta^2 \psi(x) \) value that we expect at points away from the origin.  All the expectations can now be computed.  The normalization is
%
\begin{dmath}\label{eqn:gradQuantumProblemSet7Problem3:140}
\braket{\psi}{\psi}
=
2 \int_0^\infty e^{-2 \beta x} dx
=
2 \evalrange{\frac{e^{-2 \beta x}}{-2 \beta}}{0}{\infty}
=
\inv{\beta}.
\end{dmath}
%
Observe that
%
\begin{dmath}\label{eqn:gradQuantumProblemSet7Problem3:160}
\frac{d^2}{d\beta^2} \int_0^\infty e^{-2 \beta x} dx
=
(-2)^2 \int_0^\infty x^2 e^{-2 \beta x} dx
=
\frac{d}{d\beta} \lr{ -\inv{2 \beta^2}}
=
\frac{1}{\beta^3},
\end{dmath}
%
so
\begin{equation}\label{eqn:gradQuantumProblemSet7Problem3:180}
\int_0^\infty x^2 e^{-2 \beta x} dx = \frac{1}{4 \beta^3},
\end{dmath}
%
a result we will need later.  The kinetic portion of the energy expectation is
%
\begin{dmath}\label{eqn:gradQuantumProblemSet7Problem3:200}
\bra{\psi} \frac{p^2}{2m} \ket{\psi}
=
-\frac{\Hbar^2}{2m}
\int_{-\infty}^\infty e^{-\beta x} \lr{ - 2 \beta \delta(x) + \beta^2 e^{-\beta \Abs{x} } }
=
-\frac{\Hbar^2}{2m} \beta^2 \inv{\beta} -\frac{\Hbar^2}{2m} \lr{ - 2 \beta }
= \frac{\Hbar^2}{2m} \beta.
\end{dmath}
%
The potential portion of the energy expectation is
\begin{dmath}\label{eqn:gradQuantumProblemSet7Problem3:220}
\bra{\psi} \inv{2} m \omega^2 x^2 \ket{\psi}
=
m \omega^2 \int_0^\infty x^2 e^{-2 \beta x} dx
=
m \omega^2 \frac{1}{4 \beta^3}.
\end{dmath}
%
Adding things up we have
\begin{dmath}\label{eqn:gradQuantumProblemSet7Problem3:240}
\overbar{E}(\beta)
=
\frac{\bra{\psi} H \ket{\psi}}{\braket{\psi}{\psi}}
=
\frac{\frac{\Hbar^2}{2m} \beta
+ \frac{ m \omega^2}{4 \beta^3}}{\inv{\beta}}
=
\frac{\Hbar^2}{2m} \beta^2
+ \frac{ m \omega^2}{4 \beta^2}
=
\frac{\Hbar \omega}{2}
\lr{
\frac{\Hbar}{m \omega} \beta^2 +
\frac{ m \omega}{2 \Hbar \beta^2}
}
=
\frac{\Hbar \omega}{2}
\lr{
x_0^2 \beta^2 + \inv{2 x_0^2 \beta^2}
}.
\end{dmath}
%
Minimizing gives
%
\begin{dmath}\label{eqn:gradQuantumProblemSet7Problem3:260}
0
=
\frac{d}{d\beta}
\lr{
x_0^2 \beta^2 + \inv{2 x_0^2 \beta^2}
}
=
2 x_0^2 \beta - \inv{x_0^2 \beta^3},
\end{dmath}
%
or
%
\begin{equation}\label{eqn:gradQuantumProblemSet7Problem3:280}
\beta^4 = \inv{2 x_0^4},
\end{dmath}
%
which gives
\boxedEquation{eqn:gradQuantumProblemSet7Problem3:300}{
\beta = \inv{2^{1/4} x_0}.
}

The energy at this value of \( \beta \) is
%
\begin{dmath}\label{eqn:gradQuantumProblemSet7Problem3:320}
\overbar{E}_{\textrm{min}}
=
\frac{\Hbar \omega}{2}
\lr{
x_0^2 \inv{\sqrt{2} x_0^2} + \frac{ \sqrt{2} x_0^2}{2 x_0^2 }
}
=
\frac{\Hbar \omega}{2}
\frac{2}{ \sqrt{2} },
\end{dmath}
%
or
\boxedEquation{eqn:gradQuantumProblemSet7Problem3:340}{
\overbar{E}_{\textrm{min}}
=
\frac{\Hbar \omega}{2} \sqrt{2} > \Hbar \omega \lr{ 0 + \inv{2} }.
}
We find that the trial function that minimizes the average energy is
%
\begin{equation}\label{eqn:gradQuantumProblemSet7Problem3:360}
\psi(x) = 2^{1/4} x_0 \exp\lr{ -2^{-1/4} \Abs{x}/x_0 },
\end{dmath}
%
with an average energy that is \( 1.41 \times \) the actual ground state energy for the harmonic oscillator.
}
}


\clearpage
\paragraph{Mathematica Sources}

Mathematica notebooks associated with these problems can be found under
\href{https://github.com/peeterjoot/mathematica/tree/master/phy1520-quantum/ps7}{phy1520/ps7/}
within the github repository:

\begin{itemize}
\item git@github.com:peeterjoot/mathematica.git
\end{itemize}

Notebooks created for this problem set:

\input{ps7mathematica.tex}

The data and figures referenced in these notes were generated with versions not greater than:

\begin{itemize}
\item commit c63718392da4966e13c326822c06f6233e15c5df
\end{itemize}

\EndArticle
