%
% Copyright � 2015 Peeter Joot.  All Rights Reserved.
% Licenced as described in the file LICENSE under the root directory of this GIT repository.
%
%\input{../blogpost.tex}
%\renewcommand{\basename}{LyPerturbation}
%\renewcommand{\dirname}{notes/phy1520/}
%%\newcommand{\dateintitle}{}
%%\newcommand{\keywords}{}
%
%\input{../peeter_prologue_print2.tex}
%
%\usepackage{peeters_layout_exercise}
%\usepackage{peeters_braket}
%\usepackage{peeters_figures}
%
%\beginArtNoToc
%
%\generatetitle{\(L_y\) perturbation}
%%\chapter{\(L_y\) perturbation}
%%\label{chap:LyPerturbation}
%
\makeoproblem{\( L_y \) perturbation.}{problem:LyPerturbation:1}{\citep{sakurai2014modern} pr. 5.17(a)}{
\index{angular momentum!perturbation}

Find the first non-zero energy shift for the perturbed Hamiltonian
%
\begin{equation}\label{eqn:LyPerturbation:20}
H = A \BL^2 + B L_z + C L_y = H_0 + V.
\end{dmath}
%
} % problem
%
\makeanswer{problem:LyPerturbation:1}{
%
The energy eigenvalues for state \( \ket{l, m} \) prior to perturbation are
%
\begin{equation}\label{eqn:LyPerturbation:40}
A \Hbar^2 l(l+1) + B \Hbar m.
\end{dmath}
%
The first order energy shift is zero

% Lp = Lx + iLy
% Lm = Lx - iLy
% 2i Ly = Lp - Lm
\begin{dmath}\label{eqn:LyPerturbation:60}
\Delta^1
=
\bra{l, m} C L_y \ket{l, m}
=
\frac{C}{2 i}
\bra{l, m} \lr{ L_{+} - L_{-} } \ket{l, m}
=
0,
\end{dmath}
%
so we need the second order shift.  Assuming no degeneracy to start, the perturbed state is
%
\begin{equation}\label{eqn:LyPerturbation:80}
\ket{l, m}' = \sum' \frac{\ket{l', m'} \bra{l', m'}}{E_{l,m} - E_{l', m'}} V \ket{l, m},
\end{dmath}
%
and the next order energy shift is
\begin{dmath}\label{eqn:LyPerturbation:100}
\Delta^2
=
\bra{l m} V
\sum' \frac{\ket{l', m'} \bra{l', m'}}{E_{l,m} - E_{l', m'}} V \ket{l, m}
=
\sum' \frac{\bra{l, m} V \ket{l', m'} \bra{l', m'}}{E_{l,m} - E_{l', m'}} V \ket{l, m}
=
\sum' \frac{ \Abs{ \bra{l', m'} V \ket{l, m} }^2 }{E_{l,m} - E_{l', m'}}
=
\sum_{m' \ne m} \frac{ \Abs{ \bra{l, m'} V \ket{l, m} }^2 }{E_{l,m} - E_{l, m'}}
=
\sum_{m' \ne m} \frac{ \Abs{ \bra{l, m'} V \ket{l, m} }^2 }{
\lr{ A \Hbar^2 l(l+1) + B \Hbar m }
-\lr{ A \Hbar^2 l(l+1) + B \Hbar m' }
}
=
\inv{B \Hbar} \sum_{m' \ne m} \frac{ \Abs{ \bra{l, m'} V \ket{l, m} }^2 }{
m - m'
}.
\end{dmath}
%
The sum over \( l' \) was eliminated because \( V \) only changes the \( m \) of any state \( \ket{l,m} \), so the matrix element \( \bra{l',m'} V \ket{l, m} \) must includes a \( \delta_{l', l} \) factor.
Since we are now summing over \( m' \ne m \), some of the matrix elements in the numerator should now be non-zero, unlike the case when the zero first order energy shift was calculated in \cref{eqn:LyPerturbation:60}.
%
\begin{equation}\label{eqn:LyPerturbation:120}
\begin{aligned}
&\bra{l, m'} C L_y \ket{l, m} \\
&\qquad=
\frac{C}{2 i}
\bra{l, m'} \lr{ L_{+} - L_{-} } \ket{l, m} \\
&\qquad=
\frac{C}{2 i}
\bra{l, m'}
\lr{
L_{+}
\ket{l, m}
- L_{-}
\ket{l, m}
} \\
&\qquad=
\frac{C \Hbar}{2 i}
\bra{l, m'}
\biglr{
\sqrt{(l - m)(l + m + 1)} \ket{l, m + 1} \\
&\qquad\qquad-
\sqrt{(l + m)(l - m + 1)} \ket{l, m - 1}
} \\
&\qquad=
\frac{C \Hbar}{2 i}
\lr{
\sqrt{(l - m)(l + m + 1)} \delta_{m', m + 1}
-
\sqrt{(l + m)(l - m + 1)} \delta_{m', m - 1}
}.
\end{aligned}
\end{equation}
%
After squaring and summing, the cross terms will be zero since they involve products of delta functions with different indices.  That leaves
%
\begin{dmath}\label{eqn:LyPerturbation:140}
\Delta^2
=
\frac{C^2 \Hbar}{4 B} \sum_{m' \ne m} \frac{
(l - m)(l + m + 1) \delta_{m', m + 1}
-
(l + m)(l - m + 1) \delta_{m', m - 1}
}{
m - m'
}
=
\frac{C^2 \Hbar}{4 B}
\lr{
\frac{ (l - m)(l + m + 1) }{ m - (m+1) }
-
\frac{ (l + m)(l - m + 1) }{ m - (m-1)}
}
=
\frac{C^2 \Hbar}{4 B}
\lr{
-
(l^2 - m^2 + l - m)
-
(l^2 - m^2 + l + m)
}
=
-\frac{C^2 \Hbar}{2 B} (l^2 - m^2 + l ),
\end{dmath}
%
so to first order the energy shift is
%
\boxedEquation{eqn:LyPerturbation:160}{
A \Hbar^2 l(l+1) + B \Hbar m \rightarrow
\Hbar l(l+1)
\lr{
A \Hbar
-\frac{C^2}{2 B}
}
+ B \Hbar m
+\frac{C^2 m^2 \Hbar}{2 B}.
}

\paragraph{Exact perturbation equation}

If we wanted to solve the Hamiltonian exactly, we've have to diagonalize the \( 2 m + 1 \) dimensional Hamiltonian
%
\begin{dmath}\label{eqn:LyPerturbation:180}
\bra{l, m'} H \ket{l, m}
=
\lr{ A \Hbar^2 l(l+1) + B \Hbar m } \delta_{m', m}
+
\frac{C \Hbar}{2 i}
\lr{
\sqrt{(l - m)(l + m + 1)} \delta_{m', m + 1}
-
\sqrt{(l + m)(l - m + 1)} \delta_{m', m - 1}
}.
\end{dmath}
%
This Hamiltonian matrix has a very regular structure
%
\begin{equation}\label{eqn:LyPerturbation:200}
\begin{aligned}
H &=
(A l(l+1) \Hbar^2 - B \Hbar (l+1)) I \\
&+ B \Hbar
\begin{bmatrix}
1 &   &   &        &          \\
  & 2 &   &        &          \\
  &   & 3 &        &          \\
  &   &   & \ddots &          \\
  &   &   &        & 2 l + 1
\end{bmatrix} \\
&+
\frac{C \Hbar}{i}
%\begin{bmatrix}
%0                & -\sqrt{(2l-1)(1)} &                  &                    & \\
%\sqrt{(2l-1)(1)} & 0                 & -\sqrt{(2l-2)(2)}&                    & \\
%                 & \sqrt{(2l-2)(2)}  &                  &                    & \\
%                 &                   & \ddots           &                    & \\
%&                 &                   &  0               & - \sqrt{(1)(2l-1)}  \\
%&                 &                   & \sqrt{(1)(2l-1)} & 0
%\end{bmatrix}
\begin{bmatrix}
0                & -c_{2l-1,1} &                  &                    & \\
c_{2l-1,1} & 0                 & -c_{2l-2,2}&                    & \\
                 & c_{2l-2,2}  &                  &                    & \\
                 &                   & \ddots           &                    & \\
&                 &                   &  0               & - c_{1,2l-1}  \\
&                 &                   & c_{1,2l-1} & 0
\end{bmatrix},
\end{aligned}
\end{equation}
where \( c_{a, b} = \sqrt{ a b } \).

Solving for the eigenvalues of this Hamiltonian for increasing \( l \) in Mathematica (\nbref{sakuraiProblem5.17a.nb}), it appears that the eigenvalues are
%
\begin{equation}\label{eqn:LyPerturbation:220}
\lambda_m = A \Hbar^2 (l)(l+1) + \Hbar m B \sqrt{ 1 + \frac{4 C^2}{B^2} },
\end{dmath}
%
so to first order in \( C^2 \), these are
%
\begin{equation}\label{eqn:LyPerturbation:221}
\lambda_m = A \Hbar^2 (l)(l+1) + \Hbar m B \lr{ 1 + \frac{2 C^2}{B^2} }.
\end{dmath}
%
We have a \( C^2 \Hbar/B \) term in both the perturbative energy shift
\cref{eqn:LyPerturbation:140}, and the first order expansion of the exact solution
\cref{eqn:LyPerturbation:220}.
Comparing this for the \( l = 5 \) case, the coefficients of \( C^2 \Hbar/B \) in \cref{eqn:LyPerturbation:140} are all negative
\begin{equation}\label{eqn:LyPerturbation:241}
-17.5, -17., -16.5, -16., -15.5, -15., -14.5, -14., -13.5, -13., -12.5,
\end{dmath}
whereas the coefficient of \( C^2 \Hbar/B \) in the first order expansion of the exact solution \cref{eqn:LyPerturbation:220} are \( 2 m \), ranging from \( [-10, 10] \).
} % answer

%\EndArticle
