%
% Copyright � 2015 Peeter Joot.  All Rights Reserved.
% Licenced as described in the file LICENSE under the root directory of this GIT repository.
%
%\input{../blogpost.tex}
%\renewcommand{\basename}{positionCommutator}
%\renewcommand{\dirname}{notes/phy1520/}
%%\newcommand{\dateintitle}{}
%%\newcommand{\keywords}{}
%
%\input{../peeter_prologue_print2.tex}
%
%\usepackage{peeters_layout_exercise}
%\usepackage{peeters_braket}
%\usepackage{peeters_figures}
%
%\beginArtNoToc
%
%\generatetitle{Heisenberg picture position commutator}
%\chapter{Heisenberg picture position commutator}
%\label{chap:positionCommutator}


\makeoproblem{Heisenberg picture position commutator.}{problem:positionCommutator:2.5}{\citep{sakurai2014modern} pr. 2.5}{
\index{Heisenberg picture}
Evaluate
%
\begin{dmath}\label{eqn:positionCommutator:20}
\antisymmetric{x(t)}{x(0)},
\end{dmath}
%
for a Heisenberg picture operator \( x(t) \) for a free particle.
} % problem

\makeanswer{problem:positionCommutator:2.5}{
%
The free particle Hamiltonian is
%
\begin{dmath}\label{eqn:positionCommutator:40}
H = \frac{p^2}{2m},
\end{dmath}
%
so the time evolution operator is
%
\begin{dmath}\label{eqn:positionCommutator:60}
U(t) = e^{-i p^2 t/(2 m \Hbar)}.
\end{dmath}
%
The Heisenberg picture position operator is
%
\begin{dmath}\label{eqn:positionCommutator:80}
x^\txtH
= U^\dagger x U
= e^{i p^2 t/(2 m \Hbar)} x e^{-i p^2 t/(2 m \Hbar)}
= \sum_{k = 0}^\infty \inv{k!} \lr{ \frac{i p^2 t}{2 m \Hbar} }^k
x
e^{-i p^2 t/(2 m \Hbar)}
= \sum_{k = 0}^\infty \inv{k!} \lr{ \frac{i t}{2 m \Hbar} }^k p^{2k} x
e^{-i p^2 t/(2 m \Hbar)}
=
\sum_{k = 0}^\infty \inv{k!} \lr{ \frac{i t}{2 m \Hbar} }^k \lr{ \antisymmetric{p^{2k}}{x} + x p^{2k} }
e^{-i p^2 t/(2 m \Hbar)}
= x +
\sum_{k = 0}^\infty \inv{k!} \lr{ \frac{i t}{2 m \Hbar} }^k \antisymmetric{p^{2k}}{x}
e^{-i p^2 t/(2 m \Hbar)}
= x +
\sum_{k = 0}^\infty \inv{k!} \lr{ \frac{i t}{2 m \Hbar} }^k \lr{ -i \Hbar \PD{p}{p^{2k}} }
e^{-i p^2 t/(2 m \Hbar)}
= x +
\sum_{k = 0}^\infty \inv{k!} \lr{ \frac{i t}{2 m \Hbar} }^k \lr{ -i \Hbar 2 k p^{2 k -1} }
e^{-i p^2 t/(2 m \Hbar)}
= x +
-2 i \Hbar p \frac{i t}{2 m \Hbar} \sum_{k = 1}^\infty \inv{(k-1)!} \lr{ \frac{i t}{2 m \Hbar} }^{k-1} p^{2(k - 1)}
e^{-i p^2 t/(2 m \Hbar)}
= x + t \frac{p}{m}.
\end{dmath}
%
This has the structure of a classical free particle \( x(t) = x + v t \), but in this case \( x,p \) are operators.

The evolved position commutator is
\begin{dmath}\label{eqn:positionCommutator:100}
\antisymmetric{x(t)}{x(0)}
=
\antisymmetric{x + t p/m}{x}
=
\frac{t}{m} \antisymmetric{p}{x}
=
-i \Hbar \frac{t}{m}.
\end{dmath}
%
Compare this to the classical Poisson bracket
\begin{dmath}\label{eqn:positionCommutator:120}
\antisymmetric{x(t)}{x(0)}_{\textrm{classical}}
=
\PD{x}{}\lr{x + p t/m} \PD{p}{x} - \PD{p}{}\lr{x + p t/m} \PD{x}{x}
=
- \frac{t}{m}.
\end{dmath}
%
This has the expected relation \( \antisymmetric{x(t)}{x(0)} = i \Hbar \antisymmetric{x(t)}{x(0)}_{\textrm{classical}} \).

} % answer

%\EndArticle
