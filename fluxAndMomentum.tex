%
% Copyright � 2015 Peeter Joot.  All Rights Reserved.
% Licenced as described in the file LICENSE under the root directory of this GIT repository.
%
%\input{../blogpost.tex}
%\renewcommand{\basename}{fluxAndMomentum}
%\renewcommand{\dirname}{notes/phy1520/}
%\newcommand{\dateintitle}{}
%\newcommand{\keywords}{}

%\input{../peeter_prologue_print2.tex}

%\usepackage{peeters_layout_exercise}
%\usepackage{peeters_braket}
%\usepackage{peeters_figures}
%
%\beginArtNoToc

%\generatetitle{Relation of probability flux to momentum}
%\chapter{Relation of probability flux to momentum}
%\label{chap:fluxAndMomentum}
%
\makeproblem{Relation of probability flux to momentum.}{problem:fluxAndMomentum:1}{
\index{probability!flux}
\index{momentum!expectation}

Show that the probability flux
%
\begin{equation}\label{eqn:fluxAndMomentum:20}
\Bj(\Bx, t) = -\frac{i\Hbar}{2 m} \lr{ \psi^\conj \spacegrad \psi - \psi \spacegrad \psi^\conj },
\end{dmath}
%
is related to the momentum expectation at a given time by the integral of the flux over all space
%
\begin{equation}\label{eqn:fluxAndMomentum:40}
\int d^3 x \Bj(\Bx, t) = \frac{\expectation{\Bp}_t}{m}.
\end{dmath}
%
} % problem
%
\makeanswer{problem:fluxAndMomentum:1}{
%
This can be seen by recasting the integral in bra-ket form.  Let
%
\begin{equation}\label{eqn:fluxAndMomentum:60}
\psi(\Bx, t) = \braket{\Bx}{\psi(t)},
\end{dmath}
%
and note that the momentum portions of the flux can be written as
%
\begin{equation}\label{eqn:fluxAndMomentum:80}
-i \Hbar \spacegrad \psi(\Bx, t) = \bra{\Bx} \Bp \ket{\psi(t)}.
\end{dmath}
%
The current is therefore
%
\begin{dmath}\label{eqn:fluxAndMomentum:100}
\Bj(\Bx, t)
= \frac{1}{2 m}
\lr{
\psi^\conj \bra{\Bx} \Bp \ket{\psi(t)}
+\psi {\bra{\Bx} \Bp \ket{\psi(t)} }^\conj
}
= \frac{1}{2 m}
\lr{
{\braket{\Bx}{\psi(t)}}^\conj \bra{\Bx} \Bp \ket{\psi(t)}
+ \braket{\Bx}{\psi(t)} {\bra{\Bx} \Bp \ket{\psi(t)} }^\conj
}
= \frac{1}{2 m}
\lr{
\braket{\psi(t)}{\Bx} \bra{\Bx} \Bp \ket{\psi(t)}
+
\bra{\psi(t)} \Bp \ket{\Bx} \braket{\Bx}{\psi(t)}
}.
\end{dmath}
%
Integrating and noting that the spatial identity is \( 1 = \int d^3 x \ket{\Bx}\bra{\Bx} \), we have
%
\begin{dmath}\label{eqn:fluxAndMomentum:120}
\int d^3 x \Bj(\Bx, t)
=
\bra{\psi(t)} \Bp \ket{\psi(t)},
\end{dmath}
%
This is just the expectation of \( \Bp \) with respect to a specific time-instance state, demonstrating the desired relationship.
} % answer

%\EndArticle
