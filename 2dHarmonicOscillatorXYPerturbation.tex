%
% Copyright � 2015 Peeter Joot.  All Rights Reserved.
% Licenced as described in the file LICENSE under the root directory of this GIT repository.
%
%\input{../blogpost.tex}
%\renewcommand{\basename}{2dHarmonicOscillatorXYPerturbation}
%\renewcommand{\dirname}{notes/phy1520/}
%%\newcommand{\dateintitle}{}
%%\newcommand{\keywords}{}
%
%\input{../peeter_prologue_print2.tex}
%
%\usepackage{peeters_layout_exercise}
%\usepackage{peeters_braket}
%\usepackage{peeters_figures}
%\usepackage{enumerate}
%
%\beginArtNoToc
%
%\generatetitle{2D SHO xy perturbation}
%\chapter{2D SHO xy perturbation}
%\label{chap:2dHarmonicOscillatorXYPerturbation}

\makeoproblem{2D SHO xy perturbation.}{problem:2dHarmonicOscillatorXYPerturbation:1}{\citep{sakurai2014modern} pr. 5.4}{
\index{harmonic oscillator!xy perturbation}

Given a 2D SHO with Hamiltonian
%
\begin{dmath}\label{eqn:2dHarmonicOscillatorXYPerturbation:20}
H_0 = \inv{2m} \lr{ p_x^2 + p_y^2 } + \frac{m \omega^2}{2} \lr{ x^2 + y^2 },
\end{dmath}
%
\makesubproblem{}{problem:2dHarmonicOscillatorXYPerturbation:1:a}
What are the energies and degeneracies of the three lowest states?

\makesubproblem{}{problem:2dHarmonicOscillatorXYPerturbation:1:b}
With perturbation
%
\begin{dmath}\label{eqn:2dHarmonicOscillatorXYPerturbation:40}
V = m \omega^2 x y,
\end{dmath}
%
calculate the first order energy perturbations and the zeroth order perturbed states.

\makesubproblem{}{problem:2dHarmonicOscillatorXYPerturbation:1:c}

Solve the \( H_0 + \delta V \) problem exactly, and compare.
} % problem

\makeanswer{problem:2dHarmonicOscillatorXYPerturbation:1}{

\makeSubAnswer{}{problem:2dHarmonicOscillatorXYPerturbation:1:a}


Recall that we have
%
\begin{dmath}\label{eqn:2dHarmonicOscillatorXYPerturbation:60}
H \ket{n_1, n_2} =
\Hbar\omega
\lr{
n_1 + n_2 + 1
}
\ket{n_1, n_2},
\end{dmath}
%
So the three lowest energy states are \( \ket{0,0}, \ket{1,0}, \ket{0,1} \) with energies \( \Hbar \omega, 2 \Hbar \omega, 2 \Hbar \omega \) respectively (with a two fold degeneracy for the second two energy eigenkets).

\makeSubAnswer{}{problem:2dHarmonicOscillatorXYPerturbation:1:b}
Consider the action of \( x y \) on the \( \beta = \setlr{ \ket{0,0}, \ket{1,0}, \ket{0,1} } \) subspace.  Those are
%
\begin{dmath}\label{eqn:2dHarmonicOscillatorXYPerturbation:200}
x y \ket{0,0}
=
\frac{x_0^2}{2} \lr{ a + a^\dagger } \lr{ b + b^\dagger } \ket{0,0}
=
\frac{x_0^2}{2} \lr{ b + b^\dagger } \ket{1,0}
=
\frac{x_0^2}{2} \ket{1,1}.
\end{dmath}
%
\begin{dmath}\label{eqn:2dHarmonicOscillatorXYPerturbation:220}
x y \ket{1, 0}
=
\frac{x_0^2}{2} \lr{ a + a^\dagger } \lr{ b + b^\dagger } \ket{1,0}
=
\frac{x_0^2}{2} \lr{ a + a^\dagger } \ket{1,1}
=
\frac{x_0^2}{2} \lr{ \ket{0,1} + \sqrt{2} \ket{2,1} } .
\end{dmath}
%
\begin{dmath}\label{eqn:2dHarmonicOscillatorXYPerturbation:240}
x y \ket{0, 1}
=
\frac{x_0^2}{2} \lr{ a + a^\dagger } \lr{ b + b^\dagger } \ket{0,1}
=
\frac{x_0^2}{2} \lr{ b + b^\dagger } \ket{1,1}
=
\frac{x_0^2}{2} \lr{ \ket{1,0} + \sqrt{2} \ket{1,2} }.
\end{dmath}
%
%<row|A|column>
The matrix representation of \( m \omega^2 x y \) with respect to the subspace spanned by basis \( \beta \) above is
%
\begin{dmath}\label{eqn:2dHarmonicOscillatorXYPerturbation:260}
x y
\sim
\inv{2} \Hbar \omega
\begin{bmatrix}
0 & 0 & 0 \\
0 & 0 & 1 \\
0 & 1 & 0 \\
\end{bmatrix}.
\end{dmath}

This diagonalizes with
\begin{subequations}
\label{eqn:2dHarmonicOscillatorXYPerturbation:280}
\begin{dmath}\label{eqn:2dHarmonicOscillatorXYPerturbation:300}
U =
\begin{bmatrix}
1 & 0  \\
0 & \tilde{U}
\end{bmatrix}
\end{dmath}
\begin{dmath}\label{eqn:2dHarmonicOscillatorXYPerturbation:320}
\tilde{U}
=
\inv{\sqrt{2}}
\begin{bmatrix}
1 & 1 \\
1 & -1 \\
\end{bmatrix}
\end{dmath}
\begin{dmath}\label{eqn:2dHarmonicOscillatorXYPerturbation:340}
D =
\inv{2} \Hbar \omega
\begin{bmatrix}
0 & 0 & 0 \\
0 & 1 & 0 \\
0 & 0 & -1 \\
\end{bmatrix}
\end{dmath}
\begin{equation}\label{eqn:2dHarmonicOscillatorXYPerturbation:360}
x y = U D U^\dagger = U D U.
\end{equation}
\end{subequations}

The unperturbed Hamiltonian in the original basis is
%
\begin{dmath}\label{eqn:2dHarmonicOscillatorXYPerturbation:380}
H_0
=
\Hbar \omega
\begin{bmatrix}
1 & 0 \\
0 & 2 I
\end{bmatrix},
\end{dmath}
%
So the transformation to the diagonal \( x y \) basis leaves the initial Hamiltonian unaltered
%
\begin{dmath}\label{eqn:2dHarmonicOscillatorXYPerturbation:400}
H_0'
= U^\dagger H_0 U
=
\Hbar \omega
\begin{bmatrix}
1 & 0  \\
0 & \tilde{U} 2 I \tilde{U}
\end{bmatrix}
=
\Hbar \omega
\begin{bmatrix}
1 & 0  \\
0 & 2 I
\end{bmatrix}.
\end{dmath}

Now we can compute the first order energy shifts almost by inspection.  Writing the new basis as \( \beta' = \setlr{ \ket{0}, \ket{1}, \ket{2} } \) those energy shifts are just the diagonal elements from the \( x y \) operators matrix representation
%
\begin{dmath}\label{eqn:2dHarmonicOscillatorXYPerturbation:420}
\begin{aligned}
E^{{(1)}}_0 &= \bra{0} V \ket{0} = 0 \\
E^{{(1)}}_1 &= \bra{1} V \ket{1} = \inv{2} \Hbar \omega \\
E^{{(1)}}_2 &= \bra{2} V \ket{2} = -\inv{2} \Hbar \omega.
\end{aligned}
\end{dmath}

The new energies are
%
\begin{dmath}\label{eqn:2dHarmonicOscillatorXYPerturbation:440}
\begin{aligned}
E_0 &\rightarrow \Hbar \omega \\
E_1 &\rightarrow \Hbar \omega \lr{ 2 + \delta/2 } \\
E_2 &\rightarrow \Hbar \omega \lr{ 2 - \delta/2 }.
\end{aligned}
\end{dmath}

\makeSubAnswer{}{problem:2dHarmonicOscillatorXYPerturbation:1:c}

For the exact solution, it's possible to rotate the coordinate system in a way that kills the explicit \( x y \) term of the perturbation.  That we could do this for \( x, y \) operators wasn't obvious to me, but after doing so (and rotating the momentum operators the same way) the new operators still have the required commutators.  Let
%
\begin{dmath}\label{eqn:2dHarmonicOscillatorXYPerturbation:80}
\begin{bmatrix}
u \\
v
\end{bmatrix}
=
\begin{bmatrix}
\cos\theta & \sin\theta \\
-\sin\theta & \cos\theta
\end{bmatrix}
\begin{bmatrix}
x \\
y
\end{bmatrix}
=
\begin{bmatrix}
x \cos\theta + y \sin\theta \\
-x \sin\theta + y \cos\theta
\end{bmatrix}.
\end{dmath}

Similarly, for the momentum operators, let
\begin{dmath}\label{eqn:2dHarmonicOscillatorXYPerturbation:100}
\begin{bmatrix}
p_u \\
p_v
\end{bmatrix}
=
\begin{bmatrix}
\cos\theta & \sin\theta \\
-\sin\theta & \cos\theta
\end{bmatrix}
\begin{bmatrix}
p_x \\
p_y
\end{bmatrix}
=
\begin{bmatrix}
p_x \cos\theta + p_y \sin\theta \\
-p_x \sin\theta + p_y \cos\theta
\end{bmatrix}.
\end{dmath}

For the commutators of the new operators we have
%
\begin{dmath}\label{eqn:2dHarmonicOscillatorXYPerturbation:120}
\antisymmetric{u}{p_u}
=
\antisymmetric{x \cos\theta + y \sin\theta}{p_x \cos\theta + p_y \sin\theta}
=
\antisymmetric{x}{p_x} \cos^2\theta + \antisymmetric{y}{p_y} \sin^2\theta
=
i \Hbar \lr{ \cos^2\theta + \sin^2\theta }
=
i\Hbar.
\end{dmath}
%
\begin{dmath}\label{eqn:2dHarmonicOscillatorXYPerturbation:140}
\antisymmetric{v}{p_v}
=
\antisymmetric{-x \sin\theta + y \cos\theta}{-p_x \sin\theta + p_y \cos\theta}
=
\antisymmetric{x}{p_x} \sin^2\theta + \antisymmetric{y}{p_y} \cos^2\theta
=
i \Hbar.
\end{dmath}
%
\begin{dmath}\label{eqn:2dHarmonicOscillatorXYPerturbation:160}
\antisymmetric{u}{p_v}
=
\antisymmetric{x \cos\theta + y \sin\theta}{-p_x \sin\theta + p_y \cos\theta}
= \cos\theta \sin\theta \lr{ -\antisymmetric{x}{p_x} + \antisymmetric{y}{p_p} }
=
0.
\end{dmath}
%
\begin{dmath}\label{eqn:2dHarmonicOscillatorXYPerturbation:180}
\antisymmetric{v}{p_u}
=
\antisymmetric{-x \sin\theta + y \cos\theta}{p_x \cos\theta + p_y \sin\theta}
= \cos\theta \sin\theta \lr{ -\antisymmetric{x}{p_x} + \antisymmetric{y}{p_p} }
=
0.
\end{dmath}

We see that the new operators are canonical conjugate as required.  For this problem, we just want a 45 degree rotation, with
%
\begin{equation}\label{eqn:2dHarmonicOscillatorXYPerturbation:460}
\begin{aligned}
x &= \inv{\sqrt{2}} \lr{ u + v } \\
y &= \inv{\sqrt{2}} \lr{ u - v }.
\end{aligned}
\end{equation}

We have
\begin{dmath}\label{eqn:2dHarmonicOscillatorXYPerturbation:480}
x^2 + y^2
=
\inv{2} \lr{ (u+v)^2 + (u-v)^2 }
=
\inv{2} \lr{ 2 u^2 + 2 v^2 + 2 u v - 2 u v }
=
u^2 + v^2,
\end{dmath}
%
\begin{dmath}\label{eqn:2dHarmonicOscillatorXYPerturbation:500}
p_x^2 + p_y^2
=
\inv{2} \lr{ (p_u+p_v)^2 + (p_u-p_v)^2 }
=
\inv{2} \lr{ 2 p_u^2 + 2 p_v^2 + 2 p_u p_v - 2 p_u p_v }
=
p_u^2 + p_v^2,
\end{dmath}
%
and
\begin{dmath}\label{eqn:2dHarmonicOscillatorXYPerturbation:520}
x y
=
\inv{2} \lr{ (u+v)(u-v) }
=
\inv{2} \lr{ u^2 - v^2 }.
\end{dmath}

The perturbed Hamiltonian is
%
\begin{dmath}\label{eqn:2dHarmonicOscillatorXYPerturbation:540}
H_0 + \delta V
=
\inv{2m} \lr{ p_u^2 + p_v^2 }
+ \inv{2} m \omega^2 \lr{ u^2 + v^2 + \delta u^2 - \delta v^2 }
=
\inv{2m} \lr{ p_u^2 + p_v^2 }
+ \inv{2} m \omega^2 \lr{ u^2(1 + \delta) + v^2 (1 - \delta) }.
\end{dmath}

In this coordinate system, the corresponding eigensystem is
%
\begin{dmath}\label{eqn:2dHarmonicOscillatorXYPerturbation:560}
H \ket{n_1, n_2}
= \Hbar \omega \lr{ 1 + n_1 \sqrt{1 + \delta} + n_2 \sqrt{ 1 - \delta } } \ket{n_1, n_2}.
\end{dmath}
%
For small \( \delta \)
%
\begin{dmath}\label{eqn:2dHarmonicOscillatorXYPerturbation:580}
n_1 \sqrt{1 + \delta} + n_2 \sqrt{ 1 - \delta }
\approx
n_1 + n_2
+ \inv{2} n_1 \delta
- \inv{2} n_2 \delta,
\end{dmath}
%
so
\begin{dmath}\label{eqn:2dHarmonicOscillatorXYPerturbation:600}
H \ket{n_1, n_2}
\approx \Hbar \omega \lr{ 1 + n_1 + n_2 + \inv{2} n_1 \delta - \inv{2} n_2 \delta
} \ket{n_1, n_2}.
\end{dmath}
%
The lowest order perturbed energy levels are
%
\begin{dmath}\label{eqn:2dHarmonicOscillatorXYPerturbation:620}
\ket{0,0} \rightarrow \Hbar \omega
\end{dmath}
\begin{dmath}\label{eqn:2dHarmonicOscillatorXYPerturbation:640}
\ket{1,0} \rightarrow \Hbar \omega \lr{ 2 + \inv{2} \delta }
\end{dmath}
\begin{dmath}\label{eqn:2dHarmonicOscillatorXYPerturbation:660}
\ket{0,1} \rightarrow \Hbar \omega \lr{ 2 - \inv{2} \delta }
\end{dmath}
%
The degeneracy of the \( \ket{0,1}, \ket{1,0} \) states has been split, and to first order match the zeroth order perturbation result.
} % answer

%\EndArticle
