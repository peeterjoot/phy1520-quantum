%
% Copyright � 2015 Peeter Joot.  All Rights Reserved.
% Licenced as described in the file LICENSE under the root directory of this GIT repository.
%
\makeoproblem{Density matrix.}{gradQuantum:problemSet1:1}{phy1520 2015 ps1.1}{
\index{density matrix}

Consider a spin-1/2 particle. The Hilbert space is two-dimensional, let us label the two states as \( \ket{\uparrow} \) and \( \ket{\downarrow} \).
Write down the \( 2 \times 2 \) density matrix which corresponds to the following pure states.

\begin{enumerate}[(i)]
\item \( \ket{\uparrow} \)
\item \( \inv{\sqrt{2}} \lr{ \ket{\uparrow} + \ket{\downarrow} } \)
\item \( \inv{\sqrt{2}} \lr{ \ket{\uparrow} + i \ket{\downarrow} } \)
\item At time \( t = 0 \), let us start with the state \( \inv{\sqrt{2}} \lr{ \ket{\uparrow} + i \ket{\downarrow} } \),
and consider time-evolution under the Hamiltonian \( \hatH = - B S_z \),
where \( S_z \) is the z-component of the spin operator.
This leads to eigenstates \( \ket{\uparrow} \) with energy \( -B \Hbar/2 \), and
\( \ket{\downarrow} \)
with energy \( +B \Hbar/2 \).
The state
\( \inv{\sqrt{2}} \lr{ \ket{\uparrow} + i \ket{\downarrow} } \)
is not an eigenstate of this Hamiltonian and it will evolve in time.
Find the state and the corresponding \( 2 \times 2 \) density matrix of this system at a later time \( t \).
\end{enumerate}

} % makeproblem

\makeanswer{gradQuantum:problemSet1:1}{
\withproblemsetsParagraph{
\begin{enumerate}[(i)]
\item
This state in matrix form is
%
\begin{dmath}\label{eqn:gradQuantumProblemSet1Problem1:20}
\ket{\psi} =
\begin{bmatrix}
1 \\
0
\end{bmatrix},
\end{dmath}
%
for which the density operator representation is
%
\begin{dmath}\label{eqn:gradQuantumProblemSet1Problem1:40}
\hat\rho
=
\begin{bmatrix}
1 \\
0
\end{bmatrix}
\begin{bmatrix}
1 &
0
\end{bmatrix}
=
\begin{bmatrix}
1 & 0 \\
0 & 0
\end{bmatrix}.
\end{dmath}
%
\item

This state in matrix form is
%
\begin{dmath}\label{eqn:gradQuantumProblemSet1Problem1:60}
\ket{\psi} =
\inv{\sqrt{2}}
\lr{
\begin{bmatrix}
1 \\
0
\end{bmatrix}
+
\begin{bmatrix}
0 \\
1
\end{bmatrix}
}
=
\inv{\sqrt{2}}
\begin{bmatrix}
1 \\
1
\end{bmatrix}
\end{dmath}

for which the density operator representation is
%
\begin{dmath}\label{eqn:gradQuantumProblemSet1Problem1:80}
\hat\rho
=
\inv{2}
\begin{bmatrix}
1 \\
1
\end{bmatrix}
\begin{bmatrix}
1 &
1
\end{bmatrix}
=
\inv{2}
\begin{bmatrix}
1 & 1 \\
1 & 1
\end{bmatrix}.
\end{dmath}
%
\item

This state in matrix form is
%
\begin{dmath}\label{eqn:gradQuantumProblemSet1Problem1:100}
\ket{\psi} =
\inv{\sqrt{2}}
\lr{
\begin{bmatrix}
1 \\
0
\end{bmatrix}
+
\begin{bmatrix}
0 \\
i
\end{bmatrix}
}
=
\inv{\sqrt{2}}
\begin{bmatrix}
1 \\
i
\end{bmatrix}
\end{dmath}

for which the density operator representation is
%
\begin{dmath}\label{eqn:gradQuantumProblemSet1Problem1:120}
\hat\rho
=
\inv{2}
\begin{bmatrix}
1 \\
i
\end{bmatrix}
\begin{bmatrix}
1 &
-i
\end{bmatrix}
=
\inv{2}
\begin{bmatrix}
1 & -i \\
i & 1
\end{bmatrix}.
\end{dmath}
%
\item

The time evolution operator is
%
\begin{dmath}\label{eqn:gradQuantumProblemSet1Problem1:140}
U(t)
= e^{-i \hatH t/\Hbar}
= e^{i B S_z t/\Hbar}
= e^{i B \sigma_z t/2}
= \cos( B t/2 ) + i \sigma_z \sin( B t/2)
=
\begin{bmatrix}
\cos(B t/2) & 0 \\
0 & \cos(B t/2) \\
\end{bmatrix}
+
\begin{bmatrix}
i \sin(B t/2) & 0 \\
0 & -i \sin(B t/2) \\
\end{bmatrix}
=
\begin{bmatrix}
e^{i B t/2} & 0 \\
0 & e^{-i B t/2}
\end{bmatrix}.
\end{dmath}
%
so the time evolved state is
%
\begin{dmath}\label{eqn:gradQuantumProblemSet1Problem1:160}
\ket{\psi(t)} =
\inv{\sqrt{2}}
\begin{bmatrix}
e^{i B t/2} & 0 \\
0 & e^{-i B t/2}
\end{bmatrix}
\begin{bmatrix}
1 \\
i
\end{bmatrix}
=
\inv{\sqrt{2}}
\begin{bmatrix}
e^{i B t/2} \\
i e^{-i B t/2}
\end{bmatrix}.
\end{dmath}
%
The density operator matrix representation is
%
\begin{dmath}\label{eqn:gradQuantumProblemSet1Problem1:180}
\hat\rho(t)
=
\inv{2}
\begin{bmatrix}
e^{i B t/2} \\
i e^{-i B t/2}
\end{bmatrix}
\begin{bmatrix}
e^{-i B t/2} & -i e^{i B t/2}
\end{bmatrix}
=
\inv{2}
\begin{bmatrix}
1 & -i e^{i B t} \\
i e^{-i B t} & 1
\end{bmatrix}.
\end{dmath}
%
As a check observe that this has the right value for \( t = 0 \).  This also checks against the slightly messier computation \( \hatp(t) = U \hatp(0) U^\dagger \).

\end{enumerate}
}
}
