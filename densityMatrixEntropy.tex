%
% Copyright � 2015 Peeter Joot.  All Rights Reserved.
% Licenced as described in the file LICENSE under the root directory of this GIT repository.
%
%\input{../blogpost.tex}
%\renewcommand{\basename}{densityMatrixEntropy}
%\renewcommand{\dirname}{notes/phy1520/}
%%\newcommand{\dateintitle}{}
%%\newcommand{\keywords}{}
%
%\input{../peeter_prologue_print2.tex}
%
%\usepackage{peeters_layout_exercise}
%\usepackage{peeters_braket}
%\usepackage{peeters_figures}
%
%\beginArtNoToc
%
%\generatetitle{Entropy when density operator has zero eigenvalues}
%%\label{chap:densityMatrixEntropy}

In the class notes and the text \citep{sakurai2014modern} the Von Neumann entropy is defined as
%
\begin{dmath}\label{eqn:densityMatrixEntropy:20}
S = -\tr(\rho \ln \rho).
\end{dmath}

In one of our problems I had trouble evaluating this, having calculated a density operator matrix representation
%
\begin{dmath}\label{eqn:densityMatrixEntropy:40}
\rho = E \wedge E^{-1},
\end{dmath}
%
where
%
\begin{dmath}\label{eqn:densityMatrixEntropy:60}
E = \inv{\sqrt{2}}
\begin{bmatrix}
1 & 1 \\
1 & -1
\end{bmatrix},
\end{dmath}
%
and
\begin{dmath}\label{eqn:densityMatrixEntropy:100}
\wedge =
\begin{bmatrix}
1 & 0 \\
0 & 0
\end{bmatrix}.
\end{dmath}

The usual method of evaluating a function of a matrix is to assume the function has a power series representation, and that a similarity transformation of the form \( A = E \wedge E^{-1} \) is possible, so that
%
\begin{dmath}\label{eqn:densityMatrixEntropy:80}
f(A) = E f(\wedge) E^{-1},
\end{dmath}
%
however, when attempting to do this with the matrix of \cref{eqn:densityMatrixEntropy:40} leads to an undesirable result
%
\begin{dmath}\label{eqn:densityMatrixEntropy:120}
\ln \rho =
\inv{2}
\begin{bmatrix}
1 & 1 \\
1 & -1
\end{bmatrix}
\begin{bmatrix}
\ln 1 & 0 \\
0 & \ln 0
\end{bmatrix}
\begin{bmatrix}
1 & 1 \\
1 & -1
\end{bmatrix}.
\end{dmath}

The \( \ln 0 \) makes the evaluation of this matrix logarithm rather unpleasant.  To give meaning to the entropy expression, we have to do two things, the first is treating the trace operation as a higher precedence than the logarithms that it contains.  That is
%
\begin{dmath}\label{eqn:densityMatrixEntropy:140}
-\tr ( \rho \ln \rho )
=
-\tr ( E \wedge E^{-1} E \ln \wedge E^{-1} )
=
-\tr ( E \wedge \ln \wedge E^{-1} )
=
-\tr ( E^{-1} E \wedge \ln \wedge )
=
-\tr ( \wedge \ln \wedge )
=
- \sum_k \wedge_{kk} \ln \wedge_{kk}.
\end{dmath}

Now the matrix of the logarithm need not be evaluated, but we still need to give meaning to \( \wedge_{kk} \ln \wedge_{kk} \) for zero diagonal entries.  This can be done by considering a limiting scenario
%
\begin{dmath}\label{eqn:densityMatrixEntropy:160}
-\lim_{a \rightarrow 0} a \ln a
=
-\lim_{x \rightarrow \infty} e^{-x} \ln e^{-x}
=
\lim_{x \rightarrow \infty} x e^{-x}
=
0.
\end{dmath}

The entropy can now be expressed in the unambiguous form, summing over all the non-zero eigenvalues of the density operator
%
%\begin{dmath}\label{eqn:densityMatrixEntropy:180}
\boxedEquation{eqn:densityMatrixEntropy:180}{
S = - \sum_{ \wedge_{kk} \ne 0} \wedge_{kk} \ln \wedge_{kk}.
}
%\end{dmath}

%\EndArticle
