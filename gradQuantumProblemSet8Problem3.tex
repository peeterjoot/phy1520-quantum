%
% Copyright � 2015 Peeter Joot.  All Rights Reserved.
% Licenced as described in the file LICENSE under the root directory of this GIT repository.
%
\makeoproblem{Harmonic oscillator.}{gradQuantum:problemSet8:3}{2015 ps8 p3}{
\index{harmonic oscillator!anharmonic perturbation}

Consider a 2D harmonic oscillator with

\begin{dmath}\label{eqn:gradQuantumProblemSet8Problem3:20}
H =
\frac{p_x^2}{2m}
+\frac{p_y^2}{2m}
+ \inv{2} m \omega^2 \lr{ x^2 + y^2 }
\end{dmath}

Turn on an anharmonic perturbation

\begin{dmath}\label{eqn:gradQuantumProblemSet8Problem3:40}
V = \lambda g_1 \frac{m^2 \omega^3}{\Hbar} \lr{ x^4 + y^4 } + \lambda^2 g_2 m \omega^2 x y,
\end{dmath}

Note that the potentials have been altered from the original problem statement to have dimensions of energy with dimensionless scale factors \( g_1, g_2, \lambda \).

\makesubproblem{}{gradQuantum:problemSet8:3a}

Derive the equations for the energy shifts and perturbed states for a second order perturbing potential of the form above.

\makesubproblem{}{gradQuantum:problemSet8:3b}

Find the perturbed eigenstate and the corresponding energy shifts up to \( O(\lambda^2) \) for the ground state.  Ignore terms of \( O(\lambda^3) \).

\makesubproblem{}{gradQuantum:problemSet8:3c}

Do the same for the first two degenerate states.

} % makeproblem

\makeanswer{gradQuantum:problemSet8:3}{
\withproblemsetsParagraph{
\makeSubAnswer{}{gradQuantum:problemSet8:3a}

With a \( \lambda^2 \) perturbation we have to step back and revisit the derivation of the energy level and perturbed state formulas.  Given

\begin{dmath}\label{eqn:gradQuantumProblemSet8Problem3:60}
H = H_0 + \lambda V_1 + \lambda^2 V_2,
\end{dmath}

with known solution \( H_0 \ket{n^{(0)}} = E^{(0)} \ket{n^{(0)}} \), we seek the a power series representation of the perturbed ket and an energy shift \( \Delta \)

\begin{dmath}\label{eqn:gradQuantumProblemSet8Problem3:80}
\ket{n} = \ket{n_0} + \lambda \ket{n_1} + \lambda^2 \ket{n_2} + \cdots
\end{dmath}
\begin{dmath}\label{eqn:gradQuantumProblemSet8Problem3:100}
\Delta = \lambda \Delta^{(1)} + \lambda^2 \Delta^{(2)} + \cdots
\end{dmath}

where

\begin{dmath}\label{eqn:gradQuantumProblemSet8Problem3:120}
H \ket{n} = \lr{ E^{(0)} + \Delta } \ket{n}.
\end{dmath}

We can assume that the we have the same sort of representation of the perturbed state

\begin{dmath}\label{eqn:gradQuantumProblemSet8Problem3:140}
\ket{n} = \ket{n^{(0)}} + \frac{\overbar{P}_n}{E^{(0)} - H_0} \lr{ \lambda_1 V_1 + \lambda^2 V_2 - \Delta } \ket{n},
\end{dmath}

where

\begin{dmath}\label{eqn:gradQuantumProblemSet8Problem3:160}
\overbar{P}_n = 1 - \ket{n^{(0)}}\bra{n^{(0)}} = \sum_{m \ne n} \ket{m^{(0)}}\bra{m^{(0)}}.
\end{dmath}

To check this, operating with \( E^{(0)} - H_0 \), we have

\begin{dmath}\label{eqn:gradQuantumProblemSet8Problem3:180}
\lr{ E^{(0)} - H_0 } \ket{n}
=
\lr{ E^{(0)} - H_0 } \ket{n^{(0)}} +
\overbar{P}_n \lr{ \lambda_1 V_1 + \lambda^2 V_2 - \Delta } \ket{n}
=
\lr{ 1 - \ket{n^{(0)}}\bra{n^{(0)}} } \lr{ H - H_0 - \Delta } \ket{n},
\end{dmath}

or
\begin{dmath}\label{eqn:gradQuantumProblemSet8Problem3:200}
\lr{ E^{(0)} - H + \Delta} \ket{n}
=
-\ket{n^{(0)}}\bra{n^{(0)}} \lr{ H - H_0 - \Delta } \ket{n}
=
-\ket{n^{(0)}} \lr{
\lr{ E^{(0)} + \Delta} - E^{(0)} -\Delta } \braket{n^{(0)}}{n}
= 0.
\end{dmath}

The LHS is also zero as desired, showing that \cref{eqn:gradQuantumProblemSet8Problem3:140} is the desired perturbation relationship.
For the perturbed state we are looking for just the \( \lambda^1 \) terms of \cref{eqn:gradQuantumProblemSet8Problem3:140}, which after dropping all second order and higher terms is

\begin{equation}\label{eqn:gradQuantumProblemSet8Problem3:420}
\ket{n_0} + \lambda \ket{n_1} = \ket{n_0} + \sum_{m \ne n} \frac{\ket{m^{(0)}} \bra{m^{(0)}}}{E^{(0)} - E_m} \lr{ \lambda V_1 - \lambda \Delta^{(1)} } \lr{ \ket{n_0} + \lambda \ket{n_1} },
\end{equation}

so the first order state perturbation is

\begin{equation}\label{eqn:gradQuantumProblemSet8Problem3:440}
\ket{n_1} = \sum_{m \ne n} \frac{\ket{m^{(0)}} \bra{m^{(0)}}}{E^{(0)} - E_m} \lr{ V_1 - \Delta^{(1)} } \ket{n_0}.
\end{equation}

The \( \Delta^{(1)} \) contribution drops out, leaving

\boxedEquation{eqn:gradQuantumProblemSet8Problem3:460}{
\ket{n_1} = \sum_{m \ne n} \frac{\ket{m^{(0)}} \bra{m^{(0)}}}{E^{(0)} - E_m} V_1 \ket{n_0},
}

just as we had for a strictly first order perturbing potential.

For the energy shifts consider the braket

\begin{dmath}\label{eqn:gradQuantumProblemSet8Problem3:220}
\bra{n^{(0)}} H -H_0 \ket{n}
=
\bra{n^{(0)}} { V_1 \Delta^{(1)} + V_2 \Delta^{(2)} } \ket{n}
=
\lr{ \lr{ E^{(0)} + \Delta } -E^{(0)} } \braket{n^{(0)}}{n}
=
\Delta \braket{n^{(0)}}{n},
\end{dmath}

or
%As with the a first order \( \lambda \) perturbation, we can impose a requirement that \( \braket{0^{(0)}}{n} = 1 \), so

\begin{dmath}\label{eqn:gradQuantumProblemSet8Problem3:240}
\Delta \braket{n^{(0)}}{n} = \bra{n^{(0)}} { V_1 \Delta^{(1)} + V_2 \Delta^{(2)} } \ket{n}.
\end{dmath}

Expanding both sides in powers of \( \lambda \) we have

\begin{dmath}\label{eqn:gradQuantumProblemSet8Problem3:260}
\sum_{r = 1, s = 0} \lambda^{r+s} \Delta^{(r)} \braket{n^{(0)}}{n_s}
=
\sum_{m = 0} \lambda^{m+1} \bra{n^{(0)}} V_1 \ket{n_m} + \lambda^{m+2} \bra{n^{(0)}} V_2 \ket{n_m}
\end{dmath}

With \( \ket{n^{(0)}} = \ket{n_0} \) as required in the \( \lambda \rightarrow 0 \) limit, the \( \lambda = 1 \) contribution to these sums is

%\begin{dmath}\label{eqn:gradQuantumProblemSet8Problem3:280}
\boxedEquation{eqn:gradQuantumProblemSet8Problem3:300}{
\Delta^{(1)} = \bra{n_0} V_1 \ket{n_0}.
}
%\end{dmath}

The second order contribution is

\begin{dmath}\label{eqn:gradQuantumProblemSet8Problem3:320}
\Delta^{(1)} \braket{n^{(0)}}{n_1} + \Delta^{(2)} \braket{n^{(0)}}{n_0}
=
\bra{n^{(0)}} V_1 \ket{n_1} + \bra{n^{(0)}} V_2 \ket{n_0},
\end{dmath}

or
\begin{equation}\label{eqn:gradQuantumProblemSet8Problem3:380}
\Delta^{(2)} = \bra{n_0} V_1 \ket{n_1} + \bra{n_0} V_2 \ket{n_0} - \bra{n_0} V_1 \ket{n_0} \braket{n_0}{n_1}.
\end{equation}

%We can write this as
%\begin{equation}\label{eqn:gradQuantumProblemSet8Problem3:360}
%%\boxedEquation{eqn:gradQuantumProblemSet8Problem3:400}{
%\begin{aligned}
%\Delta^{(2)} &= \bra{n_0} V_1 \ket{n_1}_\perp + \bra{n_0} V_2 \ket{n_0} \\
%\ket{n_1}_\perp &= \biglr{ 1 - \ket{n_0}\bra{n_0} } \ket{n_1},
%\end{aligned}
%%}
%\end{equation}
%
%where \( \ket{n_1}_\perp \) is the rejection of \( \ket{n_0} \) from the first order perturbed state \( \ket{n_1} \), the portions of \( \ket{n_1} \) that are orthogonal to \( \ket{n_0} \).
However, from \cref{eqn:gradQuantumProblemSet8Problem3:460} we see that \( \ket{n_1} \) has no \( \ket{n_0} \) component, this means the second order shift is just

\boxedEquation{eqn:gradQuantumProblemSet8Problem3:500}{
\Delta^{(2)} = \bra{n_0} V_1 \ket{n_1} + \bra{n_0} V_2 \ket{n_0}.
}

\makeSubAnswer{}{gradQuantum:problemSet8:3b}

In \nbref{ps8:2dHarmonicOscillatorOperators.nb}, for
an initial state \( \ket{n_0} = \ket{0,0} \), it is calculated that

\begin{dmath}\label{eqn:gradQuantumProblemSet8Problem3:480}
V_1 \ket{0,0}
=
g_1 \frac{\Hbar \omega}{x_0^4} \lr{ x^4 + y^4 } \ket{0,0}
=
g_1 \Hbar \omega
\lr{
\frac{3 }{2} \ket{0,0}
+
\frac{3 }{\sqrt{2}} \lr{ \ket{2,0} + \ket{0,2} }
+
\sqrt{\frac{3}{2}} \lr{ \ket{4,0} + \ket{0,4} }
}.
\end{dmath}

The first energy shift is

\begin{dmath}\label{eqn:gradQuantumProblemSet8Problem3:520}
\Delta^{(1)} = \bra{0,0} V_1 \ket{0,0} = \frac{3}{2} g_1 \Hbar \omega,
\end{dmath}

and the first order perturbation of the state is

\begin{dmath}\label{eqn:gradQuantumProblemSet8Problem3:540}
\ket{n_1} = g_1 \Hbar \omega
\lr{
\frac{3}{\sqrt{2}} \frac{\ket{2,0} + \ket{0,2} }{ \Hbar \omega \lr{ 1 + 0 + 0 } - \Hbar \omega \lr{ 1 + 2 + 0} }
+
\sqrt{\frac{3}{2}} \frac{ \ket{4,0} + \ket{0,4} }{ \Hbar \omega \lr{ 1 + 0 + 0 } - \Hbar \omega \lr{ 1 + 4 + 0} }
},
\end{dmath}

or
\begin{dmath}\label{eqn:gradQuantumProblemSet8Problem3:560}
\ket{n_1} = -g_1
\lr{
\frac{3}{2 \sqrt{2}} \lr{\ket{2,0} + \ket{0,2} }
+
\inv{4} \sqrt{\frac{3}{2}} \lr{ \ket{4,0} + \ket{0,4} }
},
\end{dmath}

We can calculate
\begin{dmath}\label{eqn:gradQuantumProblemSet8Problem3:580}
\begin{aligned}
\bra{0,0} V_2 \ket{0,0} &= 0 \\
\bra{0,0} V_1 \ket{n_1} &= -\frac{21}{4} \Hbar \omega g_1^2,
\end{aligned}
\end{dmath}

so the second order energy shift is
\begin{dmath}\label{eqn:gradQuantumProblemSet8Problem3:600}
\Delta^{(2)} = -\frac{21}{4} \Hbar \omega g_1^2,
\end{dmath}

so the ground state energy shift, to second order in \( \lambda \), is

%\begin{dmath}\label{eqn:gradQuantumProblemSet8Problem3:620}
\boxedEquation{eqn:gradQuantumProblemSet8Problem3:680}{
\Hbar \omega \rightarrow \Hbar \omega + \frac{3}{2} \Hbar \omega g_1 \lambda - \frac{21}{4} \Hbar \omega g_1^2 \lambda^2.
}
%\end{dmath}

For the ground state, there is no contribution from the second order potential \( \lambda^2 V_2 \).

\makeSubAnswer{}{gradQuantum:problemSet8:3c}

The next two highest states are \( \ket{1,0}, \ket{0,1} \) both with unperturbed energy eigenvalues \( 2 \Hbar \omega \).  For a basis spanning the \( \setlr{ \ket{1,0}, \ket{0,1} } \) subspace, the matrix element of the perturbed Hamiltonian is

\begin{dmath}\label{eqn:gradQuantumProblemSet8Problem3:640}
H_0 + \lambda V_1 + \lambda^2 V_2
=
2 \Hbar \omega I
+ \frac{9}{2} g_1 \lambda \Hbar \omega I
+ \frac{1}{2} g_2 \lambda^2 \Hbar \omega \sigma_1
=
\Hbar \omega
\begin{bmatrix}
2 + \frac{9}{2} g_1 \lambda & \frac{1}{2} g_2 \lambda^2 \\
\frac{1}{2} g_2 \lambda^2 & 2 + \frac{9}{2} g_1 \lambda
\end{bmatrix}.
\end{dmath}

Since the eigenvalues of \(
\begin{bmatrix}
a & b \\
b & a
\end{bmatrix} \) are just \( a \pm b \), the energy splitting to first order for these first two degenerate states is

%\begin{equation}\label{eqn:gradQuantumProblemSet8Problem3:660}
\boxedEquation{eqn:gradQuantumProblemSet8Problem3:700}{
2 \Hbar \omega \rightarrow \Hbar \omega \lr{ 2 + \frac{9}{2} g_1 \lambda \pm \frac{g_2 \lambda^2 }{2} }.
}
%\end{equation}

While the \( x y \) perturbation potential left the ground state untouched, it is responsible for the energy level splitting for the degenerate states \( \ket{1,0}, \ket{1,0} \).
}
}
