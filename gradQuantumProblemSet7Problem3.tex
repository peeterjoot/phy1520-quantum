%
% Copyright � 2015 Peeter Joot.  All Rights Reserved.
% Licenced as described in the file LICENSE under the root directory of this GIT repository.
%
% desai 24.3
\makeoproblem{Harmonic oscillator variation.}{gradQuantum:problemSet7:3}{\citep{desai2009quantum} pr. 24.3}
%\makeproblem{Harmonic oscillator variation}{gradQuantum:problemSet7:3}
{
\index{harmonic oscillator!variational method}
%\makesubproblem{}{gradQuantum:problemSet7:3a}
%
Consider a 1D harmonic oscillator with an unnormalized trial wavefunction \( \psi_v(x) = e^{-\beta \Abs{x}} \).
Minimize the ground state energy with respect to \( \beta \), thus obtaining the optimal \( \beta \) as well as the variational ground state energy.
Compare with the exact result.
Note that you need to be careful evaluating derivatives since the wavefunction has a `cusp' at \( x = 0\).
} % makeproblem
%
\makeanswer{gradQuantum:problemSet7:3}{
\withproblemsetsParagraph{
%\makeSubAnswer{}{gradQuantum:problemSet7:3a}
%
In order to make the derivatives of the trial function better behaved at the origin, we can treat it as a distribution, writing
%
\begin{dmath}\label{eqn:gradQuantumProblemSet7Problem3:20}
\psi(x) = \Theta(x) e^{-\beta x} + \Theta(-x) e^{\beta x},
\end{dmath}
%
This has a derivative
%
\begin{dmath}\label{eqn:gradQuantumProblemSet7Problem3:40}
\psi'(x)
=
\delta(x) e^{-\beta x} - \delta(-x) e^{\beta x}
+ \beta \lr{ -\Theta(x) e^{-\beta x} + \Theta(-x) e^{\beta x} }
=
-2 \delta(x) \sinh( \beta x )
+ \beta \lr{ -\Theta(x) e^{-\beta x} + \Theta(-x) e^{\beta x} }.
\end{dmath}
%
Using \( \delta'(x) = - \delta(x)/x \), the second derivative is
%
\begin{dmath}\label{eqn:gradQuantumProblemSet7Problem3:60}
\psi''(x)
=
+2 \delta(x) \sinh( \beta x )/x
-2 \beta \delta(x) \cosh( \beta x )
+ \beta \lr{ -\delta(x) e^{-\beta x} - \delta(-x) e^{\beta x} }
+ \beta^2 \lr{ \Theta(x) e^{-\beta x} + \Theta(-x) e^{\beta x} }
=
+2 \delta(x) \beta \lr{ \frac{\sinh( \beta x )}{\beta x} - \cosh( \beta x ) }
- 2 \beta \delta(x) \cosh( \beta x )
+ \beta^2 e^{-\beta \Abs{x} }.
\end{dmath}
%
Because
%
\begin{dmath}\label{eqn:gradQuantumProblemSet7Problem3:80}
\int \delta(x) \cosh( \beta x ) f(x) dx = f(0),
\end{dmath}
%
and
%
\begin{dmath}\label{eqn:gradQuantumProblemSet7Problem3:100}
\int \delta(x) \lr{ \frac{\sinh( \beta x )}{\beta x} - \cosh( \beta x ) } f(x) dx
= \lr{ 1 - 1 } f(0)
= 0,
\end{dmath}
%
this second derivative can be simplified to
\begin{dmath}\label{eqn:gradQuantumProblemSet7Problem3:120}
\psi''(x)
=
- 2 \beta \delta(x) + \beta^2 e^{-\beta \Abs{x} }.
\end{dmath}
%
This has the \( \beta^2 \psi(x) \) value that we expect at points away from the origin.  All the expectations can now be computed.  The normalization is
%
\begin{dmath}\label{eqn:gradQuantumProblemSet7Problem3:140}
\braket{\psi}{\psi}
=
2 \int_0^\infty e^{-2 \beta x} dx
=
2 \evalrange{\frac{e^{-2 \beta x}}{-2 \beta}}{0}{\infty}
=
\inv{\beta}.
\end{dmath}
%
Observe that
%
\begin{dmath}\label{eqn:gradQuantumProblemSet7Problem3:160}
\frac{d^2}{d\beta^2} \int_0^\infty e^{-2 \beta x} dx
=
(-2)^2 \int_0^\infty x^2 e^{-2 \beta x} dx
=
\frac{d}{d\beta} \lr{ -\inv{2 \beta^2}}
=
\frac{1}{\beta^3},
\end{dmath}
%
so
\begin{dmath}\label{eqn:gradQuantumProblemSet7Problem3:180}
\int_0^\infty x^2 e^{-2 \beta x} dx = \frac{1}{4 \beta^3},
\end{dmath}
%
a result we will need later.  The kinetic portion of the energy expectation is
%
\begin{dmath}\label{eqn:gradQuantumProblemSet7Problem3:200}
\bra{\psi} \frac{p^2}{2m} \ket{\psi}
=
-\frac{\Hbar^2}{2m}
\int_{-\infty}^\infty e^{-\beta x} \lr{ - 2 \beta \delta(x) + \beta^2 e^{-\beta \Abs{x} } }
=
-\frac{\Hbar^2}{2m} \beta^2 \inv{\beta} -\frac{\Hbar^2}{2m} \lr{ - 2 \beta }
= \frac{\Hbar^2}{2m} \beta.
\end{dmath}
%
The potential portion of the energy expectation is
\begin{dmath}\label{eqn:gradQuantumProblemSet7Problem3:220}
\bra{\psi} \inv{2} m \omega^2 x^2 \ket{\psi}
=
m \omega^2 \int_0^\infty x^2 e^{-2 \beta x} dx
=
m \omega^2 \frac{1}{4 \beta^3}.
\end{dmath}
%
Adding things up we have
\begin{dmath}\label{eqn:gradQuantumProblemSet7Problem3:240}
\overbar{E}(\beta)
=
\frac{\bra{\psi} H \ket{\psi}}{\braket{\psi}{\psi}}
=
\frac{\frac{\Hbar^2}{2m} \beta
+ \frac{ m \omega^2}{4 \beta^3}}{\inv{\beta}}
=
\frac{\Hbar^2}{2m} \beta^2
+ \frac{ m \omega^2}{4 \beta^2}
=
\frac{\Hbar \omega}{2}
\lr{
\frac{\Hbar}{m \omega} \beta^2 +
\frac{ m \omega}{2 \Hbar \beta^2}
}
=
\frac{\Hbar \omega}{2}
\lr{
x_0^2 \beta^2 + \inv{2 x_0^2 \beta^2}
}.
\end{dmath}
%
Minimizing gives
%
\begin{dmath}\label{eqn:gradQuantumProblemSet7Problem3:260}
0
=
\frac{d}{d\beta}
\lr{
x_0^2 \beta^2 + \inv{2 x_0^2 \beta^2}
}
=
2 x_0^2 \beta - \inv{x_0^2 \beta^3},
\end{dmath}
%
or
%
\begin{dmath}\label{eqn:gradQuantumProblemSet7Problem3:280}
\beta^4 = \inv{2 x_0^4},
\end{dmath}
%
which gives
\boxedEquation{eqn:gradQuantumProblemSet7Problem3:300}{
\beta = \inv{2^{1/4} x_0}.
}

The energy at this value of \( \beta \) is
%
\begin{dmath}\label{eqn:gradQuantumProblemSet7Problem3:320}
\overbar{E}_{\textrm{min}}
=
\frac{\Hbar \omega}{2}
\lr{
x_0^2 \inv{\sqrt{2} x_0^2} + \frac{ \sqrt{2} x_0^2}{2 x_0^2 }
}
=
\frac{\Hbar \omega}{2}
\frac{2}{ \sqrt{2} },
\end{dmath}
%
or
\boxedEquation{eqn:gradQuantumProblemSet7Problem3:340}{
\overbar{E}_{\textrm{min}}
=
\frac{\Hbar \omega}{2} \sqrt{2} > \Hbar \omega \lr{ 0 + \inv{2} }.
}

We find that the trial function that minimizes the average energy is
%
\begin{dmath}\label{eqn:gradQuantumProblemSet7Problem3:360}
\psi(x) = 2^{1/4} x_0 \exp\lr{ -2^{-1/4} \Abs{x}/x_0 },
\end{dmath}
%
with an average energy that is \( 1.41 \times \) the actual ground state energy for the harmonic oscillator.

}
}
