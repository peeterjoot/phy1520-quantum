%
% Copyright � 2015 Peeter Joot.  All Rights Reserved.
% Licenced as described in the file LICENSE under the root directory of this GIT repository.
%
%\input{../blogpost.tex}
%\renewcommand{\basename}{shoMomentumSpace}
%\renewcommand{\dirname}{notes/phy1520/}
%%\newcommand{\dateintitle}{}
%%\newcommand{\keywords}{}
%
%\input{../peeter_prologue_print2.tex}
%
%\usepackage{peeters_layout_exercise}
%\usepackage{peeters_braket}
%\usepackage{peeters_figures}
%\usepackage{macros_qed}
%
%\beginArtNoToc
%
%\generatetitle{Momentum space representation of Schr\"{o}dinger equation}
%\chapter{Momentum space representation of Schr\"{o}dinger equation}
%\label{chap:shoMomentumSpace}
%
\makeoproblem
%{Momentum space representation of Schr\"{o}dinger equation.}
{Momentum space Schr\"{o}dinger eq.}
{problem:shoMomentumSpace:15}{\citep{sakurai2014modern} pr. 2.15}{
\index{Schr\"{o}dinger equation!momentum space}

Using
%
\begin{equation}\label{eqn:shoMomentumSpace:20}
\braket{x'}{p'} = \inv{\sqrt{2 \pi \Hbar}} e^{i p' x'/\Hbar},
\end{equation}
%
show that
%
\begin{equation}\label{eqn:shoMomentumSpace:40}
\bra{p'} x \ket{\alpha} = i \Hbar \PD{p'}{} \braket{p'}{\alpha}.
\end{equation}
%
Use this to find the momentum space representation of the Schr\"{o}dinger equation for the one dimensional SHO and the energy eigenfunctions in their momentum representation.
%
} % problem
%
\makeanswer{problem:shoMomentumSpace:15}{
To expand the matrix element, introduce both momentum and position space identity operators
%
\begin{dmath}\label{eqn:shoMomentumSpace:60}
\bra{p'} x \ket{\alpha}
=
\int dx' dp'' \braket{p'}{x'}\bra{x'}x \ket{p''}\braket{p''}{\alpha}
=
\int dx' dp'' \braket{p'}{x'}x'\braket{x'}{p''}\braket{p''}{\alpha}
=
\inv{2 \pi \Hbar}
\int dx' dp'' e^{-i p' x'/\Hbar} x' e^{i p'' x'/\Hbar} \braket{p''}{\alpha}
=
\inv{2 \pi \Hbar}
\int dx' dp'' x' e^{i (p'' - p') x'/\Hbar} \braket{p''}{\alpha}
=
\inv{2 \pi \Hbar}
\int dx' dp'' \frac{d}{dp''}\lr{ \frac{-i \Hbar} e^{i (p'' - p') x'/\Hbar} } \braket{p''}{\alpha}
=
i \Hbar
\int dp''
\lr{ \inv{2 \pi \Hbar}
\int dx' e^{i (p'' - p') x'/\Hbar} } \frac{d}{dp''} \braket{p''}{\alpha}
=
i \Hbar
\int dp'' \delta(p''- p')
\frac{d}{dp''} \braket{p''}{\alpha}
=
i \Hbar
\frac{d}{dp'} \braket{p'}{\alpha}. \qedmarker
\end{dmath}

Schr\"{o}dinger's equation for a time dependent state \( \ket{\alpha} = U(t) \ket{\alpha,0} \) is
%
\begin{equation}\label{eqn:shoMomentumSpace:80}
i \Hbar \PD{t}{} \ket{\alpha} = H \ket{\alpha},
\end{equation}
%
with the momentum representation
%
\begin{equation}\label{eqn:shoMomentumSpace:100}
i \Hbar \PD{t}{} \braket{p'}{\alpha} = \bra{p'} H \ket{\alpha}.
\end{equation}
%
Expansion of the Hamiltonian matrix element for a strictly spatial dependent potential \( V(x) \) gives
%
\begin{dmath}\label{eqn:shoMomentumSpace:120}
\bra{p'} H \ket{\alpha}
=
\bra{p'} \lr{\frac{p^2}{2m} + V(x) } \ket{\alpha}
=
\frac{(p')^2}{2m}
+ \bra{p'} V(x) \ket{\alpha}.
\end{dmath}
%
Assuming a Taylor representation of the potential \( V(x) = \sum c_k x^k \), we want to calculate
%
\begin{equation}\label{eqn:shoMomentumSpace:140}
\bra{p'} V(x) \ket{\alpha}
= \sum c_k \bra{p'} x^k \ket{\alpha}.
\end{equation}
%
With \( \ket{\alpha} = \ket{p''} \) \cref{eqn:shoMomentumSpace:40} provides the \( k = 1 \) term
%
\begin{dmath}\label{eqn:shoMomentumSpace:160}
\bra{p'} x \ket{p''}
= i \Hbar \frac{d}{dp'} \braket{p'}{p''}
= i \Hbar \frac{d}{dp'} \delta(p' - p''),
\end{dmath}
%
where it is implied here that the derivative is operating on not just the delta function, but on all else that follows.

Using this the higher powers of \( \bra{p'} x^k \ket{\alpha} \) can be found easily.  For example for \( k = 2 \) we have
%
\begin{dmath}\label{eqn:shoMomentumSpace:180}
\bra{p'} x^2 \ket{\alpha}
=
\int dp''
\bra{p'} x \ket{p''}\bra{p''} x \ket{\alpha}
=
\int dp''
i \Hbar
\frac{d}{dp'} \delta(p' - p'') i \Hbar \frac{d}{dp''} \braket{p''}{\alpha}
=
\lr{ i \Hbar }^2 \frac{d^2}{d(p')^2} \braket{p'}{\alpha}.
\end{dmath}
%
This means that the potential matrix element is
%
\begin{dmath}\label{eqn:shoMomentumSpace:200}
\bra{p'} V(x) \ket{\alpha}
=
\sum c_k \lr{ i \Hbar \frac{d}{dp'} }^k \braket{p'}{\alpha}
= V\lr{ i \Hbar \frac{d}{dp'} }.
\end{dmath}
%
With 
\begin{equation}\label{eqn:shoMomentumSpace:219}
\Psi_\alpha(p') = \braket{p'}{\alpha},
\end{equation}
the momentum space representation of Schr\"{o}dinger's equation for a position dependent potential is
%
%\begin{dmath}\label{eqn:shoMomentumSpace:220}
\boxedEquation{eqn:shoMomentumSpace:220}{
i \Hbar \PD{t}{} \Psi_\alpha(p')
=
\lr{ \frac{(p')^2}{2m} + V\lr{ i \Hbar \PDi{p'}{} } } \Psi_\alpha(p').
}
%\end{dmath}
%
For the SHO Hamiltonian the potential is \( V(x) = (1/2) m \omega^2 x^2 \), so the Schr\"{o}dinger equation is
%
\begin{dmath}\label{eqn:shoMomentumSpace:240}
i \Hbar \PD{t}{} \Psi_\alpha(p')
=
\lr{ \frac{(p')^2}{2m} - \inv{2} m \omega^2 \Hbar^2 \frac{\partial^2}{\partial(p')^2} } \Psi_\alpha(p')
=
\inv{2 m} \lr{ (p')^2 - m^2 \omega^2 \Hbar^2 \frac{\partial^2}{\partial(p')^2} } \Psi_\alpha(p').
\end{dmath}
%
To determine the wave functions, let's non-dimensionalize this and compare to the position space Schr\"{o}dinger equation.  Let
%
\begin{equation}\label{eqn:shoMomentumSpace:260}
p_0^2 = m \omega \hbar,
\end{equation}
%
so
\begin{dmath}\label{eqn:shoMomentumSpace:280}
i \Hbar \PD{t}{} \Psi_\alpha(p')
=
\frac{p_0^2}{2 m} \lr{ \lr{\frac{p'}{p_0}}^2 - \frac{\partial^2}{\partial(p'/p_0)^2} } \Psi_\alpha(p')
=
\frac{\omega \Hbar}{2}\lr{
- \frac{\partial^2}{\partial(p'/p_0)^2} +
\lr{\frac{p'}{p_0}}^2
} \Psi_\alpha(p').
\end{dmath}
%
Compare this to the position space equation with \( x_0^2 = m \omega/\Hbar \),
%
\begin{dmath}\label{eqn:shoMomentumSpace:300}
i \Hbar \PD{t}{} \Psi_\alpha(x')
=
\lr{ -\frac{\Hbar^2}{2m} \frac{\partial^2}{\partial(x')^2}
+
\inv{2} m \omega^2 (x')^2 }
\Psi_\alpha(x')
=
\frac{\Hbar^2}{2m}
\lr{ -\frac{\partial^2}{\partial(x')^2}
+
\frac{m^2 \omega^2}{\Hbar^2} (x')^2 }
\Psi_\alpha(x')
=
\frac{\Hbar^2 x_0^2}{2m}
\lr{
-\frac{\partial^2}{\partial(x'/x_0)^2}
+
\lr{\frac{x'}{x_0}}^2
}
\Psi_\alpha(x')
=
\frac{\Hbar \omega}{2}
\lr{
-\frac{\partial^2}{\partial(x'/x_0)^2}
+
\lr{\frac{x'}{x_0}}^2
}
\Psi_\alpha(x').
\end{dmath}
%
It's clear that there is a straightforward duality relationship between the respective wave functions.  Since
%
\begin{equation}\label{eqn:shoMomentumSpace:320}
\braket{x'}{n} =
\inv{\pi^{1/4} \sqrt{2^n n!} x_0^{n + 1/2}}  \lr{ x' - x_0^2 \frac{d}{dx'} }^n \exp\lr{ -\inv{2} \lr{\frac{x'}{x_0}}^2 },
\end{equation}
%
\index{wave function!momentum space}
the momentum space wave functions are
%
\begin{equation}\label{eqn:shoMomentumSpace:340}
\braket{p'}{n} =
\inv{\pi^{1/4} \sqrt{2^n n!} p_0^{n + 1/2}}  \lr{ p' - p_0^2 \frac{d}{dp'} }^n \exp\lr{ -\inv{2} \lr{\frac{p'}{p_0}}^2 }.
\end{equation}
%
} % answer

%\EndArticle
