%
% Copyright � 2015 Peeter Joot.  All Rights Reserved.
% Licenced as described in the file LICENSE under the root directory of this GIT repository.
%
%\input{../blogpost.tex}
%\renewcommand{\basename}{shoSuperposition}
%\renewcommand{\dirname}{notes/phy1520/}
%%\newcommand{\dateintitle}{}
%%\newcommand{\keywords}{}
%
%\input{../peeter_prologue_print2.tex}
%
%\usepackage{peeters_layout_exercise}
%\usepackage{peeters_braket}
%\usepackage{peeters_figures}
%
%\beginArtNoToc
%
%\generatetitle{1D SHO linear superposition that maximizes expectation}
%%\chapter{1D SHO linear superposition that maximizes expectation}
%%\label{chap:shoSuperposition}

\makeoproblem{1D SHO linear superposition that maximizes expectation.}{problem:shoSuperposition:1}{\citep{sakurai2014modern} pr. 2.17}{
%
For a 1D SHO

\makesubproblem{}{problem:shoSuperposition:1:a}
%
Construct a linear combination of \( \ket{0}, \ket{1} \) that maximizes \( \expectation{x} \) without using wave functions.

\makesubproblem{}{problem:shoSuperposition:1:b}
%
How does this state evolve with time?

\makesubproblem{}{problem:shoSuperposition:1:c}
%
Evaluate \( \expectation{x} \) using the Schr\"{o}dinger picture.

\makesubproblem{}{problem:shoSuperposition:1:d}
%
Evaluate \( \expectation{x} \) using the Heisenberg picture.

\makesubproblem{}{problem:shoSuperposition:1:e}
%
Evaluate \( \expectation{(\Delta x)^2} \).

} % problem

\makeanswer{problem:shoSuperposition:1}{
%
\makeSubAnswer{}{problem:shoSuperposition:1:a}
%
Forming
%
\begin{dmath}\label{eqn:shoSuperposition:20}
\ket{\psi} = \frac{\ket{0} + \sigma \ket{1}}{\sqrt{1 + \Abs{\sigma}^2}}
\end{dmath}

the position expectation is
%
\begin{dmath}\label{eqn:shoSuperposition:40}
\bra{\psi} x \ket{\psi}
=
\inv{1 + \Abs{\sigma}^2} \lr{ \bra{0} + \sigma^\conj \bra{1} } \frac{x_0}{\sqrt{2}} \lr{ a^\dagger + a } \lr{ \ket{0} + \sigma \ket{1} }.
\end{dmath}
%
Evaluating the action of the operators on the kets, we've got
%
\begin{dmath}\label{eqn:shoSuperposition:60}
\lr{ a^\dagger + a } \lr{ \ket{0} + \sigma \ket{1} }
=
\ket{1} + \sqrt{2} \sigma \ket{2} + \sigma \ket{0}.
\end{dmath}
%
The \( \ket{2} \) term is killed by the bras, leaving
%
\begin{dmath}\label{eqn:shoSuperposition:80}
\expectation{x}
=
\inv{1 + \Abs{\sigma}^2} \frac{x_0}{\sqrt{2}} \lr{ \sigma + \sigma^\conj}
=
\frac{\sqrt{2} x_0 \Real \sigma}{1 + \Abs{\sigma}^2}.
\end{dmath}
%
Any imaginary component in \( \sigma \) will reduce the expectation, so we are constrained to picking a real value.

The derivative of
%
\begin{dmath}\label{eqn:shoSuperposition:100}
f(\sigma) = \frac{\sigma}{1 + \sigma^2},
\end{dmath}
%
is
%
\begin{dmath}\label{eqn:shoSuperposition:120}
f'(\sigma) = \frac{1 - \sigma^2}{(1 + \sigma^2)^2}.
\end{dmath}
%
That has zeros at \( \sigma = \pm 1 \).  The second derivative is
%
\begin{dmath}\label{eqn:shoSuperposition:140}
f''(\sigma) = \frac{-2 \sigma (3 - \sigma^2)}{(1 + \sigma^2)^3}.
\end{dmath}
%
That will be negative (maximum for the extreme value) at \( \sigma = 1 \), so the linear superposition of these first two energy eigenkets that maximizes the position expectation is
%
\begin{dmath}\label{eqn:shoSuperposition:160}
\psi = \inv{\sqrt{2}}\lr{ \ket{0} + \ket{1} }.
\end{dmath}
%
That maximized position expectation is
%
\begin{dmath}\label{eqn:shoSuperposition:180}
\expectation{x}
=
\frac{x_0}{\sqrt{2}}.
\end{dmath}
%
\makeSubAnswer{}{problem:shoSuperposition:1:b}
%
The time evolution is given by
%
\begin{dmath}\label{eqn:shoSuperposition:200}
\ket{\Psi(t)}
= e^{-i H t/\Hbar} \inv{\sqrt{2}}\lr{ \ket{0} + \ket{1} }
= \inv{\sqrt{2}}\lr{ e^{-i(0+ \ifrac{1}{2})\Hbar \omega t/\Hbar} \ket{0} + e^{-i(1+ \ifrac{1}{2})\Hbar \omega t/\Hbar} \ket{1} }
= \inv{\sqrt{2}}\lr{ e^{-i \omega t/2} \ket{0} + e^{-3 i \omega t/2} \ket{1} }.
\end{dmath}
%
\makeSubAnswer{}{problem:shoSuperposition:1:c}
%
The position expectation in the Schr\"{o}dinger representation is
%
\begin{dmath}\label{eqn:shoSuperposition:220}
\expectation{x(t)}
=
\inv{2}
\lr{ e^{i \omega t/2} \bra{0} + e^{3 i \omega t/2} \bra{1} } \frac{x_0}{\sqrt{2}} \lr{ a^\dagger + a }
\lr{ e^{-i \omega t/2} \ket{0} + e^{-3 i \omega t/2} \ket{1} }
=
\frac{x_0}{2\sqrt{2}}
\lr{ e^{i \omega t/2} \bra{0} + e^{3 i \omega t/2} \bra{1} }
\lr{ e^{-i \omega t/2} \ket{1} + e^{-3 i \omega t/2} \sqrt{2} \ket{2} + e^{-3 i \omega t/2} \ket{0} }
=
\frac{x_0}{\sqrt{2}} \cos(\omega t).
\end{dmath}
%
\makeSubAnswer{}{problem:shoSuperposition:1:d}
%
\begin{dmath}\label{eqn:shoSuperposition:240}
\expectation{x(t)}
=
\inv{2}
\lr{ \bra{0} + \bra{1} } \frac{x_0}{\sqrt{2}}
\lr{ a^\dagger e^{i\omega t} + a e^{-i \omega t} }
\lr{ \ket{0} + \ket{1} }
=
\frac{x_0}{2 \sqrt{2}}
\lr{ \bra{0} + \bra{1} }
\lr{ e^{i\omega t} \ket{1} + \sqrt{2} e^{i\omega t} \ket{2} + e^{-i \omega t} \ket{0} }
=
\frac{x_0}{\sqrt{2}} \cos(\omega t),
\end{dmath}
%
matching the calculation using the Schr\"{o}dinger picture.

\makeSubAnswer{}{problem:shoSuperposition:1:e}
%
Let's use the Heisenberg picture for the uncertainty calculation.  Using the calculation above we have
%
\begin{dmath}\label{eqn:shoSuperposition:260}
\expectation{x^2}
=
\inv{2} \frac{x_0^2}{2}
\lr{ e^{-i\omega t} \bra{1} + \sqrt{2} e^{-i\omega t} \bra{2} + e^{i \omega t} \bra{0} }
\lr{ e^{i\omega t} \ket{1} + \sqrt{2} e^{i\omega t} \ket{2} + e^{-i \omega t} \ket{0} }
=
\frac{x_0^2}{4} \lr{ 1 + 2 + 1}
=
x_0^2.
\end{dmath}
%
The uncertainty is
\begin{dmath}\label{eqn:shoSuperposition:280}
\expectation{(\Delta x)^2} =
\expectation{x^2} - \expectation{x}^2
=
x_0^2 - \frac{x_0^2}{2} \cos^2(\omega t)
=
\frac{x_0^2}{2} \lr{ 2 - \cos^2(\omega t) }
=
\frac{x_0^2}{2} \lr{ 1 + \sin^2(\omega t) }
\end{dmath}
} % answer

%\EndArticle
