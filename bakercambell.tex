%
% Copyright � 2015 Peeter Joot.  All Rights Reserved.
% Licenced as described in the file LICENSE under the root directory of this GIT repository.
%
%\input{../blogpost.tex}
%\renewcommand{\basename}{bakercambell}
%\renewcommand{\dirname}{notes/phy1520/}
%%\newcommand{\dateintitle}{}
%%\newcommand{\keywords}{}
%
%\input{../peeter_prologue_print2.tex}
%
%\usepackage{peeters_layout_exercise}
%\usepackage{peeters_braket}
%\usepackage{peeters_figures}
%\usepackage{macros_qed}
%
%\beginArtNoToc
%
%\generatetitle{A curious proof of the Baker-Campbell-Hausdorff formula}
%\label{chap:bakercambell}

Equation (39) of \citep{glauber1951some} states the Baker-Campbell-Hausdorff formula for two operators \( a, b\) that commute with their commutator \( \antisymmetric{a}{b} \)
%
\begin{equation}\label{eqn:bakercambell:20}
e^a e^b = e^{a + b + \antisymmetric{a}{b}/2},
\end{equation}
%
and provides the outline of an interesting method of proof.  That method is to consider the derivative of
%
\begin{equation}\label{eqn:bakercambell:40}
f(\lambda) = e^{\lambda a} e^{\lambda b} e^{-\lambda (a + b)},
\end{equation}
%
That derivative is
\begin{dmath}\label{eqn:bakercambell:60}
\frac{df}{d\lambda}
=
e^{\lambda a} a e^{\lambda b} e^{-\lambda (a + b)}
+
e^{\lambda a} b e^{\lambda b} e^{-\lambda (a + b)}
-
e^{\lambda a} b e^{\lambda b} (a + b)e^{-\lambda (a + b)}
=
e^{\lambda a} \lr{
a e^{\lambda b}
+
b e^{\lambda b}
-
e^{\lambda b} (a+b)
}
e^{-\lambda (a + b)}
=
e^{\lambda a} \lr{
\antisymmetric{a}{e^{\lambda b}}
+
\cancel{\antisymmetric{b}{e^{\lambda b}}}
}
e^{-\lambda (a + b)}
=
e^{\lambda a}
\antisymmetric{a}{e^{\lambda b}}
e^{-\lambda (a + b)}
.
\end{dmath}
%
The commutator above is proportional to \( \antisymmetric{a}{b} \)
%
\begin{dmath}\label{eqn:bakercambell:80}
\antisymmetric{a}{e^{\lambda b}}
=
\sum_{k=0}^\infty \frac{\lambda^k}{k!} \antisymmetric{a}{ b^k }
=
\sum_{k=0}^\infty \frac{\lambda^k}{k!} k b^{k-1} \antisymmetric{a}{b}
=
\lambda \sum_{k=1}^\infty \frac{\lambda^{k-1}}{(k-1)!} b^{k-1} \antisymmetric{a}{b}
=
\lambda e^{\lambda b} \antisymmetric{a}{b},
\end{dmath}
%
so
%
\begin{equation}\label{eqn:bakercambell:100}
\frac{df}{d\lambda} = \lambda \antisymmetric{a}{b} f.
\end{equation}
%
To get the above, we should also do the induction demonstration for \( \antisymmetric{a}{ b^k } = k b^{k-1} \antisymmetric{a}{b} \).
This clearly holds for \( k = 0,1 \).  For any other \( k \) we have
%
\begin{dmath}\label{eqn:bakercambell:120}
\antisymmetric{a}{b^{k+1}}
=
a b^{k+1} - b^{k+1} a
=
\lr{ \antisymmetric{a}{b^{k}} + b^k a
} b - b^{k+1} a
=
k b^{k-1} \antisymmetric{a}{b} b
+ b^k \lr{ \antisymmetric{a}{b} + \cancel{b a} }
- \cancel{b^{k+1} a}
=
k b^{k} \antisymmetric{a}{b}
+ b^k \antisymmetric{a}{b}
=
(k+1) b^k \antisymmetric{a}{b}. \qedmarker
\end{dmath}
Observe that \cref{eqn:bakercambell:100} is solved by
%
\begin{equation}\label{eqn:bakercambell:140}
f = e^{\lambda^2\antisymmetric{a}{b}/2},
\end{equation}
%
which gives
%
\begin{dmath}\label{eqn:bakercambell:160}
e^{\lambda^2 \antisymmetric{a}{b}/2} =
e^{\lambda a} e^{\lambda b} e^{-\lambda (a + b)}.
\end{dmath}
%
Right multiplication by \( e^{\lambda (a + b)} \) which commutes with \( e^{\lambda^2 \antisymmetric{a}{b}/2} \) and setting \( \lambda = 1 \) recovers \cref{eqn:bakercambell:20} as desired.

What I wonder looking at this, is what thought process led to trying this in the first place?  This is not what I would consider an obvious approach to demonstrating this identity.

%\EndArticle
