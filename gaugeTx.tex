%
% Copyright � 2015 Peeter Joot.  All Rights Reserved.
% Licenced as described in the file LICENSE under the root directory of this GIT repository.
%
%\input{../blogpost.tex}
%\renewcommand{\basename}{gaugeTx}
%\renewcommand{\dirname}{notes/phy1520/}
%%\newcommand{\dateintitle}{}
%%\newcommand{\keywords}{}
%
%\input{../peeter_prologue_print2.tex}
%
%\usepackage{peeters_layout_exercise}
%\usepackage{peeters_braket}
%\usepackage{peeters_figures}
%
%\beginArtNoToc
%
%\generatetitle{Gauge transformation of free particle Hamiltonian}
%\chapter{Gauge transformation}
%\label{chap:gaugeTx}
%
\makeoproblem{Gauge transformation, free Hamiltonian.}{problem:gaugeTx:0}{\citep{sakurai2014modern} pr. 37(a)}{
\index{gauge transformation}
\index{free particle}
Given a gauge transformation of the free particle Hamiltonian to
%
\begin{dmath}\label{eqn:gaugeTx:20}
H = \inv{2 m} \BPi \cdot \BPi + e \phi,
\end{dmath}
%
where
%
\begin{dmath}\label{eqn:gaugeTx:40}
\BPi = \Bp - \frac{e}{c} \BA,
\end{dmath}
%
calculate
\( m d\Bx/dt \),
\( \antisymmetric{\Pi_i}{\Pi_j} \), and
\( m d^2\Bx/dt^2 \), where
\( \Bx \) is the Heisenberg picture position operator, and the fields are functions only of position \( \phi = \phi(\Bx), \BA = \BA(\Bx) \).
} % problem
%
\makeanswer{problem:gaugeTx:0}{
The final results for these calculations are found in \citep{sakurai2014modern}, but seem worth deriving to exercise our commutator muscles.

\paragraph{Heisenberg picture velocity operator}

The first order of business is the Heisenberg picture velocity operator, but first note
%
\begin{dmath}\label{eqn:gaugeTx:60}
\BPi \cdot \BPi
= \lr{ \Bp - \frac{e}{c} \BA} \cdot \lr{ \Bp - \frac{e}{c} \BA}
= \Bp^2 - \frac{e}{c} \lr{ \BA \cdot \Bp + \Bp \cdot \BA } + \frac{e^2}{c^2} \BA^2.
\end{dmath}
%
The time evolution of the Heisenberg picture position operator is therefore
%
\begin{dmath}\label{eqn:gaugeTx:80}
\ddt{\Bx}
= \inv{i\Hbar} \antisymmetric{\Bx}{H}
= \inv{i\Hbar 2 m} \antisymmetric{\Bx}{\BPi^2}
= \inv{i\Hbar 2 m} \antisymmetric{\Bx}{\Bp^2 - \frac{e}{c} \lr{ \BA \cdot \Bp + \Bp \cdot \BA } + \frac{e^2}{c^2} \BA^2 }
= \inv{i\Hbar 2 m}
\lr{
\antisymmetric{\Bx}{\Bp^2}
- \frac{e}{c} \antisymmetric{\Bx}{ \BA \cdot \Bp + \Bp \cdot \BA }
}
.
\end{dmath}
%
For the \( \Bp^2 \) commutator we have
%
\begin{dmath}\label{eqn:gaugeTx:100}
\antisymmetric{x_r}{\Bp^2}
=
i \Hbar \PD{p_r}{\Bp^2}
=
2 i \Hbar p_r,
\end{dmath}
%
or
\begin{dmath}\label{eqn:gaugeTx:120}
\antisymmetric{\Bx}{\Bp^2}
=
2 i \Hbar \Bp.
\end{dmath}
%
Computing the remaining commutator, we've got
%
\begin{equation}\label{eqn:gaugeTx:140}
\begin{aligned}
\antisymmetric{x_r}{\Bp \cdot \BA + \BA \cdot \Bp}
&= x_r p_s A_s - p_s A_s x_r \\
&\quad+ x_r A_s p_s - A_s p_s x_r \\
&= \lr{ \antisymmetric{x_r}{p_s} + p_s x_r } A_s - p_s A_s x_r \\
&\quad+ x_r A_s p_s - A_s \lr{ \antisymmetric{p_s}{x_r} + x_r p_s } \\
&= \antisymmetric{x_r}{p_s} A_s + \cancel{p_s A_s x_r   - p_s A_s x_r} \\
&\quad+ \cancel{x_r A_s p_s - x_r A_s p_s} + A_s \antisymmetric{x_r}{p_s} \\
&= 2 i \Hbar \delta_{r s} A_s \\
&= 2 i \Hbar A_r,
\end{aligned}
\end{equation}

so
%
\begin{equation}\label{eqn:gaugeTx:160}
\antisymmetric{\Bx}{\Bp \cdot \BA + \BA \cdot \Bp} = 2 i \Hbar \BA.
\end{equation}
%
Assembling these results gives
%
%\begin{equation}\label{eqn:gaugeTx:180}
\boxedEquation{eqn:gaugeTx:180}{
\ddt{\Bx} = \inv{m} \lr{ \Bp - \frac{e}{c} \BA } = \inv{m} \BPi,
}
%\end{equation}

as asserted in the text.

\paragraph{Kinetic Momentum commutators}
%
\begin{dmath}\label{eqn:gaugeTx:200}
\antisymmetric{\Pi_r}{\Pi_s}
=
\antisymmetric{p_r - e A_r/c}{p_s - e A_s/c}
=
\cancel{\antisymmetric{p_r}{p_s}}
- \frac{e}{c} \lr{ \antisymmetric{p_r}{A_s} + \antisymmetric{A_r}{p_s}}
+ \frac{e^2}{c^2} \cancel{\antisymmetric{A_r}{A_s}}
=
- \frac{e}{c} \lr{ (-i\Hbar) \PD{x_r}{A_s} + (i\Hbar) \PD{x_s}{A_r} }
=
- \frac{i e \Hbar}{c} \lr{ -\PD{x_r}{A_s} + \PD{x_s}{A_r} }
=
- \frac{i e \Hbar}{c} \epsilon_{t s r} B_t,
\end{dmath}
%
or
%\begin{dmath}\label{eqn:gaugeTx:220}
\boxedEquation{eqn:gaugeTx:220}{
\antisymmetric{\Pi_r}{\Pi_s}
=
\frac{i e \Hbar}{c} \epsilon_{r s t} B_t.
}
%\end{dmath}
%
\paragraph{Quantum Lorentz force}
\index{Lorentz force}

For the force equation we have
%
\begin{dmath}\label{eqn:gaugeTx:240}
m \frac{d^2 \Bx}{dt^2}
= \ddt{\BPi}
= \inv{i \Hbar} \antisymmetric{\BPi}{H}
= \inv{i \Hbar 2 m } \antisymmetric{\BPi}{\BPi^2}
+ \inv{i \Hbar } \antisymmetric{\BPi}{e \phi}.
\end{dmath}
%
For the \( \phi \) commutator consider one component
%
\begin{dmath}\label{eqn:gaugeTx:260}
\antisymmetric{\Pi_r}{e \phi}
=
e \antisymmetric{p_r - \frac{e}{c} A_r}{\phi}
=
e \antisymmetric{p_r}{\phi}
=
e (-i\Hbar) \PD{x_r}{\phi},
\end{dmath}
%
or
\begin{equation}\label{eqn:gaugeTx:280}
\inv{i \Hbar} \antisymmetric{\BPi}{e \phi}
=
- e \spacegrad \phi
=
e \BE.
\end{equation}
%
For the \( \BPi^2 \) commutator I initially did this the hard way (it took four notebook pages, plus two for a false start.)  Realizing that I didn't use \cref{eqn:gaugeTx:220} for that expansion was the clue to doing this more expediently.

Considering a single component
%
\begin{dmath}\label{eqn:gaugeTx:300}
\antisymmetric{\Pi_r}{\BPi^2}
=
\antisymmetric{\Pi_r}{\Pi_s \Pi_s}
=
\Pi_r \Pi_s \Pi_s - \Pi_s \Pi_s \Pi_r
=
\lr{ \antisymmetric{\Pi_r}{\Pi_s}  + \cancel{\Pi_s \Pi_r} }
\Pi_s
- \Pi_s
\lr{ \antisymmetric{\Pi_s}{\Pi_r}  + \cancel{\Pi_r \Pi_s} }
= i \Hbar \frac{e}{c} \epsilon_{r s t}
\lr{ B_t \Pi_s + \Pi_s B_t },
\end{dmath}
%
or
%
\begin{dmath}\label{eqn:gaugeTx:320}
\inv{ i \Hbar 2 m} \antisymmetric{\BPi}{\BPi^2}
= \frac{e}{2 m c } \epsilon_{r s t} \Be_r
\lr{ B_t \Pi_s + \Pi_s B_t }
= \frac{e}{ 2 m c }
\lr{
 \BPi \cross \BB
- \BB \cross \BPi
}.
\end{dmath}
%
Putting all the pieces together we've got the quantum equivalent of the Lorentz force equation
%
%\begin{dmath}\label{eqn:gaugeTx:340}
\boxedEquation{eqn:gaugeTx:340}{
m \frac{d^2 \Bx}{dt^2} = e \BE + \frac{e}{2 c} \lr{
 \frac{d\Bx}{dt} \cross \BB
- \BB \cross \frac{d\Bx}{dt}
}.
}
%\end{dmath}
%
While this looks equivalent to the classical result, all the vectors here are Heisenberg picture operators dependent on position.

} % answer

%\EndArticle
