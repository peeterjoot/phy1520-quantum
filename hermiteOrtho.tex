%
% Copyright � 2015 Peeter Joot.  All Rights Reserved.
% Licenced as described in the file LICENSE under the root directory of this GIT repository.
%
%\input{../blogpost.tex}
%\renewcommand{\basename}{hermiteOrtho}
%\renewcommand{\dirname}{notes/phy1520/}
%%\newcommand{\dateintitle}{}
%%\newcommand{\keywords}{}
%
%\input{../peeter_prologue_print2.tex}
%
%\usepackage{peeters_layout_exercise}
%\usepackage{peeters_braket}
%\usepackage{peeters_figures}
%
%\beginArtNoToc
%
%\generatetitle{Hermite polynomial normalization constant}
%\chapter{Hermite polynomial normalization constant}
%\label{chap:hermiteOrtho}

\makeoproblem{Hermite polynomial normalization constant.}{problem:hermiteOrtho:2.21}{\citep{sakurai2014modern} pr. 2.21}{
\index{Hermite polynomial}
Derive the normalization constant \( c_n \) for the Harmonic oscillator solution

\begin{dmath}\label{eqn:hermiteOrtho:20}
u_n(x) = c_n H_n\lr{ x \sqrt{\frac{m\omega}{\Hbar}} } e^{-m \omega x^2/2 \Hbar},
\end{dmath}

by deriving the orthogonality relationship using generating functions

\begin{equation}\label{eqn:hermiteOrtho:40}
g(x,t) = e^{-t^2 + 2 t x} = \sum_{n=0}^\infty H_n(x) \frac{t^n}{n!}.
\end{equation}

Start by working out the integral

\begin{equation}\label{eqn:hermiteOrtho:60}
I = \int_{-\infty}^\infty g(x, t) g(x, s) e^{-x^2} dx,
\end{equation}

consider the integral twice with each side definition of the generating function.

} % problem

\makeanswer{problem:hermiteOrtho:2.21}{

First using the exponential definition of the generating function

\begin{dmath}\label{eqn:hermiteOrtho:80}
\int_{-\infty}^\infty g(x, t) g(x, s) e^{-x^2} dx
=
\int_{-\infty}^\infty
e^{-t^2 + 2 t x}
e^{-s^2 + 2 s x} e^{-x^2} dx
=
e^{-t^2 -s^2}
\int_{-\infty}^\infty
e^{-(x^2- 2 t x - 2 s x)} dx
=
e^{-t^2 -s^2 + (s + t)^2}
\int_{-\infty}^\infty
e^{-(x - t - s)^2} dx
=
e^{2 st}
\int_{-\infty}^\infty
e^{-u^2} du
= \sqrt{\pi} e^{2 st}.
\end{dmath}

With the Hermite polynomial definition of the generating function, this integral is

\begin{dmath}\label{eqn:hermiteOrtho:100}
\int_{-\infty}^\infty g(x, t) g(x, s) e^{-x^2} dx
=
\int_{-\infty}^\infty
\sum_{n=0}^\infty H_n(x) \frac{t^n}{n!}
\sum_{m=0}^\infty H_m(x) \frac{s^m}{m!}
e^{-x^2} dx
=
\sum_{n=0}^\infty \frac{t^n}{n!}
\sum_{m=0}^\infty \frac{s^m}{m!}
\int_{-\infty}^\infty H_n(x) H_m(x) e^{-x^2} dx.
\end{dmath}

Let

\begin{dmath}\label{eqn:hermiteOrtho:120}
\alpha_{n m} = \int_{-\infty}^\infty H_n(x) H_m(x) e^{-x^2} dx,
\end{dmath}

and equate the two expansions of this integral

\begin{dmath}\label{eqn:hermiteOrtho:140}
\sqrt{\pi} \sum_{n=0}^\infty \frac{(2st)^n}{n!}
=
\sum_{n=0}^\infty \frac{t^n}{n!}
\sum_{m=0}^\infty \frac{s^m}{m!}
\alpha_{n m},
\end{dmath}

or, after equating powers of \( t^n \)

\begin{dmath}\label{eqn:hermiteOrtho:160}
\sqrt{\pi} (2 s)^n =
\sum_{m=0}^\infty \frac{s^m}{m!} \alpha_{n m}.
\end{dmath}

This requires \( \alpha_{n m} \) to be zero for \( n \ne m \), so

\begin{dmath}\label{eqn:hermiteOrtho:180}
\sqrt{\pi} 2^n = \frac{1}{n!} \alpha_{n n},
\end{dmath}

and

\begin{dmath}\label{eqn:hermiteOrtho:200}
\int_{-\infty}^\infty H_n(x) H_m(x) e^{-x^2} dx = \delta_{n m} \sqrt{\pi} 2^n n!.
\end{dmath}

The SHO normalization is fixed by

\begin{dmath}\label{eqn:hermiteOrtho:220}
\int_{-\infty}^\infty u_n^2(x) dx
= c_n^2
\int_{-\infty}^\infty H_n^2(x/x_0) e^{-(x/x_0)^2} dx
= c_n^2 x_0 \sqrt{\pi} 2^n n!,
\end{dmath}

or

\begin{dmath}\label{eqn:hermiteOrtho:240}
c_n
= \inv{\sqrt{ \sqrt{\pi} 2^n n! \sqrt{\frac{\Hbar}{m \omega}}}}
= \lr{ \frac{m \omega}{\Hbar \pi} }^{1/4} 2^{-n/2} \inv{\sqrt{n!}}
\end{dmath}

} % answer

%\EndArticle
