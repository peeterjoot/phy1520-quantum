%
% Copyright � 2015 Peeter Joot.  All Rights Reserved.
% Licenced as described in the file LICENSE under the root directory of this GIT repository.
%
%{
%\input{../blogpost.tex}
%\renewcommand{\basename}{angularMomentumExpectation}
%\renewcommand{\dirname}{notes/phy1520/}
%%\newcommand{\dateintitle}{}
%%\newcommand{\keywords}{}
%
%\input{../peeter_prologue_print2.tex}
%
%\usepackage{peeters_layout_exercise}
%\usepackage{peeters_braket}
%\usepackage{peeters_figures}
%
%\beginArtNoToc
%
%\generatetitle{Angular momentum expectation}
%%\chapter{Angular momentum expectation}
%%\label{chap:angularMomentumExpectation}
%
\makeoproblem{Angular momentum expectation values.}{problem:angularMomentumExpectation:180}{\citep{sakurai2014modern} pr. 3.18}{
\index{angular momentum!expectation}
Compute the expectation values for the first and second powers of the angular momentum operators with respect to states \( \ket{lm} \).
%
} % problem
%
\makeanswer{problem:angularMomentumExpectation:180}{
We can write the expectation values for the \( L_z \) powers immediately
%
\begin{dmath}\label{eqn:angularMomentumExpectation:20}
\expectation{L_z}
= m \Hbar,
\end{dmath}
%
and
%
\begin{equation}\label{eqn:angularMomentumExpectation:40}
\expectation{L_z^2} = (m \Hbar)^2.
\end{equation}
%
For the x and y components first express the operators in terms of the ladder operators.
%
\begin{equation}\label{eqn:angularMomentumExpectation:60}
\begin{aligned}
L_{+} &= L_x + i L_y \\
L_{-} &= L_x - i L_y.
\end{aligned}
\end{equation}
%
Rearranging gives
%
\begin{equation}\label{eqn:angularMomentumExpectation:80}
\begin{aligned}
L_x &= \inv{2} \lr{ L_{+} + L_{-} } \\
L_y &= \inv{2i} \lr{ L_{+} - L_{-} }.
\end{aligned}
\end{equation}
%
The first order expectations \( \expectation{L_x}, \expectation{L_y} \) are both zero since \( \expectation{L_{+}} = \expectation{L_{-}} \).  For the second order expectation values we have
%
\begin{dmath}\label{eqn:angularMomentumExpectation:100}
L_x^2
= \inv{4} \lr{ L_{+} + L_{-} } \lr{ L_{+} + L_{-} }
= \inv{4} \lr{ L_{+} L_{+} + L_{-} L_{-} + L_{+} L_{-} + L_{-} L_{+} }
= \inv{4} \lr{ L_{+} L_{+} + L_{-} L_{-} + 2 (L_x^2 + L_y^2) }
= \inv{4} \lr{ L_{+} L_{+} + L_{-} L_{-} + 2 (\BL^2 - L_z^2) },
\end{dmath}
%
and
\begin{dmath}\label{eqn:angularMomentumExpectation:120}
L_y^2
= -\inv{4} \lr{ L_{+} - L_{-} } \lr{ L_{+} - L_{-} }
= -\inv{4} \lr{ L_{+} L_{+} + L_{-} L_{-} - L_{+} L_{-} - L_{-} L_{+} }
= -\inv{4} \lr{ L_{+} L_{+} + L_{-} L_{-} - 2 (L_x^2 + L_y^2) }
= -\inv{4} \lr{ L_{+} L_{+} + L_{-} L_{-} - 2 (\BL^2 - L_z^2) }.
\end{dmath}
%
Any expectation value \( \bra{lm} L_{+} L_{+} \ket{lm} \) or \( \bra{lm} L_{-} L_{-} \ket{lm} \) will be zero, leaving
%
\begin{dmath}\label{eqn:angularMomentumExpectation:140}
\expectation{L_x^2}
=
\expectation{L_y^2}
=
\inv{4} \expectation{2 (\BL^2 - L_z^2) }
=
\inv{2} \lr{ \Hbar^2 l(l+1) - (\Hbar m)^2 }.
\end{dmath}
%
Observe that we have
\begin{equation}\label{eqn:angularMomentumExpectation:160}
\expectation{L_x^2}
+
\expectation{L_y^2}
+
\expectation{L_z^2}
=
\Hbar^2 l(l+1)
=
\expectation{\BL^2},
\end{equation}
%
which is the quantum mechanical analogue of the classical scalar equation \( \BL^2 = L_x^2 + L_y^2 + L_z^2 \).
} % answer

%\EndArticle
