%
% Copyright � 2015 Peeter Joot.  All Rights Reserved.
% Licenced as described in the file LICENSE under the root directory of this GIT repository.
%
%\input{../blogpost.tex}
%\renewcommand{\basename}{diracPotential}
%\renewcommand{\dirname}{notes/phy1520/}
%%\newcommand{\dateintitle}{}
%%\newcommand{\keywords}{}
%
%\input{../peeter_prologue_print2.tex}
%
%\usepackage{peeters_layout_exercise}
%\usepackage{peeters_braket}
%\usepackage{peeters_figures}
%
%\beginArtNoToc
%
%\generatetitle{Dirac delta function potential}
%%\chapter{Dirac delta function potential}
%%\label{chap:diracPotential}

\makeoproblem{Dirac delta function potential.}{problem:diracPotential:1}{\citep{sakurai2014modern} pr. 2.24,2.25}{

Given a Dirac delta function potential

\begin{dmath}\label{eqn:diracPotential:20}
H = \frac{p^2}{2m} - V_0 \delta(x),
\end{dmath}

which vanishes after \( t = 0 \).

\makesubproblem{}{problem:diracPotential:1:a}
Solve for the bound state for \( t < 0 \),
\makesubproblem{}{problem:diracPotential:1:b}
Solve for the time evolution after that.
} % problem

\makeanswer{problem:diracPotential:1}{

\makeSubAnswer{}{problem:diracPotential:1:a}

%The first part of this problem was assigned back in phy356, where we solved this for a rectangular potential that had the limiting form of a delta function potential.  However,
This problem can be solved directly by considering the \( \Abs{x} > 0 \) and \( x = 0 \) regions separately.

For \( \Abs{x} > 0 \) Schr\"{o}dinger's equation takes the form

\begin{dmath}\label{eqn:diracPotential:40}
E \psi = -\frac{\Hbar^2}{2m} \frac{d^2 \psi}{dx^2}.
\end{dmath}

With

\begin{dmath}\label{eqn:diracPotential:60}
\kappa =
\frac{\sqrt{-2 m E}}{\Hbar},
\end{dmath}

this has solutions

\begin{dmath}\label{eqn:diracPotential:80}
\psi = e^{\pm \kappa x}.
\end{dmath}

For \( x > 0 \) we must have
\begin{dmath}\label{eqn:diracPotential:100}
\psi = a e^{-\kappa x},
\end{dmath}

and for \( x < 0 \)
\begin{dmath}\label{eqn:diracPotential:120}
\psi = b e^{\kappa x}.
\end{dmath}

requiring that \( \psi \) is continuous at \( x = 0 \) means \( a = b \), or

\begin{dmath}\label{eqn:diracPotential:140}
\psi = \psi(0) e^{-\kappa \Abs{x}}.
\end{dmath}

For the \( x = 0 \) region, consider an interval \( [-\epsilon, \epsilon] \) region around the origin.  We must have

\begin{dmath}\label{eqn:diracPotential:160}
E \int_{-\epsilon}^\epsilon \psi(x) dx = \frac{-\Hbar^2}{2m} \int_{-\epsilon}^\epsilon \frac{d^2 \psi}{dx^2} dx - V_0 \int_{-\epsilon}^\epsilon \delta(x) \psi(x) dx.
\end{dmath}

The RHS is zero

\begin{dmath}\label{eqn:diracPotential:180}
E \int_{-\epsilon}^\epsilon \psi(x) dx
=
E \frac{ e^{-\kappa (\epsilon)} - 1}{-\kappa}
-E \frac{ 1 - e^{\kappa (-\epsilon)}}{\kappa}
\rightarrow
0.
\end{dmath}

That leaves
\begin{dmath}\label{eqn:diracPotential:200}
V_0 \int_{-\epsilon}^\epsilon \delta(x) \psi(x) dx
=
\frac{-\Hbar^2}{2m} \int_{-\epsilon}^\epsilon \frac{d^2 \psi}{dx^2} dx
=
\frac{-\Hbar^2}{2m} \evalrange{\frac{d \psi}{dx}}{-\epsilon}{\epsilon}
=
\frac{-\Hbar^2}{2m}
\psi(0)
\lr
{
-\kappa e^{-\kappa (\epsilon)}
-
\kappa e^{\kappa (-\epsilon)}
}.
\end{dmath}

In the \( \epsilon \rightarrow 0 \) limit this gives

\begin{dmath}\label{eqn:diracPotential:220}
V_0 = \frac{\Hbar^2 \kappa}{m}.
\end{dmath}

Equating relations for \( \kappa \) we have

\begin{equation}\label{eqn:diracPotential:240}
\kappa = \frac{m V_0}{\Hbar^2} = \frac{\sqrt{-2 m E}}{\Hbar},
\end{equation}

or

\begin{equation}\label{eqn:diracPotential:260}
E = -\inv{2 m} \lr{ \frac{m V_0}{\Hbar} }^2,
\end{equation}

with

\begin{equation}\label{eqn:diracPotential:280}
\psi(x, t < 0) = C \exp\lr{ -i E t/\hbar - \kappa \Abs{x}}.
\end{equation}

The normalization requires

\begin{dmath}\label{eqn:diracPotential:300}
1
= 2 \Abs{C}^2 \int_0^\infty e^{- 2 \kappa x} dx
= 2 \Abs{C}^2 \evalrange{\frac{e^{- 2 \kappa x}}{-2 \kappa}}{0}{\infty}
= \frac{\Abs{C}^2}{\kappa},
\end{dmath}

so
%\begin{equation}\label{eqn:diracPotential:320}
\boxedEquation{eqn:diracPotential:340}{
\psi(x, t < 0) = \sqrt{\kappa} \exp\lr{ -i E t/\hbar - \kappa \Abs{x}}.
}
%\end{equation}

There is only one bound state for such a potential.

\makeSubAnswer{}{problem:diracPotential:1:b}

After turning off the potential, any plane wave

\begin{equation}\label{eqn:diracPotential:360}
\psi(x, t) = e^{i k x - i E(k) t/\Hbar},
\end{equation}

where

\begin{equation}\label{eqn:diracPotential:380}
k = \frac{\sqrt{2 m E}}{\Hbar},
\end{equation}

is a solution.  In particular, at \( t = 0 \), the wave packet

\begin{equation}\label{eqn:diracPotential:400}
\psi(x,0) = \inv{\sqrt{2\pi}} \int_{-\infty}^\infty e^{i k x} A(k) dk,
\end{equation}

is a solution.  To solve for \( A(k) \), we require

\begin{dmath}\label{eqn:diracPotential:420}
\inv{\sqrt{2\pi}} \int_{-\infty}^\infty e^{i k x} A(k) dk
= \sqrt{\kappa} e^{ - \kappa \Abs{x} },
\end{dmath}

or
%\begin{dmath}\label{eqn:diracPotential:440}
\boxedEquation{eqn:diracPotential:460}{
A(k) =
\sqrt{\frac{\kappa}{2\pi}} \int_{-\infty}^\infty e^{-i k x}
e^{ - \kappa \Abs{x} }
dx
=
\sqrt{\frac{2}{\pi } }
\frac{
 \kappa^{3/2}}{\kappa^2+k^2}.
}
%\end{dmath}

The initial time state established by the delta function potential evolves as
%\begin{equation}\label{eqn:diracPotential:480}
\boxedEquation{eqn:diracPotential:500}{
\psi(x, t > 0) =
\inv{\sqrt{2\pi}} \int_{-\infty}^\infty e^{i k x - i \Hbar k^2 t/2m} A(k) dk.
}
%\end{equation}

In terms of \( m, V_0 \) that is

\begin{dmath}\label{eqn:diracPotential:700}
\psi(x, t > 0) =
\frac{
 \lr{m V_0}^{3/2}
}{\pi \Hbar}
\int_{-\infty}^\infty
\frac{e^{i k x - i \Hbar k^2 t/2m}
}{k^2 \Hbar^2 + \ifrac{m^2 V_0^2}{\Hbar^2}}
dk
.
\end{dmath}

This integral resists an attempt to evaluate with Mathematica.
} % answer

%\EndArticle
