%
% Copyright � 2015 Peeter Joot.  All Rights Reserved.
% Licenced as described in the file LICENSE under the root directory of this GIT repository.
%
\makeoproblem{Landau Levels - Symmetric gauge.}{gradQuantum:problemSet3:2}{phy1520 2015 ps3.2}
{
\index{Landau levels!symmetric gauge}

Consider a charged particle moving in two dimensions (\(xy\)-plane) in a uniform magnetic field \( B_0 \zcap \) perpendicular
to the plane.
Let us work in a different gauge from the Landau gauge we discussed in class, namely, let us set
%
\begin{dmath}\label{eqn:gradQuantumProblemSet3Problem2:20}
\BA = \frac{B_0}{2} \lr{ x \ycap - y \xcap },
\end{dmath}
%
where \( (x,y) \) denotes the particle position.
This is called the `symmetric gauge'.

In this gauge,
\makesubproblem{}{gradQuantum:problemSet3:2a}
work out the energy spectrum, and the eigenfunctions, and
\makesubproblem{}{gradQuantum:problemSet3:2b}
provide a crude counting of the number of states per energy level (i.e., the degeneracy) for an electron on a disk of radius \( R \).
} % makeproblem
%
\makeanswer{gradQuantum:problemSet3:2}{
\withproblemsetsParagraph{

\makeSubAnswer{}{gradQuantum:problemSet3:2a}
%
Using the approach suggested by our practise problem \citep{sakurai2014modern} \textprref{2.39},
%and \citep{desai2009quantum}
the Hamiltonian for magnetic field driven motion constrained to a plane, can be factored into raising and lowering style operators
%
\begin{dmath}\label{eqn:gradQuantumProblemSet3Problem2:40}
H
= \inv{2 m} \lr{ \Bp - \frac{e}{c} \BA }^2
= \inv{2 m} \lr{ \Pi_x^2 + \Pi_y^2 }^2
= \inv{2 m} \lr{ \lr{ \Pi_x - i \Pi_y }\lr{ \Pi_x + i \Pi_y } - i \lr{ \Pi_x \Pi_y - \Pi_y \Pi_x } },
\end{dmath}
%
That commutator term is proportional to the magnetic field strength
\begin{dmath}\label{eqn:gradQuantumProblemSet3Problem2:60}
\Pi_x \Pi_y - \Pi_y \Pi_x
=
\antisymmetric{\Pi_x}{\Pi_y}
=
\antisymmetric{p_x - \frac{e}{c} A_x}{p_y - \frac{e}{c} A_y}
=
\cancel{\antisymmetric{p_x}{p_y}} - \frac{e}{c} \lr{ \antisymmetric{A_x}{p_y} + \antisymmetric{p_x}{A_y} } + \lr{\frac{e}{c}}^2 \cancel{\antisymmetric{A_x}{A_y}}
=
- \frac{e}{c} (-i \Hbar) \lr{ \PD{x}{A_y} - \PD{y}{A_x} }
=
i \frac{e \Hbar}{c} B_z
=
i \frac{e \Hbar B_0}{c},
\end{dmath}
%
so
%
\begin{dmath}\label{eqn:gradQuantumProblemSet3Problem2:80}
H
= \inv{2 m} \lr{ \Pi_x - i \Pi_y }\lr{ \Pi_x + i \Pi_y } + \frac{e \Hbar B_0}{2 m c}.
\end{dmath}
%
Writing
%
\begin{dmath}\label{eqn:gradQuantumProblemSet3Problem2:100}
\omega = \frac{e B_0}{m c},
\end{dmath}
%
this appears to have the structure of the 1D Harmonic oscillator
\begin{dmath}\label{eqn:gradQuantumProblemSet3Problem2:120}
H
= \inv{2 m} \lr{ \Pi_x - i \Pi_y }\lr{ \Pi_x + i \Pi_y } + \frac{\Hbar \omega}{2}.
\end{dmath}
%
Observe that
%
\begin{dmath}\label{eqn:gradQuantumProblemSet3Problem2:140}
\antisymmetric{ \Pi_x + i \Pi_y }{ \Pi_x - i \Pi_y }
=
i \antisymmetric{ \Pi_y }{ \Pi_x } - i \antisymmetric{ \Pi_x }{ \Pi_y }
=
- 2 i \antisymmetric{ \Pi_x }{ \Pi_y }
=
\frac{2 e \Hbar B_0}{c}
=
2 m \omega.
\end{dmath}
%
With
%
\begin{dmath}\label{eqn:gradQuantumProblemSet3Problem2:160}
b = \inv{\sqrt{2 m \omega \Hbar}} \lr{ \Pi_x + i \Pi_y },
\end{dmath}
%
the Hamiltonian has the form
%
\begin{dmath}\label{eqn:gradQuantumProblemSet3Problem2:180}
H = \Hbar \omega \lr{ b^\dagger b + \inv{2} },
\end{dmath}
%
where
%
\begin{dmath}\label{eqn:gradQuantumProblemSet3Problem2:200}
\antisymmetric{b}{b^\dagger} = 1,
\end{dmath}
%
just like the 1D SHO.  The energy levels are therefore
%
\begin{dmath}\label{eqn:gradQuantumProblemSet3Problem2:220}
E_n = \Hbar \omega \lr{ n + \inv{2} }.
\end{dmath}
%
For the symmetric gauge where \( A_x = - B_0 y/2, A_y = B_0 x/2 \), the lowering operator has the form
%
\begin{dmath}\label{eqn:gradQuantumProblemSet3Problem2:240}
b
= \inv{\sqrt{2 m \omega \Hbar}} \lr{ p_x + i p_y - \frac{e}{c} \frac{B_0}{2} ( -y + i x ) }
= \frac{-i \Hbar}{\sqrt{2 m \omega \Hbar}} \lr{ \partial_x + i \partial_y + \frac{e B_0}{2 \Hbar c} ( i y + x ) }
= \frac{-i \Hbar}{\sqrt{2 m \omega \Hbar}} \lr{ \partial_x + i \partial_y + \frac{m \omega}{2 \Hbar} ( x + i y) }.
\end{dmath}
%
With
%
\begin{equation}\label{eqn:gradQuantumProblemSet3Problem2:260}
\alpha = \frac{m \omega}{2 \Hbar} = \frac{e B_0}{2 \Hbar c},
\end{equation}
%
The first state is defined by
%
\begin{dmath}\label{eqn:gradQuantumProblemSet3Problem2:280}
0
=
\bra{x,y} b \ket{0}
\propto
\lr{ \partial_x + i \partial_y + \alpha ( x + i y) } u(x,y).
\end{dmath}
%
For integer \( n \), this has solutions of the form
%
\begin{dmath}\label{eqn:gradQuantumProblemSet3Problem2:300}
u_n(x,y) = c_n \lr{ x + i y }^n e^{-\alpha \lr{x^2 + y^2}/2},
\end{dmath}
%
which can be verified directly
%
\begin{dmath}\label{eqn:gradQuantumProblemSet3Problem2:320}
\begin{aligned}
\biglr{ &\partial_x + i \partial_y + \alpha ( x + i y) } u_n(x,y) \\
&=
\biglr{ \partial_x + i \partial_y + \alpha ( x + i y) } c_n \lr{ x + i y }^n e^{-\alpha \lr{x^2 + y^2}/2} \\
&=
c_n e^{-\alpha \lr{x^2 + y^2}/2} \lr{
n \lr{ x + i y }^{n-1} \lr{ 1 + i^2 } -\alpha \lr{ x + i y }^n \lr{ 2x + 2 y i }/2 + \alpha \lr{ x + i y }^{n+1}
} \\
&= 0.
\end{aligned}
\end{dmath}
%
The normalization is given by
%
\begin{dmath}\label{eqn:gradQuantumProblemSet3Problem2:340}
1
= \int \Abs{u_n(x,y)}^2 dx dy
= \Abs{c_n}^2 \int \Abs{x + i y}^{2n} e^{-\alpha(x^2 + y^2)} dx dy
= \Abs{c_n}^2 \int_0^\infty 2 \pi r r^{2n} e^{-\alpha r^2} dr.
\end{dmath}
%
Let \( \alpha r^2 = t \), with \( 2 r dr = dt/\alpha \), for
%
\begin{dmath}\label{eqn:gradQuantumProblemSet3Problem2:360}
1
= \Abs{c_n}^2 \frac{\pi}{\alpha} \int_0^\infty \lr{\frac{t}{\alpha}}^n e^{-t} dt
= \Abs{c_n}^2 \frac{\pi}{\alpha^{n+1}} \int_0^\infty t^{(n + 1) - 1} e^{-t} dt
= \Abs{c_n}^2 \frac{\pi}{\alpha^{n+1}} \Gamma(n+1)
= \Abs{c_n}^2 \frac{\pi}{\alpha^{n+1}} n!,
\end{dmath}
%
so
%
\begin{dmath}\label{eqn:gradQuantumProblemSet3Problem2:380}
c_n = \sqrt{\frac{\alpha^{n+1}}{n! \pi}},
\end{dmath}
%
and
%
\boxedEquation{eqn:gradQuantumProblemSet3Problem2:400}{
u_n(x,y) = \sqrt{\frac{\alpha^{n+1}}{n! \pi}} \lr{ x + i y }^n e^{-\alpha (x^2 + y^2)/2}.
}

\makeSubAnswer{}{gradQuantum:problemSet3:2b}
%
The wave functions decay exponentially, and beyond a certain threshold (dependent on \( n \)), not much of the wave function will contribute to the probability density.  To get a feel for this, the probability density \( 2 \pi r \Abs{u_n(r)}^2 \) is plotted for \( n = 2,4,6 \) with \( \alpha = 1 \) in \cref{fig:ps3:ps3Fig2}.

\imageFigure{../figures/phy1520-quantum/ps3Fig2}{Some representative probability density plots.}{fig:ps3:ps3Fig2}{0.3}

What is the value of \( r \) for which the probability density is maximized for a given value of \( n \)?  That is
%
\begin{dmath}\label{eqn:gradQuantumProblemSet3Problem2:420}
0
=
\frac{d}{dr} \lr{
2 \pi r \Abs{u_n(r)}^2
}
=
2 \pi \frac{\alpha^{n+1}}{\pi n!}
\frac{d}{dr} \lr{
r r^{2 n} e^{-\alpha r^2}
}
=
2 \pi \frac{\alpha^{n+1}}{\pi n!}
\lr{
(2 n + 1) r^{2 n}
r^{2 n+ 1} \lr{ - 2 \alpha r }
}
e^{-\alpha r^2}
=
2 \pi \frac{\alpha^{n+1}}{\pi n!}
\lr{
(2 n + 1)
-2 \alpha r^{2}
}
r^{2 n}
e^{-\alpha r^2},
\end{dmath}
%
which occurs at
%
\begin{dmath}\label{eqn:gradQuantumProblemSet3Problem2:440}
r^2
= \frac{(2 n + 1)}{2 \alpha}
= \lr{n + \inv{2}} \frac{ 2 \Hbar c}{e B_0}.
\end{dmath}
%
If this is less than \( R^2 \), and
assuming \( n \) large with respect to \( 1/2 \), and \( r^2 \le R^2 \), this is
%
\begin{equation}\label{eqn:gradQuantumProblemSet3Problem2:460}
n \frac{ 2 \Hbar c}{e B_0} \le R^2,
\end{equation}
%
or
%
\begin{dmath}\label{eqn:gradQuantumProblemSet3Problem2:480}
n
\le \frac{e B_0 R^2}{ 2 \Hbar c}
= \frac{e B_0 \pi R^2}{ h c}
= \frac{e \Phi}{ h c}.
\end{dmath}
%
The approximate number of ground states if the particle is confined to a disk of radius \( R \) is proportional to the magnetic flux \( \Phi = \int \BB \cdot d\BS = B_0 \pi R^2 \) through that disk.
}
}
