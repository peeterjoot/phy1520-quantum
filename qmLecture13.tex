%
% Copyright � 2015 Peeter Joot.  All Rights Reserved.
% Licenced as described in the file LICENSE under the root directory of this GIT repository.
%
%\input{../blogpost.tex}
%\renewcommand{\basename}{qmLecture13}
%\renewcommand{\dirname}{notes/phy1520/}
%\newcommand{\keywords}{PHY1520H}
%\input{../peeter_prologue_print2.tex}
%
%%\usepackage{phy1520}
%\usepackage{peeters_braket}
%%\usepackage{peeters_layout_exercise}
%\usepackage{peeters_figures}
%\usepackage{mathtools}
%
%\beginArtNoToc
%\generatetitle{PHY1520H Graduate Quantum Mechanics.  Lecture 13: Time reversal (cont.), and angular momentum.  Taught by Prof.\ Arun Paramekanti}
%%\chapter{Time reversal (cont.)}
%\label{chap:qmLecture13}
%
%\paragraph{Disclaimer}
%
%Peeter's lecture notes from class.  These may be incoherent and rough.
%
%These are notes for the UofT course PHY1520, Graduate Quantum Mechanics, taught by Prof. Paramekanti, covering \textchapref{{4}}, \textchapref{{3}} \citep{sakurai2014modern} content.
%
\paragraph{Time reversal (cont.)}
Given a time reversed state
%
\begin{equation}\label{eqn:qmLecture13:20}
\ket{\tilde{\Psi}(t)} = \Theta \ket{\Psi(0)},
\end{equation}
which can alternately be written
%
\begin{equation}\label{eqn:qmLecture13:40}
\Theta^{-1} \ket{\tilde{\Psi}(t)} = \ket{\Psi(-t)} = e^{i \hatH t/\Hbar} \ket{\Psi(0)}.
\end{equation}
The left hand side can be expanded as the evolution of the state as found at time \( -t \)
%
\begin{dmath}\label{eqn:qmLecture13:60}
\Theta^{-1} \ket{\tilde{\Psi}(t)}
=
\Theta^{-1} e^{-i \hatH t/\Hbar} \ket{\tilde{\Psi}(-t)}
=
\Theta^{-1} e^{-i \hatH t/\Hbar} \Theta \ket{\Psi(0)}.
\end{dmath}
%
To first order for a small time increment \( \delta t \), we have
%
\begin{dmath}\label{eqn:qmLecture13:80}
\lr{ 1 + i \frac{\hatH}{\Hbar} \delta t } \ket{\Psi(0)} =
\Theta^{-1} \lr{ 1 - i \frac{\hatH}{\Hbar} \delta t } \Theta \ket{\Psi(0)},
\end{dmath}
%
or
%
\begin{dmath}\label{eqn:qmLecture13:120}
i \frac{\hatH}{\Hbar} \delta t \ket{\Psi(0)}
=
\Theta^{-1} (- i) \frac{\hatH}{\Hbar} \delta t \Theta \ket{\Psi(0)}.
\end{dmath}
%
Since this holds for any state \( \ket{\Psi(0)} \), the time reversal operator satisfies
%
\begin{dmath}\label{eqn:qmLecture13:140}
i \hatH
=
\Theta^{-1} (- i) \hatH \Theta.
\end{dmath}
%
Note that the factors of \( i \) have not been cancelled on purpose, since we are allowing for the time reversal operator to not necessarily commute with imaginary numbers.

There are two possible solutions

\begin{itemize}
\item If \( \Theta \) is unitary where \( \Theta i = i \Theta \), then
%
\begin{equation}\label{eqn:qmLecture13:160}
\hatH
=
-\Theta^{-1} \hatH \Theta,
\end{equation}
%
or
\begin{equation}\label{eqn:qmLecture13:180}
\Theta \hatH
=
- \hatH \Theta.
\end{equation}
%
Consider the implications of this on energy eigenstates
\begin{equation}\label{eqn:qmLecture13:200}
\hatH \ket{\Psi_n} = E_n \ket{\Psi_n},
\end{equation}
%
\begin{equation}\label{eqn:qmLecture13:220}
\Theta \hatH \ket{\Psi_n} = E_n \Theta \ket{\Psi_n},
\end{equation}
%
but
%
\begin{equation}\label{eqn:qmLecture13:240}
-\hatH \Theta \ket{\Psi_n} = E_n \Theta \ket{\Psi_n},
\end{equation}
%
or
%
\begin{equation}\label{eqn:qmLecture13:260}
\hatH \lr{ \Theta \ket{\Psi_n}} = -E_n \lr{ \Theta \ket{\Psi_n} }.
\end{equation}
%
This would mean that \( \lr{ \Theta \ket{\Psi_n}} \) is an eigenket of \( \hatH \), but with a negative energy eigenvalue.

\item \( \Theta \) is antiunitary, where \( \Theta i = -i \Theta \).

This time
\begin{equation}\label{eqn:qmLecture13:280}
i \hatH = i \Theta^{-1} \hatH \Theta,
\end{equation}
%
so
%
\begin{equation}\label{eqn:qmLecture13:300}
\Theta \hatH = \hatH \Theta.
\end{equation}
%
Acting on an energy eigenket, we've got
%
\begin{equation}\label{eqn:qmLecture13:1400}
\Theta \hatH \ket{\Psi_n}
=
E_n \lr{ \Theta \ket{\Psi_n} },
\end{equation}
%
and
\begin{equation}\label{eqn:qmLecture13:1420}
\lr{ \hatH \Theta } \ket{\Psi_n}
=
\hatH \lr{ \Theta \ket{\Psi_n} },
\end{equation}
%
so \( \Theta \ket{\Psi_n} \) is an eigenstate with energy \( E_n \).

\end{itemize}

\paragraph{What properties do we expect from \( \Theta \)?}

We expect
\index{time reversal!properties}
\begin{dmath}\label{eqn:qmLecture13:320}
\begin{aligned}
\hatx &\rightarrow \hatx \\
\hatp &\rightarrow -\hatp \\
\Lcap &\rightarrow -\Lcap,
\end{aligned}
\end{dmath}
where we have a sign flip in the time dependent momentum operator (and therefore angular momentum), but not for position.  If we have
%
\begin{equation}\label{eqn:qmLecture13:340}
\Theta^{-1} \hatx \Theta = \hatx,
\end{equation}
%
%if that's true, 
then how about the momentum operator in the position basis
\begin{dmath}\label{eqn:qmLecture13:360}
\Theta^{-1} \hatp \Theta
=
\Theta^{-1} \lr{ -i \Hbar \PD{x}{} } \Theta
=
\Theta^{-1} \lr{ -i \Hbar } \Theta \PD{x}{}
=
i \Hbar \Theta^{-1} \Theta \PD{x}{}
=
-\hatp.
\end{dmath}
%
How about the \( x,p \) commutator?  For that we have
%
\begin{dmath}\label{eqn:qmLecture13:380}
\Theta^{-1} \antisymmetric{\hatx}{\hatp} \Theta
=
\Theta^{-1} \lr{ i \Hbar } \Theta
=
-i \Hbar \Theta^{-1} \Theta
=
- \antisymmetric{\hatx}{\hatp}.
\end{dmath}
%
For the angular momentum operators
%
\begin{equation}\label{eqn:qmLecture13:420}
\hatL_i = \epsilon_{ijk} \hatr_j \hatp_k,
\end{equation}
%
the time reversal operator should flip the sign due to its action on \( \hatp_k \).
%
%\begin{equation}\label{eqn:qmLecture13:400}
%\antisymmetric{\hatL_i }{\hatL_j } = i \epsilon_{ijk} \hatL_k.
%\end{equation}
%
%FIXME: lost his point about the angular momentum commutator here.

\paragraph{Time reversal acting on spin 1/2 (Fermions).  Attempt I.}

Consider two spin states \( \ket{\uparrow}, \ket{\downarrow} \).  What should the action of the time reversal operator on such a state be?  Let's (incorrectly) start by supposing that the time reversal operator effects are
%
\begin{dmath}\label{eqn:qmLecture13:440}
\begin{aligned}
\Theta \ket{\uparrow} &\questionEquals \ket{\downarrow} \\
\Theta \ket{\downarrow} &\questionEquals \ket{\uparrow}.
\end{aligned}
\end{dmath}
%
Given a general state
so that if
%
\begin{equation}\label{eqn:qmLecture13:740}
\ket{\Psi} = a \ket{\uparrow} + b \ket{\downarrow},
\end{equation}
%
the action of the time reversal operator would be
%
\begin{equation}\label{eqn:qmLecture13:760}
\Theta \ket{\Psi} = a^\conj \ket{\downarrow} + b^\conj \ket{\uparrow}.
\end{equation}
%
That action is:
%
\begin{equation}\label{eqn:qmLecture13:460}
\begin{aligned}
a \rightarrow b^\conj, \\
b \rightarrow a^\conj.
\end{aligned}
\end{equation}
Let's consider whether or not such an action a spin operator with properties
%
\begin{equation}\label{eqn:qmLecture13:480}
\antisymmetric{\hatS_i}{\hatS_j} = i \epsilon_{ijk} \hatS_k,
\end{equation}
%
will produce the desired inversion of sign
%
\begin{equation}\label{eqn:qmLecture13:500}
\Theta^{-1} \hatS_i \Theta = - \hatS_i.
\end{equation}
%
The expectations of the spin operators (without any application of time reversal) are
%
\begin{dmath}\label{eqn:qmLecture13:1440}
\bra{\Psi} \hatS_x \ket{\Psi}
=
\frac{\Hbar}{2}
\lr{ a^\conj \bra{\uparrow} + b^\conj \bra{\downarrow} }
\sigma_x
\lr{ a \ket{\uparrow} + b \ket{\downarrow} }
=
\frac{\Hbar}{2}
\lr{ a^\conj \bra{\uparrow} + b^\conj \bra{\downarrow} }
\lr{ a \ket{\downarrow} + b \ket{\uparrow} }
=
\frac{\Hbar}{2}
\lr{ a^\conj b + b^\conj a },
\end{dmath}
%
\begin{dmath}\label{eqn:qmLecture13:1460}
\bra{\Psi} \hatS_y \ket{\Psi}
=
\frac{\Hbar}{2}
\lr{ a^\conj \bra{\uparrow} + b^\conj \bra{\downarrow} }
\sigma_y
\lr{ a \ket{\uparrow} + b \ket{\downarrow} }
=
\frac{i\Hbar}{2}
\lr{ a^\conj \bra{\uparrow} + b^\conj \bra{\downarrow} }
\lr{ a \ket{\downarrow} - b \ket{\uparrow} }
=
\frac{\Hbar}{2 i} \lr{ a^\conj b - b^\conj a },
\end{dmath}
%
\begin{dmath}\label{eqn:qmLecture13:1480}
\bra{\Psi} \hatS_z \ket{\Psi}
=
\frac{\Hbar}{2}
\lr{ a^\conj \bra{\uparrow} + b^\conj \bra{\downarrow} }
\sigma_z
\lr{ a \ket{\uparrow} - b \ket{\downarrow} }
=
\frac{\Hbar}{2} \lr{ \Abs{a}^2 - \Abs{b}^2 }.
\end{dmath}
The time reversed actions are
%
\begin{dmath}\label{eqn:qmLecture13:1560}
\bra{\Psi} \Theta^{-1} \hatS_x \Theta \ket{\Psi}
=
\frac{\Hbar}{2}
\lr{ a^\conj \bra{\downarrow} + b^\conj \bra{\uparrow} }
\sigma_x
\lr{ a \ket{\downarrow} + b \ket{\uparrow} }
=
\frac{\Hbar}{2}
\lr{ a^\conj \bra{\downarrow} + b^\conj \bra{\uparrow} }
\lr{ a \ket{\uparrow} + b \ket{\downarrow} }
=
\frac{\Hbar}{2}
\lr{ a^\conj b + b^\conj a },
\end{dmath}
%
\begin{dmath}\label{eqn:qmLecture13:1580}
\bra{\Psi} \Theta^{-1} \hatS_y \Theta \ket{\Psi}
=
\frac{\Hbar}{2}
\lr{ a^\conj \bra{\downarrow} + b^\conj \bra{\uparrow} }
\sigma_y
\lr{ a \ket{\downarrow} + b \ket{\uparrow} }
=
\frac{i\Hbar}{2}
\lr{ a^\conj \bra{\downarrow} + b^\conj \bra{\uparrow} }
\lr{ -a \ket{\uparrow} + b \ket{\downarrow} }
=
\frac{\Hbar}{2 i} \lr{ -a^\conj b + b^\conj a },
\end{dmath}
%
\begin{dmath}\label{eqn:qmLecture13:1600}
\bra{\Psi} \Theta^{-1} \hatS_z \Theta \ket{\Psi}
=
\frac{\Hbar}{2}
\lr{ a^\conj \bra{\downarrow} + b^\conj \bra{\uparrow} }
\sigma_z
\lr{ a \ket{\downarrow} + b \ket{\uparrow} }
=
\frac{\Hbar}{2}
\lr{ a^\conj \bra{\downarrow} + b^\conj \bra{\uparrow} }
\lr{ -a \ket{\downarrow} + b \ket{\uparrow} }
=
\frac{\Hbar}{2} \lr{ -\Abs{a}^2 + \Abs{b}^2 }.
\end{dmath}
%
%\begin{dmath}\label{eqn:qmLecture13:1500}
%\Theta^{-1} \hatS_z \Theta \ket{\uparrow}
%=
%\frac{\Hbar}{2} \Theta^{-1} \sigma_z \ket{\downarrow}
%=
%-\frac{\Hbar}{2} \Theta^{-1} \ket{\downarrow}
%=
%-\frac{\Hbar}{2} \ket{\uparrow},
%\end{dmath}
%
%\begin{dmath}\label{eqn:qmLecture13:1520}
%\Theta^{-1} \hatS_x \Theta \ket{\uparrow}
%=
%\frac{\Hbar}{2} \Theta^{-1} \sigma_x \ket{\downarrow}
%=
%\frac{\Hbar}{2} \Theta^{-1} \ket{\uparrow}
%=
%\frac{\Hbar}{2} \ket{\downarrow},
%\end{dmath}
%
%and
%
%\begin{dmath}\label{eqn:qmLecture13:1540}
%\Theta^{-1} \hatS_y \Theta \ket{\uparrow}
%=
%\frac{\Hbar}{2} \Theta^{-1} \sigma_y \ket{\downarrow}
%=
%-i \frac{\Hbar}{2} \Theta^{-1} \ket{\uparrow}
%=
%\frac{\Hbar}{2 i} \ket{\downarrow}.
%\end{dmath}
%
%Contrast this to the time reversed action on the spin operators
%
%\bra{\Psi} \Theta^{-1} \hatS_x \Theta \ket{\Psi}
%= \frac{\Hbar}{2} \lr{ b^\conj a + a b^\conj }
%
%\bra{\Psi} \Theta^{-1} \hatS_y \Theta \ket{\Psi}
%= \frac{\Hbar}{2 i} \lr{ b a^\conj - a b^\conj }
%
%\bra{\Psi} \Theta^{-1} \hatS_z \Theta \ket{\Psi}
%= \frac{\Hbar}{2} \lr{ \Abs{b}^2 - \Abs{a}^2 }
We see that this is not right, because the sign for the x component has not been flipped.
% (only the sign of the y component has).

\paragraph{Spin 1/2 (Fermions).  Attempt II.}

Again assuming
%
\begin{equation}\label{eqn:qmLecture13:580}
\ket{\Psi} = a \ket{\uparrow} + b \ket{\downarrow},
\end{equation}
%
now try the action
%
\begin{equation}\label{eqn:qmLecture13:780}
\Theta \ket{\Psi} = a^\conj \ket{\downarrow} - b^\conj \ket{\uparrow}.
\end{equation}
%
This is the action:
%
\begin{equation}\label{eqn:qmLecture13:600}
\begin{aligned}
a \rightarrow -b^\conj, \\
b \rightarrow a^\conj.
\end{aligned}
\end{equation}
The correct action of time reversal on the basis states (up to a phase choice) is
%
\boxedEquation{eqn:qmLecture13:620}{
%\begin{boxed}\label{eqn:qmLecture13:640}
\begin{aligned}
\Theta \ket{\uparrow} &= \ket{\downarrow} \\
\Theta \ket{\downarrow} &= -\ket{\uparrow} \\
\end{aligned}
%\end{boxed}
}
Note that acting the time reversal operator twice has the effects
%
\begin{equation}\label{eqn:qmLecture13:660}
\Theta^2 \ket{\uparrow} = \Theta \ket{\downarrow} = - \ket{\uparrow},
\end{equation}
\begin{equation}\label{eqn:qmLecture13:680}
\Theta^2 \ket{\downarrow} = \Theta (-\ket{\uparrow}) = - \ket{\uparrow}.
\end{equation}
%
We end up with the same state we started with, but with the opposite sign.  This means that as an operator
\index{time reversal!squared}
%\begin{dmath}\label{eqn:qmLecture13:700}
\boxedEquation{eqn:qmLecture13:700}{
\Theta^2 = -1.
}
%\end{dmath}
%
This is try for half integer particles (Fermions) \( S = 1/2, 3/2, 5/2, \cdots \), but for Bosons with integer spin \( S \).
%
%\begin{dmath}\label{eqn:qmLecture13:720}
\boxedEquation{eqn:qmLecture13:720}{
\Theta^2 = 1.
}
%\end{dmath}
%
\paragraph{Kramer's degeneracy for Spin 1/2 (Fermions)}

\index{spin half!time reversal}

Suppose we imagine there is state for which the action of the time reversal operator produces the same state, just different in phase.  Let
%
\begin{equation}\label{eqn:qmLecture13:800}
\ket{\psi}' = \Theta \ket{\psi}
= e^{i \delta} \ket{\psi}.
\end{equation}
%
For a Fermion we have
\begin{equation}\label{eqn:qmLecture13:840}
\Theta^2 \ket{\psi} = -\ket{\psi},
\end{equation}
%
but if such the time reversal action posited above is possible we also have
%
\begin{dmath}\label{eqn:qmLecture13:860}
\Theta^2 \ket{\psi}
=
\Theta e^{i \delta} \ket{\psi}
=
e^{-i \delta} \Theta \ket{\psi}
=
e^{-i \delta} e^{i \delta} \ket{\psi}
=
\ket{\psi}
\ne
- \ket{\psi}.
\end{dmath}
%
This is a contradiction, so we must have at least a two-fold degeneracy.  This is called \textAndIndex{Kramer's degeneracy}.  In the homework we will show that this is not the case for integer spin particles.

\paragraph{Time reversal implications for wave functions}

For spin and angular momentum states, the implications of time reversal on the states is worked out above.  If a spinless Hamiltonian has time reversal symmetry then the implication is really just the fact that the wave functions can be real valued.
