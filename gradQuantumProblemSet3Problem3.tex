%
% Copyright � 2015 Peeter Joot.  All Rights Reserved.
% Licenced as described in the file LICENSE under the root directory of this GIT repository.
%
% This is 2.28 from \citep{}
\makeoproblem{Aharonov Bohm effect.}{gradQuantum:problemSet3:3}{\citep{sakurai2014modern} pr. 2.28, 2015 ps3.3}
{
\index{Aharonov-Bohm effect}

Consider an electron confined to the interior of a finite hollow cylinder with its axis being \( \zcap \).
Let the inner and outer
walls of the cylinder be at radial coordinates \( \rho_a \) and \( \rho_b > \rho_a \) respectively.
Let the cylinder have its top and bottom ends at \( z = 0,L \).
%
\makesubproblem{}{gradQuantum:problemSet3:3a}
Find the eigenstates for a particle confined to this cylinder (ignore normalization), and show that its energies are given by
%
\begin{equation}\label{eqn:gradQuantumProblemSet3Problem3:20}
E_{l m n} = \frac{\Hbar^2}{2 m } \lr{ k_{m n}^2 + \lr{ \frac{ \pi l }{L} }^2 } \qquad ( l = 1,2,3, \cdots ; m = 0, 1, 2, \cdots )
\end{equation}
%
where \( k_{m n} \) is the $n$th root of the equation
%
\begin{equation}\label{eqn:gradQuantumProblemSet3Problem3:40}
J_m (k_{m n} \rho_b ) N_m (k_{m n} \rho_a ) - N_m (k_{m n} \rho_b ) J_m (k_{m n} \rho_a ) = 0.
\end{equation}
%
\makesubproblem{}{gradQuantum:problemSet3:3b}
Repeat this problem with a uniform magnetic field \( B \zcap \) which is confined to the region \( 0 < \rho < \rho_a \) (i.e., only in the hollow part of the cylinder).
\makesubproblem{}{gradQuantum:problemSet3:3c}
Show that there is a periodicity of the energy levels with the field, with the period being such that \( \pi \rho_a^2 B = 2 \pi N \hbar c/e \).
} % makeproblem
%
\makeanswer{gradQuantum:problemSet3:3}{
\withproblemsetsParagraph{

We can model this geometrical constraints of these two configurations by
%
\begin{equation}\label{eqn:gradQuantumProblemSet3Problem3:60}
H = \inv{2 m} \lr{ -i \Hbar \spacegrad - \frac{e}{c}\BA } + V,
\end{equation}
%
where
%
\begin{equation}\label{eqn:gradQuantumProblemSet3Problem3:80}
\spacegrad = \rhocap \partial_\rho + \inv{\rho} \partial_\phi + \zcap \partial_z,
\end{equation}
%
and
%
\begin{dmath}\label{eqn:gradQuantumProblemSet3Problem3:100}
V =
\left\{
\begin{array}{l l}
0& \quad \mbox{if \( \rho \in [\rho_a,\rho_b], z \in [0,L] \) } \\
\infty & \quad \mbox{otherwise.}
\end{array}
\right.
\end{dmath}
%
The effects of the potential require that a wavefunction \( \psi \) solution of this equation satisfies
%
\begin{equation}\label{eqn:gradQuantumProblemSet3Problem3:120}
\evalbar{\psi( z )}{z = 0,L} = 0,
\end{equation}
%
and
%
\begin{equation}\label{eqn:gradQuantumProblemSet3Problem3:140}
\evalbar{\psi( \rho )}{\rho = \rho_a, \rho_b} = 0,
\end{equation}
%
leaving
%
\begin{equation}\label{eqn:gradQuantumProblemSet3Problem3:160}
H = \inv{2 m} \lr{ -i \Hbar \spacegrad - \frac{e}{c} \BA }^2.
\end{equation}
%
in the interior region of the cylinder where the electron is free to move.
%
\makeSubAnswer{}{gradQuantum:problemSet3:3a}
%
Without a magnetic field, in the interior of the cylinder, the Hamiltonian is
%
\begin{dmath}\label{eqn:gradQuantumProblemSet3Problem3:180}
H \psi
= \frac{-\Hbar^2}{2 m} \spacegrad^2 \psi
=
\frac{-\Hbar^2}{2 m} \lr{
\inv{\rho} \partial_\rho \lr{ \rho \partial_\rho \psi } + \inv{\rho^2} \partial_{\phi\phi} \psi + \partial_{z z} \psi
}.
\end{dmath}
%
Assuming a solution is possible using separation of variables, let
%
\begin{equation}\label{eqn:gradQuantumProblemSet3Problem3:200}
\psi(\rho, \phi, z) = P(\rho) \Phi(\phi) Z(z),
\end{equation}
%
so that
%
\begin{dmath}\label{eqn:gradQuantumProblemSet3Problem3:320}
H \psi = E \psi = E P \Phi Z =
\frac{-\Hbar^2}{2 m} \lr{
\Phi Z \inv{\rho} \partial_\rho \lr{ \rho \partial_\rho P } + P Z \inv{\rho^2} \partial_{\phi\phi} \Phi + P \Phi \partial_{z z} Z
},
\end{dmath}
%
or
%
\begin{dmath}\label{eqn:gradQuantumProblemSet3Problem3:220}
E =
\frac{-\Hbar^2}{2 m} \lr{
\inv{\rho P} \partial_\rho \lr{ \rho \partial_\rho P } + \inv{\rho^2 \Phi} \partial_{\phi\phi} \Phi + \inv{Z} \partial_{z z} Z
},
\end{dmath}
%
Let \( E = E' + E_z \), where
%
\begin{equation}\label{eqn:gradQuantumProblemSet3Problem3:240}
E_z = \frac{-\Hbar^2}{2 m} \inv{Z} Z''.
\end{equation}
%
This has solution
%
\begin{equation}\label{eqn:gradQuantumProblemSet3Problem3:260}
Z = e^{i k_z z},
\end{equation}
%
where \( k_z = \sqrt{2 m E_z}/\Hbar \).  The \( z = 0,L \) boundary condition requires that
%
\begin{equation}\label{eqn:gradQuantumProblemSet3Problem3:340}
k_z L = \pi l, \qquad l \in \bbZ
\end{equation}
%
or
%
\begin{equation}\label{eqn:gradQuantumProblemSet3Problem3:280}
k_z = \frac{\pi l}{L}.
\end{equation}
%
That means that the total energy is of the form
%
\begin{equation}\label{eqn:gradQuantumProblemSet3Problem3:300}
E = E' + \frac{\Hbar^2}{2 m} \lr{ \frac{\pi l}{L} }^2.
\end{equation}
%
Now we consider the remaining subset of the eigenvalue equation
%
\begin{dmath}\label{eqn:gradQuantumProblemSet3Problem3:360}
E' \rho^2 =
\frac{-\Hbar^2}{2 m} \lr{
\frac{\rho}{P} \partial_\rho \lr{ \rho \partial_\rho P } + \inv{\Phi} \partial_{\phi\phi} \Phi
}.
\end{dmath}
%
We are free to let
%
\begin{equation}\label{eqn:gradQuantumProblemSet3Problem3:380}
\frac{-\Hbar^2}{2 m} \inv{\Phi} \partial_{\phi\phi} \Phi = E_\phi,
\end{equation}
%
which has solution
%
\begin{equation}\label{eqn:gradQuantumProblemSet3Problem3:400}
\Phi = e^{i k_\phi \phi},
\end{equation}
%
where \( k_\phi = \sqrt{2 m E_\phi}/\Hbar \).  Geometry requires \( k_\phi (2 \pi) = 2 \pi \nu, \nu \in \bbZ \ge 0 \),
or
%
\begin{equation}\label{eqn:gradQuantumProblemSet3Problem3:420}
E_\phi = \frac{\Hbar^2 \nu^2}{2 m}.
\end{equation}
%
This leaves
%
\begin{equation}\label{eqn:gradQuantumProblemSet3Problem3:440}
E' \rho^2 = \frac{-\Hbar^2}{2 m}
\frac{\rho}{P} \partial_\rho \lr{ \rho \partial_\rho P } + \frac{\Hbar^2 \nu^2}{2 m},
\end{equation}
%
or
%
\begin{dmath}\label{eqn:gradQuantumProblemSet3Problem3:460}
0 =
\rho^2 \partial_{\rho \rho} P + \rho \partial_\rho P
+
\lr{ \frac{2 m}{\Hbar^2} E' \rho^2 - \nu^2 } P.
\end{dmath}
%
With \( \rho' = \sqrt{ \frac{2 m E'}{\Hbar^2} } \rho \), this is the standard form (\texteqnref{10.2.1} \citep{NIST:DLMF}) for Bessel's equation as a function of \( \rho' \), with solutions \( J_{\pm \nu}( \rho' ), N_\nu( \rho') \).  In particular, the linear combination
%
\begin{equation}\label{eqn:gradQuantumProblemSet3Problem3:480}
a J_{\pm \nu}( \rho' ) + b N_\nu( \rho'),
\end{equation}
%
is a solution.  The geometry requires this vanish at \( \rho_a, \rho_b \).  With
%
\begin{equation}\label{eqn:gradQuantumProblemSet3Problem3:540}
k = \sqrt{ \frac{2 m E'}{\Hbar^2} },
\end{equation}
%
those boundary value constraints can be written as
%
\begin{dmath}\label{eqn:gradQuantumProblemSet3Problem3:500}
0 =
\begin{bmatrix}
J_{\pm \nu}( k \rho_a ) & N_\nu( k \rho_b ) \\
J_{\pm \nu}( k \rho_b ) & N_\nu( k \rho_b )
\end{bmatrix}
\begin{bmatrix}
a \\
b
\end{bmatrix}.
\end{dmath}
%
This is satisfied when the determinant is zero
%
\begin{equation}\label{eqn:gradQuantumProblemSet3Problem3:520}
0 = J_{\pm \nu}( k \rho_a ) N_\nu( k \rho_b ) - J_{\pm \nu}( k \rho_b ) N_\nu( k \rho_a ).
\end{equation}
%
The total energy eigenvalue is
%
\begin{equation}\label{eqn:gradQuantumProblemSet3Problem3:560}
E = \inv{2 m} (\Hbar k)^2 + \frac{\Hbar^2}{2 m} \lr{ \frac{\pi l}{L} }^2.
\end{equation}
%
With minor differences in indexing notation, this completes the demonstration of \cref{eqn:gradQuantumProblemSet3Problem3:20} and \cref{eqn:gradQuantumProblemSet3Problem3:40}.

The complete wavefunction associated with this energy eigenvalue is
%
\boxedEquation{eqn:gradQuantumProblemSet3Problem3:580}{
\psi = \lr{ a J_{\pm \nu}( k \rho ) + b N_\nu( k \rho ) } e^{i \nu \phi} e^{i \pi l z/L}.
}
%
\makeSubAnswer{}{gradQuantum:problemSet3:3b}
%
A magnetic field that is isolated to the hole of the cavity is described by the piecewise vector potential
%
\begin{dmath}\label{eqn:gradQuantumProblemSet3Problem3:600}
\BA =
\left\{
\begin{array}{l l}
\frac{B}{2} \frac{\rho_a^2}{\rho} \phicap & \quad \mbox{if \( \rho \ge \rho_a \) } \\
\frac{B}{2} \rho \phicap & \quad \mbox{if \( \rho < \rho_a \) } \\
\end{array}
\right.
\end{dmath}
%
Checking this for \( \rho \ge \rho_a \) we have
%
\begin{dmath}\label{eqn:gradQuantumProblemSet3Problem3:620}
\spacegrad \cross \BA
=
\frac{B}{2} \rho_a^2 \lr{ \rhocap \partial_\rho + \frac{\phicap}{\rho} \partial_\phi } \cross \frac{\phicap}{\rho}
=
\frac{B}{2} \rho_a^2 \lr{ \rhocap \cross \frac{\phicap}{-\rho^2} + \frac{\phicap}{\rho^2} \partial_\phi \phicap }
=
\frac{B \rho_a^2}{2 \rho^2} \lr{ \rhocap \cross \phicap + \phicap \cross \lr{ -\rhocap} }
=
0,
\end{dmath}
%
and for \( \rho < \rho_a \)
%
\begin{dmath}\label{eqn:gradQuantumProblemSet3Problem3:640}
\spacegrad \cross \BA
=
\frac{B}{2} \lr{ \rhocap \partial_\rho + \frac{\phicap}{\rho} \partial_\phi } \cross \lr{ \rho \phicap }
=
\frac{B}{2} \lr{ \rhocap \cross \phicap + \frac{\phicap}{\rho} \cross \lr{ -\rho \rhocap } }
=
B \zcap.
\end{dmath}
%
In the conduction zone of the cylinder the Hamiltonian is now
%
\begin{dmath}\label{eqn:gradQuantumProblemSet3Problem3:660}
H
= \inv{2 m} \lr{ \Bp - \frac{e}{c} \BA }^2
= \inv{2 m} \lr{ -i \Hbar \spacegrad - \frac{e}{c} \frac{B}{2} \rho_a^2 \frac{\phicap}{\rho} }^2
= \inv{2 m} \lr{
- \Hbar^2 \spacegrad^2
+ i \Hbar \frac{e}{c} \frac{B}{2} \rho_a^2 \lr{ \spacegrad \cdot \frac{\phicap}{\rho} + \frac{\phicap}{\rho} \cdot \spacegrad }
+ \lr{ \frac{e B \rho_a^2}{2 c \rho} }^2
}.
\end{dmath}
Expanding the cross terms we have
%
\begin{dmath}\label{eqn:gradQuantumProblemSet3Problem3:680}
\spacegrad \cdot \lr{ \frac{\phicap}{\rho} \psi }
=
\spacegrad \psi \cdot \frac{\phicap}{\rho}
+
\psi \spacegrad \cdot \frac{\phicap}{\rho}
=
\lr{ \rhocap \partial_\rho \psi + \frac{\phicap}{\rho} \partial_\phi \psi + \partial_z \psi } \cdot \frac{\phicap}{\rho}
+
\psi
\lr{ \rhocap \cdot \frac{\phicap}{-\rho^2} + \frac{\phicap}{\rho} \cdot \frac{(-\rhocap)}{\rho} }
=
\frac{1}{\rho^2} \partial_\phi \psi,
\end{dmath}
%
and
%
\begin{dmath}\label{eqn:gradQuantumProblemSet3Problem3:700}
\frac{\phicap}{\rho} \cdot \spacegrad \psi
=
\frac{\phicap}{\rho} \cdot \frac{\phicap}{\rho} \partial_\phi \psi
=
\frac{1}{\rho^2} \partial_\phi \psi,
\end{dmath}
%
so we have
%
\begin{dmath}\label{eqn:gradQuantumProblemSet3Problem3:720}
\spacegrad \cdot \frac{\phicap}{\rho} + \frac{\phicap}{\rho} \cdot \spacegrad
=
\frac{2}{\rho^2} \partial_\phi.
\end{dmath}
%
The complete Hamiltonian is
%
\begin{dmath}\label{eqn:gradQuantumProblemSet3Problem3:740}
H \psi =
E \psi =
\frac{-\Hbar^2}{2 m} \lr{
\inv{\rho} \partial_\rho \lr{ \rho \partial_\rho \psi } + \inv{\rho^2} \partial_{\phi\phi} \psi + \partial_{z z} \psi
}
+ \frac{i \Hbar e B \rho_a^2}{2 m c \rho^2} \partial_\phi \psi
+
\inv{2m}
\lr{ \frac{e B \rho_a^2}{2 c \rho} }^2 \psi.
\end{dmath}
%
Proceeding the same way with separation of variables using \( \psi = P \Phi Z \), we have as before
%
\begin{equation}\label{eqn:gradQuantumProblemSet3Problem3:760}
\begin{aligned}
Z &= e^{i \pi l/L} \\
E &= E' + \frac{\Hbar^2}{2m} \lr{ \frac{\pi l}{L} }^2,
\end{aligned}
\end{equation}
and are left with
%
\begin{dmath}\label{eqn:gradQuantumProblemSet3Problem3:780}
E' \rho^2 =
\frac{-\Hbar^2}{2 m} \lr{
\frac{\rho}{P} \partial_\rho \lr{ \rho \partial_\rho P } + \inv{\Phi} \partial_{\phi\phi} \Phi
}
+ \frac{i \Hbar}{2m} \frac{e}{c} \frac{B}{2} \rho_a^2 \inv{ \Phi} 2 \partial_\phi \Phi
+
\inv{2m}
\lr{ \frac{e B \rho_a^2}{2 c} }^2.
\end{dmath}
%
Now set
%
\begin{dmath}\label{eqn:gradQuantumProblemSet3Problem3:800}
\frac{-\Hbar^2}{2 m \Phi} \partial_{\phi\phi} \Phi + \frac{i \Hbar e B \rho_a^2}{2 m c \Phi} \partial_\phi \Phi + \inv{2m}
\lr{ \frac{e B \rho_a^2}{2 c} }^2
= E_\phi.
\end{dmath}
%
With \( \Phi = e^{i \nu \phi} \), that gives
%
\begin{dmath}\label{eqn:gradQuantumProblemSet3Problem3:820}
E_\phi
=
\frac{-\Hbar^2}{2 m } (-i \nu)^2 + \frac{i \Hbar e B \rho_a^2}{2 m c \Phi} (i\nu) + \inv{2m}
\lr{ \frac{e B \rho_a^2}{2 c} }^2
=
\frac{\Hbar^2 \nu^2}{2 m } - \frac{\Hbar e B \rho_a^2}{2 m c} \nu + \inv{2m}
\lr{ \frac{e B \rho_a^2}{2 c} }^2
=
\frac{\Hbar^2}{2 m} \lr{ \nu^2 - \frac{e B \rho_a^2}{\Hbar c} \nu }
+ \inv{2m}
\lr{ \frac{e B \rho_a^2}{2 c} }^2
=
\frac{\Hbar^2}{2 m}
\lr{
\lr{ \nu - \frac{e B \rho_a^2}{2 \Hbar c} }^2 - \lr{ \frac{e B \rho_a^2}{2 \Hbar c} }^2
}
+ \inv{2m}
\lr{ \frac{e B \rho_a^2}{2 c} }^2
=
\frac{\Hbar^2}{2 m}
\lr{ \nu - \frac{e B \rho_a^2}{2 \Hbar c} }^2
.
\end{dmath}
%
We are left with
%
\begin{dmath}\label{eqn:gradQuantumProblemSet3Problem3:840}
E' \rho^2
=
\frac{-\Hbar^2}{2 m} \lr{
\frac{\rho}{P} \partial_\rho \lr{ \rho \partial_\rho P }
}
+
E_\phi
=
\frac{-\Hbar^2}{2 m} \lr{
\frac{\rho}{P} \partial_\rho \lr{ \rho \partial_\rho P }
}
+
\frac{\Hbar^2}{2 m}
\lr{ \nu - \frac{e B \rho_a^2}{2 \Hbar c} }^2,
\end{dmath}
%
or
\begin{dmath}\label{eqn:gradQuantumProblemSet3Problem3:860}
\rho^2 P'' + \rho P +
\lr{ \frac{2 m E'}{\Hbar^2} \rho^2 -
\lr{ \nu - \frac{e B \rho_a^2}{2 \Hbar c} }^2 } P = 0.
\end{dmath}
%
With
%
\begin{equation}\label{eqn:gradQuantumProblemSet3Problem3:880}
\nu' = \nu - \frac{e B \rho_a^2}{2 \Hbar c},
\end{equation}
%
and
\begin{equation}\label{eqn:gradQuantumProblemSet3Problem3:900}
k = \frac{\sqrt{2 m E'}}{\Hbar},
\end{equation}
%
this is, once again, a Bessel equation with solution
%
\begin{equation}\label{eqn:gradQuantumProblemSet3Problem3:920}
P = a J_{\pm \nu'}(k \rho) + b N_{\nu'}(k \rho).
\end{equation}
%
The full solution is
%
\boxedEquation{eqn:gradQuantumProblemSet3Problem3:940}{
\psi = \lr{ a J_{\pm \nu'}( k \rho ) + b N_{\nu'}( k \rho ) } e^{i \nu \phi} e^{i \pi l z/L}.
}
%
\makeSubAnswer{}{gradQuantum:problemSet3:3c}
%
With \( \theta = \ifrac{e B \rho_a^2}{2 \Hbar c} \), and \( N = \nu \), \cref{eqn:gradQuantumProblemSet3Problem3:820} takes the form
%
\begin{equation}\label{eqn:gradQuantumProblemSet3Problem3:960}
\frac{2 m E_\phi}{\Hbar^2} = \lr{ N - \theta }^2,
\end{equation}
%
which has zeros when \( N = \theta \), or
%
\begin{equation}\label{eqn:gradQuantumProblemSet3Problem3:980}
N = \frac{e B \rho_a^2}{2 \Hbar c},
\end{equation}
%
Shuffling terms, this is
%
\begin{equation}\label{eqn:gradQuantumProblemSet3Problem3:1000}
\frac{2 \pi N \Hbar c}{e} = B \pi \rho_a^2,
\end{equation}
%
which is the desired result.
}
}
