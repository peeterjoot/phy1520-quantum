%
% Copyright � 2015 Peeter Joot.  All Rights Reserved.
% Licenced as described in the file LICENSE under the root directory of this GIT repository.
%
%\input{../blogpost.tex}
%\renewcommand{\basename}{gradQuantumProblemSet6Problem1}
%\renewcommand{\dirname}{notes/phy1520/}
%%\newcommand{\dateintitle}{}
%%\newcommand{\keywords}{}
%
%\input{../peeter_prologue_print2.tex}
%
%\usepackage{peeters_layout_exercise}
%\usepackage{peeters_braket}
%\usepackage{peeters_figures}
%
%\beginArtNoToc
%
%\generatetitle{Determining the rotation angle and normal for a rotation through Euler angles}
%%\chapter{Determining the rotation angle and normal for a rotation through Euler angles}
%%\label{chap:gradQuantumProblemSet6Problem1}

\makeoproblem{Rotation angle and normal for a rotation through Euler angles.}{problem:gradQuantumProblemSet6Problem1:1}{\citep{sakurai2014modern} pr. 3.9}
%\makeproblem{Sequence of rotations.}{problem:gradQuantumProblemSet6Problem1:1}
{
\index{Euler angle}

Consider a sequence of Euler rotations represented by
%
\begin{dmath}\label{eqn:gradQuantumProblemSet6Problem1:20}
\calD^{1/2}(\alpha, \beta, \gamma)
=
e^{-i \sigma_z \alpha/2} e^{-i \sigma_y \beta/2} e^{-i \sigma_z \gamma_2 }.
\end{dmath}
} % problem

Because rotations form a group, this sequence of rotations corresponds to a single rotation by an angle \( \theta \) about a new axis \( \ncap \). What is \(\theta\)? What is \( \ncap\)?

\makeanswer{problem:gradQuantumProblemSet6Problem1:1}{
\withproblemsetsParagraph{

First expand the z-axis rotations into their Pauli matrix form
%
\begin{dmath}\label{eqn:gradQuantumProblemSet6Problem1:200}
\begin{aligned}
e^{-i \sigma_z \mu/2}
&=
\cos(\mu/2) - i \PauliZ \sin(\mu/2) \\
&=
\begin{bmatrix}
\cos(\mu/2) -i \sin(\mu/2) & 0 \\
0 & \cos(\mu/2) +i \sin(\mu/2)
\end{bmatrix} \\
&=
\begin{bmatrix}
e^{-i \mu/2} & 0 \\
0 & e^{i \mu/2}
\end{bmatrix}.
\end{aligned}
\end{dmath}

For the y-axis rotation we have
\begin{dmath}\label{eqn:gradQuantumProblemSet6Problem1:220}
\begin{aligned}
e^{-i \sigma_y \beta/2}
&=
\cos(\beta/2) - i \PauliY \sin(\beta/2) \\
&=
\cos(\beta/2)
+
\begin{bmatrix}
0 & -1 \\
1 & 0
\end{bmatrix}
\sin(\beta/2) \\
&=
\begin{bmatrix}
\cos(\beta/2) & - \sin(\beta/2) \\
\sin(\beta/2) & \cos(\beta/2)
\end{bmatrix}.
\end{aligned}
\end{dmath}

The composition of rotations is therefore
\begin{dmath}\label{eqn:gradQuantumProblemSet6Problem1:240}
\begin{aligned}
\calD^{1/2}(\alpha, \beta, \gamma)
&=
\begin{bmatrix}
e^{-i \alpha/2} & 0 \\
0 & e^{i \alpha/2}
\end{bmatrix}
\begin{bmatrix}
\cos(\beta/2) & - \sin(\beta/2) \\
\sin(\beta/2) & \cos(\beta/2)
\end{bmatrix}
\begin{bmatrix}
e^{-i \gamma/2} & 0 \\
0 & e^{i \gamma/2}
\end{bmatrix} \\
&=
\begin{bmatrix}
e^{-i \alpha/2} & 0 \\
0 & e^{i \alpha/2}
\end{bmatrix}
\begin{bmatrix}
e^{-i \gamma/2} \cos(\beta/2) & - e^{i \gamma/2} \sin(\beta/2) \\
e^{-i \gamma/2} \sin(\beta/2) & e^{i \gamma/2} \cos(\beta/2)
\end{bmatrix} \\
&=
\begin{bmatrix}
e^{-i \alpha/2} e^{-i \gamma/2} \cos(\beta/2) & - e^{-i \alpha/2} e^{i \gamma/2} \sin(\beta/2) \\
e^{i \alpha/2} e^{-i \gamma/2} \sin(\beta/2) & e^{+i \alpha/2} e^{i \gamma/2} \cos(\beta/2)
\end{bmatrix} \\
&=
\begin{bmatrix}
e^{-i(\alpha+\gamma)/2} \cos \frac{\beta}{2} & -e^{-i(\alpha-\gamma)/2} \sin \frac{\beta}{2} \\
e^{i(\alpha-\gamma)/2} \sin \frac{\beta}{2} & e^{i(\alpha+\gamma)/2} \cos \frac{\beta}{2}
\end{bmatrix}.
\end{aligned}
\end{dmath}

Compare this to the matrix for a rotation (double sided) about a normal, given by
%
\begin{dmath}\label{eqn:gradQuantumProblemSet6Problem1:40}
\calR
= e^{-i \Bsigma \cdot \ncap \theta/2}
= \cos \frac{\theta}{2} I - i \Bsigma \cdot \ncap \sin \frac{\theta}{2}.
\end{dmath}

With \( \ncap = \lr{ \sin \Theta \cos\Phi, \sin \Theta \sin\Phi, \cos\Theta} \), the normal direction in its Pauli basis is
%
\begin{dmath}\label{eqn:gradQuantumProblemSet6Problem1:60}
\Bsigma \cdot \ncap
=
\begin{bmatrix}
\cos\Theta        & \sin \Theta \cos\Phi - i \sin \Theta \sin\Phi \\
\sin \Theta \cos\Phi + i \sin \Theta \sin\Phi & -\cos\Theta
\end{bmatrix}
=
\begin{bmatrix}
\cos\Theta        & \sin \Theta e^{-i \Phi} \\
\sin \Theta e^{i \Phi} & -\cos\Theta
\end{bmatrix},
\end{dmath}

so
%
\begin{dmath}\label{eqn:gradQuantumProblemSet6Problem1:80}
\calR =
\begin{bmatrix}
\cos \frac{\theta}{2} -i \sin \frac{\theta}{2} \cos\Theta & -i \sin \Theta e^{-i \Phi} \sin \frac{\theta}{2} \\
-i \sin \Theta e^{i \Phi} \sin \frac{\theta}{2}           & \cos \frac{\theta}{2} +i \sin \frac{\theta}{2} \cos\Theta \\
\end{bmatrix}.
\end{dmath}

It's not obvious how to put this into correspondence with the matrix for the Euler rotations.  Doing so certainly doesn't look fun.  To solve this problem, let's go the opposite direction, and put the matrix for the Euler rotations into the form of \cref{eqn:gradQuantumProblemSet6Problem1:40}.

That is
\begin{dmath}\label{eqn:gradQuantumProblemSet6Problem1:100}
\begin{aligned}
\calD^{1/2}(\alpha, \beta, \gamma)
&=
\begin{bmatrix}
e^{-i(\alpha+\gamma)/2} \cos \frac{\beta}{2} & -e^{-i(\alpha-\gamma)/2} \sin \frac{\beta}{2} \\
e^{i(\alpha-\gamma)/2} \sin \frac{\beta}{2} & e^{i(\alpha+\gamma)/2} \cos \frac{\beta}{2}
\end{bmatrix} \\
&=
\begin{bmatrix}
\cos\frac{\alpha+\gamma}{2} \cos \frac{\beta}{2} & - \cos\frac{\alpha-\gamma}{2} \sin \frac{\beta}{2} \\
\cos\frac{\alpha-\gamma}{2} \sin \frac{\beta}{2} & \cos\frac{\alpha+\gamma}{2} \cos \frac{\beta}{2}
\end{bmatrix} \\
&\quad +
i
\begin{bmatrix}
- \sin\frac{\alpha+\gamma}{2} \cos \frac{\beta}{2} & \sin\frac{\alpha-\gamma}{2} \sin \frac{\beta}{2} \\
 \sin\frac{\alpha-\gamma}{2} \sin \frac{\beta}{2} & \sin\frac{\alpha+\gamma}{2} \cos \frac{\beta}{2}
\end{bmatrix} \\
&=
\lr{\cos\frac{\alpha+\gamma}{2} \cos \frac{\beta}{2} }
+ \lr{i \sin\frac{\alpha-\gamma}{2} \sin \frac{\beta}{2} } \sigma_x \\
&\qquad - \lr{i \cos\frac{\alpha-\gamma}{2} \sin \frac{\beta}{2} } \sigma_y
- \lr{i \sin\frac{\alpha+\gamma}{2} \cos \frac{\beta}{2} } \sigma_z.
\end{aligned}
\end{dmath}

This gives us
%
\begin{equation}\label{eqn:gradQuantumProblemSet6Problem1:120}
\begin{aligned}
\cos\frac{\theta}{2} &= \cos\frac{\alpha+\gamma}{2} \cos \frac{\beta}{2} \\
\ncap \sin\frac{\theta}{2} &= \lr{ -\sin\frac{\alpha-\gamma}{2} \sin \frac{\beta}{2}, \cos\frac{\alpha-\gamma}{2} \sin \frac{\beta}{2}, \sin\frac{\alpha+\gamma}{2} \cos \frac{\beta}{2} }.
\end{aligned}
\end{equation}

The angle is
%
\begin{dmath}\label{eqn:gradQuantumProblemSet6Problem1:140}
\theta
= 2 \Atan \frac{
\sqrt{\sin^2\frac{\beta}{2} + \sin^2\frac{\alpha+\gamma}{2} \cos^2\frac{\beta}{2}
}
}{\cos\frac{\alpha+\gamma}{2} \cos \frac{\beta}{2}},
\end{dmath}

or
%\begin{dmath}\label{eqn:gradQuantumProblemSet6Problem1:180}
\boxedEquation{eqn:gradQuantumProblemSet6Problem1:180}{
\theta = 2 \Atan \frac{
\sqrt{\tan^2\frac{\beta}{2} + \sin^2\frac{\alpha+\gamma}{2}
}
}{\cos\frac{\alpha+\gamma}{2}
},
}
%\end{dmath}

and the normal direction is
%\begin{dmath}\label{eqn:gradQuantumProblemSet6Problem1:160}
\boxedEquation{eqn:gradQuantumProblemSet6Problem1:160}{
\ncap
=
\inv{\sqrt{1 - \cos^2\frac{\alpha+\gamma}{2} \cos^2\frac{\beta}{2} }}
\lr{ -\sin\frac{\alpha-\gamma}{2} \sin \frac{\beta}{2}, \cos\frac{\alpha-\gamma}{2} \sin \frac{\beta}{2}, \sin\frac{\alpha+\gamma}{2} \cos \frac{\beta}{2} }.
}
%\end{dmath}
}
} % answer

%\EndArticle
