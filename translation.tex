%
% Copyright � 2015 Peeter Joot.  All Rights Reserved.
% Licenced as described in the file LICENSE under the root directory of this GIT repository.
%
%\input{../blogpost.tex}
%\renewcommand{\basename}{translation}
%\renewcommand{\dirname}{notes/phy1520/}
%%\newcommand{\dateintitle}{}
%%\newcommand{\keywords}{}
%
%\input{../peeter_prologue_print2.tex}
%
%\usepackage{peeters_layout_exercise}
%\usepackage{peeters_braket}
%\usepackage{peeters_figures}
%\usepackage{peeters_qed}
%
%\beginArtNoToc
%
%\generatetitle{Translation operator problems}
%\chapter{Translation operator problems}
%\label{chap:translation}

%
\makeoproblem{One dimensional translation operator.}{problem:translation:28}{\citep{sakurai2014modern} pr. 1.28}{
\index{translation!operator}
%
\makesubproblem{}{problem:translation:28:a}
%
Evaluate the classical Poisson bracket
\index{Poisson bracket}
%
\begin{equation}\label{eqn:translation:420}
\antisymmetric{x}{F(p)}_{\textrm{classical}}
\end{equation}
%
\makesubproblem{}{problem:translation:28:b}
Evaluate the commutator
%
\begin{equation}\label{eqn:translation:440}
\antisymmetric{x}{e^{i p a/\Hbar}}
\end{equation}
%
\makesubproblem{}{problem:translation:28:c}
%
Using the result in \ref{problem:translation:28:b}, prove that
\begin{equation}\label{eqn:translation:460}
e^{i p a/\Hbar} \ket{x'},
\end{equation}
%
is an eigenstate of the coordinate operator \( x \).
} % problem
%
\makeanswer{problem:translation:28}{
%
\makeSubAnswer{}{problem:translation:28:a}
%
\begin{dmath}\label{eqn:translation:480}
\antisymmetric{x}{F(p)}_{\textrm{classical}}
=
\PD{x}{x} \PD{p}{F(p)} - \PD{p}{x} \PD{x}{F(p)}
=
\PD{p}{F(p)}.
\end{dmath}
%
\makeSubAnswer{}{problem:translation:28:b}
%
Having worked backwards through these problems, the answer for this one dimensional problem can be obtained from \cref{eqn:translation:140} and is
%
\begin{dmath}\label{eqn:translation:500}
\antisymmetric{x}{e^{i p a/\Hbar}} = a e^{i p a/\Hbar}.
\end{dmath}
%
\makeSubAnswer{}{problem:translation:28:c}
%
\begin{dmath}\label{eqn:translation:520}
x e^{i p a/\Hbar} \ket{x'}
=
\lr{
\antisymmetric{x}{e^{i p a/\Hbar}}
e^{i p a/\Hbar} x
+
}
\ket{x'}
=
\lr{ a e^{i p a/\Hbar} + e^{i p a/\Hbar} x ' } \ket{x'}
= \lr{ a + x' } \ket{x'}.
\end{dmath}
%
This demonstrates that \( e^{i p a/\Hbar} \ket{x'} \) is an eigenstate of \( x \) with eigenvalue \( a + x' \).

} % answer
%
\makeoproblem{Polynomial commutators.}{problem:translation:29}{\citep{sakurai2014modern} pr. 1.29}{
\index{commutator!polynomial functions}
%
\makesubproblem{}{problem:translation:29:a}
For power series \( F, G \), verify
%
\begin{equation}\label{eqn:translation:180}
\antisymmetric{x_k}{G(\Bp)} = i \Hbar \PD{p_k}{G}, \qquad
\antisymmetric{p_k}{F(\Bx)} = -i \Hbar \PD{x_k}{F}.
\end{equation}
%
\makesubproblem{}{problem:translation:29:b}
%
Evaluate \( \antisymmetric{x^2}{p^2} \), and compare to the classical Poisson bracket \( \antisymmetric{x^2}{p^2}_{\textrm{classical}} \).

} % problem
%
\makeanswer{problem:translation:29}{
\makeSubAnswer{}{problem:translation:29:a}
%
Let
%
\begin{equation}\label{eqn:translation:200}
\begin{aligned}
G(\Bp) &= \sum_{k l m} a_{k l m} p_1^k p_2^l p_3^m \\
F(\Bx) &= \sum_{k l m} b_{k l m} x_1^k x_2^l x_3^m.
\end{aligned}
\end{equation}
%
It is simpler to work with a specific \( x_k \), say \( x_k = y \).  The validity of the general result will still be clear doing so.  Expanding the commutator gives
%
\begin{dmath}\label{eqn:translation:220}
\antisymmetric{y}{G(\Bp)}
=
\sum_{k l m} a_{k l m} \antisymmetric{y}{p_1^k p_2^l p_3^m }
=
\sum_{k l m} a_{k l m} \lr{
y p_1^k p_2^l p_3^m  - p_1^k p_2^l p_3^m  y
}
=
\sum_{k l m} a_{k l m} \lr{
p_1^k y p_2^l p_3^m  - p_1^k y p_2^l p_3^m
}
=
\sum_{k l m} a_{k l m}
p_1^k
\antisymmetric{y}{p_2^l}
p_3^m.
\end{dmath}
%
From \cref{eqn:translation:100}, we have \( \antisymmetric{y}{p_2^l} = l i \Hbar p_2^{l-1} \), so
%
\begin{dmath}\label{eqn:translation:240}
\antisymmetric{y}{G(\Bp)}
=
\sum_{k l m} a_{k l m}
p_1^k
\antisymmetric{y}{p_2^l}
\lr{ l
i \Hbar p_2^{l-1}
}
p_3^m
=
i \Hbar \PD{y}{G(\Bp)}.
\end{dmath}
%
It is straightforward to show that
\( \antisymmetric{p}{x^l} = -l i \Hbar x^{l-1} \), allowing for a similar computation of the momentum commutator
%
\begin{dmath}\label{eqn:translation:260}
\antisymmetric{p_y}{F(\Bx)}
=
\sum_{k l m} b_{k l m} \antisymmetric{p_y}{x_1^k x_2^l x_3^m }
=
\sum_{k l m} b_{k l m} \lr{
p_y x_1^k x_2^l x_3^m  - x_1^k x_2^l x_3^m  p_y
}
=
\sum_{k l m} b_{k l m} \lr{
x_1^k p_y x_2^l x_3^m  - x_1^k p_y x_2^l x_3^m
}
=
\sum_{k l m} b_{k l m}
x_1^k
\antisymmetric{p_y}{x_2^l}
x_3^m
=
\sum_{k l m} b_{k l m}
x_1^k
\lr{ -l i \Hbar x_2^{l-1}}
x_3^m
=
-i \Hbar \PD{p_y}{F(\Bx)}.
\end{dmath}
%
\makeSubAnswer{}{problem:translation:29:b}
%
It isn't clear to me how the results above can be used directly to compute \( \antisymmetric{x^2}{p^2} \).  However, when the first term of such a commutator is a mononomial, it can be expanded in terms of an \( x \) commutator
%
\begin{dmath}\label{eqn:translation:280}
\antisymmetric{x^2}{G(\Bp)}
= x^2 G - G x^2
= x \lr{ x G } - G x^2
= x \lr{ \antisymmetric{ x }{ G } + G x } - G x^2
= x \antisymmetric{ x }{ G } + \lr{ x G } x - G x^2
= x \antisymmetric{ x }{ G } + \lr{ \antisymmetric{ x }{ G } + \cancel{G x} } x - \cancel{G x^2}
= x \antisymmetric{ x }{ G } + \antisymmetric{ x }{ G } x.
\end{dmath}
%
Similarly,
%
\begin{dmath}\label{eqn:translation:300}
\antisymmetric{x^3}{G(\Bp)} = x^2 \antisymmetric{ x }{ G } + x \antisymmetric{ x }{ G } x + \antisymmetric{ x }{ G } x^2.
\end{dmath}
%
An induction hypothesis can be formed
%
\begin{dmath}\label{eqn:translation:320}
\antisymmetric{x^k}{G(\Bp)} = \sum_{j = 0}^{k-1} x^{k-1-j} \antisymmetric{ x }{ G } x^j,
\end{dmath}
%
and demonstrated
%
\begin{dmath}\label{eqn:translation:340}
\antisymmetric{x^{k+1}}{G(\Bp)}
=
x^{k+1} G - G x^{k+1}
=
x \lr{ x^{k} G } - G x^{k+1}
=
x \lr{ \antisymmetric{x^{k}}{G} + G x^k } - G x^{k+1}
=
x \antisymmetric{x^{k}}{G} + \lr{ x G } x^k - G x^{k+1}
=
x \antisymmetric{x^{k}}{G} + \lr{ \antisymmetric{x}{G} + G x } x^k - G x^{k+1}
=
x \antisymmetric{x^{k}}{G} + \antisymmetric{x}{G} x^k
=
x \sum_{j = 0}^{k-1} x^{k-1-j} \antisymmetric{ x }{ G } x^j + \antisymmetric{x}{G} x^k
=
\sum_{j = 0}^{k-1} x^{(k+1)-1-j} \antisymmetric{ x }{ G } x^j + \antisymmetric{x}{G} x^k
=
\sum_{j = 0}^{k} x^{(k+1)-1-j} \antisymmetric{ x }{ G } x^j. \qedmarker
\end{dmath}

That was a bit overkill for this problem, but may be useful later.  Application of this to the problem gives
%
\begin{dmath}\label{eqn:translation:360}
\antisymmetric{x^2}{p^2}
=
x \antisymmetric{x}{p^2}
+ \antisymmetric{x}{p^2} x
=
x i \Hbar \PD{x}{p^2}
+ i \Hbar \PD{x}{p^2} x
=
x 2 i \Hbar p
+ 2 i \Hbar p x
= i \Hbar \lr{ 2 x p + 2 p x }.
\end{dmath}
%
The classical commutator is
\begin{dmath}\label{eqn:translation:380}
\antisymmetric{x^2}{p^2}_{\textrm{classical}}
=
\PD{x}{x^2} \PD{p}{p^2} - \PD{p}{x^2} \PD{x}{p^2}
=
2 x 2 p
= 2 x p + 2 p x.
\end{dmath}
%
This demonstrates the expected relation between the classical and quantum commutators
%
\begin{equation}\label{eqn:translation:400}
\antisymmetric{x^2}{p^2} = i \Hbar \antisymmetric{x^2}{p^2}_{\textrm{classical}}.
\end{equation}
%
} % answer
%
\makeoproblem{Translation operator and position expectation.}{problem:translation:30}{\citep{sakurai2014modern} pr. 1.30}{
\index{translation!expectation}

The translation operator for a finite spatial displacement is given by
%
\begin{dmath}\label{eqn:translation:20}
\mathcal{T}(\Bl) = \exp\lr{ -i \Bp \cdot \Bl/\Hbar },
\end{dmath}
%
where \( \Bp \) is the momentum operator.
%
\makesubproblem{}{problem:translation:1.30:a}
%
Evaluate
%
\begin{dmath}\label{eqn:translation:40}
\antisymmetric{x_i}{\mathcal{T}(\Bl)}.
\end{dmath}
%
\makesubproblem{}{problem:translation:1.30:b}
Demonstrate how the expectation value \( \expectation{\Bx} \) changes under translation.

} % problem
%
\makeanswer{problem:translation:30}{
%
\makeSubAnswer{}{problem:translation:1.30:a}
%
For clarity, let's set \( x_i = y \).  The general result will be clear despite doing so.
%
\begin{dmath}\label{eqn:translation:60}
\antisymmetric{y}{\mathcal{T}(\Bl)}
=
\sum_{k= 0} \inv{k!} \lr{\frac{-i}{\Hbar}}
\antisymmetric{y}{
\lr{ \Bp \cdot \Bl }^k
}.
\end{dmath}
%
The commutator expands as
%
\begin{dmath}\label{eqn:translation:80}
\antisymmetric{y}{
\lr{ \Bp \cdot \Bl }^k
}
+ \lr{ \Bp \cdot \Bl }^k y
=
y \lr{ \Bp \cdot \Bl }^k
=
y \lr{ p_x l_x + p_y l_y + p_z l_z } \lr{ \Bp \cdot \Bl }^{k-1}
=
\lr{ p_x l_x y + y p_y l_y + p_z l_z y } \lr{ \Bp \cdot \Bl }^{k-1}
=
\lr{ p_x l_x y + l_y \lr{ p_y y + i \Hbar } + p_z l_z y } \lr{ \Bp \cdot \Bl }^{k-1}
=
\lr{ \Bp \cdot \Bl } y \lr{ \Bp \cdot \Bl }^{k-1}
+ i \Hbar l_y \lr{ \Bp \cdot \Bl }^{k-1}
= \cdots
=
\lr{ \Bp \cdot \Bl }^{k-1} y \lr{ \Bp \cdot \Bl }^{k-(k-1)}
+ (k-1) i \Hbar l_y \lr{ \Bp \cdot \Bl }^{k-1}
=
\lr{ \Bp \cdot \Bl }^{k} y
+ k i \Hbar l_y \lr{ \Bp \cdot \Bl }^{k-1}.
\end{dmath}
%
In the above expansion, the commutation of \( y \) with \( p_x, p_z \) has been used.  This gives, for \( k \ne 0 \),
%
\begin{dmath}\label{eqn:translation:100}
\antisymmetric{y}{
\lr{ \Bp \cdot \Bl }^k
}
=
k i \Hbar l_y \lr{ \Bp \cdot \Bl }^{k-1}.
\end{dmath}
%
Note that this also holds for the \( k = 0 \) case, since \( y \) commutes with the identity operator.  Plugging back into the \( \mathcal{T} \) commutator, we have
%
\begin{dmath}\label{eqn:translation:120}
\antisymmetric{y}{\mathcal{T}(\Bl)}
=
\sum_{k = 1} \inv{k!} \lr{\frac{-i}{\Hbar}}
k i \Hbar l_y \lr{ \Bp \cdot \Bl }^{k-1}
=
l_y \sum_{k = 1} \inv{(k-1)!} \lr{\frac{-i}{\Hbar}}
\lr{ \Bp \cdot \Bl }^{k-1}
=
l_y \mathcal{T}(\Bl).
\end{dmath}
%
The same pattern clearly applies with the other \( x_i \) values, providing the desired relation.
%
\begin{equation}\label{eqn:translation:140}
\antisymmetric{\Bx}{\mathcal{T}(\Bl)} = \sum_{m = 1}^3 \Be_m l_m \mathcal{T}(\Bl) = \Bl \mathcal{T}(\Bl).
\end{equation}
%
\makeSubAnswer{}{problem:translation:1.30:b}
%
Suppose that the translated state is defined as \( \ket{\alpha_{\Bl}} = \mathcal{T}(\Bl) \ket{\alpha} \).  The expectation value with respect to this state is
%
\begin{dmath}\label{eqn:translation:160}
\expectation{\Bx'}
=
\bra{\alpha_{\Bl}} \Bx \ket{\alpha_{\Bl}}
=
\bra{\alpha} \mathcal{T}^\dagger(\Bl) \Bx \mathcal{T}(\Bl) \ket{\alpha}
=
\bra{\alpha} \mathcal{T}^\dagger(\Bl) \lr{ \Bx \mathcal{T}(\Bl) } \ket{\alpha}
=
\bra{\alpha} \mathcal{T}^\dagger(\Bl) \lr{ \mathcal{T}(\Bl) \Bx + \Bl \mathcal{T}(\Bl) } \ket{\alpha}
=
\bra{\alpha} \mathcal{T}^\dagger \mathcal{T} \Bx + \Bl \mathcal{T}^\dagger \mathcal{T} \ket{\alpha}
=
\bra{\alpha} \Bx \ket{\alpha} + \Bl \braket{\alpha}{\alpha}
=
\expectation{\Bx} + \Bl.
\end{dmath}
%
} % answer

%\EndArticle
