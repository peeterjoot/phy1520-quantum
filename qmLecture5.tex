%
% Copyright � 2015 Peeter Joot.  All Rights Reserved.
% Licenced as described in the file LICENSE under the root directory of this GIT repository.
%
%\input{../blogpost.tex}
%\renewcommand{\basename}{qmLecture5}
%\renewcommand{\dirname}{notes/phy1520/}
%\newcommand{\keywords}{PHY1520H}
%\input{../peeter_prologue_print2.tex}
%
%%\usepackage{phy1520}
%\usepackage{peeters_braket}
%%\usepackage{peeters_layout_exercise}
%\usepackage{peeters_figures}
%\usepackage{mathtools}
%
%\beginArtNoToc
%\generatetitle{PHY1520H Graduate Quantum Mechanics.  Lecture 5: time evolution of coherent states, and charged particles in a magnetic field.  Taught by Prof.\ Arun Paramekanti}
%\label{chap:qmLecture5}
%
%\paragraph{Disclaimer}
%
%Peeter's lecture notes from class.  These may be incoherent and rough.
%
%These are notes for the UofT course PHY1520, Graduate Quantum Mechanics, taught by Prof. Paramekanti, covering \textchapref{{1}} \citep{sakurai2014modern} content.
%
\section{Expectation with respect to coherent states}

\index{expectation!coherent state}
A coherent state for the SHO \( H = \lr{ N + \inv{2} } \Hbar \omega \) was given by
%
\begin{equation}\label{eqn:qmLecture5:20}
a \ket{z} = z \ket{z},
\end{equation}
%
where we showed that
%
\begin{equation}\label{eqn:qmLecture5:40}
\ket{z} = c_0 e^{ z a^\dagger } \ket{0}.
\end{equation}
%
In the Heisenberg picture we found
%
\begin{equation}\label{eqn:qmLecture5:60}
\begin{aligned}
a_\txtH(t) &= e^{i H t/\Hbar} a e^{-i H t/\Hbar} = a e^{-i\omega t} \\
a_\txtH^\dagger(t) &= e^{i H t/\Hbar} a^\dagger e^{-i H t/\Hbar} = a^\dagger e^{i\omega t}.
\end{aligned}
\end{equation}
%
Recall that the position and momentum representation of the ladder operators was
%
\begin{equation}\label{eqn:qmLecture5:80}
\begin{aligned}
a &= \inv{\sqrt{2}} \lr{ \hatx \sqrt{\frac{m \omega}{\Hbar}} + i \hatp \sqrt{\inv{m \Hbar \omega}} } \\
a^\dagger &= \inv{\sqrt{2}} \lr{ \hatx \sqrt{\frac{m \omega}{\Hbar}} - i \hatp \sqrt{\inv{m \Hbar \omega}} },
\end{aligned}
\end{equation}

or equivalently
\begin{equation}\label{eqn:qmLecture5:100}
\begin{aligned}
\hatx &= \lr{ a + a^\dagger } \sqrt{\frac{\Hbar}{ 2 m \omega}} \\
\hatp &= i \lr{ a^\dagger - a } \sqrt{\frac{m \Hbar \omega}{2}}.
\end{aligned}
\end{equation}
%
Given this we can compute expectation value of position operator
%
\begin{dmath}\label{eqn:qmLecture5:120}
\bra{z} \hatx \ket{z}
=
\sqrt{\frac{\Hbar}{ 2 m \omega}}
\bra{z}
\lr{ a + a^\dagger }
\ket{z}
=
\lr{ z + z^\conj } \sqrt{\frac{\Hbar}{ 2 m \omega}}
=
2 \Real z \sqrt{\frac{\Hbar}{ 2 m \omega}} .
\end{dmath}
%
Similarly
%
\begin{dmath}\label{eqn:qmLecture5:140}
\bra{z} \hatp \ket{z}
=
i \sqrt{\frac{m \Hbar \omega}{2}}
\bra{z}
\lr{ a^\dagger - a }
\ket{z}
=
\sqrt{\frac{m \Hbar \omega}{2}}
2 \Imag z.
\end{dmath}
%
How about the expectation of the Heisenberg position operator?  That is
%
\begin{dmath}\label{eqn:qmLecture5:160}
\bra{z} \hatx_\txtH(t) \ket{z}
=
\sqrt{\frac{\Hbar}{2 m \omega}} \bra{z} \lr{ a + a^\dagger } \ket{z}
=
\sqrt{\frac{\Hbar}{2 m \omega}} \lr{ z e^{-i \omega t} + z^\conj e^{i \omega t}}
=
\sqrt{\frac{\Hbar}{2 m \omega}} \lr{ \lr{z + z^\conj} \cos( \omega t ) -i \lr{ z - z^\conj } \sin( \omega t) }
=
\sqrt{\frac{\Hbar}{2 m \omega}} \lr{ \expectation{x(0)} \sqrt{ \frac{2 m \omega}{\Hbar}} \cos( \omega t ) -i \expectation{p(0)} i \sqrt{\frac{2 m \omega}{\Hbar} } \sin( \omega t) }
=
\expectation{x(0)} \cos( \omega t ) + \frac{\expectation{p(0)}}{m \omega} \sin( \omega t) .
\end{dmath}
%
\index{coherent state!position operator}
We find that the average of the Heisenberg position operator evolves in time in exactly the same fashion as position in the classical Harmonic oscillator.  This phase space like trajectory is sketched in \cref{fig:zStateProjectionsAverageXandP:zStateProjectionsAverageXandPFig1}.

\imageFigure{../figures/phy1520-quantum/zStateProjectionsAverageXandPFig1}{Phase space like trajectory.}{fig:zStateProjectionsAverageXandP:zStateProjectionsAverageXandPFig1}{0.3}

In the text it is shown that we have the same structure for the Heisenberg operator itself, before taking expectations
%
\begin{dmath}\label{eqn:qmLecture5:220}
\hatx_\txtH(t)
=
{x(0)} \cos( \omega t ) + \frac{{p(0)}}{m \omega} \sin( \omega t).
\end{dmath}
%
Where the coherent states become useful is that we will see that the second moments of position and momentum are not time dependent with respect to the coherent states.  Such states remain localized.

\section{Coherent state uncertainty}
\index{coherent state!uncertainty}

First note that using the commutator relationship we have
%
\begin{dmath}\label{eqn:qmLecture5:180}
\bra{z} a a^\dagger \ket{z}
=
\bra{z} \lr{ \antisymmetric{a}{a^\dagger} + a^\dagger a } \ket{z}
=
\bra{z} \lr{ 1 + a^\dagger a } \ket{z}.
\end{dmath}
%
For the second moment we have
%
\begin{dmath}\label{eqn:qmLecture5:200}
\bra{z} \hatx^2 \ket{z}
=
\frac{\Hbar}{ 2 m \omega}
\bra{z} \lr{a + a^\dagger } \lr{a + a^\dagger }  \ket{z}
=
\frac{\Hbar}{ 2 m \omega}
\bra{z} \lr{
a^2 + {(a^\dagger)}^2 + a a^\dagger + a^\dagger a
} \ket{z}
=
\frac{\Hbar}{ 2 m \omega}
\bra{z} \lr{
a^2 + {(a^\dagger)}^2 + 2 a^\dagger a + 1
} \ket{z}
=
\frac{\Hbar}{ 2 m \omega}
\lr{ z^2 + {(z^\conj)}^2 + 2 z^\conj z + 1}  \ket{z}
=
\frac{\Hbar}{ 2 m \omega}
\lr{ z + z^\conj }^2
+
\frac{\Hbar}{ 2 m \omega}.
\end{dmath}
%
We find
%
\begin{dmath}\label{eqn:qmLecture5:240}
\sigma_x^2 = \frac{\Hbar}{ 2 m \omega},
\end{dmath}
%
and
%
\begin{dmath}\label{eqn:qmLecture5:260}
\sigma_p^2 = \frac{m \Hbar \omega}{2}
\end{dmath}

so
%
\begin{dmath}\label{eqn:qmLecture5:280}
\sigma_x^2 \sigma_p^2 = \frac{\Hbar^2}{4},
\end{dmath}
%
or
%
\begin{dmath}\label{eqn:qmLecture5:300}
\sigma_x \sigma_p = \frac{\Hbar}{2}.
\end{dmath}
%
This is the minimum uncertainty.

\section{Quantum Field theory}
\index{field theory}

In Quantum Field theory the ideas of isolated oscillators is used to model particle creation.  The lowest energy state (a no particle, vacuum state) is given the lowest energy level, with each additional quantum level modeling a new particle creation state as sketched in \cref{fig:qftEnergyLevels:qftEnergyLevelsFig2}.

\imageFigure{../figures/phy1520-quantum/qftEnergyLevelsFig2}{QFT energy levels.}{fig:qftEnergyLevels:qftEnergyLevelsFig2}{0.3}

We have to imagine many oscillators, each with a distinct vacuum energy \( \sim \Bk^2 \) .  The Harmonic oscillator can be used to model the creation of particles with \( \Hbar \omega \) energy differences from that ``vacuum energy''.

\section{Charged particle in a magnetic field}
\index{magnetic field}

In the classical case ( with SI units or \( c = 1 \) ) we have
%
\begin{dmath}\label{eqn:qmLecture5:320}
\BF = q \BE + q \Bv \cross \BB.
\end{dmath}
%
Alternately, we can look at the Hamiltonian view of the system, written in terms of potentials
%
\begin{dmath}\label{eqn:qmLecture5:340}
\BB = \spacegrad \cross \BA,
\end{dmath}
\begin{dmath}\label{eqn:qmLecture5:360}
\BE = - \spacegrad \phi - \PD{t}{\BA}.
\end{dmath}
%
Note that the curl form for the magnetic field implies one of the required Maxwell's equations \( \spacegrad \cdot \BB = 0 \).

Ignoring time dependence of the potentials, the Hamiltonian can be expressed as
%
\begin{dmath}\label{eqn:qmLecture5:380}
H = \inv{2 m} \lr{ \Bp - q \BA }^2 + q \phi.
\end{dmath}
%
In this Hamiltonian the vector \( \Bp \) is called the canonical momentum, the momentum conjugate to position in phase space.

It is left as an exercise to show that the Lorentz force equation results from application of the Hamiltonian equations of motion, and that the velocity is given by \( \Bv = (\Bp - q \BA)/m \).

For quantum mechanics, we use the same Hamiltonian, but promote our position, momentum and potentials to operators.
%
\begin{dmath}\label{eqn:qmLecture5:400}
\hatH = \inv{2 m} \lr{ \pcap - q \Acap(\Br, t) }^2 + q \hat\phi(\Br, t).
\end{dmath}
%
\section{Gauge invariance}
\index{gauge invariance}

Can we say anything about this before looking at the question of a particle in a magnetic field?

Recall that the we can make a gauge transformation of the form

\begin{subequations}
\label{eqn:qmLecture5:420a}
\begin{dmath}\label{eqn:qmLecture5:420}
\BA \rightarrow \BA + \spacegrad \chi
\end{dmath}
\begin{dmath}\label{eqn:qmLecture5:440}
\phi \rightarrow \phi - \PD{t}{\chi}
\end{dmath}
\end{subequations}

Does this notion of gauge invariance also carry over to the Quantum Hamiltonian.  After gauge transformation we have
%
\begin{dmath}\label{eqn:qmLecture5:460}
\hatH'
= \inv{2 m} \lr{ \pcap - q \BA - q \spacegrad \chi }^2 + q \lr{ \phi - \PD{t}{\chi} }
\end{dmath}
%
Now we are in a mess, since this function \( \chi \) can make the Hamiltonian horribly complicated.  We don't see how gauge invariance can easily be applied to the quantum problem.  Next time we will introduce a transformation that resolves some of this mess.
%
%\EndArticle
