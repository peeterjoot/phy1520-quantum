%
% Copyright � 2015 Peeter Joot.  All Rights Reserved.
% Licenced as described in the file LICENSE under the root directory of this GIT repository.
%
\makeoproblem{Helium-4 atom.}{gradQuantum:problemSet7:2}{2015 ps7 p2}{
\index{helium-4}

Consider the Helium atom with atomic number \( Z=2 \), which leads to the nuclear charge \( Z=2e \), and two electrons with charge \( -e \) each.
%
\makesubproblem{}{gradQuantum:problemSet7:2a}
Show that ignoring electron-electron interactions leads to a ground state energy \( E_{\textrm{He}} = 4 E_\txtH \)
where \( E_\txtH \) is the ground state energy of the hydrogen atom.
%
\makesubproblem{}{gradQuantum:problemSet7:2b}
Consider the full problem which retains the Coulomb interaction between the electrons, i.e.
%
\begin{dmath}\label{eqn:gradQuantumProblemSet7Problem2:20}
H
=
\inv{2m} \lr{ \Bp_1^2 + \Bp_2^2 }
-
2 e^2
\inv{4 \pi \epsilon_0 }
\lr{ \inv{r_1} + \inv{r_2} }
+
e^2
\inv{4 \pi \epsilon_0 }
\inv{ \Abs{\Br_1 - \Br_2} }.
\end{dmath}
%
and consider the variational wavefunction
\begin{dmath}\label{eqn:gradQuantumProblemSet7Problem2:40}
\psi(\Br_1, \Br_2)
=
N
e^{- \inv{a} \lr{ r_1 + r_2 } }.
\end{dmath}
%
where \( N \) is the normalization constant, and \( a \) is a variational parameter. Determine the variational ground state energy, and minimize with respect to a to find the best estimate for the ground state energy of Helium.
Compare with numerical estimates of the energy.
} % makeproblem
%
\makeanswer{gradQuantum:problemSet7:2}{
\withproblemsetsParagraph{
\makeSubAnswer{}{gradQuantum:problemSet7:2a}
%
%\paragraph{Comparing the Hydrogen and Helium ground state energies.}

%Having initially thought that I had to show that \( E_{\textrm{He}} = 4 E_\txtH\), and geting an 8 times difference above, I went looking for mistakes and tried the Helium ground state a different way.
Without the electron-electron interaction term, the Helium Hamiltonian is separable.  Assuming a wave function of the form
%
\begin{dmath}\label{eqn:gradQuantumProblemSet7Problem2:660}
\psi(r_1, r_2) = \psi_1(r_1) \psi_2(r_2),
\end{dmath}
%
The Hamiltonian action on this wave function is
%
\begin{dmath}\label{eqn:gradQuantumProblemSet7Problem2:680}
E \psi_1(r_1) \psi_2(r_2)
=
\lr{ \inv{2m} \Bp_1^2 \psi_1(r_1) } \psi_2(r_2) + \lr{ \inv{2m} \Bp_2^2 \psi_2(r_1) } \psi_1(r_2)
-
2 e^2
\inv{4 \pi \epsilon_0 } \inv{r_1} \psi_1(r_1) \psi_2(r_2)
-
2 e^2
\inv{4 \pi \epsilon_0 } \inv{r_1} \psi_1(r_1) \psi_2(r_2),
\end{dmath}
%
or
\begin{dmath}\label{eqn:gradQuantumProblemSet7Problem2:700}
E
=
\lr{\inv{\psi_1(r)} \lr{ \inv{2m} \Bp_1^2 \psi_1(r_1) }
-
2 e^2
\inv{4 \pi \epsilon_0 } \inv{r_1} }
+
\lr{\inv{\psi_2(r)} \lr{ \inv{2m} \Bp_2^2 \psi_2(r_2) }
-
2 e^2
\inv{4 \pi \epsilon_0 } \inv{r_2}}.
\end{dmath}
%
This can be written in separated form as
%
\begin{equation}\label{eqn:gradQuantumProblemSet7Problem2:720}
\begin{aligned}
E_1 \psi_1(r_1) &= \inv{2m} \Bp_1^2 \psi_1(r_1) - 2 e^2 \inv{4 \pi \epsilon_0 } \inv{r_1} \psi_1(r_1) \\
E_2 \psi_2(r_2) &= \inv{2m} \Bp_2^2 \psi_2(r_2) - 2 e^2 \inv{4 \pi \epsilon_0 } \inv{r_2} \psi_2(r_2) \\
E &= E_1 + E_2.
\end{aligned}
\end{equation}
%
Observe that each of these separated Hamiltonians have (with \( Z = 2 \) ) the form
%
\begin{equation}\label{eqn:gradQuantumProblemSet7Problem2:740}
E \psi(r) = \inv{2m} \Bp^2 \psi(r) - Z e^2 \inv{4 \pi \epsilon_0 } \inv{r} \psi(r).
\end{equation}
%
With \( Z = 1 \) that is precisely the Hamiltonian for the Hydrogen atom.  If the wavefunction for this Hamiltonian is assumed to be \( \psi(r) = e^{-r/a} \), we find
%
\begin{equation}\label{eqn:gradQuantumProblemSet7Problem2:760}
\frac{\bra{\psi} H \ket{\psi} }{\braket{\psi}{\psi}}
=
\frac{\Hbar^2}{2 m a^2} - \frac{Z e^2}{4 \pi \epsilon_0 a},
\end{equation}
%
which has its minimum at
%
\begin{equation}\label{eqn:gradQuantumProblemSet7Problem2:780}
a_{\mathrm{min}} = \frac{a_0}{Z},
\end{equation}
%
where
\begin{dmath}\label{eqn:gradQuantumProblemSet7Problem2:880}
a_0 = \frac{4 \pi \epsilon_0 \Hbar^2}{m e^2}.
\end{dmath}
%
The minimum energy is found to be
%
\begin{equation}\label{eqn:gradQuantumProblemSet7Problem2:800}
E_{\mathrm{min}} = -\inv{2} \frac{e^2 Z^2}{ 4 \pi \epsilon_0 a_0 }.
\end{equation}
%
With \( Z = 1 \), the Hydrogen ground state energy is
%
\begin{equation}\label{eqn:gradQuantumProblemSet7Problem2:820}
E_\txtH
= -\inv{2} \frac{e^2}{ 4 \pi \epsilon_0 a_0 },
\end{equation}
%
a value of about \( -13.6 \si{eV} \).  The Helium ground state energy is
%
\begin{equation}\label{eqn:gradQuantumProblemSet7Problem2:840}
\begin{aligned}
E_{\textrm{He}}
&= \evalbar{E_1}{Z=2} + \evalbar{E_2}{Z=2} \\
&= -\inv{2} \lr{ 2^2 + 2^2 } \frac{e^2 }{ 4 \pi \epsilon_0 a_0 } \\
&= - 4 \frac{e^2 }{ 4 \pi \epsilon_0 a_0 }.
\end{aligned}
\end{equation}
%
This is \( E_{\textrm{He}} = 8 E_\txtH\), a value of about \( -109 eV \).

The computations above can be found in \nbref{ps7:heliumAtomGroundStateWithInteraction.nb}.
%
\makeSubAnswer{}{gradQuantum:problemSet7:2b}
%
The Laplacian of an exponentially decreasing trial function \( e^{-r/a} \) is
%
\begin{dmath}\label{eqn:gradQuantumProblemSet7Problem2:340}
\begin{aligned}
\spacegrad^2 e^{-r/a}
&=
\inv{r^2} \PD{r}{} \lr{ r^2 \PD{r}{e^{-r/a}} } \\
&=
\inv{r^2} \PD{r}{} \lr{ -\frac{r^2}{a} e^{-r/a} } \\
&=
-\inv{r^2 a} \lr{ 2 r - \frac{r^2}{a} } e^{-r/a},
\end{aligned}
\end{dmath}

%To calculate \( \Bp_j^2 \psi \) first compute the Laplacian of an exponential
%
%\begin{dmath}\label{eqn:gradQuantumProblemSet7Problem2:60}
%\spacegrad^2 e^{\phi}
%=
%\spacegrad \cdot \spacegrad e^\phi
%=
%\spacegrad \cdot \lr{ e^\phi \spacegrad \phi }
%=
%e^\phi \spacegrad^2 \phi + \lr{ \spacegrad \phi }^2 e^\phi
%=
%\lr{ \spacegrad^2 \phi + \lr{ \spacegrad \phi }^2  } e^\phi.
%\end{dmath}
%
%For \( \phi = -r/a \), we have
%
%\begin{dmath}\label{eqn:gradQuantumProblemSet7Problem2:80}
%\spacegrad \phi
%=
%- \inv{a} \spacegrad \sqrt{ \Bx^2 }
%=
%- \inv{a} \spacegrad \sqrt{ x_k x_k }
%=
%- \inv{a} \inv{2 r} \Be_j ( 2 \partial_j x_k ) x_k
%=
%- \frac{\Bx}{a r},
%\end{dmath}
%
%and
%\begin{dmath}\label{eqn:gradQuantumProblemSet7Problem2:100}
%\spacegrad^2 \phi
%=
%-\inv{a} \spacegrad \cdot \frac{ \Bx}{ r}
%=
%-\inv{a} \lr{
%\inv{r} \spacegrad \cdot \Bx
%+
%\Bx \cdot \spacegrad \inv{ r }
%}
%=
%-\inv{a} \lr{
%\frac{3}{r}
%+
%\Bx \cdot \lr{ -\inv{r^3} \Bx }
%}
%=
%-\inv{a} \lr{
%\frac{3}{r}
%-
%\frac{1}{r}
%}
%=
%-\frac{2}{a r}.
%\end{dmath}
%
or
%
\begin{dmath}\label{eqn:gradQuantumProblemSet7Problem2:120}
\spacegrad^2 e^{-r/a} = \inv{a} \lr{ \inv{a} -\frac{2}{r} } e^{-r/a}.
\end{dmath}
%
%%%For the Hydrogen atom (with \( a = a_0 \)), the Hamiltonian action on the unnormalized ground state wavefunction \( \psi = e^{-r/a} \) is
%%%
%%%\begin{dmath}\label{eqn:gradQuantumProblemSet7Problem2:380}
%%%H \psi(r)
%%%=
%%%\frac{\Bp^2}{2m} \psi
%%%- e^2 \inv{4 \pi \epsilon_0 } \inv{r} \psi
%%%=
%%%-\frac{\Hbar^2}{2m} \inv{a} \lr{ \inv{a} -\frac{2}{r} } \psi
%%%- e^2 \inv{4 \pi \epsilon_0 } \inv{r} \psi
%%%=
%%%\lr{ -\frac{\Hbar^2}{2m} \inv{a^2} + \lr{ \frac{\Hbar^2}{m a} - e^2 \inv{4 \pi \epsilon_0 } \inv{r} } } e^{-r/a}.
%%%\end{dmath}
%%%
%%%The hydrogen ground state energy is
%%%\begin{dmath}\label{eqn:gradQuantumProblemSet7Problem2:160}
%%%E_\txtH
%%%%=
%%%%\frac{
%%%%   \bra{ \psi }
%%%%   \lr{ -\frac{\Hbar^2}{2m a} \lr{ \inv{a} -\frac{2}{ r} } - e^2 \inv{4 \pi \epsilon_0 r} }
%%%%   \ket{ \psi }
%%%%}
%%%%{ \braket{ \psi }{ \psi } }
%%%=
%%%   \lr{ \frac{\Hbar^2}{m a} - e^2 \inv{4 \pi \epsilon_0 } }
%%%\frac{
%%%\bra{ \psi } \inv{r} \ket{ \psi }
%%%}
%%%{
%%%   \braket{ \psi }{ \psi }
%%%}
%%%   - \frac{\Hbar^2}{2m a^2} .
%%%\end{dmath}
%
For Helium without electron-electron interaction the kinetic portion of the Hamiltonian action on this trial function \( \psi = e^{-(r_1 + r_2)/a} \) is
%
\begin{dmath}\label{eqn:gradQuantumProblemSet7Problem2:140}
H \psi(r_1, r_2)
=
\frac{\Bp_1^2}{2m} \psi
+
\frac{\Bp_2^2}{2m} \psi
- 2 e^2 \inv{4 \pi \epsilon_0 } \lr{ \inv{r_1} + \inv{r_2} } \psi
=
-\frac{\Hbar^2}{2m a} \lr{ \frac{2}{a} -\frac{2}{ r_1}  -\frac{2}{ r_2} } \psi
- 2 e^2 \inv{4 \pi \epsilon_0 } \lr{ \inv{r_1} + \inv{r_2} } \psi
=
\lr{ -\frac{\Hbar^2}{m a^2}
+
\lr{ \frac{\Hbar^2}{m a} - \frac{e^2}{2 \pi \epsilon_0} } \lr{ \inv{r_1} + \inv{r_2} }
}
e^{-(r_1 + r_2)/a}.
\end{dmath}
%
Now, assuming that \( \psi = e^{-(r_1 + r_2)/a} \) is the unnormalized ground state wavefunction for the Helium atom without electron-electron interaction, that ground state energy is given by
%
\begin{dmath}\label{eqn:gradQuantumProblemSet7Problem2:260}
E_{\textrm{He}}
%=
%\frac{
%   \bra{ \psi }
%   \lr{
%      -\frac{\Hbar^2}{m a} \lr{ \inv{a} -\frac{1}{ r_1}  -\frac{1}{ r_2} }
%      - 2 e^2 \inv{4 \pi \epsilon_0 } \lr{ \inv{r_1} + \inv{r_2} }
%   }
%   \ket{ \psi }
%}
%{ \braket{ \psi }{ \psi } }
=
   \lr{ \frac{\Hbar^2}{m a} - e^2 \inv{2 \pi \epsilon_0 } }
\frac{
\bra{ \psi } \inv{r_1} + \inv{r_2} \ket{ \psi }
}
{
   \braket{ \psi }{ \psi }
}
   - \frac{\Hbar^2}{m a^2} .
\end{dmath}
%
%%%\paragraph{Calculating Hydrogen ground state energy}

We'll need a couple helper integrals
%A couple helper integrals
%For the normalization factor we have
%
\begin{dmath}\label{eqn:gradQuantumProblemSet7Problem2:180}
%\begin{aligned}
%\braket{ \psi }{ \psi }
%&=
4 \pi \int_0^\infty r^2 dr e^{-2 r/a}
=
%&=
%{4 \pi}\frac{a^3}{2^3} \int_0^\infty r^2 dr e^{-r} \\
%&=
%\inv{2} { \pi}{a^3} \int_0^\infty 2r dr e^{-r} \\
%&=
%{\pi}{a^3} \int_0^\infty dr e^{-r} \\
%&=
{\pi}{a^3},
%\end{aligned}
\end{dmath}

and %for the inverse radial expectation we have
%
\begin{dmath}\label{eqn:gradQuantumProblemSet7Problem2:200}
%\bra{ \psi } \inv{r} \ket{ \psi }
%=
4 \pi \int_0^\infty r dr e^{-2 r/a}
%=
%{4 \pi}\frac{a^2}{2^2} \int_0^\infty r dr e^{-r}
=
{\pi}{a^2}.
\end{dmath}
%
%so
%
%%%\begin{dmath}\label{eqn:gradQuantumProblemSet7Problem2:220}
%%%E_\txtH
%%%=
%%%\lr{ \frac{\Hbar^2}{m a} - e^2 \inv{4 \pi \epsilon_0 } } \frac{\pi a^2}{\pi a^3}
%%%   - \frac{\Hbar^2}{2m a^2} ,
%%%=
%%%\frac{\Hbar^2}{2 m a^2} - \frac{e^2}{4 \pi \epsilon_0 a }.
%%%\end{dmath}
%%%
%%%A test minimization of this energy using \( a \) as a variational parameter finds
%%%
%%%\begin{dmath}\label{eqn:gradQuantumProblemSet7Problem2:620}
%%%a = \frac{4 \pi \epsilon_0 \Hbar^2}{m e^2},
%%%\end{dmath}
%%%
%%%which is the Bohr-radius as expected.  Substituting that gives
%%%
%%%%\begin{dmath}\label{eqn:gradQuantumProblemSet7Problem2:240}
%%%\boxedEquation{eqn:gradQuantumProblemSet7Problem2:240}{
%%%E_\txtH
%%%=
%%%-
%%%\frac{ m e^4 }{ 32 \pi^2 \epsilon_0^2 \Hbar^2 }
%%%=
%%%-
%%%\inv{2} \frac{e^2}{4 \pi \epsilon_0 a_0}.
%%%}
%%%%\end{dmath}
%%%
%%%Numerically this is about \( -13.6 \si{eV} \).
%
%\paragraph{Calculating Helium ground state energy}
%
To normalize the wavefunction, we need a six-fold integral over both the spatial domains.  With only radial dependence that is
%
\begin{dmath}\label{eqn:gradQuantumProblemSet7Problem2:280}
%\begin{aligned}
\braket{ \psi }{ \psi }
=
\lr{ 4 \pi}^2
\int_0^\infty r_1^2 dr_1 e^{-2 r_1/a}
\int_0^\infty r_2^2 dr_2 e^{-2 r_2/a}
%&=
%\lr{ 4 \pi}^2 \lr{ \frac{a}{2} }^6
%\lr{ \int_0^\infty r^2 dr e^{-r} }^2 \\
%&=
%2^{4 - 6 + 2}
%\pi^2 a^6 \\
= \pi^2 a^6.
%\end{aligned}
\end{dmath}
%
We also need the inverse radial expectations.  Calculating the expectation of \( 1/r_1 \) is sufficient, and is
%
\begin{dmath}\label{eqn:gradQuantumProblemSet7Problem2:300}
\bra{ \psi } \inv{r_1} \ket{ \psi }
=
\lr{ 4 \pi}^2
\int_0^\infty r_1 dr_1 e^{-2 r_1/a}
\int_0^\infty r_2^2 dr_2 e^{-2 r_2/a}
%=
%\lr{ 4 \pi}^2 \lr{ \frac{a}{2} }^5
%\lr{ \int_0^\infty r dr e^{-r} }
%\lr{ \int_0^\infty r^2 dr e^{-r} }
%=
%2^{4 - 5 + 1}
%\pi^2 a^5
= \pi^2 a^5.
\end{dmath}
%
So, without the electron-electron interaction, the ground state energy is
%
\begin{dmath}\label{eqn:gradQuantumProblemSet7Problem2:320}
E_{\textrm{He}}
=
   \lr{ \frac{\Hbar^2}{m a} - e^2 \inv{2 \pi \epsilon_0 } }
\frac{ 2 \pi^2 a^5 }
{
   \pi^2 a^6
}
   - \frac{\Hbar^2}{m a^2}
=
   \frac{\Hbar^2}{m a^2} - e^2 \inv{\pi \epsilon_0 a }.
\end{dmath}
%
%%%It appears that the value of \( a \) that minimizes this energy is not the bohr radius for this wave function, so the assumption that \( \psi(r_1, r_2) = e^{-(r_1 + r_2)/a_0} \) was an eigenfunction for the Hamiltonian was incorrect.  Performing the variation, we find that the minimum energy is found at
%%%
%%%\begin{dmath}\label{eqn:gradQuantumProblemSet7Problem2:640}
%%%a = \inv{2} \frac{4 \pi \epsilon_0 \Hbar^2}{m e^2},
%%%\end{dmath}
%%%
%%%which is half the Bohr-radius.  Substituting that into the energy above gives
%%%
%%%\boxedEquation{eqn:gradQuantumProblemSet7Problem2:400}{
%%%E_{\textrm{He}}
%%%=
%%%-\frac{e^2 m}{4 \pi^2 \epsilon_0^2 \Hbar^2}
%%%=
%%%-
%%%\frac{e^2}{\pi \epsilon_0 a_0}.
%%%}
%%%
%This is \( E_{\textrm{He}} = 8 E_\txtH\), a value of about \( -110 eV \).

To evaluate the interaction term, a Fourier transform representation of that inverse radial distance can be employed
%
\begin{dmath}\label{eqn:gradQuantumProblemSet7Problem2:900}
\inv{\Abs{\Br}}
= \inv{2 \pi^2} \int d^3 k \frac{e^{i \Bk \cdot \Br}}{\Bk^2},
\end{dmath}
%
where this is understood to be the \( \epsilon \rightarrow 0 \) limit of
%
\begin{dmath}\label{eqn:gradQuantumProblemSet7Problem2:420}
\inv{\Abs{\Br}} e^{-\epsilon \Abs{\Br}}
= \inv{2 \pi^2} \int d^3 k \frac{e^{i \Bk \cdot \Br}}{\Bk^2 + \epsilon^2}.
\end{dmath}
%
See \citep{byron1992mca} for a demonstration of this identity, and the contour used to evaluate the RHS of \cref{eqn:gradQuantumProblemSet7Problem2:420}.  Employing this inverse radial representation, the \( r_1 \) and \( r_2 \) contributions to the interaction can be decoupled
%
\begin{dmath}\label{eqn:gradQuantumProblemSet7Problem2:440}
\frac{e^2}{4 \pi \epsilon_0} \bra{\psi} \inv{\Abs{\Br_1 - \Br_2}} \ket{\psi}
=
\frac{e^2}{4 \pi \epsilon_0}
\int 2 \pi dr_1 d\theta_1 r_1^2 \sin(\theta_1)
\int 2 \pi dr_2 d\theta_2 r_2^2 \sin(\theta_2)
e^{ -2(r_1 + r_2)/a}
\inv{2 \pi^2}
\int d^3 k \frac{e^{i \Bk \cdot \lr{ \Br_1 - \Br_2} }}{\Bk^2}
=
\frac{e^2}{4 \pi \epsilon_0}
\inv{2 \pi^2}
\int d^3 k \inv{\Bk^2}
\int 2 \pi dr_1 d\theta_1 r_1^2 \sin(\theta_1) e^{-2r_1/a + i \Bk \cdot \Br_1}
\int 2 \pi dr_2 d\theta_2 r_2^2 \sin(\theta_2) e^{-2r_2/a - i \Bk \cdot \Br_2}.
\end{dmath}
%
The spatial domain integrals can now be evaluated separately.  With a coordinate system picked so that \( \Bk = \pm k \zcap \), that gives
%
\begin{dmath}\label{eqn:gradQuantumProblemSet7Problem2:460}
\begin{aligned}
2 \pi \int dr d\theta r^2 \sin(\theta) e^{-2r/a + i \Bk \cdot \Br}
&=
2 \pi \int_0^\infty dr r^2 e^{-2r/a}
\int_0^\pi d\theta
\frac{d}{d\theta} (-\cos(\theta))
e^{\pm i k r \cos\theta} \\
&=
2 \pi \int_0^\infty dr r^2 e^{-2r/a}
\int_{-1}^1 du
e^{\mp i k r u} \\
&=
2 \pi \int_0^\infty dr r^2 e^{-2r/a}
\frac{e^{\mp i k r} - e^{\pm i k r} }{ \mp i k r } \\
&=
2 \pi \frac{2}{k} \int_0^\infty dr r e^{-2r/a} \sin( k r ) \\
&=
2 \pi \frac{2}{k} \int_0^\infty dr r e^{-2r/a} \sin( k r ) \\
&=
\frac{16 \pi a^3}{\lr{1 + a^2 k^2 }^2}.
\end{aligned}
\end{dmath}
%
We see that the specific orientation used to evaluate the integral does not matter, so we have
%
\begin{dmath}\label{eqn:gradQuantumProblemSet7Problem2:480}
\frac{e^2}{4 \pi \epsilon_0} \bra{\psi} \inv{\Abs{\Br_1 - \Br_2}} \ket{\psi}
=
\frac{e^2}{4 \pi \epsilon_0}
\inv{2 \pi^2}
\int d^3 k \inv{\Bk^2}
\frac{(16 \pi a^3)^2}{\lr{1 + a^2 k^2 }^4}
=
\frac{e^2}{4 \pi \epsilon_0}
\inv{2 \pi^2}
(4\pi)
16^2 \pi^2 a^6
\int dk k^2 \inv{\Bk^2}
\inv{\lr{1 + a^2 k^2 }^4}
%=
%\frac{32 e^2 a^6}{\epsilon_0}
%\int dk
%\inv{\lr{1 + a^2 k^2 }^4}
%=
%\frac{32 e^2 a^6}{\epsilon_0}
%\frac{5 \pi}{32 a}
=
\frac{5 \pi e^2 a^5}{32 \epsilon_0}.
\end{dmath}
%
Rescaling with the normalization factor gives
%
\begin{dmath}\label{eqn:gradQuantumProblemSet7Problem2:500}
\frac{e^2}{4 \pi \epsilon_0} \bra{\psi} \inv{\Abs{\Br_1 - \Br_2}} \ket{\psi}/\braket{\psi}{\psi}
=
\frac{5 \pi e^2 a^5}{32 \epsilon_0 } \inv{\pi^2 a^6}
=
\frac{5 e^2 }{32 \pi \epsilon_0 a}.
\end{dmath}
%
Adding this electron-electron interaction to the Helium ground energy calculated in
\cref{eqn:gradQuantumProblemSet7Problem2:320} gives
%
\begin{dmath}\label{eqn:gradQuantumProblemSet7Problem2:520}
E_{\textrm{He}}
=
\frac{\Hbar^2}{m a^2} -\frac{27}{32} e^2 \inv{\pi \epsilon_0 a }.
\end{dmath}
%
For the minimum we want to solve
%
\begin{dmath}\label{eqn:gradQuantumProblemSet7Problem2:540}
0
=
\PD{a}{E}
=
-2 \frac{\Hbar^2}{m a^3} + \frac{27}{32} e^2 \inv{\pi \epsilon_0 a^2 },
\end{dmath}
%
which has the minimum at
%
\begin{dmath}\label{eqn:gradQuantumProblemSet7Problem2:560}
a = - \frac{64 \Hbar^2 \pi \epsilon_0}{27 m e^2}.
\end{dmath}
%
Note that \( m \) should really be treated as the reduced mass of the electron, but doing so isn't numerically significant.  The final result for the variational ground state energy is

%Noting that \( a_0 = \Hbar^2/m e^2 \),
%the ground state energy, after substituting this value of \( a \) is
%
%\begin{dmath}\label{eqn:gradQuantumProblemSet7Problem2:580}
\boxedEquation{eqn:gradQuantumProblemSet7Problem2:600}{
E_{\textrm{He}}
=
- \lr{\frac{27}{16}}^2 \frac{e^2}{4 \pi \epsilon_0 a_0} \approx -77.5 \si{eV}
}
%\end{dmath}

In atomic units this is
%
\begin{dmath}\label{eqn:gradQuantumProblemSet7Problem2:860}
E_{\textrm{He}}
=
- \lr{\frac{27}{16}}^2 \frac{e^2}{a_0} \approx -2.848 \frac{e^2}{a_0}.
\end{dmath}
%
In \citep{desai2009quantum} the measured value is stated as \( -2.90 \,\ifrac{e^2}{a_0} \).  Table 1 of \citep{aznabayev2015energy}, which lists high precision calculations of all the energy levels, has \( -2.903724 \,\ifrac{e^2}{a_0} \) for the 1s energy level.  The calculated value of \cref{eqn:gradQuantumProblemSet7Problem2:600} is about 2 \% off the mark.

See \nbref{ps7:heliumAtomGroundStateWithInteraction.nb}, for a complete end to end verification of the calculations above.
}
}
