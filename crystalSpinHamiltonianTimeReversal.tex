%
% Copyright � 2015 Peeter Joot.  All Rights Reserved.
% Licenced as described in the file LICENSE under the root directory of this GIT repository.
%
%{
%\input{../blogpost.tex}
%\renewcommand{\basename}{crystalSpinHamiltonianTimeReversal}
%\renewcommand{\dirname}{notes/phy1520/}
%%\newcommand{\dateintitle}{}
%%\newcommand{\keywords}{}
%
%\input{../peeter_prologue_print2.tex}
%
%\usepackage{peeters_layout_exercise}
%\usepackage{peeters_braket}
%\usepackage{peeters_figures}
%
%\beginArtNoToc
%
%\generatetitle{Time reversal behavior of solutions to crystal spin Hamiltonian}
%\chapter{Time reversal behaviour of solutions to crystal spin Hamiltonian}
%\label{chap:crystalSpinHamiltonianTimeReversal}

\makeoproblem{Time reversal behavior of solutions to crystal spin Hamiltonian.}{problem:crystalSpinHamiltonianTimeReversal:1}{\citep{sakurai2014modern} pr. 4.12}{
\index{crystal spin}
\index{time reversal}

Solve the spin 1 Hamiltonian
\begin{dmath}\label{eqn:crystalSpinHamiltonianTimeReversal:20}
H = A S_z^2 + B(S_x^2 - S_y^2).
\end{dmath}
%
Is this Hamiltonian invariant under time reversal?

How do the eigenkets change under time reversal?

} % problem

\makeanswer{problem:crystalSpinHamiltonianTimeReversal:1}{

In spinMatrices.nb the matrix representation of the Hamiltonian is found to be
\begin{dmath}\label{eqn:crystalSpinHamiltonianTimeReversal:40}
H =
\Hbar^2
\begin{bmatrix}
 A+\frac{B}{2} & 0 & \frac{B}{2} \\
 -\frac{i B}{\sqrt{2}} & B & -\frac{i B}{\sqrt{2}} \\
 \frac{B}{2} & 0 & A+\frac{B}{2} \\
\end{bmatrix}.
\end{dmath}
%
The eigenvalues are
\begin{dmath}\label{eqn:crystalSpinHamiltonianTimeReversal:60}
\setlr{ \Hbar^2 A, \Hbar^2 B, \Hbar^2(A + B)},
\end{dmath}
%
and the respective eigenvalues (unnormalized) are
%
\begin{dmath}\label{eqn:crystalSpinHamiltonianTimeReversal:80}
\setlr{
\begin{bmatrix}
-1 \\
0 \\
1
\end{bmatrix},
\begin{bmatrix}
0 \\
1 \\
0
\end{bmatrix},
\begin{bmatrix}
1 \\
-\frac{i \sqrt{2} B}{A} \\
1 \\
\end{bmatrix}
}.
\end{dmath}
%
Under time reversal, the Hamiltonian is
%
\begin{equation}\label{eqn:crystalSpinHamiltonianTimeReversal:100}
H \rightarrow A (-S_z)^2 + B ( (-S_x)^2 - (-S_y)^2 ) = H,
\end{equation}
%
so we expect the eigenkets for this Hamiltonian to vary by at most a phase factor.  To check this, first recall that the time reversal action on a spin one state is
%
\begin{dmath}\label{eqn:crystalSpinHamiltonianTimeReversal:120}
\Theta \ket{1, m} = (-1)^m \ket{1, -m},
\end{dmath}
%
or
%
\begin{equation}\label{eqn:crystalSpinHamiltonianTimeReversal:140}
\begin{aligned}
\Theta \ket{1} &= -\ket{-1} \\
\Theta \ket{0} &= \ket{0} \\
\Theta \ket{-1} &= -\ket{1}.
\end{aligned}
\end{equation}
%
Let's write the eigenkets respectively as
%
\begin{equation}\label{eqn:crystalSpinHamiltonianTimeReversal:160}
\begin{aligned}
\ket{A} &= -\ket{1} + \ket{-1} \\
\ket{B} &= \ket{0} \\
\ket{A+B} &= \ket{1} + \ket{-1} - \frac{i \sqrt{2} B}{A} \ket{0}.
\end{aligned}
\end{equation}
%
Noting that the time reversal operator maps complex numbers onto their conjugates, the time reversed eigenkets are
%
\begin{equation}\label{eqn:crystalSpinHamiltonianTimeReversal:180}
\begin{aligned}
\ket{A} &\rightarrow \ket{-1} - \ket{-1} = -\ket{A} \\
\ket{B} &\rightarrow \ket{0} = \ket{B} \\
\ket{A+B} &\rightarrow -\ket{1} - \ket{-1} + \frac{i \sqrt{2} B}{A} \ket{0} = -\ket{A+B}.
\end{aligned}
\end{equation}
%
Up to a sign, the time reversed states match the unreversed states.
} % answer

%}
%\EndArticle
%\EndNoBibArticle
