%
% Copyright � 2015 Peeter Joot.  All Rights Reserved.
% Licenced as described in the file LICENSE under the root directory of this GIT repository.
%
%\input{../blogpost.tex}
%\renewcommand{\basename}{braketManip}
%\renewcommand{\dirname}{notes/phy1520/}
%%\newcommand{\dateintitle}{}
%%\newcommand{\keywords}{}
%
%\input{../peeter_prologue_print2.tex}
%
%\usepackage{peeters_layout_exercise}
%\usepackage{peeters_braket}
%
%\beginArtNoToc
%
%\generatetitle{bra-ket manipulation problems}
%%\chapter{bra-ket manipulation problems}
%%\label{chap:braketManip}
\makeoproblem{Some bra-ket manipulation problems.}{problem:braketManip:1.4}{\citep{sakurai2014modern} pr. 1.4}{
\index{braket}
Using braket logic expand
%
\makesubproblem{}{problem:braketManip:1.4:a}
%\begin{equation}\label{eqn:braketManip:20}
\(
\trace{X Y},
\)
%\end{equation}
\makesubproblem{}{problem:braketManip:1.4:b}
%\begin{equation}\label{eqn:braketManip:40}
\(
(X Y)^\dagger,
\)
%\end{equation}
\makesubproblem{}{problem:braketManip:1.4:c}
%\begin{equation}\label{eqn:braketManip:60}
\(
e^{i f(A)},
\)
%\end{equation}
%
where \( A \) is Hermitian with a complete set of eigenvalues.
%
\makesubproblem{}{problem:braketManip:1.4:d}
%\begin{equation}\label{eqn:braketManip:80}
\(
\sum_{a'} \Psi_{a'}(\Bx')^\conj \Psi_{a'}(\Bx''),
\)
%\end{equation}
%
where \( \Psi_{a'}(\Bx'') = \braket{\Bx'}{a'} \).
%
} % problem
%
\makeanswer{problem:braketManip:1.4}{
%
\makeSubAnswer{}{problem:braketManip:1.4:a}
%
\begin{dmath}\label{eqn:braketManip:100}
\trace{X Y}
= \sum_a \bra{a} X Y \ket{a}
= \sum_{a,b} \bra{a} X \ket{b}\bra{b} Y \ket{a}
= \sum_{a,b}
\bra{b} Y \ket{a}
\bra{a} X \ket{b}
= \sum_{a,b}
\bra{b} Y
X \ket{b}
= \trace{ Y X }.
\end{dmath}
%
\makeSubAnswer{}{problem:braketManip:1.4:b}
%
\begin{dmath}\label{eqn:braketManip:120}
\bra{a} \lr{ X Y}^\dagger \ket{b}
=
\lr{ \bra{b} X Y \ket{a} }^\conj
=
\sum_c \lr{ \bra{b} X \ket{c}\bra{c} Y \ket{a} }^\conj
=
\sum_c \lr{ \bra{b} X \ket{c} }^\conj \lr{ \bra{c} Y \ket{a} }^\conj
=
\sum_c
\lr{ \bra{c} Y \ket{a} }^\conj
\lr{ \bra{b} X \ket{c} }^\conj
=
\sum_c
\bra{a} Y^\dagger \ket{c}
\bra{c} X^\dagger \ket{b}
=
\bra{a} Y^\dagger
X^\dagger \ket{b},
\end{dmath}
%
so \( \lr{ X Y }^\dagger = Y^\dagger X^\dagger \).
%
\makeSubAnswer{}{problem:braketManip:1.4:c}
%
Let's presume that the function \( f \) has a Taylor series representation
%
\begin{dmath}\label{eqn:braketManip:140}
f(A) = \sum_r b_r A^r.
\end{dmath}
%
If the eigenvalues of \( A \) are given by
%
\begin{dmath}\label{eqn:braketManip:160}
A \ket{a_s} = a_s \ket{a_s},
\end{dmath}
%
this operator can be expanded like
%
\begin{dmath}\label{eqn:braketManip:180}
A
= \sum_{a_s} A \ket{a_s} \bra{a_s}
= \sum_{a_s} a_s \ket{a_s} \bra{a_s},
\end{dmath}
%
To compute powers of this operator, consider first the square
%
\begin{dmath}\label{eqn:braketManip:200}
A^2 =
=
\sum_{a_s} a_s \ket{a_s} \bra{a_s}
\sum_{a_r} a_r \ket{a_r} \bra{a_r}
=
\sum_{a_s, a_r} a_s a_r \ket{a_s} \bra{a_s} \ket{a_r} \bra{a_r}
=
\sum_{a_s, a_r} a_s a_r \ket{a_s} \delta_{s r} \bra{a_r}
=
\sum_{a_s} a_s^2 \ket{a_s} \bra{a_s}.
\end{dmath}
%
The pattern for higher powers will clearly just be
%
\begin{dmath}\label{eqn:braketManip:220}
A^k =
\sum_{a_s} a_s^k \ket{a_s} \bra{a_s},
\end{dmath}
%
so the expansion of \( f(A) \) will be
%
\begin{dmath}\label{eqn:braketManip:240}
f(A)
= \sum_r b_r A^r
= \sum_r b_r
\sum_{a_s} a_s^r \ket{a_s} \bra{a_s}
=
\sum_{a_s} \lr{ \sum_r b_r a_s^r } \ket{a_s} \bra{a_s}
=
\sum_{a_s} f(a_s) \ket{a_s} \bra{a_s}.
\end{dmath}
%
The exponential expansion is
%
\begin{dmath}\label{eqn:braketManip:260}
e^{i f(A)}
=
\sum_t \frac{i^t}{t!} f^t(A)
=
\sum_t \frac{i^t}{t!}
\lr{ \sum_{a_s} f(a_s) \ket{a_s} \bra{a_s} }^t
=
\sum_t \frac{i^t}{t!}
\sum_{a_s} f^t(a_s) \ket{a_s} \bra{a_s}
=
\sum_{a_s}
e^{i f(a_s) }
\ket{a_s} \bra{a_s}.
\end{dmath}
%
\makeSubAnswer{}{problem:braketManip:1.4:d}
%
\begin{dmath}\label{eqn:braketManip:99}
\sum_{a'} \Psi_{a'}(\Bx')^\conj \Psi_{a'}(\Bx'')
=
\sum_{a'}
\braket{\Bx'}{a'}^\conj
\braket{\Bx''}{a'}
=
\sum_{a'}
\braket{a'}{\Bx'}
\braket{\Bx''}{a'}
=
\sum_{a'}
\braket{\Bx''}{a'}
\braket{a'}{\Bx'}
=
\braket{\Bx''}{\Bx'}
= \delta\lr{\Bx'' - \Bx'}.
\end{dmath}
%
} % answer

%\EndArticle
