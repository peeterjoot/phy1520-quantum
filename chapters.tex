%
% Copyright � 2015 Peeter Joot.  All Rights Reserved.
% Licenced as described in the file LICENSE under the root directory of this GIT repository.
%
%----------------------------------------------------------------------------------------
%\part{Reading and Lecture Notes}
   \mychapter{Fundamental concepts.}
      %\section{Lighting review}
         %
% Copyright � 2015 Peeter Joot.  All Rights Reserved.
% Licenced as described in the file LICENSE under the root directory of this GIT repository.
%
%\input{../blogpost.tex}
%\renewcommand{\basename}{gmLecture1}
%\renewcommand{\dirname}{notes/phy1520/}
%%\newcommand{\dateintitle}{}
%%\newcommand{\keywords}{}
%
%\input{../peeter_prologue_print2.tex}
%
%\usepackage{peeters_layout_exercise}
%\usepackage{peeters_braket}
%\usepackage{peeters_figures}
%
%\beginArtNoToc
%
%\generatetitle{PHY1520H Graduate Quantum Mechanics.  Lecture 1: Lighting review.  Taught by Prof.\ Arun Paramekanti}
%\chapter{Lighting review}
%\label{chap:gmLecture1}

%\paragraph{Disclaimer}
%
%Peeter's lecture notes from class.  These may be incoherent and rough.
%
%These are notes for the UofT course PHY1520, Graduate Quantum Mechanics, taught by Prof. Paramekanti, covering \textchapref{{1}} \citep{sakurai2014modern} content.
%%
%
%Text \citep{sakurai2014modern} (revised edition).
%
\section{Classical mechanics.}

We'll be talking about one body physics for most of this course.  In classical mechanics we can figure out the particle trajectories using both of \( (\Br, \Bp \), where
%
\begin{dmath}\label{eqn:qmLecture1:20}
\begin{aligned}
\ddt{\Br} &= \inv{m} \Bp, \\
\ddt{\Bp} &= \spacegrad V.
\end{aligned}
\end{dmath}
A two dimensional phase space as sketched in \cref{fig:lectureOnePhaseSpaceClassical:lectureOnePhaseSpaceClassicalFig1} shows the trajectory of a point particle subject to some equations of motion

\imageFigure{../figures/phy1520-quantum/lectureOnePhaseSpaceClassicalFig1}{One dimensional classical phase space example.}{fig:lectureOnePhaseSpaceClassical:lectureOnePhaseSpaceClassicalFig1}{0.3}

\section{Quantum mechanics.}

For this lecture, we'll work with natural units, setting
%
%\begin{dmath}\label{eqn:qmLecture1:480}
\boxedEquation{eqn:qmLecture1:480}{
\Hbar = 1.
}
%\end{dmath}
%
\index{state vector}
In QM we are no longer allowed to think of position and momentum, but have to start asking about state vectors \( \ket{\Psi} \).

\index{basis}
\index{braket}
We'll consider the state vector with respect to some basis, for example, in a position basis, we write
%
\begin{equation}\label{eqn:qmLecture1:40}
\braket{ x }{\Psi } = \Psi(x),
\end{equation}
%
\index{wave function}
\index{probability!amplitude}
a complex numbered ``wave function'', the probability amplitude for a particle in \( \ket{\Psi} \) to be in the vicinity of \( x \).

We could also consider the state in a momentum basis
\index{momentum!basis}
%
\begin{equation}\label{eqn:qmLecture1:60}
\braket{ p }{\Psi } = \Psi(p),
\end{equation}
%
a probability amplitude with respect to momentum \( p \).
More precisely,
%
\begin{equation}\label{eqn:qmLecture1:80}
\Abs{\Psi(x)}^2 dx \ge 0,
\end{equation}
is the probability of finding the particle in the range \( (x, x + dx ) \).  To have meaning as a probability, we require
\index{normalization}
\begin{equation}\label{eqn:qmLecture1:100}
\int_{-\infty}^\infty \Abs{\Psi(x)}^2 dx = 1.
\end{equation}
%
\index{probability!density function}
The average position can be calculated using this probability density function.  For example
%
\begin{equation}\label{eqn:qmLecture1:120}
\expectation{x} = \int_{-\infty}^\infty \Abs{\Psi(x)}^2 x dx,
\end{equation}
%
or
\begin{equation}\label{eqn:qmLecture1:140}
\expectation{f(x)} = \int_{-\infty}^\infty \Abs{\Psi(x)}^2 f(x) dx.
\end{equation}
%
Similarly, calculation of an average of a function of momentum can be expressed as

\index{expectation}
\begin{equation}\label{eqn:qmLecture1:160}
\expectation{f(p)} = \int_{-\infty}^\infty \Abs{\Psi(p)}^2 f(p) dp.
\end{equation}
%
\section{Transformation from a position to momentum basis.}
\index{momentum basis}
We have a problem, if we which to compute an average in momentum space such as \( \expectation{p} \), when given a wavefunction \( \Psi(x) \).

How do we convert
%
\begin{dmath}\label{eqn:qmLecture1:180}
\Psi(p)
\overset{?}{\leftrightarrow}
\Psi(x),
\end{dmath}
%
or equivalently
\begin{dmath}\label{eqn:qmLecture1:200}
\braket{p}{\Psi}
\overset{?}{\leftrightarrow}
\braket{x}{\Psi}.
\end{dmath}
%
Such a conversion can be performed by virtue of an the assumption that we have a complete orthonormal basis, for which we can introduce identity operations such as
%
\begin{equation}\label{eqn:qmLecture1:220}
\int_{-\infty}^\infty dp \ket{p}\bra{p} = 1,
\end{equation}
%
or
\begin{equation}\label{eqn:qmLecture1:240}
\int_{-\infty}^\infty dx \ket{x}\bra{x} = 1
\end{equation}

Some interpretations:

\begin{enumerate}
\item \( \ket{x_0} \leftrightarrow \text{sits at} x = x_0 \)
\item \( \braket{x}{x'} \leftrightarrow \delta(x - x') \)
\item \( \braket{p}{p'} \leftrightarrow \delta(p - p') \)
\item \( \braket{x}{p'} = \frac{e^{i p x}}{\sqrt{V}} \), where \( V \) is the volume of the box containing the particle.  We'll define the appropriate normalization for an infinite box volume later.
\end{enumerate}

The delta function interpretation of the braket \( \braket{p}{p'} \) justifies the identity operator, since we recover any state in the basis when operating with it.  For example, in momentum space
%
\begin{dmath}\label{eqn:qmLecture1:260}
1 \ket{p}
=
\lr{ \int_{-\infty}^\infty dp'
\ket{p'}\bra{p'} }
\ket{p}
=
\int_{-\infty}^\infty dp'
\ket{p'}
\braket{p'}{p}
=
\int_{-\infty}^\infty dp'
\ket{p'}
\delta(p - p')
=
\ket{p}.
\end{dmath}
%
This also the determination of an integral operator representation for the delta function
%
\begin{dmath}\label{eqn:qmLecture1:500}
\delta(x - x')
=
\braket{x}{x'}
=
\int dp \braket{x}{p} \braket{p}{x'}
=
\inv{V} \int dp e^{i p x} e^{-i p x'},
\end{dmath}
%
or
%
\begin{dmath}\label{eqn:qmLecture1:520}
\delta(x - x')
=
\inv{V} \int dp e^{i p (x- x')}.
\end{dmath}
%
Here we used the fact that \( \braket{p}{x} = \braket{x}{p}^\conj \).

FIXME: do we have a justification for that conjugation with what was defined here so far?

The conversion from a position basis to momentum space is now possible
%
\begin{dmath}\label{eqn:qmLecture1:280}
\braket{p}{\Psi}
=
\Psi(p)
= \int_{-\infty}^\infty \braket{p}{x} \braket{x}{\Psi} dx
= \int_{-\infty}^\infty \frac{e^{-ip x}}{\sqrt{V}} \Psi(x) dx.
\end{dmath}
%
The momentum space to position space conversion can be written as
%
\begin{dmath}\label{eqn:qmLecture1:300}
\Psi(x)
= \int_{-\infty}^\infty \frac{e^{ip x}}{\sqrt{V}} \Psi(p) dp.
\end{dmath}
%
Now we can go back and figure out the an expectation
%
\begin{dmath}\label{eqn:qmLecture1:320}
\expectation{p}
=
\int \Psi^\conj(p) \Psi(p) p d p
=
\int dp
\lr{
\int_{-\infty}^\infty \frac{e^{ip x}}{\sqrt{V}} \Psi^\conj(x) dx
}
\lr{
\int_{-\infty}^\infty \frac{e^{-ip x'}}{\sqrt{V}} \Psi(x') dx'
}
p
=\int dp dx dx'
\Psi^\conj(x)
\inv{V} e^{ip (x-x')} \Psi(x') p
=
\int dp dx dx'
\Psi^\conj(x)
\inv{V} \lr{ -i\PD{x}{e^{ip (x-x')}} }\Psi(x')
=
\int dp dx
\Psi^\conj(x) \lr{ -i \PD{x}{} }
\inv{V} \int dx' e^{ip (x-x')} \Psi(x')
=
\int dx
\Psi^\conj(x) \lr{ -i \PD{x}{} }
\int dx' \lr{ \inv{V} \int dp e^{ip (x-x')} } \Psi(x')
=
\int dx
\Psi^\conj(x) \lr{ -i \PD{x}{} }
\int dx' \delta(x - x') \Psi(x')
=
\int dx
\Psi^\conj(x) \lr{ -i \PD{x}{} }
\Psi(x).
\end{dmath}
%FIXME : Performing this integral we find
%
%\begin{equation}\label{eqn:qmLecture1:340}
%p e^{i p x} \leftrightarrow -i \PD{x}{} e^{i p x},
%\end{equation}
%
%so
%
%\begin{dmath}\label{eqn:qmLecture1:360}
%\expectation{p}
%=
%\int dp dx dx'
%\lr{ \frac{e^{ip x}}{\sqrt{V}} \Psi^\conj(x)  }
%\lr{ i \frac{\Psi(x')}{\sqrt{V}} \PD{x'}{} e^{-ip x'} }
%= \int \frac{dx dx'}{V} \Psi^\conj(x) \Psi(x) i \PD{x'}{} \lr{ \int dp e^{i p x - i p x'} }
%= \int \frac{dx dx'}{V} \Psi^\conj(x) \Psi(x) i \PD{x'} V \delta(x - x')
%= \int dx dx' \Psi^\conj(x) \Psi(x) i \PD{x'}{} \delta(x - x')
%= \int dx dx' \Psi^\conj(x) \lr{ -i \PD{x'}{\Psi(x')}} \delta(x - x')
%= \int dx \Psi^\conj(x) \lr{ -i \PD{x}{} } \Psi(x).
%\end{dmath}
%
Here we've essentially calculated the position space representation of the momentum operator, allowing identifications of the following form
\index{momentum!operator}
%
\begin{equation}\label{eqn:qmLecture1:380}
p \leftrightarrow -i \PD{x}{}
\end{equation}
\begin{equation}\label{eqn:qmLecture1:400}
p^2 \leftrightarrow - \PDSq{x}{}.
\end{equation}
%
\paragraph{Alternate starting point.}

Most of the above results followed from the claim that \( \braket{x}{p} = e^{i p x} \).  Note that this position space representation of the momentum operator can also be taken as the starting point.  Given that, the exponential representation of the position-momentum braket follows
%
\begin{dmath}\label{eqn:qmLecture1:540}
\bra{x} P \ket{p}
=
-i \Hbar \PD{x}{} \braket{x}{p},
\end{dmath}
%
but \( \bra{x} P \ket{p} = p \braket{x}{p} \), providing a differential equation for \( \braket{x}{p} \)
%
\begin{equation}\label{eqn:qmLecture1:560}
p \braket{x}{p} = -i \Hbar \PD{x}{} \braket{x}{p},
\end{equation}
%
with solution
%
\begin{equation}\label{eqn:qmLecture1:580}
i p x/\Hbar = \ln \braket{x}{p} + \text{const},
\end{equation}
%
or
\begin{equation}\label{eqn:qmLecture1:600}
\braket{x}{p} \propto e^{i p x/\Hbar}.
\end{equation}
%
\section{Matrix interpretation.}

\begin{enumerate}
\item Ket's \( \ket{\Psi} \leftrightarrow \text{column vector} \)
\item Bra's \( \bra{\Psi} \leftrightarrow {(\text{row vector})}^\conj \)
\item Operators \( \leftrightarrow \) matrices that act on vectors.
\end{enumerate}
%
\begin{equation}\label{eqn:qmLecture1:420}
\hatp \ket{\Psi} \rightarrow \ket{\Psi'}.
\end{equation}
\section{Time evolution.}
\index{time evolution}
\index{Hamiltonian}
For a state subject to the equations of motion given by the Hamiltonian operator \( \hatH \)
%
\begin{equation}\label{eqn:qmLecture1:440}
i \PD{t}{} \ket{\Psi} = \hatH \ket{\Psi},
\end{equation}
%
the time evolution is given by
\begin{equation}\label{eqn:qmLecture1:460}
\ket{\Psi(t)} = e^{-i \hatH t} \ket{\Psi(0)}.
\end{equation}
%
%\section{Density matrix and incomplete information}
%

%\EndArticle

      %\section{Basic concepts, time evolution, and density operators}
         %
% Copyright � 2015 Peeter Joot.  All Rights Reserved.
% Licenced as described in the file LICENSE under the root directory of this GIT repository.
%
%\input{../blogpost.tex}
%\renewcommand{\basename}{lecture2}
%\renewcommand{\dirname}{notes/phy1520/}
%%\newcommand{\dateintitle}{}
%%\newcommand{\keywords}{}
%
%\input{../peeter_prologue_print2.tex}
%
%\usepackage{peeters_layout_exercise}
%\usepackage{peeters_braket}
%\usepackage{peeters_figures}
%\usepackage{enumerate}
%
%\beginArtNoToc
%
%\generatetitle{PHY1520H Graduate Quantum Mechanics.  Lecture 2: Basic concepts, time evolution, and density operators.  Taught by Prof.\ Arun Paramekanti}
%%\label{chap:gmLecture1}
%
%\paragraph{Disclaimer}
%
%Peeter's lecture notes from class.  These may be incoherent and rough.
%
%These are notes for the UofT course PHY1520, Graduate Quantum Mechanics, taught by Prof. Paramekanti, covering \textchapref{{1}} \citep{sakurai2014modern} content.
%%

\section{Review: Basic concepts.}

We've reviewed the basic concepts that we will encounter in Quantum Mechanics.

\begin{enumerate}
\item Abstract state vector.  \( \ket{ \psi} \)
\item Basis states.  \( \ket{ x } \)
\item Observables, special Hermitian operators.  We'll only deal with linear observables.
\item Measurement.
\end{enumerate}

We can either express the wave functions \( \psi(x) = \braket{x}{\psi} \) in terms of a basis for the observable, or can express the observable in terms of the basis of the wave function (position or momentum for example).

We saw that the position space representation of a momentum operator (also an observable) was
%
\begin{dmath}\label{eqn:qmLecture2:20}
\hatp \rightarrow -i \Hbar \PD{x}{}.
\end{dmath}
%
In general we can find the matrix element representation of any operator by considering its representation in a given basis.  For example, in a position basis, that would be
%
\begin{dmath}\label{eqn:qmLecture2:40}
\bra{x'} \hatA \ket{x} \leftrightarrow A_{x x'}
\end{dmath}

The Hermitian property of the observable means that \( A_{x x'} = A_{x' x}^\conj \)
%
\begin{dmath}\label{eqn:qmLecture2:60}
\int dx \bra{x'} \hatA \ket{x} \braket{x }{\psi} = \braket{x'}{\phi}
\leftrightarrow
A_{x' x} \psi_x = \phi_{x'}.
\end{dmath}
%
\index{measurement}
\makeexample{Measurement example}{example:qmLecture2:1}{

Consider a polarization apparatus as sketched in \cref{fig:polarizerMeasurement:polarizerMeasurementFig1}, where the output is of the form \( I_{\textrm{out}} = I_{\textrm{in}} \cos^2 \theta \).

\imageFigure{../figures/phy1520-quantum/polarizerMeasurementFig1}{Polarizer apparatus.}{fig:polarizerMeasurement:polarizerMeasurementFig1}{0.3}

A general input state can be written in terms of each of the possible polarizations
%
\begin{dmath}\label{eqn:qmLecture2:80}
\alpha \ket{ \updownarrow } + \beta \ket{ \leftrightarrow } \sim
\cos\theta \ket{ \updownarrow } + \sin\theta \ket{ \leftrightarrow }.
\end{dmath}
Here \( \abs{\alpha}^2 \) is the probability that the input state is in the upwards polarization state, and \( \abs{\beta}^2 \) is the probability that the input state is in the downwards polarization state.

The measurement of the polarization results in an output state that has a specific polarization.  That measurement is said to collapse the wavefunction.
} % example

When attempting a measurement, looking for a specific value, effects the state of the system, and is call a strong or projective measurement.  Such a measurement is
\begin{enumerate}[(i)]
\item Probabilistic.
\item Requires many measurements.
\end{enumerate}
This measurement process results a determination of the eigenvalue of the operator.  The eigenvalue production of measurement is why we demand that operators be Hermitian.

It is also possible to try to do a weaker (perturbative) measurement, where some information is extracted from the input state without completely altering it.
\paragraph{Time evolution}
\begin{enumerate}
\item Schr\"{o}dinger picture.
\index{Schr\"{o}dinger picture}
\index{time evolution}
The time evolution process is governed by a Schr\"{o}dinger equation of the following form
%
\begin{equation}\label{eqn:qmLecture2:100}
i \Hbar \PD{t}{} \ket{\Psi(t)} = \hatH \ket{\Psi(t)}.
\end{equation}
%
This Hamiltonian could be, for example,
%
\begin{equation}\label{eqn:qmLecture2:120}
\hatH = \frac{\hatp^2}{2m} + V(x),
\end{equation}
%
Such a representation of time evolution is expressed in terms of operators \( \hatx, \hatp, \hatH, \cdots \) that are independent of time.
\item Heisenberg picture.
\index{Heisenberg picture}

Suppose we have a state \( \ket{\Psi(t)} \) and operate on this with an operator
%
\begin{equation}\label{eqn:qmLecture2:140}
\hatA \ket{\Psi(t)}.
\end{equation}
%
This will have time evolution of the form
%
\begin{equation}\label{eqn:qmLecture2:160}
\hatA e^{-i \hatH t/\Hbar} \ket{\Psi(0)},
\end{equation}
%
or in matrix element form
%
\begin{equation}\label{eqn:qmLecture2:180}
\bra{\phi(t)} \hatA \ket{\Psi(t)}
=
\bra{\phi(0)}
e^{i \hatH t/\Hbar}
\hatA e^{-i \hatH t/\Hbar} \ket{\Psi(0)}.
\end{equation}
%
We work with states that do not evolve in time \( \ket{\phi(0)}, \ket{\Psi(0)}, \cdots \), but operators do evolve in time according to
%
\begin{equation}\label{eqn:qmLecture2:200}
\hatA(t) =
e^{i \hatH t/\Hbar}
\hatA e^{-i \hatH t/\Hbar}.
\end{equation}
%
\end{enumerate}

\paragraph{Density operator}
\index{density operator}

We can have situations where it is impossible to determine a single state that describes the system.  For example, given the gas in the room that you are sitting in, there are things that we can measure, but it is impossible to describe the state that describes all the particles and also impossible to construct a Hamiltonian that governs all the interactions of those many many particles.

We need a probabilistic description to even describe such a complex system, and to be able to deal with concepts like entanglement.

Suppose we have a complex system that can be partitioned into two subsets, left and right, as sketched in \cref{fig:partitions:partitionsFig2}.

\imageFigure{../figures/phy1520-quantum/partitionsFig2}{System partitioned into separate set of states.}{fig:partitions:partitionsFig2}{0.2}

If the states in each partition can be enumerated separately, we can write the state of the system as sums over the probability amplitudes that for the combined states.
%
\begin{dmath}\label{eqn:qmLecture2:220}
\ket{\Psi}
=
\sum_{m, n} C_{m,n} \ket{m} \ket{n}.
%\equiv
%\sum_{m, n} C_{m,n} \ket{m} \directproduct \ket{n}.
\end{dmath}
%
Here \( C_{m, n} \) is the probability amplitude to find the state in the combined state \( \ket{m} \ket{n} \).

As an example of such a system, we could investigate a two particle configuration where spin up or spin down can be separately measured for each particle.
%
\begin{dmath}\label{eqn:qmLecture2:240}
\ket{\psi} = \inv{\sqrt{2}} \lr{
\ket{\uparrow}\ket{\downarrow}
+
\ket{\downarrow}\ket{\uparrow}
}.
\end{dmath}

Considering such a system we could ask questions such as

\begin{itemize}
\item What is the probability that the left half is in state \( m \)?  This would be
%
\begin{equation}\label{eqn:qmLecture2:260}
\sum_n \Abs{C_{m, n}}^2.
\end{equation}
%
\item Probability that the left half is in state \( m \), and the
probability that the right half is in state \( n \)?  That is
%
\begin{equation}\label{eqn:qmLecture2:280}
\Abs{C_{m, n}}^2.
\end{equation}
\end{itemize}

We define the density operator
%
\begin{dmath}\label{eqn:qmLecture2:300}
\hat\rho = \ket{\Psi} \bra{\Psi}.
\end{dmath}
%
This is \textAndIndex{idempotent}
%
\begin{dmath}\label{eqn:qmLecture2:320}
\hat\rho^2 =
\lr{ \ket{\Psi} \bra{\Psi} }
\lr{ \ket{\Psi} \bra{\Psi} }
=
\ket{\Psi} \bra{\Psi}.
\end{dmath}
%\EndArticle

      %\section{Density matrix (cont.)}
         %
% Copyright � 2015 Peeter Joot.  All Rights Reserved.
% Licenced as described in the file LICENSE under the root directory of this GIT repository.
%
%\input{../blogpost.tex}
%\renewcommand{\basename}{chapter3Notes}
%\renewcommand{\dirname}{notes/phy1520/}
%\newcommand{\keywords}{PHY1520H}
%\input{../peeter_prologue_print2.tex}
%
%%\usepackage{phy1520}
%\usepackage{peeters_braket}
%\usepackage{peeters_layout_exercise}
%\usepackage{peeters_figures}
%\usepackage{mathtools}
%
%\beginArtNoToc
%\generatetitle{PHY1520H Graduate Quantum Mechanics.  Lecture 3: Density matrix (cont.).  Taught by Prof.\ Arun Paramekanti}
%%\generatetitle{Density matrix (cont.)}
%%\chapter{Density matrix (cont.)}
%\label{chap:chapter3Notes}
%
%\paragraph{Disclaimer}
%
%Peeter's lecture notes from class.  These may be incoherent and rough.
%
%These are notes for the UofT course PHY1520, Graduate Quantum Mechanics, taught by Prof. Paramekanti, covering \textchapref{{1}} \citep{sakurai2014modern} content.
%
%\paragraph{Density matrix (cont.)}

An example of a partitioned system with four total states (two spin 1/2 particles) is sketched in \cref{fig:twoSpins:twoSpinsFig1}.

\imageFigure{../figures/phy1520-quantum/twoSpinsFig1}{Two spins.}{fig:twoSpins:twoSpinsFig1}{0.2}

An example of a partitioned system with eight total states (three spin 1/2 particles) is sketched in \cref{fig:threeSpins:threeSpinsFig2}.

\imageFigure{../figures/phy1520-quantum/threeSpinsFig2}{Three spins.}{fig:threeSpins:threeSpinsFig2}{0.3}

The density matrix
%
\begin{dmath}\label{eqn:qmLecture3:20}
\hat\rho = \ket{\Psi}\bra{\Psi}
\end{dmath}

is clearly an operator as can be seen by applying it to a state
%
\begin{dmath}\label{eqn:qmLecture3:40}
\hat\rho \ket{\phi} = \ket{\Psi} \lr{ \braket{ \Psi }{\phi} }.
\end{dmath}
%
The quantity in braces is just a complex number.

After expanding the pure state \( \ket{\Psi} \) in terms of basis states for each of the two partitions
%
\begin{dmath}\label{eqn:qmLecture3:60}
\ket{\Psi}
= \sum_{m,n} C_{m, n} \ket{m}_\txtL \ket{n}_\txtR,
\end{dmath}
%
With \( \txtL \) and \( \txtR \) implied for \( \ket{m}, \ket{n} \) indexed states respectively, this can be written
%
\begin{dmath}\label{eqn:qmLecture3:460}
\ket{\Psi}
= \sum_{m,n} C_{m, n} \ket{m} \ket{n}.
\end{dmath}
%
The density operator is
%
\begin{dmath}\label{eqn:qmLecture3:80}
\hat\rho =
\sum_{m,n}
C_{m, n}
C_{m', n'}^\conj
\ket{m} \ket{n}
\sum_{m',n'}
\bra{m'} \bra{n'}.
\end{dmath}
%
Suppose we trace over the right partition of the state space, defining such a trace as the reduced density operator \( \hat\rho_{\textrm{red}} \)
%
\begin{dmath}\label{eqn:qmLecture3:100}
\hat\rho_{\textrm{red}}
\equiv
\tr_\txtR(\hat\rho)
= \sum_{\tilde{n}} \bra{\tilde{n}} \hat\rho \ket{ \tilde{n}}
= \sum_{\tilde{n}}
\bra{\tilde{n} }
\lr{
\sum_{m,n}
C_{m, n}
\ket{m} \ket{n}
}
\lr{
\sum_{m',n'}
C_{m', n'}^\conj
\bra{m'} \bra{n'}
}
\ket{ \tilde{n} }
=
\sum_{\tilde{n}}
\sum_{m,n}
\sum_{m',n'}
C_{m, n}
C_{m', n'}^\conj
\ket{m} \delta_{\tilde{n} n}
\bra{m' }
\delta_{ \tilde{n} n' }
=
\sum_{\tilde{n}, m, m'}
C_{m, \tilde{n}}
C_{m', \tilde{n}}^\conj
\ket{m} \bra{m' }
\end{dmath}

Computing the matrix element of \( \hat\rho_{\textrm{red}} \), we have
%
\begin{dmath}\label{eqn:qmLecture3:120}
\bra{\tilde{m}} \hat\rho_{\textrm{red}} \ket{\tilde{m}}
=
\sum_{m, m', \tilde{n}} C_{m, \tilde{n}} C_{m', \tilde{n}}^\conj \braket{ \tilde{m}}{m} \braket{m'}{\tilde{m}}
=
\sum_{\tilde{n}} \Abs{C_{\tilde{m}, \tilde{n}} }^2.
\end{dmath}
%
This is the probability that the left partition is in state \( \tilde{m} \).

\section{Average of an observable}
\index{density operator!observable expectation}

Suppose we have two spin half particles.  For such a system the total magnetization is
%
\begin{dmath}\label{eqn:qmLecture3:140}
S_{\textrm{Total}} =
S_1^z
+
S_1^z,
\end{dmath}
%
as sketched in \cref{fig:magneticMomentTwoSpins:magneticMomentTwoSpinsFig3}.

\imageFigure{../figures/phy1520-quantum/magneticMomentTwoSpinsFig3}{Magnetic moments from two spins.}{fig:magneticMomentTwoSpins:magneticMomentTwoSpinsFig3}{0.1}

The average of some observable is
%
\begin{dmath}\label{eqn:qmLecture3:160}
\expectation{\hatA}
= \sum_{m, n, m', n'} C_{m, n}^\conj C_{m', n'}
\bra{m}\bra{n} \hatA \ket{n'} \ket{m'}.
\end{dmath}
%
\index{trace}
Consider the trace of the density operator observable product
%
\begin{dmath}\label{eqn:qmLecture3:180}
\tr( \hat\rho \hatA )
= \sum_{m, n} \braket{m n}{\Psi} \bra{\Psi} \hatA \ket{m, n}.
\end{dmath}
%
Let
%
\begin{dmath}\label{eqn:qmLecture3:200}
\ket{\Psi} = \sum_{m, n} C_{m n} \ket{m, n},
\end{dmath}
%
so that
%
\begin{dmath}\label{eqn:qmLecture3:220}
\tr( \hat\rho \hatA )
= \sum_{m, n, m', n', m'', n''} C_{m', n'} C_{m'', n''}^\conj
\braket{m n}{m', n'} \bra{m'', n''} \hatA \ket{m, n}
= \sum_{m, n, m'', n''} C_{m, n} C_{m'', n''}^\conj
\bra{m'', n''} \hatA \ket{m, n}.
\end{dmath}
%
This is just
%
%\begin{dmath}\label{eqn:qmLecture3:240}
\boxedEquation{eqn:qmLecture3:240}{
\bra{\Psi} \hatA \ket{\Psi} = \tr( \hat\rho \hatA ).
}
%\end{dmath}
%
\section{Left observables}
\index{density operator!left observable}

Consider
%
\begin{dmath}\label{eqn:qmLecture3:260}
\bra{\Psi} \hatA_\txtL \ket{\Psi}
= \tr(\hat\rho \hatA_\txtL)
=
\tr_\txtL
\tr_\txtR
(\hat\rho \hatA_\txtL)
=
\tr_\txtL
\lr{
\lr{
\tr_\txtR \hat\rho
}
\hatA_\txtL)
}
=
\tr_\txtL
\lr{
\hat\rho_{\textrm{red}}
\hatA_\txtL)
}.
\end{dmath}
%
We see
%
\begin{dmath}\label{eqn:qmLecture3:280}
\bra{\Psi} \hatA_\txtL \ket{\Psi}
=
\tr_\txtL \lr{ \hat\rho_{\textrm{red}, \txtL} \hatA_\txtL }.
\end{dmath}
%
We find that we don't need to know the state of the complete system to answer questions about portions of the system, but instead just need \( \hat\rho \), a ``probability operator'' that provides all the required information about the partitioning of the system.

\section{Pure states vs. mixed states}

For pure states we can assign a state vector and talk about reduced scenarios.  For mixed states we must work with reduced density matrices.

\index{spin half!two particles}
\makeexample{Two particle spin half pure states}{example:qmLecture3:1}{

Consider
%
\begin{dmath}\label{eqn:qmLecture3:300}
\ket{\psi_1} = \inv{\sqrt{2}} \lr{ \ket{ \uparrow \downarrow } - \ket{ \downarrow \uparrow } }
\end{dmath}
%
\begin{dmath}\label{eqn:qmLecture3:320}
\ket{\psi_2} = \inv{\sqrt{2}} \lr{ \ket{ \uparrow \downarrow } + \ket{ \uparrow \uparrow } }.
\end{dmath}
%
For the first pure state the density operator is
\begin{dmath}\label{eqn:qmLecture3:360}
\hat\rho = \inv{2}
\lr{ \ket{ \uparrow \downarrow } - \ket{ \downarrow \uparrow } }
\lr{ \bra{ \uparrow \downarrow } - \bra{ \downarrow \uparrow } }
\end{dmath}

What are the reduced density matrices?
\index{reduced density!operator}
%
\begin{dmath}\label{eqn:qmLecture3:340}
\hat\rho_\txtL
= \tr_\txtR \lr{ \hat\rho }
=
\inv{2} (-1)(-1) \ket{\downarrow}\bra{\downarrow}
+\inv{2} (+1)(+1) \ket{\uparrow}\bra{\uparrow},
\end{dmath}
%
so the matrix representation of this reduced density operator is
%
\begin{dmath}\label{eqn:qmLecture3:380}
\hat\rho_\txtL
=
\inv{2}
\begin{bmatrix}
1 & 0 \\
0 & 1
\end{bmatrix}.
\end{dmath}
%
For the second pure state the density operator is
\begin{dmath}\label{eqn:qmLecture3:400}
\hat\rho = \inv{2}
\lr{ \ket{ \uparrow \downarrow } + \ket{ \uparrow \uparrow } }
\lr{ \bra{ \uparrow \downarrow } + \bra{ \uparrow \uparrow } }.
\end{dmath}
%
This has a reduced density matrix
\begin{dmath}\label{eqn:qmLecture3:420}
\hat\rho_\txtL
= \tr_\txtR \lr{ \hat\rho }
=
\inv{2} \ket{\uparrow}\bra{\uparrow}
+\inv{2} \ket{\uparrow}\bra{\uparrow}
=
\ket{\uparrow}\bra{\uparrow}.
\end{dmath}
%
This has a matrix representation
\begin{dmath}\label{eqn:qmLecture3:440}
\hat\rho_\txtL
=
\begin{bmatrix}
1 & 0 \\
0 & 0
\end{bmatrix}.
\end{dmath}
\index{entanglement entropy}
In this second example, we have more information about the left partition.  That will be seen as a zero entanglement entropy in the problem set.  In contrast we have less information about the first state, and will find a non-zero positive entanglement entropy in that case.
} % example

%\EndArticle

      \section{Entropy when density operator has zero eigenvalues.}
         %
% Copyright � 2015 Peeter Joot.  All Rights Reserved.
% Licenced as described in the file LICENSE under the root directory of this GIT repository.
%
%\input{../blogpost.tex}
%\renewcommand{\basename}{densityMatrixEntropy}
%\renewcommand{\dirname}{notes/phy1520/}
%%\newcommand{\dateintitle}{}
%%\newcommand{\keywords}{}
%
%\input{../peeter_prologue_print2.tex}
%
%\usepackage{peeters_layout_exercise}
%\usepackage{peeters_braket}
%\usepackage{peeters_figures}
%
%\beginArtNoToc
%
%\generatetitle{Entropy when density operator has zero eigenvalues}
%%\label{chap:densityMatrixEntropy}

In the class notes and the text \citep{sakurai2014modern} the Von Neumann entropy is defined as
%
\begin{dmath}\label{eqn:densityMatrixEntropy:20}
S = -\tr(\rho \ln \rho).
\end{dmath}
%
In one of our problems I had trouble evaluating this, having calculated a density operator matrix representation
%
\begin{dmath}\label{eqn:densityMatrixEntropy:40}
\rho = E \wedge E^{-1},
\end{dmath}
%
where
%
\begin{dmath}\label{eqn:densityMatrixEntropy:60}
E = \inv{\sqrt{2}}
\begin{bmatrix}
1 & 1 \\
1 & -1
\end{bmatrix},
\end{dmath}
%
and
\begin{dmath}\label{eqn:densityMatrixEntropy:100}
\wedge =
\begin{bmatrix}
1 & 0 \\
0 & 0
\end{bmatrix}.
\end{dmath}
%
The usual method of evaluating a function of a matrix is to assume the function has a power series representation, and that a similarity transformation of the form \( A = E \wedge E^{-1} \) is possible, so that
%
\begin{dmath}\label{eqn:densityMatrixEntropy:80}
f(A) = E f(\wedge) E^{-1},
\end{dmath}
%
however, when attempting to do this with the matrix of \cref{eqn:densityMatrixEntropy:40} leads to an undesirable result
%
\begin{dmath}\label{eqn:densityMatrixEntropy:120}
\ln \rho =
\inv{2}
\begin{bmatrix}
1 & 1 \\
1 & -1
\end{bmatrix}
\begin{bmatrix}
\ln 1 & 0 \\
0 & \ln 0
\end{bmatrix}
\begin{bmatrix}
1 & 1 \\
1 & -1
\end{bmatrix}.
\end{dmath}
%
The \( \ln 0 \) makes the evaluation of this matrix logarithm rather unpleasant.  To give meaning to the entropy expression, we have to do two things, the first is treating the trace operation as a higher precedence than the logarithms that it contains.  That is
%
\begin{dmath}\label{eqn:densityMatrixEntropy:140}
-\tr ( \rho \ln \rho )
=
-\tr ( E \wedge E^{-1} E \ln \wedge E^{-1} )
=
-\tr ( E \wedge \ln \wedge E^{-1} )
=
-\tr ( E^{-1} E \wedge \ln \wedge )
=
-\tr ( \wedge \ln \wedge )
=
- \sum_k \wedge_{kk} \ln \wedge_{kk}.
\end{dmath}
%
Now the matrix of the logarithm need not be evaluated, but we still need to give meaning to \( \wedge_{kk} \ln \wedge_{kk} \) for zero diagonal entries.  This can be done by considering a limiting scenario
%
\begin{dmath}\label{eqn:densityMatrixEntropy:160}
-\lim_{a \rightarrow 0} a \ln a
=
-\lim_{x \rightarrow \infty} e^{-x} \ln e^{-x}
=
\lim_{x \rightarrow \infty} x e^{-x}
=
0.
\end{dmath}
%
The entropy can now be expressed in the unambiguous form, summing over all the non-zero eigenvalues of the density operator
%
%\begin{dmath}\label{eqn:densityMatrixEntropy:180}
\boxedEquation{eqn:densityMatrixEntropy:180}{
S = - \sum_{ \wedge_{kk} \ne 0} \wedge_{kk} \ln \wedge_{kk}.
}
%\end{dmath}
%
%\EndArticle

      \section{Problems.}
         %
% Copyright � 2015 Peeter Joot.  All Rights Reserved.
% Licenced as described in the file LICENSE under the root directory of this GIT repository.
%
%\input{../blogpost.tex}
%\renewcommand{\basename}{pauliProblems}
%\renewcommand{\dirname}{notes/phy1520/}
%%\newcommand{\dateintitle}{}
%%\newcommand{\keywords}{}
%
%\input{../peeter_prologue_print2.tex}
%\usepackage{peeters_layout_exercise}
%
%\beginArtNoToc
%
%\generatetitle{Pauli matrix problems}
%\chapter{Pauli matrix problems}
%\label{chap:pauliProblems}

\makeoproblem
%{Representation of \( 2 \times 2 \) matrix with Pauli matrices.}
{Pauli and \( 2 \times 2 \) matrixes.}
{problem:pauliProblems:1.2}{\citep{sakurai2014modern} pr. 1.2}{
Given an arbitrary \( 2 \times 2 \) matrix \( X = a_0 + \Bsigma \cdot \Ba \), show the relationships between \( a_\mu \) and \( \trace(X), \trace(\sigma_k X) \), and \( X_{ij} \).
\index{Pauli matrix}
\index{Pauli matrix!trace}
} % problem

\makeanswer{problem:pauliProblems:1.2}{

Observe that each of the Pauli matrices \( \sigma_k \) are traceless
%
\begin{equation}\label{eqn:pauliProblems:20}
\begin{aligned}
\sigma_x &= \PauliX \\
\sigma_y &= \PauliY \\
\sigma_z &= \PauliZ \\
\end{aligned},
\end{equation}
%
so \( \trace(X) = 2 a_0 \).  Note that \( \trace(\sigma_k \sigma_m) = 2 \delta_{k m} \), so \( \trace(\sigma_k X) = 2 a_k \).

Notationally, it would seem to make sense to define \( \sigma_0 \equiv I \), so that \( \trace(\sigma_\mu X) = a_\mu \).  I don't know if that is common practice.

For the opposite relations, given
%
\begin{dmath}\label{eqn:pauliProblems:40}
X
= a_0 + \Bsigma \cdot \Ba
= \PauliI a_0 + \PauliX a_1 + \PauliY a_2 + \PauliZ a_3
=
\begin{bmatrix}
a_0 + a_3 & a_1 - i a_2 \\
a_1 + i a_2 & a_0 - a_3
\end{bmatrix}
=
\begin{bmatrix}
X_{11} & X_{12} \\
X_{21} & X_{22} \\
\end{bmatrix},
\end{dmath}
%
so
%\begin{equation}\label{eqn:pauliProblems:60}
%\begin{aligned}
%X_{11} &= a_0 + a_3 \\
%X_{22} &= a_0 - a_3 \\
%X_{12} &= a_1 - i a_2 \\
%X_{21} &= a_1 + i a_2 \\
%\end{aligned},
%\end{equation}
%
%or
\begin{equation}\label{eqn:pauliProblems:80}
\begin{aligned}
a_0 &= \inv{2} \lr{ X_{11} + X_{22} } \\
a_1 &= \inv{2} \lr{ X_{12} + X_{21} } \\
a_2 &= \inv{2 i} \lr{ X_{21} - X_{12} } \\
a_3 &= \inv{2} \lr{ X_{11} - X_{22} }
\end{aligned}.
\end{equation}
} % answer

\makeoproblem{Rotation transformation.}{problem:pauliProblems:1.3}{\citep{sakurai2014modern} pr. 1.3}{
Determine the structure and determinant of the transformation
\index{rotation}
%
\begin{equation}\label{eqn:pauliProblems:100}
\Bsigma \cdot \Ba \rightarrow
\Bsigma \cdot \Ba' =
\exp\lr{ i \Bsigma \cdot \ncap \phi/2}
\Bsigma \cdot \Ba
\exp\lr{ -i \Bsigma \cdot \ncap \phi/2}.
\end{equation}
%
} % problem

\makeanswer{problem:pauliProblems:1.3}{

Knowing Geometric Algebra, this is recognized as a rotation transformation.  In GA, \( i \) is treated as a pseudoscalar (which commutes with all grades in \R{3}), and the expression can be reduced to one involving dot and wedge products.  Let's see how can this be reduced using only the Pauli matrix toolbox.

First, consider the determinant of one of the exponentials.  Showing that one such exponential has unit determinant is sufficient.  The matrix representation of the unit normal is
%
\begin{dmath}\label{eqn:pauliProblems:120}
\Bsigma \cdot \ncap
= n_x \PauliX
+ n_y \PauliY
+ n_z \PauliZ
=
\begin{bmatrix}
n_z & n_x - i n_y \\
n_x + i n_y & -n_z
\end{bmatrix}.
\end{dmath}
%
This is expected to have a unit square, and does
%
\begin{dmath}\label{eqn:pauliProblems:140}
\lr{ \Bsigma \cdot \ncap }^2
=
\begin{bmatrix}
n_z & n_x - i n_y \\
n_x + i n_y & -n_z
\end{bmatrix}
\begin{bmatrix}
n_z & n_x - i n_y \\
n_x + i n_y & -n_z
\end{bmatrix}
=
\lr{ n_x^2 + n_y^2 + n_z^2 }
\begin{bmatrix}
1 & 0 \\
0 & 1
\end{bmatrix}
=
1.
\end{dmath}
%
This allows for a cosine and sine expansion of the exponential, as in
%
\begin{dmath}\label{eqn:pauliProblems:160}
\exp\lr{ i \Bsigma \cdot \ncap \theta}
=
\cos\theta + i \Bsigma \cdot \ncap \sin\theta
=
\cos\theta
\begin{bmatrix}
1 & 0 \\
0 & 1
\end{bmatrix}
+
i \sin\theta
\begin{bmatrix}
n_z & n_x - i n_y \\
n_x + i n_y & -n_z
\end{bmatrix}
=
\begin{bmatrix}
\cos\theta + i n_z \sin\theta & \lr{ n_x - i n_y } i \sin\theta \\
\lr{ n_x + i n_y } i \sin\theta & \cos\theta - i n_z \sin\theta \\
\end{bmatrix}.
\end{dmath}
%
\index{rotation!determinant}
This has determinant
%
\begin{dmath}\label{eqn:pauliProblems:180}
\Abs{\exp\lr{ i \Bsigma \cdot \ncap \theta} }
=
\cos^2\theta + n_z^2 \sin^2\theta
-
\lr{ -n_x^2 + -n_y^2 } \sin^2\theta
=
\cos^2\theta + \lr{ n_x^2 + n_y^2 + n_z^2 } \sin^2\theta
= 1,
\end{dmath}
%
as expected.

Next step is to show that this transformation is a rotation, and determine the sense of the rotation.  Let \( C = \cos\phi/2, S = \sin\phi/2 \), so that
%
\begin{dmath}\label{eqn:pauliProblems:200}
\Bsigma \cdot \Ba'
=
\exp\lr{ i \Bsigma \cdot \ncap \phi/2}
\Bsigma \cdot \Ba
\exp\lr{ -i \Bsigma \cdot \ncap \phi/2}
=
\lr{ C + i \Bsigma \cdot \ncap S }
\Bsigma \cdot \Ba
\lr{ C - i \Bsigma \cdot \ncap S }
=
\lr{ C + i \Bsigma \cdot \ncap S }
\lr{ C \Bsigma \cdot \Ba - i \Bsigma \cdot \Ba \Bsigma \cdot \ncap S }
=
C^2 \Bsigma \cdot \Ba + \Bsigma \cdot \ncap \Bsigma \cdot \Ba \Bsigma \cdot \ncap S^2
+ i \lr{
-\Bsigma \cdot \Ba \Bsigma \cdot \ncap
+ \Bsigma \cdot \ncap \Bsigma \cdot \Ba
} S C
=
\inv{2} \lr{ 1 + \cos\phi}
\Bsigma \cdot \Ba
+ \Bsigma \cdot \ncap \Bsigma \cdot \Ba \Bsigma \cdot \ncap \inv{2} \lr{ 1 - \cos\phi}
+ i
\antisymmetric{
\Bsigma \cdot \ncap }{\Bsigma \cdot \Ba }
\inv{2} \sin\phi
=
\inv{2}
\Bsigma \cdot \ncap
\symmetric{
\Bsigma \cdot \ncap }{\Bsigma \cdot \Ba }
+ \inv{2}
\Bsigma \cdot \ncap
\antisymmetric{
\Bsigma \cdot \ncap }{\Bsigma \cdot \Ba } \cos\phi
+
\inv{2}
i
\antisymmetric{
\Bsigma \cdot \ncap }{\Bsigma \cdot \Ba }
\sin\phi.
\end{dmath}
%
Observe that the angle dependent portion can be written in a compact exponential form
%
\begin{dmath}\label{eqn:pauliProblems:220}
\Bsigma \cdot \Ba'
=
\inv{2}
\Bsigma \cdot \ncap
\symmetric{
\Bsigma \cdot \ncap }{\Bsigma \cdot \Ba }
+
\lr{
\cos\phi
+
i
\Bsigma \cdot \ncap
\sin\phi
}
\inv{2}
\Bsigma \cdot \ncap
\antisymmetric{
\Bsigma \cdot \ncap }{\Bsigma \cdot \Ba }
=
\inv{2}
\Bsigma \cdot \ncap
\symmetric{
\Bsigma \cdot \ncap }{\Bsigma \cdot \Ba }
+
\exp\lr{ i \Bsigma \cdot \ncap \phi }
\inv{2}
\Bsigma \cdot \ncap
\antisymmetric{
\Bsigma \cdot \ncap }{\Bsigma \cdot \Ba }.
\end{dmath}
%
The anticommutator and commutator products with the unit normal can be identified as projections and rejections respectively.  Consider the symmetric product first
%
\begin{dmath}\label{eqn:pauliProblems:240}
\inv{2}
\symmetric{
\Bsigma \cdot \ncap }{\Bsigma \cdot \Ba }
=
\inv{2}
\sum n_r a_s \lr{ \sigma_r \sigma_s + \sigma_s \sigma_r }
=
\inv{2}
\sum_{r \ne s} n_r a_s \lr{ \sigma_r \sigma_s + \sigma_s \sigma_r }
+
\inv{2}
\sum_{r } n_r a_r 2
= 2 \ncap \cdot \Ba.
\end{dmath}
%
This shows that
\begin{dmath}\label{eqn:pauliProblems:260}
\inv{2}
\Bsigma \cdot \ncap
\symmetric{
\Bsigma \cdot \ncap }{\Bsigma \cdot \Ba }
=
\lr{ \ncap \cdot \Ba } \Bsigma \cdot \ncap,
\end{dmath}
%
which is the projection of \( \Ba \) in the direction of the normal \( \ncap \).  To show that the commutator term is the rejection, consider the sum of the two
%
\begin{dmath}\label{eqn:pauliProblems:280}
\inv{2}
\Bsigma \cdot \ncap
\symmetric{
\Bsigma \cdot \ncap }{\Bsigma \cdot \Ba }
+
\inv{2}
\Bsigma \cdot \ncap
\antisymmetric{
\Bsigma \cdot \ncap }{\Bsigma \cdot \Ba }
=
\Bsigma \cdot \ncap
\Bsigma \cdot \ncap \Bsigma \cdot \Ba
=
\Bsigma \cdot \Ba,
\end{dmath}
%
so we must have
%
\begin{dmath}\label{eqn:pauliProblems:300}
\Bsigma \cdot \Ba - \lr{ \ncap \cdot \Ba } \Bsigma \cdot \ncap
=
\inv{2}
\Bsigma \cdot \ncap
\antisymmetric{
\Bsigma \cdot \ncap }{\Bsigma \cdot \Ba }.
\end{dmath}
%
This is the component of \( \Ba \) that has the projection in the \( \ncap \) direction removed.  Looking back to \cref{eqn:pauliProblems:220}, the transformation leaves components of the vector that are colinear with the unit normal unchanged, and applies an exponential operation to the component that lies in what is presumed to be the rotation plane.  To verify that this latter portion of the transformation is a rotation, and to determine the sense of the rotation, let's expand the factor of the sine of \cref{eqn:pauliProblems:200}.

That is
%
\begin{dmath}\label{eqn:pauliProblems:320}
\frac{i}{2} \antisymmetric{ \Bsigma \cdot \ncap }{\Bsigma \cdot \Ba }
=
\frac{i}{2} \sum n_r a_s \antisymmetric{ \sigma_r }{\sigma_s }
=
\frac{i}{2} \sum n_r a_s 2 i \epsilon_{r s t} \sigma_t
=
- \sum \sigma_t n_r a_s \epsilon_{r s t}
=
-\Bsigma \cdot \lr{ \ncap \cross \Ba }
=
\Bsigma \cdot \lr{ \Ba \cross \ncap }.
\end{dmath}
%
Since \( \Ba \cross \ncap = \lr{ \Ba - \ncap (\ncap \cdot \Ba) } \cross \ncap \), this vector is seen to lie in the plane normal to \( \ncap \), but perpendicular to the rejection of \( \ncap \) from \( \Ba \).  That completes the demonstration that this is a rotation transformation.

To understand the sense of this rotation, consider \( \ncap = \zcap, \Ba = \xcap \), so
%
\begin{dmath}\label{eqn:pauliProblems:340}
\Bsigma \cdot \lr{ \Ba \cross \ncap }
=
\Bsigma \cdot \lr{ \xcap \cross \zcap }
=
-\Bsigma \cdot \ycap,
\end{dmath}
%
and
\begin{dmath}\label{eqn:pauliProblems:360}
\Bsigma \cdot \Ba'
=
\xcap \cos\phi - \ycap \sin\phi,
\end{dmath}
%
showing that this rotation transformation has a clockwise sense.
} % answer

%\EndArticle

         %
% Copyright � 2015 Peeter Joot.  All Rights Reserved.
% Licenced as described in the file LICENSE under the root directory of this GIT repository.
%
%\input{../blogpost.tex}
%\renewcommand{\basename}{braketManip}
%\renewcommand{\dirname}{notes/phy1520/}
%%\newcommand{\dateintitle}{}
%%\newcommand{\keywords}{}
%
%\input{../peeter_prologue_print2.tex}
%
%\usepackage{peeters_layout_exercise}
%\usepackage{peeters_braket}
%
%\beginArtNoToc
%
%\generatetitle{bra-ket manipulation problems}
%%\chapter{bra-ket manipulation problems}
%%\label{chap:braketManip}
\makeoproblem{Some bra-ket manipulation problems.}{problem:braketManip:1.4}{\citep{sakurai2014modern} pr. 1.4}{
\index{braket}
Using braket logic expand

\makesubproblem{}{problem:braketManip:1.4:a}
\begin{equation}\label{eqn:braketManip:20}
\trace{X Y}
\end{equation}
\makesubproblem{}{problem:braketManip:1.4:b}
\begin{equation}\label{eqn:braketManip:40}
(X Y)^\dagger
\end{equation}
\makesubproblem{}{problem:braketManip:1.4:c}
\begin{equation}\label{eqn:braketManip:60}
e^{i f(A)},
\end{equation}
%
where \( A \) is Hermitian with a complete set of eigenvalues.

\makesubproblem{}{problem:braketManip:1.4:d}
\begin{equation}\label{eqn:braketManip:80}
\sum_{a'} \Psi_{a'}(\Bx')^\conj \Psi_{a'}(\Bx''),
\end{equation}
%
where \( \Psi_{a'}(\Bx'') = \braket{\Bx'}{a'} \).

} % problem

\makeanswer{problem:braketManip:1.4}{
%
\makeSubAnswer{}{problem:braketManip:1.4:a}
%
\begin{dmath}\label{eqn:braketManip:100}
\trace{X Y}
= \sum_a \bra{a} X Y \ket{a}
= \sum_{a,b} \bra{a} X \ket{b}\bra{b} Y \ket{a}
= \sum_{a,b}
\bra{b} Y \ket{a}
\bra{a} X \ket{b}
= \sum_{a,b}
\bra{b} Y
X \ket{b}
= \trace{ Y X }.
\end{dmath}
%
\makeSubAnswer{}{problem:braketManip:1.4:b}
%
\begin{dmath}\label{eqn:braketManip:120}
\bra{a} \lr{ X Y}^\dagger \ket{b}
=
\lr{ \bra{b} X Y \ket{a} }^\conj
=
\sum_c \lr{ \bra{b} X \ket{c}\bra{c} Y \ket{a} }^\conj
=
\sum_c \lr{ \bra{b} X \ket{c} }^\conj \lr{ \bra{c} Y \ket{a} }^\conj
=
\sum_c
\lr{ \bra{c} Y \ket{a} }^\conj
\lr{ \bra{b} X \ket{c} }^\conj
=
\sum_c
\bra{a} Y^\dagger \ket{c}
\bra{c} X^\dagger \ket{b}
=
\bra{a} Y^\dagger
X^\dagger \ket{b},
\end{dmath}
%
so \( \lr{ X Y }^\dagger = Y^\dagger X^\dagger \).

\makeSubAnswer{}{problem:braketManip:1.4:c}
%
Let's presume that the function \( f \) has a Taylor series representation
%
\begin{dmath}\label{eqn:braketManip:140}
f(A) = \sum_r b_r A^r.
\end{dmath}
%
If the eigenvalues of \( A \) are given by
%
\begin{dmath}\label{eqn:braketManip:160}
A \ket{a_s} = a_s \ket{a_s},
\end{dmath}
%
this operator can be expanded like
%
\begin{dmath}\label{eqn:braketManip:180}
A
= \sum_{a_s} A \ket{a_s} \bra{a_s}
= \sum_{a_s} a_s \ket{a_s} \bra{a_s},
\end{dmath}
%
To compute powers of this operator, consider first the square
%
\begin{dmath}\label{eqn:braketManip:200}
A^2 =
=
\sum_{a_s} a_s \ket{a_s} \bra{a_s}
\sum_{a_r} a_r \ket{a_r} \bra{a_r}
=
\sum_{a_s, a_r} a_s a_r \ket{a_s} \bra{a_s} \ket{a_r} \bra{a_r}
=
\sum_{a_s, a_r} a_s a_r \ket{a_s} \delta_{s r} \bra{a_r}
=
\sum_{a_s} a_s^2 \ket{a_s} \bra{a_s}.
\end{dmath}
%
The pattern for higher powers will clearly just be
%
\begin{dmath}\label{eqn:braketManip:220}
A^k =
\sum_{a_s} a_s^k \ket{a_s} \bra{a_s},
\end{dmath}
%
so the expansion of \( f(A) \) will be
%
\begin{dmath}\label{eqn:braketManip:240}
f(A)
= \sum_r b_r A^r
= \sum_r b_r
\sum_{a_s} a_s^r \ket{a_s} \bra{a_s}
=
\sum_{a_s} \lr{ \sum_r b_r a_s^r } \ket{a_s} \bra{a_s}
=
\sum_{a_s} f(a_s) \ket{a_s} \bra{a_s}.
\end{dmath}
%
The exponential expansion is
%
\begin{dmath}\label{eqn:braketManip:260}
e^{i f(A)}
=
\sum_t \frac{i^t}{t!} f^t(A)
=
\sum_t \frac{i^t}{t!}
\lr{ \sum_{a_s} f(a_s) \ket{a_s} \bra{a_s} }^t
=
\sum_t \frac{i^t}{t!}
\sum_{a_s} f^t(a_s) \ket{a_s} \bra{a_s}
=
\sum_{a_s}
e^{i f(a_s) }
\ket{a_s} \bra{a_s}.
\end{dmath}
%
\makeSubAnswer{}{problem:braketManip:1.4:d}
%
\begin{dmath}\label{eqn:braketManip:99}
\sum_{a'} \Psi_{a'}(\Bx')^\conj \Psi_{a'}(\Bx'')
=
\sum_{a'}
\braket{\Bx'}{a'}^\conj
\braket{\Bx''}{a'}
=
\sum_{a'}
\braket{a'}{\Bx'}
\braket{\Bx''}{a'}
=
\sum_{a'}
\braket{\Bx''}{a'}
\braket{a'}{\Bx'}
=
\braket{\Bx''}{\Bx'}
= \delta\lr{\Bx'' - \Bx'}.
\end{dmath}
%
} % answer

%\EndArticle

         %
% Copyright � 2015 Peeter Joot.  All Rights Reserved.
% Licenced as described in the file LICENSE under the root directory of this GIT repository.
%
%\input{../blogpost.tex}
%\renewcommand{\basename}{moreBraKetProblems}
%\renewcommand{\dirname}{notes/phy1520/}
%%\newcommand{\dateintitle}{}
%%\newcommand{\keywords}{}
%
%\input{../peeter_prologue_print2.tex}
%
%\usepackage{peeters_layout_exercise}
%\usepackage{peeters_braket}
%
%\beginArtNoToc
%
%\generatetitle{Bra-ket and spin one-half problems}
%\chapter{Bra-ket and spin one-half problems}
%\label{chap:moreBraKetProblems}


\makeoproblem{Operator matrix representation.}{problem:moreBraKetProblems:1.5}{\citep{sakurai2014modern} pr. 1.5}{
\index{matrix representation}

\makesubproblem{}{problem:moreBraKetProblems:1.5:a}
%
Determine the matrix representation of \( \ket{\alpha}\bra{\beta} \) given a complete set of eigenvectors \( \ket{a^r} \).

\makesubproblem{}{problem:moreBraKetProblems:1.5:b}
%
Verify with \( \ket{\alpha} = \ket{s_z = \Hbar/2}, \ket{s_x = \Hbar/2} \).

} % problem

\makeanswer{problem:moreBraKetProblems:1.5}{
%
\makeSubAnswer{}{problem:moreBraKetProblems:1.5:a}
%
Forming the matrix element
%
\begin{dmath}\label{eqn:moreBraKetProblems:20}
\bra{a^r} \lr{ \ket{\alpha}\bra{\beta} } \ket{a^s}
=
\braket{a^r}{\alpha}\braket{\beta}{a^s}
=
\braket{a^r}{\alpha}
\braket{a^s}{\beta}^\conj,
\end{dmath}
%
the matrix representation is seen to be
%
\begin{dmath}\label{eqn:moreBraKetProblems:40}
\ket{\alpha}\bra{\beta}
\sim
\begin{bmatrix}
\bra{a^1} \lr{ \ket{\alpha}\bra{\beta} } \ket{a^1} & \bra{a^1} \lr{ \ket{\alpha}\bra{\beta} } \ket{a^2} & \cdots \\
\bra{a^2} \lr{ \ket{\alpha}\bra{\beta} } \ket{a^1} & \bra{a^2} \lr{ \ket{\alpha}\bra{\beta} } \ket{a^2} & \cdots \\
\vdots & \vdots & \ddots \\
\end{bmatrix}
=
\begin{bmatrix}
\braket{a^1}{\alpha} \braket{a^1}{\beta}^\conj & \braket{a^1}{\alpha} \braket{a^2}{\beta}^\conj & \cdots \\
\braket{a^2}{\alpha} \braket{a^1}{\beta}^\conj & \braket{a^2}{\alpha} \braket{a^2}{\beta}^\conj & \cdots \\
\vdots & \vdots & \ddots \\
\end{bmatrix}.
\end{dmath}
%
\makeSubAnswer{}{problem:moreBraKetProblems:1.5:b}
First compute the spin-z representation of \( \ket{s_x = \Hbar/2 } \).
%
\begin{dmath}\label{eqn:moreBraKetProblems:60}
\begin{aligned}
\lr{ S_x - \Hbar/2 I }
\begin{bmatrix}
a \\
b
\end{bmatrix}
&=
\lr{
\begin{bmatrix}
0 & \Hbar/2 \\
\Hbar/2 & 0 \\
\end{bmatrix}
-
\begin{bmatrix}
\Hbar/2 & 0 \\
0 & \Hbar/2 \\
\end{bmatrix}
} \\
&=
\begin{bmatrix}
a \\
b
\end{bmatrix} \\
&=
\frac{\Hbar}{2}
\begin{bmatrix}
-1 & 1 \\
1 & -1 \\
\end{bmatrix}
\begin{bmatrix}
a \\
b
\end{bmatrix},
\end{aligned}
\end{dmath}
so \( \ket{s_x = \Hbar/2 } \propto (1,1) \).  Normalized we have
%
\begin{equation}\label{eqn:moreBraKetProblems:80}
\begin{aligned}
\ket{\alpha} &= \ket{s_z = \Hbar/2 } =
\begin{bmatrix}
1 \\
0
\end{bmatrix} \\
\ket{\beta} &= \ket{s_z = \Hbar/2 }
\inv{\sqrt{2}}
\begin{bmatrix}
1 \\
1
\end{bmatrix}.
\end{aligned}
\end{equation}
%
Using \cref{eqn:moreBraKetProblems:40} the matrix representation is
%
\begin{dmath}\label{eqn:moreBraKetProblems:100}
\ket{\alpha}\bra{\beta}
\sim
\begin{bmatrix}
(1) (1/\sqrt{2})^\conj & (1) (1/\sqrt{2})^\conj \\
(0) (1/\sqrt{2})^\conj & (0) (1/\sqrt{2})^\conj \\
\end{bmatrix}
=
\inv{\sqrt{2}}
\begin{bmatrix}
1 & 1 \\
0 & 0
\end{bmatrix}.
\end{dmath}
%
This can be confirmed with direct computation
\begin{dmath}\label{eqn:moreBraKetProblems:120}
\ket{\alpha}\bra{\beta}
=
\begin{bmatrix}
1 \\
0
\end{bmatrix}
\inv{\sqrt{2}}
\begin{bmatrix}
1 & 1
\end{bmatrix}
=
\inv{\sqrt{2}}
\begin{bmatrix}
1 & 1 \\
0 & 0
\end{bmatrix}.
\end{dmath}
%
} % answer
%
\makeoproblem{Eigenvalue of sum of kets.}{problem:moreBraKetProblems:6}{\citep{sakurai2014modern} pr. 1.6}{
Given eigenkets \( \ket{i}, \ket{j} \) of an operator \( A \), what are the conditions that \( \ket{i} + \ket{j} \) is also an eigenvector?
} % problem

\makeanswer{problem:moreBraKetProblems:6}{
Let \( A \ket{i} = i \ket{i}, A \ket{j} = j \ket{j} \), and suppose that the sum is an eigenket.  Then there must be a value \( a \) such that
%
\begin{dmath}\label{eqn:moreBraKetProblems:140}
A \lr{ \ket{i} + \ket{j} } = a \lr{ \ket{i} + \ket{j} },
\end{dmath}
%
so
%
\begin{dmath}\label{eqn:moreBraKetProblems:160}
i \ket{i} + j \ket{j} = a \lr{ \ket{i} + \ket{j} }.
\end{dmath}
%
Operating with \( \bra{i}, \bra{j} \) respectively, gives
%
\begin{equation}\label{eqn:moreBraKetProblems:180}
\begin{aligned}
i &= a \\
j &= a,
\end{aligned}
\end{equation}
so for the sum to be an eigenket, both of the corresponding energy eigenvalues must be identical (i.e. linear combinations of degenerate eigenkets are also eigenkets).
} % answer

\makeoproblem{Null operator.}{problem:moreBraKetProblems:7}{\citep{sakurai2014modern} pr. 1.7}{
\index{null operator}
Given eigenkets \( \ket{a'} \) of operator \( A \)

\makesubproblem{}{problem:moreBraKetProblems:7:a}
show that
%
\begin{equation}\label{eqn:moreBraKetProblems:200}
\prod_{a'} \lr{ A - a' }
\end{equation}
is the null operator.

\makesubproblem{}{problem:moreBraKetProblems:7:b}
%
\begin{equation}\label{eqn:moreBraKetProblems:220}
\prod_{a'' \ne a'} \frac{\lr{ A - a'' }}{a' - a''}
\end{equation}

\makesubproblem{}{problem:moreBraKetProblems:7:c}
Illustrate using \( S_z \) for a spin 1/2 system.
} % problem

\makeanswer{problem:moreBraKetProblems:7}{
\makeSubAnswer{}{problem:moreBraKetProblems:7:a}
Application of \( \ket{a} \), the eigenket of \( A \) with eigenvalue \( a \) to any term \( A - a' \) scales \( \ket{a} \) by \( a - a' \), so the product operating on \( \ket{a} \) is
%
\begin{equation}\label{eqn:moreBraKetProblems:240}
\prod_{a'} \lr{ A - a' } \ket{a} = \prod_{a'} \lr{ a - a' } \ket{a}.
\end{equation}
%
Since \( \ket{a} \) is one of the \( \setlr{\ket{a'}} \) eigenkets of \( A \), one of these terms must be zero.

\makeSubAnswer{}{problem:moreBraKetProblems:7:b}
%
Again, consider the action of the operator on \( \ket{a} \),
%
\begin{equation}\label{eqn:moreBraKetProblems:260}
\prod_{a'' \ne a'} \frac{\lr{ A - a'' }}{a' - a''} \ket{a}
=
\prod_{a'' \ne a'} \frac{\lr{ a - a'' }}{a' - a''} \ket{a}.
\end{equation}
%
If \( \ket{a} = \ket{a'} \), then \( \prod_{a'' \ne a'} \frac{\lr{ A - a'' }}{a' - a''} \ket{a} = \ket{a} \), whereas if it does not, then it equals one of the \( a'' \) energy eigenvalues.  This is a representation of the Kronecker delta function
%
\begin{dmath}\label{eqn:moreBraKetProblems:300}
\prod_{a'' \ne a'} \frac{\lr{ A - a'' }}{a' - a''} \ket{a} \equiv \delta_{a', a} \ket{a}
\end{dmath}
%
\makeSubAnswer{}{problem:moreBraKetProblems:7:c}
%
For operator \( S_z \) the eigenvalues are \( \setlr{ \Hbar/2, -\Hbar/2 } \), so the null operator must be
%
\begin{dmath}\label{eqn:moreBraKetProblems:280}
\prod_{a'} \lr{ A - a' }
=
\lr{ \frac{\Hbar}{2} }^2 \lr{ \PauliZ - \PauliI } \lr{ \PauliZ + \PauliI }
=
\begin{bmatrix}
0 & 0 \\
0 & -2
\end{bmatrix}
\begin{bmatrix}
2 & 0  \\
0 & 0 \\
\end{bmatrix}
=
\begin{bmatrix}
0 & 0  \\
0 & 0 \\
\end{bmatrix}
\end{dmath}

For the delta representation, consider the \( \ket{\pm} \) states and their eigenvalue.  The delta operators are
%
\begin{dmath}\label{eqn:moreBraKetProblems:320}
\begin{aligned}
\prod_{a'' \ne \Hbar/2} \frac{\lr{ A - a'' }}{\Hbar/2 - a''}
&=
\frac{S_z - (-\Hbar/2) I}{\Hbar/2 - (-\Hbar/2)} \\
&=
\inv{2} \lr{ \sigma_z + I } \\
&=
\inv{2} \lr{ \PauliZ + \PauliI } \\
&=
\inv{2}
\begin{bmatrix}
2 & 0 \\
0 & 0
\end{bmatrix} \\
&=
\begin{bmatrix}
1 & 0 \\
0 & 0
\end{bmatrix}.
\end{aligned}
\end{dmath}
\begin{dmath}\label{eqn:moreBraKetProblems:340}
\begin{aligned}
\prod_{a'' \ne -\Hbar/2} \frac{\lr{ A - a'' }}{-\Hbar/2 - a''}
&=
\frac{S_z - (\Hbar/2) I}{-\Hbar/2 - \Hbar/2} \\
&=
\inv{2} \lr{ \sigma_z - I } \\
&=
\inv{2} \lr{ \PauliZ - \PauliI } \\
&=
\inv{2}
\begin{bmatrix}
0 & 0 \\
0 & -2
\end{bmatrix} \\
&=
\begin{bmatrix}
0 & 0 \\
0 & 1
\end{bmatrix}.
\end{aligned}
\end{dmath}
%
These clearly have the expected delta function property acting on kets \( \ket{+} = (1,0)^\T, \ket{-} = (0, 1)^\T \).

} % answer


\makeoproblem{Spin half general normal.}{problem:moreBraKetProblems:9}{\citep{sakurai2014modern} pr. 1.9}{
%
\index{spin half!states}
Construct \( \ket{\BS \cdot \ncap ; + } \), where \( \ncap = ( \cos\alpha \sin\beta, \sin\alpha \sin\beta, \cos\beta )^\T \)  such that
%
\begin{dmath}\label{eqn:moreBraKetProblems:360}
\BS \cdot \ncap \ket{\BS \cdot \ncap ; + } =
\frac{\Hbar}{2} \ket{\BS \cdot \ncap ; + },
\end{dmath}
%
Solve this as an eigenvalue problem.
} % problem

\makeanswer{problem:moreBraKetProblems:9}{
%
The spin operator for this direction is
%
\begin{dmath}\label{eqn:moreBraKetProblems:380}
\BS \cdot \ncap
= \frac{\Hbar}{2} \Bsigma \cdot \ncap
= \frac{\Hbar}{2}
\lr{
\cos\alpha \sin\beta \PauliX + \sin\alpha \sin\beta \PauliY + \cos\beta \PauliZ
}
=
\frac{\Hbar}{2}
\begin{bmatrix}
\cos\beta &
e^{-i\alpha}
\sin\beta
\\
e^{i\alpha}
\sin\beta
& -\cos\beta
\end{bmatrix}.
\end{dmath}
%
Observed that this is traceless and has a \( -\Hbar/2 \) determinant like any of the \( x,y,z \) spin operators.

Assuming that this has an \( \Hbar/2 \) eigenvalue (to be verified later), the eigenvalue problem is
%
\begin{dmath}\label{eqn:moreBraKetProblems:400}
0 =
\BS \cdot \ncap - \Hbar/2 I
=
\frac{\Hbar}{2}
\begin{bmatrix}
\cos\beta -1 &
e^{-i\alpha}
\sin\beta
\\
e^{i\alpha}
\sin\beta
& -\cos\beta -1
\end{bmatrix}
=
\Hbar
\begin{bmatrix}
- \sin^2 \frac{\beta}{2} &
e^{-i\alpha}
\sin\frac{\beta}{2} \cos\frac{\beta}{2}
\\
e^{i\alpha}
\sin\frac{\beta}{2} \cos\frac{\beta}{2}
& -\cos^2 \frac{\beta}{2}
\end{bmatrix}
\end{dmath}

This has a zero determinant as expected, and the eigenvector \( (a,b) \) will satisfy
%
\begin{dmath}\label{eqn:moreBraKetProblems:420}
0
= - \sin^2 \frac{\beta}{2} a +
e^{-i\alpha}
\sin\frac{\beta}{2} \cos\frac{\beta}{2}
b
= \sin\frac{\beta}{2} \lr{ - \sin \frac{\beta}{2} a +
e^{-i\alpha} b
\cos\frac{\beta}{2}
}
\end{dmath}
%
\begin{dmath}\label{eqn:moreBraKetProblems:440}
\begin{bmatrix}
a \\
b
\end{bmatrix}
\propto
\begin{bmatrix}
\cos\frac{\beta}{2} \\
e^{i\alpha}
\sin\frac{\beta}{2}
\end{bmatrix}.
\end{dmath}
%
This is appropriately normalized, so the ket for \( \BS \cdot \ncap \) is
%
\begin{dmath}\label{eqn:moreBraKetProblems:460}
\ket{ \BS \cdot \ncap ; + } =
\cos\frac{\beta}{2} \ket{+} +
e^{i\alpha}
\sin\frac{\beta}{2}
\ket{-}.
\end{dmath}
%
Note that the other eigenvalue is
%
\begin{dmath}\label{eqn:moreBraKetProblems:480}
\ket{ \BS \cdot \ncap ; - } =
-\sin\frac{\beta}{2} \ket{+} +
e^{i\alpha}
\cos\frac{\beta}{2}
\ket{-}.
\end{dmath}
%
It is straightforward to show that these are orthogonal and that this has the \( -\Hbar/2 \) eigenvalue.

} % answer

\makeoproblem{Two state Hamiltonian.}{problem:moreBraKetProblems:10}{\citep{sakurai2014modern} pr. 1.10}{
\index{Hamiltonian!two state}

Solve the eigenproblem for
%
\begin{dmath}\label{eqn:moreBraKetProblems:500}
H = a \biglr{
\ket{1}\bra{1}
-\ket{2}\bra{2}
+\ket{1}\bra{2}
+\ket{2}\bra{1}
}
\end{dmath}

} % problem

\makeanswer{problem:moreBraKetProblems:10}{
%
In matrix form the Hamiltonian is
%
\begin{dmath}\label{eqn:moreBraKetProblems:520}
H = a
\begin{bmatrix}
1 & 1 \\
1 & -1
\end{bmatrix}.
\end{dmath}
%
The eigenvalue problem is
%
\begin{dmath}\label{eqn:moreBraKetProblems:540}
0
= \Abs{ H - \lambda I }
= (a - \lambda)(-a - \lambda) - a^2
= (-a + \lambda)(a + \lambda) - a^2
= \lambda^2 - a^2 - a^2,
\end{dmath}
%
or
%
\begin{dmath}\label{eqn:moreBraKetProblems:560}
\lambda = \pm \sqrt{2} a.
\end{dmath}
%
An eigenket proportional to \( (\alpha,\beta) \) must satisfy
%
\begin{dmath}\label{eqn:moreBraKetProblems:580}
0
= ( 1 \mp \sqrt{2} ) \alpha + \beta,
\end{dmath}
%
so
%
\begin{dmath}\label{eqn:moreBraKetProblems:600}
\ket{\pm} \propto
\begin{bmatrix}
-1 \\
1 \mp \sqrt{2}
\end{bmatrix},
\end{dmath}
%
or
%
\begin{dmath}\label{eqn:moreBraKetProblems:620}
\ket{\pm}
=
\inv{2(2 - \sqrt{2})}
\begin{bmatrix}
-1 \\
1 \mp \sqrt{2}
\end{bmatrix}
=
\frac{2 + \sqrt{2}}{4}
\begin{bmatrix}
-1 \\
1 \mp \sqrt{2}
\end{bmatrix}.
\end{dmath}
%
That is
\begin{dmath}\label{eqn:moreBraKetProblems:640}
\ket{\pm} =
\frac{2 + \sqrt{2}}{4} \lr{
-\ket{1} + (1 \mp \sqrt{2}) \ket{2}
}.
\end{dmath}
%
} % answer

%
%\EndArticle

         %
% Copyright © 2015 Peeter Joot.  All Rights Reserved.
% Licenced as described in the file LICENSE under the root directory of this GIT repository.
%
%
\makeoproblem{Spin half and dispersion.}{problem:moreBraKetProblems:12}{
2015 ps1.3; \citep{sakurai2014modern} pr. 1.12
}{
\index{spin half!dispersion}
A spin \( 1/2 \) system \( \BS \cdot \ncap \), with \( \ncap = \sin \theta \xcap + \cos\theta \zcap \), is in state with eigenvalue \( \Hbar/2 \).
%
\makesubproblem{}{problem:moreBraKetProblems:12:a}
If \( S_x \) is measured.  What is the probability of getting \( + \Hbar/2 \)?
%
\makesubproblem{}{problem:moreBraKetProblems:12:b}
Evaluate the dispersion in \( S_x \), that is,
%
\begin{equation}\label{eqn:moreBraKetProblems:660}
\expectation{\lr{ S_x - \expectation{S_x}}^2}.
\end{equation}
%
} % problem
%
\makeanswer{problem:moreBraKetProblems:12}{
\makeSubAnswer{}{problem:moreBraKetProblems:12:a}
In matrix form the spin operator for the system is
%
\begin{equation}\label{eqn:moreBraKetProblems:680}
\begin{aligned}
\BS \cdot \ncap
&= \frac{\Hbar}{2} \lr{ \cos\theta \PauliZ + \sin\theta \PauliX}
\\ &= \frac{\Hbar}{2}
\begin{bmatrix}
\cos\theta & \sin\theta \\
\sin\theta & -\cos\theta \\
\end{bmatrix}.
\end{aligned}
\end{equation}
An eigenket \( \ket{\BS \cdot \ncap ; + } = (a,b)^\T \) must satisfy
%
\begin{equation}\label{eqn:moreBraKetProblems:700}
\begin{aligned}
0
&= \lr{ \cos \theta - 1 } a + \sin\theta b
\\ &= \lr{ -2 \sin^2 \frac{\theta}{2} } a + 2 \sin\frac{\theta}{2} \cos\frac{\theta}{2} b
\\ &= -\sin \frac{\theta}{2} a + \cos\frac{\theta}{2} b,
\end{aligned}
\end{equation}
%
so the eigenstate is
\begin{equation}\label{eqn:moreBraKetProblems:720}
\ket{\BS \cdot \ncap ; + }
=
\begin{bmatrix}
\cos\frac{\theta}{2} \\
\sin\frac{\theta}{2}
\end{bmatrix}.
\end{equation}
%
Pick \( \ket{S_x ; \pm } = \inv{\sqrt{2}}
\begin{bmatrix}
1 \\ \pm 1
\end{bmatrix} \) as the basis for the \( S_x \) operator.  Then, for the probability that the system will end up in the \( + \Hbar/2 \) state of \( S_x \), we have
%
\begin{equation}\label{eqn:moreBraKetProblems:740}
\begin{aligned}
P
&= \Abs{\braket{ S_x ; + }{ \BS \cdot \ncap ; + } }^2
\\ &= \Abs{ \inv{\sqrt{2} }
{
\begin{bmatrix}
1 \\
1
\end{bmatrix}}^\dagger
\begin{bmatrix}
\cos\frac{\theta}{2} \\
\sin\frac{\theta}{2}
\end{bmatrix}
}^2
\\ &=\inv{2}
\Abs{
\begin{bmatrix}
1 & 1
\end{bmatrix}
\begin{bmatrix}
\cos\frac{\theta}{2} \\
\sin\frac{\theta}{2}
\end{bmatrix}
}^2
\\ &=
\inv{2}
\lr{
\cos\frac{\theta}{2} +
\sin\frac{\theta}{2}
}^2
\\ &=
\inv{2}
\lr{ 1 + 2 \cos\frac{\theta}{2} \sin\frac{\theta}{2} }
\\ &=
\inv{2}
\lr{ 1 + \sin\theta }.
\end{aligned}
\end{equation}
%
This is a reasonable seeming result, with \( P \in [0, 1] \).  Some special values also further validate this
%
\begin{equation}\label{eqn:moreBraKetProblems:760}
\begin{aligned}
\theta &= 0, \ket{\BS \cdot \ncap ; + } =
\begin{bmatrix}
1 \\
0
\end{bmatrix}
=
\ket{S_z ; +}
=
\inv{\sqrt{2}} \ket{S_x;+}
+\inv{\sqrt{2}} \ket{S_x;-}
\\
\theta &= \pi/2, \ket{\BS \cdot \ncap ; + } =
\inv{\sqrt{2}}
\begin{bmatrix}
1 \\
1
\end{bmatrix}
=
\ket{S_x ; +}
\\
\theta &= \pi, \ket{\BS \cdot \ncap ; + } =
\begin{bmatrix}
0 \\
1
\end{bmatrix}
=
\ket{S_z ; -}
=
\inv{\sqrt{2}} \ket{S_x;+}
-\inv{\sqrt{2}} \ket{S_x;-},
\end{aligned}
\end{equation}

where we see that the probabilities are in proportion to the projection of the initial state onto the measured state \( \ket{S_x ; +} \).
%
\makeSubAnswer{}{problem:moreBraKetProblems:12:b}
%
The \( S_x \) expectation is
%
\begin{equation}\label{eqn:moreBraKetProblems:780}
\begin{aligned}
\expectation{S_x}
&=
\frac{\Hbar}{2}
\begin{bmatrix}
\cos\frac{\theta}{2} & \sin\frac{\theta}{2}
\end{bmatrix}
\PauliX
\begin{bmatrix}
\cos\frac{\theta}{2} \\
\sin\frac{\theta}{2}
\end{bmatrix}
\\ &=
\frac{\Hbar}{2}
\begin{bmatrix}
\cos\frac{\theta}{2} & \sin\frac{\theta}{2}
\end{bmatrix}
\begin{bmatrix}
\sin\frac{\theta}{2} \\
\cos\frac{\theta}{2}
\end{bmatrix}
\\ &=
\frac{\Hbar}{2} 2 \sin\frac{\theta}{2} \cos\frac{\theta}{2}
\\ &=
\frac{\Hbar}{2} \sin\theta.
\end{aligned}
\end{equation}
%
Note that \( S_x^2 = (\Hbar/2)^2I \), so
%
\begin{equation}\label{eqn:moreBraKetProblems:800}
\begin{aligned}
\expectation{S_x^2}
&=
\lr{\frac{\Hbar}{2}}^2
\begin{bmatrix}
\cos\frac{\theta}{2} & \sin\frac{\theta}{2}
\end{bmatrix}
\begin{bmatrix}
\cos\frac{\theta}{2} \\
\sin\frac{\theta}{2}
\end{bmatrix}
\\ &=
\lr{ \frac{\Hbar}{2} }^2
\cos^2\frac{\theta}{2} + \sin^2 \frac{\theta}{2}
\\ &=
\lr{ \frac{\Hbar}{2} }^2.
\end{aligned}
\end{equation}
%
The dispersion is
%
\begin{equation}\label{eqn:moreBraKetProblems:820}
\begin{aligned}
\expectation{\lr{ S_x - \expectation{S_x}}^2}
&=
\expectation{S_x^2} - \expectation{S_x}^2
\\ &=
\lr{ \frac{\Hbar}{2} }^2
\lr{1 - \sin^2 \theta}
\\ &=
\lr{ \frac{\Hbar}{2} }^2
\cos^2 \theta.
\end{aligned}
\end{equation}
%
At \( \theta = \pi/2 \) the dispersion is 0, which is expected since \( \ket{\BS \cdot \ncap ; + } = \ket{ S_x ; + } \) at that point.  Similarly, the dispersion is maximized at \( \theta = 0,\pi \) where the \( \ket{\BS \cdot \ncap ; + } \) component in the \( \ket{S_x ; + } \) direction is minimized.
%
} % answer

         % p13:
         %
% Copyright � 2015 Peeter Joot.  All Rights Reserved.
% Licenced as described in the file LICENSE under the root directory of this GIT repository.
%
%\input{../blogpost.tex}
%\renewcommand{\basename}{sg}
%\renewcommand{\dirname}{notes/phy1520/}
%%\newcommand{\dateintitle}{}
%%\newcommand{\keywords}{}
%
%\input{../peeter_prologue_print2.tex}
%
%\usepackage{peeters_layout_exercise}
%\usepackage{peeters_braket}
%\usepackage{peeters_figures}
%
%\beginArtNoToc
%
%\generatetitle{Cascading Stern-Gerlach}
%\chapter{Cascading Stern-Gerlach}
%\label{chap:sg}

\makeoproblem{Cascading Stern-Gerlach.}{problem:sg:13}{\citep{sakurai2014modern} pr. 1.13}{
\index{Stern-Gerlach}

Three Stern-Gerlach type measurements are performed, the first that prepares the state in a \( \ket{S_z ; + } \) state, the next in a \( \ket{ \BS \cdot \ncap ; + } \) state where \( \ncap = \cos\beta \zcap + \sin\beta \xcap \), and the last performing a \( S_z \) \( \Hbar/2 \) state measurement, as illustrated in \cref{fig:sternGerlach:sternGerlachFig1}.

\imageFigure{../figures/phy1520-quantum/sternGerlachFig1}{Cascaded Stern-Gerlach type measurements.}{fig:sternGerlach:sternGerlachFig1}{0.3}

What is the intensity of the final \( s_z = -\Hbar/2 \) beam?  What is the orientation for the second measuring apparatus to maximize the intensity of this beam?
} % problem

\makeanswer{problem:sg:13}{

The spin operator for the second apparatus is
%
\begin{dmath}\label{eqn:sg:20}
\BS \cdot \ncap
= \frac{\Hbar}{2} \lr{ \sin\beta \PauliX + \cos\beta \PauliZ }
= \frac{\Hbar}{2}
\begin{bmatrix}
\cos\beta & \sin\beta \\
\sin\beta & -\cos\beta
\end{bmatrix}.
\end{dmath}

The intensity of the final \( \ket{S_z ; -} \) beam is
%
\begin{dmath}\label{eqn:sg:40}
P
= \Abs{ \braket{-}{\BS \cdot \ncap ; +} \braket{\BS \cdot \ncap ; +}{+} }^2,
\end{dmath}
%
(i.e. the second apparatus applies a projection operator \( \ket{\BS \cdot \ncap ; +}\bra{\BS \cdot \ncap ; +} \) to the initial \( \ket{+} \) state, and then the \( \ket{-} \) states are selected out of that.

The \( \BS \cdot \ncap \) eigenket is found to be
%
\begin{dmath}\label{eqn:sg:60}
\ket{\BS \cdot \ncap ; +} =
\begin{bmatrix}
\cos\frac{\beta}{2} \\
\sin\frac{\beta}{2} \\
\end{bmatrix},
\end{dmath}
%
so
%
\begin{dmath}\label{eqn:sg:80}
P
= \Abs{
\begin{bmatrix}
0 & 1
\end{bmatrix}
\begin{bmatrix}
\cos\frac{\beta}{2} \\
\sin\frac{\beta}{2} \\
\end{bmatrix}
\begin{bmatrix}
\cos\frac{\beta}{2} &
\sin\frac{\beta}{2} \\
\end{bmatrix}
\begin{bmatrix}
1 \\
0
\end{bmatrix}
}^2
=
\Abs{
\cos\frac{\beta}{2}
\sin\frac{\beta}{2}
}^2
=
\Abs{\inv{2} \sin\beta}^2
=
\inv{4} \sin^2\beta.
\end{dmath}

This is maximized when \( \beta = \pi/2 \), or \( \ncap = \xcap \).  At this angle the state leaving the second apparatus is
%
\begin{dmath}\label{eqn:sg:100}
\begin{bmatrix}
\cos\frac{\beta}{2} \\
\sin\frac{\beta}{2} \\
\end{bmatrix}
\begin{bmatrix}
\cos\frac{\beta}{2} &
\sin\frac{\beta}{2} \\
\end{bmatrix}
\begin{bmatrix}
1 \\
0
\end{bmatrix}
=
\inv{2}
\begin{bmatrix}
1 \\ 1
\end{bmatrix}
\begin{bmatrix}
1 & 1
\end{bmatrix}
\begin{bmatrix}
1 \\ 0
\end{bmatrix}
=
\inv{2}
\begin{bmatrix}
1 \\ 1
\end{bmatrix}
=\inv{2} \ket{+} + \inv{2}\ket{-},
\end{dmath}
%
so the state after filtering the \( \ket{-} \) states is \( \inv{2} \ket{-} \) with intensity (probability density) of \( 1/4 \) relative to a unit normalize input \( \ket{+} \) state to the \( \BS \cdot \ncap \) apparatus.

} % answer

%\EndArticle

         % p16:
         %
% Copyright � 2015 Peeter Joot.  All Rights Reserved.
% Licenced as described in the file LICENSE under the root directory of this GIT repository.
%
%\input{../blogpost.tex}
%\renewcommand{\basename}{anticommutingOperatorWithSimulaneousEigenket}
%\renewcommand{\dirname}{notes/phy1520/}
%%\newcommand{\dateintitle}{}
%%\newcommand{\keywords}{}
%
%\input{../peeter_prologue_print2.tex}
%
%\usepackage{peeters_layout_exercise}
%\usepackage{peeters_braket}
%\usepackage{peeters_figures}
%
%\beginArtNoToc
%
%\generatetitle{Can anticommuting operators have a simultaneous eigenket?}
%\chapter{Can anticommuting operators have a simulaneous eigenket?}
%\label{chap:anticommutingOperatorWithSimulaneousEigenket}
%
\makeoproblem{Anticommuting operators and eigenkets.}{problem:anticommutingOperatorWithSimulaneousEigenket:1}{\citep{sakurai2014modern} pr. 1.16}{
%Can anticommuting operators have a simultaneous eigenket?
\index{simultaneous eigenstate}
Given two Hermitian operators that anticommute
%
\begin{dmath}\label{eqn:anticommutingOperatorWithSimulaneousEigenket:20}
\symmetric{A}{B} = A B + B A = 0,
\end{dmath}
%
is it possible to have a simultaneous eigenket of \( A \) and \( B \)?  Prove or illustrate your assertion.
} % problem
%
\makeanswer{problem:anticommutingOperatorWithSimulaneousEigenket:1}{
%
Suppose that such a simultaneous non-zero eigenket \( \ket{\alpha} \) exists, then
%
\begin{dmath}\label{eqn:anticommutingOperatorWithSimulaneousEigenket:40}
A \ket{\alpha} = a \ket{\alpha},
\end{dmath}
%
and
%
\begin{dmath}\label{eqn:anticommutingOperatorWithSimulaneousEigenket:60}
B \ket{\alpha} = b \ket{\alpha}
\end{dmath}

This gives
%
\begin{dmath}\label{eqn:anticommutingOperatorWithSimulaneousEigenket:80}
\lr{ A B + B A } \ket{\alpha}
=
\lr{A b + B a} \ket{\alpha}
= 2 a b \ket{\alpha}.
\end{dmath}
%
If this is zero, one of the operators must have a zero eigenvalue.  Knowing that we can construct an example of such operators.  In matrix form, let

\begin{subequations}
\label{eqn:anticommutingOperatorWithSimulaneousEigenket:100}
\begin{dmath}\label{eqn:anticommutingOperatorWithSimulaneousEigenket:120}
A =
\begin{bmatrix}
1 & 0 & 0 \\
0 & -1 & 0 \\
0 & 0 & a \\
\end{bmatrix}
\end{dmath}
\begin{dmath}\label{eqn:anticommutingOperatorWithSimulaneousEigenket:140}
B =
\begin{bmatrix}
0 & 1 & 0 \\
1 & 0 & 0 \\
0 & 0 & b \\
\end{bmatrix}.
\end{dmath}
\end{subequations}

These are both Hermitian, and anticommute provided at least one of \( a, b\) is zero.  These have a common eigenket
%
\begin{dmath}\label{eqn:anticommutingOperatorWithSimulaneousEigenket:160}
\ket{\alpha} =
\begin{bmatrix}
0 \\
0 \\
1
\end{bmatrix}.
\end{dmath}
%
A zero eigenvalue of one of the commuting operators may not be a sufficient condition for such anticommutation.
%It also appears that not all Hermitian matrices that anticommute, where one has a zero eigenvalue, necessarily have a common eigenket.  An example is
%
%\begin{subequations}
%\label{eqn:anticommutingOperatorWithSimulaneousEigenket:180}
%\begin{dmath}\label{eqn:anticommutingOperatorWithSimulaneousEigenket:200}
%A =
%\begin{bmatrix}
%1 & 0 & 0 & 0 \\
%0 & -1 & 0  & 0\\
%0 & 0 & 1  & 0\\
%0 & 0 & 0  & 0\\
%\end{bmatrix}
%\end{dmath}
%\begin{dmath}\label{eqn:anticommutingOperatorWithSimulaneousEigenket:220}
%B =
%\begin{bmatrix}
%0 & 1 & 0 & 0 \\
%1 & 0 & 0 & 0 \\
%0 & 0 & 0 & 0 \\
%0 & 0 & 0 & 1 \\
%\end{bmatrix}.
%\end{dmath}
%\end{subequations}
%
%The eigenkets for the zero eigenvalues for \( A \) and \( B \) are \( (0,0,0,1) \) and \( (0,0,1,0) \) respectively, but neither of these are a common eigenket.
} % answer

%\EndArticle

         % p17:
         %
% Copyright � 2015 Peeter Joot.  All Rights Reserved.
% Licenced as described in the file LICENSE under the root directory of this GIT repository.
%
%\input{../blogpost.tex}
%\renewcommand{\basename}{angularMomentumAndCentralForceCommutators}
%\renewcommand{\dirname}{notes/phy1520/}
%%\newcommand{\dateintitle}{}
%%\newcommand{\keywords}{}
%
%\input{../peeter_prologue_print2.tex}
%
%\usepackage{peeters_layout_exercise}
%\usepackage{peeters_braket}
%\usepackage{peeters_figures}
%
%\beginArtNoToc
%
%\generatetitle{Commutators of angular momentum and a central force Hamiltonian}
%%\chapter{Commutators of angular momentum and a central force Hamiltonian}
%%\label{chap:angularMomentumAndCentralForceCommutators}

\makeoproblem{Degeneracy in non-commuting observables that both commute with the Hamiltonian.}{problem:angularMomentumAndCentralForceCommutators:1}{\citep{sakurai2014modern} pr. 1.17}{
\index{degeneracy}
\index{non-commuting observables}

Show that non-commuting operators that both commute with the Hamiltonian, have, in general, degenerate energy eigenvalues.  That is

\begin{equation}\label{eqn:angularMomentumAndCentralForceCommutators:320}
[A,H] = [B,H] = 0,
\end{equation}

but

\begin{dmath}\label{eqn:angularMomentumAndCentralForceCommutators:340}
[A,B] \ne 0.
\end{dmath}

\makesubproblem{}{problem:angularMomentumAndCentralForceCommutators:1:a}

Consider \( L_x, L_z \) and a central force Hamiltonian \( H = \Bp^2/2m + V(r) \) as examples.

\makesubproblem{}{problem:angularMomentumAndCentralForceCommutators:1:b}

Construct some simple matrix examples that illustrate the degeneracy conditions.

\makesubproblem{}{problem:angularMomentumAndCentralForceCommutators:1:c}

Prove the general case.

} % problem

\makeanswer{problem:angularMomentumAndCentralForceCommutators:1}{

\makeSubAnswer{}{problem:angularMomentumAndCentralForceCommutators:1:a}

Let's start with demonstrate these commutators act as expected in these cases.

With \( \BL = \Bx \cross \Bp \), we have

\begin{equation}\label{eqn:angularMomentumAndCentralForceCommutators:20}
\begin{aligned}
L_x &= y p_z - z p_y \\
L_y &= z p_x - x p_z \\
L_z &= x p_y - y p_x.
\end{aligned}
\end{equation}

The \( L_x, L_z \) commutator is

\begin{equation}\label{eqn:angularMomentumAndCentralForceCommutators:40}
\begin{aligned}
\antisymmetric{L_x}{L_z}
&=
\antisymmetric{y p_z - z p_y }{x p_y - y p_x} \\
&=
\antisymmetric{y p_z}{x p_y}
-\antisymmetric{y p_z}{y p_x}
-\antisymmetric{z p_y }{x p_y}
+\antisymmetric{z p_y }{y p_x} \\
&=
x p_z \antisymmetric{y}{p_y}
+ z p_x \antisymmetric{p_y }{y} \\
&=
i \Hbar \lr{ x p_z - z p_x } \\
&=
- i \Hbar L_y
\end{aligned}
\end{equation}

cyclically permuting the indexes shows that no pairs of different \( \BL \) components commute.  For \( L_y, L_x \) that is

\begin{equation}\label{eqn:angularMomentumAndCentralForceCommutators:60}
\begin{aligned}
\antisymmetric{L_y}{L_x}
&=
\antisymmetric{z p_x - x p_z }{y p_z - z p_y} \\
&=
\antisymmetric{z p_x}{y p_z}
-\antisymmetric{z p_x}{z p_y}
-\antisymmetric{x p_z }{y p_z}
+\antisymmetric{x p_z }{z p_y} \\
&=
y p_x \antisymmetric{z}{p_z}
+ x p_y \antisymmetric{p_z }{z} \\
&=
i \Hbar \lr{ y p_x - x p_y } \\
&=
- i \Hbar L_z,
\end{aligned}
\end{equation}

and for \( L_z, L_y \)

\begin{equation}\label{eqn:angularMomentumAndCentralForceCommutators:80}
\begin{aligned}
\antisymmetric{L_z}{L_y}
&=
\antisymmetric{x p_y - y p_x }{z p_x - x p_z} \\
&=
\antisymmetric{x p_y}{z p_x}
-\antisymmetric{x p_y}{x p_z}
-\antisymmetric{y p_x }{z p_x}
+\antisymmetric{y p_x }{x p_z} \\
&=
z p_y \antisymmetric{x}{p_x}
+ y p_z \antisymmetric{p_x }{x} \\
&=
i \Hbar \lr{ z p_y - y p_z } \\
&=
- i \Hbar L_x.
\end{aligned}
\end{equation}

If these angular momentum components are also shown to commute with themselves (which they do), the commutator relations above can be summarized as

\index{central force potential}
\begin{equation}\label{eqn:angularMomentumAndCentralForceCommutators:100}
\antisymmetric{L_a}{L_b} = i \Hbar \epsilon_{a b c} L_c.
\end{equation}

In the example to consider, we'll have to consider the commutators with \( \Bp^2 \) and \( V(r) \).  Picking any one component of \( \BL \) is sufficient due to the symmetries of the problem.  For example

\begin{equation}\label{eqn:angularMomentumAndCentralForceCommutators:120}
\begin{aligned}
\antisymmetric{L_x}{\Bp^2}
&=
\antisymmetric{y p_z - z p_y}{p_x^2 + p_y^2 + p_z^2} \\
&=
\antisymmetric{y p_z}{\cancel{p_x^2} + p_y^2 + \cancel{p_z^2}}
-\antisymmetric{z p_y}{\cancel{p_x^2} + \cancel{p_y^2} + p_z^2} \\
&=
p_z \antisymmetric{y}{p_y^2}
-p_y \antisymmetric{z}{p_z^2} \\
&=
p_z 2 i \Hbar p_y
-p_y 2 i \Hbar p_z  \\
&=
0.
\end{aligned}
\end{equation}

How about the commutator of \( \BL \) with the potential?  It is sufficient to consider one component again, for example

\begin{equation}\label{eqn:angularMomentumAndCentralForceCommutators:140}
\begin{aligned}
\antisymmetric{L_x}{V}
&=
\antisymmetric{y p_z - z p_y}{V} \\
&=
y \antisymmetric{p_z}{V} - z \antisymmetric{p_y}{V} \\
&=
-i \Hbar y \PD{z}{V(r)} + i \Hbar z \PD{y}{V(r)} \\
&=
-i \Hbar y \PD{r}{V}\PD{z}{r} + i \Hbar z \PD{r}{V}\PD{y}{r}  \\
&=
-i \Hbar y \PD{r}{V} \frac{z}{r} + i \Hbar z \PD{r}{V}\frac{y}{r}  \\
&=
0.
\end{aligned}
\end{equation}

This has shown that all the components of \( \BL \) commute with a central force Hamiltonian, and each different component of \( \BL \) do not commute.  It does not demonstrate the degeneracy, but I do recall that exists for this system.

%\paragraph{Matrix example of non-commuting commutators}
\makeSubAnswer{}{problem:angularMomentumAndCentralForceCommutators:1:b}

I thought perhaps the problem at hand would be easier if I were to construct some example matrices representing operators that did not commute, but did commuted with a Hamiltonian.  I came up with

\begin{equation}\label{eqn:angularMomentumAndCentralForceCommutators:360}
\begin{aligned}
A &=
\begin{bmatrix}
\sigma_z & 0 \\
0 & 1
\end{bmatrix}
=
\begin{bmatrix}
 1 & 0 & 0 \\
 0 & -1 & 0 \\
 0 & 0 & 1 \\
\end{bmatrix} \\
B &=
\begin{bmatrix}
\sigma_x & 0 \\
0 & 1
\end{bmatrix}
=
\begin{bmatrix}
 0 & 1 & 0 \\
 1 & 0 & 0 \\
 0 & 0 & 1 \\
\end{bmatrix} \\
H &=
\begin{bmatrix}
 0 & 0 & 0 \\
 0 & 0 & 0 \\
 0 & 0 & 1 \\
\end{bmatrix}
\end{aligned}
\end{equation}

This system has \( \antisymmetric{A}{H} = \antisymmetric{B}{H} = 0 \), and

\begin{dmath}\label{eqn:angularMomentumAndCentralForceCommutators:380}
\antisymmetric{A}{B}
=
\begin{bmatrix}
 0 & 2 & 0 \\
-2 & 0 & 0 \\
 0 & 0 & 0 \\
\end{bmatrix}
\end{dmath}

There is one shared eigenvector between all of \( A, B, H \)

\begin{dmath}\label{eqn:angularMomentumAndCentralForceCommutators:400}
\ket{3} =
\begin{bmatrix}
0 \\
0 \\
1
\end{bmatrix}.
\end{dmath}

The other eigenvectors for \( A \) are
\begin{equation}\label{eqn:angularMomentumAndCentralForceCommutators:420}
\begin{aligned}
\ket{a_1} &=
\begin{bmatrix}
1 \\
0 \\
0
\end{bmatrix} \\
\ket{a_2} &=
\begin{bmatrix}
0 \\
1 \\
0
\end{bmatrix},
\end{aligned}
\end{equation}

and for \( B \)
\begin{equation}\label{eqn:angularMomentumAndCentralForceCommutators:440}
\begin{aligned}
\ket{b_1} &=
\inv{\sqrt{2}}
\begin{bmatrix}
1 \\
1 \\
0
\end{bmatrix} \\
\ket{b_2} &=
\inv{\sqrt{2}}
\begin{bmatrix}
1 \\
-1 \\
0
\end{bmatrix},
\end{aligned}
\end{equation}

This clearly has the degeneracy sought.

Looking to \citep{commutingMatrices}, it appears that it is possible to construct an even simpler example.  Let

\begin{equation}\label{eqn:angularMomentumAndCentralForceCommutators:460}
\begin{aligned}
A &=
\begin{bmatrix}
0 & 1 \\
0 & 0
\end{bmatrix} \\
B &=
\begin{bmatrix}
1 & 0 \\
0 & 0
\end{bmatrix} \\
H &=
\begin{bmatrix}
0 & 0 \\
0 & 0
\end{bmatrix}.
\end{aligned}
\end{equation}

Here \( \antisymmetric{A}{B} = -A \), and \( \antisymmetric{A}{H} = \antisymmetric{B}{H} = 0 \), but the Hamiltonian isn't interesting at all physically.

A less boring example builds on this.  Let

\begin{equation}\label{eqn:angularMomentumAndCentralForceCommutators:480}
\begin{aligned}
A &=
\begin{bmatrix}
0 & 1 & 0 \\
0 & 0 & 0 \\
0 & 0 & 1
\end{bmatrix} \\
B &=
\begin{bmatrix}
1 & 0 & 0 \\
0 & 0 & 0 \\
0 & 0 & 1
\end{bmatrix} \\
H &=
\begin{bmatrix}
0 & 0 & 0 \\
0 & 0 & 0 \\
0 & 0 & 1 \\
\end{bmatrix}.
\end{aligned}
\end{equation}

Here \( \antisymmetric{A}{B} \ne 0 \), and \( \antisymmetric{A}{H} = \antisymmetric{B}{H} = 0 \).  I don't see a way for any exception to be constructed.

%\paragraph{The problem}
\makeSubAnswer{}{problem:angularMomentumAndCentralForceCommutators:1:c}

The concrete examples above give some intuition for solving the more abstract problem.  Suppose that we are working in a basis that simultaneously diagonalizes operator \( A \) and the Hamiltonian \( H \).  To make life easy consider the simplest case where this basis is also an eigenbasis for the second operator \( B \) for all but two of that operators eigenvectors.  For such a system let's write

\begin{equation}\label{eqn:angularMomentumAndCentralForceCommutators:160}
\begin{aligned}
H \ket{1} &= \epsilon_1 \ket{1} \\
H \ket{2} &= \epsilon_2 \ket{2} \\
A \ket{1} &= a_1 \ket{1} \\
A \ket{2} &= a_2 \ket{2},
\end{aligned}
\end{equation}

where \( \ket{1}\), and \( \ket{2} \) are not eigenkets of \( B \).  Because \( B \) also commutes with \( H \), we must have

\begin{dmath}\label{eqn:angularMomentumAndCentralForceCommutators:180}
H B \ket{1}
= H \ket{n}\bra{n} B \ket{1}
= \epsilon_n \ket{n} B_{n 1},
\end{dmath}

and
\begin{dmath}\label{eqn:angularMomentumAndCentralForceCommutators:200}
B H \ket{1}
= B \epsilon_1 \ket{1}
= \epsilon_1 \ket{n}\bra{n} B \ket{1}
= \epsilon_1 \ket{n} B_{n 1}.
\end{dmath}

The commutator is
\begin{dmath}\label{eqn:angularMomentumAndCentralForceCommutators:220}
\antisymmetric{B}{H} \ket{1}
=
\lr{ \epsilon_1 - \epsilon_n } \ket{n} B_{n 1}.
\end{dmath}

Similarly
\begin{dmath}\label{eqn:angularMomentumAndCentralForceCommutators:240}
\antisymmetric{B}{H} \ket{2}
=
\lr{ \epsilon_2 - \epsilon_n } \ket{n} B_{n 2}.
\end{dmath}

For those kets \( \ket{m} \in \setlr{ \ket{3}, \ket{4}, \cdots } \) that are eigenkets of \( B \), with \( B \ket{m} = b_m \ket{m} \), we have

\begin{dmath}\label{eqn:angularMomentumAndCentralForceCommutators:280}
\antisymmetric{B}{H} \ket{m}
=
B \epsilon_m \ket{m} - H b_m \ket{m}
=
b_m \epsilon_m \ket{m} - \epsilon_m b_m \ket{m}
=
0.
\end{dmath}

If the commutator is zero, then we require all its matrix elements
\begin{equation}\label{eqn:angularMomentumAndCentralForceCommutators:260}
\begin{aligned}
\bra{1} \antisymmetric{B}{H} \ket{1} &= \lr{ \epsilon_1 - \epsilon_1 } B_{1 1} \\
\bra{2} \antisymmetric{B}{H} \ket{1} &= \lr{ \epsilon_1 - \epsilon_2 } B_{2 1} \\
\bra{1} \antisymmetric{B}{H} \ket{2} &= \lr{ \epsilon_2 - \epsilon_1 } B_{1 2} \\
\bra{2} \antisymmetric{B}{H} \ket{2} &= \lr{ \epsilon_2 - \epsilon_2 } B_{2 2},
\end{aligned}
\end{equation}

to be zero.  Because of \cref{eqn:angularMomentumAndCentralForceCommutators:280} only the matrix elements with respect to states \( \ket{1}, \ket{2} \) need be considered.  Two of the matrix elements above are clearly zero, regardless of the values of \( B_{1 1}\), and \(B_{2 2} \), and for the other two to be zero, we must either have

\begin{itemize}
\item \( B_{2 1} = B_{1 2} = 0 \), or
\item \( \epsilon_1 = \epsilon_2 \).
\end{itemize}

If the first condition were true we would have

\begin{dmath}\label{eqn:angularMomentumAndCentralForceCommutators:300}
B \ket{1}
=
\ket{n}\bra{n} B \ket{1}
=
\ket{n} B_{n 1}
=
\ket{1} B_{1 1},
\end{dmath}

and \( B \ket{2} = B_{2 2} \ket{2} \).  This contradicts the requirement that \( \ket{1}, \ket{2} \) not be eigenkets of \( B \), leaving only the second option.  That second option means there must be a degeneracy in the system.

} % answer

%\EndArticle

         %
% Copyright � 2015 Peeter Joot.  All Rights Reserved.
% Licenced as described in the file LICENSE under the root directory of this GIT repository.
%
%\input{../blogpost.tex}
%\renewcommand{\basename}{moreKet}
%\renewcommand{\dirname}{notes/phy1520/}
%%\newcommand{\dateintitle}{}
%%\newcommand{\keywords}{}
%
%\input{../peeter_prologue_print2.tex}
%
%\usepackage{peeters_layout_exercise}
%\usepackage{peeters_braket}
%\usepackage{peeters_figures}
%
%\beginArtNoToc
%
%\generatetitle{More ket problems}
%\chapter{More ket problems}
%\label{chap:moreKet}

\makeoproblem{Uncertainty relation.}{problem:moreKet:20}{\citep{sakurai2014modern} pr. 1.20}{
\index{uncertainty}
Find the ket that maximizes the uncertainty product
%
\begin{dmath}\label{eqn:moreKet:140}
\expectation{\lr{\Delta S_x}^2}
\expectation{\lr{\Delta S_y}^2},
\end{dmath}

and compare to the uncertainty bound \( \inv{4} \Abs{ \expectation{\antisymmetric{S_x}{S_y}}}^2 \).

} % problem

\makeanswer{problem:moreKet:20}{

To parameterize the ket space, consider first the kets that where both components are both not zero, where a single complex number can parameterize the ket
%
\begin{dmath}\label{eqn:moreKet:160}
\ket{s} =
\begin{bmatrix}
\beta' e^{i\phi'} \\
\alpha' e^{i\theta'} \\
\end{bmatrix}
\propto
\begin{bmatrix}
1 \\
\alpha e^{i\theta} \\
\end{bmatrix}
\end{dmath}

The expectation values with respect to this ket are
\begin{dmath}\label{eqn:moreKet:180}
\expectation{S_x} =
\frac{\Hbar}{2}
\begin{bmatrix}
1 & \alpha e^{-i\theta} \\
\end{bmatrix}
\PauliX
\begin{bmatrix}
1 \\
\alpha e^{i\theta} \\
\end{bmatrix}
=
\frac{\Hbar}{2}
\begin{bmatrix}
1 &
\alpha e^{-i\theta} \\
\end{bmatrix}
\begin{bmatrix}
\alpha e^{i\theta} \\
1 \\
\end{bmatrix}
=
\frac{\Hbar}{2}
\alpha e^{i\theta} + \alpha e^{-i\theta}
=
\frac{\Hbar}{2}
2 \alpha \cos\theta
=
\Hbar \alpha \cos\theta.
\end{dmath}
%
\begin{dmath}\label{eqn:moreKet:200}
\expectation{S_y} =
\frac{\Hbar}{2}
\begin{bmatrix}
1 & \alpha e^{-i\theta} \\
\end{bmatrix}
\PauliY
\begin{bmatrix}
1 \\
\alpha e^{i\theta} \\
\end{bmatrix}
=
\frac{i\Hbar}{2}
\begin{bmatrix}
1 & \alpha e^{-i\theta} \\
\end{bmatrix}
\begin{bmatrix}
-\alpha e^{i\theta} \\
1 \\
\end{bmatrix}
=
\frac{-i \alpha \Hbar}{2} 2 i \sin\theta
=
\alpha \Hbar \sin\theta.
\end{dmath}

The variances are
\begin{dmath}\label{eqn:moreKet:220}
\lr{ \Delta S_x }^2
=
\lr{
\frac{\Hbar}{2}
\begin{bmatrix}
-2 \alpha \cos\theta & 1 \\
1 & -2 \alpha \cos\theta \\
\end{bmatrix}
}^2
=
\frac{\Hbar^2}{4}
\begin{bmatrix}
-2 \alpha \cos\theta & 1 \\
1 & -2 \alpha \cos\theta \\
\end{bmatrix}
\begin{bmatrix}
-2 \alpha \cos\theta & 1 \\
1 & -2 \alpha \cos\theta \\
\end{bmatrix}
=
\frac{\Hbar^2}{4}
\begin{bmatrix}
4 \alpha^2 \cos^2\theta + 1 & -4 \alpha \cos\theta \\
-4 \alpha \cos\theta & 4 \alpha^2 \cos^2\theta + 1 \\
\end{bmatrix},
\end{dmath}

and
%
\begin{dmath}\label{eqn:moreKet:240}
\lr{ \Delta S_y }^2
=
\lr{
\frac{\Hbar}{2}
\begin{bmatrix}
-2 \alpha \sin\theta & -i \\
i & -2 \alpha \sin\theta \\
\end{bmatrix}
}^2
=
\frac{\Hbar^2}{4}
\begin{bmatrix}
-2 \alpha \sin\theta & -i \\
i & -2 \alpha \sin\theta \\
\end{bmatrix}
\begin{bmatrix}
-2 \alpha \sin\theta & -i \\
i & -2 \alpha \sin\theta \\
\end{bmatrix}
=
\frac{\Hbar^2}{4}
\begin{bmatrix}
4 \alpha^2 \sin^2\theta + 1 & 4 \alpha i \sin\theta \\
-4 \alpha i \sin\theta & 4 \alpha^2 \sin^2\theta + 1 \\
\end{bmatrix}.
\end{dmath}

The uncertainty factors are
%
\begin{dmath}\label{eqn:moreKet:260}
\expectation{\lr{\Delta S_x}^2}
=
\frac{\Hbar^2}{4}
\begin{bmatrix}
1 & \alpha e^{-i\theta}
\end{bmatrix}
\begin{bmatrix}
4 \alpha^2 \cos^2\theta + 1 & -4 \alpha \cos\theta \\
-4 \alpha \cos\theta & 4 \alpha^2 \cos^2\theta + 1 \\
\end{bmatrix}
\begin{bmatrix}
1 \\
\alpha e^{i\theta}
\end{bmatrix}
=
\frac{\Hbar^2}{4}
\begin{bmatrix}
1 & \alpha e^{-i\theta}
\end{bmatrix}
\begin{bmatrix}
4 \alpha^2 \cos^2\theta + 1 -4 \alpha^2 \cos\theta e^{i\theta} \\
-4 \alpha \cos\theta + 4 \alpha^3 \cos^2\theta e^{i\theta} + \alpha e^{i\theta} \\
\end{bmatrix}
=
\frac{\Hbar^2}{4}
\lr{
4 \alpha^2 \cos^2\theta + 1 -4 \alpha^2 \cos\theta e^{i\theta}
-4 \alpha^2 \cos\theta e^{-i\theta} + 4 \alpha^4 \cos^2\theta + \alpha^2
}
=
\frac{\Hbar^2}{4}
\lr{
4 \alpha^2 \cos^2\theta + 1 -8 \alpha^2 \cos^2\theta
+ 4 \alpha^4 \cos^2\theta + \alpha^2
}
=
\frac{\Hbar^2}{4}
\lr{
-4 \alpha^2 \cos^2\theta + 1
+ 4 \alpha^4 \cos^2\theta + \alpha^2
}
=
\frac{\Hbar^2}{4}
\lr{
4 \alpha^2 \cos^2\theta \lr{ \alpha^2 - 1 }
+ \alpha^2 + 1
}
,
\end{dmath}

and
%
\begin{dmath}\label{eqn:moreKet:280}
\expectation{ \lr{ \Delta S_y }^2 }
=
\frac{\Hbar^2}{4}
\begin{bmatrix}
1 & \alpha e^{-i\theta}
\end{bmatrix}
\begin{bmatrix}
4 \alpha^2 \sin^2\theta + 1 & 4 \alpha i \sin\theta \\
-4 \alpha i \sin\theta & 4 \alpha^2 \sin^2\theta + 1 \\
\end{bmatrix}
\begin{bmatrix}
1 \\
\alpha e^{i\theta}
\end{bmatrix}
=
\frac{\Hbar^2}{4}
\begin{bmatrix}
1 & \alpha e^{-i\theta}
\end{bmatrix}
\begin{bmatrix}
4 \alpha^2 \sin^2\theta + 1 + 4 \alpha^2 i \sin\theta e^{i\theta} \\
-4 \alpha i \sin\theta + 4 \alpha^3 \sin^2\theta e^{i\theta} + \alpha e^{i\theta} \\
\end{bmatrix}
=
\frac{\Hbar^2}{4}
\lr{
4 \alpha^2 \sin^2\theta + 1 + 4 \alpha^2 i \sin\theta e^{i\theta}
-4 \alpha^2 i \sin\theta e^{-i\theta} + 4 \alpha^4 \sin^2\theta + \alpha^2
}
=
\frac{\Hbar^2}{4}
\lr{
-4 \alpha^2 \sin^2\theta + 1
+ 4 \alpha^4 \sin^2\theta + \alpha^2
}
=
\frac{\Hbar^2}{4}
\lr{
4 \alpha^2 \sin^2\theta \lr{ \alpha^2 - 1}
+ \alpha^2
+ 1
}
.
\end{dmath}

The uncertainty product can finally be calculated
%
\begin{dmath}\label{eqn:moreKet:300}
\expectation{\lr{\Delta S_x}^2}
\expectation{\lr{\Delta S_y}^2}
=
\lr{\frac{\Hbar}{2} }^4
\lr{
4 \alpha^2 \cos^2\theta \lr{ \alpha^2 - 1 }
+ \alpha^2 + 1
}
\lr{
4 \alpha^2 \sin^2\theta \lr{ \alpha^2 - 1}
+ \alpha^2
+ 1
}
=
\lr{\frac{\Hbar}{2} }^4
\lr{
4 \alpha^4 \sin^2 \lr{ 2\theta } \lr{ \alpha^2 - 1 }
+ 4 \alpha^2 \lr{ \alpha^4 - 1 }
+ \lr{\alpha^2 + 1 }^2
}
\end{dmath}

The maximum occurs when \( f = \sin^2 2 \theta \) is extremized.  Those points are
\begin{dmath}\label{eqn:moreKet:320}
0
= \PD{\theta}{f}
= 2 \sin 2 \theta \cos 2\theta
= 4 \sin 4 \theta.
\end{dmath}

Those points are at \( 4 \theta = \pi n \), for integer \( n \), or
%
\begin{equation}\label{eqn:moreKet:340}
\theta = \frac{\pi}{4} n, n \in [0, 7],
\end{equation}

Minimums will occur when
%
\begin{dmath}\label{eqn:moreKet:360}
0 < \PDSq{\theta}{f} = 8 \cos 4\theta,
\end{dmath}

or
\begin{dmath}\label{eqn:moreKet:380}
n = 0, 2, 4, 6.
\end{dmath}

At these points \( \sin^2 2\theta \) takes the values
%
\begin{dmath}\label{eqn:moreKet:400}
\sin^2 \lr{ 2 \frac{\pi}{4} \setlr{ 0, 2, 4, 6 } }
=
\sin^2 \lr{ \pi \setlr{ 0, 1, 2, 3 } }
\in \setlr{ 0 },
\end{dmath}

so the maximization of the uncertainty product can be reduced to that of
%
\begin{dmath}\label{eqn:moreKet:420}
\expectation{\lr{\Delta S_x}^2}
\expectation{\lr{\Delta S_y}^2}
=
\lr{\frac{\Hbar}{2} }^4
\lr{
4 \alpha^2 \lr{ \alpha^4 - 1 }
+ \lr{\alpha^2 + 1 }^2
}
\end{dmath}

We seek
%
\begin{dmath}\label{eqn:moreKet:440}
0 =
\PD{\alpha}{}
\lr{
4 \alpha^2 \lr{ \alpha^4 - 1 }
+ \lr{\alpha^2 + 1 }^2
}
=
\lr{
8 \alpha \lr{ \alpha^4 - 1 }
+16 \alpha^5
+ 4 \lr{\alpha^2 + 1 } \alpha
}
=
4 \alpha
\lr{
2 \alpha^4 - 2
+4 \alpha^4
+ 4 \alpha^2 + 4
}
=
8 \alpha
\lr{
3 \alpha^4 + 2 \alpha^2 + 1
}.
\end{dmath}

The only real root of this polynomial is \( \alpha = 0 \), so the ket where both \( \ket{+} \) and \( \ket{-} \) are not zero that maximizes the uncertainty product is
%
\begin{dmath}\label{eqn:moreKet:460}
\ket{s} =
\begin{bmatrix}
1 \\
0
\end{bmatrix}
= \ket{+}.
\end{dmath}

The search for this maximizing value excluded those kets proportional to \( \begin{bmatrix} 0 \\ 1 \end{bmatrix} = \ket{-} \).  Let's see the values of this uncertainty product at both \( \ket{\pm} \), and compare to the uncertainty commutator.  First \( \ket{s} = \ket{+} \)
%
\begin{dmath}\label{eqn:moreKet:480}
\expectation{S_x}
=
\begin{bmatrix}
1 & 0
\end{bmatrix}
\PauliX
\begin{bmatrix}
1 \\ 0
\end{bmatrix}
=
0.
\end{dmath}
%
\begin{dmath}\label{eqn:moreKet:500}
\expectation{S_y}
=
\begin{bmatrix}
1 & 0
\end{bmatrix}
\PauliY
\begin{bmatrix}
1 \\ 0
\end{bmatrix}
=
0.
\end{dmath}

so
%
\begin{dmath}\label{eqn:moreKet:520}
\expectation{ \lr{ \Delta S_x }^2 }
=
\lr{\frac{\Hbar}{2}}^2
\begin{bmatrix}
1 & 0
\end{bmatrix}
\begin{bmatrix}
1 \\ 0
\end{bmatrix}
=
\lr{\frac{\Hbar}{2}}^2
\end{dmath}
%
\begin{dmath}\label{eqn:moreKet:540}
\expectation{ \lr{ \Delta S_y }^2 }
=
\lr{\frac{\Hbar}{2}}^2
\begin{bmatrix}
1 & 0
\end{bmatrix}
\begin{bmatrix}
1 \\ 0
\end{bmatrix}
=
\lr{\frac{\Hbar}{2}}^2.
\end{dmath}

For the commutator side of the uncertainty relation we have
%
\begin{dmath}\label{eqn:moreKet:560}
\inv{4} \Abs{ \expectation{ \antisymmetric{ S_x}{ S_y } } }^2
=
\inv{4} \Abs{ \expectation{ i \hbar S_z } }^2
=
\lr{ \frac{\Hbar}{2} }^4
\Abs{
\begin{bmatrix}
1 & 0
\end{bmatrix}
\PauliZ
\begin{bmatrix}
1 \\ 0
\end{bmatrix}
}^2,
\end{dmath}

so for the \( \ket{+} \) state we have an equality condition for the uncertainty relation
%
\begin{dmath}\label{eqn:moreKet:580}
\expectation{\lr{\Delta S_x}^2}
\expectation{\lr{\Delta S_y}^2}
=
\inv{4} \Abs{ \expectation{\antisymmetric{S_x}{S_y}}}^2
=
\lr{ \frac{\Hbar}{2} }^4.
\end{dmath}

It's reasonable to guess that the \( \ket{-} \) state also matches the equality condition.  Let's check
\begin{dmath}\label{eqn:moreKet:600}
\expectation{S_x}
=
\begin{bmatrix}
0 & 1
\end{bmatrix}
\PauliX
\begin{bmatrix}
0 \\ 1
\end{bmatrix}
=
0.
\end{dmath}
%
\begin{dmath}\label{eqn:moreKet:620}
\expectation{S_y}
=
\begin{bmatrix}
0 & 1
\end{bmatrix}
\PauliY
\begin{bmatrix}
0 \\ 1
\end{bmatrix}
=
0.
\end{dmath}

so \( \expectation{ \lr{ \Delta S_x }^2 } = \expectation{ \lr{ \Delta S_y }^2 } = \lr{\frac{\Hbar}{2}}^2 \).

For the commutator side of the uncertainty relation will be identical, so the equality of \cref{eqn:moreKet:580} is satisfied for both \( \ket{\pm} \).  Note that it wasn't explicitly verified that \( \ket{-} \) maximized the uncertainty product, but I don't feel like working through that second set of algebraic mess.

We can see by example that equality does not mean that the equality condition means that the product is maximized.  For example, it is straightforward to show that \( \ket{ S_x ; \pm } \) also satisfy the equality condition of the uncertainty relation.  However, in that case the product is not maximized, but is zero.

} % answer

\makeoproblem{Degenerate ket space example.}{problem:moreKet:23}{\citep{sakurai2014modern} pr. 1.23}{

\index{degeneracy}
\index{simultaneous eigenstate}
Consider operators with representation
%
\begin{equation}\label{eqn:moreKet:20}
A =
\begin{bmatrix}
a & 0 & 0 \\
0 & -a & 0 \\
0 & 0 & -a
\end{bmatrix}
,
\qquad
B =
\begin{bmatrix}
b & 0 & 0 \\
0 & 0 & -ib \\
0 & ib & 0
\end{bmatrix}.
\end{equation}

Show that these both have degeneracies, commute, and compute a simultaneous ket space for both operators.

} % problem

\makeanswer{problem:moreKet:23}{
The eigenvalues and eigenvectors for \( A \) can be read off by inspection, with values of \( a, -a, -a \), and kets
%
\begin{equation}\label{eqn:moreKet:40}
\ket{a_1} =
\begin{bmatrix}
1 \\
0 \\
0
\end{bmatrix},
\ket{a_2} =
\begin{bmatrix}
0 \\
1 \\
0
\end{bmatrix},
\ket{a_3} =
\begin{bmatrix}
0 \\
0 \\
1 \\
\end{bmatrix}
\end{equation}

Notice that the lower-right \( 2 \times 2 \) submatrix of \( B \) is proportional to \( \sigma_y \), so it's eigenvalues can be formed by inspection
%
\begin{equation}\label{eqn:moreKet:60}
\ket{b_1} =
\begin{bmatrix}
1 \\
0 \\
0
\end{bmatrix},
\ket{b_2} =
\inv{\sqrt{2}}
\begin{bmatrix}
0 \\
1 \\
i
\end{bmatrix},
\ket{b_3} =
\inv{\sqrt{2}}
\begin{bmatrix}
0 \\
1 \\
-i \\
\end{bmatrix}.
\end{equation}

Computing \( B \ket{b_i} \) shows that the eigenvalues are \( b, b, -b \) respectively.

Because of the two-fold degeneracy in the \( -a \) eigenvalues of \( A \), any linear combination of \( \ket{a_2}, \ket{a_3} \) will also be an eigenket.   In particular,
%
\begin{equation}\label{eqn:moreKet:80}
\begin{aligned}
\ket{a_2} + i \ket{a_3} &= \ket{b_2} \\
\ket{a_2} - i \ket{a_3} &= \ket{b_3},
\end{aligned}
\end{equation}

so the basis \( \setlr{ \ket{b_i}} \) is a simultaneous eigenspace for both \( A \) and \(B\).  Because there is a simultaneous eigenspace, the matrices must commute.  This can be confirmed with direct computation
%
\begin{dmath}\label{eqn:moreKet:100}
A B = a b
\begin{bmatrix}
1 & 0 & 0 \\
0 & -1 & 0 \\
0 & 0 & -1
\end{bmatrix}
\begin{bmatrix}
1 & 0 & 0 \\
0 & 0 & -i \\
0 & i & 0
\end{bmatrix}
=
a b
\begin{bmatrix}
1 & 0 & 0 \\
0 & 0 & i \\
0 & -i & 0
\end{bmatrix},
\end{dmath}

and
%
\begin{dmath}\label{eqn:moreKet:120}
B A = a b
\begin{bmatrix}
1 & 0 & 0 \\
0 & 0 & -i \\
0 & i & 0
\end{bmatrix}
\begin{bmatrix}
1 & 0 & 0 \\
0 & -1 & 0 \\
0 & 0 & -1
\end{bmatrix}
=
a b
\begin{bmatrix}
1 & 0 & 0 \\
0 & 0 & i \\
0 & -i & 0
\end{bmatrix}.
\end{dmath}

} % answer

\makeoproblem{Unitary transformation.}{problem:moreKet:26}{\citep{sakurai2014modern} pr. 1.26}{
\index{unitary transformation}

Construct the transformation matrix that maps between the \( S_z \) diagonal basis, to the \( S_x \) diagonal basis.
} % problem

\makeanswer{problem:moreKet:26}{

Based on the definition
%
\begin{dmath}\label{eqn:moreKet:640}
U \ket{a^{(r)}} = \ket{b^{(r)}},
\end{dmath}

the matrix elements can be computed
%
\begin{dmath}\label{eqn:moreKet:660}
\bra{a^{(s)}} U \ket{a^{(r)}} = \braket{a^{(s)}}{b^{(r)}},
\end{dmath}

that is
%
\begin{dmath}\label{eqn:moreKet:680}
U =
\begin{bmatrix}
\bra{a^{(1)}} U \ket{a^{(1)}} & \bra{a^{(1)}} U \ket{a^{(2)}} \\
\bra{a^{(2)}} U \ket{a^{(1)}} & \bra{a^{(2)}} U \ket{a^{(2)}}
\end{bmatrix}
=
\begin{bmatrix}
\braket{a^{(1)}}{b^{(1)}} & \braket{a^{(1)}}{b^{(2)}} \\
\braket{a^{(2)}}{b^{(1)}} & \braket{a^{(2)}}{b^{(2)}}
\end{bmatrix}
=
\inv{\sqrt{2}}
\begin{bmatrix}
	\begin{bmatrix}
	1 & 0
	\end{bmatrix}
	\begin{bmatrix}
	1 \\ 1
	\end{bmatrix} &
	\begin{bmatrix}
	1 & 0
	\end{bmatrix}
	\begin{bmatrix}
	1 \\ -1
	\end{bmatrix} \\
	\begin{bmatrix}
	0 & 1
	\end{bmatrix}
	\begin{bmatrix}
	1 \\ 1
	\end{bmatrix} &
	\begin{bmatrix}
	0 & 1
	\end{bmatrix}
	\begin{bmatrix}
	1 \\ -1
	\end{bmatrix} \\
\end{bmatrix}
=
\inv{\sqrt{2}}
\begin{bmatrix}
1 & 1 \\
1 & -1
\end{bmatrix}.
\end{dmath}

\index{similarity transformation}
As a similarity transformation, we have
%
\begin{dmath}\label{eqn:moreKet:700}
\bra{b^{(r)}} S_z \ket{b^{(s)}}
=
\braket{b^{(r)}}{a^{(t)}}\bra{a^{(t)}} S_z \ket{a^{(u)}}\braket{a^{(u)}}{b^{(s)}}
=
\braket{a^{(r)}}U^\dagger {a^{(t)}}\bra{a^{(t)}} S_z \ket{a^{(u)}}\bra{a^{(u)}}U \ket{a^{(s)}},
\end{dmath}

or
%
\begin{dmath}\label{eqn:moreKet:720}
S_z' = U^\dagger S_z U.
\end{dmath}

Let's check that the computed similarity transformation does it's job.
\begin{dmath}\label{eqn:moreKet:740}
\sigma_z'
= U^\dagger \sigma_z U
= \inv{2}
\begin{bmatrix}
1 & 1 \\
1 & -1
\end{bmatrix}
\begin{bmatrix}
1 & 0 \\
0 & -1
\end{bmatrix}
\begin{bmatrix}
1 & 1 \\
1 & -1
\end{bmatrix}
=
\inv{2}
\begin{bmatrix}
1 & -1 \\
1 & 1
\end{bmatrix}
\begin{bmatrix}
1 & 1 \\
1 & -1
\end{bmatrix}
=
\inv{2}
\begin{bmatrix}
0 & 2 \\
2 & 0
\end{bmatrix}
= \sigma_x.
\end{dmath}

The transformation matrix can also be computed more directly
%
\begin{dmath}\label{eqn:moreKet:760}
U
= U \ket{a^{(r)}} \bra{a^{(r)}}
= \ket{b^{(r)}}\bra{a^{(r)}}
=
\inv{\sqrt{2}}
\begin{bmatrix}
1 \\
1
\end{bmatrix}
\begin{bmatrix}
1 & 0
\end{bmatrix}
+
\inv{\sqrt{2}}
\begin{bmatrix}
1 \\
-1
\end{bmatrix}
\begin{bmatrix}
0 & 1
\end{bmatrix}
=
\inv{\sqrt{2}}
\begin{bmatrix}
1 & 0 \\
1 & 0
\end{bmatrix}
+
\inv{\sqrt{2}}
\begin{bmatrix}
0 & 1 \\
0 & -1
\end{bmatrix}
=
\inv{\sqrt{2}}
\begin{bmatrix}
1 & 1 \\
1 & -1
\end{bmatrix}.
\end{dmath}

} % answer

%\EndArticle

         %
% Copyright � 2015 Peeter Joot.  All Rights Reserved.
% Licenced as described in the file LICENSE under the root directory of this GIT repository.
%
%\input{../blogpost.tex}
%\renewcommand{\basename}{translation}
%\renewcommand{\dirname}{notes/phy1520/}
%%\newcommand{\dateintitle}{}
%%\newcommand{\keywords}{}
%
%\input{../peeter_prologue_print2.tex}
%
%\usepackage{peeters_layout_exercise}
%\usepackage{peeters_braket}
%\usepackage{peeters_figures}
%\usepackage{peeters_qed}
%
%\beginArtNoToc
%
%\generatetitle{Translation operator problems}
%\chapter{Translation operator problems}
%\label{chap:translation}

%
\makeoproblem{One dimensional translation operator.}{problem:translation:28}{\citep{sakurai2014modern} pr. 1.28}{
\index{translation!operator}
%
\makesubproblem{}{problem:translation:28:a}
%
Evaluate the classical Poisson bracket
\index{Poisson bracket}
%
\begin{equation}\label{eqn:translation:420}
\antisymmetric{x}{F(p)}_{\textrm{classical}}
\end{equation}
%
\makesubproblem{}{problem:translation:28:b}
Evaluate the commutator
%
\begin{equation}\label{eqn:translation:440}
\antisymmetric{x}{e^{i p a/\Hbar}}
\end{equation}
%
\makesubproblem{}{problem:translation:28:c}
%
Using the result in \ref{problem:translation:28:b}, prove that
\begin{equation}\label{eqn:translation:460}
e^{i p a/\Hbar} \ket{x'},
\end{equation}
%
is an eigenstate of the coordinate operator \( x \).
} % problem
%
\makeanswer{problem:translation:28}{
%
\makeSubAnswer{}{problem:translation:28:a}
%
\begin{dmath}\label{eqn:translation:480}
\antisymmetric{x}{F(p)}_{\textrm{classical}}
=
\PD{x}{x} \PD{p}{F(p)} - \PD{p}{x} \PD{x}{F(p)}
=
\PD{p}{F(p)}.
\end{dmath}
%
\makeSubAnswer{}{problem:translation:28:b}
%
Having worked backwards through these problems, the answer for this one dimensional problem can be obtained from \cref{eqn:translation:140} and is
%
\begin{dmath}\label{eqn:translation:500}
\antisymmetric{x}{e^{i p a/\Hbar}} = a e^{i p a/\Hbar}.
\end{dmath}
%
\makeSubAnswer{}{problem:translation:28:c}
%
\begin{equation}\label{eqn:translation:520}
\begin{aligned}
x e^{i p a/\Hbar} \ket{x'}
&=
\lr{
\antisymmetric{x}{e^{i p a/\Hbar}}
e^{i p a/\Hbar} x
+
}
\ket{x'} \\
&=
\lr{ a e^{i p a/\Hbar} + e^{i p a/\Hbar} x ' } \ket{x'} \\
&= \lr{ a + x' } \ket{x'}.
\end{aligned}
\end{equation}
%
This demonstrates that \( e^{i p a/\Hbar} \ket{x'} \) is an eigenstate of \( x \) with eigenvalue \( a + x' \).
} % answer
%
\makeoproblem{Polynomial commutators.}{problem:translation:29}{\citep{sakurai2014modern} pr. 1.29}{
\index{commutator!polynomial functions}
%
\makesubproblem{}{problem:translation:29:a}
For power series \( F, G \), verify
%
\begin{equation}\label{eqn:translation:180}
\antisymmetric{x_k}{G(\Bp)} = i \Hbar \PD{p_k}{G}, \qquad
\antisymmetric{p_k}{F(\Bx)} = -i \Hbar \PD{x_k}{F}.
\end{equation}
%
\makesubproblem{}{problem:translation:29:b}
%
Evaluate \( \antisymmetric{x^2}{p^2} \), and compare to the classical Poisson bracket \( \antisymmetric{x^2}{p^2}_{\textrm{classical}} \).

} % problem
%
\makeanswer{problem:translation:29}{
\makeSubAnswer{}{problem:translation:29:a}
%
Let
%
\begin{equation}\label{eqn:translation:200}
\begin{aligned}
G(\Bp) &= \sum_{k l m} a_{k l m} p_1^k p_2^l p_3^m \\
F(\Bx) &= \sum_{k l m} b_{k l m} x_1^k x_2^l x_3^m.
\end{aligned}
\end{equation}
%
It is simpler to work with a specific \( x_k \), say \( x_k = y \).  The validity of the general result will still be clear doing so.  Expanding the commutator gives
%
\begin{dmath}\label{eqn:translation:220}
\antisymmetric{y}{G(\Bp)}
=
\sum_{k l m} a_{k l m} \antisymmetric{y}{p_1^k p_2^l p_3^m }
=
\sum_{k l m} a_{k l m} \lr{
y p_1^k p_2^l p_3^m  - p_1^k p_2^l p_3^m  y
}
=
\sum_{k l m} a_{k l m} \lr{
p_1^k y p_2^l p_3^m  - p_1^k y p_2^l p_3^m
}
=
\sum_{k l m} a_{k l m}
p_1^k
\antisymmetric{y}{p_2^l}
p_3^m.
\end{dmath}
%
From \cref{eqn:translation:100}, we have \( \antisymmetric{y}{p_2^l} = l i \Hbar p_2^{l-1} \), so
%
\begin{dmath}\label{eqn:translation:240}
\antisymmetric{y}{G(\Bp)}
=
\sum_{k l m} a_{k l m}
p_1^k
\antisymmetric{y}{p_2^l}
\lr{ l
i \Hbar p_2^{l-1}
}
p_3^m
=
i \Hbar \PD{y}{G(\Bp)}.
\end{dmath}
%
It is straightforward to show that
\( \antisymmetric{p}{x^l} = -l i \Hbar x^{l-1} \), allowing for a similar computation of the momentum commutator
%
\begin{dmath}\label{eqn:translation:260}
\antisymmetric{p_y}{F(\Bx)}
=
\sum_{k l m} b_{k l m} \antisymmetric{p_y}{x_1^k x_2^l x_3^m }
=
\sum_{k l m} b_{k l m} \lr{
p_y x_1^k x_2^l x_3^m  - x_1^k x_2^l x_3^m  p_y
}
=
\sum_{k l m} b_{k l m} \lr{
x_1^k p_y x_2^l x_3^m  - x_1^k p_y x_2^l x_3^m
}
=
\sum_{k l m} b_{k l m}
x_1^k
\antisymmetric{p_y}{x_2^l}
x_3^m
=
\sum_{k l m} b_{k l m}
x_1^k
\lr{ -l i \Hbar x_2^{l-1}}
x_3^m
=
-i \Hbar \PD{p_y}{F(\Bx)}.
\end{dmath}
%
\makeSubAnswer{}{problem:translation:29:b}
%
It isn't clear to me how the results above can be used directly to compute \( \antisymmetric{x^2}{p^2} \).  However, when the first term of such a commutator is a mononomial, it can be expanded in terms of an \( x \) commutator
%
\begin{dmath}\label{eqn:translation:280}
\antisymmetric{x^2}{G(\Bp)}
= x^2 G - G x^2
= x \lr{ x G } - G x^2
= x \lr{ \antisymmetric{ x }{ G } + G x } - G x^2
= x \antisymmetric{ x }{ G } + \lr{ x G } x - G x^2
= x \antisymmetric{ x }{ G } + \lr{ \antisymmetric{ x }{ G } + \cancel{G x} } x - \cancel{G x^2}
= x \antisymmetric{ x }{ G } + \antisymmetric{ x }{ G } x.
\end{dmath}
%
Similarly,
%
\begin{dmath}\label{eqn:translation:300}
\antisymmetric{x^3}{G(\Bp)} = x^2 \antisymmetric{ x }{ G } + x \antisymmetric{ x }{ G } x + \antisymmetric{ x }{ G } x^2.
\end{dmath}
%
An induction hypothesis can be formed
%
\begin{dmath}\label{eqn:translation:320}
\antisymmetric{x^k}{G(\Bp)} = \sum_{j = 0}^{k-1} x^{k-1-j} \antisymmetric{ x }{ G } x^j,
\end{dmath}
%
and demonstrated
%
\begin{dmath}\label{eqn:translation:340}
\antisymmetric{x^{k+1}}{G(\Bp)}
=
x^{k+1} G - G x^{k+1}
=
x \lr{ x^{k} G } - G x^{k+1}
=
x \lr{ \antisymmetric{x^{k}}{G} + G x^k } - G x^{k+1}
=
x \antisymmetric{x^{k}}{G} + \lr{ x G } x^k - G x^{k+1}
=
x \antisymmetric{x^{k}}{G} + \lr{ \antisymmetric{x}{G} + G x } x^k - G x^{k+1}
=
x \antisymmetric{x^{k}}{G} + \antisymmetric{x}{G} x^k
=
x \sum_{j = 0}^{k-1} x^{k-1-j} \antisymmetric{ x }{ G } x^j + \antisymmetric{x}{G} x^k
=
\sum_{j = 0}^{k-1} x^{(k+1)-1-j} \antisymmetric{ x }{ G } x^j + \antisymmetric{x}{G} x^k
=
\sum_{j = 0}^{k} x^{(k+1)-1-j} \antisymmetric{ x }{ G } x^j. \qedmarker
\end{dmath}

That was a bit overkill for this problem, but may be useful later.  Application of this to the problem gives
%
\begin{dmath}\label{eqn:translation:360}
\antisymmetric{x^2}{p^2}
=
x \antisymmetric{x}{p^2}
+ \antisymmetric{x}{p^2} x
=
x i \Hbar \PD{x}{p^2}
+ i \Hbar \PD{x}{p^2} x
=
x 2 i \Hbar p
+ 2 i \Hbar p x
= i \Hbar \lr{ 2 x p + 2 p x }.
\end{dmath}
%
The classical commutator is
\begin{dmath}\label{eqn:translation:380}
\antisymmetric{x^2}{p^2}_{\textrm{classical}}
=
\PD{x}{x^2} \PD{p}{p^2} - \PD{p}{x^2} \PD{x}{p^2}
=
2 x 2 p
= 2 x p + 2 p x.
\end{dmath}
%
This demonstrates the expected relation between the classical and quantum commutators
%
\begin{equation}\label{eqn:translation:400}
\antisymmetric{x^2}{p^2} = i \Hbar \antisymmetric{x^2}{p^2}_{\textrm{classical}}.
\end{equation}
%
} % answer
%
\makeoproblem{Translation op., position expectation.}{problem:translation:30}{\citep{sakurai2014modern} pr. 1.30}{
\index{translation!expectation}

The translation operator for a finite spatial displacement is given by
%
\begin{dmath}\label{eqn:translation:20}
\mathcal{T}(\Bl) = \exp\lr{ -i \Bp \cdot \Bl/\Hbar },
\end{dmath}
%
where \( \Bp \) is the momentum operator.
%
\makesubproblem{}{problem:translation:1.30:a}
%
Evaluate
%
\begin{dmath}\label{eqn:translation:40}
\antisymmetric{x_i}{\mathcal{T}(\Bl)}.
\end{dmath}
%
\makesubproblem{}{problem:translation:1.30:b}
Demonstrate how the expectation value \( \expectation{\Bx} \) changes under translation.
} % problem
%
\makeanswer{problem:translation:30}{
%
\makeSubAnswer{}{problem:translation:1.30:a}
%
For clarity, let's set \( x_i = y \).  The general result will be clear despite doing so.
%
\begin{dmath}\label{eqn:translation:60}
\antisymmetric{y}{\mathcal{T}(\Bl)}
=
\sum_{k= 0} \inv{k!} \lr{\frac{-i}{\Hbar}}
\antisymmetric{y}{
\lr{ \Bp \cdot \Bl }^k
}.
\end{dmath}
%
The commutator expands as
%
\begin{dmath}\label{eqn:translation:80}
\antisymmetric{y}{
\lr{ \Bp \cdot \Bl }^k
}
+ \lr{ \Bp \cdot \Bl }^k y
=
y \lr{ \Bp \cdot \Bl }^k
=
y \lr{ p_x l_x + p_y l_y + p_z l_z } \lr{ \Bp \cdot \Bl }^{k-1}
=
\lr{ p_x l_x y + y p_y l_y + p_z l_z y } \lr{ \Bp \cdot \Bl }^{k-1}
=
\lr{ p_x l_x y + l_y \lr{ p_y y + i \Hbar } + p_z l_z y } \lr{ \Bp \cdot \Bl }^{k-1}
=
\lr{ \Bp \cdot \Bl } y \lr{ \Bp \cdot \Bl }^{k-1}
+ i \Hbar l_y \lr{ \Bp \cdot \Bl }^{k-1}
= \cdots
=
\lr{ \Bp \cdot \Bl }^{k-1} y \lr{ \Bp \cdot \Bl }^{k-(k-1)}
+ (k-1) i \Hbar l_y \lr{ \Bp \cdot \Bl }^{k-1}
=
\lr{ \Bp \cdot \Bl }^{k} y
+ k i \Hbar l_y \lr{ \Bp \cdot \Bl }^{k-1}.
\end{dmath}
%
In the above expansion, the commutation of \( y \) with \( p_x, p_z \) has been used.  This gives, for \( k \ne 0 \),
%
\begin{dmath}\label{eqn:translation:100}
\antisymmetric{y}{
\lr{ \Bp \cdot \Bl }^k
}
=
k i \Hbar l_y \lr{ \Bp \cdot \Bl }^{k-1}.
\end{dmath}
%
Note that this also holds for the \( k = 0 \) case, since \( y \) commutes with the identity operator.  Plugging back into the \( \mathcal{T} \) commutator, we have
%
\begin{dmath}\label{eqn:translation:120}
\antisymmetric{y}{\mathcal{T}(\Bl)}
=
\sum_{k = 1} \inv{k!} \lr{\frac{-i}{\Hbar}}
k i \Hbar l_y \lr{ \Bp \cdot \Bl }^{k-1}
=
l_y \sum_{k = 1} \inv{(k-1)!} \lr{\frac{-i}{\Hbar}}
\lr{ \Bp \cdot \Bl }^{k-1}
=
l_y \mathcal{T}(\Bl).
\end{dmath}
%
The same pattern clearly applies with the other \( x_i \) values, providing the desired relation.
%
\begin{equation}\label{eqn:translation:140}
\antisymmetric{\Bx}{\mathcal{T}(\Bl)} = \sum_{m = 1}^3 \Be_m l_m \mathcal{T}(\Bl) = \Bl \mathcal{T}(\Bl).
\end{equation}
%
\makeSubAnswer{}{problem:translation:1.30:b}
%
Suppose that the translated state is defined as \( \ket{\alpha_{\Bl}} = \mathcal{T}(\Bl) \ket{\alpha} \).  The expectation value with respect to this state is
%
\begin{dmath}\label{eqn:translation:160}
\expectation{\Bx'}
=
\bra{\alpha_{\Bl}} \Bx \ket{\alpha_{\Bl}}
=
\bra{\alpha} \mathcal{T}^\dagger(\Bl) \Bx \mathcal{T}(\Bl) \ket{\alpha}
=
\bra{\alpha} \mathcal{T}^\dagger(\Bl) \lr{ \Bx \mathcal{T}(\Bl) } \ket{\alpha}
=
\bra{\alpha} \mathcal{T}^\dagger(\Bl) \lr{ \mathcal{T}(\Bl) \Bx + \Bl \mathcal{T}(\Bl) } \ket{\alpha}
=
\bra{\alpha} \mathcal{T}^\dagger \mathcal{T} \Bx + \Bl \mathcal{T}^\dagger \mathcal{T} \ket{\alpha}
=
\bra{\alpha} \Bx \ket{\alpha} + \Bl \braket{\alpha}{\alpha}
=
\expectation{\Bx} + \Bl.
\end{dmath}
%
} % answer

%\EndArticle

         %
% Copyright � 2015 Peeter Joot.  All Rights Reserved.
% Licenced as described in the file LICENSE under the root directory of this GIT repository.
%
\makeoproblem{Density matrix.}{gradQuantum:problemSet1:1}{phy1520 2015 ps1.1}{
\index{density matrix}

Consider a spin-1/2 particle. The Hilbert space is two-dimensional, let us label the two states as \( \ket{\uparrow} \) and \( \ket{\downarrow} \).
Write down the \( 2 \times 2 \) density matrix which corresponds to the following pure states.

\begin{enumerate}[(i)]
\item \( \ket{\uparrow} \)
\item \( \inv{\sqrt{2}} \lr{ \ket{\uparrow} + \ket{\downarrow} } \)
\item \( \inv{\sqrt{2}} \lr{ \ket{\uparrow} + i \ket{\downarrow} } \)
\item At time \( t = 0 \), let us start with the state \( \inv{\sqrt{2}} \lr{ \ket{\uparrow} + i \ket{\downarrow} } \),
and consider time-evolution under the Hamiltonian \( \hatH = - B S_z \),
where \( S_z \) is the z-component of the spin operator.
This leads to eigenstates \( \ket{\uparrow} \) with energy \( -B \Hbar/2 \), and
\( \ket{\downarrow} \)
with energy \( +B \Hbar/2 \).
The state
\( \inv{\sqrt{2}} \lr{ \ket{\uparrow} + i \ket{\downarrow} } \)
is not an eigenstate of this Hamiltonian and it will evolve in time.
Find the state and the corresponding \( 2 \times 2 \) density matrix of this system at a later time \( t \).
\end{enumerate}

} % makeproblem

\makeanswer{gradQuantum:problemSet1:1}{
\withproblemsetsParagraph{
\begin{enumerate}[(i)]
\item
This state in matrix form is
%
\begin{dmath}\label{eqn:gradQuantumProblemSet1Problem1:20}
\ket{\psi} =
\begin{bmatrix}
1 \\
0
\end{bmatrix},
\end{dmath}
%
for which the density operator representation is
%
\begin{dmath}\label{eqn:gradQuantumProblemSet1Problem1:40}
\hat\rho
=
\begin{bmatrix}
1 \\
0
\end{bmatrix}
\begin{bmatrix}
1 &
0
\end{bmatrix}
=
\begin{bmatrix}
1 & 0 \\
0 & 0
\end{bmatrix}.
\end{dmath}

\item

This state in matrix form is
%
\begin{dmath}\label{eqn:gradQuantumProblemSet1Problem1:60}
\ket{\psi} =
\inv{\sqrt{2}}
\lr{
\begin{bmatrix}
1 \\
0
\end{bmatrix}
+
\begin{bmatrix}
0 \\
1
\end{bmatrix}
}
=
\inv{\sqrt{2}}
\begin{bmatrix}
1 \\
1
\end{bmatrix}
\end{dmath}

for which the density operator representation is
%
\begin{dmath}\label{eqn:gradQuantumProblemSet1Problem1:80}
\hat\rho
=
\inv{2}
\begin{bmatrix}
1 \\
1
\end{bmatrix}
\begin{bmatrix}
1 &
1
\end{bmatrix}
=
\inv{2}
\begin{bmatrix}
1 & 1 \\
1 & 1
\end{bmatrix}.
\end{dmath}

\item

This state in matrix form is
%
\begin{dmath}\label{eqn:gradQuantumProblemSet1Problem1:100}
\ket{\psi} =
\inv{\sqrt{2}}
\lr{
\begin{bmatrix}
1 \\
0
\end{bmatrix}
+
\begin{bmatrix}
0 \\
i
\end{bmatrix}
}
=
\inv{\sqrt{2}}
\begin{bmatrix}
1 \\
i
\end{bmatrix}
\end{dmath}

for which the density operator representation is
%
\begin{dmath}\label{eqn:gradQuantumProblemSet1Problem1:120}
\hat\rho
=
\inv{2}
\begin{bmatrix}
1 \\
i
\end{bmatrix}
\begin{bmatrix}
1 &
-i
\end{bmatrix}
=
\inv{2}
\begin{bmatrix}
1 & -i \\
i & 1
\end{bmatrix}.
\end{dmath}

\item

The time evolution operator is
%
\begin{dmath}\label{eqn:gradQuantumProblemSet1Problem1:140}
U(t)
= e^{-i \hatH t/\Hbar}
= e^{i B S_z t/\Hbar}
= e^{i B \sigma_z t/2}
= \cos( B t/2 ) + i \sigma_z \sin( B t/2)
=
\begin{bmatrix}
\cos(B t/2) & 0 \\
0 & \cos(B t/2) \\
\end{bmatrix}
+
\begin{bmatrix}
i \sin(B t/2) & 0 \\
0 & -i \sin(B t/2) \\
\end{bmatrix}
=
\begin{bmatrix}
e^{i B t/2} & 0 \\
0 & e^{-i B t/2}
\end{bmatrix}.
\end{dmath}

so the time evolved state is
%
\begin{dmath}\label{eqn:gradQuantumProblemSet1Problem1:160}
\ket{\psi(t)} =
\inv{\sqrt{2}}
\begin{bmatrix}
e^{i B t/2} & 0 \\
0 & e^{-i B t/2}
\end{bmatrix}
\begin{bmatrix}
1 \\
i
\end{bmatrix}
=
\inv{\sqrt{2}}
\begin{bmatrix}
e^{i B t/2} \\
i e^{-i B t/2}
\end{bmatrix}.
\end{dmath}

The density operator matrix representation is
%
\begin{dmath}\label{eqn:gradQuantumProblemSet1Problem1:180}
\hat\rho(t)
=
\inv{2}
\begin{bmatrix}
e^{i B t/2} \\
i e^{-i B t/2}
\end{bmatrix}
\begin{bmatrix}
e^{-i B t/2} & -i e^{i B t/2}
\end{bmatrix}
=
\inv{2}
\begin{bmatrix}
1 & -i e^{i B t} \\
i e^{-i B t} & 1
\end{bmatrix}.
\end{dmath}

As a check observe that this has the right value for \( t = 0 \).  This also checks against the slightly messier computation \( \hatp(t) = U \hatp(0) U^\dagger \).

\end{enumerate}
}
}

         %
% Copyright � 2015 Peeter Joot.  All Rights Reserved.
% Licenced as described in the file LICENSE under the root directory of this GIT repository.
%
\makeoproblem{Reduced density matrix.}{gradQuantum:problemSet1:2}{phy1520 2015 ps1.2}{
\index{reduced density!matrix}

\index{spin half}
Consider two spin-1/2 particles, the Hilbert space is now 4-dimensional,
with states
\( \ket{ \uparrow \uparrow } \),
\( \ket{ \uparrow \downarrow } \),
\( \ket{ \downarrow \uparrow } \),
\( \ket{ \downarrow \downarrow } \).
Let us consider the following pure states:

\begin{enumerate}[(i)]
\item
\( \inv{2} \lr{
 \ket{ \uparrow \uparrow }
- \ket{ \uparrow \downarrow }
- \ket{ \downarrow \downarrow }
+ \ket{ \downarrow \uparrow }
} \)
\item \( \inv{\sqrt{2}} \lr{
 \ket{ \uparrow \uparrow }
+ \ket{ \downarrow \downarrow }
} \)
\item \( \inv{\sqrt{5}} \lr{
 \ket{ \uparrow \uparrow }
+ 2  \ket{ \downarrow \downarrow }
} \)
\end{enumerate}
In each case, obtain the reduced \( 2 \times 2 \) density matrix which describes the first spin, when we trace over the
second spin. The von Neumann entanglement entropy is defined via
\( S_{\txtv \txtN} = -\tr(\rho_\txtR \ln \rho_\txtR ) \) where \( \rho_\txtR \) is the reduced
density matrix you have obtained above and the \( \tr \) now refers to tracing over the first spin. Using the reduced
density matrices you have obtained above, compute the corresponding
\( S_{\txtv \txtN} \),
and simply explain your result in words.
Consider the Renyi entropy
\( S_n =
\inv{1-n}
\ln \lr{ \tr( \rho_\txtR^n ) } \).
Prove that
\( S_{n \rightarrow 1} = S_{\txtv \txtN} \),
and compute \( S_{n=2} \) for the above \( \rho_\txtR \).
} % makeproblem

\makeanswer{gradQuantum:problemSet1:2}{
\withproblemsetsParagraph{

\paragraph{reduced density operator matrix representation}
\index{reduced density!operator}

Some notation is useful to start.  Let \( \ket{a}_1 \) represent a spin state for the first particle and \( \ket{a}_2 \) represent a spin state for the second particle.  If the state under consideration is \( \ket{\psi} \), the reduced density operator matrix representation, after summing over all the second particle states, is
%
\begin{dmath}\label{eqn:gradQuantumProblemSet1Problem2:20}
\rho_\txtR
= \sum_a \prescript{}{2}{\bra{a}} \lr{ \ket{\psi} \bra{\psi} } \ket{a}_2.
\end{dmath}
%
Computing these \( {}_2\braket{a}{\psi} \) brackets for each state will allow the each of reduced density matrix to be expressed easily.

\begin{enumerate}[(i)]
\item For this state we have
%
\begin{dmath}\label{eqn:gradQuantumProblemSet1Problem2:40}
\prescript{}{2}{\braket{\uparrow}{\psi}} =
\inv{2}
\prescript{}{2}{\bra{\uparrow}}
\lr{
 \ket{ \uparrow \uparrow }
- \ket{ \uparrow \downarrow }
- \ket{ \downarrow \downarrow }
+ \ket{ \downarrow \uparrow }
}
=
\inv{2}
\lr{
 \ket{ \uparrow }_1
+ \ket{ \downarrow }_1
}
=
\inv{2}
\begin{bmatrix}
1 \\
1
\end{bmatrix},
\end{dmath}
%
and
%
\begin{dmath}\label{eqn:gradQuantumProblemSet1Problem2:60}
\prescript{}{2}{\braket{\downarrow}{\psi}} =
\inv{2}
\prescript{}{2}{\bra{\downarrow}}
\lr{
 \ket{ \uparrow \uparrow }
- \ket{ \uparrow \downarrow }
- \ket{ \downarrow \downarrow }
+ \ket{ \downarrow \uparrow }
}
=
\inv{2}
\lr{
- \ket{ \uparrow }_1
- \ket{ \downarrow }_1
}
=
\inv{2}
\begin{bmatrix}
-1 \\
-1
\end{bmatrix},
\end{dmath}
%
so the reduced density operator matrix representation is
%
\begin{dmath}\label{eqn:gradQuantumProblemSet1Problem2:80}
\rho_\txtR
=
\inv{4}
\begin{bmatrix}
1 \\
1
\end{bmatrix}
\begin{bmatrix}
1 & 1
\end{bmatrix}
+
\inv{4}
\begin{bmatrix}
-1 \\
-1
\end{bmatrix}
\begin{bmatrix}
-1 & -1
\end{bmatrix}
=
\inv{2}
\begin{bmatrix}
1 & 1 \\
1 & 1
\end{bmatrix}.
\end{dmath}
%
\item For this state we have
%
\begin{dmath}\label{eqn:gradQuantumProblemSet1Problem2:100}
\prescript{}{2}{\braket{\uparrow}{\psi}} =
\inv{\sqrt{2}}
\prescript{}{2}{\bra{\uparrow}}
\lr{
 \ket{ \uparrow \uparrow }
+ \ket{ \downarrow \downarrow }
}
=
\inv{\sqrt{2}}
 \ket{ \uparrow }_1
=
\inv{\sqrt{2}}
\begin{bmatrix}
1 \\
0
\end{bmatrix},
\end{dmath}
%
and
%
\begin{dmath}\label{eqn:gradQuantumProblemSet1Problem2:120}
\prescript{}{2}{\braket{\downarrow}{\psi}} =
\inv{\sqrt{2}}
\prescript{}{2}{\bra{\downarrow}}
\lr{
 \ket{ \uparrow \uparrow }
+ \ket{ \downarrow \downarrow }
}
=
\inv{\sqrt{2}}
 \ket{ \downarrow }_1
=
\inv{\sqrt{2}}
\begin{bmatrix}
0 \\
1
\end{bmatrix},
\end{dmath}
%
so the reduced density operator matrix representation is
%
\begin{dmath}\label{eqn:gradQuantumProblemSet1Problem2:140}
\rho_\txtR
=
\inv{2}
\begin{bmatrix}
1 \\
0
\end{bmatrix}
\begin{bmatrix}
1 & 0
\end{bmatrix}
+
\inv{2}
\begin{bmatrix}
0 \\
1
\end{bmatrix}
\begin{bmatrix}
0 & 1
\end{bmatrix}
=
\inv{2}
\begin{bmatrix}
1 & 0 \\
0 & 1
\end{bmatrix}.
\end{dmath}
%
\item For this state we have
%
\begin{dmath}\label{eqn:gradQuantumProblemSet1Problem2:160}
\prescript{}{2}{\braket{\uparrow}{\psi}} =
\inv{\sqrt{5}}
\prescript{}{2}{\bra{\uparrow}}
\lr{
 \ket{ \uparrow \uparrow }
+ 2 \ket{ \downarrow \downarrow }
}
=
\inv{\sqrt{5}}
 \ket{ \uparrow }_1
=
\inv{\sqrt{5}}
\begin{bmatrix}
1 \\
0
\end{bmatrix},
\end{dmath}
%
and
%
\begin{dmath}\label{eqn:gradQuantumProblemSet1Problem2:180}
\prescript{}{2}{\braket{\downarrow}{\psi}} =
\inv{\sqrt{5}}
\prescript{}{2}{\bra{\downarrow}}
\lr{
 \ket{ \uparrow \uparrow }
+ 2 \ket{ \downarrow \downarrow }
}
=
\frac{2}{\sqrt{5}}
 \ket{ \downarrow }_1
=
\frac{2}{\sqrt{5}}
\begin{bmatrix}
0 \\
1
\end{bmatrix},
\end{dmath}
%
so the reduced density operator matrix representation is
%
\begin{dmath}\label{eqn:gradQuantumProblemSet1Problem2:200}
\rho_\txtR
=
\inv{5}
\begin{bmatrix}
1 \\
0
\end{bmatrix}
\begin{bmatrix}
1 & 0
\end{bmatrix}
+
\frac{4}{5}
\begin{bmatrix}
0 \\
1
\end{bmatrix}
\begin{bmatrix}
0 & 1
\end{bmatrix}
=
\inv{5}
\begin{bmatrix}
1 & 0 \\
0 & 4
\end{bmatrix}.
\end{dmath}
%
\end{enumerate}

\paragraph{von Neumann entropy.}
\index{von Neumann entropy}

Here is the computation of the von Neumann entropy for each of these states.

\begin{enumerate}[(i)]
\item For this case, the eigenvalues of the reduced density matrix are \( \setlr{0,1} \), so the von Neumann entropy is
%
\begin{dmath}\label{eqn:gradQuantumProblemSet1Problem2:220}
S_{\txtv \txtN} = - 1 \ln 1 - 0 \ln 0 = 0,
\end{dmath}
%
where the zero logarithm is interpreted in the limit \( \lim_{x \rightarrow 0} -x \ln x = 0 \).

\item Since this reduced density matrix is diagonal, the von Neumann entropy can be calculated easily
%
\begin{dmath}\label{eqn:gradQuantumProblemSet1Problem2:240}
S_{\txtv \txtN} = 2 \lr{ - \inv{2} \ln \inv{2} } = \ln 2 \approx 0.69
\end{dmath}

\item This is also diagonal with von Neumann entropy of
%
\begin{dmath}\label{eqn:gradQuantumProblemSet1Problem2:260}
S_{\txtv \txtN}
= - \inv{5} \ln \inv{5} - \frac{4}{5} \ln \frac{4}{5}
= \inv{5} \lr{ \ln 5 + 4 \ln (5/4) }
\approx 0.50
\end{dmath}

In the diagonal representation the entropy of the reduced density operator for case (i) shows that it is the most ordered system, with a unit probability amplitude for one state of that representation, and zero amplitude for the other.  Case (iii) of the reduced system is the next most ordered, with the more of the probability amplitude weighting associated with the first particle in the spin down state.  This results in a greater than zero entropy.  Case (ii) is the least ordered of the reduced systems with equal probability amplitudes for each first particle spin state, and has the maximum von Neumann entropy possible for a two dimensional Hilbert space.

\paragraph{Renyi entropy.}
\index{Renyi entropy}

Here is the computation of the Renyi entropy for \( n = 2 \) for each of these states.  That is
%
\begin{dmath}\label{eqn:gradQuantumProblemSet1Problem2:280}
S_{n=2} = -\ln \tr \lr{ \rho_\txtR^2 }.
\end{dmath}
\end{enumerate}

\begin{enumerate}[(i)]
\item

For this case the reduced density matrix is idempotent
\index{idempotent}
%
\begin{dmath}\label{eqn:gradQuantumProblemSet1Problem2:300}
\rho_\txtR^2 = \inv{4}
\begin{bmatrix}
1 & 1  \\
1 & 1
\end{bmatrix}
\begin{bmatrix}
1 & 1  \\
1 & 1
\end{bmatrix}
=
\inv{2}
\begin{bmatrix}
1 & 1  \\
1 & 1
\end{bmatrix}.
\end{dmath}
%
So the trace is the sum of the eigenvalues of \( \rho_\txtR \), and the \( n = 2 \) Renyi entropy is
%
\begin{dmath}\label{eqn:gradQuantumProblemSet1Problem2:320}
S_{n=2} = -\ln (1 + 0) = 0.
\end{dmath}
%
\item

For this case, the \( n = 2 \) Renyi entropy is
%
\begin{dmath}\label{eqn:gradQuantumProblemSet1Problem2:340}
S_{n=2} = -\ln \lr{ \inv{2^2} + \inv{2^2}} = \ln 2 \approx 0.69.
\end{dmath}
%
\item

Finally, for this case, the \( n = 2 \) Renyi entropy is
%
\begin{dmath}\label{eqn:gradQuantumProblemSet1Problem2:360}
S_{n=2} = -\ln \lr{ \frac{1}{25} + \frac{16}{25} } = \ln \frac{17}{25} \approx 0.39.
\end{dmath}
%
\end{enumerate}

\paragraph{Limit of Renyi entropy}

In a diagonal representation of the reduced density matrix, the \( n \rightarrow 1 \) summation in the numerator of the Renyi entropy
%
\begin{dmath}\label{eqn:gradQuantumProblemSet1Problem2:380}
S_{n \rightarrow 1}
= \lim_{n \rightarrow 1} \frac{\ln \sum_{k} \rho_{kk}^n }{1 - n},
\end{dmath}
%
tends to unity, so that logarithm tends to zero.  This results in division of two nearly zero quantities, which can be evaluated using l'H\^{o}pital's rule.  That is
%
\begin{dmath}\label{eqn:gradQuantumProblemSet1Problem2:400}
S_{n \rightarrow 1}
= \frac{\evalbar{\frac{d}{dn} \ln \sum_{k} \rho_{kk}^n}{n=1} }{-1}
=
\frac{
\sum_k \evalbar{\frac{d}{dn} \rho_{kk}^n}{n=1}
}{-
\evalbar{\sum_{k} \rho_{kk}^n}{n=1}
}
=
- \sum_k \evalbar{\frac{d}{dn} \rho_{kk}^n}{n=1}.
\end{dmath}
%
To evaluate the derivatives set \( a^n = e^y \), or \( y = n \ln a \).  That gives
%
\begin{dmath}\label{eqn:gradQuantumProblemSet1Problem2:420}
\frac{d}{dn} a^n
=
\frac{d}{dn} e^{ n \ln a }
=
a^n \ln a,
\end{dmath}
%
which gives
%
\begin{dmath}\label{eqn:gradQuantumProblemSet1Problem2:440}
S_{n \rightarrow 1}
=
-
\evalbar{
\sum_k \rho_{kk}^n \ln \rho_{kk}
}{n=1}
=
\sum_k \rho_{kk} \ln \rho_{kk},
\end{dmath}
%
which is precisely the von Neumann entropy.
}
}

         % p3.10
         %
% Copyright � 2015 Peeter Joot.  All Rights Reserved.
% Licenced as described in the file LICENSE under the root directory of this GIT repository.
%
%\input{../blogpost.tex}
%\renewcommand{\basename}{ensemblesForSpinOneHalf}
%\renewcommand{\dirname}{notes/phy1520/}
%%\newcommand{\dateintitle}{}
%%\newcommand{\keywords}{}
%
%\input{../peeter_prologue_print2.tex}
%
%\usepackage{peeters_layout_exercise}
%\usepackage{peeters_braket}
%\usepackage{peeters_figures}
%
%\beginArtNoToc
%
%\generatetitle{Ensembles for spin one half}
%%\chapter{Ensembles for spin one half}
%%\label{chap:ensemblesForSpinOneHalf}

\makeoproblem{Ensembles for spin one half.}{problem:ensemblesForSpinOneHalf:1}{\citep{sakurai2014modern} pr. 3.10}{
\index{spin half!ensemble averages}

\makesubproblem{}{problem:ensemblesForSpinOneHalf:1:a}
Sakurai leaves it to the reader to verify that knowledge of the three ensemble averages [S_x], [S_y],[S_z] is sufficient to reconstruct the density operator for a spin one half system.  Show this.

\makesubproblem{}{problem:ensemblesForSpinOneHalf:1:b}
Show how the expectation values \( \expectation{S_x}, \expectation{S_y},\expectation{S_x} \) fully determine the spin orientation for a pure ensemble.
} % problem

\makeanswer{problem:ensemblesForSpinOneHalf:1}{
\makeSubAnswer{}{problem:ensemblesForSpinOneHalf:1:a}

I'll do this in two parts, the first using a spin-up/down ensemble to see what form this has, then the general case.  The general case is a bit messy algebraically.  After first attempting it the hard way, I did the grunt work portion of that calculation in Mathematica, but then realized it's not so bad to do it manually.

Consider first an ensemble with \textAndIndex{density operator}
%
\begin{dmath}\label{eqn:ensemblesForSpinOneHalf:20}
\rho =
w_{+} \ket{+}\bra{+} + w_{-} \ket{-}\bra{-},
\end{dmath}

where these are the \( \BS \cdot (\pm \zcap) \) eigenstates.  The traces are
%
\begin{dmath}\label{eqn:ensemblesForSpinOneHalf:40}
\tr( \rho \sigma_x )
=
\bra{+} \rho \sigma_x \ket{+}
+
\bra{-} \rho \sigma_x \ket{-}
=
\bra{+} \rho \PauliX \ket{+}
+
\bra{-} \rho \PauliX \ket{-}
=
\bra{+} \lr{ w_{+} \ket{+}\bra{+} + w_{-} \ket{-}\bra{-} } \ket{-}
+
\bra{-} \lr{ w_{+} \ket{+}\bra{+} + w_{-} \ket{-}\bra{-} } \ket{+}
=
\bra{+} w_{-} \ket{-}
+
\bra{-} w_{+} \ket{+}
=
0,
\end{dmath}
%
\begin{dmath}\label{eqn:ensemblesForSpinOneHalf:60}
\tr( \rho \sigma_y )
=
\bra{+} \rho \sigma_y \ket{+}
+
\bra{-} \rho \sigma_y \ket{-}
=
\bra{+} \rho \PauliY \ket{+}
+
\bra{-} \rho \PauliY \ket{-}
=
i \bra{+} \lr{ w_{+} \ket{+}\bra{+} + w_{-} \ket{-}\bra{-} } \ket{-}
-
i \bra{-} \lr{ w_{+} \ket{+}\bra{+} + w_{-} \ket{-}\bra{-} } \ket{+}
=
i \bra{+} w_{-} \ket{-}
-
i \bra{-} w_{+} \ket{+}
=
0,
\end{dmath}

and
\begin{dmath}\label{eqn:ensemblesForSpinOneHalf:100}
\tr( \rho \sigma_z )
=
\bra{+} \rho \sigma_z \ket{+}
+
\bra{-} \rho \sigma_z \ket{-}
=
\bra{+} \rho \ket{+}
-
\bra{-} \rho \ket{-}
=
 \bra{+} \lr{ w_{+} \ket{+}\bra{+} + w_{-} \ket{-}\bra{-} } \ket{+}
-
 \bra{-} \lr{ w_{+} \ket{+}\bra{+} + w_{-} \ket{-}\bra{-} } \ket{-}
=
 \bra{+} w_{+} \ket{+}
-
 \bra{-} w_{-} \ket{-}
=
w_{+} - w_{-}.
\end{dmath}

Since \( w_{+} + w_{-} = 1 \), this gives

\boxedEquation{eqn:ensemblesForSpinOneHalf:80}{
\begin{aligned}
w_{+} &= \frac{1 + \tr( \rho \sigma_z )}{2} \\
w_{-} &= \frac{1 - \tr( \rho \sigma_z )}{2}
\end{aligned}
}

Attempting to do a similar set of trace expansions this way for a more general spin basis turns out to be a really bad idea and horribly messy.  So much so that I resorted to
%\href{https://raw.githubusercontent.com/peeterjoot/mathematica/master/phy1520/spinOneHalfSymbolicManipulation.nb}{Mathematica to do this symbolic work}
\nbref{spinOneHalfSymbolicManipulation.nb} to do this symbolic work.
However, it's not so bad if the trace is done completely in matrix form.

Using the basis
%
\begin{dmath}\label{eqn:ensemblesForSpinOneHalf:120}
\begin{aligned}
\ket{\BS \cdot \ncap ; + } &=
\begin{bmatrix}
\cos(\theta/2) \\
\sin(\theta/2) e^{i \phi}
\end{bmatrix} \\
\ket{\BS \cdot \ncap ; - } &=
\begin{bmatrix}
\sin(\theta/2) e^{-i \phi} \\
-\cos(\theta/2) \\
\end{bmatrix},
\end{aligned}
\end{dmath}

\index{projection operator}
the projector matrices are
%
\begin{dmath}\label{eqn:ensemblesForSpinOneHalf:140}
\ket{\BS \cdot \ncap ; + } \bra{\BS \cdot \ncap ; + }
=
\begin{bmatrix}
\cos(\theta/2) \\
\sin(\theta/2) e^{i \phi}
\end{bmatrix}
\begin{bmatrix}
\cos(\theta/2) &
\sin(\theta/2) e^{-i \phi}
\end{bmatrix}
=
\begin{bmatrix}
\cos^2(\theta/2) & \cos(\theta/2) \sin(\theta/2) e^{-i \phi} \\
\sin(\theta/2) \cos(\theta/2) e^{i \phi} & \sin^2(\theta/2)
\end{bmatrix},
\end{dmath}
\begin{dmath}\label{eqn:ensemblesForSpinOneHalf:160}
\ket{\BS \cdot \ncap ; - } \bra{\BS \cdot \ncap ; - }
=
\begin{bmatrix}
\sin(\theta/2) e^{-i \phi} \\
-\cos(\theta/2) \\
\end{bmatrix}
\begin{bmatrix}
\sin(\theta/2) e^{i \phi} & -\cos(\theta/2) \\
\end{bmatrix}
=
\begin{bmatrix}
\sin^2(\theta/2)  & -\cos(\theta/2) \sin(\theta/2) e^{-i \phi} \\
-\cos(\theta/2) \sin(\theta/2) e^{i \phi} & \cos^2(\theta/2)
\end{bmatrix}
\end{dmath}

With \( C = \cos(\theta/2), S = \sin(\theta/2) \), a general density operator in this basis has the form
%
\begin{dmath}\label{eqn:ensemblesForSpinOneHalf:180}
\rho
=
w_{+}
\begin{bmatrix}
C^2 & C S e^{-i \phi} \\
S C e^{i \phi} & S^2
\end{bmatrix}
+
w_{-}
\begin{bmatrix}
S^2  & -C S e^{-i \phi} \\
-C S e^{i \phi} & C^2
\end{bmatrix}
=
\begin{bmatrix}
w_{+} C^2 + w_{-} S^2  & (w_{+} - w_{-})C S e^{-i \phi} \\
(w_{+} -w_{-} ) S C e^{i \phi} & w_{+} S^2  + w_{-} C^2
\end{bmatrix}.
\end{dmath}

The products with the Pauli matrices are
\index{Pauli matrix}
%
\begin{dmath}\label{eqn:ensemblesForSpinOneHalf:200}
\rho \sigma_x
=
\begin{bmatrix}
w_{+} C^2 + w_{-} S^2  & (w_{+} - w_{-})C S e^{-i \phi} \\
(w_{+} -w_{-} ) S C e^{i \phi} & w_{+} S^2  + w_{-} C^2
\end{bmatrix}
\PauliX
=
\begin{bmatrix}
(w_{+} - w_{-})C S e^{-i \phi}  & w_{+} C^2 + w_{-} S^2  \\
w_{+} S^2  + w_{-} C^2 & (w_{+} -w_{-} ) S C e^{i \phi} \\
\end{bmatrix}
\end{dmath}
%
\begin{dmath}\label{eqn:ensemblesForSpinOneHalf:220}
\rho \sigma_y
=
\begin{bmatrix}
w_{+} C^2 + w_{-} S^2  & (w_{+} - w_{-})C S e^{-i \phi} \\
(w_{+} -w_{-} ) S C e^{i \phi} & w_{+} S^2  + w_{-} C^2
\end{bmatrix}
\PauliY
=
i
\begin{bmatrix}
(w_{+} - w_{-})C S e^{-i \phi}  & -w_{+} C^2 - w_{-} S^2  \\
w_{+} S^2  + w_{-} C^2          & -(w_{+} -w_{-} ) S C e^{i \phi} \\
\end{bmatrix}
\end{dmath}
%
\begin{dmath}\label{eqn:ensemblesForSpinOneHalf:240}
\rho \sigma_z
=
\begin{bmatrix}
w_{+} C^2 + w_{-} S^2  & (w_{+} - w_{-})C S e^{-i \phi} \\
(w_{+} -w_{-} ) S C e^{i \phi} & w_{+} S^2  + w_{-} C^2
\end{bmatrix}
\PauliZ
=
\begin{bmatrix}
w_{+} C^2 + w_{-} S^2  & -(w_{+} - w_{-})C S e^{-i \phi} \\
(w_{+} -w_{-} ) S C e^{i \phi} & - (w_{+} S^2  + w_{-} C^2)
\end{bmatrix}
\end{dmath}

The respective traces can be read right off the matrices
\begin{dmath}\label{eqn:ensemblesForSpinOneHalf:260}
\begin{aligned}
\tr( \rho \sigma_x ) &= (w_{+} - w_{-}) \sin\theta \cos\phi \\
\tr( \rho \sigma_y ) &= (w_{+} - w_{-}) \sin\theta \sin\phi \\
\tr( \rho \sigma_z ) &= (w_{+} - w_{-}) \cos\theta \\
\end{aligned}.
\end{dmath}

This gives
%
\begin{dmath}\label{eqn:ensemblesForSpinOneHalf:280}
(w_{+} - w_{-}) \ncap = \lr{ \tr( \rho \sigma_x ), \tr( \rho \sigma_y ), \tr( \rho \sigma_z ) },
\end{dmath}

or

\boxedEquation{eqn:ensemblesForSpinOneHalf:320}{
w_{\pm} = \frac{1 \pm \sqrt{ \tr^2( \rho \sigma_x ) + \tr^2( \rho \sigma_y ) + \tr^2( \rho \sigma_z )} }{2} .
}

So, as claimed, it's possible to completely describe the ensemble weight factors using the ensemble averages of \( [S_x], [S_y], [S_z] \).  I used the Pauli matrices instead, but the difference is just an \( \Hbar/2 \) scaling adjustment.

\paragraph{Alternate approach}

Another easier and trig free way to look at this problem is assume the density operator's representation is given by a \( 2 \times 2 \) matrix with undetermined values
%
\begin{dmath}\label{eqn:ensemblesForSpinOneHalf:400}
\rho =
\begin{bmatrix}
a & b \\
c & d
\end{bmatrix}
\end{dmath}

For such a representation we have
%
\begin{equation}\label{eqn:ensemblesForSpinOneHalf:420}
\begin{aligned}
\rho \sigma_x
&=
\begin{bmatrix}
a & b \\
c & d
\end{bmatrix}
\PauliX
=
\begin{bmatrix}
b & a \\
d & c
\end{bmatrix} \\
\rho \sigma_y
&=
\begin{bmatrix}
a & b \\
c & d
\end{bmatrix}
\PauliY
=
i
\begin{bmatrix}
b & -a \\
d & -c
\end{bmatrix} \\
\rho \sigma_z
&=
\begin{bmatrix}
a & b \\
c & d
\end{bmatrix}
\PauliZ
=
\begin{bmatrix}
a & -b\\
c & -d
\end{bmatrix} \\
\end{aligned}
\end{equation}

The ensemble averages can be read by inspection
\begin{equation}\label{eqn:ensemblesForSpinOneHalf:440}
\begin{aligned}
[\sigma_x] &= b + c \\
[\sigma_y] &= i(b - c) \\
[\sigma_z] &= a - d \\
\end{aligned}
\end{equation}

Noting that \( \tr \lr{E A E^{-1}} = \tr \lr{A E^{-1} E} = \tr \lr{A} \), and that there must be a diagonal basis for which \( \braket{+}{-} = 0 \) and
%
\begin{dmath}\label{eqn:ensemblesForSpinOneHalf:460}
\rho = w_{+} \ket{+}\bra{+} + w_{-} \ket{-}\bra{-},
\end{dmath}

we must have
%
\begin{dmath}\label{eqn:ensemblesForSpinOneHalf:480}
\trace{\rho} = a + d = w_{+} + w_{-} = 1.
\end{dmath}

This provides one set of equations for each of \( b,c \) and \( a, d\)
%
\begin{equation}\label{eqn:ensemblesForSpinOneHalf:500}
\begin{aligned}
[\sigma_x] &= b + c \\
[\sigma_y] &= i(b - c),
\end{aligned}
\end{equation}

and
\begin{equation}\label{eqn:ensemblesForSpinOneHalf:520}
\begin{aligned}
[\sigma_z] &= a - d \\
1          &= a + d.
\end{aligned}
\end{equation}

These have solutions
\begin{equation}\label{eqn:ensemblesForSpinOneHalf:540}
\begin{aligned}
b &= \frac{[\sigma_x] - i [\sigma_y]}{2} \\
c &= \frac{[\sigma_x] + i [\sigma_y]}{2} \\
a &= \frac{1 + [\sigma_z]}{2} \\
d &= \frac{1 - [\sigma_z]}{2},
\end{aligned}
\end{equation}

or

\boxedEquation{eqn:ensemblesForSpinOneHalf:560}{
\rho = \inv{2}
\begin{bmatrix}
1 + [\sigma_z] & [\sigma_x] - i [\sigma_y] \\
[\sigma_x] + i [\sigma_y] & 1 - [\sigma_z]
\end{bmatrix}.
}

The characteristic equation for this operator is
%
\begin{equation}\label{eqn:ensemblesForSpinOneHalf:600}
0 =
\lr{ \lr{ \inv{2} - \lambda } + \frac{[\sigma_z] }{2} }
\lr{ \lr{ \inv{2} - \lambda } - \frac{[\sigma_z] }{2} }
- \inv{4} \lr{ [\sigma_x] + i [\sigma_y] } \lr{ [\sigma_x] - i [\sigma_y] },
\end{equation}

or
\boxedEquation{eqn:ensemblesForSpinOneHalf:580}{
\lambda = \frac{1 \pm \sqrt{ [\sigma_x]^2 + [\sigma_y]^2 + [\sigma_z]^2 }}{2},
}

as found above.

\makeSubAnswer{}{problem:ensemblesForSpinOneHalf:1:b}

Suppose that the system is in the state \( \ket{\BS \cdot \ncap ; + } \) as defined in \cref{eqn:ensemblesForSpinOneHalf:120}, then the expectation values of \( \sigma_x, \sigma_y, \sigma_z \) with respect to this state are
%
\begin{dmath}\label{eqn:ensemblesForSpinOneHalf:300}
\expectation{\sigma_x}
=
\begin{bmatrix}
\cos(\theta/2) &
\sin(\theta/2) e^{-i \phi}
\end{bmatrix}
\PauliX
\begin{bmatrix}
\cos(\theta/2) \\
\sin(\theta/2) e^{i \phi}
\end{bmatrix}
=
\begin{bmatrix}
\cos(\theta/2) &
\sin(\theta/2) e^{-i \phi}
\end{bmatrix}
\begin{bmatrix}
\sin(\theta/2) e^{i \phi} \\
\cos(\theta/2) \\
\end{bmatrix}
=
\sin\theta \cos\phi,
\end{dmath}
\begin{dmath}\label{eqn:ensemblesForSpinOneHalf:340}
\expectation{\sigma_y}
=
\begin{bmatrix}
\cos(\theta/2) &
\sin(\theta/2) e^{-i \phi}
\end{bmatrix}
\PauliY
\begin{bmatrix}
\cos(\theta/2) \\
\sin(\theta/2) e^{i \phi}
\end{bmatrix}
=
i
\begin{bmatrix}
\cos(\theta/2) &
\sin(\theta/2) e^{-i \phi}
\end{bmatrix}
\begin{bmatrix}
-\sin(\theta/2) e^{i \phi} \\
\cos(\theta/2) \\
\end{bmatrix}
=
\sin\theta \sin\phi,
\end{dmath}
\begin{dmath}\label{eqn:ensemblesForSpinOneHalf:360}
\expectation{\sigma_z}
=
\begin{bmatrix}
\cos(\theta/2) &
\sin(\theta/2) e^{-i \phi}
\end{bmatrix}
\PauliZ
\begin{bmatrix}
\cos(\theta/2) \\
\sin(\theta/2) e^{i \phi}
\end{bmatrix}
=
\begin{bmatrix}
\cos(\theta/2) &
\sin(\theta/2) e^{-i \phi}
\end{bmatrix}
\begin{bmatrix}
\cos(\theta/2) \\
-\sin(\theta/2) e^{i \phi}
\end{bmatrix}
=
\cos\theta.
\end{dmath}

So we have
\boxedEquation{eqn:ensemblesForSpinOneHalf:380}{
\ncap = \lr{ \expectation{\sigma_x}, \expectation{\sigma_y}, \expectation{\sigma_z} }.
}

\index{spin half!expectation}
The spin direction is completely determined by this vector of expectation values (or equivalently, the expectation values of \( S_x, S_y, S_z \)).
} % answer

%\EndArticle

   \mychapter{Quantum Dynamics.}
      %\section{Quantum Harmonic oscillator and coherent states}
         %
% Copyright � 2015 Peeter Joot.  All Rights Reserved.
% Licenced as described in the file LICENSE under the root directory of this GIT repository.
%
%\input{../blogpost.tex}
%\renewcommand{\basename}{gmLecture4}
%\renewcommand{\dirname}{notes/phy1520/}
%%\newcommand{\dateintitle}{}
%\newcommand{\keywords}{PHY1520H}
%
%\input{../peeter_prologue_print2.tex}
%%\usepackage{phy1520}
%\usepackage{peeters_braket}
%%\usepackage{peeters_layout_exercise}
%\usepackage{peeters_figures}
%\usepackage{mathtools}
%
%\beginArtNoToc
%\generatetitle{PHY1520H Graduate Quantum Mechanics.  Lecture 4: Quantum Harmonic oscillator and coherent states.  Taught by Prof.\ Arun Paramekanti}
%%\chapter{Quantum Harmonic oscillator and coherent states}
%\label{chap:lecture4}
%
%\paragraph{Disclaimer}
%
%Peeter's lecture notes from class.  These may be incoherent and rough.  This lecture reviewed a lot of quantum harmonic oscillator theory, and wouldn't make sense without having seen raising and lowering operators (ladder operators), number operators, and the like.
%
%These are notes for the UofT course PHY1520, Graduate Quantum Mechanics, taught by Prof. Paramekanti, covering \textchapref{{2}} \citep{sakurai2014modern} content.
%
\section{Classical Harmonic Oscillator}
Recall the classical Harmonic oscillator equations in their Hamiltonian form

\begin{subequations}
\label{eqn:qmLecture4:20}
\begin{dmath}\label{eqn:qmLecture4:40}
\ddt{x} = \frac{p}{m}
\end{dmath}
\begin{dmath}\label{eqn:qmLecture4:60}
\ddt{p} = -k x.
\end{dmath}
\end{subequations}

With
%
\begin{equation}\label{eqn:qmLecture4:140}
\begin{aligned}
x(t = 0) &= x_0 \\
p(t = 0) &= p_0 \\
k &= m \omega^2,
\end{aligned}
\end{equation}

\index{classical harmonic oscillator}
the solutions are ellipses in phase space

\begin{subequations}
\label{eqn:qmLecture4:80}
\begin{dmath}\label{eqn:qmLecture4:100}
x(t) = x_0 \cos(\omega t) + \frac{p_0}{m \omega} \sin(\omega t)
\end{dmath}
\begin{dmath}\label{eqn:qmLecture4:120}
p(t) = p_0 \cos(\omega t) - m \omega x_0 \sin(\omega t).
\end{dmath}
\end{subequations}

After a suitable scaling of the variables, these elliptical orbits can be transformed into circular trajectories.

\section{Quantum Harmonic Oscillator}
\index{harmonic oscillator}
%
\begin{dmath}\label{eqn:qmLecture4:160}
\hatH = \frac{\hatp^2}{2 m} + \inv{2} k \hatx^2
\end{dmath}

Set

\begin{subequations}
\label{eqn:qmLecture4:180}
\begin{equation}\label{eqn:qmLecture4:200}
\hatX = \sqrt{\frac{m \omega}{\Hbar}} \hatx
\end{equation}
\begin{equation}\label{eqn:qmLecture4:220}
\hatP = \sqrt{\inv{m \omega \Hbar}} \hatp
\end{equation}
\end{subequations}

The commutators after this change of variables goes from
%
\begin{equation}\label{eqn:qmLecture4:240}
\antisymmetric{ \hatx}{\hatp} = i \Hbar,
\end{equation}
%
to
\begin{equation}\label{eqn:qmLecture4:260}
\antisymmetric{ \hatX}{\hatP} = i.
\end{equation}
%
The Hamiltonian takes the form
%
\begin{dmath}\label{eqn:qmLecture4:280}
\hatH
= \frac{\Hbar \omega}{2} \lr{ \hatX^2 + \hatP^2 }
= \Hbar \omega \lr{ \lr{ \frac{\hatX -i \hatP}{\sqrt{2}} } \lr{ \frac{\hatX +i \hatP}{\sqrt{2}}} + \inv{2} }.
\end{dmath}
%
\index{ladder operator}
\index{raising operator}
\index{lowering operator}
Define ladder operators (raising and lowering operators respectively)

\begin{subequations}
\label{eqn:qmLecture4:300}
\begin{dmath}\label{eqn:qmLecture4:320}
\hata^\dagger = \frac{\hatX -i \hatP}{\sqrt{2}}
\end{dmath}
\begin{dmath}\label{eqn:qmLecture4:340}
\hata = \frac{\hatX +i \hatP}{\sqrt{2}}
\end{dmath}
\end{subequations}

so
%
\begin{dmath}\label{eqn:qmLecture4:360}
\hatH = \Hbar \omega \lr{ \hata^\dagger \hata + \inv{2} }.
\end{dmath}
%
We can show
%
\begin{dmath}\label{eqn:qmLecture4:380}
\antisymmetric{\hata}{\hata^\dagger} = 1,
\end{dmath}
%
and

\index{number operator}
\begin{equation}\label{eqn:qmLecture4:400}
N \ket{n} \equiv \hata^\dagger a = n \ket{n},
\end{equation}
%
where \( n \ge 0 \) is an integer.  Recall that
%
\begin{dmath}\label{eqn:qmLecture4:420}
\hata \ket{0} = 0,
\end{dmath}
%
and
%
\begin{dmath}\label{eqn:qmLecture4:440}
\bra{X} X + i P \ket{0} = 0.
\end{dmath}
%
With
%
\begin{dmath}\label{eqn:qmLecture4:460}
\braket{x}{0} = \Psi_0(x),
\end{dmath}
%
we can show
%
\begin{dmath}\label{eqn:qmLecture4:480}
\inv{\sqrt{2}} \lr{ X + \PD{X}{} } \Psi_0(X) = 0.
\end{dmath}
%
Also recall that

\begin{subequations}
\label{eqn:qmLecture4:500}
\begin{dmath}\label{eqn:qmLecture4:520}
\hata \ket{n} = \sqrt{n} \ket{n-1}
\end{dmath}
\begin{dmath}\label{eqn:qmLecture4:540}
\hata^\dagger \ket{n} = \sqrt{n + 1} \ket{n+1}
\end{dmath}
\end{subequations}

\section{Coherent states}
\index{coherent state}

Coherent states for the quantum harmonic oscillator are the eigenkets for the creation and annihilation operators

\begin{subequations}
\label{eqn:qmLecture4:560}
\begin{dmath}\label{eqn:qmLecture4:580}
\hata \ket{z} = z \ket{z}
\end{dmath}
\begin{dmath}\label{eqn:qmLecture4:600}
\hata^\dagger \ket{\tilde{z}} = \tilde{z} \ket{\tilde{z}} ,
\end{dmath}
\end{subequations}

where
%
\begin{dmath}\label{eqn:qmLecture4:620}
\ket{z} = \sum_{n = 0}^\infty c_n \ket{n},
\end{dmath}
%
and \( z \) is allowed to be a complex number.

Looking for such a state, we compute
%
\begin{dmath}\label{eqn:qmLecture4:640}
\hata \ket{z}
= \sum_{n=1}^\infty c_n \hata \ket{n}
= \sum_{n=1}^\infty c_n \sqrt{n} \ket{n-1}
\end{dmath}

compare this to
%
\begin{dmath}\label{eqn:qmLecture4:660}
z \ket{z}
=
z \sum_{n=0}^\infty c_n \ket{n}
=
\sum_{n=1}^\infty c_n \sqrt{n} \ket{n-1}
=
\sum_{n=0}^\infty c_{n+1} \sqrt{n+1} \ket{n},
\end{dmath}
%
so
%
\begin{dmath}\label{eqn:qmLecture4:680}
c_{n+1} \sqrt{n+1} = z c_n
\end{dmath}

This gives
%
\begin{dmath}\label{eqn:qmLecture4:700}
c_{n+1} = \frac{z c_n}{\sqrt{n+1}}
\end{dmath}
%
\begin{equation}\label{eqn:qmLecture4:720}
\begin{aligned}
c_1 &= c_0 z \\
c_2 &= \frac{z c_1}{\sqrt{2}} = \frac{z^2 c_0}{\sqrt{2}} \\
\vdots &
\end{aligned}
\end{equation}

or
%
\begin{dmath}\label{eqn:qmLecture4:740}
c_n = \frac{z^n}{\sqrt{n!}}.
\end{dmath}
%
So the desired state is
%
\begin{dmath}\label{eqn:qmLecture4:760}
\ket{z} = c_0 \sum_{n=0}^\infty \frac{z^n}{\sqrt{n!}} \ket{n}.
\end{dmath}
%
Also recall that
%
\begin{dmath}\label{eqn:qmLecture4:780}
\ket{n} = \frac{\lr{ \hata^\dagger }^n}{\sqrt{n!}} \ket{0},
\end{dmath}
%
which gives
%
\begin{dmath}\label{eqn:qmLecture4:800}
\ket{z}
= c_0 \sum_{n=0}^\infty \frac{\lr{z \hata^\dagger}^n }{n!} \ket{0}
= c_0 e^{z \hata^\dagger}  \ket{0}.
\end{dmath}
%
The normalization is
%
\begin{dmath}\label{eqn:qmLecture4:820}
c_0 = e^{-\Abs{z}^2/2}.
\end{dmath}
%
While we have \( \braket{n_1}{n_2} = \delta_{n_1, n_2} \), these \( \ket{z} \) states are not orthonormal.  Figuring out that this overlap
%
\begin{dmath}\label{eqn:qmLecture4:840}
\braket{z_1}{z_2} \ne 0,
\end{dmath}
%
will be left for homework.

\section{Coherent state time evolution}
\index{coherent state!time evolution}

We don't know much about these coherent states.  For example does a coherent state at time zero evolve to a coherent state?
%
\begin{dmath}\label{eqn:qmLecture4:860}
\ket{z} \overset{?}{\rightarrow} \ket{z(t)}
\end{dmath}

\index{Heisenberg picture!coherent state}
It turns out that these questions are best tackled in the Heisenberg picture, considering
%
\begin{dmath}\label{eqn:qmLecture4:880}
e^{-i \hatH t/\Hbar } \ket{z}.
\end{dmath}
%
For example, what is the average of the position operator
%
\begin{dmath}\label{eqn:qmLecture4:900}
\bra{z} e^{i \hatH t/\Hbar } \hatx e^{-i \hatH t/\Hbar } \ket{z}
=
\sum_{n, n' = 0}^\infty
\bra{n} c_n^\conj e^{i E_n t/\Hbar}
\lr{ a + a^\dagger} \sqrt{ \frac{\Hbar}{m \omega} }
c_{n'} e^{i E_{n'} t/\Hbar}
\ket{n}.
\end{dmath}
%
This is very messy to attempt.  Instead if we know how the operator evolves we can calculate
%
\begin{dmath}\label{eqn:qmLecture4:920}
\bra{z} \hatx_\txtH(t) \ket{z},
\end{dmath}
%
that is
%
\begin{dmath}\label{eqn:qmLecture4:940}
\expectation{\hatx}(t) = \bra{z} \hatx_\txtH(t) \ket{z},
\end{dmath}
%
and for momentum
%
\begin{dmath}\label{eqn:qmLecture4:960}
\expectation{\hatp}(t) = \bra{z} \hatp_\txtH(t) \ket{z}.
\end{dmath}
%
The question to ask is what are the expansions of

\begin{subequations}
\label{eqn:qmLecture4:980}
\begin{dmath}\label{eqn:qmLecture4:1000}
\hata_\txtH(t) = e^{i \hatH t/\Hbar} \hata e^{-i \hatH t/\Hbar}.
\end{dmath}
\begin{dmath}\label{eqn:qmLecture4:1020}
\hata^\dagger_\txtH(t) = e^{i \hatH t/\Hbar} \hata^\dagger e^{-i \hatH t/\Hbar}.
\end{dmath}
\end{subequations}

The question to ask is how do these operators ask on the basis states
%
\begin{dmath}\label{eqn:qmLecture4:1040}
\hata_\txtH(t) \ket{n}
= e^{i \hatH t/\Hbar} \hata e^{-i \hatH t/\Hbar} \ket{n}
= e^{i \hatH t/\Hbar} \hata e^{-i t \omega (n + 1/2)} \ket{n}
=
e^{-i t \omega (n + 1/2)}
e^{i \hatH t/\Hbar}
\sqrt{n} \ket{n-1}
=
\sqrt{n}
e^{-i t \omega (n + 1/2)}
e^{i t \omega (n - 1/2)}
\ket{n-1}
=
\sqrt{n}  e^{-i \omega t} \ket{n-1}
=
e^{-i \omega t} \ket{n}.
\end{dmath}
%
So we have found
%
\begin{dmath}\label{eqn:qmLecture4:1060}
\begin{aligned}
\hata_\txtH(t) &= a e^{-i\omega t} \\
\hata^\dagger_\txtH(t) &= a^\dagger e^{i\omega t}
\end{aligned}
\end{dmath}

%\paragraph{Position and momentum operator time evolution}

%\EndArticle

      %\section{time evolution of coherent states, and charged particles in a magnetic field}
         %
% Copyright � 2015 Peeter Joot.  All Rights Reserved.
% Licenced as described in the file LICENSE under the root directory of this GIT repository.
%
%\input{../blogpost.tex}
%\renewcommand{\basename}{qmLecture5}
%\renewcommand{\dirname}{notes/phy1520/}
%\newcommand{\keywords}{PHY1520H}
%\input{../peeter_prologue_print2.tex}
%
%%\usepackage{phy1520}
%\usepackage{peeters_braket}
%%\usepackage{peeters_layout_exercise}
%\usepackage{peeters_figures}
%\usepackage{mathtools}
%
%\beginArtNoToc
%\generatetitle{PHY1520H Graduate Quantum Mechanics.  Lecture 5: time evolution of coherent states, and charged particles in a magnetic field.  Taught by Prof.\ Arun Paramekanti}
%\label{chap:qmLecture5}
%
%\paragraph{Disclaimer}
%
%Peeter's lecture notes from class.  These may be incoherent and rough.
%
%These are notes for the UofT course PHY1520, Graduate Quantum Mechanics, taught by Prof. Paramekanti, covering \textchapref{{1}} \citep{sakurai2014modern} content.
%
\section{Expectation with respect to coherent states.}

\index{expectation!coherent state}
A coherent state for the SHO \( H = \lr{ N + \inv{2} } \Hbar \omega \) was given by
%
\begin{equation}\label{eqn:qmLecture5:20}
a \ket{z} = z \ket{z},
\end{equation}
%
where we showed that
%
\begin{equation}\label{eqn:qmLecture5:40}
\ket{z} = c_0 e^{ z a^\dagger } \ket{0}.
\end{equation}
%
In the Heisenberg picture we found
%
\begin{equation}\label{eqn:qmLecture5:60}
\begin{aligned}
a_\txtH(t) &= e^{i H t/\Hbar} a e^{-i H t/\Hbar} = a e^{-i\omega t} \\
a_\txtH^\dagger(t) &= e^{i H t/\Hbar} a^\dagger e^{-i H t/\Hbar} = a^\dagger e^{i\omega t}.
\end{aligned}
\end{equation}
%
Recall that the position and momentum representation of the ladder operators was
%
\begin{equation}\label{eqn:qmLecture5:80}
\begin{aligned}
a &= \inv{\sqrt{2}} \lr{ \hatx \sqrt{\frac{m \omega}{\Hbar}} + i \hatp \sqrt{\inv{m \Hbar \omega}} } \\
a^\dagger &= \inv{\sqrt{2}} \lr{ \hatx \sqrt{\frac{m \omega}{\Hbar}} - i \hatp \sqrt{\inv{m \Hbar \omega}} },
\end{aligned}
\end{equation}
or equivalently
\begin{equation}\label{eqn:qmLecture5:100}
\begin{aligned}
\hatx &= \lr{ a + a^\dagger } \sqrt{\frac{\Hbar}{ 2 m \omega}} \\
\hatp &= i \lr{ a^\dagger - a } \sqrt{\frac{m \Hbar \omega}{2}}.
\end{aligned}
\end{equation}
%
Given this we can compute expectation value of position operator
%
\begin{dmath}\label{eqn:qmLecture5:120}
\bra{z} \hatx \ket{z}
=
\sqrt{\frac{\Hbar}{ 2 m \omega}}
\bra{z}
\lr{ a + a^\dagger }
\ket{z}
=
\lr{ z + z^\conj } \sqrt{\frac{\Hbar}{ 2 m \omega}}
=
2 \Real z \sqrt{\frac{\Hbar}{ 2 m \omega}} .
\end{dmath}
%
Similarly
%
\begin{dmath}\label{eqn:qmLecture5:140}
\bra{z} \hatp \ket{z}
=
i \sqrt{\frac{m \Hbar \omega}{2}}
\bra{z}
\lr{ a^\dagger - a }
\ket{z}
=
\sqrt{\frac{m \Hbar \omega}{2}}
2 \Imag z.
\end{dmath}
%
How about the expectation of the Heisenberg position operator?  That is
%
\begin{dmath}\label{eqn:qmLecture5:160}
\bra{z} \hatx_\txtH(t) \ket{z}
=
\sqrt{\frac{\Hbar}{2 m \omega}} \bra{z} \lr{ a + a^\dagger } \ket{z}
=
\sqrt{\frac{\Hbar}{2 m \omega}} \lr{ z e^{-i \omega t} + z^\conj e^{i \omega t}}
=
\sqrt{\frac{\Hbar}{2 m \omega}} \lr{ \lr{z + z^\conj} \cos( \omega t ) -i \lr{ z - z^\conj } \sin( \omega t) }
=
\sqrt{\frac{\Hbar}{2 m \omega}} \lr{ \expectation{x(0)} \sqrt{ \frac{2 m \omega}{\Hbar}} \cos( \omega t ) -i \expectation{p(0)} i \sqrt{\frac{2 m \omega}{\Hbar} } \sin( \omega t) }
=
\expectation{x(0)} \cos( \omega t ) + \frac{\expectation{p(0)}}{m \omega} \sin( \omega t) .
\end{dmath}
%
\index{coherent state!position operator}
We find that the average of the Heisenberg position operator evolves in time in exactly the same fashion as position in the classical Harmonic oscillator.  This phase space like trajectory is sketched in \cref{fig:zStateProjectionsAverageXandP:zStateProjectionsAverageXandPFig1}.

\imageFigure{../figures/phy1520-quantum/zStateProjectionsAverageXandPFig1}{Phase space like trajectory.}{fig:zStateProjectionsAverageXandP:zStateProjectionsAverageXandPFig1}{0.3}

In the text it is shown that we have the same structure for the Heisenberg operator itself, before taking expectations
%
\begin{dmath}\label{eqn:qmLecture5:220}
\hatx_\txtH(t)
=
{x(0)} \cos( \omega t ) + \frac{{p(0)}}{m \omega} \sin( \omega t).
\end{dmath}
%
Where the coherent states become useful is that we will see that the second moments of position and momentum are not time dependent with respect to the coherent states.  Such states remain localized.

\section{Coherent state uncertainty.}
\index{coherent state!uncertainty}

First note that using the commutator relationship we have
%
\begin{dmath}\label{eqn:qmLecture5:180}
\bra{z} a a^\dagger \ket{z}
=
\bra{z} \lr{ \antisymmetric{a}{a^\dagger} + a^\dagger a } \ket{z}
=
\bra{z} \lr{ 1 + a^\dagger a } \ket{z}.
\end{dmath}
%
For the second moment we have
%
\begin{dmath}\label{eqn:qmLecture5:200}
\bra{z} \hatx^2 \ket{z}
=
\frac{\Hbar}{ 2 m \omega}
\bra{z} \lr{a + a^\dagger } \lr{a + a^\dagger }  \ket{z}
=
\frac{\Hbar}{ 2 m \omega}
\bra{z} \lr{
a^2 + {(a^\dagger)}^2 + a a^\dagger + a^\dagger a
} \ket{z}
=
\frac{\Hbar}{ 2 m \omega}
\bra{z} \lr{
a^2 + {(a^\dagger)}^2 + 2 a^\dagger a + 1
} \ket{z}
=
\frac{\Hbar}{ 2 m \omega}
\lr{ z^2 + {(z^\conj)}^2 + 2 z^\conj z + 1}  \ket{z}
=
\frac{\Hbar}{ 2 m \omega}
\lr{ z + z^\conj }^2
+
\frac{\Hbar}{ 2 m \omega}.
\end{dmath}
%
We find
%
\begin{equation}\label{eqn:qmLecture5:240}
\sigma_x^2 = \frac{\Hbar}{ 2 m \omega},
\end{dmath}
%
and
%
\begin{equation}\label{eqn:qmLecture5:260}
\sigma_p^2 = \frac{m \Hbar \omega}{2}
\end{dmath}
so
%
\begin{equation}\label{eqn:qmLecture5:280}
\sigma_x^2 \sigma_p^2 = \frac{\Hbar^2}{4},
\end{dmath}
%
or
%
\begin{equation}\label{eqn:qmLecture5:300}
\sigma_x \sigma_p = \frac{\Hbar}{2}.
\end{dmath}
%
This is the minimum uncertainty.

\section{Quantum Field theory.}
\index{field theory}

In Quantum Field theory the ideas of isolated oscillators is used to model particle creation.  The lowest energy state (a no particle, vacuum state) is given the lowest energy level, with each additional quantum level modeling a new particle creation state as sketched in \cref{fig:qftEnergyLevels:qftEnergyLevelsFig2}.

\imageFigure{../figures/phy1520-quantum/qftEnergyLevelsFig2}{QFT energy levels.}{fig:qftEnergyLevels:qftEnergyLevelsFig2}{0.3}

We have to imagine many oscillators, each with a distinct vacuum energy \( \sim \Bk^2 \) .  The Harmonic oscillator can be used to model the creation of particles with \( \Hbar \omega \) energy differences from that ``vacuum energy''.

\section{Charged particle in a magnetic field.}
\index{magnetic field}

In the classical case ( with SI units or \( c = 1 \) ) we have
%
\begin{equation}\label{eqn:qmLecture5:320}
\BF = q \BE + q \Bv \cross \BB.
\end{dmath}
%
Alternately, we can look at the Hamiltonian view of the system, written in terms of potentials
%
\begin{equation}\label{eqn:qmLecture5:340}
\BB = \spacegrad \cross \BA,
\end{dmath}
\begin{equation}\label{eqn:qmLecture5:360}
\BE = - \spacegrad \phi - \PD{t}{\BA}.
\end{dmath}
%
Note that the curl form for the magnetic field implies one of the required Maxwell's equations \( \spacegrad \cdot \BB = 0 \).

Ignoring time dependence of the potentials, the Hamiltonian can be expressed as
%
\begin{equation}\label{eqn:qmLecture5:380}
H = \inv{2 m} \lr{ \Bp - q \BA }^2 + q \phi.
\end{dmath}
%
In this Hamiltonian the vector \( \Bp \) is called the canonical momentum, the momentum conjugate to position in phase space.

It is left as an exercise to show that the Lorentz force equation results from application of the Hamiltonian equations of motion, and that the velocity is given by \( \Bv = (\Bp - q \BA)/m \).

For quantum mechanics, we use the same Hamiltonian, but promote our position, momentum and potentials to operators.
%
\begin{equation}\label{eqn:qmLecture5:400}
\hatH = \inv{2 m} \lr{ \pcap - q \Acap(\Br, t) }^2 + q \hat\phi(\Br, t).
\end{dmath}
%
\section{Gauge invariance.}
\index{gauge invariance}

Can we say anything about this before looking at the question of a particle in a magnetic field?

Recall that the we can make a gauge transformation of the form

\begin{subequations}
\label{eqn:qmLecture5:420a}
\begin{equation}\label{eqn:qmLecture5:420}
\BA \rightarrow \BA + \spacegrad \chi,
\end{dmath}
\begin{equation}\label{eqn:qmLecture5:440}
\phi \rightarrow \phi - \PD{t}{\chi}.
\end{dmath}
\end{subequations}
Does this notion of gauge invariance also carry over to the Quantum Hamiltonian.  After gauge transformation we have
%
\begin{dmath}\label{eqn:qmLecture5:460}
\hatH'
= \inv{2 m} \lr{ \pcap - q \BA - q \spacegrad \chi }^2 + q \lr{ \phi - \PD{t}{\chi} }.
\end{dmath}
%
Now we are in a mess, since this function \( \chi \) can make the Hamiltonian horribly complicated.  We don't see how gauge invariance can easily be applied to the quantum problem.  Next time we will introduce a transformation that resolves some of this mess.
%
%\EndArticle

      %\section{Electromagnetic gauge transformation and Aharonov-Bohm effect}
         %
% Copyright � 2015 Peeter Joot.  All Rights Reserved.
% Licenced as described in the file LICENSE under the root directory of this GIT repository.
%
%\input{../blogpost.tex}
%\renewcommand{\basename}{qmLecture6}
%\renewcommand{\dirname}{notes/phy1520/}
%\newcommand{\keywords}{PHY1520H}
%\input{../peeter_prologue_print2.tex}
%
%%\usepackage{phy1520}
%\usepackage{peeters_braket}
%%\usepackage{peeters_layout_exercise}
%\usepackage{peeters_figures}
%\usepackage{mathtools}
%
%\beginArtNoToc
%\generatetitle{PHY1520H Graduate Quantum Mechanics.  Lecture 6: Electromagnetic gauge transformation and Aharonov-Bohm effect.  Taught by Prof.\ Arun Paramekanti}
%%\chapter{Electromagnetic gauge transformation and Aharonov-Bohm effect}
%\label{chap:qmLecture6}
%
%\paragraph{Disclaimer}
%
%Peeter's lecture notes from class.  These may be incoherent and rough.
%
%These are notes for the UofT course PHY1520, Graduate Quantum Mechanics, taught by Prof. Paramekanti, covering \textchapref{{2}} \citep{sakurai2014modern} content.
%
\paragraph{Particle with \( \BE, \BB \) fields}
\index{electric field}
\index{magnetic field}

We express our fields with vector and scalar potentials
%
\begin{dmath}\label{eqn:qmLecture6:20}
\BE, \BB \rightarrow \BA, \phi
\end{dmath}
%
and apply a gauge transformed Hamiltonian
%
\begin{dmath}\label{eqn:qmLecture6:40}
H = \inv{2m} \lr{ \Bp - q \BA }^2 + q \phi.
\end{dmath}
%
Recall that in classical mechanics we have
%
\begin{dmath}\label{eqn:qmLecture6:60}
\Bp - q \BA = m \Bv
\end{dmath}

where \( \Bp \) is not gauge invariant, but the classical momentum \( m \Bv \) is.

If given a point in phase space we must also specify the gauge that we are working with.

For the quantum case, temporarily considering a Hamiltonian without any scalar potential, but introducing a gauge transformation
%
\begin{dmath}\label{eqn:qmLecture6:80}
\BA \rightarrow \BA + \spacegrad \chi,
\end{dmath}
%
which takes the Hamiltonian from
%
\begin{dmath}\label{eqn:qmLecture6:100}
H = \inv{2m} \lr{ \Bp - q \BA }^2,
\end{dmath}
%
to
\begin{dmath}\label{eqn:qmLecture6:120}
H = \inv{2m} \lr{ \Bp - q \BA -q \spacegrad \chi }^2.
\end{dmath}
%
We care that the position and momentum operators obey
%
\begin{dmath}\label{eqn:qmLecture6:140}
\antisymmetric{\hatr_i}{\hatp_j} = i \Hbar \delta_{i j}.
\end{dmath}
%
We can apply a transformation that keeps \( \Br \) the same, but changes the momentum
%
\begin{dmath}\label{eqn:qmLecture6:160}
\begin{aligned}
\rcap' &= \rcap  \\
\pcap' &= \pcap  - q \spacegrad \chi(\Br)
\end{aligned}
\end{dmath}

This maps the Hamiltonian to
%
\begin{dmath}\label{eqn:qmLecture6:101}
H = \inv{2m} \lr{ \Bp' - q \BA -q \spacegrad \chi }^2,
\end{dmath}
%
We want to check if the commutator relationships have the desired structure, that is
%
\begin{dmath}\label{eqn:qmLecture6:180}
\begin{aligned}
\antisymmetric{r_i'}{r_j'} &= 0 \\
\antisymmetric{p_i'}{p_j'} &= 0
\end{aligned}
\end{dmath}

This is confirmed in \cref{problem:qmLecture6:1}.

Another thing of interest is how are the wave functions altered by this change of variables?  The wave functions must change in response to this transformation if the energies of the Hamiltonian are to remain the same.

Considering a plane wave specified by
%
\begin{dmath}\label{eqn:qmLecture6:200}
e^{i \Bk \cdot \Br},
\end{dmath}
%
where we alter the momentum by
%
\begin{dmath}\label{eqn:qmLecture6:220}
\Bk \rightarrow \Bk - e \spacegrad \chi.
\end{dmath}
%
This takes the plane wave to
%
\begin{dmath}\label{eqn:qmLecture6:240}
e^{i \lr{ \Bk - q \spacegrad \chi } \cdot \Br}.
\end{dmath}
%
We want to try to find a wave function for the new Hamiltonian
%
\begin{dmath}\label{eqn:qmLecture6:260}
H' = \inv{2m} \lr{ \Bp' - q \BA -q \spacegrad \chi }^2,
\end{dmath}
%
of the form
%
\begin{dmath}\label{eqn:qmLecture6:280}
\psi'(\Br)
\questionEquals
e^{i \theta(\Br)} \psi(\Br),
\end{dmath}
%
where the new wave function differs from a wave function for the original Hamiltonian by only a position dependent phase factor.

Let's look at the action of the Hamiltonian on the new wave function
%
\begin{dmath}\label{eqn:qmLecture6:300}
H' \psi'(\Br) .
\end{dmath}
%
Looking at just the first action
%
\begin{dmath}\label{eqn:qmLecture6:320}
\lr{ -i \Hbar \spacegrad - q \BA - q \spacegrad \chi } e^{i \theta(\Br)} \psi(\Br)
=
e^{i\theta}
\lr{ -i \Hbar \spacegrad - q \BA - q \spacegrad \chi }
\psi(\Br)
+
\lr{
-i \Hbar i \spacegrad \theta
}
e^{i\theta}
\psi(\Br)
=
e^{i\theta}
\lr{ -i \Hbar \spacegrad - q \BA - q \spacegrad \chi
+ \Hbar \spacegrad \theta
}
\psi(\Br).
\end{dmath}
%
If we choose
%
\begin{dmath}\label{eqn:qmLecture6:340}
\theta = \frac{q \chi}{\Hbar},
\end{dmath}
%
then we are left with
%
\begin{dmath}\label{eqn:qmLecture6:360}
\lr{ -i \Hbar \spacegrad - q \BA - q \spacegrad \chi } e^{i \theta(\Br)} \psi(\Br)
=
e^{i\theta}
\lr{ -i \Hbar \spacegrad - q \BA }
\psi(\Br).
\end{dmath}
%
Let \( \BM = -i \Hbar \spacegrad - q \BA \), and act again with \( \lr{ -i \Hbar \spacegrad - q \BA - q \spacegrad \chi } \)
%
\begin{dmath}\label{eqn:qmLecture6:700}
\lr{ -i \Hbar \spacegrad - q \BA - q \spacegrad \chi } e^{i \theta} \BM \psi
=
e^{i\theta}
\lr{ -i \Hbar i \spacegrad \theta - q \BA - q \spacegrad \chi } e^{i \theta} \BM \psi
+
e^{i\theta}
\lr{ -i \Hbar \spacegrad } \BM \psi
=
e^{i\theta}
\lr{ -i \Hbar \spacegrad -q \BA + \spacegrad \lr{ \Hbar \theta - q \chi} } \BM \psi
=
e^{i\theta} \BM^2 \psi.
\end{dmath}
%
Restoring factors of \( m \), we've shown that for a choice of \( \Hbar \theta - q \chi \), we have
%
\begin{dmath}\label{eqn:qmLecture6:400}
\inv{2m} \lr{ -i \Hbar \spacegrad - q \BA - q \spacegrad \chi }^2 e^{i \theta} \psi = e^{i\theta}
\inv{2m} \lr{ -i \Hbar \spacegrad - q \BA }^2 \psi.
\end{dmath}
%
When \( \psi \) is an energy eigenfunction, this means
%
\begin{equation}\label{eqn:qmLecture6:420}
H' e^{i\theta} \psi = e^{i \theta} H \psi = e^{i\theta} E\psi = E (e^{i\theta} \psi).
\end{equation}
%
We've found a transformation of the wave function that has the same energy eigenvalues as the corresponding wave functions for the original untransformed Hamiltonian.

In summary
%%\begin{equation}\label{eqn:qmLecture6:440}
%%\end{equation}
\boxedEquation{eqn:qmLecture6:440}{
\begin{aligned}
H' &= \inv{2m} \lr{ \Bp - q \BA - q \spacegrad \chi}^2 \\
\psi'(\Br) &= e^{i \theta(\Br)} \psi(\Br), \qquad \text{where}\, \theta(\Br) = q \chi(\Br)/\Hbar
\end{aligned}
}

\paragraph{Aharonov-Bohm effect}
\index{Aharonov-Bohm effect}

Consider a periodic motion in a fixed ring as sketched in \cref{fig:l6:l6Fig1}.

\imageFigure{../figures/phy1520-quantum/l6Fig1}{Particle confined to a ring.}{fig:l6:l6Fig1}{0.3}

Here the displacement around the perimeter is \( s = R \phi \) and the Hamiltonian
%
\begin{equation}\label{eqn:qmLecture6:460}
H = - \frac{\Hbar^2}{2 m} \PDSq{s}{} = - \frac{\Hbar^2}{2 m R^2} \PDSq{\phi}{}.
\end{equation}
%
Now assume that there is a magnetic field squeezed into the point at the origin, by virtue of a flux at the origin
%
\begin{equation}\label{eqn:qmLecture6:480}
\BB = \Phi_0 \delta(\Br) \zcap.
\end{equation}
%
We know that
%
\begin{equation}\label{eqn:qmLecture6:500}
\oint \BA \cdot d\Bl = \Phi_0,
\end{equation}
%
so that
%
\begin{equation}\label{eqn:qmLecture6:520}
\BA = \frac{\Phi_0}{2 \pi r} \phicap.
\end{equation}
%
The Hamiltonian for the new configuration is
%
\begin{equation}\label{eqn:qmLecture6:540}
H
= - \lr{ -i \Hbar \spacegrad - q \frac{\Phi_0}{2 \pi r } \phicap }^2
= - \inv{2 m} \lr{ -i \Hbar \inv{R} \PD{\phi}{} - q \frac{\Phi_0}{2 \pi R } }^2.
\end{equation}
%
Here the replacement \( r \rightarrow R \) makes use of the fact that this problem as been posed with the particle forced to move around the ring at the fixed radius \( R \).

For this transformed Hamiltonian, what are the wave functions?
%
\begin{dmath}\label{eqn:qmLecture6:560}
\psi(\phi)'
\questionEquals
% \inv{\sqrt{2 \pi}}
e^{i n \phi}.
\end{dmath}
%
\begin{dmath}\label{eqn:qmLecture6:580}
H \psi
= \inv{2 m}
\lr{ -i \Hbar \inv{R} (i n) - q \frac{\Phi_0}{2 \pi R } }^2 e^{i n \phi}
=
\mathLabelBox
[ labelstyle={below of=m\themathLableNode, below of=m\themathLableNode} ]
{\inv{2 m}
\lr{ \frac{\Hbar n}{R} - q \frac{\Phi_0}{2 \pi R } }^2}{\(E_n\)} e^{i n \phi}.
\end{dmath}
%
This is very unclassical, since the energy changes in a way that depends on the flux, because particles are seeing magnetic fields that are not present at the point of the particle.

This is sketched in \cref{fig:l6:l6Fig2}.

\imageFigure{../figures/phy1520-quantum/l6Fig2}{Energy variation with flux.}{fig:l6:l6Fig2}{0.3}

we see that there are multiple points that the energies hit the minimum levels


%\EndArticle
%\EndNoBibArticle

         %
% Copyright � 2015 Peeter Joot.  All Rights Reserved.
% Licenced as described in the file LICENSE under the root directory of this GIT repository.
%
%\input{../blogpost.tex}
%\renewcommand{\basename}{qmLecture7}
%\renewcommand{\dirname}{notes/phy1520/}
%\newcommand{\keywords}{PHY1520H}
%\input{../peeter_prologue_print2.tex}
%
%%\usepackage{phy1520}
%\usepackage{peeters_braket}
%%\usepackage{peeters_layout_exercise}
%\usepackage{peeters_figures}
%\usepackage{mathtools}
%\usepackage{mhchem}
%
%
%\beginArtNoToc
%\generatetitle{PHY1520H Graduate Quantum Mechanics.  Lecture 7: Aharonov-Bohm effect and Landau levels.  Taught by Prof.\ Arun Paramekanti}
%%\chapter{Aharonov-Bohm effect and Landau levels}
%\label{chap:qmLecture7}
%
%\paragraph{Disclaimer}
%
%Peeter's lecture notes from class.  These may be incoherent and rough.
%
%These are notes for the UofT course PHY1520, Graduate Quantum Mechanics, taught by Prof. Paramekanti, covering \textchapref{{1}} \citep{sakurai2014modern} content.
%
\paragraph{problem set note.}

In the problem set we'll look at interference patterns for two slit electron interference like that of \cref{fig:lecture7:lecture7Fig1}, where a magnetic whisker that introduces flux is added to the configuration.

\imageFigure{../phy1520-quantum-figureslecture7Fig1}{Two slit interference with magnetic whisker.}{fig:lecture7:lecture7Fig1}{0.2}

\paragraph{Aharonov-Bohm effect (cont.)}

Why do we have the zeros at integral multiples of \( h/q \)?  Consider a particle in a circular trajectory as sketched in \cref{fig:lecture7:lecture7Fig3}

\imageFigure{../phy1520-quantum-figureslecture7Fig3}{Circular trajectory.}{fig:lecture7:lecture7Fig3}{0.1}

FIXME: Prof mentioned:

\begin{dmath}\label{eqn:qmLecture7:20}
\phi_{\textrm{loop}} = q \frac{ h p/ q }{\Hbar} = 2 \pi p
\end{dmath}

... I'm not sure what that was about now.

In classical mechanics we have

\begin{dmath}\label{eqn:qmLecture7:40}
\oint p dq
\end{dmath}

The integral zero points are related to such a loop, but the \( q \BA \) portion of the momentum \( \Bp - q \BA \) needs to be considered.

\paragraph{Superconductors}
\index{superconductor}

After cooling some materials sufficiently, superconductivity, a complete lack of resistance to electrical flow can be observed.  A resistivity vs temperature plot of such a material is sketched in \cref{fig:lecture7:lecture7Fig4}.

\imageFigure{../phy1520-quantum-figureslecture7Fig4}{Superconductivity with comparison to superfluidity.}{fig:lecture7:lecture7Fig4}{0.2}

Just like \ce{He^4} can undergo Bose condensation, superconductivity can be explained by a hybrid Bosonic state where electrons are paired into one state containing integral spin.

The Little-Parks experiment puts a superconducting ring around a magnetic whisker as sketched in \cref{fig:lecture7:lecture7Fig6}.

\imageFigure{../phy1520-quantum-figureslecture7Fig6}{Little-Parks superconducting ring.}{fig:lecture7:lecture7Fig6}{0.1}

This experiment shows that the effective charge of the circulating charge was \( 2 e \), validating the concept of Cooper-pairing, the Bosonic combination (integral spin) of electrons in superconduction.

\paragraph{Motion around magnetic field}
\index{magnetic field}
\index{Little-Parks superconductor}

%F7
%\cref{fig:lecture7:lecture7Fig7}.
%\imageFigure{../phy1520-quantum-figureslecture7Fig7}{CAPTION: lecture7Fig7}{fig:lecture7:lecture7Fig7}{0.2}

\begin{dmath}\label{eqn:qmLecture7:140}
\omega_{\textrm{c}} = \frac{e B}{m}
\end{dmath}

We work with what is now called the Landau gauge

\begin{dmath}\label{eqn:qmLecture7:60}
\BA = \lr{ 0, B x, 0 }
\end{dmath}

This gives

\begin{dmath}\label{eqn:qmLecture7:80}
\BB
= \spacegrad \cross \BA
= \lr{ \partial_x A_y - \partial_y A_x } \zcap
= B \zcap.
\end{dmath}

An alternate gauge choice, the symmetric gauge, is

\begin{dmath}\label{eqn:qmLecture7:100}
\BA = \lr{ -\frac{B y}{2}, \frac{B x}{2}, 0 },
\end{dmath}

that also has the same magnetic field

\begin{dmath}\label{eqn:qmLecture7:120}
\BB
= \spacegrad \BA
= \lr{ \partial_x A_y - \partial_y A_x } \zcap
= \lr{ \frac{B}{2} - \lr{ - \frac{B}{2} } } \zcap
= B \zcap.
\end{dmath}

We expect the physics for each to have the same results, although the wave functions in one gauge may be more complicated than in the other.

Our Hamiltonian is

\begin{dmath}\label{eqn:qmLecture7:160}
H
= \inv{2 m} \lr{ \Bp - e \BA }^2
= \inv{2 m} \hatp_x^2 + \inv{2 m} \lr{ \hatp_y - e B \hatx }^2
\end{dmath}

We can solve after noting that

\begin{dmath}\label{eqn:qmLecture7:180}
\antisymmetric{\hatp_y}{H} = 0
\end{dmath}

means that

\begin{dmath}\label{eqn:qmLecture7:200}
\Psi(x,y) = e^{i k_y y} \phi(x)
\end{dmath}

The eigensystem

\begin{dmath}\label{eqn:qmLecture7:220}
H \psi(x, y) = E \phi(x, y) ,
\end{dmath}

becomes

\begin{dmath}\label{eqn:qmLecture7:240}
\lr{ \inv{2 m} \hatp_x^2 + \inv{2 m} \lr{ \Hbar k_y - e B \hatx}^2 } \phi(x)
= E \phi(x).
\end{dmath}

This reduced Hamiltonian can be rewritten as

\begin{dmath}\label{eqn:qmLecture7:320}
H_x
= \inv{2 m} p_x^2 + \inv{2 m} e^2 B^2 \lr{ \hatx - \frac{\Hbar k_y}{e B} }^2
\equiv \inv{2 m} p_x^2 + \inv{2} m \omega^2 \lr{ \hatx - x_0 }^2
\end{dmath}

where

\begin{dmath}\label{eqn:qmLecture7:260}
\inv{2 m} e^2 B^2 = \inv{2} m \omega^2,
\end{dmath}

or
\begin{dmath}\label{eqn:qmLecture7:280}
\omega = \frac{ e B}{m} \equiv \omega_\txtc.
\end{dmath}

and

\begin{dmath}\label{eqn:qmLecture7:300}
x_0 = \frac{\Hbar}{k_y}{e B}.
\end{dmath}

But what is this \( x_0 \)?  Because \( k_y \) is not really specified in this problem, we can consider that we have a zero point energy for every \( k_y \), but the oscillator position is shifted for every such value of \( k_y \).  For each set of energy levels \cref{fig:lecture7:lecture7Fig8} we can consider that there is a different zero point energy for each possible \( k_y \).

\imageFigure{../phy1520-quantum-figureslecture7Fig8}{Energy levels, and Energy vs flux.}{fig:lecture7:lecture7Fig8}{0.1}

\index{degeneracy}
This is an infinitely degenerate system with an infinite number of states for any given energy level.

This tells us that there is a problem, and have to reconsider the assumption that any \( k_y \) is acceptable.

To resolve this we can introduce periodic boundary conditions, imagining that a square is rotated in space forming a cylinder as sketched in \cref{fig:lecture7:lecture7Fig9}.

\imageFigure{../phy1520-quantum-figureslecture7Fig9}{Landau degeneracy region.}{fig:lecture7:lecture7Fig9}{0.1}

Requiring quantized momentum

\begin{dmath}\label{eqn:qmLecture7:340}
k_y L_y = 2 \pi n,
\end{dmath}

or

\begin{equation}\label{eqn:qmLecture7:360}
k_y = \frac{2 \pi n}{L_y}, \qquad n \in \bbZ,
\end{equation}

gives

\begin{dmath}\label{eqn:qmLecture7:380}
x_0(n) = \frac{\Hbar}{e B} \frac{ 2 \pi n}{L_y},
\end{dmath}

with \( x_0 \le L_x \).  The range is thus restricted to

\begin{dmath}\label{eqn:qmLecture7:400}
\frac{\Hbar}{e B} \frac{ 2 \pi n_{\textrm{max}}}{L_y} = L_x,
\end{dmath}

or

\begin{dmath}\label{eqn:qmLecture7:420}
n_{\textrm{max}} =
\mathLabelBox
[ labelstyle={below of=m\themathLableNode, below of=m\themathLableNode} ]
{L_x L_y}{area} \frac{ e B }{2 \pi \Hbar }
\end{dmath}

That is

\begin{dmath}\label{eqn:qmLecture7:440}
n_{\textrm{max}}
= \frac{\Phi_{\textrm{total}}}{h/e}
= \frac{\Phi_{\textrm{total}}}{\Phi_0}.
\end{dmath}

%F10
%\cref{fig:lecture7:lecture7Fig10}.
%\imageFigure{../phy1520-quantum-figureslecture7Fig10}{CAPTION: lecture7Fig10}{fig:lecture7:lecture7Fig10}{0.2}

Attempting to measure Hall-effect systems, it was found that the Hall conductivity was quantized like

\begin{dmath}\label{eqn:qmLecture7:460}
\sigma_{x y} = p \frac{e^2}{h}.
\end{dmath}

\index{Landau levels}
This quantization is explained by these Landau levels, and this experimental apparatus provides one of the more accurate ways to measure the fine structure constant.

%\cref{fig:lecture7:lecture7Fig11}.
%\imageFigure{../phy1520-quantum-figureslecture7Fig11}{CAPTION: lecture7Fig11}{fig:lecture7:lecture7Fig11}{0.2}

%\EndArticle

      \section{Diagonalizating the Quantum Harmonic Oscillator.}
         %
% Copyright � 2015 Peeter Joot.  All Rights Reserved.
% Licenced as described in the file LICENSE under the root directory of this GIT repository.
%
%\input{../blogpost.tex}
%\renewcommand{\basename}{harmonicOscDiagonalize}
%\renewcommand{\dirname}{notes/phy1520/}
%%\newcommand{\dateintitle}{}
%%\newcommand{\keywords}{}
%
%\input{../peeter_prologue_print2.tex}
%
%\usepackage{peeters_layout_exercise}
%\usepackage{peeters_braket}
%\usepackage{peeters_figures}
%
%\beginArtNoToc
%
%\generatetitle{Quantum SHO ladder operators as a diagonal change of basis for the Heisenberg EOMs}
%\generatetitle{Diagonalizing the Quantum Harmonic Oscillator}
%\label{chap:harmonicOscDiagonalize}

Many authors pull the definitions of the raising and lowering (or ladder) operators out of their butt with no attempt at motivation.  This is pointed out nicely in \citep{eli:quantumLadderOperators} by Eli along with one justification based on factoring the Hamiltonian.

In \citep{sakurai2014modern:shoTimeEvolution} is a small exception to the usual presentation.  In that text, these operators are defined as usual with no motivation.  However, after the utility of these operators has been shown, the raising and lowering operators show up in a context that does provide that missing motivation as a side effect.
It doesn't look like the author was trying to provide a motivation, but it can be interpreted that way.

When seeking the time evolution of Heisenberg-picture position and momentum operators, we will see that those solutions can be trivially expressed using the raising and lowering operators.  No special tools nor black magic is required to find the structure of these operators.  Unfortunately, we must first switch to both the Heisenberg picture representation of the position and momentum operators, and also employ the Heisenberg equations of motion.  Neither of these last two fit into standard narrative of most introductory quantum mechanics treatments.  We will also see that these raising and lowering ``operators'' could also be introduced in classical mechanics, provided we were attempting to solve the SHO system using the Hamiltonian equations of motion.

I'll outline this route to finding the structure of the ladder operators below.  Because these are encountered trying to solve the time evolution problem, I'll first show a simpler way to solve that problem.  Because that simpler method depends a bit on lucky observation and is somewhat unstructured, I'll then outline a more structured procedure that leads to the ladder operators directly, also providing the solution to the time evolution problem as a side effect.

The starting point is the Heisenberg equations of motion.  For a time independent Hamiltonian \( H \), and a Heisenberg operator \( A^{(H)} \), those equations are
%
\begin{dmath}\label{eqn:harmonicOscDiagonalize:20}
\ddt{A^{(H)}} = \inv{i \Hbar} \antisymmetric{A^{(H)}}{H}.
\end{dmath}

Here the Heisenberg operator \( A^{(H)} \) is related to the Schr\"{o}dinger operator \( A^{(S)} \) by
%
\begin{dmath}\label{eqn:harmonicOscDiagonalize:60}
A^{(H)} = U^\dagger A^{(S)} U,
\end{dmath}
%
where \( U \) is the time evolution operator.  For this discussion, we need only know that \( U \) commutes with \( H \), and do not need to know the specific structure of that operator.  In particular, the Heisenberg equations of motion take the form
%
\begin{dmath}\label{eqn:harmonicOscDiagonalize:80}
\ddt{A^{(H)}}
= \inv{i \Hbar}
\antisymmetric{A^{(H)}}{H}
= \inv{i \Hbar}
\antisymmetric{U^\dagger A^{(S)} U}{H}
= \inv{i \Hbar}
\lr{
U^\dagger A^{(S)} U H
- H U^\dagger A^{(S)} U
}
= \inv{i \Hbar}
\lr{
U^\dagger A^{(S)} H U
- U^\dagger H A^{(S)} U
}
= \inv{i \Hbar} U^\dagger \antisymmetric{A^{(S)}}{H} U.
\end{dmath}

The Hamiltonian for the harmonic oscillator, with Schr\"{o}dinger-picture position and momentum operators \( x, p \) is
%
\begin{dmath}\label{eqn:harmonicOscDiagonalize:40}
H = \frac{p^2}{2m} + \inv{2} m \omega^2 x^2,
\end{dmath}
%
so the equations of motions are
%
\begin{dmath}\label{eqn:harmonicOscDiagonalize:100}
\ddt{x^{(H)}}
= \inv{i \Hbar} U^\dagger \antisymmetric{x}{H} U
= \inv{i \Hbar} U^\dagger \antisymmetric{x}{\frac{p^2}{2m}} U
= \inv{2 m i \Hbar} U^\dagger \lr{ i \Hbar \PD{p}{p^2} } U
= \inv{m } U^\dagger p U
= \inv{m } p^{(H)},
\end{dmath}
%
and
\begin{dmath}\label{eqn:harmonicOscDiagonalize:120}
\ddt{p^{(H)}}
= \inv{i \Hbar} U^\dagger \antisymmetric{p}{H} U
= \inv{i \Hbar} U^\dagger \antisymmetric{p}{\inv{2} m \omega^2 x^2 } U
= \frac{m \omega^2}{2 i \Hbar} U^\dagger \lr{ -i \Hbar \PD{x}{x^2} } U
= -m \omega^2 U^\dagger x U
= -m \omega^2 x^{(H)}.
\end{dmath}

In the Heisenberg picture the equations of motion are precisely those of classical Hamiltonian mechanics, except that we are dealing with operators instead of scalars
%
\begin{dmath}\label{eqn:harmonicOscDiagonalize:140}
\begin{aligned}
\ddt{p^{(H)}} &= -m \omega^2 x^{(H)} \\
\ddt{x^{(H)}} &= \inv{m } p^{(H)}.
\end{aligned}
\end{dmath}

In the text the ladder operators are used to simplify the solution of these coupled equations, since they can decouple them.  That's not really required since we can solve them directly in matrix form with little work
%
\begin{dmath}\label{eqn:harmonicOscDiagonalize:160}
\ddt{}
\begin{bmatrix}
p^{(H)} \\
x^{(H)}
\end{bmatrix}
=
\begin{bmatrix}
0 & -m \omega^2 \\
\inv{m} & 0
\end{bmatrix}
\begin{bmatrix}
p^{(H)} \\
x^{(H)}
\end{bmatrix},
\end{dmath}
%
or, with length scaled variables
%
\begin{dmath}\label{eqn:harmonicOscDiagonalize:180}
\ddt{}
\begin{bmatrix}
\frac{p^{(H)}}{m \omega} \\
x^{(H)}
\end{bmatrix}
=
\begin{bmatrix}
0 & -\omega \\
\omega & 0
\end{bmatrix}
\begin{bmatrix}
\frac{p^{(H)}}{m \omega} \\
x^{(H)}
\end{bmatrix}
=
-i \omega
\PauliY
\begin{bmatrix}
\frac{p^{(H)}}{m \omega} \\
x^{(H)}
\end{bmatrix}
=
-i \omega
\sigma_y
\begin{bmatrix}
\frac{p^{(H)}}{m \omega} \\
x^{(H)}
\end{bmatrix}.
\end{dmath}

Writing \( y = \begin{bmatrix} \frac{p^{(H)}}{m \omega} \\ x^{(H)} \end{bmatrix} \), the solution can then be written immediately as
%
\begin{dmath}\label{eqn:harmonicOscDiagonalize:200}
y(t)
=
\exp\lr{ -i \omega \sigma_y t } y(0)
=
\lr{ \cos \lr{ \omega t } I - i \sigma_y \sin\lr{ \omega t } } y(0)
=
\begin{bmatrix}
\cos\lr{ \omega t } & \sin\lr{ \omega t } \\
-\sin\lr{ \omega t } & \cos\lr{ \omega t }
\end{bmatrix}
y(0),
\end{dmath}
%
or
%
\begin{dmath}\label{eqn:harmonicOscDiagonalize:220}
\begin{aligned}
\frac{p^{(H)}(t)}{m \omega} &= \cos\lr{ \omega t } \frac{p^{(H)}(0)}{m \omega} + \sin\lr{ \omega t } x^{(H)}(0) \\
x^{(H)}(t) &= -\sin\lr{ \omega t } \frac{p^{(H)}(0)}{m \omega} + \cos\lr{ \omega t } x^{(H)}(0).
\end{aligned}
\end{dmath}

This solution depends on being lucky enough to recognize that the matrix has a Pauli matrix as a factor (which squares to unity, and allows the exponential to be evaluated easily.)

If we hadn't been that observant, then the first tool we'd have used instead would have been to diagonalize the matrix.  For such diagonalization, it's natural to work in completely dimensionless variables.  Such a non-dimensionalisation can be had by defining
%
\begin{dmath}\label{eqn:harmonicOscDiagonalize:240}
x_0 = \sqrt{\frac{\Hbar}{m \omega}},
\end{dmath}
%
and dividing the working (operator) variables through by those values.  Let \( z = \inv{x_0} y \), and \( \tau = \omega t \) so that the equations of motion are
%
\begin{dmath}\label{eqn:harmonicOscDiagonalize:260}
\frac{dz}{d\tau}
=
\begin{bmatrix}
0 & -1 \\
1 & 0
\end{bmatrix}
z.
\end{dmath}

This matrix can be diagonalized as
%
\begin{dmath}\label{eqn:harmonicOscDiagonalize:280}
A =
\begin{bmatrix}
0 & -1 \\
1 & 0
\end{bmatrix}
=
V
\begin{bmatrix}
i & 0  \\
0 & -i
\end{bmatrix}
V^{-1},
\end{dmath}
%
where
%
\begin{dmath}\label{eqn:harmonicOscDiagonalize:300}
V =
\inv{\sqrt{2}}
\begin{bmatrix}
i & -i \\
1 & 1
\end{bmatrix}.
\end{dmath}

The equations of motion can now be written
%
\begin{dmath}\label{eqn:harmonicOscDiagonalize:320}
\frac{d}{d\tau} \lr{ V^{-1} z } =
\begin{bmatrix}
i & 0  \\
0 & -i
\end{bmatrix}
\lr{ V^{-1} z }.
\end{dmath}

This final change of variables \( V^{-1} z \) decouples the system as desired.  Expanding that gives
%
\begin{dmath}\label{eqn:harmonicOscDiagonalize:340}
V^{-1} z
=
\inv{\sqrt{2}}
\begin{bmatrix}
-i & 1 \\
 i & 1
\end{bmatrix}
\begin{bmatrix}
\frac{p^{(H)}}{x_0 m \omega} \\
\frac{x^{(H)}}{x_0}
\end{bmatrix}
=
\inv{\sqrt{2} x_0}
\begin{bmatrix}
-i \frac{p^{(H)}}{m \omega} + x^{(H)} \\
i \frac{p^{(H)}}{m \omega} + x^{(H)}
\end{bmatrix}
=
\begin{bmatrix}
a^\dagger \\
a
\end{bmatrix},
\end{dmath}
%
where
\begin{dmath}\label{eqn:harmonicOscDiagonalize:400}
\begin{aligned}
a^\dagger &= \sqrt{\frac{m \omega}{2 \Hbar}} \lr{ -i \frac{p^{(H)}}{m \omega} + x^{(H)} } \\
a &= \sqrt{\frac{m \omega}{2 \Hbar}} \lr{ i \frac{p^{(H)}}{m \omega} + x^{(H)} }.
\end{aligned}
\end{dmath}

Lo and behold, we have the standard form of the raising and lowering operators, and can write the system equations as
%
\begin{dmath}\label{eqn:harmonicOscDiagonalize:360}
\begin{aligned}
\ddt{a^\dagger} &= i \omega a^\dagger \\
\ddt{a} &= -i \omega a.
\end{aligned}
\end{dmath}

It is actually a bit fluky that this matched exactly, since we could have chosen eigenvectors that differ by constant phase factors, like
%
\begin{dmath}\label{eqn:harmonicOscDiagonalize:380}
V = \inv{\sqrt{2}}
\begin{bmatrix}
i e^{i\phi} & -i e^{i \psi} \\
1 e^{i\phi} & e^{i \psi}
\end{bmatrix},
\end{dmath}
%
so
%
\begin{dmath}\label{eqn:harmonicOscDiagonalize:341}
V^{-1} z
=
\frac{e^{-i(\phi + \psi)}}{\sqrt{2}}
\begin{bmatrix}
-i e^{i\psi} & e^{i \psi} \\
i e^{i\phi} & e^{i \phi}
\end{bmatrix}
\begin{bmatrix}
\frac{p^{(H)}}{x_0 m \omega} \\
\frac{x^{(H)}}{x_0}
\end{bmatrix}
=
\inv{\sqrt{2} x_0}
\begin{bmatrix}
-i e^{i\phi} \frac{p^{(H)}}{m \omega} + e^{i\phi} x^{(H)} \\
i e^{i\psi} \frac{p^{(H)}}{m \omega} + e^{i\psi} x^{(H)}
\end{bmatrix}
=
\begin{bmatrix}
e^{i\phi} a^\dagger \\
e^{i\psi} a
\end{bmatrix}.
\end{dmath}

To make the resulting pairs of operators Hermitian conjugates, we'd want to constrain those constant phase factors by setting \( \phi = -\psi \).  If we were only interested in solving the time evolution problem no such additional constraints are required.

The raising and lowering operators are seen to naturally occur when seeking the solution of the Heisenberg equations of motion.  This is found using the standard technique of non-dimensionalisation and then seeking a change of basis that diagonalizes the system matrix.  Because the Heisenberg equations of motion are identical to the classical Hamiltonian equations of motion in this case, what we call the raising and lowering operators in quantum mechanics could also be utilized in the classical simple harmonic oscillator problem.  However, in a classical context we wouldn't have a justification to call this more than a change of basis.

%\EndArticle

      \section{Constant magnetic solenoid field.}
         %
% Copyright � 2015 Peeter Joot.  All Rights Reserved.
% Licenced as described in the file LICENSE under the root directory of this GIT repository.
%
%\input{../blogpost.tex}
%\renewcommand{\basename}{solenoidConstantField}
%\renewcommand{\dirname}{notes/phy1520/}
%%\newcommand{\dateintitle}{}
%%\newcommand{\keywords}{}
%
%\input{../peeter_prologue_print2.tex}
%
%\usepackage{peeters_layout_exercise}
%%\usepackage{peeters_braket}
%\usepackage{peeters_figures}
%
%\beginArtNoToc
%
%\generatetitle{Constant magnetic solenoid field}
%\label{chap:solenoidConstantField}

In \citep{sakurai2014modern} the following vector potential

\begin{dmath}\label{eqn:solenoidConstantField:20}
\BA = \frac{B \rho_a^2}{2 \rho} \phicap,
\end{dmath}

is introduced in a discussion on the Aharonov-Bohm effect, for configurations where the interior field of a solenoid is either a constant \( \BB \) or zero.

I wasn't able to make sense of this since the field I was calculating was zero for all \( \rho \ne 0 \)

\begin{dmath}\label{eqn:solenoidConstantField:40}
\BB
= \spacegrad \cross \BA
= \lr{ \rhocap \partial_\rho + \zcap \partial_z + \frac{\phicap}{\rho} \partial_\phi } \cross \frac{B \rho_a^2}{2 \rho} \phicap
= \lr{ \rhocap \partial_\rho + \frac{\phicap}{\rho} \partial_\phi } \cross \frac{B \rho_a^2}{2 \rho} \phicap
=
\frac{B \rho_a^2}{2}
\rhocap \cross \phicap \partial_\rho \lr{ \inv{\rho} }
+
\frac{B \rho_a^2}{2 \rho}
\frac{\phicap}{\rho} \cross \partial_\phi \phicap
=
\frac{B \rho_a^2}{2 \rho^2} \lr{ -\zcap + \phicap \cross \partial_\phi \phicap}.
\end{dmath}

Note that the \( \rho \) partial requires that \( \rho \ne 0 \).  To expand the cross product in the second term let \( j = \Be_1 \Be_2 \), and expand using a Geometric Algebra representation of the unit vector

\begin{dmath}\label{eqn:solenoidConstantField:60}
\phicap \cross \partial_\phi \phicap
=
\Be_2 e^{j \phi} \cross \lr{ \Be_2 \Be_1 \Be_2 e^{j \phi} }
=
- \Be_1 \Be_2 \Be_3
\gpgradetwo{
\Be_2 e^{j \phi} (-\Be_1) e^{j \phi}
}
=
\Be_1 \Be_2 \Be_3 \Be_2 \Be_1
= \Be_3
= \zcap.
\end{dmath}

So, provided \( \rho \ne 0 \), \( \BB = 0 \).

The errata \citep{sakurai2014modernErrata} provides the clarification, showing that a \( \rho > \rho_a \) constraint is required for this potential to produce the desired results.  Continuity at \( \rho = \rho_a \) means that in the interior (or at least on the boundary) we must have one of

\begin{dmath}\label{eqn:solenoidConstantField:80}
\BA = \frac{B \rho_a}{2} \phicap,
\end{dmath}

or

\begin{dmath}\label{eqn:solenoidConstantField:100}
\BA = \frac{B \rho}{2} \phicap.
\end{dmath}

The first doesn't work, but the second does

\begin{dmath}\label{eqn:solenoidConstantField:120}
\BB
= \spacegrad \cross \BA
= \lr{ \rhocap \partial_\rho + \zcap \partial_z + \frac{\phicap}{\rho} \partial_\phi } \cross \frac{B \rho}{2 } \phicap
=
\frac{B }{2 } \rhocap \cross \phicap
+
\frac{B \rho}{2 }
\frac{\phicap}{\rho} \cross \partial_\phi \phicap
= B \zcap.
\end{dmath}

So the vector potential that we want for a constant \( B \zcap \) field in the interior \( \rho < \rho_a \) of a cylindrical space, we need

\begin{dmath}\label{eqn:solenoidConstantField:140}
\BA =
\left\{
\begin{array}{l l}
\frac{B \rho_a^2}{2 \rho} \phicap & \quad \mbox{if \( \rho \ge \rho_a \) } \\
\frac{B \rho}{2} \phicap & \quad \mbox{if \( \rho < \rho_a \).}
\end{array}
\right.
\end{dmath}

An example of the magnitude of potential is graphed in \cref{fig:solenoidPotential:solenoidPotentialFig1}.

\mathImageFigure{../figures/phy1520-quantum/solenoidPotentialFig1}{Vector potential for constant field in cylindrical region.}{fig:solenoidPotential:solenoidPotentialFig1}{0.3}{vectorSolenoid.jl}

%\EndArticle

      \section{Lagrangian for magnetic portion of Lorentz force.}
         %
% Copyright � 2015 Peeter Joot.  All Rights Reserved.
% Licenced as described in the file LICENSE under the root directory of this GIT repository.
%
%\input{../blogpost.tex}
%\renewcommand{\basename}{magneticLorentzForceLagrangian}
%\renewcommand{\dirname}{notes/phy1520/}
%%\newcommand{\dateintitle}{}
%%\newcommand{\keywords}{}
%
%\input{../peeter_prologue_print2.tex}
%
%\usepackage{peeters_layout_exercise}
%\usepackage{peeters_braket}
%\usepackage{peeters_figures}
%
%\beginArtNoToc

%\generatetitle{Lagrangian for magnetic portion of Lorentz force}
%\label{chap:magneticLorentzForceLagrangian}

In \citep{sakurai2014modern} it is claimed in an Aharonov-Bohm discussion that a Lagrangian modification to include electromagnetism is
%
\begin{equation}\label{eqn:magneticLorentzForceLagrangian:20}
\LL \rightarrow \LL + \frac{e}{c} \Bv \cdot \BA.
\end{equation}
%
That can't be the full Lagrangian since there is no \( \phi \) term, so what exactly do we get?

If you have somehow, like I did, forgot the exact form of the Euler-Lagrange equations (i.e. where do the dots go), then the derivation of those equations can come to your rescue.  The starting point is the action
%
\begin{equation}\label{eqn:magneticLorentzForceLagrangian:40}
S = \int \LL(x, \xdot, t) dt,
\end{equation}
%
where the end points of the integral are fixed, and we assume we have no variation at the end points.  The variational calculation is
%
\begin{dmath}\label{eqn:magneticLorentzForceLagrangian:60}
\delta S
= \int \delta \LL(x, \xdot, t) dt
= \int \lr{ \PD{x}{\LL} \delta x + \PD{\xdot}{\LL} \delta \xdot } dt
= \int \lr{ \PD{x}{\LL} \delta x + \PD{\xdot}{\LL} \delta \ddt{x} } dt
= \int \lr{ \PD{x}{\LL} - \ddt{}\lr{\PD{\xdot}{\LL}} } \delta x dt
+ \delta x \PD{\xdot}{\LL}.
\end{dmath}
%
The boundary term is killed after evaluation at the end points where the variation is zero.  For the result to hold for all variations \( \delta x \), we must have
%
%\begin{dmath}\label{eqn:magneticLorentzForceLagrangian:80}
\boxedEquation{eqn:magneticLorentzForceLagrangian:80}{
\PD{x}{\LL} = \ddt{}\lr{\PD{\xdot}{\LL}}.
}
%\end{dmath}
%
Now lets apply this to the Lagrangian at hand.  For the position derivative we have
%
\begin{dmath}\label{eqn:magneticLorentzForceLagrangian:100}
\PD{x_i}{\LL}
=
\frac{e}{c} v_j \PD{x_i}{A_j}.
\end{dmath}
%
For the canonical momentum term, assuming \( \BA = \BA(\Bx) \) we have
%
\begin{dmath}\label{eqn:magneticLorentzForceLagrangian:120}
\ddt{} \PD{\xdot_i}{\LL}
=
\ddt{}
\lr{ m \xdot_i
+
\frac{e}{c} A_i
}
=
m \ddot{x}_i
+
\frac{e}{c}
\ddt{A_i}
=
m \ddot{x}_i
+
\frac{e}{c}
\PD{x_j}{A_i} \frac{dx_j}{dt}.
\end{dmath}
%
Assembling the results, we've got
%
\begin{dmath}\label{eqn:magneticLorentzForceLagrangian:140}
0
=
\ddt{} \PD{\xdot_i}{\LL}
-
\PD{x_i}{\LL}
=
m \ddot{x}_i
+
\frac{e}{c}
\PD{x_j}{A_i} \frac{dx_j}{dt}
-
\frac{e}{c} v_j \PD{x_i}{A_j},
\end{dmath}
%
or
\begin{dmath}\label{eqn:magneticLorentzForceLagrangian:160}
m \ddot{x}_i
=
\frac{e}{c} v_j \PD{x_i}{A_j}
-
\frac{e}{c}
\PD{x_j}{A_i} v_j
=
\frac{e}{c} v_j
\lr{
\PD{x_i}{A_j}
-
\PD{x_j}{A_i}
}
=
\frac{e}{c} v_j B_k \epsilon_{i j k}.
\end{dmath}
%
In vector form that is
%
%\begin{dmath}\label{eqn:magneticLorentzForceLagrangian:180}
\boxedEquation{eqn:magneticLorentzForceLagrangian:180}{
m \ddot{\Bx}
=
\frac{e}{c} \Bv \cross \BB.
}
%\end{dmath}
%
So, we get the magnetic term of the Lorentz force.  Also note that this shows the Lagrangian (and the end result), was not in SI units.  The \( 1/c \) term would have to be dropped for SI.

%\EndArticle

      \section{Problems.}
         %
% Copyright © 2015 Peeter Joot.  All Rights Reserved.
% Licenced as described in the file LICENSE under the root directory of this GIT repository.
%
\makeproblem{Lorentz force, electrodynamic Hamiltonian.}{problem:qmLecture5:1}{
\index{classical Hamiltonian!Lorentz force}
Given the classical Hamiltonian
%
\begin{equation}\label{eqn:qmLecture5:381}
H = \inv{2 m} \lr{ \Bp - q \BA }^2 + q \phi.
\end{dmath}
%
apply the Hamiltonian equations of motion
%
\begin{equation}\label{eqn:qmLecture5:480}
\begin{aligned}
\ddt{\Bp} &= - \PD{\Bq}{H} \\
\ddt{\Bq} &= \PD{\Bp}{H},
\end{aligned}
\end{equation}

to show that this is the Hamiltonian that describes the Lorentz force equation, and to find the velocity in terms of the canonical momentum and vector potential.
} % problem
%
\makeanswer{problem:qmLecture5:1}{
%
The particle velocity follows easily
%
\begin{dmath}\label{eqn:qmLecture5:500}
\Bv
= \ddt{\Br}
= \PD{\Bp}{H}
= \inv{m} \lr{ \Bp - q \BA }.
\end{dmath}
%
For the Lorentz force we can proceed in the coordinate representation
%
\begin{dmath}\label{eqn:qmLecture5:520}
\ddt{p_k}
= - \PD{x_k}{H}
= - \frac{2}{2m} \lr{ p_m - q A_m } \PD{x_k}{}\lr{ p_m - q A_m } - q \PD{x_k}{\phi}
= q v_m \PD{x_k}{A_m} - q \PD{x_k}{\phi}.
\end{dmath}
%
We also have
%
\begin{dmath}\label{eqn:qmLecture5:540}
\ddt{p_k}
=
\ddt{} \lr{m x_k + q A_k }
=
m \frac{d^2 x_k}{dt^2} + q \PD{x_m}{A_k} \frac{d x_m}{dt} + q \PD{t}{A_k}.
\end{dmath}
%
Putting these together we've got
%
\begin{dmath}\label{eqn:qmLecture5:560}
m \frac{d^2 x_k}{dt^2}
= q v_m \PD{x_k}{A_m} - q \PD{x_k}{\phi},
- q \PD{x_m}{A_k} \frac{d x_m}{dt} - q \PD{t}{A_k}
=
q v_m \lr{ \PD{x_k}{A_m} - \PD{x_m}{A_k} } + q E_k
=
q v_m \epsilon_{k m s} B_s + q E_k,
\end{dmath}
%
or
%
\begin{dmath}\label{eqn:qmLecture5:580}
m \frac{d^2 \Bx}{dt^2}
=
q \Be_k v_m \epsilon_{k m s} B_s + q E_k
= q \Bv \cross \BB + q \BE.
\end{dmath}
%
} % answer
%
\makeproblem{Show gauge invariance of the magnetic and electric fields.}{problem:qmLecture5:2}{
\index{gauge invariance}

After the gauge transformation of \cref{eqn:qmLecture5:420a} show that the electric and magnetic fields are unaltered.
} % problem
%
\makeanswer{problem:qmLecture5:2}{
%
For the magnetic field the transformed field is
%
\begin{dmath}\label{eqn:qmLecture5:600}
\BB'
= \spacegrad \cross \lr{ \BA + \spacegrad \chi }
= \spacegrad \cross \BA + \spacegrad \cross \lr{ \spacegrad \chi }
= \spacegrad \cross \BA
= \BB.
\end{dmath}
%
\begin{dmath}\label{eqn:qmLecture5:620}
\BE'
=
- \PD{t}{\BA'} - \spacegrad \phi'
=
- \PD{t}{}\lr{\BA + \spacegrad \chi} - \spacegrad \lr{ \phi - \PD{t}{\chi}}
=
- \PD{t}{\BA} - \spacegrad \phi
=
\BE.
\end{dmath}
%
} % answer

         %
% Copyright © 2015 Peeter Joot.  All Rights Reserved.
% Licenced as described in the file LICENSE under the root directory of this GIT repository.
%
\makeproblem{Gauge transformation.}{problem:qmLecture6:1}{
\index{gauge transformation}

Show that after a transformation of position and momentum of the following form
%
\begin{dmath}\label{eqn:qmLecture6:600}
\begin{aligned}
\rcap' &= \rcap  \\
\pcap' &= \pcap  - q \spacegrad \chi(\Br)
\end{aligned}
\end{dmath}

all the commutators have the expected values.
} % problem

\makeanswer{problem:qmLecture6:1}{

The position commutators don't need consideration.  Of interest is the momentum-position commutators
%
\begin{dmath}\label{eqn:qmLecture6:620}
\antisymmetric{\hatp_k'}{\hatx_k'}
=
\antisymmetric{\hatp_k - q \partial_k \chi}{\hatx_k}
=
\antisymmetric{\hatp_k}{\hatx_k} - q \antisymmetric{\partial_k \chi}{\hatx_k}
=
\antisymmetric{\hatp_k}{\hatx_k},
\end{dmath}

and the momentum commutators
%
\begin{dmath}\label{eqn:qmLecture6:640}
\antisymmetric{\hatp_k'}{\hatp_j'}
=
\antisymmetric{\hatp_k - q \partial_k \chi}{\hatp_j - q \partial_j \chi}
=
\antisymmetric{\hatp_k}{\hatp_j}
- q \lr{ \antisymmetric{\partial_k \chi}{\hatp_j} + \antisymmetric{\hatp_k}{\partial_j \chi} }.
\end{dmath}

That last sum of commutators is
%
\begin{dmath}\label{eqn:qmLecture6:660}
\antisymmetric{\partial_k \chi}{\hatp_j} + \antisymmetric{\hatp_k}{\partial_j \chi}
=
- i \Hbar \lr{ \PD{k}{(\partial_j \chi)} -  \PD{j}{(\partial_k \chi)} }
= 0.
\end{dmath}

We've shown that
%
\begin{dmath}\label{eqn:qmLecture6:680}
\begin{aligned}
\antisymmetric{\hatp_k'}{\hatx_k'} &= \antisymmetric{\hatp_k}{\hatx_k} \\
\antisymmetric{\hatp_k'}{\hatp_j'} &= \antisymmetric{\hatp_k}{\hatp_j}.
\end{aligned}
\end{dmath}

All the other commutators clearly have the desired transformation properties.
} % answer

         % p1
         %
% Copyright � 2015 Peeter Joot.  All Rights Reserved.
% Licenced as described in the file LICENSE under the root directory of this GIT repository.
%
%\input{../blogpost.tex}
%\renewcommand{\basename}{heisenbergSpinPrecession}
%\renewcommand{\dirname}{notes/phy1520/}
%%\newcommand{\dateintitle}{}
%%\newcommand{\keywords}{}
%
%\input{../peeter_prologue_print2.tex}
%
%\usepackage{peeters_layout_exercise}
%\usepackage{peeters_braket}
%\usepackage{peeters_figures}
%
%\beginArtNoToc
%
%\generatetitle{Heisenberg picture spin precession}
%\chapter{Heisenberg picture spin precession}
%\label{chap:heisenbergSpinPrecession}

\makeoproblem{Heisenberg picture spin precession.}{problem:heisenbergSpinPrecession:2.1}{\citep{sakurai2014modern} pr. 2.1}{
For the spin Hamiltonian
\index{spin precession}

\begin{dmath}\label{eqn:heisenbergSpinPrecession:20}
H = -\frac{e B}{m c} S_z = \omega S_z,
\end{dmath}

express and solve the Heisenberg equations of motion for \( S_x(t), S_y(t) \), and \( S_z(t) \).
} % problem

\makeanswer{problem:heisenbergSpinPrecession:2.1}{

The equations of motion are of the form

\begin{dmath}\label{eqn:heisenbergSpinPrecession:40}
\frac{dS_i^\txtH}{dt}
= \inv{i \Hbar} \antisymmetric{S_i^\txtH}{H}
= \inv{i \Hbar} \antisymmetric{U^\dagger S_i U}{H}
= \inv{i \Hbar} \lr{U^\dagger S_i U H - H U^\dagger S_i U }
= \inv{i \Hbar} U^\dagger \lr{ S_i H - H S_i } U
= \frac{\omega}{i \Hbar} U^\dagger \antisymmetric{ S_i}{S_z } U.
\end{dmath}

These are

\begin{dmath}\label{eqn:heisenbergSpinPrecession:60}
\begin{aligned}
\frac{dS_x^\txtH}{dt} &= -\omega U^\dagger S_y U \\
\frac{dS_y^\txtH}{dt} &= \omega U^\dagger S_x U \\
\frac{dS_z^\txtH}{dt} &= 0.
\end{aligned}
\end{dmath}

To completely specify these equations, we need to expand \( U(t) \), which is

\begin{dmath}\label{eqn:heisenbergSpinPrecession:80}
U(t)
= e^{-i H t /\Hbar}
= e^{-i \omega S_z t /\Hbar}
= e^{-i \omega \sigma_z t /2}
= \cos\lr{ \omega t/2 } -i \sigma_z \sin\lr{ \omega t/2 }
=
\begin{bmatrix}
\cos\lr{ \omega t/2 } -i \sin\lr{ \omega t/2 } & 0 \\
0 & \cos\lr{ \omega t/2 } + i \sin\lr{ \omega t/2 }
\end{bmatrix}
=
\begin{bmatrix}
e^{-i\omega t/2} & 0 \\
0 & e^{i\omega t/2}
\end{bmatrix}.
\end{dmath}

The equations of motion can now be written out in full.  To do so seems a bit silly since we also know that \( S_x^\txtH = U^\dagger S_x U, S_y^\txtH U^\dagger S_x U \).  However, if that is temporarily forgotten, we can show that the Heisenberg equations of motion can be solved for these too.

\begin{dmath}\label{eqn:heisenbergSpinPrecession:100}
U^\dagger S_x U
=
\frac{\Hbar}{2}
\begin{bmatrix}
e^{i\omega t/2} & 0 \\
0 & e^{-i\omega t/2}
\end{bmatrix}
\PauliX
\begin{bmatrix}
e^{-i\omega t/2} & 0 \\
0 & e^{i\omega t/2}
\end{bmatrix}
=
\frac{\Hbar}{2}
\begin{bmatrix}
0 & e^{i\omega t/2} \\
e^{-i\omega t/2} & 0
\end{bmatrix}
\begin{bmatrix}
e^{-i\omega t/2} & 0 \\
0 & e^{i\omega t/2}
\end{bmatrix}
=
\frac{\Hbar}{2}
\begin{bmatrix}
0 & e^{i\omega t} \\
e^{-i\omega t} & 0
\end{bmatrix},
\end{dmath}

and
\begin{dmath}\label{eqn:heisenbergSpinPrecession:120}
U^\dagger S_y U
=
\frac{\Hbar}{2}
\begin{bmatrix}
e^{i\omega t/2} & 0 \\
0 & e^{-i\omega t/2}
\end{bmatrix}
\PauliY
\begin{bmatrix}
e^{-i\omega t/2} & 0 \\
0 & e^{i\omega t/2}
\end{bmatrix}
=
\frac{i\Hbar}{2}
\begin{bmatrix}
0 & -e^{i\omega t/2} \\
e^{-i\omega t/2} & 0
\end{bmatrix}
\begin{bmatrix}
e^{-i\omega t/2} & 0 \\
0 & e^{i\omega t/2}
\end{bmatrix}
=
\frac{i \Hbar}{2}
\begin{bmatrix}
0 & -e^{i\omega t} \\
e^{-i\omega t} & 0
\end{bmatrix}.
\end{dmath}

The equations of motion are now fully specified

\begin{dmath}\label{eqn:heisenbergSpinPrecession:140}
\begin{aligned}
\frac{dS_x^\txtH}{dt} &=
-\frac{i \Hbar \omega}{2}
\begin{bmatrix}
0 & -e^{i\omega t} \\
e^{-i\omega t} & 0
\end{bmatrix} \\
\frac{dS_y^\txtH}{dt} &=
\frac{\Hbar \omega}{2}
\begin{bmatrix}
0 & e^{i\omega t} \\
e^{-i\omega t} & 0
\end{bmatrix} \\
\frac{dS_z^\txtH}{dt} &= 0.
\end{aligned}
\end{dmath}

Integration gives

\begin{dmath}\label{eqn:heisenbergSpinPrecession:160}
\begin{aligned}
S_x^\txtH &=
\frac{\Hbar}{2}
\begin{bmatrix}
0 & e^{i\omega t} \\
e^{-i\omega t} & 0
\end{bmatrix} + C \\
S_y^\txtH &=
\frac{\Hbar}{2}
\begin{bmatrix}
0 & -i e^{i\omega t} \\
i e^{-i\omega t} & 0
\end{bmatrix} + C \\
S_z^\txtH &= C.
\end{aligned}
\end{dmath}

The integration constants are fixed by the boundary condition \( S_i^\txtH(0) = S_i \), so

\begin{dmath}\label{eqn:heisenbergSpinPrecession:180}
\begin{aligned}
S_x^\txtH &=
\frac{\Hbar}{2}
\begin{bmatrix}
0 & e^{i\omega t} \\
e^{-i\omega t} & 0
\end{bmatrix} \\
S_y^\txtH &=
\frac{i \Hbar}{2}
\begin{bmatrix}
0 & - e^{i\omega t} \\
 e^{-i\omega t} & 0
\end{bmatrix} \\
S_z^\txtH &= S_z.
\end{aligned}
\end{dmath}

Observe that these integrated values \( S_x^\txtH, S_y^\txtH \) match \cref{eqn:heisenbergSpinPrecession:100}, and \cref{eqn:heisenbergSpinPrecession:120} as expected.

} % answer

%\EndArticle

         % p2
         %
% Copyright � 2015 Peeter Joot.  All Rights Reserved.
% Licenced as described in the file LICENSE under the root directory of this GIT repository.
%
%\input{../blogpost.tex}
%\renewcommand{\basename}{dynamicsNonHermitian}
%\renewcommand{\dirname}{notes/phy1520/}
%\newcommand{\dateintitle}{}
%\newcommand{\keywords}{}

%\input{../peeter_prologue_print2.tex}
%
%\usepackage{peeters_layout_exercise}
%\usepackage{peeters_braket}
%\usepackage{peeters_figures}
%\usepackage{peeters_qed}
%
%\beginArtNoToc

%\generatetitle{Dynamics of non-Hermitian Hamiltonian}
%\chapter{Dynamics of non-Hermitian Hamiltonian}
%\label{chap:dynamicsNonHermitian}

\makeoproblem{Dynamics of non-Hermitian Hamiltonian.}{problem:dynamicsNonHermitian:2.2}{\citep{sakurai2014modern} pr. 2.2}{
\index{Hamiltonian!non-Hermitian}
Revisiting an earlier Hamiltonian, but assuming it was entered incorrectly as
%
\begin{dmath}\label{eqn:dynamicsNonHermitian:20}
H = H_{11} \ket{1}\bra{1}
  + H_{22} \ket{2}\bra{2}
  + H_{12} \ket{1}\bra{2}.
\end{dmath}
%
What principle is now violated?  Illustrate your point explicitly by attempting to solve the most general time-dependent problem using an illegal Hamiltonian of this kind.  You may assume that \( H_{11} = H_{22} \) for simplicity.
} % problem

\makeanswer{problem:dynamicsNonHermitian:2.2}{
%
In matrix form this Hamiltonian is
%
\begin{dmath}\label{eqn:dynamicsNonHermitian:40}
H =
\begin{bmatrix}
\bra{1} H \ket{1} & \bra{1} H \ket{2} \\
\bra{2} H \ket{1} & \bra{2} H \ket{2} \\
\end{bmatrix}
=
\begin{bmatrix}
H_{11} & H_{12} \\
0      & H_{22} \\
\end{bmatrix}.
\end{dmath}
%
This is not a Hermitian operator.  What is the physical implication of this non-Hermicity?  Consider the simpler case where \( H_{11} = H_{22} \).  Such a Hamiltonian has the form
%
\begin{dmath}\label{eqn:dynamicsNonHermitian:60}
H =
\begin{bmatrix}
a & b \\
0 & a
\end{bmatrix}.
\end{dmath}
%
This has only one unique eigenvector ( \( (1,0) \), but we can still solve the time evolution equation
%
\begin{dmath}\label{eqn:dynamicsNonHermitian:80}
i \Hbar \PD{t}{U} = H U,
\end{dmath}
%
since for constant \( H \), we have
%
\begin{dmath}\label{eqn:dynamicsNonHermitian:100}
U = e^{-i H t/\Hbar}.
\end{dmath}
%
To exponentiate, note that we have
%
\begin{dmath}\label{eqn:dynamicsNonHermitian:120}
{\begin{bmatrix}
a & b \\
0 & a
\end{bmatrix}}^n
=
\begin{bmatrix}
a^n & n a^{n-1} b \\
0 & a^n
\end{bmatrix}.
\end{dmath}
%
To prove the induction, the \( n = 2 \) case follows easily
%
\begin{dmath}\label{eqn:dynamicsNonHermitian:140}
\begin{bmatrix}
a & b \\
0 & a
\end{bmatrix}
\begin{bmatrix}
a & b \\
0 & a
\end{bmatrix}
=
\begin{bmatrix}
a^2 & 2 a b \\
0 & a^2
\end{bmatrix},
\end{dmath}
%
as does the general case
%
\begin{dmath}\label{eqn:dynamicsNonHermitian:160}
\begin{bmatrix}
a^n & n a^{n-1} b \\
0 & a^n
\end{bmatrix}
\begin{bmatrix}
a & b \\
0 & a
\end{bmatrix}
=
\begin{bmatrix}
a^{n+1} & (n +1 ) a^{n} b \\
0 & a^{n+1}
\end{bmatrix}.
\end{dmath}
%
The exponential sum is thus
\begin{dmath}\label{eqn:dynamicsNonHermitian:180}
e^{H \tau}
=
\begin{bmatrix}
e^{a \tau} & 0 + \frac{b \tau}{1!} + \frac{2 a b \tau^2}{2!} + \frac{3 a^2 b \tau^3}{3!} + \cdots \\
0 & e^{a \tau}
\end{bmatrix}.
\end{dmath}
%
That sum simplifies to
%
\begin{dmath}\label{eqn:dynamicsNonHermitian:200}
\frac{b \tau}{0!} + \frac{a b \tau^2}{1!} + \frac{a^2 b \tau^3}{2!} + \cdots \\
=
b \tau \lr{ 1 + \frac{a \tau}{1!} + \frac{(a \tau)^2}{2!} + \cdots }
=
b \tau e^{a \tau}.
\end{dmath}
%
The exponential is thus
\begin{dmath}\label{eqn:dynamicsNonHermitian:220}
e^{H \tau} =
\begin{bmatrix}
e^{a\tau} & b \tau e^{a\tau} \\
0 & e^{a\tau}
\end{bmatrix}
=
\begin{bmatrix}
1 & b \tau \\
0 & 1
\end{bmatrix}
e^{a\tau}.
\end{dmath}
%
In particular
%
\begin{dmath}\label{eqn:dynamicsNonHermitian:240}
U = e^{-i H t/\Hbar} =
\begin{bmatrix}
1 & -i b t/\Hbar \\
0 & 1
\end{bmatrix}
e^{-i a t /\Hbar }.
\end{dmath}
%
We can verify that this is a solution to \cref{eqn:dynamicsNonHermitian:80}.  The left hand side is
%
\begin{dmath}\label{eqn:dynamicsNonHermitian:260}
i \Hbar \PD{t}{U}
=
i \Hbar
\begin{bmatrix}
-i a/\Hbar & -i b /\Hbar + (-i b t/\Hbar)(-i a/\Hbar) \\
0 & -i a /\Hbar
\end{bmatrix}
e^{-i a t /\Hbar }
=
\begin{bmatrix}
a & b - i a b t/\Hbar \\
0 & a
\end{bmatrix}
e^{-i a t /\Hbar },
\end{dmath}
%
and for the right hand side
\begin{dmath}\label{eqn:dynamicsNonHermitian:280}
H U
=
\begin{bmatrix}
a & b \\
0 & a
\end{bmatrix}
\begin{bmatrix}
1 & -i b t/\Hbar \\
0 & 1
\end{bmatrix}
e^{-i a t /\Hbar }
=
\begin{bmatrix}
a & b - i a b t/\Hbar \\
0 & a
\end{bmatrix}
e^{-i a t /\Hbar }
=
i \Hbar \PD{t}{U}. \qedmarker
\end{dmath}

While the Schr\"{o}dinger is satisfied, we don't have the unitary inversion physical property that is desired for the time evolution operator \( U \).  Namely
%
\begin{dmath}\label{eqn:dynamicsNonHermitian:300}
U^\dagger U
=
\begin{bmatrix}
1 & 0 \\
i b t/\Hbar & 1
\end{bmatrix}
e^{i a t /\Hbar }
\begin{bmatrix}
1 & -i b t/\Hbar \\
0 & 1
\end{bmatrix}
e^{-i a t /\Hbar }
=
\begin{bmatrix}
1 & -i b t/\Hbar \\
i b t/\Hbar & (b t)^2/\Hbar^2
\end{bmatrix}
\ne I.
\end{dmath}
%
We required \( U^\dagger U = I \) for the time evolution operator, but don't have that property for this non-Hermitian Hamiltonian.
} % answer

%\EndArticle

         % p2.3
         %
% Copyright � 2015 Peeter Joot.  All Rights Reserved.
% Licenced as described in the file LICENSE under the root directory of this GIT repository.
%
%\input{../blogpost.tex}
%\renewcommand{\basename}{spinTimeEvolution}
%\renewcommand{\dirname}{notes/phy1520/}
%%\newcommand{\dateintitle}{}
%%\newcommand{\keywords}{}
%
%\input{../peeter_prologue_print2.tex}
%
%\usepackage{peeters_layout_exercise}
%\usepackage{peeters_braket}
%\usepackage{peeters_figures}
%
%\beginArtNoToc
%
%\generatetitle{Time evolution of spin half probability and dispersion}
%%\label{chap:spinTimeEvolution}
%
\makeoproblem
%{Time evolution of spin half probability and dispersion.}
{Spin \(1/2\) evolution and dispersion.}
{problem:spinTimeEvolution:3}{\citep{sakurai2014modern} pr. 2.3}{
\index{spin half!dispersion}
\index{dispersion}
A spin \( 1/2 \) system \( \BS \cdot \ncap \), with \( \ncap = \sin \beta \xcap + \cos\beta \zcap \), is in state with eigenvalue \( \Hbar/2 \), acted on by a magnetic field of strength \( B \) in the \( +z \) direction.
%
\makesubproblem{}{problem:spinTimeEvolution:3:a}
%
If \( S_x \) is measured at time \( t \), what is the probability of getting \( + \Hbar/2 \)?
%
\makesubproblem{}{problem:spinTimeEvolution:3:b}
%
Evaluate the dispersion in \( S_x \) as a function of t, that is,
%
\begin{equation}\label{eqn:spinTimeEvolution:20}
\expectation{\lr{ S_x - \expectation{S_x}}^2}.
\end{equation}
%
\makesubproblem{}{problem:spinTimeEvolution:3:c}
%
Check your answers for \( \beta \rightarrow 0, \pi/2 \) to see if they make sense.
} % problem
%
\makeanswer{problem:spinTimeEvolution:3}{
%
\makeSubAnswer{}{problem:spinTimeEvolution:3:a}
%
The spin operator in matrix form is
\begin{dmath}\label{eqn:spinTimeEvolution:40}
S \cdot \ncap
=
\frac{\Hbar}{2} \lr{ \sigma_z \cos\beta + \sigma_x \sin\beta }
=
\frac{\Hbar}{2} \lr{ \PauliZ \cos\beta + \PauliX \sin\beta }
=
\frac{\Hbar}{2}
\begin{bmatrix}
\cos\beta & \sin\beta \\
\sin\beta & -\cos\beta
\end{bmatrix}.
\end{dmath}
%
The \( \ket{S \cdot \ncap ; + } \) eigenstate is found from
%
\begin{dmath}\label{eqn:spinTimeEvolution:60}
\lr{ S \cdot \ncap - \Hbar/2}
\begin{bmatrix}
a \\
b
\end{bmatrix}
= 0,
\end{dmath}
%
or
%
\begin{dmath}\label{eqn:spinTimeEvolution:80}
0
=
\lr{ \cos\beta - 1 } a + \sin\beta b
=
\lr{ -2 \sin^2(\beta/2) } a + 2 \sin(\beta/2) \cos(\beta/2) b
=
\lr{ - \sin(\beta/2) } a + \cos(\beta/2) b,
\end{dmath}
%
or
%
\begin{dmath}\label{eqn:spinTimeEvolution:100}
\ket{ S \cdot \ncap ; + }
=
\begin{bmatrix}
\cos(\beta/2) \\
\sin(\beta/2) \\
\end{bmatrix}.
\end{dmath}
%
The Hamiltonian is
%
\begin{equation}\label{eqn:spinTimeEvolution:120}
H
= - \frac{e B}{m c} S_z
= - \frac{e B \Hbar}{2 m c} \sigma_z,
\end{equation}
%
so the time evolution operator is
%
\begin{dmath}\label{eqn:spinTimeEvolution:140}
U
= e^{-i H t/\Hbar}
= e^{ \frac{i e B t }{2 m c} \sigma_z }.
\end{dmath}
%
Let \( \omega = e B/(2 m c) \), so
%
\begin{dmath}\label{eqn:spinTimeEvolution:160}
U
=
e^{i \sigma_z \omega t}
=
\cos(\omega t) + i \sigma_z \sin(\omega t)
=
\begin{bmatrix}
1 & 0 \\
0 & 1
\end{bmatrix}
\cos(\omega t)
+
i  \PauliZ \sin(\omega t)
=
\begin{bmatrix}
e^{i \omega t} & 0 \\
0 & e^{-i \omega t}
\end{bmatrix}.
\end{dmath}
%
The time evolution of the initial state is
%
\begin{dmath}\label{eqn:spinTimeEvolution:180}
\ket{S \cdot \ncap ; + }(t)
=
U \ket{S \cdot \ncap ; + }(0)
=
\begin{bmatrix}
e^{i \omega t} & 0 \\
0 & e^{-i \omega t}
\end{bmatrix}
\begin{bmatrix}
\cos(\beta/2) \\
\sin(\beta/2) \\
\end{bmatrix}
=
\begin{bmatrix}
\cos(\beta/2) e^{i \omega t} \\
\sin(\beta/2) e^{-i \omega t} \\
\end{bmatrix}.
\end{dmath}
%
The probability of finding the state in \( \ket{S \cdot \xcap ; + } \) at time \( t \) (i.e. measuring \( S_x \) and finding \( \Hbar/2 \)) is
%
\begin{dmath}\label{eqn:spinTimeEvolution:200}
\begin{aligned}
\Abs{\braket{S \cdot \xcap ; + }{S \cdot \ncap ; + }}^2
&=
\Abs{\inv{\sqrt{2}}
\begin{bmatrix}
1 & 1 \\
\end{bmatrix}
\begin{bmatrix}
\cos(\beta/2) e^{i \omega t} \\
\sin(\beta/2) e^{-i \omega t} \\
\end{bmatrix}
}^2 \\
&=
\inv{2}
\Abs{
\cos(\beta/2) e^{i \omega t} +
\sin(\beta/2) e^{-i \omega t} }^2 \\
&=
\inv{2} \lr{ 1 + 2 \cos(\beta/2) \sin(\beta/2) \cos(2 \omega t) } \\
&=
\inv{2} \lr{ 1 + \sin(\beta) \cos( 2 \omega t) }.
\end{aligned}
\end{dmath}
%
\makeSubAnswer{}{problem:spinTimeEvolution:3:b}
%
To calculate the dispersion first note that
%
\begin{dmath}\label{eqn:spinTimeEvolution:300}
S_x^2
= \lr{ \frac{\Hbar}{2} }^2 \PauliX^2
= \lr{ \frac{\Hbar}{2} }^2,
\end{dmath}
%
so only the first order expectation is non-trivial to calculate.  That is
%
\begin{dmath}\label{eqn:spinTimeEvolution:320}
\expectation{S_x}
=
\frac{\Hbar}{2}
\begin{bmatrix}
\cos(\beta/2) e^{-i \omega t} &
\sin(\beta/2) e^{i \omega t}
\end{bmatrix}
\PauliX
\begin{bmatrix}
\cos(\beta/2) e^{i \omega t} \\
\sin(\beta/2) e^{-i \omega t} \\
\end{bmatrix}
=
\frac{\Hbar}{2}
\begin{bmatrix}
\cos(\beta/2) e^{-i \omega t} &
\sin(\beta/2) e^{i \omega t}
\end{bmatrix}
\begin{bmatrix}
\sin(\beta/2) e^{-i \omega t} \\
\cos(\beta/2) e^{i \omega t} \\
\end{bmatrix}
=
\frac{\Hbar}{2}
\sin(\beta/2) \cos(\beta/2) \lr{ e^{-2 i \omega t} + e^{ 2 i \omega t} }
=
\frac{\Hbar}{2} \sin\beta \cos( 2 \omega t ).
\end{dmath}
%
This gives
%
\boxedEquation{eqn:spinTimeEvolution:340}{
\expectation{(\Delta S_x)^2}
=
\lr{ \frac{\Hbar}{2} }^2 \lr{ 1 - \sin^2\beta \cos^2( 2 \omega t ) }.
}
%
\makeSubAnswer{}{problem:spinTimeEvolution:3:c}
%
For \( \beta = 0 \), \( \ncap = \zcap \), and \( \beta = \pi/2 \), \( \ncap = \xcap \).  For the first case, the state is in an eigenstate of \( S_z \), so must evolve as
%
\begin{dmath}\label{eqn:spinTimeEvolution:220}
\ket{S \cdot \ncap ; + }(t) = \ket{S \cdot \ncap ; + }(0) e^{i \omega t}.
\end{dmath}
%
The probability of finding it in state \( \ket{S \cdot \xcap ; + } \) is therefore
%
\begin{dmath}\label{eqn:spinTimeEvolution:240}
\Abs{
\inv{\sqrt{2}}
\begin{bmatrix}
1 & 1
\end{bmatrix}
\begin{bmatrix}
e^{i \omega t} \\
0
\end{bmatrix}
}^2
=
\inv{2} \Abs{ e^{i\omega t} }^2
=
\inv{2}
=
\inv{2} \lr{ 1 + \sin(0) \cos(2 \omega t) }.
\end{dmath}
%
This matches \cref{eqn:spinTimeEvolution:200} as expected.

For \( \beta = \pi/2 \) we have
%
\begin{dmath}\label{eqn:spinTimeEvolution:260}
\ket{S \cdot \xcap ; + }(t) =
\inv{\sqrt{2}}
\begin{bmatrix}
e^{i \omega t} & 0 \\
0 & e^{-i \omega t}
\end{bmatrix}
\begin{bmatrix}
1 \\
1
\end{bmatrix}
=
\inv{\sqrt{2}}
\begin{bmatrix}
e^{i \omega t} \\
e^{-i \omega t}
\end{bmatrix}.
\end{dmath}
%
The probability for the \( \Hbar/2 \) \( S_x \) measurement at time \( t \) is
\begin{dmath}\label{eqn:spinTimeEvolution:280}
\Abs{
\inv{2}
\begin{bmatrix}
1 & 1
\end{bmatrix}
\begin{bmatrix}
e^{i \omega t} \\
e^{-i \omega t}
\end{bmatrix}
}^2
=
\inv{4} \Abs{ e^{i \omega t}  + e^{-i \omega t} }^2
=
\cos^2(\omega t)
=
\inv{2}\lr{ 1 + \sin(\pi/2) \cos( 2 \omega t )}.
\end{dmath}
%
Again, this matches the expected value.
For the dispersions, at \( \beta = 0 \), the dispersion is
%
\begin{dmath}\label{eqn:spinTimeEvolution:360}
\lr{\frac{\Hbar}{2}}^2.
\end{dmath}
This is the maximum dispersion, which makes sense since we are measuring \( S_x \) when the initial state is \( \ket{S \cdot \zcap ; + } \).  For \( \beta = \pi/2 \) the dispersion is
%
\begin{dmath}\label{eqn:spinTimeEvolution:380}
\lr{\frac{\Hbar}{2}}^2 \sin^2 ( 2 \omega t ).
\end{dmath}
%
This starts off as zero dispersion (because the initial state is \( \ket{ S \cdot \xcap ; + } \), but then oscillates.
%
} % answer
%\EndArticle

         % p5:
         %
% Copyright � 2015 Peeter Joot.  All Rights Reserved.
% Licenced as described in the file LICENSE under the root directory of this GIT repository.
%
%\input{../blogpost.tex}
%\renewcommand{\basename}{positionCommutator}
%\renewcommand{\dirname}{notes/phy1520/}
%%\newcommand{\dateintitle}{}
%%\newcommand{\keywords}{}
%
%\input{../peeter_prologue_print2.tex}
%
%\usepackage{peeters_layout_exercise}
%\usepackage{peeters_braket}
%\usepackage{peeters_figures}
%
%\beginArtNoToc
%
%\generatetitle{Heisenberg picture position commutator}
%\chapter{Heisenberg picture position commutator}
%\label{chap:positionCommutator}

%
\makeoproblem{Heisenberg picture position commutator.}{problem:positionCommutator:2.5}{\citep{sakurai2014modern} pr. 2.5}{
\index{Heisenberg picture}
Evaluate
%
\begin{dmath}\label{eqn:positionCommutator:20}
\antisymmetric{x(t)}{x(0)},
\end{dmath}
%
for a Heisenberg picture operator \( x(t) \) for a free particle.
} % problem
%
\makeanswer{problem:positionCommutator:2.5}{
%
The free particle Hamiltonian is
%
\begin{dmath}\label{eqn:positionCommutator:40}
H = \frac{p^2}{2m},
\end{dmath}
%
so the time evolution operator is
%
\begin{dmath}\label{eqn:positionCommutator:60}
U(t) = e^{-i p^2 t/(2 m \Hbar)}.
\end{dmath}
%
The Heisenberg picture position operator is
%
\begin{dmath}\label{eqn:positionCommutator:80}
x^\txtH
= U^\dagger x U
= e^{i p^2 t/(2 m \Hbar)} x e^{-i p^2 t/(2 m \Hbar)}
= \sum_{k = 0}^\infty \inv{k!} \lr{ \frac{i p^2 t}{2 m \Hbar} }^k
x
e^{-i p^2 t/(2 m \Hbar)}
= \sum_{k = 0}^\infty \inv{k!} \lr{ \frac{i t}{2 m \Hbar} }^k p^{2k} x
e^{-i p^2 t/(2 m \Hbar)}
=
\sum_{k = 0}^\infty \inv{k!} \lr{ \frac{i t}{2 m \Hbar} }^k \lr{ \antisymmetric{p^{2k}}{x} + x p^{2k} }
e^{-i p^2 t/(2 m \Hbar)}
= x +
\sum_{k = 0}^\infty \inv{k!} \lr{ \frac{i t}{2 m \Hbar} }^k \antisymmetric{p^{2k}}{x}
e^{-i p^2 t/(2 m \Hbar)}
= x +
\sum_{k = 0}^\infty \inv{k!} \lr{ \frac{i t}{2 m \Hbar} }^k \lr{ -i \Hbar \PD{p}{p^{2k}} }
e^{-i p^2 t/(2 m \Hbar)}
= x +
\sum_{k = 0}^\infty \inv{k!} \lr{ \frac{i t}{2 m \Hbar} }^k \lr{ -i \Hbar 2 k p^{2 k -1} }
e^{-i p^2 t/(2 m \Hbar)}
= x +
-2 i \Hbar p \frac{i t}{2 m \Hbar} \sum_{k = 1}^\infty \inv{(k-1)!} \lr{ \frac{i t}{2 m \Hbar} }^{k-1} p^{2(k - 1)}
e^{-i p^2 t/(2 m \Hbar)}
= x + t \frac{p}{m}.
\end{dmath}
%
This has the structure of a classical free particle \( x(t) = x + v t \), but in this case \( x,p \) are operators.

The evolved position commutator is
\begin{dmath}\label{eqn:positionCommutator:100}
\antisymmetric{x(t)}{x(0)}
=
\antisymmetric{x + t p/m}{x}
=
\frac{t}{m} \antisymmetric{p}{x}
=
-i \Hbar \frac{t}{m}.
\end{dmath}
%
Compare this to the classical Poisson bracket
\begin{dmath}\label{eqn:positionCommutator:120}
\antisymmetric{x(t)}{x(0)}_{\textrm{classical}}
=
\PD{x}{}\lr{x + p t/m} \PD{p}{x} - \PD{p}{}\lr{x + p t/m} \PD{x}{x}
=
- \frac{t}{m}.
\end{dmath}
%
This has the expected relation \( \antisymmetric{x(t)}{x(0)} = i \Hbar \antisymmetric{x(t)}{x(0)}_{\textrm{classical}} \).

} % answer

%\EndArticle

         % p7
         %%
% Copyright � 2015 Peeter Joot.  All Rights Reserved.
% Licenced as described in the file LICENSE under the root directory of this GIT repository.
%
%\input{../blogpost.tex}
%\renewcommand{\basename}{qmVirialTheorem}
%\renewcommand{\dirname}{notes/phy1520/}
%%\newcommand{\dateintitle}{}
%%\newcommand{\keywords}{}
%
%\input{../peeter_prologue_print2.tex}
%
%\usepackage{peeters_layout_exercise}
%\usepackage{peeters_braket}
%\usepackage{peeters_figures}
%
%\beginArtNoToc
%
%\generatetitle{Quantum Virial Theorem}
%\chapter{Quantum Virial Theorem}
%\label{chap:qmVirialTheorem}
%
\makeoproblem{Quantum virial Theorem.}{problem:qmVirialTheorem:7}{\citep{sakurai2014modern} pr. 2.7}{
\index{virial theorem}
Consider a particle with Hamiltonian
%
\begin{equation}\label{eqn:qmVirialTheorem:20}
H = \frac{\Bp^2}{2 m} + V(\Bx),
\end{dmath}
%
By calculating the time evolution of \( \antisymmetric{\Bx \cdot \Bp}{H} \), identify the quantum virial theorem and show the conditions where it is satisfied.
%
} % problem
%
\makeanswer{problem:qmVirialTheorem:7}{
%
\begin{dmath}\label{eqn:qmVirialTheorem:40}
\antisymmetric{\Bx \cdot \Bp}{H}
=
\inv{2 m} \antisymmetric{\Bx \cdot \Bp}{\Bp^2} + \antisymmetric{\Bx \cdot \Bp}{V(\Bx)}
=
\inv{2 m} \lr{ x_r p_r \Bp^2 - \Bp^2 x_r p_r}
+
\lr{ x_r p_r V(\Bx) - V(\Bx) x_r p_r }
=
\inv{2 m} \antisymmetric{ x_r }{\Bp^2} p_r
+
x_r \antisymmetric{ p_r}{ V(\Bx)},
\end{dmath}
%
Evaluating those commutators separately, gives
%
\begin{dmath}\label{eqn:qmVirialTheorem:60}
\begin{aligned}
\antisymmetric{ x_r }{\Bp^2}
&=
\antisymmetric{ x_r }{p_r^2}\qquad \text{no sum} \\
&=
2 i \Hbar p_r,
\end{aligned}
\end{dmath}

and
%
\begin{dmath}\label{eqn:qmVirialTheorem:80}
\antisymmetric{ p_r}{ V(\Bx)}
= -i \Hbar \PD{x_r}{V(\Bx)},
\end{dmath}
%
so
\begin{dmath}\label{eqn:qmVirialTheorem:100}
\ddt{}\lr{\Bx \cdot \Bp}
=
\inv{i \Hbar}
\antisymmetric{\Bx \cdot \Bp}{H}
=
\inv{2 m} 2 p_r p_r - x_r \PD{x_r}{V(\Bx)}
=
\frac{\Bp^2}{m} - \Bx \cdot \spacegrad V(\Bx).
\end{dmath}
%
Taking expectation values, assuming that the states are independent of time, we have
%
\begin{dmath}\label{eqn:qmVirialTheorem:120}
0
= \ddt{} \expectation{ \Bx \cdot \Bp }
= \expectation{\frac{\Bp^2}{m}} - \expectation{\Bx \cdot \spacegrad V(\Bx)}.
\end{dmath}
%
Note that taking the expectation with respect to stationary states was required to reverse the order of the time derivative with the expectation operation.

The right hand side is the quantum equivalent of the virial theorem, relating the average kinetic energy to the potential
%
\begin{equation}\label{eqn:qmVirialTheorem:140}
2 \expectation{T} = \expectation{\Bx \cdot \spacegrad V(\Bx)}
\end{dmath}
%
} % answer

%\EndArticle

         % ps2. virial theorem
         %
% Copyright � 2015 Peeter Joot.  All Rights Reserved.
% Licenced as described in the file LICENSE under the root directory of this GIT repository.
%

% Augmented version of problem 2.7 from the text.
\makeoproblem{Virial theorem.}{gradQuantum:problemSet2:2}{phy1520 2015 ps2.2}{
\index{virial theorem}

Consider a three-dimensional Hamiltonian
%
\begin{dmath}\label{eqn:gradQuantumProblemSet2Problem2:21}
H = \frac{\Bp^2}{2m} + V(\Bx).
\end{dmath}

\makesubproblem{}{gradQuantum:problemSet2:2a}

Calculate \( \antisymmetric{\Bx \cdot \Bp}{H} \) and show that
%
\begin{dmath}\label{eqn:gradQuantumProblemSet2Problem2:41}
\ddt{} \expectation{ \Bx \cdot \Bp } = \expectation{ \frac{\Bp^2}{m} } - \expectation{ \Bx \cdot \spacegrad V }.
\end{dmath}

When the l.h.s. vanishes, the result that the r.h.s. is zero is called the quantum virial theorem.

\makesubproblem{}{gradQuantum:problemSet2:2c}
Consider the 3D isotropic harmonic oscillator, and show explicitly that its eigenstates obey the virial theorem.

\makesubproblem{}{gradQuantum:problemSet2:2d}
Evaluate the r.h.s. for the superposition state \( \ket{0,0,0} + \ket{0,0,2} \) where the notation stands for \( \ket{ n_x, n_y, n_z } \) occupation numbers.

\makesubproblem{}{gradQuantum:problemSet2:2b}
Under what conditions does the left-hand side vanish?
} % makeproblem

%%%%%%%%%%%%%%
%
% from: ../phy1520/gradQuantumProblemSet2Problem2.tex
%
%
\makeanswer{gradQuantum:problemSet2:2}{
\withproblemsetsParagraph{

\makeSubAnswer{}{gradQuantum:problemSet2:2a}

We'll need various commutators to evaluate the Heisenberg equation of motion.
\begin{dmath}\label{eqn:gradQuantumProblemSet2Problem2:40}
\antisymmetric{\Bx \cdot \Bp}{H}
=
\inv{2 m} \antisymmetric{\Bx \cdot \Bp}{\Bp^2} + \antisymmetric{\Bx \cdot \Bp}{V(\Bx)}
=
\inv{2 m} \lr{ x_r p_r \Bp^2 - \Bp^2 x_r p_r}
+
\lr{ x_r p_r V(\Bx) - V(\Bx) x_r p_r }
=
\inv{2 m} \antisymmetric{ x_r }{\Bp^2} p_r
+
x_r \antisymmetric{ p_r}{ V(\Bx)},
\end{dmath}
%
Evaluating those commutators separately, gives
%
\begin{dmath}\label{eqn:gradQuantumProblemSet2Problem2:60}
\begin{aligned}
\antisymmetric{ x_r }{\Bp^2}
&=
\antisymmetric{ x_r }{p_r^2}\qquad \text{(no sum)} \\
&=
2 i \Hbar p_r,
\end{aligned}
\end{dmath}

and
%
\begin{dmath}\label{eqn:gradQuantumProblemSet2Problem2:80}
\antisymmetric{ p_r}{ V(\Bx)}
= -i \Hbar \PD{x_r}{V(\Bx)},
\end{dmath}
%
so
\begin{dmath}\label{eqn:gradQuantumProblemSet2Problem2:100}
\ddt{}\lr{\Bx \cdot \Bp}
=
\inv{i \Hbar}
\antisymmetric{\Bx \cdot \Bp}{H}
=
\inv{2 m} 2 p_r p_r - x_r \PD{x_r}{V(\Bx)}
=
\frac{\Bp^2}{m} - \Bx \cdot \spacegrad V(\Bx).
\end{dmath}

Evaluating the expectation of this identity with respect to a stationary state \( \ket{\psi} \) (i.e. a state independent of time), we have
%
\begin{dmath}\label{eqn:gradQuantumProblemSet2Problem2:120}
\bra{\psi} \ddt{} \Bx \cdot \Bp \ket{\psi}
=
\ddt{} \bra{\psi} \Bx \cdot \Bp \ket{\psi}
=
\ddt{} \expectation{ \Bx \cdot \Bp }
= \expectation{\frac{\Bp^2}{m}} - \expectation{\Bx \cdot \spacegrad V(\Bx)}.
\end{dmath}

Because the expectation was with respect to a stationary state, the time derivative could be moved outside of the expectation operation.

\makeSubAnswer{}{gradQuantum:problemSet2:2c}
When \cref{eqn:gradQuantumProblemSet2Problem2:120} is zero, we have the quantum equivalent of the virial theorem, relating the average kinetic energy to the potential
%
\begin{dmath}\label{eqn:gradQuantumProblemSet2Problem2:140}
2 \expectation{T} = \expectation{\Bx \cdot \spacegrad V(\Bx)}
\end{dmath}

To evaluate these expectations operations with respect to the 3D SHO eigenstates, let
%
\begin{dmath}\label{eqn:gradQuantumProblemSet2Problem2:240}
\begin{aligned}
a_x(t) &= a_x e^{-i \omega t} \\
a_y(t) &= a_y e^{-i \omega t} \\
a_z(t) &= a_z e^{-i \omega t},
\end{aligned}
\end{dmath}

Note that
%
\begin{dmath}\label{eqn:gradQuantumProblemSet2Problem2:480}
\expectation{ \Bx \cdot \spacegrad V }
=
m \omega^2 \expectation{ \Bx^2(t) }
=
m \omega^2 \frac{x_0^2}{2}
\sum_{k=1}^3
\expectation{ \lr{a_k(t) + a_k^\dagger(t)}^2 }
=
\frac{ \Hbar \omega }{2}
\sum_{k=1}^3
\bra{\psi} \lr{a_k(t) + a_k^\dagger(t)}^2 \ket{\psi}
=
\frac{ \Hbar \omega }{2}
\sum_{k=1}^3
\bra{\psi} \lr{a_k  e^{-i \omega t} + a_k^\dagger e^{i \omega t}}^2 \ket{\psi},
\end{dmath}
%
and
%
\begin{dmath}\label{eqn:gradQuantumProblemSet2Problem2:500}
\expectation{ \frac{\Bp^2}{m} }
=
\frac{-\Hbar^2}{2 m x_0^2}
\sum_{k=1}^3
\expectation{ \lr{a_k^\dagger(t) - a_k(t) }^2 }
=
\frac{ \Hbar \omega }{2}
\sum_{k=1}^3
\bra{\psi}
\lr{ a_k(t) - a_k^\dagger(t) }
\lr{ a_k^\dagger(t) - a_k(t) }
\ket{\psi}
=
\frac{ \Hbar \omega }{2}
\sum_{k=1}^3
\bra{\psi}
\lr{ a_k e^{-i \omega t}- a_k^\dagger e^{i \omega t}}
\lr{ a_k^\dagger e^{i \omega t} - a_k e^{-i \omega t}}
\ket{\psi}.
\end{dmath}

In both cases the pairs of raising and lowering operators have been factored into conjugate pairs so that only the action of the latter on \( \ket{\psi} \) need be considered.  Considering the \( x \) component for example, we've got
%
\begin{dmath}\label{eqn:gradQuantumProblemSet2Problem2:540}
\lr{ a_x^\dagger e^{i \omega t} \pm a_x e^{-i \omega t}}
\ket{n_x, n_y, n_z}
=
e^{i \omega t} \sqrt{n_x + 1} \ket{n_x + 1, n_y, n_z}
\pm
e^{-i \omega t} \sqrt{n_x} \ket{n_x - 1, n_y, n_z}.
\end{dmath}
%
This gives
%
\begin{equation}\label{eqn:gradQuantumProblemSet2Problem2:560}
\expectation{ \Bx \cdot \spacegrad V } = \expectation{ \frac{\Bp^2}{m} }
=
\frac{\Hbar \omega}{2}
\lr{
(n_x + 1) + n_x
+(n_y + 1) + n_y
+(n_z + 1) + n_z
},
\end{equation}
%
or
%
\boxedEquation{eqn:gradQuantumProblemSet2Problem2:580}{
\expectation{ \Bx \cdot \spacegrad V } = \expectation{ \frac{\Bp^2}{m} }
=
\Hbar \omega
\lr{ n_x + n_y + n_z + \frac{3}{2} }.
}

Neither expectation has any time dependence, and the virial theorem has been confirmed.

\makeSubAnswer{}{gradQuantum:problemSet2:2d}

With \( \ket{\psi} = \ket{0,0,0} + \ket{0,0,2} \), the gradient portion of the RHS is
%
\begin{dmath}\label{eqn:gradQuantumProblemSet2Problem2:600}
\expectation{ \Bx \cdot \spacegrad V }
=
m \omega^2 \expectation{ \Bx^2(t) }
=
m \omega^2 \frac{x_0^2}{2}
\sum_{k=1}^3 \bra{\psi} \lr{ a_k e^{-i \omega t} + a_k^\dagger e^{i \omega t} }^2 \ket{\psi}
=
\frac{\Hbar \omega}{2}
\sum_{k=1}^3 \bra{\psi} \lr{ a_k e^{-i \omega t} + a_k^\dagger e^{i \omega t} }^2 \ket{\psi}
\end{dmath}
%
%\begin{dmath}\label{eqn:gradQuantumProblemSet2Problem2:260}
%\begin{aligned}
%\expectation{ \Bx \cdot \spacegrad V }
%&=
%%m \omega^2 \expectation{ \Bx^2(t) } \\
%%&=
%%m \omega^2 \frac{x_0^2}{2}
%\frac{\Hbar \omega}{2}
%\Biglr{
%\bra{\psi} \lr{ a_x e^{-i \omega t} + a_x^\dagger e^{i \omega t} }^2 \ket{\psi} \\
%&\qquad  +\bra{\psi} \lr{ a_y e^{-i \omega t} + a_y^\dagger e^{i \omega t} }^2 \ket{\psi} \\
%&\qquad  +\bra{\psi} \lr{ a_z e^{-i \omega t} + a_z^\dagger e^{i \omega t} }^2 \ket{\psi} } \\
%&=
%\frac{\Hbar \omega}{2}
%\bra{0_x} \lr{ a_x e^{-i \omega t} + a_x^\dagger e^{i \omega t} }^2 \ket{0_x}
%\braket{0_y}{0_y}\lr{\bra{0_z} + \bra{2_z}}\lr{\ket{0_z} + \ket{2_z}} \\
%&+
%\frac{\Hbar \omega}{2}
%\braket{0_x}{0_x}
%\bra{0_y} \lr{ a_y e^{-i \omega t} + a_y^\dagger e^{i \omega t} }^2 \ket{0_y}
%\lr{\bra{0_z} + \bra{2_z}}\lr{\ket{0_z} + \ket{2_z}} \\
%&+
%\frac{\Hbar \omega}{2}
%\braket{0_x}{0_x}\braket{0_y}{0_y}
%\lr{\bra{0_z} + \bra{2_z}}\lr{ a_z e^{-i \omega t} + a_z^\dagger e^{i \omega t} }^2
%\lr{\ket{0_z} + \ket{2_z}}
%\end{aligned}
%\end{dmath}

We require the following for each \( k \)
%
\begin{dmath}\label{eqn:gradQuantumProblemSet2Problem2:620}
\lr{ a_k e^{-i \omega t} + a_k^\dagger e^{i \omega t} } \lr{ \ket{0,0,0} + \ket{0,0,2} }.
\end{dmath}
%
For \( k = 1 \) this is
%
\begin{dmath}\label{eqn:gradQuantumProblemSet2Problem2:640}
\ket{1,0,0} + \ket{1,0,2},
\end{dmath}
%
for \( k = 2 \) this is
%
\begin{dmath}\label{eqn:gradQuantumProblemSet2Problem2:660}
\ket{0,1,0} + \ket{0,1,2},
\end{dmath}
%
and for \( k = 3 \)
%
\begin{dmath}\label{eqn:gradQuantumProblemSet2Problem2:680}
\begin{aligned}
e^{-i \omega t} &\sqrt{2} \ket{0,0,1}
+
e^{i \omega t}
\lr{
\ket{0,0,1} + \sqrt{3} \ket{0,0,3}
} \\
&=
\ket{0,0,1}
\lr{
e^{-i \omega t} \sqrt{2} + e^{i \omega t}
}
+
\sqrt{3} e^{i \omega t}
\ket{0,0,3}.
\end{aligned}
\end{dmath}

This gives
%
\begin{dmath}\label{eqn:gradQuantumProblemSet2Problem2:700}
\expectation{ \Bx \cdot \spacegrad V }
=
\frac{\Hbar\omega}{2}
\lr{
2 + 2 + 3 + \Abs{ e^{-i \omega t} \sqrt{2} + e^{i \omega t} }^2
}
=
\frac{\Hbar\omega}{2}
\lr{
2 + 2 + 3 + 1 + 2 + 2 \sqrt{2} \cos(2 \omega t)
}
\end{dmath}

or
%
%\begin{dmath}\label{eqn:gradQuantumProblemSet2Problem2:340}
\boxedEquation{eqn:gradQuantumProblemSet2Problem2:340}{
\expectation{ \Bx \cdot \spacegrad V }
=
\Hbar \omega
\lr{ 5 + \sqrt{2} \cos( 2 \omega t ) }.
}
%\end{dmath}

For the kinetic portion, we've got
%
\begin{dmath}\label{eqn:gradQuantumProblemSet2Problem2:360}
\expectation{\frac{\Bp^2}{m}}
=
%\frac{\Hbar^2}{2 x_0^2 m}
\frac{\Hbar \omega}{2}
\sum_{k=1}^3
\bra{\psi}
\lr{ a_k e^{-i\omega t} - a_k^\dagger e^{i\omega t} }
\lr{ a_k^\dagger e^{i\omega t} - a_k e^{-i\omega t} } \ket{\psi}.
\end{dmath}

The squared momentum operator has been factored into conjugate pairs, so only the action on \( \ket{\psi} \) need be computed.  For the \( x \) component of that operation we have
%
\begin{dmath}\label{eqn:gradQuantumProblemSet2Problem2:380}
\lr{ a_x^\dagger e^{i\omega t} - a_x e^{-i\omega t} } \lr{ \ket{0,0,0} + \ket{0,0,2} }
=
a_x^\dagger e^{i\omega t} \lr{ \ket{0,0,0} + \ket{0,0,2} }
=
e^{i\omega t}
\lr{ \ket{1,0,0} + \ket{1,0,2} }.
\end{dmath}
%
The \( y \) component, by inspection, must be
%
\begin{dmath}\label{eqn:gradQuantumProblemSet2Problem2:400}
\lr{ a_x^\dagger e^{i\omega t} - a_x e^{-i\omega t} } \lr{ \ket{0,0,0} + \ket{0,0,2} }
=
e^{i\omega t}
\lr{ \ket{0,1,0} + \ket{0,1,2} }.
\end{dmath}
%
The \( z \) component is slightly messier to compute
%
\begin{dmath}\label{eqn:gradQuantumProblemSet2Problem2:420}
\lr{ a_z^\dagger e^{i\omega t} - a_z e^{-i\omega t} } \lr{ \ket{0,0,0} + \ket{0,0,2} }
=
e^{i\omega t} \lr{ \ket{0,0,1} + \sqrt{3} \ket{0,0,3} }
- e^{-i\omega t} \sqrt{2} \ket{0,0,1}
=
\lr{ e^{i\omega t} - e^{-i\omega t} \sqrt{2} } \ket{0,0,1}
+ e^{i\omega t} \sqrt{3} \ket{0,0,3}.
\end{dmath}
%
Summation of all the component contributions gives
%
\begin{dmath}\label{eqn:gradQuantumProblemSet2Problem2:440}
\expectation{\frac{\Bp^2}{m}}
=
\frac{\Hbar \omega}{2} \lr{
2 + 2 + 3 +
+ \Abs{e^{i\omega t} - e^{-i\omega t} \sqrt{2}}^2
}
=
\frac{\Hbar \omega}{2} \lr{ 4 + 3 + 1 + 2 - 2 \sqrt{2} \cos(2 \omega t)},
\end{dmath}
%
or
\boxedEquation{eqn:gradQuantumProblemSet2Problem2:460}{
\expectation{\frac{\Bp^2}{m}}
=
\Hbar \omega \lr{ 5 - \sqrt{2} \cos(2 \omega t) }.
}

This does not match \cref{eqn:gradQuantumProblemSet2Problem2:340}.

\makeSubAnswer{}{gradQuantum:problemSet2:2b}

To derive \cref{eqn:gradQuantumProblemSet2Problem2:41} the expectation had to be calculated with respect to stationary (non-time dependent) states.  As an example, we confirmed this by showing the r.h.s was zero with respect to the energy eigenstates.  For the superposition state \( \ket{0,0,0} + \ket{0,0,2} \) this was observed to be insufficient.  It is clear that the perfect cancellation of the time dependence that was required so that \( \expectation{\frac{\Bp^2}{m}} = \expectation{ \Bx \cdot \spacegrad V } \) will not, in general, be possible for such superposition states.  The virial theorem requires not only expectations with respect to stationary states, but requires those stationary states to also be energy eigenstates.

%%%\paragraph{Junk?}
%%%We've seen above for the harmonic oscillator that the r.h.s vanished when the expectations were with respect to energy eigenstates.
%%%
%%%This derivative will be equal zero when
%%%\begin{dmath}\label{eqn:gradQuantumProblemSet2Problem2:160}
%%%0
%%%=
%%%\ddt{} \expectation{ \Bx \cdot \Bp }
%%%=
%%%\bra{\psi} \ddt{} \Bx \cdot \Bp \ket{\psi}
%%%=
%%%\bra{\psi} \ddt{\Bx} \cdot \Bp + \Bx \cdot \ddt{\Bp} \ket{\psi}
%%%=
%%%\bra{\psi} \PD{\Bp}{H} \cdot \Bp - \Bx \cdot \PD{\Bx}{H} \ket{\psi},
%%%\end{dmath}
%%%
%%%or when
%%%
%%%\begin{dmath}\label{eqn:gradQuantumProblemSet2Problem2:180}
%%%\expectation{ \PD{\Bp}{H} \cdot \Bp } = \expectation{\Bx \cdot \PD{\Bx}{H}}.
%%%\end{dmath}
%%%
%%%Consider the SHO Hamiltonian for example.  For that we have
%%%
%%%\begin{dmath}\label{eqn:gradQuantumProblemSet2Problem2:200}
%%%\expectation{ \PD{\Bp}{H} \cdot \Bp }
%%%=
%%%\expectation{\frac{\Bp}{m} \cdot \Bp}
%%%=
%%%\expectation{\frac{\Bp^2}{m}},
%%%\end{dmath}
%%%
%%%and
%%%\begin{dmath}\label{eqn:gradQuantumProblemSet2Problem2:220}
%%%\expectation{ \Bx \cdot \PD{\Bx}{H} }
%%%=
%%%\expectation{ \Bx \cdot m \omega^2 \Bx }
%%%=
%%%m \omega^2 \expectation{ \Bx^2 }.
%%%\end{dmath}
%%%
%%%In one dimension, with expectation relative to state \( \ket{n} \) these both equal \( \Hbar \omega (n + 1/2) \), the energy of state \( \ket{n} \).  It seems plausible that there is an additional argument that could be used to show that is the case more generally.
}
}

         % p9:
         %
% Copyright � 2015 Peeter Joot.  All Rights Reserved.
% Licenced as described in the file LICENSE under the root directory of this GIT repository.
%
%\input{../blogpost.tex}
%\renewcommand{\basename}{symmetricHamiltonianEvolution}
%\renewcommand{\dirname}{notes/phy1520/}
%%\newcommand{\dateintitle}{}
%%\newcommand{\keywords}{}
%
%\input{../peeter_prologue_print2.tex}
%
%\usepackage{peeters_layout_exercise}
%\usepackage{peeters_braket}
%\usepackage{peeters_figures}
%
%\beginArtNoToc
%
%\generatetitle{A symmetric real Hamiltonian}
%\chapter{A symmetric real Hamiltonian}
%\label{chap:symmetricHamiltonianEvolution}
%
\makeoproblem{A symmetric real Hamiltonian.}{problem:symmetricHamiltonianEvolution:9}{\citep{sakurai2014modern} pr. 2.9}{
\index{Hamiltonian!symmetric real}
Find the time evolution for the state \( \ket{a'} \) for a Hamiltonian of the form
%
\begin{equation}\label{eqn:symmetricHamiltonianEvolution:20}
H = \delta \lr{ \ket{a'}\bra{a'} + \ket{a''}\bra{a''} }.
\end{equation}
} % problem
%
\makeanswer{problem:symmetricHamiltonianEvolution:9}{
%
This Hamiltonian has the matrix representation
%
\begin{dmath}\label{eqn:symmetricHamiltonianEvolution:40}
H =
\begin{bmatrix}
0 & \delta \\
\delta & 0
\end{bmatrix},
\end{dmath}
%
which has a characteristic equation of
%
\begin{equation}\label{eqn:symmetricHamiltonianEvolution:60}
\lambda^2 -\delta^2 = 0,
\end{equation}
%
so the energy eigenvalues are \( \pm \delta \).

The diagonal basis states are respectively
%
\begin{dmath}\label{eqn:symmetricHamiltonianEvolution:80}
\ket{\pm\delta} =
\inv{\sqrt{2}}
\begin{bmatrix}
\pm 1 \\
1
\end{bmatrix}.
\end{dmath}
%
The time evolution operator is
%
\begin{dmath}\label{eqn:symmetricHamiltonianEvolution:100}
U
= e^{-i H t/\Hbar}
=
  e^{-i \delta t/\Hbar} \ket{+\delta}\bra{+\delta}
+ e^{i \delta t/\Hbar} \ket{-\delta}\bra{-\delta}
=
\frac{e^{-i \delta t/\Hbar} }{2}
\begin{bmatrix}
1 & 1
\end{bmatrix}
\begin{bmatrix}
1  \\
1
\end{bmatrix}
+ \frac{e^{i \delta t/\Hbar} }{2}
\begin{bmatrix}
-1 & 1
\end{bmatrix}
\begin{bmatrix}
-1  \\
1
\end{bmatrix}
=
\frac{e^{-i \delta t/\Hbar} }{2}
\begin{bmatrix}
1 & 1 \\
1 & 1
\end{bmatrix}
+\frac{e^{i \delta t/\Hbar} }{2}
\begin{bmatrix}
1 & -1 \\
-1 & 1
\end{bmatrix}
=
\begin{bmatrix}
\cos(\delta t/\Hbar) & -i\sin(\delta t/\Hbar) \\
-i \sin(\delta t/\Hbar) & \cos(\delta t/\Hbar) \\
\end{bmatrix}.
\end{dmath}
%
%The non-diagonal states have the matrix representation
%
%\begin{equation}\label{eqn:symmetricHamiltonianEvolution:120}
%\begin{aligned}
%\ket{a'} &= \inv{\sqrt{2}} \lr{ \ket{+\delta} - \ket{-\delta} } \\
%\ket{a''} &= \inv{\sqrt{2}} \lr{ \ket{+\delta} + \ket{-\delta} },
%\end{aligned}
%\end{equation}
%
%so
The desired time evolution in the original basis is
%
\begin{dmath}\label{eqn:symmetricHamiltonianEvolution:140}
\ket{a', t}
=
e^{-i H t/\Hbar}
\ket{a', 0}
=
\begin{bmatrix}
\cos(\delta t/\Hbar) & -i\sin(\delta t/\Hbar) \\
-i \sin(\delta t/\Hbar) & \cos(\delta t/\Hbar) \\
\end{bmatrix}
\begin{bmatrix}
1 \\
0
\end{bmatrix}
=
\begin{bmatrix}
\cos(\delta t/\Hbar) \\
-i \sin(\delta t/\Hbar)
\end{bmatrix}
=
\cos(\delta t/\Hbar) \ket{a',0} -i \sin(\delta t/\Hbar) \ket{a'',0}.
\end{dmath}
%
This evolution has the same structure as left circularly polarized light.

The probability of finding the system in state \( \ket{a''} \) given an initial state of \( \ket{a',0} \) is
%
\begin{equation}\label{eqn:symmetricHamiltonianEvolution:160}
P
=
\Abs{\braket{a''}{a',t}}^2
=
\sin^2 \lr{ \delta t/\Hbar }.
\end{equation}
} % answer

%\EndArticle

         % p12
         %
% Copyright � 2015 Peeter Joot.  All Rights Reserved.
% Licenced as described in the file LICENSE under the root directory of this GIT repository.
%
%\input{../blogpost.tex}
%\renewcommand{\basename}{translationExpectation}
%\renewcommand{\dirname}{notes/phy1520/}
%%\newcommand{\dateintitle}{}
%%\newcommand{\keywords}{}
%
%\input{../peeter_prologue_print2.tex}
%
%\usepackage{peeters_layout_exercise}
%\usepackage{peeters_braket}
%\usepackage{peeters_figures}
%
%\beginArtNoToc
%
%\generatetitle{SHO translation operator expectation}
%\chapter{SHO translation operator expectation}
%\label{chap:translationExpectation}

\makeoproblem{SHO translation operator expectation.}{problem:translationExpectation:2.12}{\citep{sakurai2014modern} pr. 2.12}{

\index{translation!expectation}
\index{harmonic oscillator!translation operator}
Using the Heisenberg picture evaluate the expectation of the position operator \( \expectation{x} \) with respect to the initial time state

\begin{dmath}\label{eqn:translationExpectation:20}
\ket{\alpha, 0} = e^{-i p_0 a/\Hbar} \ket{0},
\end{dmath}

where \( p_0 \) is the initial time position operator, and \( a \) is a constant with dimensions of position.

} % problem

\makeanswer{problem:translationExpectation:2.12}{

Recall that the Heisenberg picture position operator expands to

\begin{dmath}\label{eqn:translationExpectation:40}
x^\txtH(t)
= U^\dagger x U
= x_0 \cos(\omega t) + \frac{p_0}{m \omega} \sin(\omega t),
\end{dmath}

so the expectation of the position operator is
\begin{dmath}\label{eqn:translationExpectation:60}
\expectation{x}
=
\bra{0} e^{i p_0 a/\Hbar} \lr{ x_0 \cos(\omega t) + \frac{p_0}{m \omega} \sin(\omega t) } e^{-i p_0 a/\Hbar} \ket{0}
=
\bra{0} \lr{ e^{i p_0 a/\Hbar} x_0 \cos(\omega t) e^{-i p_0 a/\Hbar} \cos(\omega t) + \frac{p_0}{m \omega} \sin(\omega t) } \ket{0}.
\end{dmath}

The exponential sandwich above can be expanded using the Baker-Campbell-Hausdorff \citep{wiki:bakercampbellHausdorff} formula

\begin{dmath}\label{eqn:translationExpectation:80}
\begin{aligned}
e^{i p_0 a/\Hbar} x_0 e^{-i p_0 a/\Hbar}
&=
x_0
+ \frac{i a}{\Hbar} \antisymmetric{p_0}{x_0}
+ \inv{2!} \lr{\frac{i a}{\Hbar}}^2 \antisymmetric{p_0}{\antisymmetric{p_0}{x_0}}
+ \cdots \\
&=
x_0
+ \frac{i a}{\Hbar} \lr{ -i \Hbar }
+ \inv{2!} \lr{\frac{i a}{\Hbar}}^2 \antisymmetric{p_0}{-i \Hbar}
+ \cdots \\
&=
x_0 + a.
\end{aligned}
\end{dmath}

The position expectation with respect to this translated state is

\begin{dmath}\label{eqn:translationExpectation:100}
\expectation{x}
= \bra{0} \lr{ (x_0 + a)\cos(\omega t) + \frac{p_0}{m \omega} \sin(\omega t) }\ket{0}
= a \cos(\omega t).
\end{dmath}

The final simplification above follows from \( \bra{n} x \ket{n} = \bra{n} p \ket{n} = 0 \).

} % answer

%\EndArticle

         % p14
         %
% Copyright � 2015 Peeter Joot.  All Rights Reserved.
% Licenced as described in the file LICENSE under the root directory of this GIT repository.
%
%\input{../blogpost.tex}
%\renewcommand{\basename}{shoExpectations}
%\renewcommand{\dirname}{notes/phy1520/}
%%\newcommand{\dateintitle}{}
%%\newcommand{\keywords}{}
%
%\input{../peeter_prologue_print2.tex}
%
%\usepackage{peeters_layout_exercise}
%\usepackage{peeters_braket}
%\usepackage{peeters_figures}
%
%\beginArtNoToc
%
%\generatetitle{Expectations for SHO Hamiltonian, and virial theorem.}
%%\label{chap:shoExpectations}
%
\makeoproblem
%{Expectations for SHO Hamiltonian, and virial theorem.}
{SHO Hamiltonian, and virial theorem.}
{problem:shoExpectations:1}{\citep{sakurai2014modern} pr. 2.14}{
\index{harmonic oscillator}
\index{virial theorem}
\makesubproblem{}{problem:shoExpectations:1:a}
%
For a 1D SHO, compute \(
\bra{m} x \ket{n},
\bra{m} x^2 \ket{n},
\bra{m} p \ket{n},
\bra{m} p^2 \ket{n} \) and \( \bra{m} \symmetric{x}{p} \ket{n} \).
%
\makesubproblem{}{problem:shoExpectations:1:b}
%
Verify the virial theorem is satisfied for energy eigenstates.
} % problem
%
\makeanswer{problem:shoExpectations:1}{
%
\makeSubAnswer{}{problem:shoExpectations:1:a}
%
Using
%
\begin{equation}\label{eqn:shoExpectations:20}
\begin{aligned}
x &= \frac{x_0}{\sqrt{2}} \lr{ a + a^\dagger } \\
p &= \frac{i\Hbar}{x_0 \sqrt{2}} \lr{ a^\dagger - a} \\
a(t) &= a(0) e^{-i \omega t} \\
a(0) \ket{n} &= \sqrt{n} \ket{n-1} \\
a^\dagger(0) \ket{n} &= \sqrt{n+1} \ket{n+1} \\
x_0^2 &= \frac{\Hbar}{\omega m},
\end{aligned}
\end{equation}
we have
%
\begin{dmath}\label{eqn:shoExpectations:40}
\bra{m} x \ket{n}
=
\frac{x_0}{\sqrt{2}} \bra{m} \lr{ a + a^\dagger } \ket{n}
=
\frac{x_0}{\sqrt{2}} \bra{m}
\lr{
e^{-i \omega t} \sqrt{n} \ket{n-1}
+
e^{i \omega t} \sqrt{n+1} \ket{n+1}
}
=
\frac{x_0}{\sqrt{2}} \lr{
\delta_{m, n-1} e^{-i \omega t} \sqrt{n}
+
\delta_{m, n+1} e^{i \omega t} \sqrt{n+1}
},
\end{dmath}
%
\begin{dmath}\label{eqn:shoExpectations:60}
\bra{m} x^2 \ket{n}
=
\frac{x_0^2}{2} \bra{m} \lr{ a + a^\dagger }^2 \ket{n}
=
\frac{x_0^2}{2}
\lr{
e^{i \omega t} \sqrt{m} \bra{m-1}
+
e^{-i \omega t} \sqrt{m+1} \bra{m+1}
}
\lr{
e^{-i \omega t} \sqrt{n} \ket{n-1}
+
e^{i \omega t} \sqrt{n+1} \ket{n+1}
}
=
\frac{x_0^2}{2}
\lr{
\delta_{m+1,n+1} \sqrt{(m+1)(n+1)}
+\delta_{m+1,n-1} \sqrt{(m+1)n} e^{-2 i \omega t}
+\delta_{m-1,n+1} \sqrt{m(n+1)} e^{2 i \omega t}
+\delta_{m-1,n-1} \sqrt{m n}
},
\end{dmath}
%
\begin{dmath}\label{eqn:shoExpectations:80}
\bra{m} p \ket{n}
=
\frac{i\Hbar}{\sqrt{2} x_0} \bra{m} \lr{ a^\dagger - a} \ket{n}
=
\frac{i\Hbar}{\sqrt{2} x_0} \bra{m} \lr{
e^{i \omega t} \sqrt{n+1} \ket{n+1}
-
e^{-i \omega t} \sqrt{n} \ket{n-1}
}
=
\frac{i\Hbar}{\sqrt{2} x_0} \lr{
\delta_{m,n+1} e^{i \omega t} \sqrt{n+1}
-
\delta_{m,n-1} e^{-i \omega t} \sqrt{n}
},
\end{dmath}
%
\begin{dmath}\label{eqn:shoExpectations:100}
\bra{m} p^2 \ket{n}
=
\frac{\Hbar^2}{2 x_0^2} \ket{m} \lr{ a - a^\dagger }  \lr{ a^\dagger - a} \ket{n}
=
\frac{\Hbar^2}{2 x_0^2}
\lr{
-e^{-i \omega t} \sqrt{m+1} \bra{m+1}
+
e^{i \omega t} \sqrt{m} \bra{m-1}
}
\lr{
e^{i \omega t} \sqrt{n+1} \ket{n+1}
-
e^{-i \omega t} \sqrt{n} \ket{n-1}
}
=
\frac{\Hbar^2}{2 x_0^2}
\lr{
\delta_{m+1,n+1} \sqrt{(m+1)(n+1)}
+\delta_{m+1,n-1} \sqrt{(m+1)n} e^{-2 i \omega t}
+\delta_{m-1,n+1} \sqrt{m(n+1)} e^{2 i \omega t}
+\delta_{m-1,n-1} \sqrt{m n}
}.
\end{dmath}
%
For the anticommutator \( \symmetric{x}{p} \), we have
%
\begin{dmath}\label{eqn:shoExpectations:120}
\symmetric{x}{p}
=
\frac{i\Hbar}{2}
\lr{
\lr{ a e^{-i \omega t} + a^\dagger e^{i \omega t} } \lr{ a^\dagger e^{i \omega t} - a e^{-i \omega t} }
-
\lr{ a^\dagger e^{i \omega t} - a e^{-i \omega t} }
\lr{ a e^{-i \omega t} + a^\dagger e^{i \omega t} }
}
=
\frac{i\Hbar}{2}
\lr{
- a^2 e^{- 2 i \omega t}
+ (a^\dagger)^2 e^{ 2 i \omega t}
+ a a^\dagger
- a^\dagger a
+ a^2 e^{- 2 i \omega t}
- (a^\dagger)^2 e^{ 2 i \omega t}
- a^\dagger a
+ a a^\dagger
}
=
i\Hbar
\lr{
a a^\dagger - a^\dagger a
},
\end{dmath}
%
so
%
\begin{dmath}\label{eqn:shoExpectations:140}
\bra{m} \symmetric{x}{p} \ket{n}
=
i\Hbar
\bra{m}
\lr{
a a^\dagger - a^\dagger a
}
\ket{n}
=
i\Hbar
\bra{m}
\lr{
\sqrt{(n+1)^2}\ket{n}
-\sqrt{n^2}\ket{n}
}
=
i\Hbar
\bra{m}
\lr{
2 n + 1
}
\ket{n}.
\end{dmath}
%
\makeSubAnswer{}{problem:shoExpectations:1:b}
%
For the SHO, the virial theorem requires \( \expectation{p^2/m} = \expectation{m \omega x^2} \).  That momentum expectation with respect to the eigenstate \( \ket{n} \) is
%
\begin{dmath}\label{eqn:shoExpectations:160}
\expectation{p^2/m}
=
\frac{\Hbar^2}{2 x_0^2 m}
\lr{
\sqrt{(n+1)(n+1)}
+
\sqrt{n n}
}
=
\frac{\Hbar^2 m \omega}{2 \Hbar m} \lr{ 2 n + 1 }
=
\Hbar \omega \lr{ n + \inv{2} }.
\end{dmath}
%
For the position expectation we've got
%
\begin{dmath}\label{eqn:shoExpectations:180}
\expectation{m \omega x^2}
=
\frac{m \omega^2 x_0^2}{2}
\lr{
\sqrt{(n+1)(n+1)}
+ \sqrt{n n}
}
=
\frac{m \omega^2 \Hbar}{2 m \omega}
\lr{
\sqrt{(n+1)(n+1)}
+ \sqrt{n n}
}
=
\frac{\omega \Hbar}{2 }
\lr{ 2 n + 1 }
=
\omega \Hbar
\lr{ n + \inv{2} }.
\end{dmath}
%
This shows that the virial theorem holds for the SHO Hamiltonian for eigenstates.
%
} % answer

%\EndArticle

         % p15
         %
% Copyright � 2015 Peeter Joot.  All Rights Reserved.
% Licenced as described in the file LICENSE under the root directory of this GIT repository.
%
%\input{../blogpost.tex}
%\renewcommand{\basename}{shoMomentumSpace}
%\renewcommand{\dirname}{notes/phy1520/}
%%\newcommand{\dateintitle}{}
%%\newcommand{\keywords}{}
%
%\input{../peeter_prologue_print2.tex}
%
%\usepackage{peeters_layout_exercise}
%\usepackage{peeters_braket}
%\usepackage{peeters_figures}
%\usepackage{macros_qed}
%
%\beginArtNoToc
%
%\generatetitle{Momentum space representation of Schr\"{o}dinger equation}
%\chapter{Momentum space representation of Schr\"{o}dinger equation}
%\label{chap:shoMomentumSpace}

\makeoproblem{Momentum space representation of Schr\"{o}dinger equation.}{problem:shoMomentumSpace:15}{\citep{sakurai2014modern} pr. 2.15}{
\index{Schr\"{o}dinger equation!momentum space}

Using
%
\begin{dmath}\label{eqn:shoMomentumSpace:20}
\braket{x'}{p'} = \inv{\sqrt{2 \pi \Hbar}} e^{i p' x'/\Hbar},
\end{dmath}
%
show that
%
\begin{dmath}\label{eqn:shoMomentumSpace:40}
\bra{p'} x \ket{\alpha} = i \Hbar \PD{p'}{} \braket{p'}{\alpha}.
\end{dmath}

Use this to find the momentum space representation of the Schr\"{o}dinger equation for the one dimensional SHO and the energy eigenfunctions in their momentum representation.

} % problem

\makeanswer{problem:shoMomentumSpace:15}{
To expand the matrix element, introduce both momentum and position space identity operators
%
\begin{dmath}\label{eqn:shoMomentumSpace:60}
\bra{p'} x \ket{\alpha}
=
\int dx' dp'' \braket{p'}{x'}\bra{x'}x \ket{p''}\braket{p''}{\alpha}
=
\int dx' dp'' \braket{p'}{x'}x'\braket{x'}{p''}\braket{p''}{\alpha}
=
\inv{2 \pi \Hbar}
\int dx' dp'' e^{-i p' x'/\Hbar} x' e^{i p'' x'/\Hbar} \braket{p''}{\alpha}
=
\inv{2 \pi \Hbar}
\int dx' dp'' x' e^{i (p'' - p') x'/\Hbar} \braket{p''}{\alpha}
=
\inv{2 \pi \Hbar}
\int dx' dp'' \frac{d}{dp''}\lr{ \frac{-i \Hbar} e^{i (p'' - p') x'/\Hbar} } \braket{p''}{\alpha}
=
i \Hbar
\int dp''
\lr{ \inv{2 \pi \Hbar}
\int dx' e^{i (p'' - p') x'/\Hbar} } \frac{d}{dp''} \braket{p''}{\alpha}
=
i \Hbar
\int dp'' \delta(p''- p')
\frac{d}{dp''} \braket{p''}{\alpha}
=
i \Hbar
\frac{d}{dp'} \braket{p'}{\alpha}. \qedmarker
\end{dmath}

Schr\"{o}dinger's equation for a time dependent state \( \ket{\alpha} = U(t) \ket{\alpha,0} \) is
%
\begin{dmath}\label{eqn:shoMomentumSpace:80}
i \Hbar \PD{t}{} \ket{\alpha} = H \ket{\alpha},
\end{dmath}
%
with the momentum representation
%
\begin{dmath}\label{eqn:shoMomentumSpace:100}
i \Hbar \PD{t}{} \braket{p'}{\alpha} = \bra{p'} H \ket{\alpha}.
\end{dmath}

Expansion of the Hamiltonian matrix element for a strictly spatial dependent potential \( V(x) \) gives
%
\begin{dmath}\label{eqn:shoMomentumSpace:120}
\bra{p'} H \ket{\alpha}
=
\bra{p'} \lr{\frac{p^2}{2m} + V(x) } \ket{\alpha}
=
\frac{(p')^2}{2m}
+ \bra{p'} V(x) \ket{\alpha}.
\end{dmath}

Assuming a Taylor representation of the potential \( V(x) = \sum c_k x^k \), we want to calculate
%
\begin{dmath}\label{eqn:shoMomentumSpace:140}
\bra{p'} V(x) \ket{\alpha}
= \sum c_k \bra{p'} x^k \ket{\alpha}.
\end{dmath}

With \( \ket{\alpha} = \ket{p''} \) \cref{eqn:shoMomentumSpace:40} provides the \( k = 1 \) term
%
\begin{dmath}\label{eqn:shoMomentumSpace:160}
\bra{p'} x \ket{p''}
= i \Hbar \frac{d}{dp'} \braket{p'}{p''}
= i \Hbar \frac{d}{dp'} \delta(p' - p''),
\end{dmath}
%
where it is implied here that the derivative is operating on not just the delta function, but on all else that follows.

Using this the higher powers of \( \bra{p'} x^k \ket{\alpha} \) can be found easily.  For example for \( k = 2 \) we have
%
\begin{dmath}\label{eqn:shoMomentumSpace:180}
\bra{p'} x^2 \ket{\alpha}
=
\int dp''
\bra{p'} x \ket{p''}\bra{p''} x \ket{\alpha}
=
\int dp''
i \Hbar
\frac{d}{dp'} \delta(p' - p'') i \Hbar \frac{d}{dp''} \braket{p''}{\alpha}
=
\lr{ i \Hbar }^2 \frac{d^2}{d(p')^2} \braket{p'}{\alpha}.
\end{dmath}

This means that the potential matrix element is
%
\begin{dmath}\label{eqn:shoMomentumSpace:200}
\bra{p'} V(x) \ket{\alpha}
=
\sum c_k \lr{ i \Hbar \frac{d}{dp'} }^k \braket{p'}{\alpha}
= V\lr{ i \Hbar \frac{d}{dp'} }.
\end{dmath}

Writing \( \Psi_\alpha(p') = \braket{p'}{\alpha} \), the momentum space representation of Schr\"{o}dinger's equation for a position dependent potential is
%
%\begin{dmath}\label{eqn:shoMomentumSpace:220}
\boxedEquation{eqn:shoMomentumSpace:220}{
i \Hbar \PD{t}{} \Psi_\alpha(p')
=
\lr{ \frac{(p')^2}{2m} + V\lr{ i \Hbar \PDi{p'}{} } } \Psi_\alpha(p').
}
%\end{dmath}

For the SHO Hamiltonian the potential is \( V(x) = (1/2) m \omega^2 x^2 \), so the Schr\"{o}dinger equation is
%
\begin{dmath}\label{eqn:shoMomentumSpace:240}
i \Hbar \PD{t}{} \Psi_\alpha(p')
=
\lr{ \frac{(p')^2}{2m} - \inv{2} m \omega^2 \Hbar^2 \frac{\partial^2}{\partial(p')^2} } \Psi_\alpha(p')
=
\inv{2 m} \lr{ (p')^2 - m^2 \omega^2 \Hbar^2 \frac{\partial^2}{\partial(p')^2} } \Psi_\alpha(p').
\end{dmath}

To determine the wave functions, let's non-dimensionalize this and compare to the position space Schr\"{o}dinger equation.  Let
%
\begin{dmath}\label{eqn:shoMomentumSpace:260}
p_0^2 = m \omega \hbar,
\end{dmath}
%
so
\begin{dmath}\label{eqn:shoMomentumSpace:280}
i \Hbar \PD{t}{} \Psi_\alpha(p')
=
\frac{p_0^2}{2 m} \lr{ \lr{\frac{p'}{p_0}}^2 - \frac{\partial^2}{\partial(p'/p_0)^2} } \Psi_\alpha(p')
=
\frac{\omega \Hbar}{2}\lr{
- \frac{\partial^2}{\partial(p'/p_0)^2} +
\lr{\frac{p'}{p_0}}^2
} \Psi_\alpha(p').
\end{dmath}

Compare this to the position space equation with \( x_0^2 = m \omega/\Hbar \),
%
\begin{dmath}\label{eqn:shoMomentumSpace:300}
i \Hbar \PD{t}{} \Psi_\alpha(x')
=
\lr{ -\frac{\Hbar^2}{2m} \frac{\partial^2}{\partial(x')^2}
+
\inv{2} m \omega^2 (x')^2 }
\Psi_\alpha(x')
=
\frac{\Hbar^2}{2m}
\lr{ -\frac{\partial^2}{\partial(x')^2}
+
\frac{m^2 \omega^2}{\Hbar^2} (x')^2 }
\Psi_\alpha(x')
=
\frac{\Hbar^2 x_0^2}{2m}
\lr{
-\frac{\partial^2}{\partial(x'/x_0)^2}
+
\lr{\frac{x'}{x_0}}^2
}
\Psi_\alpha(x')
=
\frac{\Hbar \omega}{2}
\lr{
-\frac{\partial^2}{\partial(x'/x_0)^2}
+
\lr{\frac{x'}{x_0}}^2
}
\Psi_\alpha(x').
\end{dmath}

It's clear that there is a straightforward duality relationship between the respective wave functions.  Since
%
\begin{dmath}\label{eqn:shoMomentumSpace:320}
\braket{x'}{n} =
\inv{\pi^{1/4} \sqrt{2^n n!} x_0^{n + 1/2}}  \lr{ x' - x_0^2 \frac{d}{dx'} }^n \exp\lr{ -\inv{2} \lr{\frac{x'}{x_0}}^2 },
\end{dmath}
%
\index{wave function!momentum space}
the momentum space wave functions are
%
\begin{dmath}\label{eqn:shoMomentumSpace:340}
\braket{p'}{n} =
\inv{\pi^{1/4} \sqrt{2^n n!} p_0^{n + 1/2}}  \lr{ p' - p_0^2 \frac{d}{dp'} }^n \exp\lr{ -\inv{2} \lr{\frac{p'}{p_0}}^2 }.
\end{dmath}

} % answer

%\EndArticle

         % p16:
         %%
% Copyright � 2015 Peeter Joot.  All Rights Reserved.
% Licenced as described in the file LICENSE under the root directory of this GIT repository.
%
%\input{../blogpost.tex}
%\renewcommand{\basename}{correlationSHO}
%\renewcommand{\dirname}{notes/phy1520/}
%%\newcommand{\dateintitle}{}
%%\newcommand{\keywords}{}
%
%\input{../peeter_prologue_print2.tex}
%
%\usepackage{peeters_layout_exercise}
%\usepackage{peeters_braket}
%\usepackage{peeters_figures}
%
%\beginArtNoToc
%
%\generatetitle{Correlation function}
%\chapter{Correlation function}
%\label{chap:correlationSHO}

\makeoproblem{Correlation function.}{problem:correlationSHO:16}{\citep{sakurai2014modern} pr. 2.16}{
\index{correlation function}

A correlation function can be defined as

\begin{dmath}\label{eqn:correlationSHO:20}
C(t) = \expectation{ x(t) x(0) }.
\end{dmath}

Using a Heisenberg picture \( x(t) \) calculate this correlation for the one dimensional SHO ground state.

} % problem

\makeanswer{problem:correlationSHO:16}{
The time dependent Heisenberg picture position operator was found to be

\begin{dmath}\label{eqn:correlationSHO:40}
x(t) = x(0) \cos(\omega t) + \frac{p(0)}{m \omega} \sin(\omega t),
\end{dmath}

so the correlation function is

\begin{dmath}\label{eqn:correlationSHO:60}
C(t)
=
\bra{0} \lr{ x(0) \cos(\omega t) + \frac{p(0)}{m \omega} \sin(\omega t)} x(0) \ket{0}
=
\cos(\omega t) \bra{0} x(0)^2 \ket{0} + \frac{\sin(\omega t)}{m \omega} \bra{0} p(0) x(0) \ket{0}
=
\frac{\Hbar \cos(\omega t) }{2 m \omega} \bra{0} \lr{ a + a^\dagger}^2 \ket{0} - \frac{i \Hbar}{m \omega} \sin(\omega t),
\end{dmath}

But
\begin{dmath}\label{eqn:correlationSHO:80}
\lr{ a + a^\dagger} \ket{0}
=
a^\dagger \ket{0}
=
\sqrt{1} \ket{1}
=
\ket{1},
\end{dmath}

so

\begin{dmath}\label{eqn:correlationSHO:100}
C(t) = x_0^2 \lr{ \inv{2} \cos(\omega t) - i \sin(\omega t) },
\end{dmath}

where \( x_0^2 = \Hbar/(m \omega) \), not to be confused with \( x(0)^2 \).

} % answer

%\EndArticle

         % ps2. correlation
         %
% Copyright � 2015 Peeter Joot.  All Rights Reserved.
% Licenced as described in the file LICENSE under the root directory of this GIT repository.
%
%
\makeoproblem{Correlation function.}{gradQuantum:problemSet2:3}{2015 ps2.3}{
\index{correlation function}
Consider \( \expectation{x(0)x(t)} \) and \( \expectation{p(0)p(t)} \) where operators are in the Heisenberg representation. These are called correlation functions. Evaluate this for the 1D harmonic oscillator in an energy eigenstate \( \ket{n} \).
} % makeproblem
%
%
% ground state version of this in ../phy1520/correlationSHO.tex
%
\makeanswer{gradQuantum:problemSet2:3}{
\withproblemsetsParagraph{

In the Heisenberg representation the position and momentum operators evolve as
%
\begin{equation}\label{eqn:gradQuantumProblemSet2Problem3:20}
\begin{aligned}
x(t) &= x(0) \cos(\omega t) + \frac{p(0)}{m \omega} \sin(\omega t) \\
p(t) &= p(0) \cos(\omega t) - m \omega x(0) \sin(\omega t).
\end{aligned}
\end{equation}
%
To evaluate the expectation operations, we'll want the ladder operator representations of the position and momentum operators
%
\begin{equation}\label{eqn:gradQuantumProblemSet2Problem3:40}
\begin{aligned}
x(0) &= \frac{x_0}{\sqrt{2}} \evalbar{\lr{ a + a^\dagger }}{t = 0} \\
p(0) &= \frac{i \Hbar}{x_0 \sqrt{2}} \evalbar{\lr{ a^\dagger - a}}{t = 0},
\end{aligned}
\end{equation}

where
%
\begin{equation}\label{eqn:gradQuantumProblemSet2Problem3:60}
x_0 = \sqrt{\frac{\Hbar}{m \omega}}.
\end{equation}
%
The expectations of interest, with the raising and lowering operators evaluated at \( t = 0 \), are
%
\begin{dmath}\label{eqn:gradQuantumProblemSet2Problem3:80}
\bra{n} x(0) x(0) \ket{n}
=
\frac{x_0^2}{2} \bra{n} \lr{ a + a^\dagger }^2 \ket{n}
=
\frac{x_0^2}{2}
\lr{ \sqrt{n+1} \bra{n+1} + \sqrt{n} \bra{n-1} }
\lr{ \sqrt{n+1} \ket{n+1} + \sqrt{n} \ket{n-1} }
=
\frac{x_0^2}{2} \lr{ 2 n + 1 },
\end{dmath}
%
and
\begin{dmath}\label{eqn:gradQuantumProblemSet2Problem3:100}
\bra{n} x(0) p(0) \ket{n}
=
\frac{i \Hbar}{2}
\bra{n} \lr{ a + a^\dagger } \lr{ a^\dagger - a }  \ket{n}
=
\frac{i \Hbar}{2}
\lr{ \sqrt{n+1} \bra{n+1} + \sqrt{n} \bra{n-1} }
\lr{ \sqrt{n+1} \ket{n+1} - \sqrt{n} \ket{n-1} }
=
\frac{i \Hbar}{2},
\end{dmath}
%
\begin{dmath}\label{eqn:gradQuantumProblemSet2Problem3:101}
\bra{n} p(0) p(0) \ket{n}
=
\frac{-\Hbar^2 }{ 2 x_0^2}
\bra{n} \lr{ a^\dagger - a } \lr{ a^\dagger - a }  \ket{n}
=
\frac{-\Hbar^2 }{ 2 x_0^2}
\lr{ \sqrt{n+1} \bra{n+1} - \sqrt{n} \bra{n-1} }
\lr{ \sqrt{n+1} \ket{n+1} - \sqrt{n} \ket{n-1} }
=
\frac{(-\Hbar^2) }{ 2 x_0^2}
\lr{
2 n + 1
},
\end{dmath}
%
and finally
%
\begin{dmath}\label{eqn:gradQuantumProblemSet2Problem3:120}
\bra{n} p(0) x(0) \ket{n}
=
\bra{n}
\lr{
\antisymmetric{p(0)}{x(0)} + x(0) p(0)
}
\ket{n}
=
-i \Hbar + \frac{i\Hbar}{2}
=
\frac{i \Hbar}{2}.
\end{dmath}
%
Now we are ready to compute the correlations.  The position correlation is
%
\begin{dmath}\label{eqn:gradQuantumProblemSet2Problem3:140}
\bra{n} x(0) x(t) \ket{n}
=
\bra{n} x(0) \lr{
x(0) \cos(\omega t) + \frac{p(0)}{m \omega} \sin(\omega t)
} \ket{n}
=
\cos(\omega t) \bra{n} x(0) x(0) \ket{n}
+
\frac{1}{m \omega} \sin(\omega t) \bra{n} x(0) p(0) \ket{n}
=
\cos(\omega t)
\frac{x_0^2}{2} \lr{ 2 n + 1 }
+ \frac{1}{m \omega} \sin(\omega t) \frac{i \Hbar}{2},
\end{dmath}
%
which is
%
%\begin{equation}\label{eqn:gradQuantumProblemSet2Problem3:160}
\boxedEquation{eqn:gradQuantumProblemSet2Problem3:180}{
\bra{n} x(0) x(t) \ket{n}
=
\frac{x_0^2}{2}
\lr{
\lr{ 2 n + 1 }
\cos(\omega t)
+ i \sin(\omega t)
}.
}
%\end{equation}
%
The momentum correlation is
%
\begin{dmath}\label{eqn:gradQuantumProblemSet2Problem3:200}
\bra{n} p(0) p(t) \ket{n}
=
\bra{n} p(0) \lr{
p(0) \cos(\omega t) - m \omega x(0) \sin(\omega t)
} \ket{n}
=
\cos(\omega t)
\lr{ 2 n + 1 }
\frac{(-\Hbar^2) }{ 2 x_0^2}
- m \omega \sin(\omega t) \frac{i \Hbar}{2}.
\end{dmath}
%
With \( p_0^2 = m \omega \Hbar \), this is
%
\boxedEquation{eqn:gradQuantumProblemSet2Problem3:220}{
\bra{n} p(0) p(t) \ket{n}
=
-\frac{p_0^2}{2} \lr{
\lr{ 2 n + 1 }
\cos(\omega t)
+ i \sin(\omega t)
}.
}

}
}

         % p17
         %
% Copyright � 2015 Peeter Joot.  All Rights Reserved.
% Licenced as described in the file LICENSE under the root directory of this GIT repository.
%
%\input{../blogpost.tex}
%\renewcommand{\basename}{shoSuperposition}
%\renewcommand{\dirname}{notes/phy1520/}
%%\newcommand{\dateintitle}{}
%%\newcommand{\keywords}{}
%
%\input{../peeter_prologue_print2.tex}
%
%\usepackage{peeters_layout_exercise}
%\usepackage{peeters_braket}
%\usepackage{peeters_figures}
%
%\beginArtNoToc
%
%\generatetitle{1D SHO linear superposition that maximizes expectation}
%%\chapter{1D SHO linear superposition that maximizes expectation}
%%\label{chap:shoSuperposition}
%
\makeoproblem
%{1D SHO linear superposition that maximizes expectation.}
{Linear superposition expectation maximization.}
{problem:shoSuperposition:1}{\citep{sakurai2014modern} pr. 2.17}{
%
For a 1D SHO
%
\makesubproblem{}{problem:shoSuperposition:1:a}
%
Construct a linear combination of \( \ket{0}, \ket{1} \) that maximizes \( \expectation{x} \) without using wave functions.
%
\makesubproblem{}{problem:shoSuperposition:1:b}
%
How does this state evolve with time?
%
\makesubproblem{}{problem:shoSuperposition:1:c}
%
Evaluate \( \expectation{x} \) using the Schr\"{o}dinger picture.
%
\makesubproblem{}{problem:shoSuperposition:1:d}
%
Evaluate \( \expectation{x} \) using the Heisenberg picture.
%
\makesubproblem{}{problem:shoSuperposition:1:e}
%
Evaluate \( \expectation{(\Delta x)^2} \).

} % problem
%
\makeanswer{problem:shoSuperposition:1}{
%
\makeSubAnswer{}{problem:shoSuperposition:1:a}
%
Forming
%
\begin{dmath}\label{eqn:shoSuperposition:20}
\ket{\psi} = \frac{\ket{0} + \sigma \ket{1}}{\sqrt{1 + \Abs{\sigma}^2}}
\end{dmath}

the position expectation is
%
\begin{dmath}\label{eqn:shoSuperposition:40}
\bra{\psi} x \ket{\psi}
=
\inv{1 + \Abs{\sigma}^2} \lr{ \bra{0} + \sigma^\conj \bra{1} } \frac{x_0}{\sqrt{2}} \lr{ a^\dagger + a } \lr{ \ket{0} + \sigma \ket{1} }.
\end{dmath}
%
Evaluating the action of the operators on the kets, we've got
%
\begin{dmath}\label{eqn:shoSuperposition:60}
\lr{ a^\dagger + a } \lr{ \ket{0} + \sigma \ket{1} }
=
\ket{1} + \sqrt{2} \sigma \ket{2} + \sigma \ket{0}.
\end{dmath}
%
The \( \ket{2} \) term is killed by the bras, leaving
%
\begin{dmath}\label{eqn:shoSuperposition:80}
\expectation{x}
=
\inv{1 + \Abs{\sigma}^2} \frac{x_0}{\sqrt{2}} \lr{ \sigma + \sigma^\conj}
=
\frac{\sqrt{2} x_0 \Real \sigma}{1 + \Abs{\sigma}^2}.
\end{dmath}
%
Any imaginary component in \( \sigma \) will reduce the expectation, so we are constrained to picking a real value.

The derivative of
%
\begin{dmath}\label{eqn:shoSuperposition:100}
f(\sigma) = \frac{\sigma}{1 + \sigma^2},
\end{dmath}
%
is
%
\begin{dmath}\label{eqn:shoSuperposition:120}
f'(\sigma) = \frac{1 - \sigma^2}{(1 + \sigma^2)^2}.
\end{dmath}
%
That has zeros at \( \sigma = \pm 1 \).  The second derivative is
%
\begin{dmath}\label{eqn:shoSuperposition:140}
f''(\sigma) = \frac{-2 \sigma (3 - \sigma^2)}{(1 + \sigma^2)^3}.
\end{dmath}
%
That will be negative (maximum for the extreme value) at \( \sigma = 1 \), so the linear superposition of these first two energy eigenkets that maximizes the position expectation is
%
\begin{dmath}\label{eqn:shoSuperposition:160}
\psi = \inv{\sqrt{2}}\lr{ \ket{0} + \ket{1} }.
\end{dmath}
%
That maximized position expectation is
%
\begin{dmath}\label{eqn:shoSuperposition:180}
\expectation{x}
=
\frac{x_0}{\sqrt{2}}.
\end{dmath}
%
\makeSubAnswer{}{problem:shoSuperposition:1:b}
%
The time evolution is given by
%
\begin{dmath}\label{eqn:shoSuperposition:200}
\ket{\Psi(t)}
= e^{-i H t/\Hbar} \inv{\sqrt{2}}\lr{ \ket{0} + \ket{1} }
= \inv{\sqrt{2}}\lr{ e^{-i(0+ \ifrac{1}{2})\Hbar \omega t/\Hbar} \ket{0} + e^{-i(1+ \ifrac{1}{2})\Hbar \omega t/\Hbar} \ket{1} }
= \inv{\sqrt{2}}\lr{ e^{-i \omega t/2} \ket{0} + e^{-3 i \omega t/2} \ket{1} }.
\end{dmath}
%
\makeSubAnswer{}{problem:shoSuperposition:1:c}
%
The position expectation in the Schr\"{o}dinger representation is
%
\begin{dmath}\label{eqn:shoSuperposition:220}
\expectation{x(t)}
=
\inv{2}
\lr{ e^{i \omega t/2} \bra{0} + e^{3 i \omega t/2} \bra{1} } \frac{x_0}{\sqrt{2}} \lr{ a^\dagger + a }
\lr{ e^{-i \omega t/2} \ket{0} + e^{-3 i \omega t/2} \ket{1} }
=
\frac{x_0}{2\sqrt{2}}
\lr{ e^{i \omega t/2} \bra{0} + e^{3 i \omega t/2} \bra{1} }
\lr{ e^{-i \omega t/2} \ket{1} + e^{-3 i \omega t/2} \sqrt{2} \ket{2} + e^{-3 i \omega t/2} \ket{0} }
=
\frac{x_0}{\sqrt{2}} \cos(\omega t).
\end{dmath}
%
\makeSubAnswer{}{problem:shoSuperposition:1:d}
%
\begin{dmath}\label{eqn:shoSuperposition:240}
\expectation{x(t)}
=
\inv{2}
\lr{ \bra{0} + \bra{1} } \frac{x_0}{\sqrt{2}}
\lr{ a^\dagger e^{i\omega t} + a e^{-i \omega t} }
\lr{ \ket{0} + \ket{1} }
=
\frac{x_0}{2 \sqrt{2}}
\lr{ \bra{0} + \bra{1} }
\lr{ e^{i\omega t} \ket{1} + \sqrt{2} e^{i\omega t} \ket{2} + e^{-i \omega t} \ket{0} }
=
\frac{x_0}{\sqrt{2}} \cos(\omega t),
\end{dmath}
%
matching the calculation using the Schr\"{o}dinger picture.
%
\makeSubAnswer{}{problem:shoSuperposition:1:e}
%
Let's use the Heisenberg picture for the uncertainty calculation.  Using the calculation above we have
%
\begin{dmath}\label{eqn:shoSuperposition:260}
\expectation{x^2}
=
\inv{2} \frac{x_0^2}{2}
\lr{ e^{-i\omega t} \bra{1} + \sqrt{2} e^{-i\omega t} \bra{2} + e^{i \omega t} \bra{0} }
\lr{ e^{i\omega t} \ket{1} + \sqrt{2} e^{i\omega t} \ket{2} + e^{-i \omega t} \ket{0} }
=
\frac{x_0^2}{4} \lr{ 1 + 2 + 1}
=
x_0^2.
\end{dmath}
%
The uncertainty is
\begin{dmath}\label{eqn:shoSuperposition:280}
\expectation{(\Delta x)^2} =
\expectation{x^2} - \expectation{x}^2
=
x_0^2 - \frac{x_0^2}{2} \cos^2(\omega t)
=
\frac{x_0^2}{2} \lr{ 2 - \cos^2(\omega t) }
=
\frac{x_0^2}{2} \lr{ 1 + \sin^2(\omega t) }
\end{dmath}
} % answer

%\EndArticle

         % p.18:
         %
% Copyright � 2015 Peeter Joot.  All Rights Reserved.
% Licenced as described in the file LICENSE under the root directory of this GIT repository.
%
%\input{../blogpost.tex}
%\renewcommand{\basename}{exponentialExpectationGroundState}
%\renewcommand{\dirname}{notes/phy1520/}
%%\newcommand{\dateintitle}{}
%%\newcommand{\keywords}{}
%
%\input{../peeter_prologue_print2.tex}
%
%\usepackage{peeters_layout_exercise}
%\usepackage{peeters_braket}
%\usepackage{peeters_figures}
%
%\beginArtNoToc
%
%\generatetitle{Plane wave ground state expectation}
%%\chapter{Plane wave ground state expectation}
%%\label{chap:exponentialExpectationGroundState}

\makeoproblem{Plane wave ground state expectation for 1D SHO.}{problem:exponentialExpectationGroundState:1}{\citep{sakurai2014modern} pr. 2.18}{
\index{harmonic oscillator!ground state}

For a 1D SHO, show that
%
\begin{dmath}\label{eqn:exponentialExpectationGroundState:20}
\bra{0} e^{i k x} \ket{0} = \exp\lr{ -k^2 \bra{0} x^2 \ket{0}/2 }.
\end{dmath}
%
} % problem

\makeanswer{problem:exponentialExpectationGroundState:1}{
Despite the simple appearance of this problem, I found this quite involved to show.  To do so, start with a series expansion of the expectation
%
\begin{dmath}\label{eqn:exponentialExpectationGroundState:40}
\bra{0} e^{i k x} \ket{0}
=
\sum_{m=0}^\infty \frac{(i k)^m}{m!} \bra{0} x^m \ket{0}.
\end{dmath}
%
Let
%
\begin{dmath}\label{eqn:exponentialExpectationGroundState:60}
X = \lr{ a + a^\dagger },
\end{dmath}
%
so that
%
\begin{equation}\label{eqn:exponentialExpectationGroundState:80}
x
= \sqrt{\frac{\Hbar}{2 \omega m}} X
= \frac{x_0}{\sqrt{2}} X.
\end{equation}
%
Consider the first few values of \( \bra{0} X^n \ket{0} \)
%
\begin{dmath}\label{eqn:exponentialExpectationGroundState:100}
\bra{0} X \ket{0}
=
\bra{0} \lr{ a + a^\dagger } \ket{0}
=
\braket{0}{1}
=
0,
\end{dmath}
%
\begin{dmath}\label{eqn:exponentialExpectationGroundState:120}
\bra{0} X^2 \ket{0}
=
\bra{0} \lr{ a + a^\dagger }^2 \ket{0}
=
\braket{1}{1}
=
1,
\end{dmath}
%
\begin{dmath}\label{eqn:exponentialExpectationGroundState:140}
\bra{0} X^3 \ket{0}
=
\bra{0} \lr{ a + a^\dagger }^3 \ket{0}
=
\bra{1} \lr{ \sqrt{2} \ket{2} + \ket{0} }
=
0.
\end{dmath}
%
Whenever the power \( n \) in \( X^n \) is even, the braket can be split into a bra that has only contributions from odd eigenstates and a ket with even eigenstates.  We conclude that \( \bra{0} X^n \ket{0} = 0 \) when \( n \) is odd.

Noting that \( \bra{0} x^2 \ket{0} = \ifrac{x_0^2}{2} \), this leaves
%
\begin{dmath}\label{eqn:exponentialExpectationGroundState:160}
\bra{0} e^{i k x} \ket{0}
=
\sum_{m=0}^\infty \frac{(i k)^{2 m}}{(2 m)!} \bra{0} x^{2m} \ket{0}
=
\sum_{m=0}^\infty \frac{(i k)^{2 m}}{(2 m)!} \lr{ \frac{x_0^2}{2} }^m \bra{0} X^{2m} \ket{0}
=
\sum_{m=0}^\infty \frac{1}{(2 m)!} \lr{ -k^2 \bra{0} x^2 \ket{0} }^m \bra{0} X^{2m} \ket{0}.
\end{dmath}
%
This problem is now reduced to showing that
%
\begin{dmath}\label{eqn:exponentialExpectationGroundState:180}
\frac{1}{(2 m)!} \bra{0} X^{2m} \ket{0} = \inv{m! 2^m},
\end{dmath}
%
or
%
\begin{dmath}\label{eqn:exponentialExpectationGroundState:200}
\bra{0} X^{2m} \ket{0}
= \frac{(2m)!}{m! 2^m}
= \frac{ (2m)(2m-1)(2m-2) \cdots (2)(1) }{2^m m!}
= \frac{ 2^m (m)(2m-1)(m-1)(2m-3)(m-2) \cdots (2)(3)(1)(1) }{2^m m!}
= (2m-1)!!,
\end{dmath}
%
where \( n!! = n(n-2)(n-4)\cdots \).

It looks like \( \bra{0} X^{2m} \ket{0} \) can be expanded by inserting an identity operator and proceeding recursively, like
%
\begin{dmath}\label{eqn:exponentialExpectationGroundState:220}
\bra{0} X^{2m} \ket{0}
=
\bra{0} X^2 \lr{ \sum_{n=0}^\infty \ket{n}\bra{n} } X^{2m-2} \ket{0}
=
\bra{0} X^2 \lr{ \ket{0}\bra{0} + \ket{2}\bra{2} } X^{2m-2} \ket{0}
=
\bra{0} X^{2m-2} \ket{0} + \bra{0} X^2 \ket{2} \bra{2} X^{2m-2} \ket{0}.
\end{dmath}
%
This has made use of the observation that \( \bra{0} X^2 \ket{n} = 0 \) for all \( n \ne 0,2 \).  The remaining term includes the factor
%
\begin{dmath}\label{eqn:exponentialExpectationGroundState:240}
\bra{0} X^2 \ket{2}
=
\bra{0} \lr{a + a^\dagger}^2 \ket{2}
=
\lr{ \bra{0} + \sqrt{2} \bra{2} } \ket{2}
=
\sqrt{2},
\end{dmath}
%
Since \( \sqrt{2} \ket{2} = \lr{a^\dagger}^2 \ket{0} \), the expectation of interest can be written
%
\begin{dmath}\label{eqn:exponentialExpectationGroundState:260}
\bra{0} X^{2m} \ket{0}
=
\bra{0} X^{2m-2} \ket{0} + \bra{0} a^2 X^{2m-2} \ket{0}.
\end{dmath}
%
How do we expand the second term.  Let's look at how \( a \) and \( X \) commute
%
\begin{dmath}\label{eqn:exponentialExpectationGroundState:280}
a X
=
\antisymmetric{a}{X} + X a
=
\antisymmetric{a}{a + a^\dagger} + X a
=
\antisymmetric{a}{a^\dagger} + X a
=
1 + X a,
\end{dmath}
%
\begin{dmath}\label{eqn:exponentialExpectationGroundState:300}
a^2 X
=
a \lr{ a X }
=
a \lr{ 1 + X a }
=
a + a X a
=
a + \lr{ 1 + X a } a
=
2 a + X a^2.
\end{dmath}
%
Proceeding to expand \( a^2 X^n \) we find
\begin{equation}\label{eqn:exponentialExpectationGroundState:320}
\begin{aligned}
a^2 X^3 &= 6 X + 6 X^2 a + X^3 a^2 \\
a^2 X^4 &= 12 X^2 + 8 X^3 a + X^4 a^2 \\
a^2 X^5 &= 20 X^3 + 10 X^4 a + X^5 a^2 \\
a^2 X^6 &= 30 X^4 + 12 X^5 a + X^6 a^2.
\end{aligned}
\end{equation}
%
It appears that we have
\begin{equation}\label{eqn:exponentialExpectationGroundState:340}
\antisymmetric{a^2 X^n}{X^n a^2} = \beta_n X^{n-2} + 2 n X^{n-1} a,
\end{equation}
%
where
%
\begin{equation}\label{eqn:exponentialExpectationGroundState:360}
\beta_n = \beta_{n-1} + 2 (n-1),
\end{equation}
%
and \( \beta_2 = 2 \).  Some goofing around shows that \( \beta_n = n(n-1) \), so the induction hypothesis is
%
\begin{equation}\label{eqn:exponentialExpectationGroundState:380}
\antisymmetric{a^2 X^n}{X^n a^2} = n(n-1) X^{n-2} + 2 n X^{n-1} a.
\end{equation}
%
Let's check the induction
\begin{dmath}\label{eqn:exponentialExpectationGroundState:400}
a^2 X^{n+1}
=
a^2 X^{n} X
=
\lr{ n(n-1) X^{n-2} + 2 n X^{n-1} a + X^n a^2 } X
=
n(n-1) X^{n-1} + 2 n X^{n-1} a X + X^n a^2 X
=
n(n-1) X^{n-1} + 2 n X^{n-1} \lr{ 1 + X a } + X^n \lr{ 2 a + X a^2 }
=
n(n-1) X^{n-1} + 2 n X^{n-1}  + 2 n X^{n} a
+ 2 X^n a
+ X^{n+1} a^2
=
X^{n+1} a^2 + (2 + 2 n) X^{n} a + \lr{ 2 n + n(n-1) }  X^{n-1}
=
X^{n+1} a^2 + 2(n + 1) X^{n} a + (n+1) n X^{n-1},
\end{dmath}
%
which concludes the induction, giving
%
\begin{dmath}\label{eqn:exponentialExpectationGroundState:420}
\bra{ 0 } a^2 X^{n} \ket{0 } = n(n-1) \bra{0} X^{n-2} \ket{0},
\end{dmath}
%
and
%
\begin{dmath}\label{eqn:exponentialExpectationGroundState:440}
\bra{0} X^{2m} \ket{0}
=
\bra{0} X^{2m-2} \ket{0} + (2m-2)(2m-3) \bra{0} X^{2m-4} \ket{0}.
\end{dmath}
%
Let
%
\begin{dmath}\label{eqn:exponentialExpectationGroundState:460}
\sigma_{n} = \bra{0} X^n \ket{0},
\end{dmath}
%
so that the recurrence relation, for \( 2n \ge 4 \) is
%
\begin{dmath}\label{eqn:exponentialExpectationGroundState:480}
\sigma_{2n} = \sigma_{2n -2} + (2n-2)(2n-3) \sigma_{2n -4}
\end{dmath}

We want to show that this simplifies to
%
\begin{dmath}\label{eqn:exponentialExpectationGroundState:500}
\sigma_{2n} = (2n-1)!!
\end{dmath}

The first values are

\begin{subequations}
\label{eqn:exponentialExpectationGroundState:520}
\begin{equation}\label{eqn:exponentialExpectationGroundState:540}
\sigma_0 = \bra{0} X^0 \ket{0} = 1
\end{equation}
\begin{equation}\label{eqn:exponentialExpectationGroundState:560}
\sigma_2 = \bra{0} X^2 \ket{0} = 1
\end{equation}
\end{subequations}

which gives us the right result for the first term in the induction
%
\begin{dmath}\label{eqn:exponentialExpectationGroundState:580}
\sigma_4
= \sigma_2 + 2 \times 1 \times \sigma_0
= 1 + 2
= 3!!
\end{dmath}

For the general induction term, consider
%
\begin{dmath}\label{eqn:exponentialExpectationGroundState:600}
\sigma_{2n + 2}
= \sigma_{2n} + 2 n (2n - 1) \sigma_{2n -2}
= (2n-1)!! + 2n ( 2n - 1) (2n -3)!!
= (2n + 1) (2n -1)!!
= (2n + 1)!!,
\end{dmath}
%
which completes the final induction.  That was also the last thing required to complete the proof, so we are done!
} % answer

%\EndArticle

         % made up problem:
         %
% Copyright � 2015 Peeter Joot.  All Rights Reserved.
% Licenced as described in the file LICENSE under the root directory of this GIT repository.
%
%\input{../blogpost.tex}
%\renewcommand{\basename}{fluxAndMomentum}
%\renewcommand{\dirname}{notes/phy1520/}
%\newcommand{\dateintitle}{}
%\newcommand{\keywords}{}

%\input{../peeter_prologue_print2.tex}

%\usepackage{peeters_layout_exercise}
%\usepackage{peeters_braket}
%\usepackage{peeters_figures}
%
%\beginArtNoToc

%\generatetitle{Relation of probability flux to momentum}
%\chapter{Relation of probability flux to momentum}
%\label{chap:fluxAndMomentum}

\makeproblem{Relation of probability flux to momentum.}{problem:fluxAndMomentum:1}{
\index{probability!flux}
\index{momentum!expectation}

Show that the probability flux
%
\begin{dmath}\label{eqn:fluxAndMomentum:20}
\Bj(\Bx, t) = -\frac{i\Hbar}{2 m} \lr{ \psi^\conj \spacegrad \psi - \psi \spacegrad \psi^\conj },
\end{dmath}
%
is related to the momentum expectation at a given time by the integral of the flux over all space
%
\begin{dmath}\label{eqn:fluxAndMomentum:40}
\int d^3 x \Bj(\Bx, t) = \frac{\expectation{\Bp}_t}{m}.
\end{dmath}
%
} % problem

\makeanswer{problem:fluxAndMomentum:1}{

This can be seen by recasting the integral in bra-ket form.  Let
%
\begin{dmath}\label{eqn:fluxAndMomentum:60}
\psi(\Bx, t) = \braket{\Bx}{\psi(t)},
\end{dmath}
%
and note that the momentum portions of the flux can be written as
%
\begin{dmath}\label{eqn:fluxAndMomentum:80}
-i \Hbar \spacegrad \psi(\Bx, t) = \bra{\Bx} \Bp \ket{\psi(t)}.
\end{dmath}
%
The current is therefore
%
\begin{dmath}\label{eqn:fluxAndMomentum:100}
\Bj(\Bx, t)
= \frac{1}{2 m}
\lr{
\psi^\conj \bra{\Bx} \Bp \ket{\psi(t)}
+\psi {\bra{\Bx} \Bp \ket{\psi(t)} }^\conj
}
= \frac{1}{2 m}
\lr{
{\braket{\Bx}{\psi(t)}}^\conj \bra{\Bx} \Bp \ket{\psi(t)}
+ \braket{\Bx}{\psi(t)} {\bra{\Bx} \Bp \ket{\psi(t)} }^\conj
}
= \frac{1}{2 m}
\lr{
\braket{\psi(t)}{\Bx} \bra{\Bx} \Bp \ket{\psi(t)}
+
\bra{\psi(t)} \Bp \ket{\Bx} \braket{\Bx}{\psi(t)}
}.
\end{dmath}
%
Integrating and noting that the spatial identity is \( 1 = \int d^3 x \ket{\Bx}\bra{\Bx} \), we have
%
\begin{dmath}\label{eqn:fluxAndMomentum:n}
\int d^3 x \Bj(\Bx, t)
=
\bra{\psi(t)} \Bp \ket{\psi(t)},
\end{dmath}
%
This is just the expectation of \( \Bp \) with respect to a specific time-instance state, demonstrating the desired relationship.
} % answer

%\EndArticle

         % p21:
         %
% Copyright � 2015 Peeter Joot.  All Rights Reserved.
% Licenced as described in the file LICENSE under the root directory of this GIT repository.
%
%\input{../blogpost.tex}
%\renewcommand{\basename}{hermiteOrtho}
%\renewcommand{\dirname}{notes/phy1520/}
%%\newcommand{\dateintitle}{}
%%\newcommand{\keywords}{}
%
%\input{../peeter_prologue_print2.tex}
%
%\usepackage{peeters_layout_exercise}
%\usepackage{peeters_braket}
%\usepackage{peeters_figures}
%
%\beginArtNoToc
%
%\generatetitle{Hermite polynomial normalization constant}
%\chapter{Hermite polynomial normalization constant}
%\label{chap:hermiteOrtho}

\makeoproblem{Hermite polynomial normalization constant.}{problem:hermiteOrtho:2.21}{\citep{sakurai2014modern} pr. 2.21}{
\index{Hermite polynomial}
Derive the normalization constant \( c_n \) for the Harmonic oscillator solution
%
\begin{dmath}\label{eqn:hermiteOrtho:20}
u_n(x) = c_n H_n\lr{ x \sqrt{\frac{m\omega}{\Hbar}} } e^{-m \omega x^2/2 \Hbar},
\end{dmath}
%
by deriving the orthogonality relationship using generating functions
%
\begin{equation}\label{eqn:hermiteOrtho:40}
g(x,t) = e^{-t^2 + 2 t x} = \sum_{n=0}^\infty H_n(x) \frac{t^n}{n!}.
\end{equation}
%
Start by working out the integral
%
\begin{equation}\label{eqn:hermiteOrtho:60}
I = \int_{-\infty}^\infty g(x, t) g(x, s) e^{-x^2} dx,
\end{equation}
%
consider the integral twice with each side definition of the generating function.

} % problem

\makeanswer{problem:hermiteOrtho:2.21}{

First using the exponential definition of the generating function
%
\begin{dmath}\label{eqn:hermiteOrtho:80}
\int_{-\infty}^\infty g(x, t) g(x, s) e^{-x^2} dx
=
\int_{-\infty}^\infty
e^{-t^2 + 2 t x}
e^{-s^2 + 2 s x} e^{-x^2} dx
=
e^{-t^2 -s^2}
\int_{-\infty}^\infty
e^{-(x^2- 2 t x - 2 s x)} dx
=
e^{-t^2 -s^2 + (s + t)^2}
\int_{-\infty}^\infty
e^{-(x - t - s)^2} dx
=
e^{2 st}
\int_{-\infty}^\infty
e^{-u^2} du
= \sqrt{\pi} e^{2 st}.
\end{dmath}

With the Hermite polynomial definition of the generating function, this integral is
%
\begin{dmath}\label{eqn:hermiteOrtho:100}
\int_{-\infty}^\infty g(x, t) g(x, s) e^{-x^2} dx
=
\int_{-\infty}^\infty
\sum_{n=0}^\infty H_n(x) \frac{t^n}{n!}
\sum_{m=0}^\infty H_m(x) \frac{s^m}{m!}
e^{-x^2} dx
=
\sum_{n=0}^\infty \frac{t^n}{n!}
\sum_{m=0}^\infty \frac{s^m}{m!}
\int_{-\infty}^\infty H_n(x) H_m(x) e^{-x^2} dx.
\end{dmath}

Let
%
\begin{dmath}\label{eqn:hermiteOrtho:120}
\alpha_{n m} = \int_{-\infty}^\infty H_n(x) H_m(x) e^{-x^2} dx,
\end{dmath}
%
and equate the two expansions of this integral
%
\begin{dmath}\label{eqn:hermiteOrtho:140}
\sqrt{\pi} \sum_{n=0}^\infty \frac{(2st)^n}{n!}
=
\sum_{n=0}^\infty \frac{t^n}{n!}
\sum_{m=0}^\infty \frac{s^m}{m!}
\alpha_{n m},
\end{dmath}
%
or, after equating powers of \( t^n \)
%
\begin{dmath}\label{eqn:hermiteOrtho:160}
\sqrt{\pi} (2 s)^n =
\sum_{m=0}^\infty \frac{s^m}{m!} \alpha_{n m}.
\end{dmath}

This requires \( \alpha_{n m} \) to be zero for \( n \ne m \), so
%
\begin{dmath}\label{eqn:hermiteOrtho:180}
\sqrt{\pi} 2^n = \frac{1}{n!} \alpha_{n n},
\end{dmath}
%
and
%
\begin{dmath}\label{eqn:hermiteOrtho:200}
\int_{-\infty}^\infty H_n(x) H_m(x) e^{-x^2} dx = \delta_{n m} \sqrt{\pi} 2^n n!.
\end{dmath}

The SHO normalization is fixed by
%
\begin{dmath}\label{eqn:hermiteOrtho:220}
\int_{-\infty}^\infty u_n^2(x) dx
= c_n^2
\int_{-\infty}^\infty H_n^2(x/x_0) e^{-(x/x_0)^2} dx
= c_n^2 x_0 \sqrt{\pi} 2^n n!,
\end{dmath}
%
or
%
\begin{dmath}\label{eqn:hermiteOrtho:240}
c_n
= \inv{\sqrt{ \sqrt{\pi} 2^n n! \sqrt{\frac{\Hbar}{m \omega}}}}
= \lr{ \frac{m \omega}{\Hbar \pi} }^{1/4} 2^{-n/2} \inv{\sqrt{n!}}
\end{dmath}

} % answer

%\EndArticle

         % p2.24,2.25
         %
% Copyright � 2015 Peeter Joot.  All Rights Reserved.
% Licenced as described in the file LICENSE under the root directory of this GIT repository.
%
%\input{../blogpost.tex}
%\renewcommand{\basename}{diracPotential}
%\renewcommand{\dirname}{notes/phy1520/}
%%\newcommand{\dateintitle}{}
%%\newcommand{\keywords}{}
%
%\input{../peeter_prologue_print2.tex}
%
%\usepackage{peeters_layout_exercise}
%\usepackage{peeters_braket}
%\usepackage{peeters_figures}
%
%\beginArtNoToc
%
%\generatetitle{Dirac delta function potential}
%%\chapter{Dirac delta function potential}
%%\label{chap:diracPotential}
%
\makeoproblem{Dirac delta function potential.}{problem:diracPotential:1}{\citep{sakurai2014modern} pr. 2.24,2.25}{
%
Given a Dirac delta function potential
%
\begin{equation}\label{eqn:diracPotential:20}
H = \frac{p^2}{2m} - V_0 \delta(x),
\end{equation}
%
which vanishes after \( t = 0 \).
%
\makesubproblem{}{problem:diracPotential:1:a}
Solve for the bound state for \( t < 0 \),
\makesubproblem{}{problem:diracPotential:1:b}
Solve for the time evolution after that.
} % problem
%
\makeanswer{problem:diracPotential:1}{
%
\makeSubAnswer{}{problem:diracPotential:1:a}
%
%The first part of this problem was assigned back in phy356, where we solved this for a rectangular potential that had the limiting form of a delta function potential.  However,
This problem can be solved directly by considering the \( \Abs{x} > 0 \) and \( x = 0 \) regions separately.

For \( \Abs{x} > 0 \) Schr\"{o}dinger's equation takes the form
%
\begin{equation}\label{eqn:diracPotential:40}
E \psi = -\frac{\Hbar^2}{2m} \frac{d^2 \psi}{dx^2}.
\end{equation}
%
With
%
\begin{equation}\label{eqn:diracPotential:60}
\kappa =
\frac{\sqrt{-2 m E}}{\Hbar},
\end{equation}
%
this has solutions
%
\begin{equation}\label{eqn:diracPotential:80}
\psi = e^{\pm \kappa x}.
\end{equation}
%
For \( x > 0 \) we must have
\begin{equation}\label{eqn:diracPotential:100}
\psi = a e^{-\kappa x},
\end{equation}
%
and for \( x < 0 \)
\begin{equation}\label{eqn:diracPotential:120}
\psi = b e^{\kappa x}.
\end{equation}
%
requiring that \( \psi \) is continuous at \( x = 0 \) means \( a = b \), or
%
\begin{equation}\label{eqn:diracPotential:140}
\psi = \psi(0) e^{-\kappa \Abs{x}}.
\end{equation}
%
For the \( x = 0 \) region, consider an interval \( [-\epsilon, \epsilon] \) region around the origin.  We must have
%
\begin{equation}\label{eqn:diracPotential:160}
E \int_{-\epsilon}^\epsilon \psi(x) dx = \frac{-\Hbar^2}{2m} \int_{-\epsilon}^\epsilon \frac{d^2 \psi}{dx^2} dx - V_0 \int_{-\epsilon}^\epsilon \delta(x) \psi(x) dx.
\end{equation}
%
The RHS is zero
%
\begin{equation}\label{eqn:diracPotential:180}
\begin{aligned}
E \int_{-\epsilon}^\epsilon \psi(x) dx
&=
E \frac{ e^{-\kappa (\epsilon)} - 1}{-\kappa}
-E \frac{ 1 - e^{\kappa (-\epsilon)}}{\kappa}
\\ &\rightarrow
0.
\end{aligned}
\end{equation}
%
That leaves
\begin{equation}\label{eqn:diracPotential:200}
\begin{aligned}
V_0 \int_{-\epsilon}^\epsilon \delta(x) \psi(x) dx
&=
\frac{-\Hbar^2}{2m} \int_{-\epsilon}^\epsilon \frac{d^2 \psi}{dx^2} dx
\\ &=
\frac{-\Hbar^2}{2m} \evalrange{\frac{d \psi}{dx}}{-\epsilon}{\epsilon}
\\ &=
\frac{-\Hbar^2}{2m}
\psi(0)
\lr
{
-\kappa e^{-\kappa (\epsilon)}
-
\kappa e^{\kappa (-\epsilon)}
}.
\end{aligned}
\end{equation}
%
In the \( \epsilon \rightarrow 0 \) limit this gives
%
\begin{equation}\label{eqn:diracPotential:220}
V_0 = \frac{\Hbar^2 \kappa}{m}.
\end{equation}
%
Equating relations for \( \kappa \) we have
%
\begin{equation}\label{eqn:diracPotential:240}
\kappa = \frac{m V_0}{\Hbar^2} = \frac{\sqrt{-2 m E}}{\Hbar},
\end{equation}
%
or
%
\begin{equation}\label{eqn:diracPotential:260}
E = -\inv{2 m} \lr{ \frac{m V_0}{\Hbar} }^2,
\end{equation}
%
with
%
\begin{equation}\label{eqn:diracPotential:280}
\psi(x, t < 0) = C \exp\lr{ -i E t/\hbar - \kappa \Abs{x}}.
\end{equation}
%
The normalization requires
%
\begin{equation}\label{eqn:diracPotential:300}
\begin{aligned}
1
&= 2 \Abs{C}^2 \int_0^\infty e^{- 2 \kappa x} dx
\\ &= 2 \Abs{C}^2 \evalrange{\frac{e^{- 2 \kappa x}}{-2 \kappa}}{0}{\infty}
\\ &= \frac{\Abs{C}^2}{\kappa},
\end{aligned}
\end{equation}
%
so
%\begin{equation}\label{eqn:diracPotential:320}
\boxedEquation{eqn:diracPotential:340}{
\psi(x, t < 0) = \sqrt{\kappa} \exp\lr{ -i E t/\hbar - \kappa \Abs{x}}.
}
%\end{equation}
%
There is only one bound state for such a potential.
%
\makeSubAnswer{}{problem:diracPotential:1:b}
%
After turning off the potential, any plane wave
%
\begin{equation}\label{eqn:diracPotential:360}
\psi(x, t) = e^{i k x - i E(k) t/\Hbar},
\end{equation}
%
where
%
\begin{equation}\label{eqn:diracPotential:380}
k = \frac{\sqrt{2 m E}}{\Hbar},
\end{equation}
%
is a solution.  In particular, at \( t = 0 \), the wave packet
%
\begin{equation}\label{eqn:diracPotential:400}
\psi(x,0) = \inv{\sqrt{2\pi}} \int_{-\infty}^\infty e^{i k x} A(k) dk,
\end{equation}
%
is a solution.  To solve for \( A(k) \), we require
%
\begin{equation}\label{eqn:diracPotential:420}
\inv{\sqrt{2\pi}} \int_{-\infty}^\infty e^{i k x} A(k) dk
= \sqrt{\kappa} e^{ - \kappa \Abs{x} },
\end{equation}
%
or
%\begin{dmath}\label{eqn:diracPotential:440}
\boxedEquation{eqn:diracPotential:460}{
A(k) =
\sqrt{\frac{\kappa}{2\pi}} \int_{-\infty}^\infty e^{-i k x}
e^{ - \kappa \Abs{x} }
dx
=
\sqrt{\frac{2}{\pi } }
\frac{
 \kappa^{3/2}}{\kappa^2+k^2}.
}
%\end{dmath}
%
The initial time state established by the delta function potential evolves as
%\begin{equation}\label{eqn:diracPotential:480}
\boxedEquation{eqn:diracPotential:500}{
\psi(x, t > 0) =
\inv{\sqrt{2\pi}} \int_{-\infty}^\infty e^{i k x - i \Hbar k^2 t/2m} A(k) dk.
}
%\end{equation}
%
In terms of \( m, V_0 \) that is
%
\begin{equation}\label{eqn:diracPotential:700}
\psi(x, t > 0) =
\frac{
 \lr{m V_0}^{3/2}
}{\pi \Hbar}
\int_{-\infty}^\infty
\frac{e^{i k x - i \Hbar k^2 t/2m}
}{k^2 \Hbar^2 + \ifrac{m^2 V_0^2}{\Hbar^2}}
dk
.
\end{equation}
%
This integral resists an attempt to evaluate with Mathematica.
} % answer

%\EndArticle

         % p31
         %
% Copyright � 2015 Peeter Joot.  All Rights Reserved.
% Licenced as described in the file LICENSE under the root directory of this GIT repository.
%
%\input{../blogpost.tex}
%\renewcommand{\basename}{freeParticlePropagator}
%\renewcommand{\dirname}{notes/phy1520/}
%%\newcommand{\dateintitle}{}
%%\newcommand{\keywords}{}
%
%\input{../peeter_prologue_print2.tex}
%
%\usepackage{peeters_layout_exercise}
%\usepackage{peeters_braket}
%\usepackage{peeters_figures}
%
%\beginArtNoToc
%
%\generatetitle{Free particle propagator}
%%\chapter{Free particle propagator}
%\label{chap:freeParticlePropagator}
%
\makeoproblem{Free particle propagator.}{problem:freeParticlePropagator:31}{\citep{sakurai2014modern} pr. 2.31}{
\index{propagator!free particle}
Derive the free particle propagator in one and three dimensions.
} % problem
%
\makeanswer{problem:freeParticlePropagator:31}{
%
I found the description in the text confusing, so let's start from scratch with the definition of the propagator.  This is the kernel of the spatial convolution integral that encodes time evolution, and can be expressed by expanding a general time state with two sets of identity operators.  Let the position relative state at time \( t \), relative to an initial time \( t_0 \) be given by \( \braket{\Bx}{\alpha, t ; t_0 } \), and expand this in terms of a complete basis of energy eigenstates \( \ket{a'} \) and the time evolution operator
%
\begin{dmath}\label{eqn:freeParticlePropagator:20}
\begin{aligned}
\braket{\Bx''}{\alpha, t ; t_0 }
&= \bra{\Bx''} U \ket{\alpha, t_0 } \\
&= \bra{\Bx''} e^{-i H (t -t_0)/\Hbar} \ket{\alpha, t_0 } \\
&= \bra{\Bx''} e^{-i H (t -t_0)/\Hbar} \lr{ \sum_{a'} \ket{a'} \bra{a' }} \ket{\alpha, t_0 } \\
&= \bra{\Bx''} \sum_{a'} e^{-i E_{a'} (t -t_0)/\Hbar} \ket{a'} \braket{a' }{\alpha, t_0 } \\
&=
\bra{\Bx''} \sum_{a'} e^{-i E_{a'} (t -t_0)/\Hbar} \ket{a'} \bra{a' }
\lr{ \int d^3 \Bx'
\ket{\Bx'}\bra{\Bx'}
}
\ket{\alpha, t_0 } \\
&=
\int d^3 \Bx'
\lr{
\bra{\Bx''} \sum_{a'} e^{-i E_{a'} (t -t_0)/\Hbar} \ket{a'} \braket{a' }{\Bx'}
}
\braket{\Bx'}{\alpha, t_0 } \\
&=
\int d^3 \Bx' K(\Bx'', t ; \Bx', t_0) \braket{\Bx'}{\alpha, t_0 },
\end{aligned}
\end{dmath}
where
%
\begin{dmath}\label{eqn:freeParticlePropagator:40}
K(\Bx'', t ; \Bx', t_0) =
\sum_{a'}
\braket{\Bx''}{a'}\braket{a' }{\Bx'}
e^{-i E_{a'} (t -t_0)/\Hbar},
\end{dmath}
%
the propagator, is the kernel of the convolution integral that takes the state \( \ket{\alpha, t_0} \) to state \( \ket{\alpha, t ; t_0} \).  Evaluating this over the momentum states (where integration and not plain summation is required), we have
%
\begin{dmath}\label{eqn:freeParticlePropagator:60}
\begin{aligned}
K(\Bx'', t ; \Bx', t_0)
&=
\int d^3 \Bp'
\braket{\Bx''}{\Bp'}\braket{\Bp' }{\Bx'}
e^{-i E_{\Bp'} (t -t_0)/\Hbar} \\
&=
\int d^3 \Bp'
\braket{\Bx''}{\Bp'}\braket{\Bp' }{\Bx'}
\exp\lr{-i \frac{(\Bp')^2 (t -t_0)}{2 m \Hbar}} \\
&=
\int d^3 \Bp'
\frac{e^{i \Bx'' \cdot \Bp'/\Hbar}}{(\sqrt{2 \pi \Hbar})^3}
\frac{e^{-i \Bx' \cdot \Bp'/\Hbar}}{(\sqrt{2 \pi \Hbar})^3}
\exp\lr{-i \frac{(\Bp')^2 (t -t_0)}{2 m \Hbar}} \\
&=
\inv{(2 \pi \Hbar)^3}
\int d^3 \Bp'
e^{i (\Bx'' -\Bx') \cdot \Bp'/\Hbar}
\exp\lr{-i \frac{(\Bp')^2 (t -t_0)}{2 m \Hbar}} \\
&=
\inv{ 2 \pi \Hbar }
\int_{-\infty}^\infty dp_1'
e^{i (x_1'' -x_1') p_1'/\Hbar}
\exp\lr{-i \frac{(p_1')^2 (t -t_0)}{2 m \Hbar}} \times \\
&\quad \inv{ 2 \pi \Hbar }
\int_{-\infty}^\infty dp_2'
e^{i (x_2'' -x_2') p_2'/\Hbar}
\exp\lr{-i \frac{(p_2')^2 (t -t_0)}{2 m \Hbar}} \times \\
&\quad \inv{ 2 \pi \Hbar }
\int_{-\infty}^\infty dp_3'
e^{i (x_3'' -x_3') p_3'/\Hbar}
\exp\lr{-i \frac{(p_3')^2 (t -t_0)}{2 m \Hbar}}.
\end{aligned}
\end{dmath}
With \( a = \ifrac{(t -t_0)}{2 m \Hbar} \), each of these three integral factors is of the form
%
\begin{equation}\label{eqn:freeParticlePropagator:80}
\begin{aligned}
\inv{ 2 \pi \Hbar } &
\int_{-\infty}^\infty dp
e^{i \Delta x p/\Hbar }
\exp\lr{-i a p^2} \\
&=
\inv{2 \pi \Hbar \sqrt{a}}
\int_{-\infty}^\infty du
e^{i \Delta x u/(\sqrt{a}\Hbar) }
\exp\lr{-i u^2} \\
&=
\inv{2 \pi \Hbar \sqrt{a}}
\int_{-\infty}^\infty du
e^{i \Delta x u/(\sqrt{a} \Hbar) }
\exp\lr{-i (u - \Delta x /(2\sqrt{a}\Hbar))^2 + i(\Delta x/(2\sqrt{a}\Hbar))^2} \\
&=
\inv{2 \pi \Hbar \sqrt{a}}
\exp\lr{ \frac{i(\Delta x)^2 2 m \Hbar}{4 (t -t_0) \Hbar^2} }
\int_{-\infty}^\infty dz
e^{-i z^2} \\
&= \sqrt{ \frac{ -i \pi 2 m \Hbar}{ 4 \pi^2 \Hbar^2 (t -t_0)} }
\exp\lr{ \frac{i(\Delta x)^2 m}{2 (t -t_0) \Hbar} } \\
&= \sqrt{ \frac{ m }{ 2 \pi i \Hbar (t -t_0)} }
\exp\lr{ \frac{i(\Delta x)^2 m}{2 (t -t_0) \Hbar} }.
\end{aligned}
\end{equation}
%
Note that the integral above has value \( \sqrt{-i\pi} \) which can be found by integrating over the contour of \cref{fig:contour:contourFig1}, letting \( R \rightarrow \infty \).

\imageFigure{../figures/phy1520-quantum/contourFig1}{Integration contour for \( \int e^{-i z^2} \).}{fig:contour:contourFig1}{0.2}

Multiplying out each of the spatial direction factors gives the propagator in its closed form
\boxedEquation{eqn:freeParticlePropagator:120}{
K(\Bx'', t ; \Bx', t_0)
= \lr{ \sqrt{ \frac{ m }{ 2 \pi i \Hbar (t -t_0)} } }^3
\exp\lr{ \frac{i(\Bx'' - \Bx')^2 m}{2 (t -t_0) \Hbar} }.
}
%\begin{equation}\label{eqn:freeParticlePropagator:120}
%\boxed{
%}
%\end{equation}

In one or two dimensions the exponential power \( 3 \) need only be adjusted appropriately.
} % answer

         % p32
         %
% Copyright � 2015 Peeter Joot.  All Rights Reserved.
% Licenced as described in the file LICENSE under the root directory of this GIT repository.
%
%\input{../blogpost.tex}
%\renewcommand{\basename}{partitionFunction}
%\renewcommand{\dirname}{notes/phy1520/}
%%\newcommand{\dateintitle}{}
%%\newcommand{\keywords}{}
%
%\input{../peeter_prologue_print2.tex}
%
%\usepackage{peeters_layout_exercise}
%\usepackage{peeters_braket}
%\usepackage{peeters_figures}
%
%\beginArtNoToc
%
%\generatetitle{Partition function and ground state energy}
%\chapter{Partition function and ground state energy}
%\label{chap:partitionFunction}
%
\makeoproblem{Partition function, ground state energy.}{problem:partitionFunction:32}{\citep{sakurai2014modern} pr. 2.32}{
\index{partition function}
\index{ground state}

Define the partition function as
%
\begin{equation}\label{eqn:partitionFunction:20}
Z = \int d^3 x' \evalbar{ K( \Bx', t ; \Bx', 0 ) }{\beta = i t/\Hbar},
\end{dmath}
%
Show that the ground state energy is given by
%
\begin{equation}\label{eqn:partitionFunction:40}
-\inv{Z} \PD{\beta}{Z}, \qquad \beta \rightarrow \infty.
\end{dmath}
%
} % problem
%
\makeanswer{problem:partitionFunction:32}{
%
The propagator evaluated at the same point is
%
\begin{dmath}\label{eqn:partitionFunction:60}
K( \Bx', t ; \Bx', 0 )
=
\sum_{a'} \braket{\Bx'}{a'} \ket{a'}{\Bx'} \exp\lr{ -\frac{i E_{a'} t}{\Hbar}}
=
\sum_{a'} \Abs{\braket{\Bx'}{a'}}^2 \exp\lr{ -\frac{i E_{a'} t}{\Hbar}}
=
\sum_{a'} \Abs{\braket{\Bx'}{a'}}^2 \exp\lr{ -E_{a'} \beta}.
\end{dmath}
%
The derivative is
\begin{dmath}\label{eqn:partitionFunction:80}
\PD{\beta}{Z}
=
-\int d^3 x' \sum_{a'} E_{a'} \Abs{\braket{\Bx'}{a'}}^2 \exp\lr{ -E_{a'} \beta}.
\end{dmath}
%
In the \( \beta \rightarrow \infty \) this sum will be dominated by the term with the lowest value of \( E_{a'} \).  Suppose that state is \( a' = 0 \), then
%
\begin{dmath}\label{eqn:partitionFunction:100}
\lim_{ \beta \rightarrow \infty }
-\inv{Z} \PD{\beta}{Z}
= \frac{
\int d^3 x' E_{0} \Abs{\braket{\Bx'}{0}}^2 \exp\lr{ -E_{0} \beta}
}
{
\int d^3 x' \Abs{\braket{\Bx'}{0}}^2 \exp\lr{ -E_{0} \beta}
}
= E_0.
\end{dmath}
%
This stat mech like result seems very striking and profound, and makes me want to go off and study the QM formulation of stat mech that I recall seeing in \citep{pathriastatistical}, but not covered back in phy452.
} % answer

%\EndArticle

         % p33
         %
% Copyright © 2015 Peeter Joot.  All Rights Reserved.
% Licenced as described in the file LICENSE under the root directory of this GIT repository.
%
\makeoproblem{Momentum space free propagator.}{problem:freeParticlePropagator:33}{\citep{sakurai2014modern} pr. 2.33}{
\index{free particle propagator!momentum space}
Derive the free particle propagator in momentum space.
} % problem
%
\makeanswer{problem:freeParticlePropagator:33}{
%
The momentum space propagator follows in the same fashion as the spatial propagator
%
\begin{equation}\label{eqn:freeParticlePropagator:140}
\begin{aligned}
\braket{\Bp''}{\alpha, t ; t_0 }
&= \bra{\Bp''} U \ket{\alpha, t_0 } \\
&= \bra{\Bp''} e^{-i H (t -t_0)/\Hbar} \ket{\alpha, t_0 } \\
&= \bra{\Bp''} e^{-i H (t -t_0)/\Hbar} \lr{ \sum_{a'} \ket{a'} \bra{a' }} \ket{\alpha, t_0 } \\
&= \bra{\Bp''} \sum_{a'} e^{-i E_{a'} (t -t_0)/\Hbar} \ket{a'} \braket{a' }{\alpha, t_0 } \\
&=
\bra{\Bp''} \sum_{a'} e^{-i E_{a'} (t -t_0)/\Hbar} \ket{a'} \bra{a' }
\lr{ \int d^3 \Bp'
\ket{\Bp'}\bra{\Bp'}
}
\ket{\alpha, t_0 } \\
&=
\int d^3 \Bp'
\lr{
\bra{\Bp''} \sum_{a'} e^{-i E_{a'} (t -t_0)/\Hbar} \ket{a'} \braket{a' }{\Bp'}
}
\braket{\Bp'}{\alpha, t_0 } \\
&=
\int d^3 \Bp' K(\Bp'', t ; \Bp', t_0) \braket{\Bp'}{\alpha, t_0 },
\end{aligned}
\end{equation}
so
%
\begin{dmath}\label{eqn:freeParticlePropagator:160}
K(\Bp'', t ; \Bp', t_0)
=
\sum_{a'}
\braket{\Bp''}{a'}
\braket{a' }{\Bp'}
e^{-i E_{a'} (t -t_0)/\Hbar}.
\end{dmath}
%
For the free particle Hamiltonian, this can be evaluated over a momentum space basis
%
\begin{dmath}\label{eqn:freeParticlePropagator:170}
K(\Bp'', t ; \Bp', t_0)
=
\int d^3 \Bp'''
\braket{\Bp''}{\Bp'''}
\braket{\Bp''' }{\Bp'}
e^{-i E_{\Bp'''} (t -t_0)/\Hbar}
=
\int d^3 \Bp'''
\braket{\Bp''}{\Bp'''}
\delta(\Bp''' - \Bp')
\exp\lr{ -i \frac{(\Bp''')^2 (t -t_0)}{2 m \Hbar}}
=
\braket{\Bp''}{\Bp'}
\exp\lr{ -i \frac{(\Bp')^2 (t -t_0)}{2 m \Hbar}},
\end{dmath}
or
%
%\begin{dmath}\label{eqn:freeParticlePropagator:200}
\boxedEquation{eqn:freeParticlePropagator:200}{
K(\Bp'', t ; \Bp', t_0)
=
\delta( \Bp'' - \Bp' )
\exp\lr{ -i \frac{(\Bp')^2 (t -t_0)}{2 m \Hbar}}.
}
%\end{dmath}
%
This is what we expect since the time evolution is given by just this exponential factor
%
\begin{equation}\label{eqn:freeParticlePropagator:220}
\begin{aligned}
\braket{\Bp'}{\alpha, t_0 ; t}
&= \bra{\Bp'} \exp\lr{ -i \frac{(\Bp')^2 (t -t_0)}{2 m \Hbar}} \ket{\alpha, t_0} \\
&=
\exp\lr{ -i \frac{(\Bp')^2 (t -t_0)}{2 m \Hbar}}
\braket{\Bp'}
{\alpha, t_0}.
\end{aligned}
\end{equation}
%
} % answer

%\EndArticle

         % includes p37(a):
         %
% Copyright � 2015 Peeter Joot.  All Rights Reserved.
% Licenced as described in the file LICENSE under the root directory of this GIT repository.
%
%\input{../blogpost.tex}
%\renewcommand{\basename}{gaugeTx}
%\renewcommand{\dirname}{notes/phy1520/}
%%\newcommand{\dateintitle}{}
%%\newcommand{\keywords}{}
%
%\input{../peeter_prologue_print2.tex}
%
%\usepackage{peeters_layout_exercise}
%\usepackage{peeters_braket}
%\usepackage{peeters_figures}
%
%\beginArtNoToc
%
%\generatetitle{Gauge transformation of free particle Hamiltonian}
%\chapter{Gauge transformation}
%\label{chap:gaugeTx}
%
\makeoproblem{Gauge transformation, free Hamiltonian.}{problem:gaugeTx:0}{\citep{sakurai2014modern} pr. 37(a)}{
\index{gauge transformation}
\index{free particle}
Given a gauge transformation of the free particle Hamiltonian to
%
\begin{equation}\label{eqn:gaugeTx:20}
H = \inv{2 m} \BPi \cdot \BPi + e \phi,
\end{equation}
%
where
%
\begin{equation}\label{eqn:gaugeTx:40}
\BPi = \Bp - \frac{e}{c} \BA,
\end{equation}
%
calculate
\( m d\Bx/dt \),
\( \antisymmetric{\Pi_i}{\Pi_j} \), and
\( m d^2\Bx/dt^2 \), where
\( \Bx \) is the Heisenberg picture position operator, and the fields are functions only of position \( \phi = \phi(\Bx), \BA = \BA(\Bx) \).
} % problem
%
\makeanswer{problem:gaugeTx:0}{
The final results for these calculations are found in \citep{sakurai2014modern}, but seem worth deriving to exercise our commutator muscles.

\paragraph{Heisenberg picture velocity operator}

The first order of business is the Heisenberg picture velocity operator, but first note
%
\begin{equation}\label{eqn:gaugeTx:60}
\begin{aligned}
\BPi \cdot \BPi
&= \lr{ \Bp - \frac{e}{c} \BA} \cdot \lr{ \Bp - \frac{e}{c} \BA}
\\ &= \Bp^2 - \frac{e}{c} \lr{ \BA \cdot \Bp + \Bp \cdot \BA } + \frac{e^2}{c^2} \BA^2.
\end{aligned}
\end{equation}
%
The time evolution of the Heisenberg picture position operator is therefore
%
\begin{equation}\label{eqn:gaugeTx:80}
\begin{aligned}
\ddt{\Bx}
&= \inv{i\Hbar} \antisymmetric{\Bx}{H}
\\ &= \inv{i\Hbar 2 m} \antisymmetric{\Bx}{\BPi^2}
\\ &= \inv{i\Hbar 2 m} \antisymmetric{\Bx}{\Bp^2 - \frac{e}{c} \lr{ \BA \cdot \Bp + \Bp \cdot \BA } + \frac{e^2}{c^2} \BA^2 }
\\ &= \inv{i\Hbar 2 m}
\lr{
\antisymmetric{\Bx}{\Bp^2}
- \frac{e}{c} \antisymmetric{\Bx}{ \BA \cdot \Bp + \Bp \cdot \BA }
}
.
\end{aligned}
\end{equation}
%
For the \( \Bp^2 \) commutator we have
%
\begin{equation}\label{eqn:gaugeTx:100}
\begin{aligned}
\antisymmetric{x_r}{\Bp^2}
&=
i \Hbar \PD{p_r}{\Bp^2}
\\ &=
2 i \Hbar p_r,
\end{aligned}
\end{equation}
%
or
\begin{equation}\label{eqn:gaugeTx:120}
\antisymmetric{\Bx}{\Bp^2}
=
2 i \Hbar \Bp.
\end{equation}
%
Computing the remaining commutator, we've got
%
\begin{equation}\label{eqn:gaugeTx:140}
\begin{aligned}
\antisymmetric{x_r}{\Bp \cdot \BA + \BA \cdot \Bp}
&= x_r p_s A_s - p_s A_s x_r \\
&\quad+ x_r A_s p_s - A_s p_s x_r \\
&= \lr{ \antisymmetric{x_r}{p_s} + p_s x_r } A_s - p_s A_s x_r \\
&\quad+ x_r A_s p_s - A_s \lr{ \antisymmetric{p_s}{x_r} + x_r p_s } \\
&= \antisymmetric{x_r}{p_s} A_s + \cancel{p_s A_s x_r   - p_s A_s x_r} \\
&\quad+ \cancel{x_r A_s p_s - x_r A_s p_s} + A_s \antisymmetric{x_r}{p_s} \\
&= 2 i \Hbar \delta_{r s} A_s \\
&= 2 i \Hbar A_r,
\end{aligned}
\end{equation}

so
%
\begin{equation}\label{eqn:gaugeTx:160}
\antisymmetric{\Bx}{\Bp \cdot \BA + \BA \cdot \Bp} = 2 i \Hbar \BA.
\end{equation}
%
Assembling these results gives
%
%\begin{equation}\label{eqn:gaugeTx:180}
\boxedEquation{eqn:gaugeTx:180}{
\ddt{\Bx} = \inv{m} \lr{ \Bp - \frac{e}{c} \BA } = \inv{m} \BPi,
}
%\end{equation}
as asserted in the text.
\paragraph{Kinetic Momentum commutators}
%
\begin{equation}\label{eqn:gaugeTx:200}
\begin{aligned}
\antisymmetric{\Pi_r}{\Pi_s}
&=
\antisymmetric{p_r - e A_r/c}{p_s - e A_s/c}
\\ &=
\cancel{\antisymmetric{p_r}{p_s}}
- \frac{e}{c} \lr{ \antisymmetric{p_r}{A_s} + \antisymmetric{A_r}{p_s}}
+ \frac{e^2}{c^2} \cancel{\antisymmetric{A_r}{A_s}}
\\ &=
- \frac{e}{c} \lr{ (-i\Hbar) \PD{x_r}{A_s} + (i\Hbar) \PD{x_s}{A_r} }
\\ &=
- \frac{i e \Hbar}{c} \lr{ -\PD{x_r}{A_s} + \PD{x_s}{A_r} }
\\ &=
- \frac{i e \Hbar}{c} \epsilon_{t s r} B_t,
\end{aligned}
\end{equation}
%
or
%\begin{dmath}\label{eqn:gaugeTx:220}
\boxedEquation{eqn:gaugeTx:220}{
\antisymmetric{\Pi_r}{\Pi_s}
=
\frac{i e \Hbar}{c} \epsilon_{r s t} B_t.
}
%\end{dmath}
%
\paragraph{Quantum Lorentz force}
\index{Lorentz force}

For the force equation we have
%
\begin{equation}\label{eqn:gaugeTx:240}
\begin{aligned}
m \frac{d^2 \Bx}{dt^2}
&= \ddt{\BPi}
\\ &= \inv{i \Hbar} \antisymmetric{\BPi}{H}
\\ &= \inv{i \Hbar 2 m } \antisymmetric{\BPi}{\BPi^2}
+ \inv{i \Hbar } \antisymmetric{\BPi}{e \phi}.
\end{aligned}
\end{equation}
%
For the \( \phi \) commutator consider one component
%
\begin{equation}\label{eqn:gaugeTx:260}
\begin{aligned}
\antisymmetric{\Pi_r}{e \phi}
&= e \antisymmetric{p_r - \frac{e}{c} A_r}{\phi}
\\ &= e \antisymmetric{p_r}{\phi}
\\ &= e (-i\Hbar) \PD{x_r}{\phi},
\end{aligned}
\end{equation}
%
or
\begin{equation}\label{eqn:gaugeTx:280}
\inv{i \Hbar} \antisymmetric{\BPi}{e \phi}
=
- e \spacegrad \phi
=
e \BE.
\end{equation}
%
For the \( \BPi^2 \) commutator I initially did this the hard way (it took four notebook pages, plus two for a false start.)  Realizing that I didn't use \cref{eqn:gaugeTx:220} for that expansion was the clue to doing this more expediently.

Considering a single component
%
\begin{equation}\label{eqn:gaugeTx:300}
\begin{aligned}
\antisymmetric{\Pi_r}{\BPi^2}
&=
\antisymmetric{\Pi_r}{\Pi_s \Pi_s}
\\ &=
\Pi_r \Pi_s \Pi_s - \Pi_s \Pi_s \Pi_r
\\ &=
\lr{ \antisymmetric{\Pi_r}{\Pi_s}  + \cancel{\Pi_s \Pi_r} } \Pi_s
- \Pi_s \lr{ \antisymmetric{\Pi_s}{\Pi_r}  + \cancel{\Pi_r \Pi_s} }
\\ &= i \Hbar \frac{e}{c} \epsilon_{r s t}
\lr{ B_t \Pi_s + \Pi_s B_t },
\end{aligned}
\end{equation}
%
or
%
\begin{equation}\label{eqn:gaugeTx:320}
\begin{aligned}
\inv{ i \Hbar 2 m} \antisymmetric{\BPi}{\BPi^2}
&= \frac{e}{2 m c } \epsilon_{r s t} \Be_r
\lr{ B_t \Pi_s + \Pi_s B_t }
\\ &= \frac{e}{ 2 m c }
\lr{
 \BPi \cross \BB
- \BB \cross \BPi
}.
\end{aligned}
\end{equation}
%
Putting all the pieces together we've got the quantum equivalent of the Lorentz force equation
%
%\begin{dmath}\label{eqn:gaugeTx:340}
\boxedEquation{eqn:gaugeTx:340}{
m \frac{d^2 \Bx}{dt^2} = e \BE + \frac{e}{2 c} \lr{
 \frac{d\Bx}{dt} \cross \BB
- \BB \cross \frac{d\Bx}{dt}
}.
}
%\end{dmath}
%
While this looks equivalent to the classical result, all the vectors here are Heisenberg picture operators dependent on position.
%
} % answer

%\EndArticle

         % p37(b):
         %
% Copyright � 2015 Peeter Joot.  All Rights Reserved.
% Licenced as described in the file LICENSE under the root directory of this GIT repository.
%
%\input{../blogpost.tex}
%\renewcommand{\basename}{gaugeTxCurrent}
%\renewcommand{\dirname}{notes/phy1520/}
%%\newcommand{\dateintitle}{}
%%\newcommand{\keywords}{}
%
%\input{../peeter_prologue_print2.tex}
%
%\usepackage{peeters_layout_exercise}
%\usepackage{peeters_braket}
%\usepackage{peeters_figures}
%
%\beginArtNoToc
%
%\generatetitle{Gauge transformed probability current}
%%\chapter{Gauge transformed probability current}
%\label{chap:gaugeTxCurrent}

\makeoproblem{Gauge transformed probability current.}{problem:gaugeTxCurrent:1}{\citep{sakurai2014modern} pr. 2.37 (b)}{
\makesubproblem{}{problem:gaugeTxCurrent:1:a}
\index{probability!current}
\index{gauge transformation!probability current}

For the gauge transformed Schr\"{o}dinger equation
%
\begin{dmath}\label{eqn:gaugeTxCurrent:20}
\inv{2m} \BPi(\Bx) \cdot \BPi(\Bx) \psi(\Bx, t) + e \phi(\Bx) \psi(\Bx, t) = i \Hbar \PD{t}{}\psi(\Bx, t),
\end{dmath}

where
%
\begin{dmath}\label{eqn:gaugeTxCurrent:40}
\BPi(\Bx) = -i \Hbar \spacegrad - \frac{e}{c} \BA(\Bx),
\end{dmath}

find the probability current defined by
%
\begin{dmath}\label{eqn:gaugeTxCurrent:60}
\PD{t}{\psi} + \spacegrad \cdot \Bj.
\end{dmath}

\makesubproblem{}{problem:gaugeTxCurrent:1:b}
Once obtained, let use a \( \psi = \sqrt{\rho} e^{i S/\Hbar} \) wavefunction representation, and find the corresponding form for the probability current.

\makesubproblem{}{problem:gaugeTxCurrent:1:c}
Evaluate \( \int d^3 x \Bj \).

} % problem

\makeanswer{problem:gaugeTxCurrent:1}{

\makeSubAnswer{}{problem:gaugeTxCurrent:1:a}

Equation \cref{eqn:gaugeTxCurrent:20} and its conjugate are
%
\begin{dmath}\label{eqn:gaugeTxCurrent:22}
\begin{aligned}
\inv{2m} \BPi \cdot \BPi \psi + e \phi \psi &= i \Hbar \PD{t}{\psi} \\
\inv{2m} \BPi^\conj \cdot \BPi^\conj \psi^\conj + e \phi \psi^\conj &= -i \Hbar \PD{t}{\psi^\conj}
\end{aligned}
\end{dmath}

which can be used immediately in a chain rule expansion of the probability time derivative
%
\begin{dmath}\label{eqn:gaugeTxCurrent:80}
i \Hbar \PD{t}{\rho}
=
i \Hbar \psi^\conj \PD{t}{\psi} +
i \Hbar \psi \PD{t}{\psi^\conj}
=
\psi^\conj \lr{ \inv{2m} \BPi \cdot \BPi \psi + e \phi \psi } -
\psi \lr{ \inv{2m} \BPi^\conj \cdot \BPi^\conj \psi^\conj + e \phi \psi^\conj }
=
\inv{2m} \lr{
\psi^\conj \BPi \cdot \BPi \psi
-\psi \BPi^\conj \cdot \BPi^\conj \psi^\conj
}.
\end{dmath}

We have a difference of conjugates, so can get away with expanding just the first term
%
\begin{dmath}\label{eqn:gaugeTxCurrent:100}
\psi^\conj \BPi \cdot \BPi \psi
=
\psi^\conj
\psi
=
\psi^\conj
\lr{ -i \Hbar \spacegrad - \frac{e}{c} \BA } \cdot \lr{ -i \Hbar \spacegrad - \frac{e}{c} \BA }
\psi
=
\psi^\conj
\lr{
-\Hbar^2 \spacegrad^2 + \frac{i \Hbar e}{c} \lr{ \BA \cdot \spacegrad + \spacegrad \cdot \BA }
+ \frac{e^2}{c^2} \BA^2
}
\psi.
\end{dmath}

Note that in the directional derivative terms, the gradient operates on everything to its right, including \( \BA \).  Also note that the last term has no imaginary component, so it will not contribute to the difference of conjugates.

This gives
%
\begin{dmath}\label{eqn:gaugeTxCurrent:120}
\begin{aligned}
\psi^\conj \BPi \cdot \BPi \psi - \psi \BPi^\conj \cdot \BPi^\conj \psi^\conj
&=
\psi^\conj
\lr{
-\Hbar^2 \spacegrad^2 \psi + \frac{i \Hbar e}{c} \lr{ \BA \cdot \spacegrad \psi + \spacegrad \cdot (\BA \psi) }
}  \\
&\quad -
\psi
\lr{
-\Hbar^2 \spacegrad^2 \psi^\conj - \frac{i \Hbar e}{c} \lr{ \BA \cdot \spacegrad \psi^\conj + \spacegrad \cdot (\BA \psi^\conj) }
}  \\
&=
-\Hbar^2 \lr{ \psi^\conj \spacegrad^2 \psi - \psi \spacegrad^2 \psi^\conj } \\
&\quad +
\frac{i \Hbar e}{c}
\lr{
\psi^\conj
\BA \cdot \spacegrad \psi + \psi^\conj \spacegrad \cdot (\BA \psi)
+
\psi
\BA \cdot \spacegrad \psi^\conj + \psi \spacegrad \cdot (\BA \psi^\conj)
}
\end{aligned}
\end{dmath}

The first term is recognized as a divergence
%
\begin{dmath}\label{eqn:gaugeTxCurrent:140}
\spacegrad \cdot \lr{ \psi^\conj \spacegrad \psi - \psi \spacegrad \psi^\conj }
=
\psi^\conj \spacegrad \cdot \spacegrad \psi
+
\spacegrad \psi \cdot \spacegrad \psi^\conj
-
\psi \spacegrad \cdot \spacegrad \psi^\conj
-
\spacegrad \psi^\conj \cdot \spacegrad \psi
= \psi^\conj \spacegrad^2 \psi - \psi \spacegrad^2 \psi^\conj.
\end{dmath}

The second term can also be factored into a divergence operation
%
\begin{dmath}\label{eqn:gaugeTxCurrent:160}
\begin{aligned}
\psi^\conj
\BA \cdot \spacegrad \psi &+ \psi^\conj \spacegrad \cdot (\BA \psi)
+
\psi
\BA \cdot \spacegrad \psi^\conj + \psi \spacegrad \cdot (\BA \psi^\conj)  \\
%&=
%\BA \cdot \lr{
%\psi^\conj \spacegrad \psi
%+
%\psi \spacegrad \psi^\conj
%}
%+\psi^\conj \spacegrad \cdot (\BA \psi)
%+\psi \spacegrad \cdot (\BA \psi^\conj) \\
&=
\lr{ \psi^\conj\BA \cdot \spacegrad \psi
+\psi \spacegrad \cdot (\BA \psi^\conj)
}
+\lr{
\psi \BA \cdot \spacegrad \psi^\conj
+\psi^\conj \spacegrad \cdot (\BA \psi)
} \\
&= 2 \spacegrad \cdot \lr{ \BA \psi \psi^\conj } \\
%&= 2 \spacegrad \cdot \lr{ \rho \BA }
\end{aligned}
\end{dmath}

Putting all the pieces back together we have
%
\begin{dmath}\label{eqn:gaugeTxCurrent:180}
\PD{t}{\rho}
=
\inv{2m i \Hbar} \lr{
\psi^\conj \BPi \cdot \BPi \psi
-\psi \BPi^\conj \cdot \BPi^\conj \psi^\conj
}
=
\spacegrad \cdot
\inv{2m i \Hbar} \lr{
-\Hbar^2
\lr{ \psi^\conj \spacegrad \psi - \psi \spacegrad \psi^\conj }
+ \frac{ i \Hbar e}{c} 2 \BA \psi \psi^\conj
}
=
\spacegrad \cdot
\lr{
\frac{i \Hbar}{2 m} \lr{ \psi^\conj \spacegrad \psi - \psi \spacegrad \psi^\conj }
+ \frac{e}{m c} \BA \psi \psi^\conj
}.
\end{dmath}

From \cref{eqn:gaugeTxCurrent:60}, the probability current must be
%
\begin{dmath}\label{eqn:gaugeTxCurrent:200}
\Bj
=
\frac{\Hbar}{2 i m} \lr{ \psi^\conj \spacegrad \psi - \psi \spacegrad \psi^\conj }
- \frac{e}{m c} \BA \psi \psi^\conj,
\end{dmath}

or
%\begin{dmath}\label{eqn:gaugeTxCurrent:220}
\boxedEquation{eqn:gaugeTxCurrent:220}{
\Bj
=
\frac{\Hbar}{m} \Imag \lr{ \psi^\conj \spacegrad \psi }
- \frac{e}{m c} \BA \psi \psi^\conj.
}
%\end{dmath}

\makeSubAnswer{}{problem:gaugeTxCurrent:1:b}

To find the \( \psi = \sqrt{\rho} e^{i S/\Hbar} \) form of the current, note that
%
\begin{dmath}\label{eqn:gaugeTxCurrent:240}
\spacegrad \psi = e^{i S/\Hbar} \spacegrad \sqrt{\rho} + \sqrt{\rho} e^{i S/\Hbar} \spacegrad \lr{ i S/\Hbar },
\end{dmath}

so
\begin{dmath}\label{eqn:gaugeTxCurrent:260}
\psi^\conj \spacegrad \psi
=
\sqrt{\rho} \spacegrad \sqrt{\rho} + \frac{ i \rho}{\Hbar} \spacegrad S.
\end{dmath}

Discarding the real part of this product, we have
%
\begin{dmath}\label{eqn:gaugeTxCurrent:280}
\Bj
= \frac{\Hbar}{m} \rho \spacegrad S - \frac{e }{m c} \BA \rho,
\end{dmath}

or
%\begin{dmath}\label{eqn:gaugeTxCurrent:300}
\boxedEquation{eqn:gaugeTxCurrent:300}{
\Bj = \frac{\rho}{m} \lr{ \spacegrad S - \frac{e}{c} \BA }.
}
%\end{dmath}

\makeSubAnswer{}{problem:gaugeTxCurrent:1:c}
Finally, note that
%
\begin{dmath}\label{eqn:gaugeTxCurrent:320}
-i \Hbar \spacegrad \psi = \bra{ \Bx } \Bp \ket{\psi},
\end{dmath}

so
%
\begin{dmath}\label{eqn:gaugeTxCurrent:340}
\Bj
= \frac{\Hbar}{m} \Imag \lr{ \braket{\Psi}{\Bx} \lr{ \frac{i}{ \Hbar} } \bra{\Bx} \Bp \ket{\psi} }
- \frac{e}{m c} \BA \braket{ \psi}{\Bx} \braket{\Bx}{\psi}.
\end{dmath}

Integrating over all space to eliminate the identity operators, this is
%
\begin{dmath}\label{eqn:gaugeTxCurrent:360}
\int d^3 x \Bj
=
\frac{1}{m} \Imag \lr{ i \bra{\Psi} \Bp \ket{\psi} }
- \frac{e}{m c} \BA \braket{ \psi}{\psi}
=
\inv{m} \bra{\psi} \lr{ \Bp - \frac{e}{c} \BA } \ket{\psi}
=
\inv{m} \expectation{ \BPi }.
\end{dmath}

} % answer

%\EndArticle

         % ps2. coherent states
         %
% Copyright � 2015 Peeter Joot.  All Rights Reserved.
% Licenced as described in the file LICENSE under the root directory of this GIT repository.
%

\makeoproblem{Coherent States.}{gradQuantum:problemSet2:1}{phy1520 2015 ps2.1, and \citep{sakurai2014modern} pr. 2.19(c)}{
\index{coherent state}

Consider the harmonic oscillator Hamiltonian \( H = p^2/2m + m \omega^2 x^2/2\). Define the coherent state \( \ket{z} \) as the eigenfunction of the annihilation operator, via \( a \ket{z} = z \ket{z} \), where \( a \) is the oscillator annihilation operator and \( z \) is some complex number which characterizes the coherent state.

\makesubproblem{}{gradQuantum:problemSet2:1a}
Expanding \( \ket{ z } \) in terms of oscillator energy eigenstates \( \ket{n} \), show that \( \ket{z} = C e^{z a^\dagger} \ket{0} \). Find the normalization constant C.

\makesubproblem{}{gradQuantum:problemSet2:1b}
Calculate the overlap \( \braket{z}{z'} \) for normalized coherent states \( \ket{z} \).

\makesubproblem{}{gradQuantum:problemSet2:1c}
Using the wavefunction \( \ket{z} \), compute \( \expectation{x}, \expectation{p}, \expectation{x^2}\), and \( \expectation{p^2} \) by defining \( x,p \) in terms of \( a, a^\dagger \).

\makesubproblem{}{gradQuantum:problemSet2:1d}

The time evolution of any observed quantity in quantum mechanics can be described in two ways:

\begin{enumerate}[(a)]
\item Schr\"{o}dinger: the wavefunction evolves as \( \ket{\psi(t)} = e^{-i H t/\Hbar} \ket{\psi(0)} \) and the operator \( A \) is time-independent, or
\item Heisenberg: the wavefunction is fixed to its value at \( t = 0 \), say \(  \ket{\psi} \) , and operators evolve as \( A(t) = e^{i H t/\Hbar} A e^{-i H t/\Hbar}\) .
\end{enumerate}

Show that both prescriptions yield the same result for any matrix elements or measured quantities.

\makesubproblem{}{gradQuantum:problemSet2:1e}
Using the Heisenberg picture, compute the time evolution of \( \expectation{x(t)}, \expectation{p(t)}, \expectation{x^2(t)}\), and \( \expectation{p^2(t)} \) in the coherent state \( \ket{z} \).  Comment on connections to classical dynamics of the oscillator in phase space.

\makesubproblem{}{gradQuantum:problemSet2:1f}

Show that \( \Abs{f(n)}^2 \) for a coherent state written as
%
\begin{dmath}\label{eqn:gradQuantumProblemSet2Problem1:561}
\ket{z} = \sum_{n=0}^\infty f(n) \ket{n}
\end{dmath}

has the form of a Poisson distribution, and find the most probable value of \( n\), and thus the most probable energy.

} % makeproblem

\makeanswer{gradQuantum:problemSet2:1}{
\withproblemsetsParagraph{
\makeSubAnswer{}{gradQuantum:problemSet2:1a}

Let
%
\begin{dmath}\label{eqn:gradQuantumProblemSet2Problem1:20}
\ket{z} = \sum_{n=0}^\infty c_n \ket{n},
\end{dmath}
%
The defining identity for \( \ket{z} \) becomes
%
\begin{dmath}\label{eqn:gradQuantumProblemSet2Problem1:40}
a \ket{z}
=
\sum_{n=0}^\infty c_n a \ket{n}
=
\sum_{n=1}^\infty c_n \sqrt{n} \ket{n-1}
=
\sum_{n=0}^\infty c_{n+1} \sqrt{n+1} \ket{n}
=
\sum_{n=0}^\infty c_{n} z \ket{n}.
\end{dmath}

Equating like terms provides a recurrence relation for \( c_n \)
%
\begin{dmath}\label{eqn:gradQuantumProblemSet2Problem1:60}
c_{n} = \frac{c_{n-1} z}{\sqrt{n}},
\end{dmath}
%
or
%
\begin{dmath}\label{eqn:gradQuantumProblemSet2Problem1:80}
\begin{aligned}
c_1 &= \frac{c_0 z}{\sqrt{1}} \\
c_2 &= \frac{c_1 z}{\sqrt{2}} = \frac{c_0 z^2}{\sqrt{2 \times 1}}  \\
c_3 &= \frac{c_2 z}{\sqrt{3}} = \frac{c_0 z^3}{\sqrt{3!}},
\end{aligned}
\end{dmath}

or more generally
%
\boxedEquation{eqn:gradQuantumProblemSet2Problem1:100}{
c_n = \frac{c_0 z^n}{\sqrt{n!}}.
}

or
\begin{dmath}\label{eqn:gradQuantumProblemSet2Problem1:120}
\ket{z}
= c_0
\sum_{n=0}^\infty \frac{z^n}{\sqrt{n!}} \ket{n}.
\end{dmath}

A similar recurrence relation can be constructed for \( \ket{n} \)
%
\begin{dmath}\label{eqn:gradQuantumProblemSet2Problem1:140}
\ket{n}
= a^\dagger \frac{\ket{ n - 1 }}{\sqrt{n}}
= (a^\dagger)^2 \frac{\ket{ n - 2 }}{\sqrt{(n)(n-1)}}
= (a^\dagger)^{n-1} \frac{\ket{ n - (n-1) }}{\sqrt{n(n-1)(n-2)...(n- (n-2))}}
= (a^\dagger)^{n} \frac{\ket{ 0 }}{\sqrt{n!}},
\end{dmath}
%
so that
%
\begin{dmath}\label{eqn:gradQuantumProblemSet2Problem1:160}
\ket{z}
= c_0
\sum_{n=0}^\infty \frac{(a^\dagger z)^n}{n!} \ket{0},
\end{dmath}
%
or
\boxedEquation{eqn:gradQuantumProblemSet2Problem1:180}{
\ket{z}
= c_0 e^{a^\dagger z} \ket{0}.
}

The normalization follows nicely from the \cref{eqn:gradQuantumProblemSet2Problem1:120} representation
%
\begin{dmath}\label{eqn:gradQuantumProblemSet2Problem1:200}
\begin{aligned}
\braket{z}{z}
&=
\Abs{c_0}^2
\sum_{n,m=0}^\infty
\bra{m} \frac{(z^\conj)^m}{\sqrt{m!}}
\frac{z^n}{\sqrt{n!}} \ket{n} \\
&= \Abs{c_0}^2
\sum_{n=0}^\infty
\bra{n} \frac{\Abs{z}^{2n}}{n!} \ket{n} \\
&= \Abs{c_0}^2
\sum_{n=0}^\infty
\frac{\Abs{z}^{2n}}{n!}  \\
&= \Abs{c_0}^2 e^{\Abs{z}^2} \\
&= 1.
\end{aligned}
\end{dmath}

Picking a real value for the constant provides a z-dependent normalization for the state
%
\boxedEquation{eqn:gradQuantumProblemSet2Problem1:220}{
c_0 = e^{-\Abs{z}^2/2},
}

or
\begin{dmath}\label{eqn:gradQuantumProblemSet2Problem1:240}
\ket{z} = e^{-\Abs{z}^2/2 + a^\dagger z} \ket{0}.
\end{dmath}

\makeSubAnswer{}{gradQuantum:problemSet2:1b}
%
\begin{dmath}\label{eqn:gradQuantumProblemSet2Problem1:260}
\begin{aligned}
\braket{z}{z'}
&=
e^{-\Abs{z}^2/2 } e^{-\Abs{z'}^2/2 }
\sum_{n,m=0}^\infty
\bra{m} \frac{(z^\conj)^m}{\sqrt{m!}}
\frac{(z')^n}{\sqrt{n!}} \ket{n}  \\
&=
e^{-\Abs{z}^2/2 -\Abs{z'}^2/2 }
\sum_{n=0}^\infty
\bra{n} \frac{(z^\conj)^n}{\sqrt{n!}}
\frac{(z')^n}{\sqrt{n!}} \ket{n}  \\
&=
\exp\lr{ -\Abs{z}^2/2 -\Abs{z'}^2/2 + z^\conj z' }.
\end{aligned}
\end{dmath}

This can be rewritten in terms of the absolute difference between the two z values
%
\begin{dmath}\label{eqn:gradQuantumProblemSet2Problem1:280}
\braket{z}{z'} =
\exp\lr{ -\inv{2} \lr{ \Abs{z - z'}^2 - 2 i \Imag\lr{ z' z^\conj } } },
\end{dmath}
%
however I'm not sure that's any prettier.

\makeSubAnswer{}{gradQuantum:problemSet2:1c}

First note that
%
\begin{dmath}\label{eqn:gradQuantumProblemSet2Problem1:300}
\bra{z} a^\dagger
=
\lr{a \ket{z}}^\dagger
=
\lr{z \ket{z}}^\dagger
=
\bra{z} z^\conj,
\end{dmath}
%
so
%
\begin{dmath}\label{eqn:gradQuantumProblemSet2Problem1:320}
\expectation{x}
=
\frac{x_0}{\sqrt{2}} \bra{z} a + a^\dagger \ket{z}
=
\frac{x_0}{\sqrt{2}} \bra{z} z + z^\conj \ket{z}
=
\frac{x_0}{\sqrt{2}} \lr{ z + z^\conj }
%=
%\frac{2 x_0}{\sqrt{2}} \Real z
%=
%\sqrt{2} x_0 \Real z
%=
%\sqrt{\frac{2 \Hbar}{ m \omega}} \Real z,
=
\sqrt{\frac{2 \Hbar}{ m \omega}} \frac{ z + z^\conj }{2},
\end{dmath}
%
\begin{dmath}\label{eqn:gradQuantumProblemSet2Problem1:340}
\expectation{p}
=
\frac{i \Hbar}{x_0 \sqrt{2}} \bra{z} a^\dagger - a\ket{z}
=
\frac{-i \Hbar}{x_0 \sqrt{2}} \bra{z} z - z^\conj \ket{z}
%=
%\frac{\sqrt{2} \Hbar}{x_0} \Imag z
%=
%\sqrt{ 2 m \Hbar \omega } \Imag z,
=
\sqrt{ 2 m \Hbar \omega } \frac{z - z^\conj}{2i}
\end{dmath}
%
\begin{dmath}\label{eqn:gradQuantumProblemSet2Problem1:360}
\expectation{x^2}
=
\frac{x_0^2}{2} \bra{z} \lr{a + a^\dagger}^2 \ket{z}
=
\frac{x_0^2}{2} \bra{z} \lr{a^2 + (a^\dagger)^2 + a a^\dagger + a^\dagger a} \ket{z}
=
\frac{x_0^2}{2} \bra{z} \lr{a^2 + (a^\dagger)^2 + \antisymmetric{a}{a^\dagger} + 2 a^\dagger a} \ket{z}
=
\frac{x_0^2}{2} \lr{z^2 + (z^\conj)^2 + 1 + 2 z^\conj z}
=
\frac{x_0^2}{2} \lr{(z + z^\conj)^2 + 1}
=
\frac{\Hbar}{2 m \omega} \lr{(z + z^\conj)^2 + 1},
\end{dmath}
%
and
\begin{dmath}\label{eqn:gradQuantumProblemSet2Problem1:380}
\expectation{p^2}
=
\frac{- \Hbar^2}{2 x_0^2} \bra{z} \lr{a^\dagger - a}^2 \ket{z}
=
\frac{- \Hbar^2}{2 x_0^2} \bra{z} \lr{(a^\dagger)^2 + a^2 - a a^\dagger - a^\dagger a} \ket{z}
=
\frac{- \Hbar^2}{2 x_0^2} \bra{z} \lr{(a^\dagger)^2 + a^2 - \antisymmetric{a}{a^\dagger} - 2 a^\dagger a} \ket{z}
=
\frac{- \Hbar^2}{2 x_0^2} \lr{(z^\conj)^2 + z^2 - 1 - 2 z^\conj z}
=
\frac{m \Hbar \omega}{2} \lr{ 1 - (z - z^\conj)^2 }.
\end{dmath}

As a check against what was stated in class, observe that the minimum uncertainty are satisfied
%
\begin{dmath}\label{eqn:gradQuantumProblemSet2Problem1:400}
\expectation{x^2} - \expectation{x}^2
=
\frac{x_0^2}{2} \lr{ (z + z^\conj)^2 + 1 - (z + z^\conj)^2 }
= \frac{x_0^2}{2},
\end{dmath}
%
and
\begin{dmath}\label{eqn:gradQuantumProblemSet2Problem1:420}
\expectation{p^2} - \expectation{p}^2
=
\frac{\Hbar^2}{2 x_0^2} \lr{ 1 - (z + z^\conj)^2 - -(z - z^\conj)^2 }
=
\frac{\Hbar^2}{2 x_0^2},
\end{dmath}
%
so we have
\begin{dmath}\label{eqn:gradQuantumProblemSet2Problem1:440}
\Delta x \Delta p = \frac{\Hbar}{2}.
\end{dmath}

\makeSubAnswer{}{gradQuantum:problemSet2:1d}

Suppose that \( \setlr{ \ket{\psi} } \) is a basis for the observable \( A \).  In the Heisenberg picture the matrix element for that operator is
%
\begin{dmath}\label{eqn:gradQuantumProblemSet2Problem1:501}
\bra{\psi} A(t) \ket{\psi'}
=
\bra{\psi} e^{i H t/\Hbar} A e^{-i H t/\Hbar} \ket{\psi'}
=
\sum_{\psi'',\psi'''}
\bra{\psi} e^{i H t/\Hbar} \ket{\psi''} \bra{\psi''} A \ket{\psi'''} \bra{\psi'''} e^{-i H t/\Hbar} \ket{\psi'}.
\end{dmath}

This product of three matrix elements has the structure of a similarity transformation \( \tilde{U}^\dagger \tilde{A} \tilde{U} \), where \( \tilde{U} \) is the matrix element of the time evolution operator and \( \tilde{A} \) is the matrix element of the observable \( A \).

Compare this to the Schr\"{o}dinger picture matrix element with respect to time evolved states
%
\begin{dmath}\label{eqn:gradQuantumProblemSet2Problem1:521}
\bra{\psi(t)} A \ket{\psi'(t)}
=
\lr{ \bra{\psi} e^{i H t/\Hbar} } A \lr{ e^{-i H t/\Hbar} \ket{\psi'} }
=
\sum_{\psi'',\psi'''}
\bra{\psi} e^{i H t/\Hbar} \ket{\psi''} \bra{\psi''} A \ket{\psi'''} \bra{\psi'''} e^{-i H t/\Hbar} \ket{\psi'}.
\end{dmath}

This has exactly the same structure as in the Heisenberg picture.  Since average quantities are matrix elements with respect to the same pair of states, this shows that measurements are independent of whether the Heisenberg or Schr\"{o}dinger picture is used to describe those measurements.

\makeSubAnswer{}{gradQuantum:problemSet2:1e}

With time evolution in the mix using the Heisenberg representation of the annihilation operator \( a(t) = a e^{-i \omega t} \), the \( x \) expectation is
%
\begin{dmath}\label{eqn:gradQuantumProblemSet2Problem1:460}
\expectation{x(t)}
=
\frac{x_0}{\sqrt{2}} \bra{z} \lr{ a e^{-i \omega t} + a^\dagger e^{i \omega t} } \ket{z}
=
\frac{x_0}{\sqrt{2}} \lr{ z e^{-i \omega t} + z^\conj e^{i \omega t} } .
\end{dmath}

It's clear how to generalize the stationary state calculations in \partref{gradQuantum:problemSet2:1c}, and can do so by inspection
%
\begin{dmath}\label{eqn:gradQuantumProblemSet2Problem1:480}
\begin{aligned}
\expectation{x} &= \sqrt{ \frac{\Hbar}{2 m \omega} } \lr{ z e^{-i \omega t} + z^\conj e^{i \omega t} } \\
\expectation{xp} &= \frac{i \Hbar}{2} \lr{ z e^{-i \omega t} + z^\conj e^{i \omega t} } \lr{ z e^{-i \omega t} - z^\conj e^{i \omega t} } \\
\expectation{p} &= -i \sqrt{ \frac{m \Hbar \omega}{2} } \lr{ z e^{-i \omega t} - z^\conj e^{i \omega t} } \\
\expectation{x^2} &= \frac{\Hbar}{2 m \omega} \lr{(z e^{-i \omega t} + z^\conj e^{i \omega t} )^2 + 1} = \expectation{x}^2 + \frac{\Hbar}{2 m \omega} \\
\expectation{p^2} &= \frac{m \Hbar \omega}{2} \lr{ 1 - (z e^{-i \omega t} - z^\conj e^{i \omega t})^2 } = \expectation{p}^2 + \frac{m \Hbar \omega}{2}.
\end{aligned}
\end{dmath}

In class, symmetric and antisymmetric conjugate sums of \( z \) were identified as position and momentum, with
%
\begin{dmath}\label{eqn:gradQuantumProblemSet2Problem1:541}
\begin{aligned}
x_0 &\equiv \sqrt{\frac{2 \Hbar}{m \omega}} \Real z = \sqrt{\frac{\Hbar}{2 m \omega}} \lr{ z + z^\conj } \\
p_0 &\equiv \sqrt{2 m \Hbar \omega} \Imag z = -i \sqrt{\frac{m \Hbar \omega}{2}} \lr{ z - z^\conj }.
\end{aligned}
\end{dmath}

With that identification the expectation values above are
%
\begin{dmath}\label{eqn:gradQuantumProblemSet2Problem1:481}
\begin{aligned}
\expectation{x} &= x_0 \cos(\omega t) + \frac{p_0}{m \omega} \sin(\omega t) \\
\expectation{p} &= p_0 \cos(\omega t) - m \omega x_0 \sin(\omega t) \\
\end{aligned}
\end{dmath}

These expectations are analogous to the phase space trajectories of classical particles.

\makeSubAnswer{}{gradQuantum:problemSet2:1f}

The Poisson distribution has the form
%
\begin{dmath}\label{eqn:gradQuantumProblemSet2Problem1:581}
P(n) = \frac{\mu^{n} e^{-\mu}}{n!}.
\end{dmath}

Here \( \mu \) is the mean of the distribution
%
\begin{dmath}\label{eqn:gradQuantumProblemSet2Problem1:601}
\expectation{n}
= \sum_{n=0}^\infty n P(n)
= \sum_{n=1}^\infty n \frac{\mu^{n} e^{-\mu}}{n!}
= \mu e^{-\mu} \sum_{n=1}^\infty \frac{\mu^{n-1}}{(n-1)!}
= \mu e^{-\mu} e^{\mu}
= \mu.
\end{dmath}

We found that the coherent state had the form
%
\begin{dmath}\label{eqn:gradQuantumProblemSet2Problem1:621}
\ket{z} = c_0 \sum_{n=0} \frac{z^n}{\sqrt{n!}} \ket{n},
\end{dmath}
%
so the probability coefficients for \( \ket{n} \) are
%
\begin{dmath}\label{eqn:gradQuantumProblemSet2Problem1:641}
P(n)
= c_0^2 \frac{\Abs{z^n}^2}{n!}
= e^{-\Abs{z}^2} \frac{\Abs{z^n}^2}{n!}.
\end{dmath}

This has the structure of the Poisson distribution with mean \( \mu = \Abs{z}^2 \).  The most probable value of \( n \) is that for which \( \Abs{f(n)}^2 \) is the largest.  This is, in general, hard to compute, since we have a maximization problem in the integer domain that falls outside the normal toolbox.  If we assume that \( n \) is large, so that Stirling's approximation can be used to approximate the factorial, and also seek a non-integer value that maximizes the distribution, the most probable value will be the closest integer to that, and this can be computed.  Let
%
\begin{dmath}\label{eqn:gradQuantumProblemSet2Problem1:661}
g(n)
= \Abs{f(n)}^2
= \frac{e^{-\mu} \mu^n}{n!}
= \frac{e^{-\mu} \mu^n}{e^{\ln n!}}
\approx e^{-\mu - n \ln n + n } \mu^n.
= e^{-\mu - n \ln n + n + n \ln \mu }
\end{dmath}

This is maximized when
%
\begin{dmath}\label{eqn:gradQuantumProblemSet2Problem1:681}
0
= \frac{dg}{dn}
= \lr{ - \ln n - 1 + 1 + \ln \mu } g(n),
\end{dmath}
%
which is maximized at \( n = \mu \).  One of the integers \( n = \lfloor \mu \rfloor \) or \( n = \lceil \mu \rceil \) that brackets this value \( \mu = \Abs{z}^2 \) is the most probable.  So, if an energy measurement is made of a coherent state \( \ket{z} \), the most probable value will be one of
%
\begin{dmath}\label{eqn:gradQuantumProblemSet2Problem1:701}
E = \Hbar \lr{
\largestIntLessThan{\Abs{z}^2}
 + \inv{2} },
\end{dmath}
%
or
%
\begin{dmath}\label{eqn:gradQuantumProblemSet2Problem1:721}
E = \Hbar \lr{
\largestIntGreaterThan{\Abs{z}^2}
 + \inv{2} },
\end{dmath}
%
}
}

         % Bohm effect, magnetic interaction, and Landau levels:
         %
% Copyright � 2015 Peeter Joot.  All Rights Reserved.
% Licenced as described in the file LICENSE under the root directory of this GIT repository.
%
\makeoproblem{Aharonov Bohm effect.}{gradQuantum:problemSet3:1}{2015 ps3.1}
{
\index{Aharonov-Bohm effect}
%
\makesubproblem{}{gradQuantum:problemSet3:1a}
Consider Young's double slit experiment with electrons, having a monoenergetic source of electrons hitting a double slit with slit spacing \( d \), with the electrons then landing on a screen at a distance \( D \) away from the double slit.
For electrons with energy E, find the de Broglie wavelength \( \lambda \), and hence the spacing between the fringes on the screen.
You can ignore the drop in intensity as the electron beam `spreads' when it travels from the slits to the screen (recall that the slits act as effective point sources), so just take phase changes into account along the travel path.
%
\makesubproblem{}{gradQuantum:problemSet3:1b}
Next, imagine a thin solenoidal flux \( \Phi \) being placed between the two slits, so that electron paths which encircle the flux once will pick up an Aharonov Bohm phase \( e \Phi/\Hbar c \).
Compute the resulting shift in the interference pattern on the screen.
Show that when the flux \( \Phi \) is increased from \( 0 \rightarrow h c/e \), the interference pattern shifts by exactly one fringe, so the new pattern appears the same as the old.
This is the same flux periodicity we saw in class for the energy levels versus flux for a particle on a ring.
%
} % makeproblem
%
\makeanswer{gradQuantum:problemSet3:1}{
\withproblemsetsParagraph{
\makeSubAnswer{}{gradQuantum:problemSet3:1a}
%
In general, the superposition of two equal amplitude wave packets with the same wavelength can be factored into a phase and amplitude
%
\begin{dmath}\label{eqn:gradQuantumProblemSet3Problem1:20}
e^{i (\omega t - k L_1)} + e^{i (\omega t - k L_1)}
=
e^{i (\omega t - k L_1/2 - k L_2/2)}
\lr{
e^{-i k L_1/2 + i k L_2/2}
+
e^{i k L_1/2 - i k L_2/2}
}
=
2 e^{i(\omega t - k L_1/2 - k L_2/2)} \cos\lr{ k (L_1 - L_2)/2 }.
\end{dmath}
%
Now consider the geometry of this screen configuration as sketched in \cref{fig:ps3p1:ps3p1Fig1}.
\imageFigure{../figures/phy1520-quantum/ps3p1Fig1}{Double slit interference.}{fig:ps3p1:ps3p1Fig1}{0.2}
The upper and lower path lengths are
%
\begin{dmath}\label{eqn:gradQuantumProblemSet3Problem1:40}
L_{1,2}
= \sqrt{ D^2 + (y \mp d/2)^2 }
= D \sqrt{ 1 + \frac{(y \mp d/2)^2}{D^2} }
\approx D \lr{ 1 + \inv{2} \frac{(y \mp d/2)^2}{D^2} }
= D + \inv{2 D} (y \mp d/2)^2.
\end{dmath}
%
To first order the length difference is
%
\begin{dmath}\label{eqn:gradQuantumProblemSet3Problem1:60}
L_1 - L_2
=
\inv{2 D} (y - d/2)^2
-\inv{2 D} (y + d/2)^2
=
-\frac{y d}{D}.
\end{dmath}
%
The amplitude of the interference pattern, at height \( y \) on the screen is gated by the cosine
%
\begin{equation}\label{eqn:gradQuantumProblemSet3Problem1:80}
\cos\lr{ \frac{k y d}{ 2 D} }.
\end{equation}
%
This has peaks and zeros separated by
%
\begin{equation}\label{eqn:gradQuantumProblemSet3Problem1:100}
\frac{k \Delta y d}{ 2 D} = \pi,
\end{equation}
%
or
\begin{equation}\label{eqn:gradQuantumProblemSet3Problem1:120}
\Delta y = \frac{2 \pi D}{k d}.
\end{equation}
%
The electron wave number ( \( k = 2 \pi/\lambda \) ) is
%
\begin{equation}\label{eqn:gradQuantumProblemSet3Problem1:140}
k = \frac{\sqrt{2 m E}}{\Hbar},
\end{equation}
%
so the peak separation is
%
\boxedEquation{eqn:gradQuantumProblemSet3Problem1:160}{
\Delta y
%= \frac{2 \pi D \Hbar}{d \sqrt{2 m E}}
= \frac{D h}{d \sqrt{2 m E}}.
}
%
\makeSubAnswer{}{gradQuantum:problemSet3:1b}
%
Suppose the upper electron path has a positive orientation with respect to the vector potential direction, while the lower electron path has a negative orientation.
The sum of the wave packets will have the form
%
\begin{equation}\label{eqn:gradQuantumProblemSet3Problem1:180}
\begin{aligned}
&e^{i \omega t - k L_1 - e \Phi/\Hbar c}
+
e^{i \omega t - k L_2 + e \Phi/\Hbar c} \\
&\qquad=
2
e^{i (\omega t - k L_1/2 - k L_2/2)}
\lr{
e^{-i k L_1/2 + i k L_2/2 - e \Phi/\Hbar c}
+
e^{i k L_1/2 - i k L_2/2 + e \Phi/\Hbar c}
} \\
&\qquad=
2 e^{i(\omega t - k L_1/2 - k L_2/2)} \cos\lr{ k (L_1 - L_2)/2 + \frac{e \Phi}{\Hbar c}}.
\end{aligned}
\end{equation}
%
As \( \Phi \rightarrow c h/e \) the additional phase term approaches
%
\begin{equation}\label{eqn:gradQuantumProblemSet3Problem1:200}
\frac{h}{\Hbar} = 2 \pi,
\end{equation}
%
so the entire interference pattern is shifted exactly one full cycle.
}
}

         %
% Copyright � 2015 Peeter Joot.  All Rights Reserved.
% Licenced as described in the file LICENSE under the root directory of this GIT repository.
%
\makeoproblem{Landau Levels - Symmetric gauge.}{gradQuantum:problemSet3:2}{phy1520 2015 ps3.2}
{
\index{Landau levels!symmetric gauge}

Consider a charged particle moving in two dimensions (\(xy\)-plane) in a uniform magnetic field \( B_0 \zcap \) perpendicular
to the plane.
Let us work in a different gauge from the Landau gauge we discussed in class, namely, let us set

\begin{dmath}\label{eqn:gradQuantumProblemSet3Problem2:20}
\BA = \frac{B_0}{2} \lr{ x \ycap - y \xcap },
\end{dmath}

where \( (x,y) \) denotes the particle position.
This is called the `symmetric gauge'.

In this gauge,
\makesubproblem{}{gradQuantum:problemSet3:2a}
work out the energy spectrum, and the eigenfunctions, and
\makesubproblem{}{gradQuantum:problemSet3:2b}
provide a crude counting of the number of states per energy level (i.e., the degeneracy) for an electron on a disk of radius \( R \).
} % makeproblem

\makeanswer{gradQuantum:problemSet3:2}{
\withproblemsetsParagraph{

\makeSubAnswer{}{gradQuantum:problemSet3:2a}

Using the approach suggested by our practise problem \citep{sakurai2014modern} \textprref{2.39},
%and \citep{desai2009quantum}
the Hamiltonian for magnetic field driven motion constrained to a plane, can be factored into raising and lowering style operators

\begin{dmath}\label{eqn:gradQuantumProblemSet3Problem2:40}
H
= \inv{2 m} \lr{ \Bp - \frac{e}{c} \BA }^2
= \inv{2 m} \lr{ \Pi_x^2 + \Pi_y^2 }^2
= \inv{2 m} \lr{ \lr{ \Pi_x - i \Pi_y }\lr{ \Pi_x + i \Pi_y } - i \lr{ \Pi_x \Pi_y - \Pi_y \Pi_x } },
\end{dmath}

That commutator term is proportional to the magnetic field strength
\begin{dmath}\label{eqn:gradQuantumProblemSet3Problem2:60}
\Pi_x \Pi_y - \Pi_y \Pi_x
=
\antisymmetric{\Pi_x}{\Pi_y}
=
\antisymmetric{p_x - \frac{e}{c} A_x}{p_y - \frac{e}{c} A_y}
=
\cancel{\antisymmetric{p_x}{p_y}} - \frac{e}{c} \lr{ \antisymmetric{A_x}{p_y} + \antisymmetric{p_x}{A_y} } + \lr{\frac{e}{c}}^2 \cancel{\antisymmetric{A_x}{A_y}}
=
- \frac{e}{c} (-i \Hbar) \lr{ \PD{x}{A_y} - \PD{y}{A_x} }
=
i \frac{e \Hbar}{c} B_z
=
i \frac{e \Hbar B_0}{c},
\end{dmath}

so

\begin{dmath}\label{eqn:gradQuantumProblemSet3Problem2:80}
H
= \inv{2 m} \lr{ \Pi_x - i \Pi_y }\lr{ \Pi_x + i \Pi_y } + \frac{e \Hbar B_0}{2 m c}.
\end{dmath}

Writing

\begin{dmath}\label{eqn:gradQuantumProblemSet3Problem2:100}
\omega = \frac{e B_0}{m c},
\end{dmath}

this appears to have the structure of the 1D Harmonic oscillator
\begin{dmath}\label{eqn:gradQuantumProblemSet3Problem2:120}
H
= \inv{2 m} \lr{ \Pi_x - i \Pi_y }\lr{ \Pi_x + i \Pi_y } + \frac{\Hbar \omega}{2}.
\end{dmath}

Observe that

\begin{dmath}\label{eqn:gradQuantumProblemSet3Problem2:140}
\antisymmetric{ \Pi_x + i \Pi_y }{ \Pi_x - i \Pi_y }
=
i \antisymmetric{ \Pi_y }{ \Pi_x } - i \antisymmetric{ \Pi_x }{ \Pi_y }
=
- 2 i \antisymmetric{ \Pi_x }{ \Pi_y }
=
\frac{2 e \Hbar B_0}{c}
=
2 m \omega.
\end{dmath}

With

\begin{dmath}\label{eqn:gradQuantumProblemSet3Problem2:160}
b = \inv{\sqrt{2 m \omega \Hbar}} \lr{ \Pi_x + i \Pi_y },
\end{dmath}

the Hamiltonian has the form

\begin{dmath}\label{eqn:gradQuantumProblemSet3Problem2:180}
H = \Hbar \omega \lr{ b^\dagger b + \inv{2} },
\end{dmath}

where

\begin{dmath}\label{eqn:gradQuantumProblemSet3Problem2:200}
\antisymmetric{b}{b^\dagger} = 1,
\end{dmath}

just like the 1D SHO.  The energy levels are therefore

\begin{dmath}\label{eqn:gradQuantumProblemSet3Problem2:220}
E_n = \Hbar \omega \lr{ n + \inv{2} }.
\end{dmath}

For the symmetric gauge where \( A_x = - B_0 y/2, A_y = B_0 x/2 \), the lowering operator has the form

\begin{dmath}\label{eqn:gradQuantumProblemSet3Problem2:240}
b
= \inv{\sqrt{2 m \omega \Hbar}} \lr{ p_x + i p_y - \frac{e}{c} \frac{B_0}{2} ( -y + i x ) }
= \frac{-i \Hbar}{\sqrt{2 m \omega \Hbar}} \lr{ \partial_x + i \partial_y + \frac{e B_0}{2 \Hbar c} ( i y + x ) }
= \frac{-i \Hbar}{\sqrt{2 m \omega \Hbar}} \lr{ \partial_x + i \partial_y + \frac{m \omega}{2 \Hbar} ( x + i y) }.
\end{dmath}

With

\begin{equation}\label{eqn:gradQuantumProblemSet3Problem2:260}
\alpha = \frac{m \omega}{2 \Hbar} = \frac{e B_0}{2 \Hbar c},
\end{equation}

The first state is defined by

\begin{dmath}\label{eqn:gradQuantumProblemSet3Problem2:280}
0
=
\bra{x,y} b \ket{0}
\propto
\lr{ \partial_x + i \partial_y + \alpha ( x + i y) } u(x,y).
\end{dmath}

For integer \( n \), this has solutions of the form

\begin{dmath}\label{eqn:gradQuantumProblemSet3Problem2:300}
u_n(x,y) = c_n \lr{ x + i y }^n e^{-\alpha \lr{x^2 + y^2}/2},
\end{dmath}

which can be verified directly

\begin{dmath}\label{eqn:gradQuantumProblemSet3Problem2:320}
\begin{aligned}
\biglr{ &\partial_x + i \partial_y + \alpha ( x + i y) } u_n(x,y) \\
&=
\biglr{ \partial_x + i \partial_y + \alpha ( x + i y) } c_n \lr{ x + i y }^n e^{-\alpha \lr{x^2 + y^2}/2} \\
&=
c_n e^{-\alpha \lr{x^2 + y^2}/2} \lr{
n \lr{ x + i y }^{n-1} \lr{ 1 + i^2 } -\alpha \lr{ x + i y }^n \lr{ 2x + 2 y i }/2 + \alpha \lr{ x + i y }^{n+1}
} \\
&= 0.
\end{aligned}
\end{dmath}

The normalization is given by

\begin{dmath}\label{eqn:gradQuantumProblemSet3Problem2:340}
1
= \int \Abs{u_n(x,y)}^2 dx dy
= \Abs{c_n}^2 \int \Abs{x + i y}^{2n} e^{-\alpha(x^2 + y^2)} dx dy
= \Abs{c_n}^2 \int_0^\infty 2 \pi r r^{2n} e^{-\alpha r^2} dr.
\end{dmath}

Let \( \alpha r^2 = t \), with \( 2 r dr = dt/\alpha \), for

\begin{dmath}\label{eqn:gradQuantumProblemSet3Problem2:360}
1
= \Abs{c_n}^2 \frac{\pi}{\alpha} \int_0^\infty \lr{\frac{t}{\alpha}}^n e^{-t} dt
= \Abs{c_n}^2 \frac{\pi}{\alpha^{n+1}} \int_0^\infty t^{(n + 1) - 1} e^{-t} dt
= \Abs{c_n}^2 \frac{\pi}{\alpha^{n+1}} \Gamma(n+1)
= \Abs{c_n}^2 \frac{\pi}{\alpha^{n+1}} n!,
\end{dmath}

so

\begin{dmath}\label{eqn:gradQuantumProblemSet3Problem2:380}
c_n = \sqrt{\frac{\alpha^{n+1}}{n! \pi}},
\end{dmath}

and

\boxedEquation{eqn:gradQuantumProblemSet3Problem2:400}{
u_n(x,y) = \sqrt{\frac{\alpha^{n+1}}{n! \pi}} \lr{ x + i y }^n e^{-\alpha (x^2 + y^2)/2}.
}

\makeSubAnswer{}{gradQuantum:problemSet3:2b}

The wave functions decay exponentially, and beyond a certain threshold (dependent on \( n \)), not much of the wave function will contribute to the probability density.  To get a feel for this, the probability density \( 2 \pi r \Abs{u_n(r)}^2 \) is plotted for \( n = 2,4,6 \) with \( \alpha = 1 \) in \cref{fig:ps3:ps3Fig2}.

\imageFigure{../phy1520-quantum-figuresps3Fig2}{Some representative probability density plots.}{fig:ps3:ps3Fig2}{0.3}

What is the value of \( r \) for which the probability density is maximized for a given value of \( n \)?  That is

\begin{dmath}\label{eqn:gradQuantumProblemSet3Problem2:420}
0
=
\frac{d}{dr} \lr{
2 \pi r \Abs{u_n(r)}^2
}
=
2 \pi \frac{\alpha^{n+1}}{\pi n!}
\frac{d}{dr} \lr{
r r^{2 n} e^{-\alpha r^2}
}
=
2 \pi \frac{\alpha^{n+1}}{\pi n!}
\lr{
(2 n + 1) r^{2 n}
r^{2 n+ 1} \lr{ - 2 \alpha r }
}
e^{-\alpha r^2}
=
2 \pi \frac{\alpha^{n+1}}{\pi n!}
\lr{
(2 n + 1)
-2 \alpha r^{2}
}
r^{2 n}
e^{-\alpha r^2},
\end{dmath}

which occurs at

\begin{dmath}\label{eqn:gradQuantumProblemSet3Problem2:440}
r^2
= \frac{(2 n + 1)}{2 \alpha}
= \lr{n + \inv{2}} \frac{ 2 \Hbar c}{e B_0}.
\end{dmath}

If this is less than \( R^2 \), and
assuming \( n \) large with respect to \( 1/2 \), and \( r^2 \le R^2 \), this is

\begin{equation}\label{eqn:gradQuantumProblemSet3Problem2:460}
n \frac{ 2 \Hbar c}{e B_0} \le R^2,
\end{equation}

or

\begin{dmath}\label{eqn:gradQuantumProblemSet3Problem2:480}
n
\le \frac{e B_0 R^2}{ 2 \Hbar c}
= \frac{e B_0 \pi R^2}{ h c}
= \frac{e \Phi}{ h c}.
\end{dmath}

The approximate number of ground states if the particle is confined to a disk of radius \( R \) is proportional to the magnetic flux \( \Phi = \int \BB \cdot d\BS = B_0 \pi R^2 \) through that disk.
}
}

         %
% Copyright � 2015 Peeter Joot.  All Rights Reserved.
% Licenced as described in the file LICENSE under the root directory of this GIT repository.
%
% This is 2.28 from \citep{}
\makeoproblem{Aharonov Bohm effect.}{gradQuantum:problemSet3:3}{\citep{sakurai2014modern} pr. 2.28, 2015 ps3.3}
{
\index{Aharonov-Bohm effect}

Consider an electron confined to the interior of a finite hollow cylinder with its axis being \( \zcap \).
Let the inner and outer
walls of the cylinder be at radial coordinates \( \rho_a \) and \( \rho_b > \rho_a \) respectively.
Let the cylinder have its top and bottom ends at \( z = 0,L \).
%
\makesubproblem{}{gradQuantum:problemSet3:3a}
Find the eigenstates for a particle confined to this cylinder (ignore normalization), and show that its energies are given by
%
\begin{equation}\label{eqn:gradQuantumProblemSet3Problem3:20}
E_{l m n} = \frac{\Hbar^2}{2 m } \lr{ k_{m n}^2 + \lr{ \frac{ \pi l }{L} }^2 } \qquad ( l = 1,2,3, \cdots ; m = 0, 1, 2, \cdots )
\end{equation}
%
where \( k_{m n} \) is the $n$th root of the equation
%
\begin{equation}\label{eqn:gradQuantumProblemSet3Problem3:40}
J_m (k_{m n} \rho_b ) N_m (k_{m n} \rho_a ) - N_m (k_{m n} \rho_b ) J_m (k_{m n} \rho_a ) = 0.
\end{equation}
%
\makesubproblem{}{gradQuantum:problemSet3:3b}
Repeat this problem with a uniform magnetic field \( B \zcap \) which is confined to the region \( 0 < \rho < \rho_a \) (i.e., only in the hollow part of the cylinder).
\makesubproblem{}{gradQuantum:problemSet3:3c}
Show that there is a periodicity of the energy levels with the field, with the period being such that \( \pi \rho_a^2 B = 2 \pi N \hbar c/e \).
} % makeproblem
%
\makeanswer{gradQuantum:problemSet3:3}{
\withproblemsetsParagraph{

We can model this geometrical constraints of these two configurations by
%
\begin{equation}\label{eqn:gradQuantumProblemSet3Problem3:60}
H = \inv{2 m} \lr{ -i \Hbar \spacegrad - \frac{e}{c}\BA } + V,
\end{equation}
%
where
%
\begin{equation}\label{eqn:gradQuantumProblemSet3Problem3:80}
\spacegrad = \rhocap \partial_\rho + \inv{\rho} \partial_\phi + \zcap \partial_z,
\end{equation}
%
and
%
\begin{dmath}\label{eqn:gradQuantumProblemSet3Problem3:100}
V =
\left\{
\begin{array}{l l}
0& \quad \mbox{if \( \rho \in [\rho_a,\rho_b], z \in [0,L] \) } \\
\infty & \quad \mbox{otherwise.}
\end{array}
\right.
\end{dmath}
%
The effects of the potential require that a wavefunction \( \psi \) solution of this equation satisfies
%
\begin{equation}\label{eqn:gradQuantumProblemSet3Problem3:120}
\evalbar{\psi( z )}{z = 0,L} = 0,
\end{equation}
%
and
%
\begin{equation}\label{eqn:gradQuantumProblemSet3Problem3:140}
\evalbar{\psi( \rho )}{\rho = \rho_a, \rho_b} = 0,
\end{equation}
%
leaving
%
\begin{equation}\label{eqn:gradQuantumProblemSet3Problem3:160}
H = \inv{2 m} \lr{ -i \Hbar \spacegrad - \frac{e}{c} \BA }^2.
\end{equation}
%
in the interior region of the cylinder where the electron is free to move.
%
\makeSubAnswer{}{gradQuantum:problemSet3:3a}
%
Without a magnetic field, in the interior of the cylinder, the Hamiltonian is
%
\begin{dmath}\label{eqn:gradQuantumProblemSet3Problem3:180}
H \psi
= \frac{-\Hbar^2}{2 m} \spacegrad^2 \psi
=
\frac{-\Hbar^2}{2 m} \lr{
\inv{\rho} \partial_\rho \lr{ \rho \partial_\rho \psi } + \inv{\rho^2} \partial_{\phi\phi} \psi + \partial_{z z} \psi
}.
\end{dmath}
%
Assuming a solution is possible using separation of variables, let
%
\begin{equation}\label{eqn:gradQuantumProblemSet3Problem3:200}
\psi(\rho, \phi, z) = P(\rho) \Phi(\phi) Z(z),
\end{equation}
%
so that
%
\begin{dmath}\label{eqn:gradQuantumProblemSet3Problem3:320}
H \psi = E \psi = E P \Phi Z =
\frac{-\Hbar^2}{2 m} \lr{
\Phi Z \inv{\rho} \partial_\rho \lr{ \rho \partial_\rho P } + P Z \inv{\rho^2} \partial_{\phi\phi} \Phi + P \Phi \partial_{z z} Z
},
\end{dmath}
%
or
%
\begin{dmath}\label{eqn:gradQuantumProblemSet3Problem3:220}
E =
\frac{-\Hbar^2}{2 m} \lr{
\inv{\rho P} \partial_\rho \lr{ \rho \partial_\rho P } + \inv{\rho^2 \Phi} \partial_{\phi\phi} \Phi + \inv{Z} \partial_{z z} Z
},
\end{dmath}
%
Let \( E = E' + E_z \), where
%
\begin{equation}\label{eqn:gradQuantumProblemSet3Problem3:240}
E_z = \frac{-\Hbar^2}{2 m} \inv{Z} Z''.
\end{equation}
%
This has solution
%
\begin{equation}\label{eqn:gradQuantumProblemSet3Problem3:260}
Z = e^{i k_z z},
\end{equation}
%
where \( k_z = \sqrt{2 m E_z}/\Hbar \).  The \( z = 0,L \) boundary condition requires that
%
\begin{equation}\label{eqn:gradQuantumProblemSet3Problem3:340}
k_z L = \pi l, \qquad l \in \bbZ
\end{equation}
%
or
%
\begin{equation}\label{eqn:gradQuantumProblemSet3Problem3:280}
k_z = \frac{\pi l}{L}.
\end{equation}
%
That means that the total energy is of the form
%
\begin{equation}\label{eqn:gradQuantumProblemSet3Problem3:300}
E = E' + \frac{\Hbar^2}{2 m} \lr{ \frac{\pi l}{L} }^2.
\end{equation}
%
Now we consider the remaining subset of the eigenvalue equation
%
\begin{dmath}\label{eqn:gradQuantumProblemSet3Problem3:360}
E' \rho^2 =
\frac{-\Hbar^2}{2 m} \lr{
\frac{\rho}{P} \partial_\rho \lr{ \rho \partial_\rho P } + \inv{\Phi} \partial_{\phi\phi} \Phi
}.
\end{dmath}
%
We are free to let
%
\begin{equation}\label{eqn:gradQuantumProblemSet3Problem3:380}
\frac{-\Hbar^2}{2 m} \inv{\Phi} \partial_{\phi\phi} \Phi = E_\phi,
\end{equation}
%
which has solution
%
\begin{equation}\label{eqn:gradQuantumProblemSet3Problem3:400}
\Phi = e^{i k_\phi \phi},
\end{equation}
%
where \( k_\phi = \sqrt{2 m E_\phi}/\Hbar \).  Geometry requires \( k_\phi (2 \pi) = 2 \pi \nu, \nu \in \bbZ \ge 0 \),
or
%
\begin{equation}\label{eqn:gradQuantumProblemSet3Problem3:420}
E_\phi = \frac{\Hbar^2 \nu^2}{2 m}.
\end{equation}
%
This leaves
%
\begin{equation}\label{eqn:gradQuantumProblemSet3Problem3:440}
E' \rho^2 = \frac{-\Hbar^2}{2 m}
\frac{\rho}{P} \partial_\rho \lr{ \rho \partial_\rho P } + \frac{\Hbar^2 \nu^2}{2 m},
\end{equation}
%
or
%
\begin{dmath}\label{eqn:gradQuantumProblemSet3Problem3:460}
0 =
\rho^2 \partial_{\rho \rho} P + \rho \partial_\rho P
+
\lr{ \frac{2 m}{\Hbar^2} E' \rho^2 - \nu^2 } P.
\end{dmath}
%
With \( \rho' = \sqrt{ \frac{2 m E'}{\Hbar^2} } \rho \), this is the standard form (\texteqnref{10.2.1} \citep{NIST:DLMF}) for Bessel's equation as a function of \( \rho' \), with solutions \( J_{\pm \nu}( \rho' ), N_\nu( \rho') \).  In particular, the linear combination
%
\begin{equation}\label{eqn:gradQuantumProblemSet3Problem3:480}
a J_{\pm \nu}( \rho' ) + b N_\nu( \rho'),
\end{equation}
%
is a solution.  The geometry requires this vanish at \( \rho_a, \rho_b \).  With
%
\begin{equation}\label{eqn:gradQuantumProblemSet3Problem3:540}
k = \sqrt{ \frac{2 m E'}{\Hbar^2} },
\end{equation}
%
those boundary value constraints can be written as
%
\begin{dmath}\label{eqn:gradQuantumProblemSet3Problem3:500}
0 =
\begin{bmatrix}
J_{\pm \nu}( k \rho_a ) & N_\nu( k \rho_b ) \\
J_{\pm \nu}( k \rho_b ) & N_\nu( k \rho_b )
\end{bmatrix}
\begin{bmatrix}
a \\
b
\end{bmatrix}.
\end{dmath}
%
This is satisfied when the determinant is zero
%
\begin{equation}\label{eqn:gradQuantumProblemSet3Problem3:520}
0 = J_{\pm \nu}( k \rho_a ) N_\nu( k \rho_b ) - J_{\pm \nu}( k \rho_b ) N_\nu( k \rho_a ).
\end{equation}
%
The total energy eigenvalue is
%
\begin{equation}\label{eqn:gradQuantumProblemSet3Problem3:560}
E = \inv{2 m} (\Hbar k)^2 + \frac{\Hbar^2}{2 m} \lr{ \frac{\pi l}{L} }^2.
\end{equation}
%
With minor differences in indexing notation, this completes the demonstration of \cref{eqn:gradQuantumProblemSet3Problem3:20} and \cref{eqn:gradQuantumProblemSet3Problem3:40}.

The complete wavefunction associated with this energy eigenvalue is
%
\boxedEquation{eqn:gradQuantumProblemSet3Problem3:580}{
\psi = \lr{ a J_{\pm \nu}( k \rho ) + b N_\nu( k \rho ) } e^{i \nu \phi} e^{i \pi l z/L}.
}
%
\makeSubAnswer{}{gradQuantum:problemSet3:3b}
%
A magnetic field that is isolated to the hole of the cavity is described by the piecewise vector potential
%
\begin{dmath}\label{eqn:gradQuantumProblemSet3Problem3:600}
\BA =
\left\{
\begin{array}{l l}
\frac{B}{2} \frac{\rho_a^2}{\rho} \phicap & \quad \mbox{if \( \rho \ge \rho_a \) } \\
\frac{B}{2} \rho \phicap & \quad \mbox{if \( \rho < \rho_a \) } \\
\end{array}
\right.
\end{dmath}
%
Checking this for \( \rho \ge \rho_a \) we have
%
\begin{dmath}\label{eqn:gradQuantumProblemSet3Problem3:620}
\spacegrad \cross \BA
=
\frac{B}{2} \rho_a^2 \lr{ \rhocap \partial_\rho + \frac{\phicap}{\rho} \partial_\phi } \cross \frac{\phicap}{\rho}
=
\frac{B}{2} \rho_a^2 \lr{ \rhocap \cross \frac{\phicap}{-\rho^2} + \frac{\phicap}{\rho^2} \partial_\phi \phicap }
=
\frac{B \rho_a^2}{2 \rho^2} \lr{ \rhocap \cross \phicap + \phicap \cross \lr{ -\rhocap} }
=
0,
\end{dmath}
%
and for \( \rho < \rho_a \)
%
\begin{dmath}\label{eqn:gradQuantumProblemSet3Problem3:640}
\spacegrad \cross \BA
=
\frac{B}{2} \lr{ \rhocap \partial_\rho + \frac{\phicap}{\rho} \partial_\phi } \cross \lr{ \rho \phicap }
=
\frac{B}{2} \lr{ \rhocap \cross \phicap + \frac{\phicap}{\rho} \cross \lr{ -\rho \rhocap } }
=
B \zcap.
\end{dmath}
%
In the conduction zone of the cylinder the Hamiltonian is now
%
\begin{dmath}\label{eqn:gradQuantumProblemSet3Problem3:660}
H
= \inv{2 m} \lr{ \Bp - \frac{e}{c} \BA }^2
= \inv{2 m} \lr{ -i \Hbar \spacegrad - \frac{e}{c} \frac{B}{2} \rho_a^2 \frac{\phicap}{\rho} }^2
= \inv{2 m} \lr{
- \Hbar^2 \spacegrad^2
+ i \Hbar \frac{e}{c} \frac{B}{2} \rho_a^2 \lr{ \spacegrad \cdot \frac{\phicap}{\rho} + \frac{\phicap}{\rho} \cdot \spacegrad }
+ \lr{ \frac{e B \rho_a^2}{2 c \rho} }^2
}.
\end{dmath}
Expanding the cross terms we have
%
\begin{dmath}\label{eqn:gradQuantumProblemSet3Problem3:680}
\spacegrad \cdot \lr{ \frac{\phicap}{\rho} \psi }
=
\spacegrad \psi \cdot \frac{\phicap}{\rho}
+
\psi \spacegrad \cdot \frac{\phicap}{\rho}
=
\lr{ \rhocap \partial_\rho \psi + \frac{\phicap}{\rho} \partial_\phi \psi + \partial_z \psi } \cdot \frac{\phicap}{\rho}
+
\psi
\lr{ \rhocap \cdot \frac{\phicap}{-\rho^2} + \frac{\phicap}{\rho} \cdot \frac{(-\rhocap)}{\rho} }
=
\frac{1}{\rho^2} \partial_\phi \psi,
\end{dmath}
%
and
%
\begin{dmath}\label{eqn:gradQuantumProblemSet3Problem3:700}
\frac{\phicap}{\rho} \cdot \spacegrad \psi
=
\frac{\phicap}{\rho} \cdot \frac{\phicap}{\rho} \partial_\phi \psi
=
\frac{1}{\rho^2} \partial_\phi \psi,
\end{dmath}
%
so we have
%
\begin{dmath}\label{eqn:gradQuantumProblemSet3Problem3:720}
\spacegrad \cdot \frac{\phicap}{\rho} + \frac{\phicap}{\rho} \cdot \spacegrad
=
\frac{2}{\rho^2} \partial_\phi.
\end{dmath}
%
The complete Hamiltonian is
%
\begin{dmath}\label{eqn:gradQuantumProblemSet3Problem3:740}
H \psi =
E \psi =
\frac{-\Hbar^2}{2 m} \lr{
\inv{\rho} \partial_\rho \lr{ \rho \partial_\rho \psi } + \inv{\rho^2} \partial_{\phi\phi} \psi + \partial_{z z} \psi
}
+ \frac{i \Hbar e B \rho_a^2}{2 m c \rho^2} \partial_\phi \psi
+
\inv{2m}
\lr{ \frac{e B \rho_a^2}{2 c \rho} }^2 \psi.
\end{dmath}
%
Proceeding the same way with separation of variables using \( \psi = P \Phi Z \), we have as before
%
\begin{equation}\label{eqn:gradQuantumProblemSet3Problem3:760}
\begin{aligned}
Z &= e^{i \pi l/L} \\
E &= E' + \frac{\Hbar^2}{2m} \lr{ \frac{\pi l}{L} }^2,
\end{aligned}
\end{equation}
and are left with
%
\begin{dmath}\label{eqn:gradQuantumProblemSet3Problem3:780}
E' \rho^2 =
\frac{-\Hbar^2}{2 m} \lr{
\frac{\rho}{P} \partial_\rho \lr{ \rho \partial_\rho P } + \inv{\Phi} \partial_{\phi\phi} \Phi
}
+ \frac{i \Hbar}{2m} \frac{e}{c} \frac{B}{2} \rho_a^2 \inv{ \Phi} 2 \partial_\phi \Phi
+
\inv{2m}
\lr{ \frac{e B \rho_a^2}{2 c} }^2.
\end{dmath}
%
Now set
%
\begin{dmath}\label{eqn:gradQuantumProblemSet3Problem3:800}
\frac{-\Hbar^2}{2 m \Phi} \partial_{\phi\phi} \Phi + \frac{i \Hbar e B \rho_a^2}{2 m c \Phi} \partial_\phi \Phi + \inv{2m}
\lr{ \frac{e B \rho_a^2}{2 c} }^2
= E_\phi.
\end{dmath}
%
With \( \Phi = e^{i \nu \phi} \), that gives
%
\begin{dmath}\label{eqn:gradQuantumProblemSet3Problem3:820}
E_\phi
=
\frac{-\Hbar^2}{2 m } (-i \nu)^2 + \frac{i \Hbar e B \rho_a^2}{2 m c \Phi} (i\nu) + \inv{2m}
\lr{ \frac{e B \rho_a^2}{2 c} }^2
=
\frac{\Hbar^2 \nu^2}{2 m } - \frac{\Hbar e B \rho_a^2}{2 m c} \nu + \inv{2m}
\lr{ \frac{e B \rho_a^2}{2 c} }^2
=
\frac{\Hbar^2}{2 m} \lr{ \nu^2 - \frac{e B \rho_a^2}{\Hbar c} \nu }
+ \inv{2m}
\lr{ \frac{e B \rho_a^2}{2 c} }^2
=
\frac{\Hbar^2}{2 m}
\lr{
\lr{ \nu - \frac{e B \rho_a^2}{2 \Hbar c} }^2 - \lr{ \frac{e B \rho_a^2}{2 \Hbar c} }^2
}
+ \inv{2m}
\lr{ \frac{e B \rho_a^2}{2 c} }^2
=
\frac{\Hbar^2}{2 m}
\lr{ \nu - \frac{e B \rho_a^2}{2 \Hbar c} }^2
.
\end{dmath}
%
We are left with
%
\begin{dmath}\label{eqn:gradQuantumProblemSet3Problem3:840}
E' \rho^2
=
\frac{-\Hbar^2}{2 m} \lr{
\frac{\rho}{P} \partial_\rho \lr{ \rho \partial_\rho P }
}
+
E_\phi
=
\frac{-\Hbar^2}{2 m} \lr{
\frac{\rho}{P} \partial_\rho \lr{ \rho \partial_\rho P }
}
+
\frac{\Hbar^2}{2 m}
\lr{ \nu - \frac{e B \rho_a^2}{2 \Hbar c} }^2,
\end{dmath}
%
or
\begin{dmath}\label{eqn:gradQuantumProblemSet3Problem3:860}
\rho^2 P'' + \rho P +
\lr{ \frac{2 m E'}{\Hbar^2} \rho^2 -
\lr{ \nu - \frac{e B \rho_a^2}{2 \Hbar c} }^2 } P = 0.
\end{dmath}
%
With
%
\begin{equation}\label{eqn:gradQuantumProblemSet3Problem3:880}
\nu' = \nu - \frac{e B \rho_a^2}{2 \Hbar c},
\end{equation}
%
and
\begin{equation}\label{eqn:gradQuantumProblemSet3Problem3:900}
k = \frac{\sqrt{2 m E'}}{\Hbar},
\end{equation}
%
this is, once again, a Bessel equation with solution
%
\begin{equation}\label{eqn:gradQuantumProblemSet3Problem3:920}
P = a J_{\pm \nu'}(k \rho) + b N_{\nu'}(k \rho).
\end{equation}
%
The full solution is
%
\boxedEquation{eqn:gradQuantumProblemSet3Problem3:940}{
\psi = \lr{ a J_{\pm \nu'}( k \rho ) + b N_{\nu'}( k \rho ) } e^{i \nu \phi} e^{i \pi l z/L}.
}
%
\makeSubAnswer{}{gradQuantum:problemSet3:3c}
%
With \( \theta = \ifrac{e B \rho_a^2}{2 \Hbar c} \), and \( N = \nu \), \cref{eqn:gradQuantumProblemSet3Problem3:820} takes the form
%
\begin{equation}\label{eqn:gradQuantumProblemSet3Problem3:960}
\frac{2 m E_\phi}{\Hbar^2} = \lr{ N - \theta }^2,
\end{equation}
%
which has zeros when \( N = \theta \), or
%
\begin{equation}\label{eqn:gradQuantumProblemSet3Problem3:980}
N = \frac{e B \rho_a^2}{2 \Hbar c},
\end{equation}
%
Shuffling terms, this is
%
\begin{equation}\label{eqn:gradQuantumProblemSet3Problem3:1000}
\frac{2 \pi N \Hbar c}{e} = B \pi \rho_a^2,
\end{equation}
%
which is the desired result.
}
}

         %
% Copyright � 2015 Peeter Joot.  All Rights Reserved.
% Licenced as described in the file LICENSE under the root directory of this GIT repository.
%
%\input{../blogpost.tex}
%\renewcommand{\basename}{twoSpinHamiltonian}
%\renewcommand{\dirname}{notes/phy1520/}
%%\newcommand{\dateintitle}{}
%%\newcommand{\keywords}{}
%
%\input{../peeter_prologue_print2.tex}
%
%\usepackage{peeters_layout_exercise}
%\usepackage{peeters_braket}
%\usepackage{peeters_figures}
%
%\beginArtNoToc
%
%\generatetitle{Two spin time evolution}
%%\chapter{Two spin time evolution}
%%\label{chap:twoSpinHamiltonian}
%\section{Motivation}

%Our midterm posed a (low mark ``quick question'') that I didn't complete (or at least not properly).  This shouldn't have been a difficult question, but I spend way too much time on it, costing me time that I needed for other questions.
%
%It turns out that there isn't anything fancy required for this question, just perseverance and careful work.
%
%\section{Guts}
%
\makeoproblem{Two spin time evolution.}{problem:twoSpinHamiltonian:1}{midterm pr. 1.iii}{
Compute the time evolution of a two particle state
%
\begin{equation}\label{eqn:twoSpinHamiltonian:20}
\psi = \inv{\sqrt{2}} \lr{ \ket{\uparrow \downarrow} - \ket{\downarrow \uparrow} },
\end{equation}
under the action of the Hamiltonian
%
\begin{equation}\label{eqn:twoSpinHamiltonian:40}
H = - B S_{z,1} + 2 B S_{x,2} = \frac{\Hbar B}{2}\lr{  -\sigma_{z,1} + 2 \sigma_{x,2} } .
\end{equation}
%
} % problem
%
\makeanswer{problem:twoSpinHamiltonian:1}{
We have to know the action of the Hamiltonian on all the states
%
\begin{equation}\label{eqn:twoSpinHamiltonian:60}
\begin{aligned}
H \ket{\uparrow \uparrow} &= \frac{B \Hbar}{2} \lr{ -\ket{\uparrow \uparrow} + 2 \ket{\uparrow \downarrow} } \\
H \ket{\uparrow \downarrow} &= \frac{B \Hbar}{2} \lr{ -\ket{\uparrow \downarrow} + 2 \ket{\uparrow \uparrow} } \\
H \ket{\downarrow \uparrow} &= \frac{B \Hbar}{2} \lr{ \ket{\downarrow \uparrow} + 2 \ket{\downarrow \downarrow} } \\
H \ket{\downarrow \downarrow} &= \frac{B \Hbar}{2} \lr{ \ket{\downarrow \downarrow} + 2 \ket{\downarrow \uparrow} }.
\end{aligned}
\end{equation}
With respect to the basis \( \setlr{ \ket{\uparrow \uparrow}, \ket{\uparrow \downarrow}, \ket{\downarrow \uparrow}, \ket{\downarrow \downarrow} } \), the matrix of the Hamiltonian is
%
\begin{dmath}\label{eqn:twoSpinHamiltonian:80}
H =
% upup updown downup downdown
\frac{ \Hbar B }{2}
\begin{bmatrix}
-1 &  2 & 0 & 0 \\
 2 & -1 & 0 & 0 \\
 0 &  0 & 1 & 2 \\
 0 &  0 & 2 & 1 \\
\end{bmatrix}.
\end{dmath}
Utilizing the block diagonal form (and ignoring the \( \Hbar B/2 \) factor for now), the characteristic equation is
%
\begin{dmath}\label{eqn:twoSpinHamiltonian:100}
0
=
\begin{vmatrix}
-1 -\lambda &  2 \\
 2 & -1 - \lambda
\end{vmatrix}
\begin{vmatrix}
1 -\lambda &  2 \\
 2 & 1 - \lambda
\end{vmatrix}
=
\lr{(1 + \lambda)^2 - 4}
\lr{(1 - \lambda)^2 - 4}.
\end{dmath}
%
This has solutions
%
\begin{dmath}\label{eqn:twoSpinHamiltonian:120}
1 \pm \lambda = \pm 2,
\end{dmath}
%
or, with the \( \Hbar B/2 \) factors put back in
%
\begin{dmath}\label{eqn:twoSpinHamiltonian:140}
\lambda = \pm \Hbar B/2 , \pm 3 \Hbar B/2.
\end{dmath}
%
I was thinking that we needed to compute the time evolution operator
%
\begin{dmath}\label{eqn:twoSpinHamiltonian:160}
U = e^{-i H t/\Hbar},
\end{dmath}
%
but we actually only need the eigenvectors, and the inverse relations.  We can find the eigenvectors by inspection in each case from
%
\begin{dmath}\label{eqn:twoSpinHamiltonian:180}
\begin{aligned}
H - (1) \frac{ \Hbar B }{2}
&=
\frac{ \Hbar B }{2}
\begin{bmatrix}
-2 &  2 & 0 & 0 \\
 2 & -2 & 0 & 0 \\
 0 &  0 & 0 & 2 \\
 0 &  0 & 2 & 0 \\
\end{bmatrix} \\
H - (-1) \frac{ \Hbar B }{2}
&=
\frac{ \Hbar B }{2}
\begin{bmatrix}
 0 &  2 & 0 & 0 \\
 2 &  0 & 0 & 0 \\
 0 &  0 & 2 & 2 \\
 0 &  0 & 2 & 2 \\
\end{bmatrix} \\
H - (3) \frac{ \Hbar B }{2}
&=
\frac{ \Hbar B }{2}
\begin{bmatrix}
-4 &  2 & 0 & 0 \\
 2 & -4 & 0 & 0 \\
 0 &  0 &-2 & 2 \\
 0 &  0 & 2 &-2 \\
\end{bmatrix} \\
H - (-3) \frac{ \Hbar B }{2}
&=
\frac{ \Hbar B }{2}
\begin{bmatrix}
 2 &  2 & 0 & 0 \\
 2 &  2 & 0 & 0 \\
 0 &  0 & 4 & 2 \\
 0 &  0 & 2 & 1 \\
\end{bmatrix}.
\end{aligned}
\end{dmath}
%
The eigenkets are
%
\begin{equation}\label{eqn:twoSpinHamiltonian:280}
\begin{aligned}
\ket{1} &=
\inv{\sqrt{2}}
\begin{bmatrix}
1 \\
1 \\
0 \\
0 \\
\end{bmatrix},\qquad
\ket{-1} =
\inv{\sqrt{2}}
\begin{bmatrix}
0 \\
0 \\
1 \\
-1 \\
\end{bmatrix}, \\
\ket{3} &=
\inv{\sqrt{2}}
\begin{bmatrix}
0 \\
0 \\
1 \\
1 \\
\end{bmatrix},\qquad
\ket{-3} =
\inv{\sqrt{2}}
\begin{bmatrix}
1 \\
-1 \\
0 \\
0 \\
\end{bmatrix},
\end{aligned}
\end{equation}
or
%
\begin{equation}\label{eqn:twoSpinHamiltonian:300}
\begin{aligned}
\sqrt{2} \ket{1} &= \ket{\uparrow \uparrow} + \ket{\uparrow \downarrow} \\
\sqrt{2} \ket{-1} &= \ket{\downarrow \uparrow} - \ket{\downarrow \downarrow} \\
\sqrt{2} \ket{3} &= \ket{\downarrow \uparrow} + \ket{\downarrow \downarrow} \\
\sqrt{2} \ket{-3} &= \ket{\uparrow \uparrow} - \ket{\uparrow \downarrow}.
\end{aligned}
\end{equation}
%
We can invert these
%
\begin{dmath}\label{eqn:twoSpinHamiltonian:220}
\begin{aligned}
\ket{\uparrow \uparrow} &= \inv{\sqrt{2}} \lr{ \ket{1} + \ket{-3} } \\
\ket{\uparrow \downarrow} &= \inv{\sqrt{2}} \lr{ \ket{1} - \ket{-3} } \\
\ket{\downarrow \uparrow} &= \inv{\sqrt{2}} \lr{ \ket{3} + \ket{-1} } \\
\ket{\downarrow \downarrow} &= \inv{\sqrt{2}} \lr{ \ket{3} - \ket{-1} }.
\end{aligned}
\end{dmath}
The original state of interest can now be expressed in terms of the eigenkets
%
\begin{dmath}\label{eqn:twoSpinHamiltonian:240}
\psi
=
\inv{2} \lr{
\ket{1} - \ket{-3} -
\ket{3} - \ket{-1}
}.
\end{dmath}
The time evolution of this ket is
%
\begin{equation}\label{eqn:twoSpinHamiltonian:260}
\begin{aligned}
\psi(t)
&=
\inv{2}
\lr{
e^{-i B t/2} \ket{1}
- e^{3 i B t/2} \ket{-3}
- e^{-3 i B t/2} \ket{3}
- e^{i B t/2} \ket{-1}
} \\
&=
\inv{2 \sqrt{2}}
\Biglr{
e^{-i B t/2} \lr{ \ket{\uparrow \uparrow} + \ket{\uparrow \downarrow} }
- e^{3 i B t/2} \lr{ \ket{\uparrow \uparrow} - \ket{\uparrow \downarrow} } \\
&\qquad - e^{-3 i B t/2} \lr{ \ket{\downarrow \uparrow} + \ket{\downarrow \downarrow} }
- e^{i B t/2} \lr{ \ket{\downarrow \uparrow} - \ket{\downarrow \downarrow} }
} \\
&=
\inv{2 \sqrt{2}}
\Biglr{
  \lr{ e^{-i B t/2} - e^{3 i B t/2} } \ket{\uparrow \uparrow}
+ \lr{ e^{-i B t/2} + e^{3 i B t/2} } \ket{\uparrow \downarrow}  \\
&\qquad - \lr{ e^{-3 i B t/2} + e^{i B t/2} } \ket{\downarrow \uparrow}
+ \lr{ e^{i B t/2} - e^{-3 i B t/2} } \ket{\downarrow \downarrow}
} \\
&=
\inv{2 \sqrt{2}}
\Biglr{
  e^{i B t/2} \lr{ e^{-2 i B t/2} - e^{2 i B t/2} } \ket{\uparrow \uparrow}
+ e^{i B t/2}  \lr{ e^{-2 i B t/2} + e^{2 i B t/2} } \ket{\uparrow \downarrow}  \\
&\qquad - e^{- i B t/2} \lr{ e^{-2 i B t/2} + e^{2 i B t/2} } \ket{\downarrow \uparrow}
+ e^{- i B t/2} \lr{ e^{2 i B t/2} - e^{-2 i B t/2} } \ket{\downarrow \downarrow}
} \\
&=
\inv{\sqrt{2}}
\biglr{
i \sin( B t )
\lr{
 e^{- i B t/2} \ket{\downarrow \downarrow} - e^{i B t/2} \ket{\uparrow \uparrow}
} \\
&\qquad
+ \cos( B t ) \lr{
e^{i B t/2} \ket{\uparrow \downarrow}
- e^{- i B t/2} \ket{\downarrow \uparrow}
}
}.
\end{aligned}
\end{equation}
Note that this returns to the original state when \( t = \frac{2 \pi n}{B}, n \in \bbZ \).  I think I've got it right this time (although I got a slightly different answer on paper before typing it up.)

%This doesn't exactly seem like a quick answer question, at least to me.  Is there some easier way to do it?
} % answer

%\EndNoBibArticle

         % midterm p2:
         %
% Copyright � 2015 Peeter Joot.  All Rights Reserved.
% Licenced as described in the file LICENSE under the root directory of this GIT repository.
%
%{
%\input{../blogpost.tex}
%\renewcommand{\basename}{particleInUniformElectricAndMagneticField}
%\renewcommand{\dirname}{notes/phy1520/}
%%\newcommand{\dateintitle}{}
%%\newcommand{\keywords}{}
%
%\input{../peeter_prologue_print2.tex}
%
%\usepackage{peeters_layout_exercise}
%\usepackage{peeters_braket}
%\usepackage{peeters_figures}
%
%\beginArtNoToc
%
%\generatetitle{Particle in charged uniform electric and magnetic fields}
%%\chapter{Particle in charged uniform electric and magnetic fields}
%%\label{chap:particleInUniformElectricAndMagneticField}

\makeoproblem{Particle in uniform electric and magnetic fields.}{problem:particleInUniformElectricAndMagneticField:1}{2015 midterm p2}{

Find the energy eigenvalues and states for a charged particle moving the in the \( x, y \) plane in a uniform magnetic field \( B \zcap \) and a uniform electric field \( E \ycap \).
} % problem

\makeanswer{problem:particleInUniformElectricAndMagneticField:1}{

The Hamiltonian for such a problem has the form
%
\begin{dmath}\label{eqn:particleInUniformElectricAndMagneticField:20}
H =
\frac{(p_x - q A_x/c)^2}{2m} +
\frac{(p_y - q A_y/c)^2}{2m} + q \phi,
\end{dmath}
%
where \( \BA, \phi \) are the potentials for the electromagnetic field.  Since we don't want a time dependent electric potential, our only choice is
%
\begin{equation}\label{eqn:particleInUniformElectricAndMagneticField:40}
E \ycap = -\spacegrad \phi = -\spacegrad (-E y).
\end{equation}

We have many choices for the magnetic field, but it will have to be of the form \( \BA = B(-a y, b x,0) \), for example \( a = b = 1/2 \).  The Hamiltonian becomes
%
\begin{dmath}\label{eqn:particleInUniformElectricAndMagneticField:60}
H =
\frac{(p_x + q B a y /c)^2}{2m} +
\frac{(p_y - q B b x /c)^2}{2m} - q E y,
\end{dmath}
%
We seek a wave function that makes this separable.  If we try \( \psi = e^{i k y} \phi(x) \) we get
%
\begin{equation}\label{eqn:particleInUniformElectricAndMagneticField:80}
\frac{H \psi}{e^{iky}}
=
\lr{ \frac{(p_x + q B a y /c)^2}{2m} +
\frac{(\Hbar k - q B b x /c)^2}{2m} - q E y
} \phi(x),
\end{equation}
%
and if we try \( \psi = e^{i k y} \phi(x) \) we get
%
\begin{equation}\label{eqn:particleInUniformElectricAndMagneticField:100}
\frac{H \psi}{e^{ikx}}
=
\lr{ \frac{(\Hbar k + q B a y /c)^2}{2m} +
\frac{(p_y - q B b x /c)^2}{2m} - q E y
} \phi(y).
\end{equation}

The latter is separable if we set \( b = 0 \), which requires \( a = 1 \), leaving an eigenvalue equation for \( \phi(y) \)
%
\begin{dmath}\label{eqn:particleInUniformElectricAndMagneticField:120}
H'
=
\frac{(\Hbar k + q B y /c)^2}{2m} +
\frac{p_y^2}{2m}
- q E y
=
\frac{p_y^2}{2m}
- \lr{ q E - \Hbar k q B /m c } y
+ \inv{2 m }\lr{\frac{q B y}{c}}^2
+ \frac{(\Hbar k)^2}{2m}
=
\frac{p_y^2}{2m}
+ \inv{2m} \lr{ \frac{q B}{c} }^2
\lr{
- \frac{2}{m} \lr{ \frac{m c}{q B} }^2 \lr{ q E - \Hbar k q B /m c } y
+ y^2
}
+ \frac{(\Hbar k)^2}{2m}
=
\frac{p_y^2}{2m}
+
\inv{2} m
\lr{ \frac{q B}{m c} }^2
\lr{
y - \inv{m} \lr{ \frac{m c}{q B} }^2 \lr{ q E - \Hbar k q B /m c }
}^2
-
\inv{2 m} \lr{ \frac{m c}{q B} }^2 \lr{ q E - \Hbar k q B /m c }^2
+ \frac{(\Hbar k)^2}{2m}.
\end{dmath}

Let
\begin{equation}\label{eqn:particleInUniformElectricAndMagneticField:140}
\begin{aligned}
\omega &= \frac{q B}{m c} \\
y_0 &= \frac{1}{m \omega^2} \lr{ q E - \Hbar k \omega } \\
E_0 &= \frac{(\Hbar k)^2}{2m} - \inv{2} m \omega^2 y_0^2,
\end{aligned}
\end{equation}

leaving
%
\begin{equation}\label{eqn:particleInUniformElectricAndMagneticField:160}
H' = \frac{p_y^2}{2m} + \inv{2} m\omega^2 (y -y_0)^2 + E_0.
\end{equation}

The energy eigenvalues are therefore
%
\begin{equation}\label{eqn:particleInUniformElectricAndMagneticField:180}
E = \Hbar \omega \lr{ n + \inv{2} } + E_0,
\end{equation}
%
and the eigenfunctions are
%
\begin{equation}\label{eqn:particleInUniformElectricAndMagneticField:200}
\psi = e^{i k x} \phi_n(y - y_0),
\end{equation}
%
where \( \phi_n(y) \) is the n-th Harmonic oscillator wavefunction.
} % answer

%}
%\EndNoBibArticle

   \mychapter{Dirac equation in 1D.}
      %
% Copyright � 2015 Peeter Joot.  All Rights Reserved.
% Licenced as described in the file LICENSE under the root directory of this GIT repository.
%
%\input{../blogpost.tex}
%\renewcommand{\basename}{qmLecture8}
%\renewcommand{\dirname}{notes/phy1520/}
%\newcommand{\keywords}{PHY1520H}
%\input{../peeter_prologue_print2.tex}
%
%%\usepackage{phy1520}
%\usepackage{peeters_braket}
%%\usepackage{peeters_layout_exercise}
%\usepackage{peeters_figures}
%\usepackage{mathtools}
%
%\beginArtNoToc
%\generatetitle{PHY1520H Graduate Quantum Mechanics.  Lecture 8: Dirac equation in 1D.  Taught by Prof.\ Arun Paramekanti}
\label{chap:qmLecture8}

%\paragraph{Disclaimer}
%
%Peeter's lecture notes from class.  These may be incoherent and rough.
%
%These are notes for the UofT course PHY1520, Graduate Quantum Mechanics, taught by Prof. Paramekanti.
%
\section{Construction of the Dirac equation}
\index{Dirac equation}
\paragraph{Schr\"{o}dinger Derivation}

Recall that a ``derivation'' of the Schr\"{o}dinger equation can be associated with the following equivalences
%
\begin{equation}\label{eqn:qmLecture8:300}
E \leftrightarrow \Hbar \omega \leftrightarrow i \Hbar \PD{t}{}
\end{equation}
\begin{equation}\label{eqn:qmLecture8:320}
p \leftrightarrow \Hbar k \leftrightarrow -i \Hbar \PD{t}{}
\end{equation}

so that the classical energy relationship
%
\begin{dmath}\label{eqn:qmLecture8:20}
E = \frac{p^2}{2m}
\end{dmath}

takes the form
%
\begin{dmath}\label{eqn:qmLecture8:40}
i \Hbar \PD{t}{} = -\frac{\Hbar^2}{2m}.
\end{dmath}

How do we do this in a relativistic context where the energy momentum relationship is
%
\begin{dmath}\label{eqn:qmLecture8:60}
E = \sqrt{ p^2 c^2 + m^2 c^4 } \approx m c^2 + \frac{p^2}{2m} + \cdots
\end{dmath}

where \( m \) is the rest mass and \( c \) is the speed of light.

\paragraph{Attempt I}
%
\begin{dmath}\label{eqn:qmLecture8:80}
E = m c^2 + \frac{p^2}{2m} + (...) p^4 + (...) p^6 + \cdots
\end{dmath}

First order in time, but infinite order in space \( \partial/\partial x \).  Useless.

\paragraph{Attempt II}
%
\begin{dmath}\label{eqn:qmLecture8:100}
E^2 = p^2 c^2 + m^2 c^4.
\end{dmath}

This gives
%
\begin{dmath}\label{eqn:qmLecture8:120}
-\Hbar^2 \PDSq{t}{\psi} = - \Hbar^2 c^2 \PDSq{x}{\psi} + m^2 c^4 \psi.
\end{dmath}

\index{Klein-Gordon equation}
This is the Klein-Gordon equation, which is second order in time.

\paragraph{Attempt III}

Suppose that we have the matrix
%
\begin{dmath}\label{eqn:qmLecture8:140}
\begin{bmatrix}
p c & m c^2 \\
m c^2 & - p c
\end{bmatrix},
\end{dmath}
%
or
%
\begin{dmath}\label{eqn:qmLecture8:160}
\begin{bmatrix}
m c^2 & i p c \\
-i p c & - m c^2
\end{bmatrix},
\end{dmath}
%
These both happen to have eigenvalues \( \lambda_{\pm} = \pm \sqrt{p^2 c^2} \).  For those familiar with the Dirac matrices, this amounts to a choice for different representations of the gamma matrices.

Working with \cref{eqn:qmLecture8:140}, which has some nicer features than other possible representations, we seek a state
%
\begin{dmath}\label{eqn:qmLecture8:180}
\Bpsi =
\begin{bmatrix}
\psi_1(x, t) \\
\psi_2(x, t) \\
\end{bmatrix},
\end{dmath}
%
where we aim to write down an equation for this composite state.
%
\begin{dmath}\label{eqn:qmLecture8:200}
i \Hbar \PD{t}{\Bpsi} = \BH \Bpsi
\end{dmath}

Assuming the matrix is the Hamiltonian, multiplying that with the composite state gives
%
\begin{dmath}\label{eqn:qmLecture8:220}
\begin{bmatrix}
i \Hbar \PD{t}{\psi_1} \\
i \Hbar \PD{t}{\psi_1}
\end{bmatrix}
=
\begin{bmatrix}
\hatp c & m c^2 \\
m c^2 & - \hatp c
\end{bmatrix}
\begin{bmatrix}
\psi_1(x, t) \\
\psi_2(x, t) \\
\end{bmatrix}
=
\begin{bmatrix}
\hatp c \psi_1 + m c^2 \psi_2  \\
m c^2 \psi_1 - \hatp c \psi_2
\end{bmatrix}.
\end{dmath}

What happens when we square this
%
\begin{dmath}\label{eqn:qmLecture8:240}
\lr{ i \Hbar \PD{t}{} }^2 \Bpsi
= \BH \BH \Bpsi
=
\begin{bmatrix}
\hatp c & m c^2 \\
m c^2 & - \hatp c
\end{bmatrix}
\begin{bmatrix}
\hatp c & m c^2 \\
m c^2 & - \hatp c
\end{bmatrix}
\Bpsi
=
\begin{bmatrix}
\hatp^2 c^2 + m^2 c^4 & 0 \\
0 & \hatp^2 c^2 + m^2 c^4 \\
\end{bmatrix}
\end{dmath}
%
\begin{dmath}\label{eqn:qmLecture8:260}
- \Hbar^2 \PDSq{t}{} \Bpsi
=
\lr{ \hatp^2 c^2 + m^2 c^4 } \BOne \Bpsi,
\end{dmath}
%
or more exactly
%
\begin{dmath}\label{eqn:qmLecture8:280}
- \Hbar^2 \PDSq{t}{} \psi_{1,2}
=
\lr{ \hatp^2 c^2 + m^2 c^4 } \psi_{1,2}.
\end{dmath}
%
This recovers the Klein Gordon equation for each of the wave functions \( \psi_1, \psi_2 \).

\section{Plane wave solution}
\index{Dirac equation!plane wave}

Instead of squaring the operators, lets try to solve the first order equation.     To do so we'll want to diagonalize \( \BH \).

Before doing that, let's write out the Hamiltonian in an alternate but useful form
%
\begin{dmath}\label{eqn:qmLecture8:340}
\BH =
\hatp c
\begin{bmatrix}
1 & 0 \\
0 & -1
\end{bmatrix}
+
m c^2
\begin{bmatrix}
0 & 1 \\
1 & 0
\end{bmatrix}
= \hatp c \hat\sigma_z + m c^2 \hat\sigma_x.
\end{dmath}

We have two types of operators in the mix here.  We have matrix operators that act on the wave function matrices, as well as derivative operators that act on the components of those matrices.

We have
%
\begin{dmath}\label{eqn:qmLecture8:360}
\hat\sigma_z
\begin{bmatrix}
\psi_1 \\
\psi_2 \\
\end{bmatrix}
=
\begin{bmatrix}
\psi_1 \\
-\psi_2 \\
\end{bmatrix},
\end{dmath}
%
and
%
\begin{dmath}\label{eqn:qmLecture8:380}
\hat\sigma_x
\begin{bmatrix}
\psi_1 \\
\psi_2 \\
\end{bmatrix}
=
\begin{bmatrix}
\psi_2 \\
\psi_1 \\
\end{bmatrix}.
\end{dmath}

Because the derivative actions of \( \hatp \) and the matrix operators are independent, we see that these operators commute.  For example
%
\begin{dmath}\label{eqn:qmLecture8:400}
\hat\sigma_z \hatp
\begin{bmatrix}
\psi_1 \\
\psi_2 \\
\end{bmatrix}
=
\hat\sigma_z
\begin{bmatrix}
-i \Hbar \PD{x}{\psi_1} \\
-i \Hbar \PD{x}{\psi_2} \\
\end{bmatrix}
=
\begin{bmatrix}
-i \Hbar \PD{x}{\psi_1} \\
i \Hbar \PD{x}{\psi_2} \\
\end{bmatrix}
=
\hatp
\hat\sigma_z
\begin{bmatrix}
\psi_1 \\
\psi_2 \\
\end{bmatrix}.
\end{dmath}

\paragraph{Diagonalizing it}
\index{Dirac equation!diagonalization}

Suppose the wave function matrix has the structure
%
\begin{dmath}\label{eqn:qmLecture8:420}
\Bpsi =
\begin{bmatrix}
f_{+} \\
f_{-} \\
\end{bmatrix}
e^{i k x}.
\end{dmath}

We'll plug this into the Schr\"{o}dinger equation and see what we get.

%\EndNoBibArticle

      %
% Copyright � 2015 Peeter Joot.  All Rights Reserved.
% Licenced as described in the file LICENSE under the root directory of this GIT repository.
%
%\input{../blogpost.tex}
%\renewcommand{\basename}{qmLecture9}
%\renewcommand{\dirname}{notes/phy1520/}
%\newcommand{\keywords}{PHY1520H}
%\input{../peeter_prologue_print2.tex}
%
%%\usepackage{phy1520}
%\usepackage{peeters_braket}
%%\usepackage{peeters_layout_exercise}
%\usepackage{peeters_figures}
%\usepackage{mathtools}
%
%\beginArtNoToc
%\generatetitle{PHY1520H Graduate Quantum Mechanics.  Lecture 9: Dirac equation (cont.).  Taught by Prof.\ Arun Paramekanti}
%%\chapter{Dirac equation (cont.)}
%\label{chap:qmLecture9}
%
%\paragraph{Disclaimer}
%
%Peeter's lecture notes from class.  These may be incoherent and rough.
%
%These are notes for the UofT course PHY1520, Graduate Quantum Mechanics, taught by Prof. Paramekanti.
%
\paragraph{Where we left off}

\begin{dmath}\label{eqn:qmLecture9:20}
-i \Hbar \PD{t}{}
\begin{bmatrix}
\psi_1 \\
\psi_2
\end{bmatrix}
=
\begin{bmatrix}
-i \Hbar c \PD{x}{} & m c^2 \\
m c^2 & i \Hbar c \PD{x}{} \\
\end{bmatrix}.
\end{dmath}

\index{Dirac equation!with potential}
With a potential this would be

\begin{dmath}\label{eqn:qmLecture9:40}
-i \Hbar \PD{t}{}
\begin{bmatrix}
\psi_1 \\
\psi_2
\end{bmatrix}
=
\begin{bmatrix}
-i \Hbar c \PD{x}{} + V(x) & m c^2 \\
m c^2 & i \Hbar c \PD{x}{} + V(x) \\
\end{bmatrix}.
\end{dmath}

This means that the potential is raising the energy eigenvalue of the system.

\paragraph{Free Particle}
\index{Dirac equation!free particle}

Assuming a form

\begin{dmath}\label{eqn:qmLecture9:60}
\begin{bmatrix}
\psi_1(x,t) \\
\psi_2(x,t)
\end{bmatrix}
=
e^{i k x}
\begin{bmatrix}
f_1(t) \\
f_2(t) \\
\end{bmatrix},
\end{dmath}

and plugging back into the Dirac equation we have

\begin{dmath}\label{eqn:qmLecture9:80}
-i \Hbar \PD{t}{}
\begin{bmatrix}
f_1 \\
f_2
\end{bmatrix}
=
\begin{bmatrix}
k \Hbar c & m c^2 \\
m c^2 & - \Hbar k c \\
\end{bmatrix}
\begin{bmatrix}
f_1 \\
f_2
\end{bmatrix}.
\end{dmath}

We can use a diagonalizing rotation

\begin{dmath}\label{eqn:qmLecture9:100}
\begin{bmatrix}
f_1 \\
f_2
\end{bmatrix}
=
\begin{bmatrix}
\cos\theta_k & -\sin\theta_k \\
\sin\theta_k & \cos\theta_k \\
\end{bmatrix}
\begin{bmatrix}
f_{+} \\
f_{-} \\
\end{bmatrix}.
\end{dmath}

Plugging this in reduces the system to the form

\begin{dmath}\label{eqn:qmLecture9:140}
-i \Hbar \PD{t}{}
\begin{bmatrix}
f_{+} \\
f_{-} \\
\end{bmatrix}
=
\begin{bmatrix}
E_k & 0 \\
0 & -E_k
\end{bmatrix}
\begin{bmatrix}
f_{+} \\
f_{-} \\
\end{bmatrix}.
\end{dmath}

Where the rotation angle is found to be given by

\begin{dmath}\label{eqn:qmLecture9:160}
\begin{aligned}
\sin(2 \theta_k) &= \frac{m c^2}{\sqrt{(\Hbar k c)^2 + m^2 c^4}} \\
\cos(2 \theta_k) &= \frac{\Hbar k c}{\sqrt{(\Hbar k c)^2 + m^2 c^4}} \\
E_k &= \sqrt{(\Hbar k c)^2 + m^2 c^4}
\end{aligned}
\end{dmath}

\section{Dirac sea and pair creation}
\index{Dirac sea}
\index{pair creation}

See \cref{fig:l9:l9Fig1} for a sketch of energy vs momentum.  The asymptotes are the limiting cases when \( m c^2 \rightarrow 0 \).  The \( + \) branch is what we usually associate with particles.  What about the other energy states.  For Fermions Dirac argued that the lower energy states could be thought of as ``filled up'', using the Pauli principle to leave only the positive energy states available.  This was called the ``Dirac Sea''.  This isn't a good solution, and won't work for example for Bosons.

\imageFigure{../phy1520-quantum-figuresl9Fig1}{Dirac equation solution space.}{fig:l9:l9Fig1}{0.2}

Another way to rationalize this is to employ ideas from solid state theory.  For example consider a semiconductor with a valence and conduction band as sketched in \cref{fig:l9:l9Fig2}.

\imageFigure{../phy1520-quantum-figuresl9Fig2}{Solid state valence and conduction band transition.}{fig:l9:l9Fig2}{0.2}

A photon can excite an electron from the valence band to the conduction band, leaving all the valence band states filled except for one (a hole).  For an electron we can use almost the same picture, as sketched in \cref{fig:l9:l9Fig3}.

\imageFigure{../phy1520-quantum-figuresl9Fig3}{Pair creation.}{fig:l9:l9Fig3}{0.2}

A photon with energy \( E_k - (-E_k) \) can create a positron-electron pair from the vacuum, where the energy of the electron and positron pair is \( E_k \).
%, and the energy of the positron is \( E_k \).
At high enough energies, we can see this pair creation occur.

\section{Zitterbewegung}
\index{zitterbewegung}

If a particle is created at a non-eigenstate such as one on the asymptotes, then oscillations between the positive and negative branches are possible as sketched in \cref{fig:l9:l9Fig4}.

\imageFigure{../phy1520-quantum-figuresl9Fig4}{Zitterbewegung oscillation.}{fig:l9:l9Fig4}{0.15}

Only ``vertical" oscillations between the positive and negative locations on these branches is possible since those are the points that match the particle momentum.  Examining this will be the aim of one of the problem set problems.

\section{Probability and current density}
\index{Dirac equation!probability density}
\index{Dirac equation!current density}

If we define a probability density

\begin{dmath}\label{eqn:qmLecture9:180}
\rho(x, t) = \Abs{\psi_1}^2 + \Abs{\psi_2}^2,
\end{dmath}

does this satisfy a probability conservation relation

\begin{dmath}\label{eqn:qmLecture9:200}
\PD{t}{\rho} + \PD{x}{j} = 0,
\end{dmath}

where \( j \) is the probability current.  Plugging in the density, we have

\begin{dmath}\label{eqn:qmLecture9:220}
\PD{t}{\rho}
=
\PD{t}{\psi_1^\conj} \psi_1
+
\psi_1^\conj \PD{t}{\psi_1}
+
\PD{t}{\psi_2^\conj} \psi_2
+
\psi_2^\conj \PD{t}{\psi_2}.
\end{dmath}

It turns out that the probability current has the form

\begin{dmath}\label{eqn:qmLecture9:240}
j(x,t) = c \lr{ \psi_1^\conj \psi_1 - \psi_2^\conj \psi_2 }.
\end{dmath}

Here the speed of light \( c \) is the slope of the line in the plots above.  We can think of this current density as right movers minus the left movers.  Any state that is given can be thought of as a combination of right moving and left moving states, neither of which are eigenstates of the free particle Hamiltonian.

\section{Potential step}

The next logical thing to think about, as in non-relativistic quantum mechanics, is to think about what occurs when the particle hits a potential step, as in \cref{fig:l9:l9Fig5}.

\imageFigure{../phy1520-quantum-figuresl9Fig5}{Reflection off a potential barrier.}{fig:l9:l9Fig5}{0.2}

The approach is the same.  We write down the wave functions for the \( V = 0 \) region (I), and the higher potential region (II).

The eigenstates are found on the solid lines above the asymptotes on the right hand movers side as sketched in \cref{fig:l9:l9Fig6}.  The right and left moving designations are based on the phase velocity \( \PDi{k}{E} \) (approaching \( \pm c \) on the top-right and top-left quadrants respectively).

\imageFigure{../phy1520-quantum-figuresl9Fig6}{Right movers and left movers.}{fig:l9:l9Fig6}{0.2}

For \( k > 0 \), an eigenstate for the incident wave is

\begin{dmath}\label{eqn:qmLecture9:261}
\Bpsi_{\textrm{inc}}(x) =
\begin{bmatrix}
\cos\theta_k \\
\sin\theta_k
\end{bmatrix}
e^{i k x},
\end{dmath}

For the reflected wave function, we pick a function on the left moving side of the positive energy branch.

\begin{dmath}\label{eqn:qmLecture9:260}
\Bpsi_{\textrm{ref}}(x) =
\begin{bmatrix}
? \\
?
\end{bmatrix}
e^{-i k x},
\end{dmath}

We'll go through this in more detail next time.

%\EndNoBibArticle

      %
% Copyright � 2015 Peeter Joot.  All Rights Reserved.
% Licenced as described in the file LICENSE under the root directory of this GIT repository.
%
%\input{../blogpost.tex}
%\renewcommand{\basename}{qmLecture10}
%\renewcommand{\dirname}{notes/phy1520/}
%\newcommand{\keywords}{PHY1520H}
%\input{../peeter_prologue_print2.tex}
%
%%\usepackage{phy1520}
%\usepackage{peeters_braket}
%%\usepackage{peeters_layout_exercise}
%\usepackage{peeters_figures}
%\usepackage{mathtools}
%
%\beginArtNoToc
%\generatetitle{PHY1520H Graduate Quantum Mechanics.  Lecture 10: 1D Dirac scattering off potential step.  Taught by Prof.\ Arun Paramekanti}
%%\chapter{1D Dirac scattering off potential step}
%\label{chap:qmLecture10}
%
%\paragraph{Disclaimer}
%
%Peeter's lecture notes from class.  These may be incoherent and rough.
%
%These are notes for the UofT course PHY1520, Graduate Quantum Mechanics, taught by Prof. Paramekanti.
\section{Dirac scattering off a potential step.}
\index{Dirac equation!scattering}
For the non-relativistic case we have
%
\begin{equation}\label{eqn:qmLecture10:20}
\begin{aligned}
E < V_0 &\implies T = 0, R = 1 \\
E > V_0 &\implies T > 0, R < 1.
\end{aligned}
\end{equation}
%
What happens for a relativistic 1D particle?
Referring to \cref{fig:lecture10:lecture10Fig1}.
\imageFigure{../figures/phy1520-quantum/lecture10Fig1}{Potential step.}{fig:lecture10:lecture10Fig1}{0.1}
the region I Hamiltonian is
%
\begin{equation}\label{eqn:qmLecture10:40}
H =
\begin{bmatrix}
\hatp c & m c^2 \\
m c^2 & - \hatp c
\end{bmatrix},
\end{equation}
%
for which the solution is
%
\begin{dmath}\label{eqn:qmLecture10:60}
\Phi = e^{i k_1 x }
\begin{bmatrix}
\cos \theta_1 \\
\sin \theta_1
\end{bmatrix},
\end{dmath}
%
where
\begin{dmath}\label{eqn:qmLecture10:80}
\begin{aligned}
\cos 2 \theta_1 &= \frac{ \Hbar c k_1 }{E_{k_1}} \\
\sin 2 \theta_1 &= \frac{ m c^2 }{E_{k_1}}.
\end{aligned}
\end{dmath}
To consider the \( k_1 < 0 \) case, note that
%
\begin{equation}\label{eqn:qmLecture10:100}
\begin{aligned}
\cos^2 \theta_1 - \sin^2 \theta_1 &= \cos 2 \theta_1 \\
2 \sin\theta_1 \cos\theta_1 &= \sin 2 \theta_1,
\end{aligned}
\end{equation}
so after flipping the signs on all the \( k_1 \) terms we find for the reflected wave
%
\begin{dmath}\label{eqn:qmLecture10:120}
\Phi = e^{-i k_1 x}
\begin{bmatrix}
\sin\theta_1 \\
\cos\theta_1
\end{bmatrix}.
\end{dmath}
%
FIXME: this reasoning doesn't entirely make sense to me.  Make sense of this by trying this solution as was done for the form of the incident wave solution.

The region I wave has the form
%
\begin{dmath}\label{eqn:qmLecture10:140}
\Phi_I
=
A e^{i k_1 x}
\begin{bmatrix}
\cos\theta_1 \\
\sin\theta_1 \\
\end{bmatrix}
+
B e^{-i k_1 x}
\begin{bmatrix}
\sin\theta_1 \\
\cos\theta_1 \\
\end{bmatrix}.
\end{dmath}
%
By the time we are done we want to have computed the reflection coefficient
%
\begin{dmath}\label{eqn:qmLecture10:160}
R =
\frac{\Abs{B}^2}{\Abs{A}^2}.
\end{dmath}
%
The region I energy is
%
\begin{equation}\label{eqn:qmLecture10:180}
E = \sqrt{ \lr{ m c^2}^2 + \lr{ \Hbar c k_1 }^2 }.
\end{equation}
%
We must have
\begin{equation}\label{eqn:qmLecture10:200}
E
=
\sqrt{ \lr{ m c^2}^2 + \lr{ \Hbar c k_2 }^2 } + V_0
=
\sqrt{ \lr{ m c^2}^2 + \lr{ \Hbar c k_1 }^2 },
\end{equation}
%
so
%
\begin{dmath}\label{eqn:qmLecture10:220}
\lr{ \Hbar c k_2 }^2
=
\lr{ E - V_0 }^2 - \lr{ m c^2}^2
=
\mathLabelBox
[ labelstyle={below of=m\themathLableNode, below of=m\themathLableNode} ]
{\lr{ E - V_0 + m c^2 }}{\(r_1\)}
\mathLabelBox
[ labelstyle={below of=m\themathLableNode, below of=m\themathLableNode} ]
{\lr{ E - V_0 - m c^2 }}{\(r_2\)}.
\end{dmath}
%
The \( r_1 \) and \( r_2 \) branches are sketched in \cref{fig:lecture10:lecture10Fig2}.

\imageFigure{../figures/phy1520-quantum/lecture10Fig2}{Energy signs.}{fig:lecture10:lecture10Fig2}{0.2}

For low energies, we have a set of potentials for which we will have propagation, despite having a potential barrier.  For still higher values of the potential barrier the product \( r_1 r_2 \) will be negative, so the solutions will be decaying.  Finally, for even higher energies, there will again be propagation.

The non-relativistic case is sketched in \cref{fig:lecture10:lecture10Fig3}.
\imageFigure{../figures/phy1520-quantum/lecture10Fig3}{Effects of increasing potential for non-relativistic case.}{fig:lecture10:lecture10Fig3}{0.1}
For the relativistic case we must consider three different cases, sketched in
\cref{fig:lecture10:lecture10Fig4a},
\cref{fig:lecture10:lecture10Fig4b}, and
\cref{fig:lecture10:lecture10Fig4c} respectively.  For the low potential energy, a particle with positive group velocity (what we've called right moving) can be matched to an equal energy portion of the potential shifted parabola in region II.  This is a case where we have transmission, but no antiparticle creation.  There will be an energy region where the region II wave function has only a dissipative term, since there is no region of either of the region II parabolic branches available at the incident energy.  When the potential is shifted still higher so that \( V_0 > E + m c^2 \), a positive group velocity in region I with a given energy can be matched to an antiparticle branch in the region II parabolic energy curve.

\imageFigure{../figures/phy1520-quantum/lecture10Fig4a}{Low potential energy.}{fig:lecture10:lecture10Fig4a}{0.1}
\imageFigure{../figures/phy1520-quantum/lecture10Fig4b}{High enough potential energy for no propagation.}{fig:lecture10:lecture10Fig4b}{0.1}
\imageFigure{../figures/phy1520-quantum/lecture10Fig4c}{High potential energy.}{fig:lecture10:lecture10Fig4c}{0.1}

\paragraph{Boundary value conditions}
\index{Dirac equation!boundary conditions}
We want to ensure that the current across the barrier is conserved (no particles are lost), as sketched in \cref{fig:lecture10:lecture10Fig5}.
\imageFigure{../figures/phy1520-quantum/lecture10Fig5}{Transmitted, reflected and incident components.}{fig:lecture10:lecture10Fig5}{0.1}
Recall that given a wave function
%
\begin{dmath}\label{eqn:qmLecture10:240}
\Psi =
\begin{bmatrix}
\psi_1 \\
\psi_2
\end{bmatrix},
\end{dmath}
%
the density and currents are respectively
%
\begin{equation}\label{eqn:qmLecture10:260}
\begin{aligned}
\rho &= \psi_1^\conj \psi_1 + \psi_2^\conj \psi_2 \\
j &= \psi_1^\conj \psi_1 - \psi_2^\conj \psi_2.
\end{aligned}
\end{equation}
Matching boundary value conditions requires
\begin{enumerate}
\item For both the relativistic and non-relativistic cases we must have
%
\begin{equation}\label{eqn:qmLecture10:280}
\Psi_\txtL = \Psi_\txtR, \qquad \mbox{at \( x = 0 \).}
\end{equation}
\item For the non-relativistic case we want
\begin{equation}\label{eqn:qmLecture10:300}
\int_{-\epsilon}^\epsilon -\frac{\Hbar^2}{2m} \PDSq{x}{\Psi} =
\cancel{\int_{-\epsilon}^\epsilon \lr{ E - V(x) } \Psi(x)}
\end{equation}
%
\begin{equation}\label{eqn:qmLecture10:320}
-\frac{\Hbar^2}{2m} \lr{ \evalbar{\PD{x}{\Psi}}{\txtR} - \evalbar{\PD{x}{\Psi}}{\txtL} }  = 0.
\end{equation}
%
%We have to match
For the relativistic case
%
\begin{equation}\label{eqn:qmLecture10:460}
-i \Hbar \sigma_z \int_{-\epsilon}^\epsilon \PD{x}{\Psi} +
\cancel{m c^2 \sigma_x \int_{-\epsilon}^\epsilon \psi}
=
\cancel{\int_{-\epsilon}^\epsilon \lr{ E - V_0 } \psi},
\end{equation}
\end{enumerate}
so
%
\begin{equation}\label{eqn:qmLecture10:340}
-i \Hbar c \sigma_z \lr{ \psi(\epsilon) - \psi(-\epsilon) }
=
-i \Hbar c \sigma_z \lr{ \psi_\txtR - \psi_\txtL }.
\end{equation}
%
so we must match
%
\begin{equation}\label{eqn:qmLecture10:360}
\sigma_z \psi_\txtR = \sigma_z \psi_\txtL .
\end{equation}
%
It appears that things are simpler, because we only have to match the wave function values at the boundary, and don't have to match the derivatives too.  However, we have a two component wave function, so there are still two tasks.

\paragraph{Solving the system}
Let's look for a solution for the \( E + m c^2 > V_0 \) case on the right branch, as sketched in \cref{fig:lecture10:lecture10Fig6}.
\imageFigure{../figures/phy1520-quantum/lecture10Fig6}{High potential region.  Anti-particle transmission.}{fig:lecture10:lecture10Fig6}{0.15}
While the right branch in this case is left going, this might work out since that is an antiparticle.  We could try both.
First
%
\begin{dmath}\label{eqn:qmLecture10:480}
\Psi_{II} = D e^{i k_2 x}
\begin{bmatrix}
-\sin\theta_2 \\
\cos\theta_2
\end{bmatrix}.
\end{dmath}
%
This is justified by
%
\begin{dmath}\label{eqn:qmLecture10:500}
+E \rightarrow
\begin{bmatrix}
\cos\theta \\
\sin\theta
\end{bmatrix},
\end{dmath}
%
so
%
\begin{dmath}\label{eqn:qmLecture10:520}
-E \rightarrow
\begin{bmatrix}
-\sin\theta \\
\cos\theta \\
\end{bmatrix}.
\end{dmath}
At \( x = 0 \) the exponentials vanish, so equating the waves at that point means
%
\begin{dmath}\label{eqn:qmLecture10:380}
\begin{bmatrix}
\cos\theta_1 \\
\sin\theta_1 \\
\end{bmatrix}
+
\frac{B}{A}
\begin{bmatrix}
\sin\theta_1 \\
\cos\theta_1 \\
\end{bmatrix}
=
\frac{D}{A}
\begin{bmatrix}
-\sin\theta_2 \\
\cos\theta_2
\end{bmatrix}.
\end{dmath}
%
Solving this yields
%
\begin{equation}\label{eqn:qmLecture10:400}
\frac{B}{A} = - \frac{\cos(\theta_1 - \theta_2)}{\sin(\theta_1 + \theta_2)}.
\end{equation}
%
This yields
%
\boxedEquation{eqn:qmLecture10:420}{
R = \frac{1 + \cos( 2 \theta_1 - 2 \theta_2) }{1 - \cos( 2 \theta_1 - 2 \theta_2)}.
}
As \( V_0 \rightarrow \infty \) this simplifies to
%
\begin{equation}\label{eqn:qmLecture10:440}
R = \frac{ E - \sqrt{ E^2 - \lr{ m c^2 }^2 } }{ E + \sqrt{ E^2 - \lr{ m c^2 }^2 } }.
\end{equation}
%
Filling in the details for these results is left to problem set 4.
%\EndNoBibArticle

      \section{Problems.}
         %
% Copyright © 2015 Peeter Joot.  All Rights Reserved.
% Licenced as described in the file LICENSE under the root directory of this GIT repository.
%
%
\makeproblem{Calculate the right going diagonalization.}{problem:qmLecture9Problems:1}{
\index{Dirac equation!right going solution}
%
\makesubproblem{}{problem:qmLecture9Problems:1:a}
%
Prove \cref{eqn:qmLecture9:160}.
%
% \makesubproblem{}{problem:qmLecture9Problems:1:b}
%
% Do the same thing for a left going wave solution.
%
} % problem
%
\makeanswer{problem:qmLecture9Problems:1}{
%
\makeSubAnswer{}{problem:qmLecture9Problems:1:a}
To determine the relations for \( \theta_k \) we have to solve
%
\begin{dmath}\label{eqn:qmLecture9Problems:280}
\begin{bmatrix}
E_k & 0 \\
0 & -E_k
\end{bmatrix}
= R^{-1} H R.
\end{dmath}
%
Working with \( \Hbar = c = 1 \) temporarily, and \( C = \cos\theta_k, S = \sin\theta_k \), that is
%
\begin{dmath}\label{eqn:qmLecture9Problems:300}
\begin{bmatrix}
E_k & 0 \\
0 & -E_k
\end{bmatrix}
=
\begin{bmatrix}
C & S \\
-S & C
\end{bmatrix}
\begin{bmatrix}
k & m \\
m & -k
\end{bmatrix}
\begin{bmatrix}
C & -S \\
S & C
\end{bmatrix}
=
\begin{bmatrix}
C & S \\
-S & C
\end{bmatrix}
\begin{bmatrix}
k C + m S & -k S + m C \\
m C - k S & -m S - k C
\end{bmatrix}
=
\begin{bmatrix}
k C^2 + m S C + m C S - k S^2   & -k S C + m C^2 -m S^2 - k C S \\
-k C S - m S^2 + m C^2 - k S C & k S^2 - m C S -m S C - k C^2
\end{bmatrix}
=
\begin{bmatrix}
k \cos(2 \theta_k) + m \sin(2 \theta_k) & m \cos(2 \theta_k) - k \sin(2 \theta_k) \\
m \cos(2 \theta_k) - k \sin(2 \theta_k) & -k \cos(2 \theta_k) - m \sin(2 \theta_k) \\
\end{bmatrix}.
\end{dmath}
%
This gives
%
\begin{dmath}\label{eqn:qmLecture9Problems:320}
E_k
\begin{bmatrix}
1 \\
0
\end{bmatrix}
=
\begin{bmatrix}
k \cos(2 \theta_k) + m \sin(2 \theta_k) \\
m \cos(2 \theta_k) - k \sin(2 \theta_k) \\
\end{bmatrix}
=
\begin{bmatrix}
k & m \\
m & -k
\end{bmatrix}
\begin{bmatrix}
\cos(2 \theta_k) \\
\sin(2 \theta_k) \\
\end{bmatrix}.
\end{dmath}
%
Adding back in the \(\Hbar\)'s and \(c\)'s this is
%
\begin{dmath}\label{eqn:qmLecture9Problems:340}
\begin{bmatrix}
\cos(2 \theta_k) \\
\sin(2 \theta_k) \\
\end{bmatrix}
=
\frac{E_k}{-(\Hbar k c)^2 -(m c^2)^2}
\begin{bmatrix}
- \Hbar k c & - m c^2 \\
- m c^2     & \Hbar k c
\end{bmatrix}
\begin{bmatrix}
1 \\
0
\end{bmatrix}
=
\inv{E_k}
\begin{bmatrix}
\Hbar k c \\
m c^2
\end{bmatrix}.
\end{dmath}
%
% \makeSubAnswer{}{problem:qmLecture9Problems:1:b}
%
% For a wave function of the form
%
% \begin{dmath}\label{eqn:qmLecture9Problems:440}
% \Phi =
% e^{-i k x}
% \begin{bmatrix}
% f_1 \\
% f_2
% \end{bmatrix},
% \end{dmath}
%
% the Dirac equation has the form
% \begin{dmath}\label{eqn:qmLecture9Problems:460}
% H \Phi =
% \begin{bmatrix}
% \hatp c & m c^2 \\
% m c^2 & - \hatp c
% \end{bmatrix}
% e^{-i k x}
% \begin{bmatrix}
% f_1 \\
% f_2
% \end{bmatrix}
% =
% \begin{bmatrix}
% -\Hbar k c & m c^2 \\
% m c^2 & \Hbar k c
% \end{bmatrix}
% \begin{bmatrix}
% f_1 \\
% f_2
% \end{bmatrix}.
% \end{dmath}
%
% Again introducing
%
% \begin{dmath}\label{eqn:qmLecture9Problems:480}
% \begin{bmatrix}
% f_1 \\
% f_2
% \end{bmatrix}
% =
% R
% \begin{bmatrix}
% f_{+} \\
% f_{-} \\
% \end{bmatrix},
% \end{dmath}
%
% so that
%
% \begin{dmath}\label{eqn:qmLecture9Problems:500}
% H R
% \begin{bmatrix}
% f_{+} \\
% f_{-} \\
% \end{bmatrix}
% =
% \begin{bmatrix}
% -\Hbar k c & m c^2 \\
% m c^2 & \Hbar k c
% \end{bmatrix}
% R
% \begin{bmatrix}
% f_{+} \\
% f_{-} \\
% \end{bmatrix}.
% \end{dmath}
%
% With
%
% \begin{dmath}\label{eqn:qmLecture9Problems:520}
% H_k
% =
% \begin{bmatrix}
% -\Hbar k c & m c^2 \\
% m c^2 & \Hbar k c
% \end{bmatrix},
% \end{dmath}
%
% We either wish to solve
% \begin{dmath}\label{eqn:qmLecture9Problems:540}
% \begin{bmatrix}
% E_k & 0 \\
% 0 & -E_k
% \end{bmatrix}
% =
% R^{-1} H_k R,
% \end{dmath}
%
% which was the approach used above (the hard way).  A better idea is to solve
%
% \begin{dmath}\label{eqn:qmLecture9Problems:560}
% R
% \begin{bmatrix}
% E_k & 0 \\
% 0 & -E_k
% \end{bmatrix}
% R^{-1}
% =
% H_k.
% \end{dmath}
%
% Expanding this we have
% \begin{dmath}\label{eqn:qmLecture9Problems:580}
% H_k
% =
% \begin{bmatrix}
% C & -S \\
% S & C
% \end{bmatrix}
% \begin{bmatrix}
% E_k & 0 \\
% 0 & -E_k
% \end{bmatrix}
% \begin{bmatrix}
% C & S \\
% -S & C
% \end{bmatrix}
% =
% E_k
% \begin{bmatrix}
% C & S \\
% S & -C
% \end{bmatrix}
% \begin{bmatrix}
% C & S \\
% -S & C
% \end{bmatrix}
% =
% \begin{bmatrix}
% C^2 - S^2 & 2 C S \\
% 2 C S & S^2 - C^2
% \end{bmatrix},
% \end{dmath}
%
% or
% \begin{dmath}\label{eqn:qmLecture9Problems:600}
% \begin{aligned}
% \cos(2 \theta_k) &= - \frac{\Hbar k c}{E_k} \\
% \sin(2 \theta_k) &= \frac{m c^2}{E_k}.
% \end{aligned}
% \end{dmath}
} % answer
%
\makeproblem{Verify the plane wave eigenstate.}{problem:qmLecture9Problems:3}{
%
\makesubproblem{}{problem:qmLecture9Problems:3:a}
%
Verify \cref{eqn:qmLecture9:261}.
%
\makesubproblem{}{problem:qmLecture9Problems:3:b}
%
Find the form of the reflected wave.
%
} % problem
%
\makeanswer{problem:qmLecture9Problems:3}{
%
\makeSubAnswer{}{problem:qmLecture9Problems:3:a}
With
%
\begin{dmath}\label{eqn:qmLecture9Problems:620}
H_k
=
\begin{bmatrix}
\Hbar k c & m c^2 \\
m c^2 & -\Hbar k c
\end{bmatrix}
=
R
\begin{bmatrix}
E_k & 0 \\
0 & -E_k
\end{bmatrix}
R^{-1}
\end{dmath},
%
We wish to show that
%
\begin{dmath}\label{eqn:qmLecture9Problems:640}
H_k
\begin{bmatrix}
\cos\theta_k \\
\sin\theta_k
\end{bmatrix}
e^{i k x}
=
E_k
\begin{bmatrix}
\cos\theta_k \\
\sin\theta_k
\end{bmatrix}
e^{i k x}.
\end{dmath}
%
The LHS side expands to
\begin{dmath}\label{eqn:qmLecture9Problems:660}
\begin{bmatrix}
C & - S \\
S & C
\end{bmatrix}
\begin{bmatrix}
E_k & 0 \\
0 & -E_k
\end{bmatrix}
\begin{bmatrix}
C & S \\
-S & C
\end{bmatrix}
\begin{bmatrix}
C \\
S \\
\end{bmatrix}
e^{i k x}
=
\begin{bmatrix}
C & - S \\
S & C
\end{bmatrix}
\begin{bmatrix}
E_k & 0 \\
0 & -E_k
\end{bmatrix}
\begin{bmatrix}
C^2 + S^2 \\
0 \\
\end{bmatrix}
e^{i k x}
=
\begin{bmatrix}
C & - S \\
S & C
\end{bmatrix}
\begin{bmatrix}
E_k \\
0
\end{bmatrix}
e^{i k x}
=
E_k
\begin{bmatrix}
C \\
S
\end{bmatrix}
e^{i k x}. \qquad \qedmarker
\end{dmath}
%
\makeSubAnswer{}{problem:qmLecture9Problems:3:b}
%
For the reflected wave, let's assume that the reflected wave has the form
%
\begin{dmath}\label{eqn:qmLecture9Problems:680}
\Psi =
\begin{bmatrix}
\sin\theta_k \\
-\cos\theta_k \\
\end{bmatrix}
e^{-i k x }.
\end{dmath}
%
Let's verify this
%
\begin{dmath}\label{eqn:qmLecture9Problems:700}
\begin{bmatrix}
C & - S \\
S & C
\end{bmatrix}
\begin{bmatrix}
E_k & 0 \\
0 & -E_k
\end{bmatrix}
\begin{bmatrix}
C & S \\
-S & C
\end{bmatrix}
\begin{bmatrix}
S \\
-C \\
\end{bmatrix}
e^{-i k x}
=
\begin{bmatrix}
C & - S \\
S & C
\end{bmatrix}
\begin{bmatrix}
E_k & 0 \\
0 & -E_k
\end{bmatrix}
\begin{bmatrix}
C S - S C \\
-S^2 - C^2
\end{bmatrix}
e^{-i k x}
=
\begin{bmatrix}
C & - S \\
S & C
\end{bmatrix}
\begin{bmatrix}
0 \\
E_k
\end{bmatrix}
e^{-i k x}
=
E_k
\begin{bmatrix}
- S \\
C
\end{bmatrix}
e^{-i k x}
=
-E_k
\begin{bmatrix}
S \\
-C
\end{bmatrix}
e^{-i k x}.
\end{dmath}
%
However, note that we have a different rotation angle \( \theta_k \) for the forward going and reverse going waves.

For the incident wave, \( k > 0 \), we have
%
\begin{equation}\label{eqn:qmLecture9Problems:720}
\tan 2 \theta_k = \frac{ \Hbar k c }{ m c^2 },
\end{equation}
%
and for the reflected wave, \( k < 0 \), we have
%
\begin{equation}\label{eqn:qmLecture9Problems:740}
\tan 2 \theta_k = \frac{ -\Hbar k c }{ m c^2 }.
\end{equation}
%
The rotation angle for both cases can therefore be expressed as
%
\begin{equation}\label{eqn:qmLecture9Problems:760}
\theta_k = \inv{2} \Atan\lr{ \Abs{\frac{\Hbar k }{m c}} }.
\end{equation}
%
} % answer
%
\makeproblem{Verify the Dirac current relationship.}{problem:qmLecture9Problems:2}{
\index{Dirac equation!current}
Prove \cref{eqn:qmLecture9:240}.
} % problem
%
\makeanswer{problem:qmLecture9Problems:2}{
%
The components of the Schr\"{o}dinger equation are
%
\begin{equation}\label{eqn:qmLecture9Problems:360}
\begin{aligned}
-i \Hbar \PD{t}{\psi_1} &= -i \Hbar c \PD{x}{\psi_1} + m c^2 \psi_2  \\
-i \Hbar \PD{t}{\psi_2} &= m c^2 \psi_1 + i \Hbar c \PD{x}{\psi_2},
\end{aligned}
\end{equation}

The conjugates of these are
\begin{equation}\label{eqn:qmLecture9Problems:380}
\begin{aligned}
i \Hbar \PD{t}{\psi_1^\conj} &= i \Hbar c \PD{x}{\psi_1^\conj} + m c^2 \psi_2^\conj \\
i \Hbar \PD{t}{\psi_2^\conj} &= m c^2 \psi_1^\conj - i \Hbar c \PD{x}{\psi_2^\conj}.
\end{aligned}
\end{equation}
%
This gives
\begin{dmath}\label{eqn:qmLecture9Problems:400}
\begin{aligned}
i \Hbar \PD{t}{\rho}
&=
\lr{ i \Hbar c \PD{x}{\psi_1^\conj} + m c^2 \psi_2^\conj } \psi_1 \\
&+ \psi_1^\conj \lr{ i \Hbar c \PD{x}{\psi_1} - m c^2 \psi_2 } \\
&+ \lr{ m c^2 \psi_1^\conj - i \Hbar c \PD{x}{\psi_2^\conj} } \psi_2 \\
&+ \psi_2^\conj \lr{ -m c^2 \psi_1 - i \Hbar c \PD{x}{\psi_2} }.
\end{aligned}
\end{dmath}
%
All the non-derivative terms cancel leaving
%
\begin{dmath}\label{eqn:qmLecture9Problems:420}
\inv{c} \PD{t}{\rho}
=
\PD{x}{\psi_1^\conj} \psi_1
+ \psi_1^\conj \PD{x}{\psi_1}
- \PD{x}{\psi_2^\conj} \psi_2
- \psi_2^\conj \PD{x}{\psi_2}
=
\PD{x}{}
\lr{
\psi_1^\conj \psi_1 -
\psi_2^\conj \psi_2
}.
\end{dmath}
%
} % answer

         %
% Copyright � 2015 Peeter Joot.  All Rights Reserved.
% Licenced as described in the file LICENSE under the root directory of this GIT repository.
%
\makeoproblem{Zitterbewegung in one dimension.}{gradQuantum:problemSet4:1}{2015 ps4 p1}{
\index{zitterbewegung}
Consider the Dirac Hamiltonian \( H = c \hatp \sigma_z + m c^2 \sigma_x \) . Using the Heisenberg equations of motion, derive a second order equation of motion (eom) for the velocity operator \( \hatv = d\hatx/dt \). For a state
%
\begin{dmath}\label{eqn:gradQuantumProblemSet4Problem1:20}
\Psi =
\begin{bmatrix}
1 \\
0
\end{bmatrix}
e^{i k x}
\end{dmath}

with \( k = 0 \), average this eom in state \( \Psi \) to get a homogeneous second order differential equation for \( \expectation{\hatv} \). Using this equation of motion and the initial conditions on the velocity and its time-derivative, obtain \( \expectation{\hatv}(t) \) and \( \expectation{\hatx(t)} \).
Show that these oscillate with a rapid frequency \( 2 m c^2/h \), with the oscillation of the position having an amplitude \( \Hbar/m c \) which is the Compton wavelength.  This `trembling' motion is called zitterbewegung.

%\makesubproblem{}{gradQuantum:problemSet4:1a}
} % makeproblem

\makeanswer{gradQuantum:problemSet4:1}{
\withproblemsetsParagraph{
%\makeSubAnswer{}{gradQuantum:problemSet4:1a}

Using the hint from class that zitterbewegung is associated with oscillation of the particle between ordinary-particle and anti-particle states, let's look for a combination of such \( k > 0 \) states to represent the state of this problem
%
\begin{dmath}\label{eqn:gradQuantumProblemSet4Problem1:40}
\begin{bmatrix}
1 \\
0
\end{bmatrix}
e^{i k x}
=
\evalbar{
\lr{
a
\begin{bmatrix}
\cos\theta \\
\sin\theta \\
\end{bmatrix}
e^{i k x - i E t/\Hbar}
+ b
\begin{bmatrix}
-\sin\theta \\
\cos\theta \\
\end{bmatrix}
e^{i k x + i E t/\Hbar}
}
}{t = 0}.
\end{dmath}

Observe that \( \begin{bmatrix}
-\sin\theta \\
\cos\theta \\
\end{bmatrix} \) and \( \begin{bmatrix}
\cos\theta \\
\sin\theta \\
\end{bmatrix} \) are orthogonal, so
%
\begin{equation}\label{eqn:gradQuantumProblemSet4Problem1:60}
a =
\begin{bmatrix}
1 & 0
\end{bmatrix}
\begin{bmatrix}
\cos\theta \\
\sin\theta \\
\end{bmatrix}
= \cos\theta.
\end{equation}
\begin{equation}\label{eqn:gradQuantumProblemSet4Problem1:80}
b =
\begin{bmatrix}
1 & 0
\end{bmatrix}
\begin{bmatrix}
-\sin\theta \\
\cos\theta \\
\end{bmatrix}
= -\sin\theta.
\end{equation}

With
%
\begin{equation}\label{eqn:gradQuantumProblemSet4Problem1:100}
\begin{aligned}
\ket{a} &=
\begin{bmatrix}
\cos\theta \\
\sin\theta \\
\end{bmatrix}
e^{i k x - i E t/\Hbar}
\\
\ket{b} &=
\begin{bmatrix}
-\sin\theta \\
\cos\theta \\
\end{bmatrix}
e^{i k x + i E t/\Hbar}
\end{aligned}
\end{equation}

The wave function of \cref{eqn:gradQuantumProblemSet4Problem1:20} can be represented as
%
\begin{dmath}\label{eqn:gradQuantumProblemSet4Problem1:120}
\Psi =
\cos\theta
\ket{a}
- \sin\theta
\ket{b}.
\end{dmath}

The action of the Hamiltonian on this wave function is
%
\begin{dmath}\label{eqn:gradQuantumProblemSet4Problem1:140}
H \psi
= i \Hbar \lr{ -i \frac{E}{\Hbar} C \ket{a} - i \frac{E}{\Hbar} S \ket{b} }
= E \lr{ C \ket{a} + S \ket{b} },
\end{dmath}
%
Writing \( \epsilon = \sqrt{ (m c^2)^2 + (\Hbar k c)^2 } \), and noting that \( \ket{a} \), and \( \ket{b} \) are positive and negative eigenstates respectively,
%using the diagonalization from class,
the spatial action of the Hamiltonian on this wave function is
\begin{dmath}\label{eqn:gradQuantumProblemSet4Problem1:160}
H \Psi
=
%\begin{bmatrix}
%\hatp c & m c^2 \\
%m c^2 & -\hatp c
%\end{bmatrix}
%\Psi
%=
%\begin{bmatrix}
%\Hbar k c & m c^2 \\
%m c^2 & -\Hbar k c
%\end{bmatrix}
%\Psi
%=
%\begin{bmatrix}
%C & -S \\
%S & C
%\end{bmatrix}
%\begin{bmatrix}
%\epsilon & 0 \\
%0 & -\epsilon
%\end{bmatrix}
%\begin{bmatrix}
%C & S \\
%-S & C
%\end{bmatrix}
%\Psi
%=
%\begin{bmatrix}
%C & -S \\
%S & C
%\end{bmatrix}
%\begin{bmatrix}
%\epsilon & 0 \\
%0 & -\epsilon
%\end{bmatrix}
%\begin{bmatrix}
%C & S \\
%-S & C
%\end{bmatrix}
%\lr{
%C
%\begin{bmatrix}
%C \\
%S
%\end{bmatrix}
%e^{i k x - i E t/\Hbar}
%-S
%\begin{bmatrix}
%-S \\
%C \\
%\end{bmatrix}
%e^{i k x + i E t/\Hbar}
%}
%=
%\begin{bmatrix}
%C & -S \\
%S & C
%\end{bmatrix}
%\begin{bmatrix}
%\epsilon & 0 \\
%0 & -\epsilon
%\end{bmatrix}
%\lr{
%C
%\begin{bmatrix}
%C^2 + S^2 \\
%-S C + C S
%\end{bmatrix}
%e^{i k x - i E t/\Hbar}
%-S
%\begin{bmatrix}
%-S C + S C \\
%S^2 + C^2
%\end{bmatrix}
%e^{i k x + i E t/\Hbar}
%}
%=
%\begin{bmatrix}
%C & -S \\
%S & C
%\end{bmatrix}
%\lr{
%C \epsilon
%\begin{bmatrix}
%1 \\
%0
%\end{bmatrix}
%e^{i k x - i E t/\Hbar}
%+ S \epsilon
%\begin{bmatrix}
%0 \\
%1
%\end{bmatrix}
%e^{i k x + i E t/\Hbar}
%}
%=
%\lr{
%C \epsilon
%\begin{bmatrix}
%C \\
%S
%\end{bmatrix}
%e^{i k x - i E t/\Hbar}
%+S \epsilon
%\begin{bmatrix}
%-S \\
%C
%\end{bmatrix}
%e^{i k x + i E t/\Hbar}
%}
= (+\epsilon) C \ket{a} - (-\epsilon) S \ket{b}
= \epsilon \lr{ C \ket{a} + S\ket{b} }.
\end{dmath}

This provides a relationship between the energy \( E \) and the eigenvalues
%
\begin{equation}\label{eqn:gradQuantumProblemSet4Problem1:180}
E = \epsilon = \sqrt{ (m c^2)^2 + (\Hbar k c)^2 }.
\end{equation}

In particular, when \( k = 0 \), this is \( E = m c^2 \).

Using the Heisenberg equations of motion, the velocity operator is
%
\begin{dmath}\label{eqn:gradQuantumProblemSet4Problem1:200}
\hatv I
=
\ddt{\hatx} I
= \inv{i\Hbar }\antisymmetric{\hatx I}{ \hatp c \sigma_z + m c^2 \sigma_x }
=
\inv{i\Hbar }
\lr{
\hatx \hatp c \sigma_z - \hatp c \sigma_z \hatx
+ \cancel{\hatx m c^2 \sigma_x} - \cancel{m c^2 \sigma_x \hatx}
}
=
\inv{i\Hbar }
c \sigma_z \antisymmetric{\hatx}{\hatp}
=
c \sigma_z.
\end{dmath}

As we've seen with the Harmonic oscillator, an expectation with respect to a state that is not a single eigenstate, will have non-zero time dependence.  With
\( C = \cos\theta \), \( S = \sin\theta \)
that expectation value with respect to the state as expressed in \cref{eqn:gradQuantumProblemSet4Problem1:120} is
%
\begin{dmath}\label{eqn:gradQuantumProblemSet4Problem1:220}
\expectation{\hatv}
=
\bra{\Psi} \hatv \ket{\Psi}
=
c \lr{ C \bra{a} - S \bra{b} } \sigma_z \lr{ C \ket{a} - S \ket{b} }
=
c \lr{
C \bra{a} - S \bra{b} }
\begin{bmatrix}
1 & 0 \\
0 & -1
\end{bmatrix}
\lr{
\begin{bmatrix}
C \\
S
\end{bmatrix}
e^{i k x - i E t/\Hbar}
- S
\begin{bmatrix}
-S \\
C
\end{bmatrix}
e^{i k x + i E t/\Hbar}
}
=
c
\lr{
C
\begin{bmatrix}
C &
S
\end{bmatrix}
e^{-i k x + i E t/\Hbar}
- S
\begin{bmatrix}
-S & C
\end{bmatrix}
e^{-i k x - i E t/\Hbar}
}
\lr{
C
\begin{bmatrix}
C \\
-S
\end{bmatrix}
e^{i k x - i E t/\Hbar}
+ S
\begin{bmatrix}
S \\
C
\end{bmatrix}
e^{i k x + i E t/\Hbar}
}
= c \lr{
C^2 \cos(2\theta)
- S^2 \cos( 2 \theta )
+ S C \sin( 2 \theta ) e^{-2 i E t/\Hbar}
+ S C \sin( 2 \theta ) e^{2 i E t/\Hbar}
}
=
c \lr{ \cos^2( 2 \theta ) + \sin^2( 2 \theta ) \cos(2 E t/\Hbar) }
=
c \lr{
\frac{ (\Hbar c k)^2 }{ (m c^2)^2 + (\Hbar k c)^2 }
+\frac{ (m c^2)^2 }{ (m c^2)^2 + (\Hbar k c)^2 } \cos\lr{ 2 \pi \frac{2 E}{h} t }
}
=
\frac{c}{ (m c^2)^2 + (\Hbar k c)^2 } \lr{
(\Hbar c k)^2
+
(m c^2)^2
\cos\lr{ 2 \pi \frac{2 E}{h} t }
}.
\end{dmath}

For \( k = 0 \), this is
%
\boxedEquation{eqn:gradQuantumProblemSet4Problem1:240}{
\expectation{\hatv} = c \cos\lr{ 2 \pi \frac{2 m c^2}{h} t }.
}

The frequency of this oscillation is \( \ifrac{2 m c^2}{h} \) as the problem states.  For the position expectation with respect to this state, we have
%
\begin{dmath}\label{eqn:gradQuantumProblemSet4Problem1:260}
\expectation{\hatx}
= \expectation{\hatx}(0) + \frac{c \Hbar}{2 m c^2} \sin \lr{ \frac{2 m c^2}{\Hbar} t }
= \expectation{\hatx}(0) + \frac{\Hbar}{2 m c} \sin \lr{ \frac{2 m c^2}{\Hbar} t }.
\end{dmath}

The amplitude of this oscillation is
%
\begin{dmath}\label{eqn:gradQuantumProblemSet4Problem1:280}
2 \times \frac{\Hbar}{2 m c}
=
\frac{\Hbar}{m c},
\end{dmath}
%
as expected.
}
}

         %
% Copyright � 2015 Peeter Joot.  All Rights Reserved.
% Licenced as described in the file LICENSE under the root directory of this GIT repository.
%
\makeoproblem{Jackiw-Rebbi problem.}{gradQuantum:problemSet4:2}{2015 ps4 p2}{
\index{Jackiw-Rebbi problem}

Recall that the energy of a relativistic particle is \( E(p) = \sqrt{p^2 c^2 + m^2 c^4 } \), which is independent of the sign of \( m \).
Thus \( m > 0 \) and \( m < 0 \) lead to the same dispersion relation.
Set aside for now, the physical meaning of \( m < 0 \).
Assume \( V(x) = 0 \) but let us assume the mass \( m \) is a function of position \( m(x) \).
This leads to
%
\begin{dmath}\label{eqn:gradQuantumProblemSet4Problem2:20}
H =
\begin{bmatrix}
c \hatp & m(x) c^2 \\
m(x) c^2 & -c \hatp
\end{bmatrix}
.
\end{dmath}
%
Let \( m(x) \) be such that \( m(-x) = -m(x) \), i.e., an odd function of position which changes sign at \( x = 0 \).

\makesubproblem{}{gradQuantum:problemSet4:2a}
Show that the operator \( \hatcalP_{\textrm{Dirac}} = \sigma_y \hatcalP \) commutes with the Hamiltonian, where
%
\begin{dmath}\label{eqn:gradQuantumProblemSet4Problem2:40}
\sigma_y
=
\PauliY
\end{dmath}

is the y-Pauli matrix, and \( \hatcalP \) is the parity operator which sends \( x \rightarrow -x\).
\makesubproblem{}{gradQuantum:problemSet4:2b}
Consider the wavefunction
%
\begin{dmath}\label{eqn:gradQuantumProblemSet4Problem2:60}
\Phi(x) =
\begin{bmatrix}
f(x) \\
- i f(x)
\end{bmatrix},
\end{dmath}
%
where \( f(-x) = f(x) \) is an even function.  Show \( \Phi(x) \) is an eigenstate of \( \hatcalP_{\textrm{Dirac}} \) with eigenvalue \( -1 \).

\makesubproblem{}{gradQuantum:problemSet4:2c}
Next, assuming \( m(x > 0) = m_0 \) and \( m(x < 0) = -m_0 \) , where \( m_0 > 0 \), find \( f(x) \) such that \( \Phi(x) \) is an eigenstate of H with zero energy.

\makesubproblem{}{gradQuantum:problemSet4:2d}
Normalize the wavefunction \( \Phi(x) \).

} % makeproblem

\makeanswer{gradQuantum:problemSet4:2}{
\withproblemsetsParagraph{
\makeSubAnswer{}{gradQuantum:problemSet4:2a}

Let \( \pi \) represent a parity operator that acts on a scalar (operator), so the Dirac parity operator \( \hatcalP_{\textrm{Dirac}} \) takes the form
%
\begin{dmath}\label{eqn:gradQuantumProblemSet4Problem2:80}
\hatcalP_{\textrm{Dirac}} =
\begin{bmatrix}
0 & -i \pi \\
i \pi & 0
\end{bmatrix}.
\end{dmath}
%
The commutator with the Hamiltonian is
%
\begin{dmath}\label{eqn:gradQuantumProblemSet4Problem2:100}
\antisymmetric{H}{\hatcalP_{\textrm{Dirac}}}
= H \hatcalP_{\textrm{Dirac}} - \hatcalP_{\textrm{Dirac}} H
=
\begin{bmatrix}
c \hatp & m(x) c^2 \\
m(x) c^2 & -c \hatp
\end{bmatrix}
\begin{bmatrix}
0 & -i \pi \\
i \pi & 0
\end{bmatrix}
-
\begin{bmatrix}
0 & -i \pi \\
i \pi & 0
\end{bmatrix}
\begin{bmatrix}
c \hatp & m(x) c^2 \\
m(x) c^2 & -c \hatp
\end{bmatrix}
=
\begin{bmatrix}
i c^2 m(x) \pi & -i c \hatp \pi \\
-i c \hatp \pi & - i c^2 m(x) \pi
\end{bmatrix}
-
\begin{bmatrix}
-i c^2 \pi m(x) & i c \pi \hatp \\
i c \pi \hatp & i c^2 \pi m(x)
\end{bmatrix}
=
\begin{bmatrix}
i c^2 \symmetric{m(x)}{ \pi } & -i c \symmetric{\hatp}{\pi} \\
-i c \symmetric{\hatp}{\pi} & - i c^2 \symmetric{m(x)}{\pi}
\end{bmatrix}.
\end{dmath}
%
Now consider the matrix element of this commutator with respect to a position basis wavefunction \( \bra{x} \antisymmetric{H}{\hatcalP_{\textrm{Dirac}}} \ket{\Psi} \).  To compute this we must first understand the behaviour of the anticommutators of scalar operators \( m, \hatp \) with \( \pi \).  That is
%
\begin{dmath}\label{eqn:gradQuantumProblemSet4Problem2:120}
\bra{x} \symmetric{m(x)}{\pi} \ket{\psi}
=
\symmetric{m(x)}{\pi} \psi(x)
=
m(x) \pi \psi(x) + \pi ( m(x) \psi(x) )
=
m(x) \psi(-x) + m(-x) \psi(-x)
=
m(x) \psi(-x) - m(x) \psi(-x)
=
0,
\end{dmath}
%
and
%
\begin{dmath}\label{eqn:gradQuantumProblemSet4Problem2:140}
\bra{x} \symmetric{\hatp}{\pi} \ket{\psi}
=
\symmetric{-i \Hbar \PD{x}{} }{\pi} \psi(x)
=
-i \Hbar \PD{x}{} \pi \psi(x) + \pi ( -i \Hbar \PD{x}{} \psi(x) )
=
-i \Hbar \PD{x}{} \psi(-x) + (-i \Hbar) \PD{(-x)}{} \psi(-x)
=
i \Hbar \PD{(-x)}{} \psi(-x) - i \Hbar \PD{(-x)}{} \psi(-x)
=
0.
\end{dmath}
%
This shows that \( \bra{x} \antisymmetric{H}{\hatcalP_{\textrm{Dirac}}} \ket{\Psi} = 0 \).  Since this is true for any \( \ket{\Psi} \), we've shown that the Hamiltonian commutes with the Dirac parity operator
%
\boxedEquation{eqn:gradQuantumProblemSet4Problem2:160}{
H \hatcalP_{\textrm{Dirac}} = \hatcalP_{\textrm{Dirac}} H.
}

\makeSubAnswer{}{gradQuantum:problemSet4:2b}

This follows with direct substitution
%
\begin{dmath}\label{eqn:gradQuantumProblemSet4Problem2:180}
\hatcalP_{\textrm{Dirac}} \Phi(x)
=
\begin{bmatrix}
0 & -i \pi \\
i \pi & 0
\end{bmatrix}
\begin{bmatrix}
f(x) \\
-i f(x)
\end{bmatrix}
=
\begin{bmatrix}
- \pi f(x) \\
i \pi f(x)
\end{bmatrix}
=
\begin{bmatrix}
- f(-x) \\
i f(-x)
\end{bmatrix}
=
-
\begin{bmatrix}
f(x) \\
-i f(x)
\end{bmatrix}.
\end{dmath}
%
\makeSubAnswer{}{gradQuantum:problemSet4:2c}

The eigenvalue equation has the form
%
\begin{dmath}\label{eqn:gradQuantumProblemSet4Problem2:380}
\lambda \Phi
= H \Phi
=
\begin{bmatrix}
c \hatp & m(x) c^2 \\
m(x) c^2 & -c \hatp
\end{bmatrix}
\begin{bmatrix}
f \\
-i f
\end{bmatrix}
=
\begin{bmatrix}
c \hatp f -i m(x) c^2 f \\
m(x) c^2 f + i c \hatp f,
\end{bmatrix}
\end{dmath}

or
\begin{dmath}\label{eqn:gradQuantumProblemSet4Problem2:200}
\begin{aligned}
c (-i \Hbar) \PD{x}{f} -i m(x) c^2 f &= \lambda f \\
i c (-i \Hbar) \PD{x}{f} + m(x) c^2 f &= -i \lambda f \\
\end{aligned}.
\end{dmath}
%
This special selection of \( \Phi \) leads to two identical equations, which should simplify things nicely.  For the zero energy eigenstate, we must solve
%
\begin{dmath}\label{eqn:gradQuantumProblemSet4Problem2:220}
\PD{x}{f} = - \inv{c \Hbar} m(x) c^2 f,
\end{dmath}
%
which integrates to
%
\begin{dmath}\label{eqn:gradQuantumProblemSet4Problem2:240}
\ln f
= \ln f(0) - \int_0^x \inv{c \Hbar} m(x) c^2 dx,
\end{dmath}
%
or
%
\begin{dmath}\label{eqn:gradQuantumProblemSet4Problem2:260}
f(x) = f(0) \exp\lr{ - \int_0^x \frac{c}{\Hbar} m(x) dx }.
\end{dmath}
%
For \( x > 0 \) this is
%
\begin{dmath}\label{eqn:gradQuantumProblemSet4Problem2:280}
f(x)
= f(0) \exp\lr{ - \int_0^x \frac{c}{\Hbar} m_0 dx }
= f(0) \exp\lr{ - \frac{m_0 c x}{\Hbar}},
\end{dmath}
%
and for \( x < 0 \) this is
\begin{dmath}\label{eqn:gradQuantumProblemSet4Problem2:300}
f(x)
= f(0) \exp\lr{ \int_0^x \frac{c}{\Hbar} m_0 dx }
= f(0) \exp\lr{ \frac{m_0 c x}{\Hbar}}.
\end{dmath}
%
These can be combined as
%
\begin{dmath}\label{eqn:gradQuantumProblemSet4Problem2:320}
f(x) = f(0) \exp\lr{ - \frac{m_0 c \Abs{x}}{\Hbar}}.
\end{dmath}
%
\makeSubAnswer{}{gradQuantum:problemSet4:2d}

The normalization is given by
%
\begin{dmath}\label{eqn:gradQuantumProblemSet4Problem2:340}
1
=
\int_{-\infty}^\infty \Phi^\dagger \Phi dx
=
\int_{-\infty}^\infty 2 \Abs{f}^2 dx
=
2 f^2(0) \int_{-\infty}^\infty e^{- 2 m_0 c \Abs{x}/\Hbar} dx
=
4 f^2(0) \int_{0}^\infty e^{- 2 m_0 c x/\Hbar} dx
=
4 f^2(0) \evalrange{\frac{e^{- 2 m_0 c x/\Hbar}}{ - 2 m_0 c/\Hbar}}{0}{\infty}
=
\frac{2 \Hbar}{ m_0 c} f^2(0),
\end{dmath}
%
or
%
\begin{dmath}\label{eqn:gradQuantumProblemSet4Problem2:360}
f(0) = \sqrt{ \frac{ m_0 c}{2 \Hbar} }.
\end{dmath}
%
The fully normalized wave function is therefore
%
\boxedEquation{eqn:gradQuantumProblemSet4Problem2:400}{
\Phi(x) = \sqrt{ \frac{ m_0 c}{2 \Hbar} } e^{ - m_0 c \Abs{x}/\Hbar}
\begin{bmatrix}
1 \\
-i
\end{bmatrix}.
}
}
}

         %
% Copyright � 2015 Peeter Joot.  All Rights Reserved.
% Licenced as described in the file LICENSE under the root directory of this GIT repository.
%
\makeoproblem{Scattering off a potential step.}{gradQuantum:problemSet4:3}{2015 ps4 p3}{
\index{Dirac equation!scattering}
\index{Dirac equation!potential step}

Consider the 1D Dirac Hamiltonian as
%
\begin{dmath}\label{eqn:gradQuantumProblemSet4Problem3:21}
H =
\begin{bmatrix}
c \hatp + V(x) & m c^2 \\
m c^2 & - c \hatp + V(x)
\end{bmatrix},
\end{dmath}
where the operator \( \hatp = - i \Hbar \PDi{x}{} \) is the rest mass, and \( c \) is the speed of light, and with 2-component wavefunctions
%
\begin{dmath}\label{eqn:gradQuantumProblemSet4Problem3:41}
\Psi(x,t) \equiv
\begin{bmatrix}
\psi_1(x, t) \\
\psi_2(x, t) \\
\end{bmatrix},
\end{dmath}
such that \( i \Hbar \PDi{t}{\Psi(x,t)} = H \Psi(x, t) \).  Assuming a potential step, where \( V(x < 0) = 0 \) and \( V(x > 0) = V_0 \), with \( V_0 > 0 \) as in class, complete the details of the scattering onto the step which was done in class. Discuss the incident current, reflected current, and transmitted current for the case where the incident energy is such that \( E > 0 \) and \( V_0 > 2 m c^2 \) , and \( E > 0 \) and \( V_0 < 2 m c^2 \). Draw pictures to illustrate the parabola and the location in momentum of the incident and reflected particles.
%
%\makesubproblem{}{gradQuantum:problemSet4:3a}
} % makeproblem
%
\makeanswer{gradQuantum:problemSet4:3}{
\withproblemsetsParagraph{
%\makeSubAnswer{}{gradQuantum:problemSet4:3a}
%
%
% Copyright © 2016 Peeter Joot.  All Rights Reserved.
% Licenced as described in the file LICENSE under the root directory of this GIT repository.
%

\paragraph{Mostly background.}
% (really for myself).  Grading can probably skip to around \cref{eqn:gradQuantumProblemSet4Problem3:560}

We talked about diagonalization of the Dirac Hamiltonian by introducing a rotation, and then figuring out the rotation angle required.

To understand the form form of the eigenkets for particles and antiparticles in both forward and backwards moving configurations, lets do this diagonalization explicitly for both forwards and backwards solutions.

For the forward solution, given \( \Psi = \Psi_0 e^{i(k x - E t/\Hbar) } \), the Dirac equation is

\begin{dmath}\label{eqn:gradQuantumProblemSet4Problem3:20}
i \Hbar (-i E/\Hbar) \Psi =
\begin{bmatrix}
-i \Hbar (i k) + V_0 & m c^2 \\
m c^2 & i \Hbar (i k)
\end{bmatrix},
\Psi
\end{dmath}

or
\begin{dmath}\label{eqn:gradQuantumProblemSet4Problem3:40}
\begin{bmatrix}
E - V_0 & 0 \\
0 & E - V_0
\end{bmatrix}
\Psi
=
\begin{bmatrix}
\Hbar k & m c^2 \\
m c^2 &  - \Hbar k
\end{bmatrix}
\Psi.
\end{dmath}

Similarly, for the backwards moving wave \( \Psi = e^{i(-k x - E t/\Hbar)} \), we have

\begin{dmath}\label{eqn:gradQuantumProblemSet4Problem3:60}
\begin{bmatrix}
E - V_0 & 0 \\
0 & E - V_0
\end{bmatrix}
\Psi
=
\begin{bmatrix}
-\Hbar k & m c^2 \\
m c^2 & \Hbar k
\end{bmatrix}
\Psi.
\end{dmath}

Working with \( \Hbar = c = 1 \) temporarily, we want to compute the eigensolutions for the matrix

\begin{dmath}\label{eqn:gradQuantumProblemSet4Problem3:80}
H_{\pm k}
=
\begin{bmatrix}
\pm k & m \\
m & \mp k
\end{bmatrix}.
\end{dmath}

The eigenvalues \( \epsilon \) of both are the same

\begin{dmath}\label{eqn:gradQuantumProblemSet4Problem3:100}
0
=
\Abs{ H_{\pm k} - \epsilon }
=
(\pm k - \epsilon)(\mp k - \epsilon) - m^2
=
(\mp k + \epsilon)(\pm k + \epsilon) - m^2
=
\epsilon^2 - k^2 - m^2,
\end{dmath}

or

\begin{dmath}\label{eqn:gradQuantumProblemSet4Problem3:120}
\epsilon = \pm \sqrt{k^2 + m^2}.
\end{dmath}

\paragraph{Eigenkets for \( H_k \)}

For the positive(negative) energy eigenvalues, we have

\begin{dmath}\label{eqn:gradQuantumProblemSet4Problem3:140}
0
=
\begin{bmatrix}
k \mp \epsilon & m \\
m & -k \mp \epsilon
\end{bmatrix}
\begin{bmatrix}
a \\
b
\end{bmatrix},
\end{dmath}

for some \( a, b\).  That is

\begin{dmath}\label{eqn:gradQuantumProblemSet4Problem3:160}
(k \mp \epsilon) a + m b = 0,
\end{dmath}

or

\begin{dmath}\label{eqn:gradQuantumProblemSet4Problem3:180}
\begin{bmatrix}
a \\
b
\end{bmatrix}
\propto
\begin{bmatrix}
- m \\
k \mp \epsilon
\end{bmatrix}.
\end{dmath}

For the normalization note that

\begin{dmath}\label{eqn:gradQuantumProblemSet4Problem3:200}
m^2 + \lr{ k \mp \epsilon }^2
=
m^2 + k^2 + \epsilon^2 \mp 2 k \epsilon
=
2 \epsilon^2 \mp 2 k \epsilon
=
2 \epsilon (\epsilon \mp k),
\end{dmath}

so the normalized kets are

\begin{dmath}\label{eqn:gradQuantumProblemSet4Problem3:220}
\ket{k; \pm\epsilon} =
\inv{\sqrt{2 \epsilon(\epsilon \mp k)}}
\begin{bmatrix}
\pm m \\
\epsilon \mp k
\end{bmatrix}.
\end{dmath}

\paragraph{Eigenkets for \( H_{-k} \)}

This time, for the positive(negative) energy eigenvalues, we have

\begin{dmath}\label{eqn:gradQuantumProblemSet4Problem3:141}
0
=
\begin{bmatrix}
-k \mp \epsilon & m \\
m & k \mp \epsilon
\end{bmatrix}
\begin{bmatrix}
a \\
b
\end{bmatrix},
\end{dmath}

for some \( a, b\).  That is

\begin{dmath}\label{eqn:gradQuantumProblemSet4Problem3:161}
(-k \mp \epsilon) a + m b = 0,
\end{dmath}

or

\begin{dmath}\label{eqn:gradQuantumProblemSet4Problem3:181}
\begin{bmatrix}
a \\
b
\end{bmatrix}
\propto
\begin{bmatrix}
- m \\
-k \mp \epsilon
\end{bmatrix}
\propto
\begin{bmatrix}
m \\
k \pm \epsilon
\end{bmatrix}.
\end{dmath}

For the normalization note that

\begin{dmath}\label{eqn:gradQuantumProblemSet4Problem3:201}
m^2 + \lr{ k \pm \epsilon }^2
=
m^2 + k^2 + \epsilon^2 \pm 2 k \epsilon
=
2 \epsilon^2 \pm 2 k \epsilon
=
2 \epsilon (\epsilon \pm k),
\end{dmath}

so the normalized kets are

\begin{dmath}\label{eqn:gradQuantumProblemSet4Problem3:221}
\ket{-k; \pm\epsilon} =
\inv{\sqrt{2 \epsilon(\epsilon \mp k)}}
\begin{bmatrix}
\pm m \\
\epsilon \pm k
\end{bmatrix}.
\end{dmath}

\paragraph{Simplification of the rotation matrices}

The eigenvalue equations have the form

\begin{dmath}\label{eqn:gradQuantumProblemSet4Problem3:240}
H
\begin{bmatrix}
\ket{+} & \ket{-}
\end{bmatrix}
=
\begin{bmatrix}
\ket{+} & \ket{-}
\end{bmatrix}
\begin{bmatrix}
\epsilon & 0 \\
0 & -\epsilon
\end{bmatrix}
\end{dmath}

With \( R = \begin{bmatrix} \ket{+} & \ket{-} \end{bmatrix} \), this has the form \( H R = R \Omega \), or \( H = R \Omega R^{-1} \).  The rotation matrices have been found to be

\begin{dmath}\label{eqn:gradQuantumProblemSet4Problem3:260}
R_{+k}
=
\inv{\sqrt{2\epsilon}}
\begin{bmatrix}
\frac{m}{\sqrt{\epsilon - k}} & \frac{-m}{\sqrt{\epsilon + k}} \\
\sqrt{\epsilon - k} & \sqrt{\epsilon + k}
\end{bmatrix},
\end{dmath}

and
\begin{dmath}\label{eqn:gradQuantumProblemSet4Problem3:280}
R_{-k}
=
\inv{\sqrt{2\epsilon}}
\begin{bmatrix}
\frac{m}{\sqrt{\epsilon + k}} & \frac{-m}{\sqrt{\epsilon - k}} \\
\sqrt{\epsilon + k} & \sqrt{\epsilon - k}
\end{bmatrix}.
\end{dmath}

These don't look very much like rotation matrices as is, but not all the terms are independent.  Writing

\begin{dmath}\label{eqn:gradQuantumProblemSet4Problem3:300}
\begin{aligned}
a &= \frac{m}{\sqrt{2\epsilon(\epsilon - k)}}  \\
b &= \frac{m}{\sqrt{2\epsilon(\epsilon + k)}} \\
\alpha &= \sqrt{\frac{\epsilon + k}{2 \epsilon}}  \\
\beta &= \sqrt{\frac{\epsilon - k}{2 \epsilon}}
\end{aligned},
\end{dmath}

the rotation matrices are

\begin{dmath}\label{eqn:gradQuantumProblemSet4Problem3:320}
R_{+k}
=
\begin{bmatrix}
a & - b \\
\beta & \alpha
\end{bmatrix},
\end{dmath}

and
\begin{dmath}\label{eqn:gradQuantumProblemSet4Problem3:340}
R_{-k}
=
\begin{bmatrix}
b & -a \\
\alpha & \beta
\end{bmatrix}.
\end{dmath}

Note that
\begin{equation}\label{eqn:gradQuantumProblemSet4Problem3:360}
\begin{aligned}
\frac{a}{\alpha} &= \frac{b}{\beta} \\
&= \frac{m}{\sqrt{\epsilon^2 - k^2}}  \\
&= \frac{m}{\sqrt{m^2}}  \\
&= 1.
\end{aligned}
\end{equation}

So
\begin{dmath}\label{eqn:gradQuantumProblemSet4Problem3:380}
R_{+k}
=
\begin{bmatrix}
a & - b \\
b & a
\end{bmatrix},
\end{dmath}

and
\begin{dmath}\label{eqn:gradQuantumProblemSet4Problem3:400}
R_{-k}
=
\begin{bmatrix}
b & -a \\
a & b
\end{bmatrix}.
\end{dmath}

These are expected to have a unit determinant, which is verified easily

\begin{dmath}\label{eqn:gradQuantumProblemSet4Problem3:420}
a^2 + b^2
=
\frac{m^2}{2 \epsilon}
\lr{
\frac{1}{\epsilon + k}
+
\frac{1}{\epsilon - k}
}
=
\frac{m^2}{2 \epsilon}
\frac{2 \epsilon}{\epsilon_2 - k^2}
= 1.
\end{dmath}

We see that both sets of matrices invert by transposition, so we are free to make a trigonometric identification

\begin{equation}\label{eqn:gradQuantumProblemSet4Problem3:440}
a = \cos\theta
=
\frac{m}{\sqrt{2\epsilon(\epsilon - k)}}
\end{equation}

\begin{equation}\label{eqn:gradQuantumProblemSet4Problem3:460}
b = \sin\theta
=
\frac{m}{\sqrt{2\epsilon(\epsilon + k)}}.
\end{equation}

Using the double angle formulation used in class these are

\begin{dmath}\label{eqn:gradQuantumProblemSet4Problem3:480}
\cos(2 \theta)
=
\frac{m^2}{2\epsilon(\epsilon - k)} -
\frac{m^2}{2\epsilon(\epsilon + k)}
=
\frac{m^2}{2 \epsilon} \lr{ \inv{\epsilon - k} - \inv{\epsilon + k} }
=
\frac{2 k m^2}{2 \epsilon (\epsilon^2 - k^2)}
=
\frac{k}{\epsilon},
\end{dmath}

and
\begin{dmath}\label{eqn:gradQuantumProblemSet4Problem3:500}
\sin(2 \theta)
=
2 \frac{m^2}{2 \epsilon} \inv{\sqrt{\epsilon^2 - k^2}}
=
\frac{m}{\epsilon}.
\end{dmath}

Putting back in the \( \Hbar \), and \( c\) factors, the diagonalizing transformation is now fully specified

\begin{equation}\label{eqn:gradQuantumProblemSet4Problem3:520}
\begin{aligned}
H_{\pm k} &=
\begin{bmatrix}
\pm \Hbar k c & m c^2 \\
m c^2 & \mp \Hbar k c
\end{bmatrix}
=
R_{\pm k} \Omega R_{\pm k}^\T \\
R_{+ k} &=
\begin{bmatrix}
\cos\theta & -\sin\theta  \\
\sin\theta & \cos\theta
\end{bmatrix}
=
\begin{bmatrix}
\ket{+k;+\epsilon} &
\ket{+k;-\epsilon}
\end{bmatrix}
\\
R_{- k} &=
\begin{bmatrix}
\sin\theta & -\cos\theta  \\
\cos\theta & \sin\theta
\end{bmatrix}
=
\begin{bmatrix}
\ket{-k;+\epsilon} &
\ket{-k;-\epsilon}
\end{bmatrix}
\\
\tan(2 \theta) &= \frac{m c}{\Hbar \Abs{k}} \\
\Omega &=
\begin{bmatrix}
\epsilon & 0 \\
0 & -\epsilon
\end{bmatrix} \\
\epsilon &= \sqrt{(\Hbar k c)^2 + (m c^2)^2}.
\end{aligned}
\end{equation}

The functions \( \Psi = \ket{\pm k; \pm \epsilon} e^{\pm i k x - i E t/\Hbar} \) are eigenfunctions of the Hamiltonian.  For example, for a non-antiparticle forward moving state

\begin{dmath}\label{eqn:gradQuantumProblemSet4Problem3:540}
(E - V_0) \ket{k; +\epsilon}
=
H_k \ket{k; +\epsilon}
=
\begin{bmatrix}
\ket{k;+\epsilon} &
\ket{k;-\epsilon}
\end{bmatrix}
\begin{bmatrix}
\epsilon & 0 \\
0 & -\epsilon
\end{bmatrix}
\begin{bmatrix}
\bra{k;+\epsilon} \\
\bra{k;-\epsilon}
\end{bmatrix}
\ket{k; \epsilon}
=
\begin{bmatrix}
\ket{k;+\epsilon} &
\ket{k;-\epsilon}
\end{bmatrix}
\begin{bmatrix}
\epsilon & 0 \\
0 & -\epsilon
\end{bmatrix}
\begin{bmatrix}
\braket{k;+\epsilon}{k; \epsilon} \\
\braket{k;-\epsilon}{k; \epsilon}
\end{bmatrix}
=
\begin{bmatrix}
\ket{k;+\epsilon} &
\ket{k;-\epsilon}
\end{bmatrix}
\begin{bmatrix}
\epsilon & 0 \\
0 & -\epsilon
\end{bmatrix}
\begin{bmatrix}
1 \\
0
\end{bmatrix}
=
\begin{bmatrix}
\ket{k;+\epsilon} &
\ket{k;-\epsilon}
\end{bmatrix}
\begin{bmatrix}
\epsilon \\
0
\end{bmatrix}
=
\epsilon
\ket{k;+\epsilon}.
\end{dmath}

\paragraph{Back to the core problem.}

The total energy for this particle is

\begin{dmath}\label{eqn:gradQuantumProblemSet4Problem3:560}
E = V_0 \pm \sqrt{(\Hbar k c)^2 + (m c^2)^2}.
\end{dmath}

This can be plotted nicely, as \( \frac{E}{m c^2} \) vs. \( \frac{\Hbar k}{m c} \) if non-dimensionalized as

\begin{dmath}\label{eqn:gradQuantumProblemSet4Problem3:580}
\frac{E}{m c^2} = \frac{V_0}{m c^2} + \sqrt{1 + \frac{(\Hbar k)^2}{(m c)^2} }.
\end{dmath}

For the step potential we have

\begin{dmath}\label{eqn:gradQuantumProblemSet4Problem3:600}
\frac{E}{m c^2}
= \sqrt{ 1 + {\frac{ \Hbar k_1}{m c} }^2 }
= \pm \sqrt{ 1 + {\frac{ \Hbar k_2}{m c} }^2 } + \frac{V_0}{m c^2},
\end{dmath}

or
\begin{dmath}\label{eqn:gradQuantumProblemSet4Problem3:620}
\frac{ \Hbar k_2}{m c}
=
\lr{ \lr{ \pm \sqrt{ 1 + {\frac{ \Hbar k_1}{m c} }^2 } - \frac{V_0}{m c^2} }^2 - 1 }^{1/2}.
\end{dmath}

When this is real valued, there is transmission in the barrier region.  There are a few cases of interest, the \( V_0 < 2 m c^2 \) cases are plotted in \cref{fig:ps4DiracStepPotential:ps4DiracStepPotentialFig1} showing the transmission (\( V_0 = 0.7 m c^2 \)) and decaying cases (\( V_0 = 1.4 m c^2 \)) respectively.
Three \( V_0 > 2 m c^2 \) cases are plotted in \cref{fig:ps4DiracStepPotential:ps4DiracStepPotentialFig3},
showing the
anti-particle transmission,
decaying solution,
and ordinary-particle transmission
(\( (V_0/m c^2,k \Hbar/m c) = (3.4,1.7), (2.6,1.9), (2.7,6.3) \)).  At low enough momentum there is antiparticle transmission, then reflection as momentum is increased, and finally at high enough momentum ordinary particle transmission.

\mathImageTwoFigures
{../figures/phy1520-quantum/ps4DiracStepPotentialFig2}
{../figures/phy1520-quantum/ps4DiracStepPotentialFig1}
{\( V_0 < 2 m c^2 \)}{fig:ps4DiracStepPotential:ps4DiracStepPotentialFig1}{scale=0.3}
{energyVsMomentumForDiracStepPotentialWaveFunctions.nb}
%%
% v_0, k
% 3.4, 1.7
% 2.6, 1.9
% 2.7, 6.3
\mathImageThreeFiguresOneLine
{../figures/phy1520-quantum/ps4DiracStepPotentialFig3}
{../figures/phy1520-quantum/ps4DiracStepPotentialFig4}
{../figures/phy1520-quantum/ps4DiracStepPotentialFig5}
{\( V_0 > 2 m c^2 \)}{fig:ps4DiracStepPotential:ps4DiracStepPotentialFig3}{scale=0.3}
{energyVsMomentumForDiracStepPotentialWaveFunctions.nb}

We want to find the wave incident, reflected, and transmitted wave function.  From the diagonalization above, these are

\begin{dmath}\label{eqn:gradQuantumProblemSet4Problem3:640}
\Psi_{\textrm{inc}} =
\begin{bmatrix}
\cos\theta_{k_1} \\
\sin\theta_{k_1} \\
\end{bmatrix}
e^{i (k_1 x - E t/\Hbar)}
\end{dmath}
\begin{dmath}\label{eqn:gradQuantumProblemSet4Problem3:660}
\Psi_{\textrm{ref}} =
\begin{bmatrix}
\sin\theta_{k_1} \\
\cos\theta_{k_1} \\
\end{bmatrix}
e^{i (-k_1 x - E t/\Hbar)}
\end{dmath}

The currents \( j = c (\psi_1^\conj \psi_1 - \psi_2^\conj \psi_2) \) for these wave functions are

\begin{equation}\label{eqn:gradQuantumProblemSet4Problem3:960}
j_{\textrm{inc}} = c (\cos^2 \theta_{k_1} - \sin^2 \theta_{k_1}) = c \cos(2 \theta_{k_1} ),
\end{equation}

and

\begin{equation}\label{eqn:gradQuantumProblemSet4Problem3:980}
j_{\textrm{ref}} = c (\sin^2 \theta_{k_1} - \cos^2 \theta_{k_1}) = -c \cos(2 \theta_{k_1} ).
\end{equation}

When there is a transmitted particle (anti-particle) in region II, the transmitted wave function are respectively

\begin{dmath}\label{eqn:gradQuantumProblemSet4Problem3:680}
\Psi_{\textrm{trans}} =
\begin{bmatrix}
\cos\theta_{k_2} \\
\sin\theta_{k_2} \\
\end{bmatrix}
e^{i (k_2 x - E t/\Hbar)}
\end{dmath}
\begin{dmath}\label{eqn:gradQuantumProblemSet4Problem3:700}
\Psi_{\textrm{trans}} =
\begin{bmatrix}
-\sin\theta_{k_2} \\
\cos\theta_{k_2} \\
\end{bmatrix}
e^{i (k_2 x - E t/\Hbar)}.
\end{dmath}

The currents for these are respectively
\begin{dmath}\label{eqn:gradQuantumProblemSet4Problem3:1321}
j = c \lr{ \cos^2\theta_{k_2} - \sin^2\theta_{k_2} } = c \cos(2\theta_{k_2}),
\end{dmath}

and

\begin{dmath}\label{eqn:gradQuantumProblemSet4Problem3:1341}
j = c \lr{ \sin^2\theta_{k_2} - \cos^2\theta_{k_2} } = -c \cos(2\theta_{k_2}).
\end{dmath}

To determine the weighting of the wave functions, we require matching at the boundary.  There are a few cases to consider

\begin{enumerate}[(i)]
\item Total reflection.  Are there any special values of momentum that allow for total reflection?  If so, at the boundary both components must be zero

\begin{dmath}\label{eqn:gradQuantumProblemSet4Problem3:720}
\begin{bmatrix}
\cos\theta_{k_1} \\
\sin\theta_{k_1} \\
\end{bmatrix}
+
\frac{B}{A}
\begin{bmatrix}
\sin\theta_{k_1} \\
\cos\theta_{k_1} \\
\end{bmatrix}
=
\begin{bmatrix}
0 \\
0
\end{bmatrix}
\end{dmath}

This is possible if \( -B/A = \cot\theta_{k_1} = \tan\theta_{k_1} \), which requires

\begin{equation}\label{eqn:gradQuantumProblemSet4Problem3:740}
\theta_{k_1} = \frac{\pi}{4} \lr{ 1 + 2 n }, \qquad n \in \bbZ.
\end{equation}

However, we must also have

\begin{dmath}\label{eqn:gradQuantumProblemSet4Problem3:760}
\tan 2 \theta_1 = \frac{m c}{\Hbar k},
\end{dmath}

so

\begin{dmath}\label{eqn:gradQuantumProblemSet4Problem3:780}
\theta_1 = \inv{2} \Atan\lr{ \frac{m c}{\Hbar k} }.
\end{dmath}

Simultaneous solutions of \cref{eqn:gradQuantumProblemSet4Problem3:740}, \cref{eqn:gradQuantumProblemSet4Problem3:780} only occur at \( k = \pm 0 \), where \( \tan( (1 + 2 n)\pi/2 ) = \pm \infty \).  Since a particle at rest is not at interest in a reflection scenario, this shows that a decaying solution in region II must be introduced to match the boundary value constraints.

\item Ordinary matter transmission.

For this case at \( x = 0 \), we have

\begin{dmath}\label{eqn:gradQuantumProblemSet4Problem3:920}
A
\begin{bmatrix}
\cos\theta_{k_1} \\
\sin\theta_{k_1} \\
\end{bmatrix}
+
B
\begin{bmatrix}
\sin\theta_{k_1} \\
\cos\theta_{k_1} \\
\end{bmatrix}
=
D
\begin{bmatrix}
\cos \theta_2 \\
\sin \theta_2
\end{bmatrix}.
\end{dmath}

With \( a = B/A \) and \( b = D/A \), \( S_{1,2} = \sin\theta_{k_{1,2}}, C_{1,2} = \cos\theta_{k_{1,2}} \), this is

\begin{dmath}\label{eqn:gradQuantumProblemSet4Problem3:800}
\begin{bmatrix}
C_1 \\
S_1
\end{bmatrix}
=
\begin{bmatrix}
- S_1 & C_2 \\
- C_1 & S_2
\end{bmatrix}
\begin{bmatrix}
a \\
b
\end{bmatrix},
\end{dmath}

or
\begin{dmath}\label{eqn:gradQuantumProblemSet4Problem3:820}
\begin{bmatrix}
a \\
b
\end{bmatrix}
=
\inv{- S_1 S_2 + C_1 C_2 }
\begin{bmatrix}
S_2 & -C_2 \\
C_1 & -S_1
\end{bmatrix}
\begin{bmatrix}
C_1 \\
S_1
\end{bmatrix}
=
\inv{\cos(\theta_{k_1} + \theta_{k_2})}
\begin{bmatrix}
C_1 S_2 - S_1 C_2 \\
C_1^2 - S_1^2
\end{bmatrix},
\end{dmath}

which is
\boxedEquation{eqn:gradQuantumProblemSet4Problem3:840}{
\begin{aligned}
\frac{B}{A} &= \frac{ \cos(\theta_{k_2} - \theta_{k_1}) }{\cos(\theta_{k_1} + \theta_{k_2})} \\
\frac{D}{A} &= \frac{ \cos(2 \theta_{k_1}) }{\cos(\theta_{k_1} + \theta_{k_2})} \\
\end{aligned}
}

Let's verify that the currents in both regions match.  With \( A = 1 \), the region I current sum is

\begin{dmath}\label{eqn:gradQuantumProblemSet4Problem3:1000}
j_{\textrm{inc}}
+ j_{\textrm{ref}}
=
c \cos( 2 \theta_{k_1} ) - B^2 c \cos( 2 \theta_{k_1} )
=
c \cos( 2 \theta_{k_1} )
\lr{ 1 -
\frac{ \sin^2(\theta_{k_2} - \theta_{k_1}) }{\cos^2(\theta_{k_1} + \theta_{k_2})} }
=
c \cos( 2 \theta_{k_1} )
\frac{ \cos^2(\theta_{k_1} + \theta_{k_2}) - \sin^2(\theta_{k_2} - \theta_{k_1}) }{\cos^2(\theta_{k_1} + \theta_{k_2})}
=
c
\frac{ \cos( 2 \theta_{k_1} ) \cos(2 \theta_{k_1}) \cos(2 \theta_{k_2})}
{\cos^2(\theta_{k_1} + \theta_{k_2})}.
\end{dmath}

Whereas, the transmitted (region II) current is
\begin{dmath}\label{eqn:gradQuantumProblemSet4Problem3:1020}
j_{\textrm{trans}}
=
 c D^2 \cos( 2 \theta_{k_2} )
=
 c \cos( 2 \theta_{k_2} )
\frac{ \cos^2(2 \theta_{k_1}) }{\cos^2(\theta_{k_1} + \theta_{k_2})},
\end{dmath}

so we see \( j_{\textrm{inc}} + j_{\textrm{ref}} = j_{\textrm{trans}} \), as expected.

\item Anti-matter transmission.
\index{anti-matter transmission}

For this case at \( x = 0 \), we have

\begin{dmath}\label{eqn:gradQuantumProblemSet4Problem3:940}
A
\begin{bmatrix}
\cos\theta_{k_1} \\
\sin\theta_{k_1} \\
\end{bmatrix}
+
B
\begin{bmatrix}
\sin\theta_{k_1} \\
\cos\theta_{k_1} \\
\end{bmatrix}
=
D
\begin{bmatrix}
-\sin \theta_2 \\
\cos \theta_2 \\
\end{bmatrix}.
\end{dmath}

With the same substitutions as above, this is

\begin{dmath}\label{eqn:gradQuantumProblemSet4Problem3:860}
\begin{bmatrix}
C_1 \\
S_1
\end{bmatrix}
=
\begin{bmatrix}
- S_1 & -S_2 \\
- C_1 & C_2
\end{bmatrix}
\begin{bmatrix}
a \\
b
\end{bmatrix},
\end{dmath}

or
\begin{dmath}\label{eqn:gradQuantumProblemSet4Problem3:880}
\begin{bmatrix}
a \\
b
\end{bmatrix}
=
\inv{- (S_1 C_2 + S_2 C_1) }
\begin{bmatrix}
C_2 & -S_2 \\
C_1 & -S_1
\end{bmatrix}
\begin{bmatrix}
C_1 \\
S_1
\end{bmatrix}
=
\inv{-\sin(\theta_{k_1} + \theta_{k_2})}
\begin{bmatrix}
C_1 C_2 + S_1 S_2 \\
C_1^2 - S_1^2
\end{bmatrix},
\end{dmath}

which is
\boxedEquation{eqn:gradQuantumProblemSet4Problem3:900}{
\begin{aligned}
\frac{B}{A} &= \frac{ \cos(\theta_{k_1} - \theta_{k_2}) }{\sin(\theta_{k_1} + \theta_{k_2})} \\
\frac{D}{A} &= -\frac{ \cos(2 \theta_{k_1}) }{\sin(\theta_{k_1} + \theta_{k_2})} \\
\end{aligned}
}

For the anti-particle transmission, the region I current is

\begin{dmath}\label{eqn:gradQuantumProblemSet4Problem3:1001}
j_{\textrm{inc}}
+ j_{\textrm{ref}}
=
c \cos( 2 \theta_{k_1} ) - B^2 c \cos( 2 \theta_{k_1} )
=
c \cos( 2 \theta_{k_1} )
\lr{ 1 -
\frac{ \cos^2(\theta_{k_1} - \theta_{k_2}) }{\sin^2(\theta_{k_1} + \theta_{k_2})} }
=
c \cos( 2 \theta_{k_1} )
\frac{ \sin^2(\theta_{k_1} + \theta_{k_2}) - \cos^2(\theta_{k_1} - \theta_{k_2}) }{\cos^2(\theta_{k_1} + \theta_{k_2})}
=
-c
\frac{ \cos( 2 \theta_{k_1} ) \cos(2 \theta_{k_1}) \cos(2 \theta_{k_2})}
{\sin^2(\theta_{k_1} + \theta_{k_2})}.
\end{dmath}

Whereas, the transmitted (region II) current is
\begin{dmath}\label{eqn:gradQuantumProblemSet4Problem3:1021}
j_{\textrm{trans}}
=
 -c D^2 \cos( 2 \theta_{k_2} )
=
 -c \cos( 2 \theta_{k_2} )
\frac{ \cos^2(2 \theta_{k_1}) }{\sin^2(\theta_{k_1} + \theta_{k_2})},
\end{dmath}

and again we see \( j_{\textrm{inc}} + j_{\textrm{ref}} = j_{\textrm{trans}} \), as expected.

\item Decaying transmission.

In units where \( \Hbar = c = 1 \), the eigenkets found for the Dirac Hamiltonian, before normalization, were found to be

\begin{dmath}\label{eqn:gradQuantumProblemSet4Problem3:1041}
\ket{\pm} \propto
\begin{bmatrix}
\mp m \\
\epsilon_1 \pm k
\end{bmatrix}
e^{\pm i k x - i E t/\Hbar},
\end{dmath}

where \( \epsilon_1^2 = m^2 + k^2 \).  Letting \( k \rightarrow i k \), this provides the form of the non-oscillatory wavefunctions in region II when there is no ordinary nor anti-particle transmission.

\begin{dmath}\label{eqn:gradQuantumProblemSet4Problem3:1061}
\ket{\pm} \propto
\begin{bmatrix}
\mp m \\
\epsilon_2 \pm i k
\end{bmatrix}
e^{\mp k x - i E t/\Hbar},
\end{dmath}

where

\begin{dmath}\label{eqn:gradQuantumProblemSet4Problem3:1081}
\epsilon_2^2 = m^2 - k^2
\end{dmath}

Observe that the \( \psi_2 \) component of this wave function sits on a circle in the complex plane

\begin{dmath}\label{eqn:gradQuantumProblemSet4Problem3:1101}
\Abs{\epsilon_2 \pm i k}^2
= \epsilon^2 + k^2
= m^2 - k^2 + k^2
= m^2.
\end{dmath}

Putting back in the factors of \( \Hbar \), \( c\), this allows the identification

\begin{dmath}\label{eqn:gradQuantumProblemSet4Problem3:1121}
\epsilon_2 \pm i \Hbar k c = m c^2 e^{i\phi}.
\end{dmath}

So, like the trigonometric representation of the oscillatory wave function, the normalized wave functions for exponential decay(increase) can be written

\begin{dmath}\label{eqn:gradQuantumProblemSet4Problem3:1141}
\ket{\pm}
=
\inv{\sqrt{2}}
\begin{bmatrix}
\pm e^{\pm i \phi/2} \\
e^{\mp i \phi/2}
\end{bmatrix}
e^{\mp k x - i E t/\Hbar}
,
\end{dmath}

where the respective eigenvalues are \( \epsilon_2 = \pm \sqrt{ (m c^2)^2 - (\Hbar k c)^2} \), and \( \Hbar k c < m c^2 \).

Observe that the currents for the exponential wave functions are both light-like

\begin{equation}\label{eqn:gradQuantumProblemSet4Problem3:1161}
j_{\pm} = c \lr{ \Abs{\pm e^{\pm i \phi/2}}^2 - \Abs{e^{\mp i \phi/2}}^2 } = 0,
\end{equation}

which makes some sense since the particle is not able to propagate freely in the barrier region.

At the interface, we wish to solve

\begin{dmath}\label{eqn:gradQuantumProblemSet4Problem3:1181}
A
\begin{bmatrix}
C_1 \\
S_1
\end{bmatrix}
+
B
\begin{bmatrix}
S_1 \\
C_1 \\
\end{bmatrix}
=
\frac{D}{\sqrt{2}}
\begin{bmatrix}
e^{i \phi/2} \\
e^{-i \phi/2} \\
\end{bmatrix}.
\end{dmath}

With \( a = A/D \), and \( b = B/D \), this has solution

\begin{dmath}\label{eqn:gradQuantumProblemSet4Problem3:1201}
\begin{bmatrix}
a \\
b
\end{bmatrix}
=
\inv{\sqrt{2}}
{\begin{bmatrix}
C_1 & S_1 \\
S_1 & C_1
\end{bmatrix}}^{-1}
\begin{bmatrix}
e^{i \phi/2} \\
e^{-i \phi/2} \\
\end{bmatrix}
=
\inv{\sqrt{2} \cos(2 \theta_{k_1})}
\begin{bmatrix}
C_1 & -S_1 \\
-S_1 & C_1
\end{bmatrix}
\begin{bmatrix}
e^{i \phi/2} \\
e^{-i \phi/2} \\
\end{bmatrix}
=
\inv{\sqrt{2} \cos(2 \theta_{k_1})}
\begin{bmatrix}
C_1 e^{i \phi/2} -S_1 e^{-i \phi/2} \\
-S_1 e^{i \phi/2} + C_1 e^{-i \phi/2}
\end{bmatrix}.
\end{dmath}

The reflection coefficient is unity
\begin{dmath}\label{eqn:gradQuantumProblemSet4Problem3:1221}
R
= \Abs{\frac{B}{A}}^2
= \Abs{\frac{B/D}{A/D}}^2
= \frac{\Abs{-S_1 e^{i \phi/2} + C_1 e^{-i \phi/2}}^2}
{\Abs{C_1 e^{i \phi/2} -S_1 e^{-i \phi/2} }^2}
=
\frac
{ C_1^2 + S_1^2 - 2 S_1 C_1 \Real e^{-i\phi} }
{ C_1^2 + S_1^2 - 2 S_1 C_1 \Real e^{i\phi} }
=
\frac
{ 1 - \sin( 2 \theta_{k_1}) \cos \phi }
{ 1 - \sin( 2 \theta_{k_1}) \cos \phi }
=
1.
\end{dmath}

%%%\paragraph{Confusion:}
%%%Completely counter to (my) intuition, the transmission coefficient is not 0, so we don't appear to have \( R + T = 1 \).  Instead
%%%
%%%\begin{dmath}\label{eqn:gradQuantumProblemSet4Problem3:1241}
%%%T
%%%= \Abs{\frac{D}{A}}^2
%%%=
%%%\frac{2 \cos^2(2 \theta_{k_1})}
%%%{
%%%1 - \sin( 2 \theta_{k_1}) \cos \phi
%%%}
%%%\ne 0.
%%%\end{dmath}
%%%
%%%
%%%Temporarily reverting to natural units, note that
%%%
%%%\begin{dmath}\label{eqn:gradQuantumProblemSet4Problem3:1261}
%%%\cos\phi
%%%= \Real e^{i \phi}
%%%= \Real \lr{ \frac{\epsilon_2}{m} + i \frac{k_2}{m} }
%%%%=
%%%%\frac{
%%%%\sqrt{
%%%%m^2 - k_2^2
%%%%}
%%%%}{m}
%%%\frac{\epsilon_2}{m},
%%%\end{dmath}
%%%
%%%so
%%%\begin{dmath}\label{eqn:gradQuantumProblemSet4Problem3:1281}
%%%T
%%%=
%%%\frac{
%%%   2 \lr{\frac{k_1}{\epsilon_1}}^2
%%%}{
%%%1 - \frac{m}{\epsilon_1} \frac{\epsilon_2}{m}
%%%}
%%%=
%%%\frac{
%%%   2 k_1^2
%%%}{
%%%\epsilon_1^2 - \epsilon_1 \epsilon_2
%%%}
%%%\end{dmath}
%%%
%%%In terms of \(k_1, k_2 \) after putting back \( \Hbar, c \)'s this is
%%%
%%%\begin{dmath}\label{eqn:gradQuantumProblemSet4Problem3:1301}
%%%T
%%%=
%%%\frac{
%%%   2 (\Hbar k_1 c)^2
%%%}{
%%%\lr{ m c^2 }^2 + \lr{ \Hbar k_1 c }^2
%%%- \sqrt{
%%%\lr{(m c^2)^2 + \lr{ \Hbar k_1 c}^2 }
%%%\lr{(m c^2)^2 - \lr{ \Hbar k_2 c}^2 }
%%%}
%%%}.
%%%\end{dmath}

\end{enumerate}

}
}

         %
% Copyright � 2015 Peeter Joot.  All Rights Reserved.
% Licenced as described in the file LICENSE under the root directory of this GIT repository.
%
%\input{../blogpost.tex}
%\renewcommand{\basename}{diracAlternate}
%\renewcommand{\dirname}{notes/phy1520/}
%%\newcommand{\dateintitle}{}
%%\newcommand{\keywords}{}
%
%\input{../peeter_prologue_print2.tex}
%
%\usepackage{peeters_layout_exercise}
%\usepackage{peeters_braket}
%\usepackage{peeters_figures}
%
%\beginArtNoToc
%
%\generatetitle{Alternate Dirac equation representation}
%\chapter{AlternateDiracEquation}
%\label{chap:diracAlternate}
%
\makeoproblem
%{Alternate Dirac equation representation.}
{Dirac equation representations.}
{problem:diracAlternate:1}{2015 midterm pr. 2}{
%
Given an alternate representation of the Dirac equation
%
\begin{dmath}\label{eqn:diracAlternate:20}
H =
\begin{bmatrix}
m c^2 + V_0 & c \hatp \\
c \hatp & - m c^2 + V_0
\end{bmatrix},
\end{dmath}
%
calculate
%
\makesubproblem{}{problem:diracAlternate:1:a}
the constant momentum plane wave solutions,
%
\makesubproblem{}{problem:diracAlternate:1:b}
the constant momentum hyperbolic solutions,
%
\makesubproblem{}{problem:diracAlternate:1:c}
the Heisenberg velocity operator \( \hatv \), and
%
\makesubproblem{}{problem:diracAlternate:1:d}
find the form of the probability density current.
} % problem
%
\makeanswer{problem:diracAlternate:1}{
%
\makeSubAnswer{}{problem:diracAlternate:1:a}
%
The action of the Hamiltonian on
%
\begin{dmath}\label{eqn:diracAlternate:40}
\psi =
e^{i k x - i E t/\Hbar}
\begin{bmatrix}
\psi_1 \\
\psi_2
\end{bmatrix},
\end{dmath}
is
\begin{dmath}\label{eqn:diracAlternate:60}
H \psi
=
\begin{bmatrix}
m c^2 + V_0 & c (-i \Hbar) i k \\
c (-i \Hbar) i k & - m c^2 + V_0
\end{bmatrix}
\begin{bmatrix}
\psi_1 \\
\psi_2
\end{bmatrix}
e^{i k x - i E t/\Hbar}
=
\begin{bmatrix}
m c^2 + V_0 & c \Hbar k \\
c \Hbar k & - m c^2 + V_0
\end{bmatrix}
\psi.
\end{dmath}
%
Writing
%
\begin{dmath}\label{eqn:diracAlternate:80}
H_k
=
\begin{bmatrix}
m c^2 + V_0 & c \Hbar k \\
c \Hbar k & - m c^2 + V_0
\end{bmatrix},
\end{dmath}
the characteristic equation is
%
\begin{dmath}\label{eqn:diracAlternate:100}
0 =
(m c^2 + V_0 - \lambda)
(-m c^2 + V_0 - \lambda)
%(-m c^2 - V_0 + \lambda)
%(m c^2 - V_0 + \lambda)
- (c \Hbar k)^2
=
\lr{ (\lambda - V_0)^2 - (m c^2)^2 } - (c \Hbar k)^2,
\end{dmath}
%
so
%
\begin{equation}\label{eqn:diracAlternate:120}
\lambda = V_0 \pm \epsilon,
\end{equation}
%
where
\begin{equation}\label{eqn:diracAlternate:140}
\epsilon^2 = (m c^2)^2 + (c \Hbar k)^2.
\end{equation}
%
We've got
%
\begin{dmath}\label{eqn:diracAlternate:160}
\begin{aligned}
H - ( V_0 + \epsilon )
&=
\begin{bmatrix}
m c^2 - \epsilon & c \Hbar k \\
c \Hbar k & - m c^2 - \epsilon
\end{bmatrix} \\
H - ( V_0 - \epsilon )
&=
\begin{bmatrix}
m c^2 + \epsilon & c \Hbar k \\
c \Hbar k & - m c^2 + \epsilon
\end{bmatrix},
\end{aligned}
\end{dmath}

so the eigenkets are
%
\begin{dmath}\label{eqn:diracAlternate:180}
\begin{aligned}
\ket{V_0+\epsilon}
&\propto
\begin{bmatrix}
-c \Hbar k \\
m c^2 - \epsilon
\end{bmatrix} \\
\ket{V_0-\epsilon}
&\propto
\begin{bmatrix}
-c \Hbar k \\
m c^2 + \epsilon
\end{bmatrix}.
\end{aligned}
\end{dmath}
%
% (c \Hbar k)^2 + (m c^2 - \epsilon)^2 = 2 \epsilon^2 - 2 m c^2 \epsilon = 2 \epsilon ( \epsilon - m c^2)
% (c \Hbar k)^2 + (m c^2 + \epsilon)^2 = 2 \epsilon^2 + 2 m c^2 \epsilon = 2 \epsilon ( \epsilon + m c^2)

Up to an arbitrary phase for each, these are
%
\begin{dmath}\label{eqn:diracAlternate:200}
\begin{aligned}
\ket{V_0 + \epsilon}
&=
\inv{\sqrt{ 2 \epsilon ( \epsilon - m c^2) }}
\begin{bmatrix}
c \Hbar k \\
\epsilon -m c^2
\end{bmatrix}, \\
\ket{V_0 - \epsilon}
&=
\inv{\sqrt{ 2 \epsilon ( \epsilon + m c^2) }}
\begin{bmatrix}
-c \Hbar k \\
\epsilon + m c^2
\end{bmatrix}.
\end{aligned}
\end{dmath}

We can now write
%
\begin{dmath}\label{eqn:diracAlternate:220}
H_k =
E
\begin{bmatrix}
V_0 + \epsilon & 0 \\
0         & V_0 - \epsilon
\end{bmatrix}
E^{-1},
\end{dmath}
%
where
\begin{equation}\label{eqn:diracAlternate:240}
\begin{aligned}
E &=
\inv{\sqrt{2 \epsilon} }
\begin{bmatrix}
\frac{c \Hbar k}{ \sqrt{ \epsilon - m c^2 } } & -\frac{c \Hbar k}{ \sqrt{ \epsilon + m c^2 } } \\
\sqrt{ \epsilon - m c^2 } & \sqrt{ \epsilon + m c^2 }
\end{bmatrix}, \qquad k > 0 \\
E &=
\inv{\sqrt{2 \epsilon} }
\begin{bmatrix}
-\frac{c \Hbar k}{ \sqrt{ \epsilon - m c^2 } } & -\frac{c \Hbar k}{ \sqrt{ \epsilon + m c^2 } } \\
-\sqrt{ \epsilon - m c^2 } & \sqrt{ \epsilon + m c^2 }
\end{bmatrix}, \qquad k < 0.
\end{aligned}
\end{equation}
%
Here the signs have been adjusted to ensure the transformation matrix has a unit determinant.

Observe that there's redundancy in this matrix since \( \ifrac{c \Hbar \Abs{k}}{ \sqrt{ \epsilon - m c^2 } } = \sqrt{ \epsilon + m c^2 } \), and \( \ifrac{c \Hbar \Abs{k}}{ \sqrt{ \epsilon + m c^2 } } = \sqrt{ \epsilon - m c^2 } \), which allows the transformation matrix to be written in the form of a rotation matrix
%
\begin{equation}\label{eqn:diracAlternate:260}
\begin{aligned}
E &=
\inv{\sqrt{2 \epsilon} }
\begin{bmatrix}
\frac{c \Hbar k}{ \sqrt{ \epsilon - m c^2 } } & -\frac{c \Hbar k}{ \sqrt{ \epsilon + m c^2 } } \\
\frac{c \Hbar k}{ \sqrt{ \epsilon + m c^2 } } & \frac{c \Hbar k}{ \sqrt{ \epsilon - m c^2 } }
\end{bmatrix}, \qquad k > 0 \\
E &=
\inv{\sqrt{2 \epsilon} }
\begin{bmatrix}
-\frac{c \Hbar k}{ \sqrt{ \epsilon - m c^2 } } & -\frac{c \Hbar k}{ \sqrt{ \epsilon + m c^2 } } \\
\frac{c \Hbar k}{ \sqrt{ \epsilon + m c^2 } }  & -\frac{c \Hbar k}{ \sqrt{ \epsilon - m c^2 } }
\end{bmatrix}, \qquad k < 0.
\end{aligned}
\end{equation}
With
%
\begin{equation}\label{eqn:diracAlternate:280}
\begin{aligned}
\cos\theta &= \frac{c \Hbar \Abs{k}}{ \sqrt{ 2 \epsilon( \epsilon - m c^2) } } = \frac{\sqrt{ \epsilon + m c^2} }{ \sqrt{ 2 \epsilon}}\\
\sin\theta &= \frac{c \Hbar k}{ \sqrt{ 2 \epsilon( \epsilon + m c^2) } } = \frac{\sgn(k) \sqrt{ \epsilon - m c^2}}{ \sqrt{ 2 \epsilon } },
\end{aligned}
\end{equation}
the transformation matrix (and eigenkets) is
%\begin{equation}\label{eqn:diracAlternate:300}
\boxedEquation{eqn:diracAlternate:300}{
E =
\begin{bmatrix}
\ket{V_0 + \epsilon} & \ket{V_0 - \epsilon}
\end{bmatrix}
=
\begin{bmatrix}
\cos\theta & -\sin\theta \\
\sin\theta & \cos\theta
\end{bmatrix}.
}
%\end{equation}
%
Observe that \cref{eqn:diracAlternate:280} can be simplified by using double angle formulas
%
\begin{dmath}\label{eqn:diracAlternate:320}
\cos(2 \theta)
=
\frac{\lr{ \epsilon + m c^2} }{ 2 \epsilon }
-
\frac{\lr{ \epsilon - m c^2}}{ 2 \epsilon }
=
\frac{1}{ 2 \epsilon } \lr{ \epsilon + m c^2 - \epsilon + m c^2 }
=
\frac{m c^2 }{ \epsilon },
\end{dmath}
%
and
\begin{dmath}\label{eqn:diracAlternate:340}
\sin(2\theta)
=
2 \frac{1}{2 \epsilon} \sgn(k ) \sqrt{ \epsilon^2 - (m c^2)^2 }
=
\frac{\Hbar k c}{\epsilon}.
\end{dmath}
%
This allows all the \( \theta \) dependence on \( \Hbar k c \) and \( m c^2 \) to be expressed as a ratio of momenta
%
%\begin{dmath}\label{eqn:diracAlternate:360}
\boxedEquation{eqn:diracAlternate:360}{
\tan(2\theta) = \frac{\Hbar k}{m c}.
}
%\end{dmath}
%
\makeSubAnswer{}{problem:diracAlternate:1:b}
For a wave function of the form
%
\begin{dmath}\label{eqn:diracAlternate:380}
\psi =
e^{k x - i E t/\Hbar}
\begin{bmatrix}
\psi_1 \\
\psi_2
\end{bmatrix},
\end{dmath}
%
some of the work above can be recycled if we substitute \( k \rightarrow -i k \), which yields unnormalized eigenfunctions
%
\begin{dmath}\label{eqn:diracAlternate:400}
\begin{aligned}
\ket{V_0+\epsilon}
&\propto
\begin{bmatrix}
i c \Hbar k \\
m c^2 - \epsilon
\end{bmatrix} \\
\ket{V_0-\epsilon}
&\propto
\begin{bmatrix}
i c \Hbar k \\
m c^2 + \epsilon
\end{bmatrix},
\end{aligned}
\end{dmath}

where
%
\begin{equation}\label{eqn:diracAlternate:420}
\epsilon^2 = (m c^2)^2 - (c \Hbar k)^2.
\end{equation}
%
The squared magnitude of these wavefunctions are
%
\begin{dmath}\label{eqn:diracAlternate:440}
(c \Hbar k)^2 + (m c^2 \mp \epsilon)^2
=
(c \Hbar k)^2 + (m c^2)^2 + \epsilon^2 \mp 2 m c^2 \epsilon
=
(c \Hbar k)^2 + (m c^2)^2 + (m c^2)^2 \mp (c \Hbar k)^2 - 2 m c^2 \epsilon
= 2 (m c^2)^2 \mp 2 m c^2 \epsilon
= 2 m c^2 ( m c^2 \mp \epsilon ),
\end{dmath}
%
so, up to a constant phase for each, the normalized kets are
%
\begin{dmath}\label{eqn:diracAlternate:460}
\begin{aligned}
\ket{V_0+\epsilon}
&=
\inv{\sqrt{ 2 m c^2 ( m c^2 - \epsilon ) }}
\begin{bmatrix}
i c \Hbar k \\
m c^2 - \epsilon
\end{bmatrix} \\
\ket{V_0-\epsilon}
&=
\inv{\sqrt{ 2 m c^2 ( m c^2 + \epsilon ) }}
\begin{bmatrix}
i c \Hbar k \\
m c^2 + \epsilon
\end{bmatrix}.
\end{aligned}
\end{dmath}
After the \( k \rightarrow -i k \) substitution, \( H_k \) is not Hermitian, so these kets aren't expected to be orthonormal, which is readily verified
%
\begin{dmath}\label{eqn:diracAlternate:480}
\begin{aligned}
\braket{V_0+\epsilon}{V_0-\epsilon}
&=
\inv{\sqrt{ 2 m c^2 ( m c^2 - \epsilon ) }}
\inv{\sqrt{ 2 m c^2 ( m c^2 + \epsilon ) }} ,\times \\
&\qquad \begin{bmatrix}
-i c \Hbar k &
m c^2 - \epsilon
\end{bmatrix}
\begin{bmatrix}
i c \Hbar k \\
m c^2 + \epsilon
\end{bmatrix} \\
&=
\frac{ 2 ( c \Hbar k )^2 }{2 m c^2 \sqrt{(\Hbar k c)^2} }  \\
&=
\sgn(k)
\frac{
\Hbar k }{m c } .
\end{aligned}
\end{dmath}
%
\makeSubAnswer{}{problem:diracAlternate:1:c}
\begin{dmath}\label{eqn:diracAlternate:500}
\hatv
= \inv{i \Hbar} \antisymmetric{ \hatx }{ H}
= \inv{i \Hbar} \antisymmetric{ \hatx }{ m c^2 \sigma_z + V_0 + c \hatp \sigma_x }
= \frac{c \sigma_x}{i \Hbar} \antisymmetric{ \hatx }{ \hatp }
= c \sigma_x.
\end{dmath}
%
\makeSubAnswer{}{problem:diracAlternate:1:d}
%
Acting against a completely general wavefunction the Hamiltonian action \( H \psi \) is
%
\begin{dmath}\label{eqn:diracAlternate:520}
i \Hbar \PD{t}{\psi}
= m c^2 \sigma_z \psi + V_0 \psi + c \hatp \sigma_x \psi
= m c^2 \sigma_z \psi + V_0 \psi -i \Hbar c \sigma_x \PD{x}{\psi}.
\end{dmath}
%
Conversely, the conjugate \( (H \psi)^\dagger \) is
%
\begin{dmath}\label{eqn:diracAlternate:540}
-i \Hbar \PD{t}{\psi^\dagger}
= m c^2 \psi^\dagger \sigma_z + V_0 \psi^\dagger +i \Hbar c \PD{x}{\psi^\dagger} \sigma_x.
\end{dmath}
%
These give
%
\begin{dmath}\label{eqn:diracAlternate:560}
\begin{aligned}
i \Hbar \psi^\dagger \PD{t}{\psi}
&=
m c^2 \psi^\dagger \sigma_z \psi + V_0 \psi^\dagger \psi -i \Hbar c \psi^\dagger \sigma_x \PD{x}{\psi} \\
-i \Hbar \PD{t}{\psi^\dagger} \psi
&= m c^2 \psi^\dagger \sigma_z \psi + V_0 \psi^\dagger \psi +i \Hbar c \PD{x}{\psi^\dagger} \sigma_x \psi.
\end{aligned}
\end{dmath}
%
Taking differences
\begin{dmath}\label{eqn:diracAlternate:580}
\psi^\dagger \PD{t}{\psi} + \PD{t}{\psi^\dagger} \psi
=
- c \psi^\dagger \sigma_x \PD{x}{\psi} - c \PD{x}{\psi^\dagger} \sigma_x \psi,
\end{dmath}
%
or
%
\begin{dmath}\label{eqn:diracAlternate:600}
0
=
\PD{t}{}
\lr{
\psi^\dagger \psi
}
+
\PD{x}{}
\lr{
c \psi^\dagger \sigma_x \psi
}.
\end{dmath}
%
The probability current still has the usual form \( \rho = \psi^\dagger \psi = \psi_1^\conj \psi_1 + \psi_2^\conj \psi_2 \), but the probability current with this representation of the Dirac Hamiltonian is
%
\begin{dmath}\label{eqn:diracAlternate:620}
j
= c \psi^\dagger \sigma_x \psi
= c
\begin{bmatrix}
\psi_1^\conj &
\psi_2^\conj
\end{bmatrix}
\begin{bmatrix}
\psi_2 \\
\psi_1
\end{bmatrix}
= c \lr{ \psi_1^\conj \psi_2 + \psi_2^\conj \psi_1 }.
\end{dmath}
} % answer

%\EndNoBibArticle

   \mychapter{Symmetries in quantum mechanics.}
      %
% Copyright � 2015 Peeter Joot.  All Rights Reserved.
% Licenced as described in the file LICENSE under the root directory of this GIT repository.
%%
%\input{../blogpost.tex}
%\renewcommand{\basename}{qmLecture11}
%\renewcommand{\dirname}{notes/phy1520/}
%\newcommand{\keywords}{PHY1520H}
%\input{../peeter_prologue_print2.tex}
%
%%\usepackage{phy1520}
%\usepackage{peeters_braket}
%\usepackage{peeters_layout_exercise}
%\usepackage{peeters_figures}
%\usepackage{mathtools}
%
%\beginArtNoToc
%\generatetitle{PHY1520H Graduate Quantum Mechanics.  Lecture 11: Symmetries in QM.  Taught by Prof.\ Arun Paramekanti}
%%\chapter{Symmetries in QM}
%\label{chap:qmLecture11}
%
%\paragraph{Disclaimer}
%
%Peeter's lecture notes from class.  These may be incoherent and rough.
%
%These are notes for the UofT course PHY1520, Graduate Quantum Mechanics, taught by Prof. Paramekanti, covering \textchapref{{4}} \citep{sakurai2014modern} content.
%
\section{Symmetry in classical mechanics.}
\index{symmetry!classical}

In a classical context considering a Hamiltonian
%
\begin{equation}\label{eqn:qmLecture11:20}
H(q_i, p_i),
\end{equation}
%
a symmetry means that certain \( q_i \) don't appear.  In that case the rate of change of one of the generalized momenta is zero
%
\begin{equation}\label{eqn:qmLecture11:40}
\ddt{p_k} = - \PD{q_k}{H} = 0,
\end{equation}
%
so \( p_k \) is a constant of motion.  This simplifies the problem by reducing the number of degrees of freedom.  Another aspect of such a symmetry is that it \underline{relates trajectories}.  For example, assuming a rotational symmetry as in \cref{fig:lecture11:lecture11Fig1}.
\imageFigure{../figures/phy1520-quantum/lecture11Fig1}{Trajectory under rotational symmetry.}{fig:lecture11:lecture11Fig1}{0.2}
the trajectory of a particle after rotation is related by rotation to the trajectory of the unrotated particle.
\section{Symmetry in quantum mechanics.}

Suppose that we have a symmetry operation that takes states from
%
\begin{equation}\label{eqn:qmLecture11:60}
\ket{\psi} \rightarrow \ket{U \psi}
\end{equation}
\begin{equation}\label{eqn:qmLecture11:80}
\ket{\phi} \rightarrow \ket{U \phi},
\end{equation}
%
we expect that
%
\begin{equation}\label{eqn:qmLecture11:100}
\Abs{\braket{ \psi}{\phi} }^2 = \Abs{\braket{ U\psi}{ U\phi} }^2.
\end{equation}
%
This won't hold true for a general operator.   Two cases where this does hold true is when

\begin{itemize}
\item \( \braket{\psi}{\phi} = \braket{ U\psi}{ U\phi} \).  Here \( U \) is \textAndIndex{unitary}, and the equivalence follows from
%
\begin{equation}\label{eqn:qmLecture11:120}
\braket{ U\psi}{ U\phi} = \bra{ \psi} U^\dagger U { \phi} = \bra{ \psi} 1 { \phi} = \braket{\psi}{\phi}.
\end{equation}
%
\item \( \braket{\psi}{\phi} = \braket{ U\psi}{ U\phi}^\conj \).  Here \( U \) is \textAndIndex{anti-unitary}.

% FIXME: This second sort of symmetry will be useful in translation operations?
\end{itemize}

\paragraph{Unitary case}
\index{symmetry!unitary}

If an ``observable'' is not changed by a unitary operation representing a symmetry we must have
%
\begin{equation}\label{eqn:qmLecture11:140}
\bra{\psi} \hatA \ket{\psi}
\rightarrow
\bra{U \psi} \hatA \ket{U \psi}
=
\bra{\psi} U^\dagger \hatA U \ket{\psi},
\end{equation}
%
so
\begin{equation}\label{eqn:qmLecture11:160}
U^\dagger \hatA U = \hatA,
\end{equation}
%
or
\boxedEquation{eqn:qmLecture11:180}{
\hatA U  = U \hatA.
}
An observable that is unchanged by a unitary symmetry commutes \( \antisymmetric{\hatA}{U} \) with the operator \( U \) for that transformation.
\paragraph{Symmetries of the Hamiltonian}
Given
\begin{equation}\label{eqn:qmLecture11:200}
\antisymmetric{H}{U} = 0,
\end{equation}
%
\( H \) is invariant.
Given
%
\begin{equation}\label{eqn:qmLecture11:220}
H \ket{\phi_n} = \epsilon_n \ket{\phi_n},
\end{equation}
%
\begin{equation}\label{eqn:qmLecture11:240}
\begin{aligned}
U H \ket{\phi_n}
&= H U \ket{\phi_n}
\\ &= \epsilon_n U \ket{\phi_n}.
\end{aligned}
\end{equation}
Such a state
%
\begin{equation}\label{eqn:qmLecture11:260}
\ket{\psi_n} = U \ket{\phi_n},
\end{equation}
is also an eigenstate with the \underline{same} energy.
Suppose this process is repeated, finding other states
%
\begin{equation}\label{eqn:qmLecture11:280}
U \ket{\psi_n} = \ket{\chi_n}
\end{equation}
\begin{equation}\label{eqn:qmLecture11:300}
U \ket{\chi_n} = \ket{\alpha_n}
\end{equation}
Because such a transformation only generates states with the initial energy, this process cannot continue forever.  At some point this process will enumerate a fixed size set of states.  These states can be orthonormalized.

We can say that symmetry operations are generators of a \underlineAndIndex{group}.  For a set of symmetry operations we can
\index{symmetry!group}
\begin{itemize}
\item
form products that lie in a closed set
%
\begin{equation}\label{eqn:qmLecture11:320}
U_1 U_2 = U_3,
\end{equation}
\item and can define an inverse operation
\begin{equation}\label{eqn:qmLecture11:340}
U \leftrightarrow U^{-1}.
\end{equation}
\item Such operators obey associative rules for multiplication
\begin{equation}\label{eqn:qmLecture11:360}
U_1 ( U_2 U_3 ) = (U_1 U_2) U_3.
\end{equation}
\item and have an identity operation.
\end{itemize}
When \( H \) has a symmetry, then degenerate eigenstates form \underlineAndIndex{irreducible} representations (which cannot be further block diagonalized).

\paragraph{Parity symmetry}
\index{parity}
\makeexample{Inversion.}{example:qmLecture11:1}{
Consider a state and a parity operation \( \hat\Pi \), with the transformation
%
\begin{equation}\label{eqn:qmLecture11:380}
\ket{\psi} \rightarrow \hat\Pi \ket{\psi}.
\end{equation}
In one dimension, the parity operation is just inversion.  In two dimensions, this is a set of flipping operations on two axes \cref{fig:lecture11:lecture11Fig2}.
\imageFigure{../figures/phy1520-quantum/lecture11Fig2}{2D parity operation.}{fig:lecture11:lecture11Fig2}{0.2}
The operational effects of this operator are
%
\begin{equation}\label{eqn:qmLecture11:400}
\begin{aligned}
\hatx &\rightarrow - \hatx \\
\hatp &\rightarrow - \hatp.
\end{aligned}
\end{equation}
%
Acting again with the \textAndIndex{parity operator} produces the original value, so it is its own inverse, and \( \hat\Pi^\dagger = \hat\Pi = \hat\Pi^{-1} \).  In an expectation value
%
\begin{equation}\label{eqn:qmLecture11:420}
\bra{ \hat\Pi \psi } \hatx \ket{ \hat\Pi \psi } = - \bra{\psi} \hatx \ket{\psi}.
\end{equation}
%
This means that
%
\begin{equation}\label{eqn:qmLecture11:440}
\hat\Pi^\dagger \hatx \hat\Pi = - \hatx,
\end{equation}
%
or
\begin{equation}\label{eqn:qmLecture11:460}
\hatx \hat\Pi = - \hat\Pi \hatx,
\end{equation}
%
%FIXME: show that \( \hat\Pi^\dagger \hatp \hat\Pi = - \hatp \).
\begin{equation}\label{eqn:qmLecture11:480}
\begin{aligned}
\hatx \hat\Pi \ket{x_0}
&= - \hat\Pi \hatx  \ket{x_0}
\\ &= - \hat\Pi x_0 \ket{x_0}
\\ &= - x_0 \hat\Pi \ket{x_0},
\end{aligned}
\end{equation}
so
%
\begin{equation}\label{eqn:qmLecture11:500}
\hat\Pi \ket{x_0} = \ket{-x_0}.
\end{equation}
%
Acting on a wave function
%
\begin{equation}\label{eqn:qmLecture11:520}
\begin{aligned}
\bra{x} \hat\Pi \ket{\psi}
&=
\braket{-x}{\psi}
\\ &= \psi(-x).
\end{aligned}
\end{equation}
%
What does this mean for eigenfunctions.  Eigenfunctions are supposed to form irreducible representations of the group.  The group has just two elements
%
\begin{equation}\label{eqn:qmLecture11:540}
\setlr{ 1, \hat\Pi },
\end{equation}
%
where \( \hat\Pi^2 = 1 \).

Suppose we have a Hamiltonian
%
\begin{equation}\label{eqn:qmLecture11:560}
H = \frac{\hatp^2}{2m} + V(\hatx),
\end{equation}
%
where \( V(\hatx) \) is even, or \( \antisymmetric{V(\hatx)}{\hat\Pi } = 0 \).  The squared momentum commutes with the parity operator
%
\begin{equation}\label{eqn:qmLecture11:580}
\begin{aligned}
\antisymmetric{\hatp^2}{\hat\Pi}
&=
\hatp^2 \hat\Pi
- \hat\Pi \hatp^2
\\ &=
\hatp^2 \hat\Pi
- (\hat\Pi \hatp) \hatp
\\ &=
\hatp^2 \hat\Pi
-(- \hatp \hat\Pi) \hatp
\\ &=
\hatp^2 \hat\Pi
+ \hatp (-\hatp \hat\Pi)
\\ &=
0.
\end{aligned}
\end{equation}
%
Only two functions are possible in the symmetry set \( \setlr{ \Psi(x), \hat\Pi \Psi(x) } \), since
%
\begin{equation}\label{eqn:qmLecture11:600}
\begin{aligned}
\hat\Pi^2 \Psi(x)
&= \hat\Pi \Psi(-x)
\\ &= \Psi(x).
\end{aligned}
\end{equation}
%
This symmetry severely restricts the possible solutions, making it so there can be only one dimensional forms of this problem with solutions that are either even or odd respectively
%
\begin{equation}\label{eqn:qmLecture11:620}
\begin{aligned}
\phi_e(x) &= \psi(x ) + \psi(-x) \\
\phi_o(x) &= \psi(x ) - \psi(-x).
\end{aligned}
\end{equation}
%
} % example

%\EndArticle

      %
% Copyright � 2015 Peeter Joot.  All Rights Reserved.
% Licenced as described in the file LICENSE under the root directory of this GIT repository.
%
%\input{../blogpost.tex}
%\renewcommand{\basename}{qmLecture12}
%\renewcommand{\dirname}{notes/phy1520/}
%\newcommand{\keywords}{PHY1520H}
%\input{../peeter_prologue_print2.tex}
%
%%\usepackage{phy1520}
%\usepackage{peeters_braket}
%\usepackage{peeters_layout_exercise}
%\usepackage{peeters_figures}
%\usepackage{mathtools}
%
%\beginArtNoToc
%\generatetitle{PHY1520H Graduate Quantum Mechanics.  Lecture 12: Symmetry (cont.).  Taught by Prof.\ Arun Paramekanti}
%%\chapter{Symmetry (cont.)}
%\label{chap:qmLecture12}
%
%\paragraph{Disclaimer}
%
%Peeter's lecture notes from class.  These may be incoherent and rough.
%
%These are notes for the UofT course PHY1520, Graduate Quantum Mechanics, taught by Prof. Paramekanti, covering \textchapref{{1}} \citep{sakurai2014modern} content.
%
\paragraph{Parity (review)}
%
The action of the parity operators on the position and momentum operators are
\begin{dmath}\label{eqn:qmLecture12:20}
\hat\Pi \hatx \hat\Pi = - \hatx,
\end{dmath}
and
\begin{dmath}\label{eqn:qmLecture12:40}
\hat\Pi \hatp \hat\Pi = - \hatp.
\end{dmath}
These are \underline{polar} vectors, in contrast to an \underline{axial} vector such as \( \BL = \Br \cross \Bp \).
%
\begin{dmath}\label{eqn:qmLecture12:60}
\hat\Pi^2 = 1,
\end{dmath}
%
\begin{dmath}\label{eqn:qmLecture12:80}
\Psi(x) \rightarrow \Psi(-x).
\end{dmath}
If \( \antisymmetric{\hat\Pi}{\hatH} = 0 \) then all the eigenstates are either
\begin{itemize}
\item even: \( \hat\Pi \) eigenvalue is \( + 1 \),
\item odd: \( \hat\Pi \) eigenvalue is \( - 1 \).
\end{itemize}
\paragraph{Note on parity in multiple dimensions}
A Hamiltonian can be constructed with parity symmetries in one or more directions.  For example, given a potential
%
\begin{dmath}\label{eqn:qmLecture12:81}
V(x,y) = a x + b x^2 + c y^2.
\end{dmath}
%
We don't have parity symmetry for \( \Bx = (x,y) \), but do have parity symmetry in the \( y \) direction.  Assuming a separated variables form for the wave function, say \( \psi(x,y) = X(x)Y(y) \), we can't say much about \( X \) on the grounds of symmetry considerations only, but know that \( Y \) has to be either an even or odd function.

%We are done with discrete symmetry operators for now.

\section{Translations.}
\index{translation!symmetry}
\index{translation!operator}
Define a (continuous) translation operator
%
\begin{dmath}\label{eqn:qmLecture12:100}
\hatT_\epsilon \ket{x} = \ket{x + \epsilon}.
\end{dmath}
The action of this operator is sketched in \cref{fig:lecture12:lecture12Fig1}.
\imageFigure{../figures/phy1520-quantum/lecture12Fig1}{Translation operation.}{fig:lecture12:lecture12Fig1}{0.1}
This is a unitary operator
%
\begin{equation}\label{eqn:qmLecture12:120}
\hatT_{-\epsilon} = \hatT_{\epsilon}^\dagger = \hatT_{\epsilon}^{-1}.
\end{equation}
In a position basis, the action of this operator is
%
\begin{equation}\label{eqn:qmLecture12:140}
\bra{x} \hatT_{\epsilon} \ket{\psi} = \braket{x-\epsilon}{\psi} = \psi(x - \epsilon)
\end{equation}
%
\begin{equation}\label{eqn:qmLecture12:160}
\Psi(x - \epsilon) \approx \Psi(x) - \epsilon \PD{x}{\Psi}
\end{equation}
%
\begin{equation}\label{eqn:qmLecture12:180}
\bra{x} \hatT_{\epsilon} \ket{\Psi}
= \braket{x}{\Psi} - \frac{\epsilon}{\Hbar} \bra{ x} i \hatp \ket{\Psi}
\end{equation}
%
\begin{equation}\label{eqn:qmLecture12:200}
\hatT_{\epsilon} \approx \lr{ 1 - i \frac{\epsilon}{\Hbar} \hatp }.
\end{equation}
A non-infinitesimal translation can be composed of many small translations, as sketched in \cref{fig:lecture12:lecture12Fig2}.
\imageFigure{../figures/phy1520-quantum/lecture12Fig2}{Composition of small translations.}{fig:lecture12:lecture12Fig2}{0.15}
For \( \epsilon \rightarrow 0, N \rightarrow \infty, N \epsilon = a \), the total translation operator is
%
\begin{dmath}\label{eqn:qmLecture12:220}
\hatT_{a}
= \hatT_{\epsilon}^N
= \lim_{\epsilon \rightarrow 0, N \rightarrow \infty, N \epsilon = a }
\lr{ 1 - \frac{\epsilon}{\Hbar} \hatp }^N
= e^{-i a \hatp/\Hbar}.
\end{dmath}
The momentum \( \hatp \) is called a ``Generator'' \index{generator!translation} of translations.  If a Hamiltonian \( H \) is translationally invariant, then
%
\begin{dmath}\label{eqn:qmLecture12:240}
\antisymmetric{\hatT_{a}}{H} = 0, \qquad \forall a.
\end{dmath}
%
This means that momentum will be a good quantum number
%
\begin{dmath}\label{eqn:qmLecture12:260}
\antisymmetric{\hatp}{H} = 0.
\end{dmath}
%
\section{Rotations.}
\index{infitesimal rotation}
Rotations form a non-Abelian group \index{Abelian group}, since the order of rotations \( \hatR_1 \hatR_2 \ne \hatR_2 \hatR_1 \).
Given a rotation acting on a ket
%
\begin{dmath}\label{eqn:qmLecture12:280}
\hatR \ket{\Br} = \ket{R \Br},
\end{dmath}
%
observe that the action of the rotation operator on a wave function is inverted
%
\begin{dmath}\label{eqn:qmLecture12:300}
\bra{\Br} \hatR \ket{\Psi}
=
\bra{R^{-1} \Br} \ket{\Psi}
= \Psi(R^{-1} \Br).
\end{dmath}
%
\makeexample{Z axis normal rotation}{example:qmLecture12:1}{

Consider an infinitesimal rotation about the z-axis as sketched in \cref{fig:lecture12:lecture12Fig3}.

\imageTwoFigures
{../figures/phy1520-quantum/lecture12Fig3}
{../figures/phy1520-quantum/lecture12Fig4}
{Rotation about z-axis.}{fig:lecture12:lecture12Fig3}{scale=0.1}
%
\begin{equation}\label{eqn:qmLecture12:320}
\begin{aligned}
x' &= x - \epsilon y \\
y' &= y + \epsilon y \\
z' &= z.
\end{aligned}
\end{equation}
The rotated wave function is
%
\begin{dmath}\label{eqn:qmLecture12:340}
\tilde{\Psi}(x,y,z)
= \Psi( x + \epsilon y, y - \epsilon x, z )
=
 \Psi( x, y, z )
+
\epsilon y
\mathLabelBox
[ labelstyle={below of=m\themathLableNode, below of=m\themathLableNode} ]
{\PD{x}{\Psi}}{\(i \hatp_x/\Hbar\)}
-
\epsilon x
\mathLabelBox
[ labelstyle={below of=m\themathLableNode, below of=m\themathLableNode} ]
{\PD{y}{\Psi}}{\(i \hatp_y/\Hbar\)}.
\end{dmath}
%
The state must then transform as
%
\begin{dmath}\label{eqn:qmLecture12:360}
\ket{\tilde{\Psi}}
=
\lr{
1
+ i \frac{\epsilon}{\Hbar} \haty \hatp_x
- i \frac{\epsilon}{\Hbar} \hatx \hatp_y
}
\ket{\Psi}.
\end{dmath}
%
Observe that the combination \( \hatx \hatp_y - \haty \hatp_x \) is the \( \hatL_z \) component of angular momentum \( \Lcap = \rcap \cross \pcap \), so the infinitesimal rotation can be written
%
%\begin{dmath}\label{eqn:qmLecture12:380}
\boxedEquation{eqn:qmLecture12:400}{
\hatR_z(\epsilon) \ket{\Psi}
=
\lr{ 1 - i \frac{\epsilon}{\Hbar} \hatL_z } \ket{\Psi}.
}
%\end{dmath}
%
For a finite rotation \( \epsilon \rightarrow 0, N \rightarrow \infty, \phi = \epsilon N \), the total rotation is
%
\begin{dmath}\label{eqn:qmLecture12:420}
\hatR_z(\phi)
=
\lr{ 1 - \frac{i \epsilon}{\Hbar} \hatL_z }^N,
\end{dmath}
%
or
%\begin{dmath}\label{eqn:qmLecture12:440}
\boxedEquation{eqn:qmLecture12:460}{
\hatR_z(\phi)
=
e^{-i \frac{\phi}{\Hbar} \hatL_z}.
}
%\end{dmath}
%
Note that \( \antisymmetric{\hatL_x}{\hatL_y} \ne 0 \).
} % example
By construction using Euler angles or any other method, a general rotation will include contributions from components of all the angular momentum operator, and will have the structure
%
%\begin{dmath}\label{eqn:qmLecture12:480}
\boxedEquation{eqn:qmLecture12:500}{
\hatR_\ncap(\phi)
=
e^{-i \frac{\phi}{\Hbar} \lr{ \Lcap \cdot \ncap }}.
}
%\end{dmath}
%
\paragraph{Rotationally invariant \( \hatH \).}
\index{rotation invariance}

Given a rotationally invariant Hamiltonian
%
\begin{dmath}\label{eqn:qmLecture12:520}
\antisymmetric{\hatR_\ncap(\phi)}{\hatH} = 0 \qquad \forall \ncap, \phi,
\end{dmath}
%
then every
%
\begin{dmath}\label{eqn:qmLecture12:540}
\antisymmetric{\BL \cdot \ncap}{\hatH} = 0,
\end{dmath}
%
or
\begin{dmath}\label{eqn:qmLecture12:560}
\antisymmetric{L_i}{\hatH} = 0,
\end{dmath}
%
Non-Abelian implies degeneracies in the spectrum.

\section{Time-reversal.}
\index{time reversal}
Imagine that we have something moving along a curve at time \( t = 0 \), and ending up at the final position at time \( t = t_f \), as sketched in \cref{fig:lecture12:lecture12Fig5}.
\imageFigure{../figures/phy1520-quantum/lecture12Fig5}{Time reversal trajectory.}{fig:lecture12:lecture12Fig5}{0.1}
Now imagine that we flip the direction of motion (i.e. flipping the velocity) and run time backwards so the final-time state becomes the initial state.
If the time reversal operator is designated \( \hat\Theta \), with operation
%
\begin{dmath}\label{eqn:qmLecture12:580}
\hat\Theta \ket{\Psi} = \ket{\tilde{\Psi}},
\end{dmath}
%
so that
%
\begin{dmath}\label{eqn:qmLecture12:600}
\hat\Theta^{-1} e^{-i \hatH t/\Hbar} \hat\Theta \ket{\Psi(t)} = \ket{\Psi(0)},
\end{dmath}
%
or
%
\begin{dmath}\label{eqn:qmLecture12:620}
\hat\Theta^{-1} e^{-i \hatH t/\Hbar} \hat\Theta \ket{\Psi(0)} = \ket{\Psi(-t)}.
\end{dmath}
%
%\EndArticle

      %
% Copyright � 2015 Peeter Joot.  All Rights Reserved.
% Licenced as described in the file LICENSE under the root directory of this GIT repository.
%
%\input{../blogpost.tex}
%\renewcommand{\basename}{qmLecture13}
%\renewcommand{\dirname}{notes/phy1520/}
%\newcommand{\keywords}{PHY1520H}
%\input{../peeter_prologue_print2.tex}
%
%%\usepackage{phy1520}
%\usepackage{peeters_braket}
%%\usepackage{peeters_layout_exercise}
%\usepackage{peeters_figures}
%\usepackage{mathtools}
%
%\beginArtNoToc
%\generatetitle{PHY1520H Graduate Quantum Mechanics.  Lecture 13: Time reversal (cont.), and angular momentum.  Taught by Prof.\ Arun Paramekanti}
%%\chapter{Time reversal (cont.)}
%\label{chap:qmLecture13}
%
%\paragraph{Disclaimer}
%
%Peeter's lecture notes from class.  These may be incoherent and rough.
%
%These are notes for the UofT course PHY1520, Graduate Quantum Mechanics, taught by Prof. Paramekanti, covering \textchapref{{4}}, \textchapref{{3}} \citep{sakurai2014modern} content.
%
\paragraph{Time reversal (cont.)}

Given a time reversed state
%
\begin{dmath}\label{eqn:qmLecture13:20}
\ket{\tilde{\Psi}(t)} = \Theta \ket{\Psi(0)}
\end{dmath}

which can alternately be written
%
\begin{equation}\label{eqn:qmLecture13:40}
\Theta^{-1} \ket{\tilde{\Psi}(t)} = \ket{\Psi(-t)} = e^{i \hatH t/\Hbar} \ket{\Psi(0)}
\end{equation}

The left hand side can be expanded as the evolution of the state as found at time \( -t \)
%
\begin{dmath}\label{eqn:qmLecture13:60}
\Theta^{-1} \ket{\tilde{\Psi}(t)}
=
\Theta^{-1} e^{-i \hatH t/\Hbar} \ket{\tilde{\Psi}(-t)}
=
\Theta^{-1} e^{-i \hatH t/\Hbar} \Theta \ket{\Psi(0)}.
\end{dmath}
%
To first order for a small time increment \( \delta t \), we have
%
\begin{dmath}\label{eqn:qmLecture13:80}
\lr{ 1 + i \frac{\hatH}{\Hbar} \delta t } \ket{\Psi(0)} =
\Theta^{-1} \lr{ 1 - i \frac{\hatH}{\Hbar} \delta t } \Theta \ket{\Psi(0)},
\end{dmath}
%
or
%
\begin{dmath}\label{eqn:qmLecture13:120}
i \frac{\hatH}{\Hbar} \delta t \ket{\Psi(0)}
=
\Theta^{-1} (- i) \frac{\hatH}{\Hbar} \delta t \Theta \ket{\Psi(0)}.
\end{dmath}
%
Since this holds for any state \( \ket{\Psi(0)} \), the time reversal operator satisfies
%
\begin{dmath}\label{eqn:qmLecture13:140}
i \hatH
=
\Theta^{-1} (- i) \hatH \Theta.
\end{dmath}
%
Note that the factors of \( i \) have not been canceled on purpose, since we are allowing for the time reversal operator to not necessarily commute with imaginary numbers.

There are two possible solutions

\begin{itemize}
\item If \( \Theta \) is unitary where \( \Theta i = i \Theta \), then
%
\begin{dmath}\label{eqn:qmLecture13:160}
\hatH
=
-\Theta^{-1} \hatH \Theta,
\end{dmath}
%
or
\begin{dmath}\label{eqn:qmLecture13:180}
\Theta \hatH
=
- \hatH \Theta.
\end{dmath}
%
Consider the implications of this on energy eigenstates
\begin{dmath}\label{eqn:qmLecture13:200}
\hatH \ket{\Psi_n} = E_n \ket{\Psi_n},
\end{dmath}
%
\begin{dmath}\label{eqn:qmLecture13:220}
\Theta \hatH \ket{\Psi_n} = E_n \Theta \ket{\Psi_n},
\end{dmath}
%
but
%
\begin{dmath}\label{eqn:qmLecture13:240}
-\hatH \Theta \ket{\Psi_n} = E_n \Theta \ket{\Psi_n},
\end{dmath}
%
or
%
\begin{dmath}\label{eqn:qmLecture13:260}
\hatH \lr{ \Theta \ket{\Psi_n}} = -E_n \lr{ \Theta \ket{\Psi_n} }.
\end{dmath}
%
This would mean that \( \lr{ \Theta \ket{\Psi_n}} \) is an eigenket of \( \hatH \), but with a negative energy eigenvalue.

\item \( \Theta \) is antiunitary, where \( \Theta i = -i \Theta \).

This time
\begin{dmath}\label{eqn:qmLecture13:280}
i \hatH = i \Theta^{-1} \hatH \Theta,
\end{dmath}
%
so
%
\begin{dmath}\label{eqn:qmLecture13:300}
\Theta \hatH = \hatH \Theta.
\end{dmath}
%
Acting on an energy eigenket, we've got
%
\begin{dmath}\label{eqn:qmLecture13:1400}
\Theta \hatH \ket{\Psi_n}
=
E_n \lr{ \Theta \ket{\Psi_n} },
\end{dmath}
%
and
\begin{dmath}\label{eqn:qmLecture13:1420}
\lr{ \hatH \Theta } \ket{\Psi_n}
=
\hatH \lr{ \Theta \ket{\Psi_n} },
\end{dmath}
%
so \( \Theta \ket{\Psi_n} \) is an eigenstate with energy \( E_n \).

\end{itemize}

\paragraph{What properties do we expect from \( \Theta \)?}

We expect
\index{time reversal!properties}
\begin{dmath}\label{eqn:qmLecture13:320}
\begin{aligned}
\hatx &\rightarrow \hatx \\
\hatp &\rightarrow -\hatp \\
\Lcap &\rightarrow -\Lcap
\end{aligned}
\end{dmath}

where we have a sign flip in the time dependent momentum operator (and therefore angular momentum), but not for position.  If we have
%
\begin{dmath}\label{eqn:qmLecture13:340}
\Theta^{-1} \hatx \Theta = \hatx,
\end{dmath}
%
if that's true, then how about the momentum operator in the position basis
\begin{dmath}\label{eqn:qmLecture13:360}
\Theta^{-1} \hatp \Theta
=
\Theta^{-1} \lr{ -i \Hbar \PD{x}{} } \Theta
=
\Theta^{-1} \lr{ -i \Hbar } \Theta \PD{x}{}
=
i \Hbar \Theta^{-1} \Theta \PD{x}{}
=
-\hatp.
\end{dmath}
%
How about the \( x,p \) commutator?  For that we have
%
\begin{dmath}\label{eqn:qmLecture13:380}
\Theta^{-1} \antisymmetric{\hatx}{\hatp} \Theta
=
\Theta^{-1} \lr{ i \Hbar } \Theta
=
-i \Hbar \Theta^{-1} \Theta
=
- \antisymmetric{\hatx}{\hatp}.
\end{dmath}
%
For the the angular momentum operators
%
\begin{dmath}\label{eqn:qmLecture13:420}
\hatL_i = \epsilon_{ijk} \hatr_j \hatp_k,
\end{dmath}
%
the time reversal operator should flip the sign due to its action on \( \hatp_k \).
%
%\begin{dmath}\label{eqn:qmLecture13:400}
%\antisymmetric{\hatL_i }{\hatL_j } = i \epsilon_{ijk} \hatL_k.
%\end{dmath}
%
%FIXME: lost his point about the angular momentum commutator here.

\paragraph{Time reversal acting on spin 1/2 (Fermions).  Attempt I.}

Consider two spin states \( \ket{\uparrow}, \ket{\downarrow} \).  What should the action of the time reversal operator on such a state be?  Let's (incorrectly) start by supposing that the time reversal operator effects are
%
\begin{dmath}\label{eqn:qmLecture13:440}
\begin{aligned}
\Theta \ket{\uparrow} &\questionEquals \ket{\downarrow} \\
\Theta \ket{\downarrow} &\questionEquals \ket{\uparrow}.
\end{aligned}
\end{dmath}
%
Given a general state
so that if
%
\begin{equation}\label{eqn:qmLecture13:740}
\ket{\Psi} = a \ket{\uparrow} + b \ket{\downarrow},
\end{equation}
%
the action of the time reversal operator would be
%
\begin{equation}\label{eqn:qmLecture13:760}
\Theta \ket{\Psi} = a^\conj \ket{\downarrow} + b^\conj \ket{\uparrow}.
\end{equation}
%
That action is:
%
\begin{equation}\label{eqn:qmLecture13:460}
\begin{aligned}
a \rightarrow b^\conj \\
b \rightarrow a^\conj
\end{aligned}
\end{equation}

Let's consider whether or not such an action a spin operator with properties
%
\begin{dmath}\label{eqn:qmLecture13:480}
\antisymmetric{\hatS_i}{\hatS_j} = i \epsilon_{ijk} \hatS_k.
\end{dmath}
%
produce the desired inversion of sign
%
\begin{dmath}\label{eqn:qmLecture13:500}
\Theta^{-1} \hatS_i \Theta = - \hatS_i.
\end{dmath}
%
The expectations of the spin operators (without any application of time reversal) are
%
\begin{dmath}\label{eqn:qmLecture13:1440}
\bra{\Psi} \hatS_x \ket{\Psi}
=
\frac{\Hbar}{2}
\lr{ a^\conj \bra{\uparrow} + b^\conj \bra{\downarrow} }
\sigma_x
\lr{ a \ket{\uparrow} + b \ket{\downarrow} }
=
\frac{\Hbar}{2}
\lr{ a^\conj \bra{\uparrow} + b^\conj \bra{\downarrow} }
\lr{ a \ket{\downarrow} + b \ket{\uparrow} }
=
\frac{\Hbar}{2}
\lr{ a^\conj b + b^\conj a },
\end{dmath}
%
\begin{dmath}\label{eqn:qmLecture13:1460}
\bra{\Psi} \hatS_y \ket{\Psi}
=
\frac{\Hbar}{2}
\lr{ a^\conj \bra{\uparrow} + b^\conj \bra{\downarrow} }
\sigma_y
\lr{ a \ket{\uparrow} + b \ket{\downarrow} }
=
\frac{i\Hbar}{2}
\lr{ a^\conj \bra{\uparrow} + b^\conj \bra{\downarrow} }
\lr{ a \ket{\downarrow} - b \ket{\uparrow} }
=
\frac{\Hbar}{2 i} \lr{ a^\conj b - b^\conj a },
\end{dmath}
%
\begin{dmath}\label{eqn:qmLecture13:1480}
\bra{\Psi} \hatS_z \ket{\Psi}
=
\frac{\Hbar}{2}
\lr{ a^\conj \bra{\uparrow} + b^\conj \bra{\downarrow} }
\sigma_z
\lr{ a \ket{\uparrow} - b \ket{\downarrow} }
=
\frac{\Hbar}{2} \lr{ \Abs{a}^2 - \Abs{b}^2 }
\end{dmath}

The time reversed actions are
%
\begin{dmath}\label{eqn:qmLecture13:1560}
\bra{\Psi} \Theta^{-1} \hatS_x \Theta \ket{\Psi}
=
\frac{\Hbar}{2}
\lr{ a^\conj \bra{\downarrow} + b^\conj \bra{\uparrow} }
\sigma_x
\lr{ a \ket{\downarrow} + b \ket{\uparrow} }
=
\frac{\Hbar}{2}
\lr{ a^\conj \bra{\downarrow} + b^\conj \bra{\uparrow} }
\lr{ a \ket{\uparrow} + b \ket{\downarrow} }
=
\frac{\Hbar}{2}
\lr{ a^\conj b + b^\conj a },
\end{dmath}
%
\begin{dmath}\label{eqn:qmLecture13:1580}
\bra{\Psi} \Theta^{-1} \hatS_y \Theta \ket{\Psi}
=
\frac{\Hbar}{2}
\lr{ a^\conj \bra{\downarrow} + b^\conj \bra{\uparrow} }
\sigma_y
\lr{ a \ket{\downarrow} + b \ket{\uparrow} }
=
\frac{i\Hbar}{2}
\lr{ a^\conj \bra{\downarrow} + b^\conj \bra{\uparrow} }
\lr{ -a \ket{\uparrow} + b \ket{\downarrow} }
=
\frac{\Hbar}{2 i} \lr{ -a^\conj b + b^\conj a },
\end{dmath}
%
\begin{dmath}\label{eqn:qmLecture13:1600}
\bra{\Psi} \Theta^{-1} \hatS_z \Theta \ket{\Psi}
=
\frac{\Hbar}{2}
\lr{ a^\conj \bra{\downarrow} + b^\conj \bra{\uparrow} }
\sigma_z
\lr{ a \ket{\downarrow} + b \ket{\uparrow} }
=
\frac{\Hbar}{2}
\lr{ a^\conj \bra{\downarrow} + b^\conj \bra{\uparrow} }
\lr{ -a \ket{\downarrow} + b \ket{\uparrow} }
=
\frac{\Hbar}{2} \lr{ -\Abs{a}^2 + \Abs{b}^2 }
\end{dmath}
%
%\begin{dmath}\label{eqn:qmLecture13:1500}
%\Theta^{-1} \hatS_z \Theta \ket{\uparrow}
%=
%\frac{\Hbar}{2} \Theta^{-1} \sigma_z \ket{\downarrow}
%=
%-\frac{\Hbar}{2} \Theta^{-1} \ket{\downarrow}
%=
%-\frac{\Hbar}{2} \ket{\uparrow},
%\end{dmath}
%
%\begin{dmath}\label{eqn:qmLecture13:1520}
%\Theta^{-1} \hatS_x \Theta \ket{\uparrow}
%=
%\frac{\Hbar}{2} \Theta^{-1} \sigma_x \ket{\downarrow}
%=
%\frac{\Hbar}{2} \Theta^{-1} \ket{\uparrow}
%=
%\frac{\Hbar}{2} \ket{\downarrow},
%\end{dmath}
%
%and
%
%\begin{dmath}\label{eqn:qmLecture13:1540}
%\Theta^{-1} \hatS_y \Theta \ket{\uparrow}
%=
%\frac{\Hbar}{2} \Theta^{-1} \sigma_y \ket{\downarrow}
%=
%-i \frac{\Hbar}{2} \Theta^{-1} \ket{\uparrow}
%=
%\frac{\Hbar}{2 i} \ket{\downarrow}.
%\end{dmath}
%
%Contrast this to the time reversed action on the spin operators
%
%\bra{\Psi} \Theta^{-1} \hatS_x \Theta \ket{\Psi}
%= \frac{\Hbar}{2} \lr{ b^\conj a + a b^\conj }
%
%\bra{\Psi} \Theta^{-1} \hatS_y \Theta \ket{\Psi}
%= \frac{\Hbar}{2 i} \lr{ b a^\conj - a b^\conj }
%
%\bra{\Psi} \Theta^{-1} \hatS_z \Theta \ket{\Psi}
%= \frac{\Hbar}{2} \lr{ \Abs{b}^2 - \Abs{a}^2 }

We see that this is not right, because the sign for the x component has not been flipped.
% (only the sign of the y component has).

\paragraph{Spin 1/2 (Fermions).  Attempt II.}

Again assuming
%
\begin{equation}\label{eqn:qmLecture13:580}
\ket{\Psi} = a \ket{\uparrow} + b \ket{\downarrow},
\end{equation}
%
now try the action
%
\begin{equation}\label{eqn:qmLecture13:780}
\Theta \ket{\Psi} = a^\conj \ket{\downarrow} - b^\conj \ket{\uparrow}.
\end{equation}
%
This is the action:
%
\begin{equation}\label{eqn:qmLecture13:600}
\begin{aligned}
a \rightarrow -b^\conj \\
b \rightarrow a^\conj
\end{aligned}
\end{equation}

The correct action of time reversal on the basis states (up to a phase choice) is
%
\boxedEquation{eqn:qmLecture13:620}{
%\begin{boxed}\label{eqn:qmLecture13:640}
\begin{aligned}
\Theta \ket{\uparrow} &= \ket{\downarrow} \\
\Theta \ket{\downarrow} &= -\ket{\uparrow} \\
\end{aligned}
%\end{boxed}
}

Note that acting the time reversal operator twice has the effects
%
\begin{dmath}\label{eqn:qmLecture13:660}
\Theta^2 \ket{\uparrow} = \Theta \ket{\downarrow} = - \ket{\uparrow}
\end{dmath}
\begin{dmath}\label{eqn:qmLecture13:680}
\Theta^2 \ket{\downarrow} = \Theta (-\ket{\uparrow}) = - \ket{\uparrow}.
\end{dmath}
%
We end up with the same state we started with, but with the opposite sign.  This means that as an operator

\index{time reversal!squared}
%\begin{dmath}\label{eqn:qmLecture13:700}
\boxedEquation{eqn:qmLecture13:700}{
\Theta^2 = -1.
}
%\end{dmath}
%
This is try for half integer particles (Fermions) \( S = 1/2, 3/2, 5/2, \cdots \), but for Bosons with integer spin \( S \).
%
%\begin{dmath}\label{eqn:qmLecture13:720}
\boxedEquation{eqn:qmLecture13:720}{
\Theta^2 = 1.
}
%\end{dmath}
%
\paragraph{Kramer's degeneracy for Spin 1/2 (Fermions)}

\index{spin half!time reversal}

Suppose we imagine there is state for which the action of the time reversal operator produces the same state, just different in phase.  Let
%
\begin{equation}\label{eqn:qmLecture13:800}
\ket{\psi}' = \Theta \ket{\psi}
= e^{i \delta} \ket{\psi}.
\end{equation}
%
For a Fermion we have
\begin{dmath}\label{eqn:qmLecture13:840}
\Theta^2 \ket{\psi} = -\ket{\psi},
\end{dmath}
%
but if such the time reversal action posited above is possible we also have
%
\begin{dmath}\label{eqn:qmLecture13:860}
\Theta^2 \ket{\psi}
=
\Theta e^{i \delta} \ket{\psi}
=
e^{-i \delta} \Theta \ket{\psi}
=
e^{-i \delta} e^{i \delta} \ket{\psi}
=
\ket{\psi}
\ne
- \ket{\psi}.
\end{dmath}
%
This is a contradiction, so we must have at least a two-fold degeneracy.  This is called \textAndIndex{Kramer's degeneracy}.  In the homework we will show that this is not the case for integer spin particles.

\paragraph{Time reversal implications for wave functions}

For spin and angular momentum states, the implications of time reversal on the states is worked out above.  If a spinless Hamiltonian has time reversal symmetry then the implication is really just the fact that the wave functions can be real valued.

      \section{Problems.}
         %
% Copyright � 2015 Peeter Joot.  All Rights Reserved.
% Licenced as described in the file LICENSE under the root directory of this GIT repository.
%
\makeoproblem{No Kramers theorem for spin-1.}{gradQuantum:problemSet5:1}{2015 ps5 p1}{
\index{Kramer's theorem}
\index{spin one}
\index{time reversal}
%\makesubproblem{}{gradQuantum:problemSet5:1a}

Consider a spin-1 particle. Even with time-reversal invariance, there is no Kramers theorem, so each eigenvalue of
a generic spin-1 Hamiltonian will be non-degenerate. Systematically construct a Hamiltonian which is time-reversal
invariant and obviously also Hermitian to illustrate this point, making clear the logic of your construction (i.e., why you are including terms which you are including).
} % makeproblem

\makeanswer{gradQuantum:problemSet5:1}{
\withproblemsetsParagraph{

One representation of the spin-one \( J_z \) operator is

\begin{dmath}\label{eqn:gradQuantumProblemSet5Problem1:20}
J_z
=
\Hbar
\begin{bmatrix}
1 & 0 & 0 \\
0 & 0 & 0 \\
0 & 0 & -1 \\
\end{bmatrix}
\end{dmath}

This operator changes sign under time reversal.  The \( L_z \) operator also changes sign under time reversal, and can be written

\begin{dmath}\label{eqn:gradQuantumProblemSet5Problem1:40}
L_z = -i \Hbar \PD{\phi}{}
\end{dmath}

A product of these will be time reversal invariant, but not Hermitian.  To make it Hermitian we can scale with an imaginary constant

\begin{dmath}\label{eqn:gradQuantumProblemSet5Problem1:60}
H
= -\frac{i}{2 m \rho^2} J_z L_z
= -\frac{i (-i \Hbar) \Hbar}{2m \rho^2}
\begin{bmatrix}
1 & 0 & 0 \\
0 & 0 & 0 \\
0 & 0 & -1 \\
\end{bmatrix} \PD{\phi}{}
=
-\frac{\Hbar^2}{2m \rho^2}
\begin{bmatrix}
\partial_\phi & 0 & 0 \\
0 & 0 & 0 \\
0 & 0 & -\partial_\phi \\
\end{bmatrix}.
\end{dmath}

A \( \ifrac{1}{2 m \rho^2} \) factor has been included to ensure the dimensions of this spin-one time-reversal invariant Hamiltonian also has the dimensions of energy.

This Hamiltonian can be solved directly using the trial function \( \psi = e^{-i E t/\Hbar} \tilde{\psi} \), which provides an eigenvalue equation

\begin{dmath}\label{eqn:gradQuantumProblemSet5Problem1:80}
E \tilde{\psi} = H \tilde{\psi},
\end{dmath}

leading to solutions that have a hyperbolic form

\begin{dmath}\label{eqn:gradQuantumProblemSet5Problem1:100}
\psi \propto
\begin{bmatrix}
C e^{-2 m \rho^2 E \phi/\Hbar^2} \\
0 \\
C' e^{2 m \rho^2 E \phi/\Hbar^2} \\
\end{bmatrix}
e^{-i E t/\Hbar}
.
\end{dmath}

Consistent with the no-Kramer's theorem result for a spin-one Hamiltonian, there is a continuum of energy solutions for this Hamiltonian, with nothing to constrain them.
}
}

         %
% Copyright � 2015 Peeter Joot.  All Rights Reserved.
% Licenced as described in the file LICENSE under the root directory of this GIT repository.
%
\makeoproblem{Boosts.}{gradQuantum:problemSet5:2}{2015 ps5 p2}{
\index{translation!generator}
\index{boost}

The momentum operator \( \hatp \) was shown, in class, to act as the generator of space translations. Show by following the exact same steps that the position operator \( \hatx \) acts as the generator of momentum boosts (i.e., \( \hatx \) is a generator of `momentum translation').

%\makesubproblem{}{gradQuantum:problemSet5:2a}
} % makeproblem

\makeanswer{gradQuantum:problemSet5:2}{
\withproblemsetsParagraph{
%\makeSubAnswer{}{gradQuantum:problemSet5:2a}

Borrowing the same notation as in class, for an infinitesimal change in momentum \( \delta \tilde{p} \), define a momentum translation operator with the action on a momentum space state of
%
\begin{dmath}\label{eqn:gradQuantumProblemSet5Problem2:20}
\hatT_{\delta \tilde{p}} \ket{p}
=
\ket{p + \delta \tilde{p}}.
\end{dmath}

A wave function matrix element for this momentum translation is
\begin{dmath}\label{eqn:gradQuantumProblemSet5Problem2:40}
\bra{p} \hatT_{\delta \tilde{p}} \ket{\psi}
=
\braket{p - \delta \tilde{p}}{\psi}
=
\psi_p(p - \delta \tilde{p})
\approx
\psi_p(p) - \delta \tilde{p} \PD{p}{\psi_p(p)}.
\end{dmath}

Since the momentum space representation of the position operator is
\index{position operator!momentum space representation}
%
\begin{dmath}\label{eqn:gradQuantumProblemSet5Problem2:60}
x \dotEquals i \Hbar \PD{p}{},
\end{dmath}

we have
%
\begin{dmath}\label{eqn:gradQuantumProblemSet5Problem2:80}
\bra{p} \hatT_{\delta \tilde{p}} \ket{\psi}
\approx
\psi_p(p) - \delta \tilde{p} \frac{x}{i \Hbar} \psi_p(p)
=
\lr{ 1 + \frac{\delta \tilde{p} i x}{\Hbar} } \psi_p(p)
\end{dmath}

Given a finite change of momentum \( \tilde{p} = N \delta \tilde{p} \), the matrix element has the limiting form
%
\begin{dmath}\label{eqn:gradQuantumProblemSet5Problem2:100}
\bra{p} \hatT_{\tilde{p}} \ket{\psi}
= \lim_{N \rightarrow \infty, \delta \tilde{p} \rightarrow 0 }
\lr{ 1 + \frac{\tilde{p} i x}{N \Hbar} }^N \psi_p(p)
=
e^{ i \tilde{p} x/\Hbar } \psi_p(p),
\end{dmath}

showing that \( x \) is the generator of the momentum translation operator
%
%\begin{dmath}\label{eqn:gradQuantumProblemSet5Problem2:120}
\boxedEquation{eqn:gradQuantumProblemSet5Problem2:140}{
\hatT_{\tilde{p}} = e^{ i \tilde{p} x/\Hbar }.
}
%\end{dmath}
}
}

         %
% Copyright � 2015 Peeter Joot.  All Rights Reserved.
% Licenced as described in the file LICENSE under the root directory of this GIT repository.
%
\makeoproblem{Simultaneous eigenstates.}{gradQuantum:problemSet5:3}{2015 ps5 p3}{
\index{simultaneous eigenstate}
\index{anticommuting}
\index{parity}
%\makesubproblem{}{gradQuantum:problemSet5:3a}
%
A quantum state \( \ket{\Psi} \) is a simultaneous eigenstate of two anticommuting Hermitian operators \( A, B \), with \( A B + B A = 0\).
What can you say about the eigenvalues of \( A, B \) for the state \(\ket{\Psi}\)? Illustrate your point using the parity operator and the momentum operator.
} % makeproblem
%
\makeanswer{gradQuantum:problemSet5:3}{
\withproblemsetsParagraph{

For the eigenvalues of the respective states, let
%
\begin{equation}\label{eqn:gradQuantumProblemSet5Problem3:20}
A \ket{\Psi} = a \ket{\Psi},
\end{equation}
%
and
\begin{equation}\label{eqn:gradQuantumProblemSet5Problem3:40}
B \ket{\Psi} = b \ket{\Psi}.
\end{equation}
%
The action of the anticommutator on this state is
%
\begin{equation}\label{eqn:gradQuantumProblemSet5Problem3:60}
\begin{aligned}
\lr{ A B + B A } \ket{\Psi}
&=
\lr{ a b + b a } \ket{\Psi}
\\ &=
0.
\end{aligned}
\end{equation}
%
This means the product \( 2 a b \), is zero, and that at least one of \( a \) or \( b \) must be zero.

The concrete example of the parity operator \( \hat\Pi \) and the momentum operator \( \hatp \) demonstrates this.
Because the eigenvalues of the parity operator can only be \( \pm 1 \), if such a simultaneous eigenstate exists in the momentum basis, it must have a \( p = 0 \) eigenvalue.  To see this explicitly, assume there is some value \( p \) for which \( \ket{p} \) is a simultaneous eigenstate of \( \hat\Pi \) and \( \hatp \).  First consider
%
\begin{equation}\label{eqn:gradQuantumProblemSet5Problem3:80}
\begin{aligned}
\hat\Pi \hatp \ket{p}
&=
p \hat\Pi \ket{p}
\\ &=
- p \ket{p}.
\end{aligned}
\end{equation}
%
Since these operators anticommute
%
\begin{equation}\label{eqn:gradQuantumProblemSet5Problem3:100}
\hat\Pi \hatp = - \hatp \hat\Pi,
\end{equation}
%
we must also have
%
\begin{equation}\label{eqn:gradQuantumProblemSet5Problem3:120}
\begin{aligned}
\hat\Pi \hatp \ket{p}
&= - \hatp \hat\Pi \ket{p}
\\ &= \hatp \ket{p}
\\ &= p \ket{p}.
\end{aligned}
\end{equation}
%
The only way this can be satisfied is when the eigenvalue of this state is \( p = 0 \).
}
}

         %
% Copyright � 2015 Peeter Joot.  All Rights Reserved.
% Licenced as described in the file LICENSE under the root directory of this GIT repository.
%
\makeoproblem{Angular momentum.}{gradQuantum:problemSet5:4}{2015 ps5 p4}{
\index{angular momentum}
\index{time reversal}
\makesubproblem{}{gradQuantum:problemSet5:4a}

What is the time-reversed version of \( \calD(R) \ket{ j, m}\)?
\makesubproblem{}{gradQuantum:problemSet5:4b}
Prove that time-reversal implies \( \Theta \ket{j , m} = (-1)^{m} \ket{j, -m}\).

} % makeproblem

\makeanswer{gradQuantum:problemSet5:4}{
\withproblemsetsParagraph{
\makeSubAnswer{}{gradQuantum:problemSet5:4a}

Assuming the rotation is around the normal \( \ncap = (n_1, n_2, n_3) \) by \( \phi \) radians, the action of the time reversal operator on the rotated state can be expanded in series

\begin{dmath}\label{eqn:gradQuantumProblemSet5Problem4:20}
\Theta \calD(R) \ket{j, m}
=
\Theta e^{-i \BJ \cdot \ncap \phi/\Hbar} \ket{j, m}
=
\sum_{k = 0}^\infty \inv{k!} \lr{\frac{-i \phi}{\Hbar}}^k \lr{ \BJ \cdot \ncap }^k \ket{j, m}
=
\sum_{k = 0}^\infty \inv{k!} \lr{\frac{-i \phi}{\Hbar}}^k
\Theta \lr{ \BJ \cdot \ncap }^k
\ket{j, m}.
\end{dmath}

For any \( J_i \), we know that \( \Theta J_i = -J_i \Theta \), so

\begin{dmath}\label{eqn:gradQuantumProblemSet5Problem4:40}
\Theta (J_m n_m)^k
=
-J_r n_r \Theta (J_m n_m)^{k-1}
=
(-J_r n_r)^2 \Theta (J_m n_m)^{k-2}
=
\cdots
=
(-J_r n_r)^k \Theta.
\end{dmath}

This effectively flips the sign of the rotation angle, so we have

\begin{dmath}\label{eqn:gradQuantumProblemSet5Problem4:60}
\Theta \calD(R) \ket{j, m}
=
\calD(R^{-1}) \Theta \ket{j, m}.
\end{dmath}

\makeSubAnswer{}{gradQuantum:problemSet5:4b}

%Using spherical harmonics \( Y_l^m \) \citep{sakurai2014modern} shows that for orbital angular momentum we have \( \Theta \ket{l m} = (-1)^m \ket{l, -m} \).
For general total angular momentum states \( \BJ + \BL + \BS \), we can utilize the postulated properties of the time reversal operator.  In particular

\begin{dmath}\label{eqn:gradQuantumProblemSet5Problem4:80}
\Theta J_i = - J_i \Theta.
\end{dmath}

Considering each of \( J_z \) and \( J_{\pm} \) in turn, we first have

\begin{dmath}\label{eqn:gradQuantumProblemSet5Problem4:100}
\Theta J_z \ket{j, m}
=
\Hbar m \Theta \ket{j, m}
=
- J_z \Theta \ket{j, m},
\end{dmath}

or

\begin{dmath}\label{eqn:gradQuantumProblemSet5Problem4:120}
J_z \lr{ \Theta \ket{j, m} } = - m \Hbar \lr{ \Theta \ket{j, m} },
\end{dmath}

This means that \( \Theta \ket{j, m} \) is an eigenket of \( J_z \) with eigenvalue \( - m \Hbar \), so that we must have

\begin{dmath}\label{eqn:gradQuantumProblemSet5Problem4:140}
\Theta \ket{j, m} = c_m \ket{j, -m} .
\end{dmath}

The interplay of the ladder and time-reversal operators can be used to determine the factor \( c_m \).  Note that

\begin{dmath}\label{eqn:gradQuantumProblemSet5Problem4:160}
\Theta J_{+} \Theta^{-1}
=
\Theta \lr{ J_x + i J_y } \Theta^{-1}
=
-J_x - i \Theta J_y \Theta^{-1}
=
-J_x + i J_y
=
-\lr{ J_x - i J_y }
=
-J_{-},
\end{dmath}

and

\begin{dmath}\label{eqn:gradQuantumProblemSet5Problem4:180}
\Theta J_{-} \Theta^{-1}
=
\Theta \lr{ J_x - i J_y } \Theta^{-1}
=
-J_x + i \Theta J_y \Theta^{-1}
=
-J_x - i J_y
=
-\lr{ J_x + i J_y }
=
-J_{+},
\end{dmath}

or
\begin{equation}\label{eqn:gradQuantumProblemSet5Problem4:200}
\begin{aligned}
J_{-} \Theta &= -\Theta J_{+} \\
J_{+} \Theta &= -\Theta J_{-}.
\end{aligned}
\end{equation}

Acting with the raising operator on a time reversed state we have

\begin{dmath}\label{eqn:gradQuantumProblemSet5Problem4:220}
J_{+} \Theta \ket{j, m}
=
J_{+} c_m \ket{j, -m}
=
\Hbar \sqrt{ (j-(-m))(j+(-m)+1) } c_{m} \ket{j, -m + 1}.
\end{dmath}

But we must also have

\begin{dmath}\label{eqn:gradQuantumProblemSet5Problem4:240}
J_{+} \Theta \ket{j, m}
=
-\Theta J_{-} \ket{j, m}
=
-\Theta \Hbar \sqrt{(j+m)(j-m+1)} \ket{j, m - 1}
=
- \Hbar \sqrt{(j+m)(j-m+1)} c_{m-1} \ket{j, -m + 1}.
\end{dmath}

This provides a recurrence relation for the undetermined \( c_m \) factors

\begin{dmath}\label{eqn:gradQuantumProblemSet5Problem4:260}
c_m = - c_{m-1} = (-1)^k c_{m-k}
\end{dmath}

In particular for \( m -k = -j \), this is

\begin{dmath}\label{eqn:gradQuantumProblemSet5Problem4:280}
c_m = (-1)^{m + j} c_{-j}.
\end{dmath}

Acting twice with the time reversal operator

\begin{dmath}\label{eqn:gradQuantumProblemSet5Problem4:300}
\Theta^2 \ket{j, m}
=
\Theta (-1)^{m + j} c_{-j} \ket{j, -m}
=
(-1)^{ m -m + 2 j} c_{-j}^\conj c_{-j} \ket{j, m}
=
 (-1)^{2 j} \Abs{c_{-j}}^2 \ket{j, m}.
\end{dmath}

The \( c_{-j} \) factor can be absorbed into the definition of the time reversal operator, so that, up to a phase factor, we have

%\begin{dmath}\label{eqn:gradQuantumProblemSet5Problem4:320}
\boxedEquation{eqn:gradQuantumProblemSet5Problem4:340}{
\Theta \ket{j, m} = (-1)^{m} \ket{j, m},
}
%\end{dmath}

and
%\begin{dmath}\label{eqn:gradQuantumProblemSet5Problem4:360}
\boxedEquation{eqn:gradQuantumProblemSet5Problem4:380}{
\Theta^2 = (-1)^{2 j}.
}
%\end{dmath}
}
}

         % pr 4.1
         %
% Copyright � 2015 Peeter Joot.  All Rights Reserved.
% Licenced as described in the file LICENSE under the root directory of this GIT repository.
%
%{
%\input{../blogpost.tex}
%\renewcommand{\basename}{noninteractingParticlesInABox}
%\renewcommand{\dirname}{notes/phy1520/}
%%\newcommand{\dateintitle}{}
%%\newcommand{\keywords}{}
%
%\input{../peeter_prologue_print2.tex}
%
%\usepackage{peeters_layout_exercise}
%\usepackage{peeters_braket}
%\usepackage{peeters_figures}
%
%\beginArtNoToc
%
%\generatetitle{Non-interacting particles in a box}
%%\chapter{Non-interacting particles in a box}
%%\label{chap:noninteractingParticlesInABox}

\makeoproblem{Non-interacting particles in a box.}{problem:noninteractingParticlesInABox:1}{\citep{sakurai2014modern} pr. 4.1}{

\index{spin half}
\index{distinguishable particles}
\index{particle in a box}
Calculate the three lowest energy levels and their degeneracies for equal mass distinguishable spin half particles in a box of length \( L \).

Consider

\makesubproblem{}{problem:noninteractingParticlesInABox:1:a}
Two particles.
\makesubproblem{}{problem:noninteractingParticlesInABox:1:b}
Three particles.
\makesubproblem{}{problem:noninteractingParticlesInABox:1:c}
Four particles.

} % problem

\makeanswer{problem:noninteractingParticlesInABox:1}{

\makeSubAnswer{}{problem:noninteractingParticlesInABox:1:a}
The problem statement doesn't include the dimensionality of the box.  The simplest case is the one dimensional box, for which the wave function of one particle is
%
\begin{dmath}\label{eqn:noninteractingParticlesInABox:20}
\psi_1(x) = \sqrt{\frac{2}{L}} \sin\lr{ \frac{n \pi x}{L} },
\end{dmath}

and the energy of that particle is
%
\begin{dmath}\label{eqn:noninteractingParticlesInABox:40}
E = \inv{2 m} \lr{ \frac{\Hbar \pi}{L} }^2 n^2.
\end{dmath}

If the box is two dimensional the energy is
%
\begin{dmath}\label{eqn:noninteractingParticlesInABox:60}
E = \inv{2 m} \lr{ \frac{\Hbar \pi}{L} }^2 \lr{ n_1^2 + n_2^2 },
\end{dmath}

and if it's a 3D box, we have
%
\begin{dmath}\label{eqn:noninteractingParticlesInABox:80}
E = \inv{2 m} \lr{ \frac{\Hbar \pi}{L} }^2 \lr{ n_1^2 + n_2^2 + n_3^2}.
\end{dmath}

Suppose we are considering the 3D box.  In statistical mechanics when we are considering particles Fermions, they are indistinguishable, and thus not allowed to share the same spin state at a given energy level.  However, for distinguishable particles, that restriction doesn't exist, and we can have two (or more) such particles in the lowest order energy state.  The lowest such energy is
%
\begin{dmath}\label{eqn:noninteractingParticlesInABox:100}
E_{1,1,1 ;1,1,1}
=
\inv{2 m} \lr{ \frac{\Hbar \pi}{L} }^2 \lr{ 6 \times 1^2 }
=
\frac{6}{2 m} \lr{ \frac{\Hbar \pi}{L} }^2.
\end{dmath}

The particle spin states can be any of \( \ket{++}, \ket{+-}, \ket{-+}, \ket{--} \), so there is a four way degeneracy in the ground state.

the next lowest energy level is
%
\begin{dmath}\label{eqn:noninteractingParticlesInABox:120}
E_{1,1,2 ;1,1,1}
=
\inv{2 m} \lr{ \frac{\Hbar \pi}{L} }^2 \lr{ 5 \times 1^2 + 2^2 }
=
\frac{9}{2 m} \lr{ \frac{\Hbar \pi}{L} }^2,
\end{dmath}

where there are \( \binom{6}{1} = 6 \) ways to pick such a state for each variation of spin, for a total \( 6 \times 4 = 24 \) way degeneracy.  Finally, since \( 2^2 + 2^2 < 3^2 + 1^2 \), the next lowest energy level is
%
\begin{dmath}\label{eqn:noninteractingParticlesInABox:140}
E_{1,2,2 ;1,1,1}
=
\inv{2 m} \lr{ \frac{\Hbar \pi}{L} }^2 \lr{ 4 \times 1^2 + 2 \times 2^2 }
=
\frac{12}{2 m} \lr{ \frac{\Hbar \pi}{L} }^2,
\end{dmath}

with a \( \binom{6}{2} \times 4 = 15 \times 4 = 60 \) way degeneracy for this energy level.

\makeSubAnswer{}{problem:noninteractingParticlesInABox:1:b}

For three particles (the two particle case wasn't actually in the problem statement, but seemed an easier starting place), the lowest energy state for a 3D box is
%
\begin{dmath}\label{eqn:noninteractingParticlesInABox:160}
E
=
\inv{2 m} \lr{ \frac{\Hbar \pi}{L} }^2 \lr{ 9 \times 1^2 }
=
\frac{9}{2 m} \lr{ \frac{\Hbar \pi}{L} }^2.
\end{dmath}

There are now \( 2^3 = 8 \) variations of spin \( \ket{+++}, \ket{++-}, \cdots \), so the ground state is 8-way degenerate.  Next up is
%
\begin{dmath}\label{eqn:noninteractingParticlesInABox:180}
E
=
\inv{2 m} \lr{ \frac{\Hbar \pi}{L} }^2 \lr{ 8 \times 1^2 + 2^2 }
=
\frac{12}{2 m} \lr{ \frac{\Hbar \pi}{L} }^2,
\end{dmath}

where there is a \( \binom{9}{1} \times 8 = 9 \times 8 = 72 \) way degeneracy in this energy level.  Finally, the next lowest energy level is
%
\begin{dmath}\label{eqn:noninteractingParticlesInABox:200}
E
=
\inv{2 m} \lr{ \frac{\Hbar \pi}{L} }^2 \lr{ 7 \times 1^2 + 2 \times 2^2 }
=
\frac{15}{2 m} \lr{ \frac{\Hbar \pi}{L} }^2,
\end{dmath}

with a \( \binom{9}{2} \times 8 = 36 \times 8 = 288 \) way degeneracy for this energy level.

\makeSubAnswer{}{problem:noninteractingParticlesInABox:1:c}

For four particles the lowest energy state for a 3D box is
%
\begin{dmath}\label{eqn:noninteractingParticlesInABox:220}
E
=
\inv{2 m} \lr{ \frac{\Hbar \pi}{L} }^2 \lr{ 12 \times 1^2 }
=
\frac{12}{2 m} \lr{ \frac{\Hbar \pi}{L} }^2.
\end{dmath}

There are now \( 2^4 = 16 \) variations of spin \( \ket{++++}, \ket{+++-}, \cdots \), so the ground state is 16-way degenerate.  For the second level
%
\begin{dmath}\label{eqn:noninteractingParticlesInABox:240}
E
=
\inv{2 m} \lr{ \frac{\Hbar \pi}{L} }^2 \lr{ 11 \times 1^2 + 2^2 }
=
\frac{15}{2 m} \lr{ \frac{\Hbar \pi}{L} }^2,
\end{dmath}

where there is a \( \binom{12}{1} \times 16 = 12 \times 16 = 192 \) way degeneracy in this energy level.  Finally, the next lowest energy level is
%
\begin{dmath}\label{eqn:noninteractingParticlesInABox:260}
E
=
\inv{2 m} \lr{ \frac{\Hbar \pi}{L} }^2 \lr{ 7 \times 1^2 + 2 \times 2^2 }
=
\frac{15}{2 m} \lr{ \frac{\Hbar \pi}{L} }^2,
\end{dmath}

with a \( \binom{12}{2} \times 16 = 66 \times 16 = 1056 \) way degeneracy for this energy level.

} % answer

%}
%\EndArticle

         % pr 4.2
         %
% Copyright � 2015 Peeter Joot.  All Rights Reserved.
% Licenced as described in the file LICENSE under the root directory of this GIT repository.
%
%{
%\input{../blogpost.tex}
%\renewcommand{\basename}{symmetryOperatorCommutators}
%\renewcommand{\dirname}{notes/phy1520/}
%%\newcommand{\dateintitle}{}
%%\newcommand{\keywords}{}
%
%\input{../peeter_prologue_print2.tex}
%
%\usepackage{peeters_layout_exercise}
%\usepackage{peeters_braket}
%\usepackage{peeters_figures}
%\usepackage{enumerate}
%\usepackage{macros_cal}
%
%\beginArtNoToc
%
%\generatetitle{Commutators for some symmetry operators}
%\chapter{Commutators for some symmetry operators}
%\label{chap:symmetryOperatorCommutators}

\makeoproblem{Commutators for some symmetry operators.}{problem:symmetryOperatorCommutators:1}{\citep{sakurai2014modern} pr. 4.2}{

If \( \calT_\Bd \), \( \calD(\ncap, \phi) \), and \( \pi \) denote the translation, rotation, and parity operators respectively.  Which of the following commute and why

\makesubproblem{}{problem:symmetryOperatorCommutators:1:a}
\( \calT_\Bd \) and \( \calT_{\Bd'} \), translations in different directions.
\makesubproblem{}{problem:symmetryOperatorCommutators:1:b}
\( \calD(\ncap, \phi) \) and \( \calD(\ncap', \phi') \), rotations in different directions.
\makesubproblem{}{problem:symmetryOperatorCommutators:1:c}
\( \calT_\Bd \) and \( \pi \).
\makesubproblem{}{problem:symmetryOperatorCommutators:1:d}
\( \calD(\ncap,\phi)\) and \( \pi \).

} % problem

\makeanswer{problem:symmetryOperatorCommutators:1}{

\makeSubAnswer{}{problem:symmetryOperatorCommutators:1:a}

Consider
\begin{dmath}\label{eqn:symmetryOperatorCommutators:20}
\calT_\Bd \calT_{\Bd'} \ket{\Bx}
=
\calT_\Bd \ket{\Bx + \Bd'}
=
\ket{\Bx + \Bd' + \Bd},
\end{dmath}

and the reverse application of the translation operators
\begin{dmath}\label{eqn:symmetryOperatorCommutators:40}
\calT_{\Bd'} \calT_{\Bd} \ket{\Bx}
=
\calT_{\Bd'} \ket{\Bx + \Bd}
=
\ket{\Bx + \Bd + \Bd'}
=
\ket{\Bx + \Bd' + \Bd}.
\end{dmath}

so we see that

\begin{dmath}\label{eqn:symmetryOperatorCommutators:60}
\antisymmetric{\calT_\Bd}{\calT_{\Bd'}} \ket{\Bx} = 0,
\end{dmath}

for any position state \( \ket{\Bx} \), and therefore in general they commute.

\makeSubAnswer{}{problem:symmetryOperatorCommutators:1:b}

That rotations do not commute when they are in different directions (like any two orthogonal directions) need not be belaboured.

\makeSubAnswer{}{problem:symmetryOperatorCommutators:1:c}
We have
\begin{dmath}\label{eqn:symmetryOperatorCommutators:80}
\calT_\Bd \pi \ket{\Bx}
=
\calT_\Bd \ket{-\Bx}
=
\ket{-\Bx + \Bd},
\end{dmath}

yet
\begin{dmath}\label{eqn:symmetryOperatorCommutators:100}
\pi \calT_\Bd \ket{\Bx}
=
\pi \ket{\Bx + \Bd}
=
\ket{-\Bx - \Bd}
\ne
\ket{-\Bx + \Bd}.
\end{dmath}

so, in general \( \antisymmetric{\calT_\Bd}{\pi} \ne 0 \).

\makeSubAnswer{}{problem:symmetryOperatorCommutators:1:d}

We have

\begin{dmath}\label{eqn:symmetryOperatorCommutators:120}
\pi \calD(\ncap, \phi) \ket{\Bx}
=
\pi \calD(\ncap, \phi) \pi^\dagger \pi \ket{\Bx}
=
\pi \calD(\ncap, \phi) \pi^\dagger \pi \ket{\Bx}
=
\pi \lr{ \sum_{k=0}^\infty \frac{(-i \BJ \cdot \ncap)^k}{k!} } \pi^\dagger \pi \ket{\Bx}
=
\sum_{k=0}^\infty \frac{(-i (\pi \BJ \pi^\dagger) \cdot (\pi \ncap \pi^\dagger) )^k}{k!} \pi \ket{\Bx}
=
\sum_{k=0}^\infty \frac{(-i \BJ \cdot \ncap)^k}{k!} \pi \ket{\Bx}
=
\calD(\ncap, \phi) \pi \ket{\Bx},
\end{dmath}

so \( \antisymmetric{\calD(\ncap, \phi)}{\pi} \ket{\Bx} = 0 \), for any position state \( \ket{\Bx} \), and therefore these operators commute in general.
} % answer

%}
%\EndArticle

         % pr 4.7
         %
% Copyright � 2015 Peeter Joot.  All Rights Reserved.
% Licenced as described in the file LICENSE under the root directory of this GIT repository.
%
%{
%\input{../blogpost.tex}
%\renewcommand{\basename}{timeReversalPlaneWaveAndSpinor}
%\renewcommand{\dirname}{notes/phy1520/}
%%\newcommand{\dateintitle}{}
%%\newcommand{\keywords}{}
%
%\input{../peeter_prologue_print2.tex}
%
%\usepackage{peeters_layout_exercise}
%\usepackage{peeters_braket}
%\usepackage{peeters_figures}
%\usepackage{enumerate}
%\usepackage{macros_qed}
%
%\beginArtNoToc
%
%\generatetitle{Plane wave and spinor under time reversal}
%\chapter{Plane wave and spinor under time reversal}
%\label{chap:timeReversalPlaneWaveAndSpinor}
%\paragraph{Q: \citep{sakurai2014modern} pr 4.7}

\makeoproblem{Plane wave and spinor under time reversal.}{problem:timeReversalPlaneWaveAndSpinor:1}{\citep{sakurai2014modern} pr. 4.7}{
\index{time reversal!plane wave}
\index{time reversal!spinor}

\makesubproblem{}{problem:timeReversalPlaneWaveAndSpinor:1:a}
Find the time reversed form of a spinless plane wave state in three dimensions.

\makesubproblem{}{problem:timeReversalPlaneWaveAndSpinor:1:b}
For the eigenspinor of \( \Bsigma \cdot \ncap \) expressed in terms of polar and azimuthal angles \( \beta\) and \( \gamma \), show that \( -i \sigma_y \chi^\conj(\ncap) \) has the reversed spin direction.

} % problem

\makeanswer{problem:timeReversalPlaneWaveAndSpinor:1}{

\makeSubAnswer{}{problem:timeReversalPlaneWaveAndSpinor:1:a}

The Hamiltonian for a plane wave is
%
\begin{equation}\label{eqn:timeReversalPlaneWaveAndSpinor:20}
H = \frac{\Bp^2}{2m} = i \PD{t}.
\end{equation}
%
Under time reversal the momentum side transforms as
%
\begin{dmath}\label{eqn:timeReversalPlaneWaveAndSpinor:40}
\Theta \frac{\Bp^2}{2m} \Theta^{-1}
=
\frac{\lr{ \Theta \Bp \Theta^{-1}} \cdot \lr{ \Theta \Bp \Theta^{-1}} }{2m}
=
\frac{(-\Bp) \cdot (-\Bp)}{2m}
=
\frac{\Bp^2}{2m}.
\end{dmath}
%
The time derivative side of the equation is also time reversal invariant
\begin{dmath}\label{eqn:timeReversalPlaneWaveAndSpinor:60}
\Theta i \PD{t}{} \Theta^{-1}
=
\Theta i \Theta^{-1} \Theta \PD{t}{} \Theta^{-1}
=
-i \PD{(-t)}{}
=
i \PD{t}{}.
\end{dmath}
%
Solutions to this equation are linear combinations of
%
\begin{dmath}\label{eqn:timeReversalPlaneWaveAndSpinor:80}
\psi(\Bx, t) = e^{i \Bk \cdot \Bx - i E t/\Hbar},
\end{dmath}
%
where \( \Hbar^2 \Bk^2/2m = E \), the energy of the particle.  Under time reversal we have
%
\begin{dmath}\label{eqn:timeReversalPlaneWaveAndSpinor:100}
\psi(\Bx, t)
\rightarrow e^{-i \Bk \cdot \Bx + i E (-t)/\Hbar}
= \lr{ e^{i \Bk \cdot \Bx - i E (-t)/\Hbar} }^\conj
=
\psi^\conj(\Bx, -t)
\end{dmath}
%
\makeSubAnswer{}{problem:timeReversalPlaneWaveAndSpinor:1:b}

The text uses a requirement for time reversal of spin states to show that the Pauli matrix form of the time reversal operator is
%
\begin{dmath}\label{eqn:timeReversalPlaneWaveAndSpinor:120}
\Theta = -i \sigma_y \eta K,
\end{dmath}
%
where \( K \) is a complex conjugating operator, and \( \eta \) is a phase factor with \( \Abs{\eta}^2 = 1 \).  The form of the spin up state used in that demonstration was
%
\begin{dmath}\label{eqn:timeReversalPlaneWaveAndSpinor:140}
\ket{\ncap ; +}
= e^{-i S_z \beta/\Hbar} e^{-i S_y \gamma/\Hbar} \ket{+}
= e^{-i \sigma_z \beta/2} e^{-i \sigma_y \gamma/2} \ket{+}
= \lr{ \cos(\beta/2) - i \sigma_z \sin(\beta/2) }
 \lr{ \cos(\gamma/2) - i \sigma_y \sin(\gamma/2) } \ket{+}
= \lr{ \cos(\beta/2) - i \PauliZ \sin(\beta/2) }
 \lr{ \cos(\gamma/2) - i \PauliY \sin(\gamma/2) } \ket{+}
=
\begin{bmatrix}
e^{-i\beta/2} & 0 \\
0 & e^{i \beta/2}
\end{bmatrix}
\begin{bmatrix}
\cos(\gamma/2) & -\sin(\gamma/2)  \\
\sin(\gamma/2) & \cos(\gamma/2)
\end{bmatrix}
\begin{bmatrix}
1 \\
0
\end{bmatrix}
=
\begin{bmatrix}
e^{-i\beta/2} & 0 \\
0 & e^{i \beta/2}
\end{bmatrix}
\begin{bmatrix}
\cos(\gamma/2) \\
\sin(\gamma/2) \\
\end{bmatrix}
=
\begin{bmatrix}
\cos(\gamma/2)
e^{-i\beta/2}
\\
\sin(\gamma/2)
e^{i \beta/2}
\end{bmatrix}.
\end{dmath}
%
The state orthogonal to this one is claimed to be
%
\begin{dmath}\label{eqn:timeReversalPlaneWaveAndSpinor:180}
\ket{\ncap ; -}
= e^{-i S_z \beta/\Hbar} e^{-i S_y (\gamma + \pi)/\Hbar} \ket{+}
= e^{-i \sigma_z \beta/2} e^{-i \sigma_y (\gamma + \pi)/2} \ket{+}.
\end{dmath}
%
We have
%
\begin{dmath}\label{eqn:timeReversalPlaneWaveAndSpinor:200}
\cos((\gamma + \pi)/2)
=
\Real e^{i(\gamma + \pi)/2}
=
\Real i e^{i\gamma/2}
=
-\sin(\gamma/2),
\end{dmath}
%
and
\begin{dmath}\label{eqn:timeReversalPlaneWaveAndSpinor:220}
\sin((\gamma + \pi)/2)
=
\Imag e^{i(\gamma + \pi)/2}
=
\Imag i e^{i\gamma/2}
=
\cos(\gamma/2),
\end{dmath}
%
so we should have
%
\begin{dmath}\label{eqn:timeReversalPlaneWaveAndSpinor:240}
\ket{\ncap ; -}
=
\begin{bmatrix}
-\sin(\gamma/2)
e^{-i\beta/2}
\\
\cos(\gamma/2)
e^{i \beta/2}
\end{bmatrix}.
\end{dmath}
%
This looks right, but we can sanity check orthogonality
%
\begin{dmath}\label{eqn:timeReversalPlaneWaveAndSpinor:260}
\braket{\ncap ; -}{\ncap ; +}
=
\begin{bmatrix}
-\sin(\gamma/2)
e^{i\beta/2}
&
\cos(\gamma/2)
e^{-i \beta/2}
\end{bmatrix}
\begin{bmatrix}
\cos(\gamma/2)
e^{-i\beta/2}
\\
\sin(\gamma/2)
e^{i \beta/2}
\end{bmatrix}
=
0,
\end{dmath}
%
as expected.

The task at hand appears to be the operation on the column representation of \( \ket{\ncap; +} \) using the Pauli representation of the time reversal operator.  With the phase factor \( \eta = 1 \) the time reversal action on the spin up state is
%
\begin{dmath}\label{eqn:timeReversalPlaneWaveAndSpinor:160}
\Theta \ket{\ncap ; +}
=
-i \sigma_y K
\begin{bmatrix}
e^{-i\beta/2} \cos(\gamma/2) \\
e^{i \beta/2} \sin(\gamma/2)
\end{bmatrix}
=
-i \PauliY
\begin{bmatrix}
e^{i\beta/2} \cos(\gamma/2) \\
e^{-i \beta/2} \sin(\gamma/2)
\end{bmatrix}
=
\begin{bmatrix}
0 & -1 \\
1 & 0
\end{bmatrix}
\begin{bmatrix}
e^{i\beta/2} \cos(\gamma/2) \\
e^{-i \beta/2} \sin(\gamma/2)
\end{bmatrix}
=
\begin{bmatrix}
-e^{-i \beta/2} \sin(\gamma/2) \\
e^{i\beta/2} \cos(\gamma/2) \\
\end{bmatrix}
= \ket{\ncap ; -}. \qedmarker
\end{dmath}
} % answer

Observe that we need \( \eta = i \) to have this match eq. (4.79) in the text where \( \Theta = i^{2m} \ket{j, -m} \).

%}
%\EndArticle

         % pr 4.11
         %
% Copyright � 2015 Peeter Joot.  All Rights Reserved.
% Licenced as described in the file LICENSE under the root directory of this GIT repository.
%
%{
%\input{../blogpost.tex}
%\renewcommand{\basename}{totallyAsymmetricPotential}
%\renewcommand{\dirname}{notes/phy1520/}
%%\newcommand{\dateintitle}{}
%%\newcommand{\keywords}{}
%
%\input{../peeter_prologue_print2.tex}
%
%\usepackage{peeters_layout_exercise}
%\usepackage{peeters_braket}
%\usepackage{peeters_figures}
%
%\beginArtNoToc
%
%\generatetitle{Totally asymmetric potential}
%%\chapter{Totally asymmetric potential}
%%\label{chap:totallyAsymmetricPotential}

\makeoproblem{Totally asymmetric potential.}{problem:totallyAsymmetricPotential:1}{\citep{sakurai2014modern} pr. 4.11}{
\index{spherical harmonics}
\index{time reversal}

\makesubproblem{}{problem:totallyAsymmetricPotential:1:a}
Given a time reversal invariant Hamiltonian, show that for any energy eigenket
%
\begin{dmath}\label{eqn:totallyAsymmetricPotential:20}
\expectation{\BL} = 0.
\end{dmath}
%
\makesubproblem{}{problem:totallyAsymmetricPotential:1:b}
%
If the wave function of such a state is expanded as
%
\begin{dmath}\label{eqn:totallyAsymmetricPotential:40}
\sum_{l,m} F_{l m} Y_{l m}(\theta, \phi),
\end{dmath}
%
what are the phase restrictions on \( F_{lm} \)?

} % problem

\makeanswer{problem:totallyAsymmetricPotential:1}{
%
\makeSubAnswer{}{problem:totallyAsymmetricPotential:1:a}
%
For a time reversal invariant Hamiltonian \( H \) we have
%
\begin{dmath}\label{eqn:totallyAsymmetricPotential:60}
H \Theta = \Theta H.
\end{dmath}
%
If \( \ket{\psi} \) is an energy eigenstate with eigenvalue \( E \), we have
%
\begin{dmath}\label{eqn:totallyAsymmetricPotential:80}
H \Theta \ket{\psi}
= \Theta H \ket{\psi}
= \lambda \Theta \ket{\psi},
\end{dmath}
%
so \( \Theta \ket{\psi} \) is also an eigenvalue of \( H \), so can only differ from \( \ket{\psi} \) by a phase factor.  That is
%
\begin{dmath}\label{eqn:totallyAsymmetricPotential:100}
\ket{\psi'}
=
\Theta \ket{\psi}
= e^{i\delta} \ket{\psi}.
\end{dmath}
%
Now consider the expectation of \( \BL \) with respect to a time reversed state
%
\begin{dmath}\label{eqn:totallyAsymmetricPotential:120}
\bra{ \psi'} \BL \ket{\psi'}
=
\bra{ \psi} \Theta^{-1} \BL \Theta \ket{\psi}
=
\bra{ \psi} (-\BL) \ket{\psi},
\end{dmath}
%
however, we also have
%
\begin{dmath}\label{eqn:totallyAsymmetricPotential:140}
\bra{ \psi'} \BL \ket{\psi'}
=
\lr{ \bra{ \psi} e^{-i\delta} } \BL \lr{ e^{i\delta} \ket{\psi} }
=
\bra{\psi} \BL \ket{\psi},
\end{dmath}
%
so we have \( \bra{\psi} \BL \ket{\psi} = -\bra{\psi} \BL \ket{\psi} \) which is only possible if \( \expectation{\BL} = \bra{\psi} \BL \ket{\psi} = 0\).

\makeSubAnswer{}{problem:totallyAsymmetricPotential:1:b}
%
Consider the expansion of the wave function of a time reversed energy eigenstate
%
\begin{dmath}\label{eqn:totallyAsymmetricPotential:160}
\bra{\Bx} \Theta \ket{\psi}
=
\bra{\Bx} e^{i\delta} \ket{\psi}
=
e^{i\delta} \braket{\Bx}{\psi},
\end{dmath}
%
and then consider the same state expanded in the position basis
%
\begin{dmath}\label{eqn:totallyAsymmetricPotential:180}
\bra{\Bx} \Theta \ket{\psi}
=
\bra{\Bx} \Theta \int d^3 \Bx' \lr{ \ket{\Bx'}\bra{\Bx'} } \ket{\psi}
=
\bra{\Bx} \Theta \int d^3 \Bx' \lr{ \braket{\Bx'}{\psi} } \ket{\Bx'}
=
\bra{\Bx} \int d^3 \Bx' \lr{ \braket{\Bx'}{\psi} }^\conj \Theta \ket{\Bx'}
=
\bra{\Bx} \int d^3 \Bx' \lr{ \braket{\Bx'}{\psi} }^\conj \ket{\Bx'}
=
\int d^3 \Bx' \lr{ \braket{\Bx'}{\psi} }^\conj \braket{\Bx}{\Bx'}
=
\int d^3 \Bx' \braket{\psi}{\Bx'} \delta(\Bx- \Bx')
=
\braket{\psi}{\Bx}.
\end{dmath}
%
This demonstrates a relationship between the wave function and its complex conjugate
%
\begin{dmath}\label{eqn:totallyAsymmetricPotential:200}
\braket{\Bx}{\psi} = e^{-i\delta} \braket{\psi}{\Bx}.
\end{dmath}
%
Now expand the wave function in the spherical harmonic basis
%
\begin{dmath}\label{eqn:totallyAsymmetricPotential:220}
\begin{aligned}
\braket{\Bx}{\psi}
&=
\int d\Omega \braket{\Bx}{\ncap}\braket{\ncap}{\psi} \\
&=
\sum_{lm} F_{lm}(r) Y_{lm}(\theta, \phi) \\
&=
e^{-i\delta}
\lr{
\sum_{lm} F_{lm}(r) Y_{lm}(\theta, \phi) }^\conj \\
&=
e^{-i\delta}
\sum_{lm} \lr{ F_{lm}(r)}^\conj Y_{lm}^\conj(\theta, \phi)  \\
&=
e^{-i\delta}
\sum_{lm} \lr{ F_{lm}(r)}^\conj (-1)^m Y_{l,-m}(\theta, \phi)  \\
&=
e^{-i\delta}
\sum_{lm} \lr{ F_{l,-m}(r)}^\conj (-1)^m Y_{l,m}(\theta, \phi),
\end{aligned}
\end{dmath}

so the \( F_{lm} \) functions are constrained by
%
\begin{dmath}\label{eqn:totallyAsymmetricPotential:240}
F_{lm}(r) = e^{-i\delta} \lr{ F_{l,-m}(r)}^\conj (-1)^m.
\end{dmath}
} % answer

%}
%\EndArticle

         % pr 4.12
         %
% Copyright � 2015 Peeter Joot.  All Rights Reserved.
% Licenced as described in the file LICENSE under the root directory of this GIT repository.
%
%{
%\input{../blogpost.tex}
%\renewcommand{\basename}{crystalSpinHamiltonianTimeReversal}
%\renewcommand{\dirname}{notes/phy1520/}
%%\newcommand{\dateintitle}{}
%%\newcommand{\keywords}{}
%
%\input{../peeter_prologue_print2.tex}
%
%\usepackage{peeters_layout_exercise}
%\usepackage{peeters_braket}
%\usepackage{peeters_figures}
%
%\beginArtNoToc
%
%\generatetitle{Time reversal behavior of solutions to crystal spin Hamiltonian}
%\chapter{Time reversal behaviour of solutions to crystal spin Hamiltonian}
%\label{chap:crystalSpinHamiltonianTimeReversal}

\makeoproblem{Time reversal behavior of solutions to crystal spin Hamiltonian.}{problem:crystalSpinHamiltonianTimeReversal:1}{\citep{sakurai2014modern} pr. 4.12}{
\index{crystal spin}
\index{time reversal}

Solve the spin 1 Hamiltonian
\begin{dmath}\label{eqn:crystalSpinHamiltonianTimeReversal:20}
H = A S_z^2 + B(S_x^2 - S_y^2).
\end{dmath}
%
Is this Hamiltonian invariant under time reversal?

How do the eigenkets change under time reversal?

} % problem

\makeanswer{problem:crystalSpinHamiltonianTimeReversal:1}{

In spinMatrices.nb the matrix representation of the Hamiltonian is found to be
\begin{dmath}\label{eqn:crystalSpinHamiltonianTimeReversal:40}
H =
\Hbar^2
\begin{bmatrix}
 A+\frac{B}{2} & 0 & \frac{B}{2} \\
 -\frac{i B}{\sqrt{2}} & B & -\frac{i B}{\sqrt{2}} \\
 \frac{B}{2} & 0 & A+\frac{B}{2} \\
\end{bmatrix}.
\end{dmath}
%
The eigenvalues are
\begin{dmath}\label{eqn:crystalSpinHamiltonianTimeReversal:60}
\setlr{ \Hbar^2 A, \Hbar^2 B, \Hbar^2(A + B)},
\end{dmath}
%
and the respective eigenvalues (unnormalized) are
%
\begin{dmath}\label{eqn:crystalSpinHamiltonianTimeReversal:80}
\setlr{
\begin{bmatrix}
-1 \\
0 \\
1
\end{bmatrix},
\begin{bmatrix}
0 \\
1 \\
0
\end{bmatrix},
\begin{bmatrix}
1 \\
-\frac{i \sqrt{2} B}{A} \\
1 \\
\end{bmatrix}
}.
\end{dmath}
%
Under time reversal, the Hamiltonian is
%
\begin{equation}\label{eqn:crystalSpinHamiltonianTimeReversal:100}
H \rightarrow A (-S_z)^2 + B ( (-S_x)^2 - (-S_y)^2 ) = H,
\end{equation}
%
so we expect the eigenkets for this Hamiltonian to vary by at most a phase factor.  To check this, first recall that the time reversal action on a spin one state is
%
\begin{dmath}\label{eqn:crystalSpinHamiltonianTimeReversal:120}
\Theta \ket{1, m} = (-1)^m \ket{1, -m},
\end{dmath}
%
or
%
\begin{equation}\label{eqn:crystalSpinHamiltonianTimeReversal:140}
\begin{aligned}
\Theta \ket{1} &= -\ket{-1} \\
\Theta \ket{0} &= \ket{0} \\
\Theta \ket{-1} &= -\ket{1}.
\end{aligned}
\end{equation}
%
Let's write the eigenkets respectively as
%
\begin{equation}\label{eqn:crystalSpinHamiltonianTimeReversal:160}
\begin{aligned}
\ket{A} &= -\ket{1} + \ket{-1} \\
\ket{B} &= \ket{0} \\
\ket{A+B} &= \ket{1} + \ket{-1} - \frac{i \sqrt{2} B}{A} \ket{0}.
\end{aligned}
\end{equation}
%
Noting that the time reversal operator maps complex numbers onto their conjugates, the time reversed eigenkets are
%
\begin{equation}\label{eqn:crystalSpinHamiltonianTimeReversal:180}
\begin{aligned}
\ket{A} &\rightarrow \ket{-1} - \ket{-1} = -\ket{A} \\
\ket{B} &\rightarrow \ket{0} = \ket{B} \\
\ket{A+B} &\rightarrow -\ket{1} - \ket{-1} + \frac{i \sqrt{2} B}{A} \ket{0} = -\ket{A+B}.
\end{aligned}
\end{equation}
%
Up to a sign, the time reversed states match the unreversed states.
} % answer

%}
%\EndArticle
%\EndNoBibArticle

   \mychapter{Theory of angular momentum.}
      %
% Copyright � 2015 Peeter Joot.  All Rights Reserved.
% Licenced as described in the file LICENSE under the root directory of this GIT repository.
%
\section{Angular momentum}

In classical mechanics the (orbital) angular momentum is
\index{orbital angular momentum}
\index{angular momentum operator}
%
\begin{dmath}\label{eqn:qmLecture13:880}
\BL = \Br \cross \Bp.
\end{dmath}

Here ``orbital'' is to distinguish from spin angular momentum.

In quantum mechanics, the mapping to operators, in component form, is
%
\begin{dmath}\label{eqn:qmLecture13:900}
\hatL_i = \epsilon_{ijk} \hatr_j \hatp_k.
\end{dmath}

These operators do not commute
\begin{dmath}\label{eqn:qmLecture13:920}
\antisymmetric{\hatL_i}{\hatL_j}
%= \epsilon_{i m n} \epsilon_{ijk}
%\antisymmetric{\hatr_m \hatp_n}{\hatr_k \hatp_l}
%=
%\delta^{[mn]}_{jk}
%\lr{
%\hatr_m \hatp_n \hatr_k \hatp_l
%-
%\hatr_k \hatp_l
%\hatr_m \hatp_n
%}
%=
%\lr{
%\hatr_j \hatp_k \hatr_k \hatp_l
%-
%\hatr_k \hatp_j \hatr_k \hatp_l
%-
%\hatr_k \hatp_l
%\hatr_j \hatp_k
%+
%\hatr_k \hatp_l
%\hatr_k \hatp_j
%}
=
i \Hbar \epsilon_{ijk} \hatL_k.
\end{dmath}

%FIXME: fill in the details.

\index{vector operator}
which means that we can't simultaneously determine \( \hatL_i \) for all \( i \).

Aside: In quantum mechanics, we define an operator \( \Vcap \) to be a vector operator if
%
\begin{dmath}\label{eqn:qmLecture13:940}
\antisymmetric{\hatL_i}{\hatV_j}
=
i \Hbar \epsilon_{ijk} \hatV_k.
\end{dmath}

The commutator of the squared angular momentum operator with any \( \hatL_i \), say \( \hatL_x \) is zero
%
\begin{dmath}\label{eqn:qmLecture13:960}
\begin{aligned}
\antisymmetric{
\hatL_x^2 +
\hatL_y^2 +
\hatL_z^2
}
{\hatL_x}
&=
\hatL_y \hatL_y \hatL_x
- \hatL_x \hatL_y \hatL_y
+
\hatL_z \hatL_z \hatL_x
- \hatL_x \hatL_z \hatL_z \\
&=
\hatL_y \lr{ \antisymmetric{\hatL_y}{\hatL_x} + \cancel{\hatL_x \hatL_y} }
-\lr{ \antisymmetric{\hatL_x}{\hatL_y} + \cancel{\hatL_y \hatL_x} } \hatL_y \\
&\quad +\hatL_z \lr{ \antisymmetric{\hatL_z}{\hatL_x} + \cancel{\hatL_x \hatL_z} }
-\lr{ \antisymmetric{\hatL_x}{\hatL_z} + \cancel{\hatL_z \hatL_x} } \hatL_z \\
&=
\hatL_y \antisymmetric{\hatL_y}{\hatL_x}
-\antisymmetric{\hatL_x}{\hatL_y} \hatL_y
+\hatL_z \antisymmetric{\hatL_z}{\hatL_x}
-\antisymmetric{\hatL_x}{\hatL_z} \hatL_z \\
&=
i \Hbar \lr{
-\hatL_y \hatL_z
- \hatL_z \hatL_y
+\hatL_z \hatL_y
+ \hatL_y \hatL_z
} \\
&=
0.
\end{aligned}
\end{dmath}
%
%In fact
%\begin{dmath}\label{eqn:qmLecture13:980}
%\antisymmetric{\Lcap^2 }{\hatL_i} = 0.
%\end{dmath}

Suppose we have a state \( \ket{\Psi} \) with a well defined \( \hatL_z \) eigenvalue and well defined \( \hat{\BL^2} \) eigenvalue, written as
%
\begin{dmath}\label{eqn:qmLecture13:1000}
\ket{\Psi} = \ket{a, b},
\end{dmath}

where the label \( a \) is used for the eigenvalue of \( \Lcap^2 \) and \( b \) labels the eigenvalue of \( \hatL_z \).  Then
%
\begin{equation}\label{eqn:qmLecture13:1020}
\begin{aligned}
\Lcap^2 \ket{a , b} &= \Hbar^2 a \ket{a ,b} \\
\hatL_z \ket{a , b} &= \Hbar b \ket{a ,b}.
\end{aligned}
\end{equation}

Things aren't so nice when we act with other angular momentum operators, producing a scrambled mess
%
\begin{equation}\label{eqn:qmLecture13:1040}
\begin{aligned}
\hatL_x \ket{a , b} &= \sum_{a', b'} \calA^x_{a, b, a', b'} \ket{a', b'} \\
\hatL_y \ket{a , b} &= \sum_{a', b'} \calA^y_{a, b, a', b'} \ket{a', b'} \\
\end{aligned}
\end{equation}

With this representation, we have
%
\begin{dmath}\label{eqn:qmLecture13:1060}
\hatL_x \Lcap^2 \ket{a, b}
=
\hatL_x \Hbar^2 a
\sum_{a', b'} \calA^x_{a, b, a', b'} \ket{a', b'}.
\end{dmath}
%
\begin{dmath}\label{eqn:qmLecture13:1080}
\Lcap^2 \hatL_x \ket{a, b}
=
\Hbar^2
\sum_{a', b'} a' \calA^x_{a, b, a', b'} \ket{a', b'}.
\end{dmath}

Since \( \Lcap^2, \hatL_x \) commute, we must have
%
\begin{dmath}\label{eqn:qmLecture13:1100}
\calA^x_{a, b, a', b'} = \delta_{a, a'} \calA^x_{a'; b, b'},
\end{dmath}

and similarly
\begin{dmath}\label{eqn:qmLecture13:1120}
\calA^y_{a, b, a', b'} = \delta_{a, a'} \calA^y_{a'; b, b'}.
\end{dmath}

Simplifying things we can write the action of \( \hatL_x, \hatL_y \) on the state as
%
\begin{dmath}\label{eqn:qmLecture13:1140}
\begin{aligned}
\hatL_x \ket{a , b} &= \sum_{ b'} \calA^x_{a; b, b'} \ket{a, b'} \\
\hatL_y \ket{a , b} &= \sum_{ b'} \calA^y_{a; b, b'} \ket{a, b'} \\
\end{aligned}
\end{dmath}

Let's define
\begin{dmath}\label{eqn:qmLecture13:1160}
\begin{aligned}
\hatL_{+} &\equiv \hatL_x + i \hatL_y \\
\hatL_{-} &\equiv \hatL_x - i \hatL_y \\
\end{aligned}
\end{dmath}

Because these are sums of \( \hatL_x, \hatL_y \) they must also commute with \( \Lcap^2 \)
%
\begin{dmath}\label{eqn:qmLecture13:1180}
\antisymmetric{\Lcap^2}{\hatL_{\pm}} = 0.
\end{dmath}

The commutators with \( \hatL_z \) are non-zero
%
\begin{dmath}\label{eqn:qmLecture13:1740}
\antisymmetric{\hatL_z}{\hatL_{\pm}}
=
\hatL_z \lr{ \hatL_x \pm i \hatL_y }
- \lr{ \hatL_x \pm i \hatL_y } \hatL_z
=
\antisymmetric{\hatL_z}{\hatL_x}
\pm i
\antisymmetric{\hatL_z}{\hatL_y}
=
i \Hbar \lr{
\hatL_y \mp i \hatL_x
}
=
\Hbar \lr{ i \hatL_y \pm \hatL_x }
=
\pm \Hbar \lr{ \hatL_x \pm i \hatL_y }
=
\pm \Hbar \hatL_{\pm}.
\end{dmath}
%
%\begin{dmath}\label{eqn:qmLecture13:1200}
%\antisymmetric{\hatL_z}{\hatL_{\pm}} = \pm \Hbar \hatL_{\pm}.
%\end{dmath}
%
Explicitly, that is
%
\begin{dmath}\label{eqn:qmLecture13:1220}
\begin{aligned}
\hatL_z \hatL_{+} - \hatL_{+} \hatL_z &= \Hbar \hatL_{+} \\
\hatL_z \hatL_{-} - \hatL_{-} \hatL_z &= -\Hbar \hatL_{-}
\end{aligned}
\end{dmath}

Now we are set to compute actions of these (assumed) raising and lowering operators on the eigenstate of \( \hatL_z, \Lcap^2 \)
%
\begin{dmath}\label{eqn:qmLecture13:1240}
\hatL_z \hatL_{\pm} \ket{a, b}
=
\Hbar \hatL_{\pm} \ket{a,b} \pm \hatL_{\pm} \hatL_z \ket{a,b}
=
\Hbar \hatL_{\pm} \ket{a,b} \pm \Hbar b \hatL_{\pm} \ket{a,b}
=
\Hbar \lr{ b \pm 1 } \hatL_{\pm} \ket{a, b} .
\end{dmath}

There must be a proportionality of the form
%
\begin{dmath}\label{eqn:qmLecture13:1260}
\ket{\hatL_{\pm}} \propto \ket{a, b \pm 1},
\end{dmath}

The products of the raising and lowering operators are
%
\begin{dmath}\label{eqn:qmLecture13:1280}
\hatL_{-} \hatL_{+}
=
\lr{ \hatL_x - i \hatL_y }
\lr{ \hatL_x + i \hatL_y }
=
\hatL_x^2 + \hatL_y^2 + i \hatL_x \hatL_y - i \hatL_y \hatL_x
=
\lr{ \Lcap^2 - \hatL_z^2 } + i \antisymmetric{\hatL_x}{\hatL_y}
=
\Lcap^2 - \hatL_z^2 - \Hbar \hatL_z,
\end{dmath}

and
\begin{dmath}\label{eqn:qmLecture13:1300}
\hatL_{+} \hatL_{-}
=
\lr{ \hatL_x + i \hatL_y }
\lr{ \hatL_x - i \hatL_y }
=
\hatL_x^2 + \hatL_y^2 - i \hatL_x \hatL_y + i \hatL_y \hatL_x
=
\lr{ \Lcap^2 - \hatL_z^2 } - i \antisymmetric{\hatL_x}{\hatL_y}
=
\Lcap^2 - \hatL_z^2 + \Hbar \hatL_z,
\end{dmath}

So we must have
%
\begin{equation}\label{eqn:qmLecture13:1320}
0
\le \bra{a, b} \hatL_{-} \hatL_{+} \ket{a, b}
=
\bra{a, b}
\lr{ \Lcap^2 - \hatL_z^2 - \Hbar \hatL_z }
\ket{a, b}
=
\Hbar^2 a - \Hbar^2 b^2 - \Hbar^2 b,
\end{equation}

and
%
\begin{equation}\label{eqn:qmLecture13:1340}
0
\le \bra{a, b} \hatL_{+} \hatL_{-} \ket{a, b}
=
\bra{a, b}
\lr{ \Lcap^2 - \hatL_z^2 + \Hbar \hatL_z }
\ket{a, b}
=
\Hbar^2 a - \Hbar^2 b^2 + \Hbar^2 b.
\end{equation}

This puts constraints on \( a, b \), roughly of the form

\begin{enumerate}
\item
\begin{equation}\label{eqn:qmLecture13:1360}
a - b( b + 1) \ge 0
\end{equation}

With \( b_{\textrm{max}} > 0 \), \( b_{\textrm{max}} \approx \sqrt{a} \).

\item
\begin{equation}\label{eqn:qmLecture13:1380}
a - b( b - 1) \ge 0
\end{equation}

With \( b_{\textrm{min}} < 0 \), \( b_{\textrm{min}} \approx -\sqrt{a} \).

\end{enumerate}

%\paragraph{I Love and Desire Sofia Always!}


      %
% Copyright � 2015 Peeter Joot.  All Rights Reserved.
% Licenced as described in the file LICENSE under the root directory of this GIT repository.
%
%%%\input{../blogpost.tex}
%%%\renewcommand{\basename}{qmLecture14}
%%%\renewcommand{\dirname}{notes/phy1520/}
%%%\newcommand{\keywords}{PHY1520H}
%%%\input{../peeter_prologue_print2.tex}
%%%
%%%%\usepackage{phy1520}
%%%\usepackage{peeters_braket}
%%%\usepackage{macros_cal}
%%%%\usepackage{peeters_layout_exercise}
%%%\usepackage{peeters_figures}
%%%\usepackage{mathtools}
%%%
%%%\beginArtNoToc
%%%\generatetitle{PHY1520H Graduate Quantum Mechanics.  Lecture 14: Angular momentum (cont.).  Taught by Prof.\ Arun Paramekanti}
%%%%\chapter{Angular momentum (cont.)}
%%%\label{chap:qmLecture14}
%%%
%%%\paragraph{Disclaimer}
%%%
%%%Peeter's lecture notes from class.  These may be incoherent and rough.
%%%
%%%These are notes for the UofT course PHY1520, Graduate Quantum Mechanics, taught by Prof. Paramekanti, covering \textchapref{{3}} \citep{sakurai2014modern} content.
%%%
%%\paragraph{Review: Angular momentum}
%%
%%Given eigenket \( \ket{a, b} \), where
%%
%%\begin{equation}\label{eqn:qmLecture14:20}
%%\begin{aligned}
%%\Lcap^2 \ket{a, b} &= \Hbar^2 a \ket{a,b} \\
%%\hatL_z \ket{a, b} &= \Hbar b \ket{a,b}
%%\end{aligned}
%%\end{equation}
%%
%%We were looking for
%%
%%\begin{dmath}\label{eqn:qmLecture14:40}
%%\hatL_{x,y} \ket{a,b} = \sum_{b'} \calA^{x,y}_{a; b, b'} \ket{a,b'},
%%\end{dmath}
%%
%%by applying
%%
%%\begin{dmath}\label{eqn:qmLecture14:60}
%%\hatL_{\pm} = \hatL_x \pm i \hatL_y.
%%\end{dmath}
%%
%%We found
%%
%%\begin{dmath}\label{eqn:qmLecture14:80}
%%\hatL_{\pm} \propto \ket{a, b \pm 1}.
%%\end{dmath}
%%
%%Let
%%
%%\begin{dmath}\label{eqn:qmLecture14:100}
%%\ket{\phi_\pm} = \hatL_{\pm} \ket{a, b}.
%%\end{dmath}
%%
%%We want
%%
%%\begin{dmath}\label{eqn:qmLecture14:120}
%%\braket{\phi_\pm}{\phi_\pm} \ge 0,
%%\end{dmath}
%%
%%or
%%\begin{dmath}\label{eqn:qmLecture14:140}
%%\begin{aligned}
%%\bra{a,b} \hatL_{+} \hatL_{-} \ket{a, b} &\ge 0 \\
%%\bra{a,b} \hatL_{-} \hatL_{+} \ket{a, b} &\ge 0
%%\end{aligned}
%%\end{dmath}
%%
%%We found
%%
%%\begin{dmath}\label{eqn:qmLecture14:160}
%%\hatL_{+} \hatL_{-} =
%%\lr{ \hatL_x + i \hatL_y } \lr{ \hatL_x - i \hatL_y }
%%= \lr{ \Lcap^2 - \hatL_z^2 } -i \antisymmetric{\hatL_x}{\hatL_y}
%%= \lr{ \Lcap^2 - \hatL_z^2 } -i \lr{ i \Hbar \hatL_z }
%%= \lr{ \Lcap^2 - \hatL_z^2 } + \Hbar \hatL_z,
%%\end{dmath}
%%
%%so
%%
%%\begin{dmath}\label{eqn:qmLecture14:180}
%%\bra{a,b} \hatL_{+} \hatL_{-} \ket{a, b}
%%=
%%\expectation{ \Lcap^2 - \hatL_z^2 + \Hbar \hatL_z }.
%%\end{dmath}
%%
%%Similarly
%%\begin{dmath}\label{eqn:qmLecture14:200}
%%\bra{a,b} \hatL_{-} \hatL_{+} \ket{a, b}
%%=
%%\expectation{ \Lcap^2 - \hatL_z^2 - \Hbar \hatL_z }.
%%\end{dmath}
%%
%%\paragraph{Constraints}
%%
%%\begin{equation}\label{eqn:qmLecture14:220}
%%\begin{aligned}
%%a - b^2 + b &\ge 0 \\
%%a - b^2 - b &\ge 0
%%\end{aligned}
%%\end{equation}

Assuming that the \( b_{\textrm{max}} \) and \( b_{\textrm{min}} \) values are
satisfied at the equality extremes we have

\begin{equation}\label{eqn:qmLecture14:240}
\begin{aligned}
b_{\textrm{max}} \lr{ b_{\textrm{max}} + 1 } &= a \\
b_{\textrm{min}} \lr{ b_{\textrm{min}} - 1 } &= a.
\end{aligned}
\end{equation}

Equating these pair of equations and rearranging, we have

\begin{dmath}\label{eqn:qmLecture14:680}
\lr{ b_{\textrm{max}} + \inv{2} }^2 - \inv{4} = \lr{ b_{\textrm{min}} - \inv{2} }^2 - \inv{4},
\end{dmath}

which has solutions at

\begin{dmath}\label{eqn:qmLecture14:700}
b_{\textrm{max}} + \inv{2} = \pm \lr{ b_{\textrm{min}} - \inv{2} }.
\end{dmath}

One of the solutions is

\begin{dmath}\label{eqn:qmLecture14:260}
-b_{\textrm{min}} = b_{\textrm{max}},
\end{dmath}

as desired.  The other solution is \( b_{\textrm{max}} = b_{\textrm{min}} - 1 \), which we discard.

The final constraint is therefore

%\begin{equation}\label{eqn:qmLecture14:280}
\boxedEquation{eqn:qmLecture14:300}{
- b_{\textrm{max}} \le b \le b_{\textrm{max}},
}
%\end{equation}

and

\begin{equation}\label{eqn:qmLecture14:320}
\begin{aligned}
\hatL_{+} \ket{a, b_{\textrm{max}}} &= 0 \\
\hatL_{-} \ket{a, b_{\textrm{min}}} &= 0
\end{aligned}
\end{equation}

If we had the sequence, which must terminate at \( b_{\textrm{min}} \) or else it will go on forever

\begin{dmath}\label{eqn:qmLecture14:340}
\ket{a, b_{\textrm{max}}}
\overset{\hatL_{-}}{\rightarrow}
\ket{a, b_{\textrm{max}} - 1}
\overset{\hatL_{-}}{\rightarrow}
\ket{a, b_{\textrm{max}} - 2}
\cdots
\overset{\hatL_{-}}{\rightarrow}
\ket{a, b_{\textrm{min}}},
\end{dmath}

then we know that \( b_{\textrm{max}} - b_{\textrm{min}} \in \bbZ \), or

\begin{equation}\label{eqn:qmLecture14:360}
b_{\textrm{max}} - n = b_{\textrm{min}} = -b_{\textrm{max}}
\end{equation}

or

\begin{equation}\label{eqn:qmLecture14:380}
b_{\textrm{max}} = \frac{n}{2},
\end{equation}

this is either an integer or a \( 1/2 \) odd integer, depending on whether \( n \) is even or odd.  These are called ``orbital'' or ``spin'' respectively.

The convention is to write \( m \) for the \( \hatL_z \) eigenvalue (the magnetic quantum number), and \( j \), the azimthual quantum number, to describe the \( j(j+1) \) eigenvalue of the \( \Lcap^2 \) operator.

\begin{equation}\label{eqn:qmLecture14:400}
\begin{aligned}
b_{\textrm{max}} &= j \\
a &= j(j + 1).
\end{aligned}
\end{equation}
\index{angular momentum!magnetic quantum number}
\index{angular momentum!azimthual quantum number}

so for \( m \in -j, -j + 1, \cdots, +j \)

%\begin{equation}\label{eqn:qmLecture14:420}
\boxedEquation{eqn:qmLecture14:440}{
\begin{aligned}
\Lcap^2 \ket{j, m} &= \Hbar^2 j (j + 1) \ket{j, m} \\
L_z \ket{j, m} &= \Hbar m \ket{j, m}.
\end{aligned}
}
%\end{equation}

\section{Schwinger's Harmonic oscillator representation of angular momentum operators.}
\index{angular momentum!Schwinger}

In \citep{schwinger1955quantum} a powerful method for describing angular momentum with harmonic oscillators was introduced, which will be outlined here.  The question is whether we can construct a set of harmonic oscillators that allows a mapping from

\begin{dmath}\label{eqn:qmLecture14:460}
\hatL_{+} \leftrightarrow a^{+}?
\end{dmath}

Picture two harmonic oscillators, one with states counted from one zero towards \( \infty \) and another with states counted from a different zero towards \( -\infty \), as pictured in \cref{fig:qmLecture14:qmLecture14Fig1}.

\imageFigure{../figures/phy1520-quantum/qmLecture14Fig1}{Overlapping SHO domains.}{fig:qmLecture14:qmLecture14Fig1}{0.2}

Is it possible that such an overlapping set of harmonic oscillators can provide the properties of the angular momentum operators?  Let's relabel the counting so that we have two sets of positive counted SHO systems, each counted in a positive direction as sketched in \cref{fig:qmLecture14:qmLecture14Fig2}.

\imageFigure{../figures/phy1520-quantum/qmLecture14Fig2}{Relabeling the counting for overlapping SHO systems.}{fig:qmLecture14:qmLecture14Fig2}{0.2}

It turns out that given a constraint there the number of ways to distribute particles between a pair of SHO systems, the process that can be viewed as reproducing the angular momentum action is a transfer of particles from one harmonic oscillator to the other.  For \( \hatL_z = +j \)

\begin{equation}\label{eqn:qmLecture14:480}
\begin{aligned}
n_1 &= n_{\textrm{max}} \\
n_2 &= 0,
\end{aligned}
\end{equation}

and for \( \hatL_z = -j \)

\begin{equation}\label{eqn:qmLecture14:500}
\begin{aligned}
n_1 &= 0 \\
n_2 &= n_{\textrm{max}}.
\end{aligned}
\end{equation}

We can make the identifications

\begin{dmath}\label{eqn:qmLecture14:520}
\hatL_z = \lr{ n_1 - n_2 } \frac{\Hbar}{2},
\end{dmath}

and
\begin{equation}\label{eqn:qmLecture14:540}
j = \inv{2} n_{\textrm{max}},
\end{equation}

or

\begin{equation}\label{eqn:qmLecture14:560}
n_1 + n_2 = \text{fixed} = n_{\textrm{max}}
\end{equation}

Changes that keep \( n_1 + n_2 \) fixed are those that change \( n_1 \), \( n_2 \) by \( +1 \) or \( -1 \) respectively, as sketched in \cref{fig:qmLecture14:qmLecture14Fig3}.

\imageFigure{../figures/phy1520-quantum/qmLecture14Fig3}{Number conservation constraint.}{fig:qmLecture14:qmLecture14Fig3}{0.15}

Can we make an identification that takes

\begin{equation}\label{eqn:qmLecture14:580}
\ket{n_1, n_2} \overset{\hatL_{-}}{\rightarrow} \ket{n_1 - 1, n_2 + 1}?
\end{equation}

What operator in the SHO problem has this effect?  Let's try

%\begin{equation}\label{eqn:qmLecture14:600}
\boxedEquation{eqn:qmLecture14:620}{
\begin{aligned}
\hatL_{-} &= \Hbar a_2^\dagger a_1  \\
\hatL_{+} &= \Hbar a_1^\dagger a_2  \\
\hatL_z &= \frac{\Hbar}{2} \lr{ n_1 - n_2 }
\end{aligned}
}
%\end{equation}

Is this correct?  Do we need to make any scalar adjustments?  We want

\begin{equation}\label{eqn:qmLecture14:640}
\antisymmetric{\hatL_z}{\hatL_{\pm}} = \pm \Hbar \hatL_{\pm}.
\end{equation}

First check this with the \( \hatL_{+} \) commutator

\begin{dmath}\label{eqn:qmLecture14:660}
\antisymmetric{\hatL_z}{\hatL_{+}}
=
\inv{2} \Hbar^2 \antisymmetric{ n_1 - n_2}{a_1^\dagger a_2 }
=
\inv{2} \Hbar^2 \antisymmetric{ a_1^\dagger a_1 - a_2^\dagger a_2 }{a_1^\dagger a_2 }
=
\inv{2} \Hbar^2
\lr{
\antisymmetric{ a_1^\dagger a_1 }{a_1^\dagger a_2 }
-\antisymmetric{ a_2^\dagger a_2 }{a_1^\dagger a_2 }
}
=
\inv{2} \Hbar^2
\lr{
a_2 \antisymmetric{ a_1^\dagger a_1 }{a_1^\dagger }
-a_1^\dagger \antisymmetric{ a_2^\dagger a_2 }{a_2 }
}.
\end{dmath}

But

\begin{dmath}\label{eqn:qmLecture14:720}
\antisymmetric{ a^\dagger a }{a^\dagger }
=
a^\dagger a
a^\dagger
-
a^\dagger
a^\dagger a
=
a^\dagger \lr{ 1 +
a^\dagger a}
-
a^\dagger
a^\dagger a
=
a^\dagger,
\end{dmath}

and
\begin{dmath}\label{eqn:qmLecture14:740}
\antisymmetric{ a^\dagger a }{a}
=
a^\dagger a a
-a a^\dagger a
=
a^\dagger a a
-\lr{ 1 + a^\dagger a } a
=
-a,
\end{dmath}

so
\begin{equation}\label{eqn:qmLecture14:760}
\antisymmetric{\hatL_z}{\hatL_{+}} = \Hbar^2 a_2 a_1^\dagger = \Hbar \hatL_{+},
\end{equation}

as desired.  Similarly

\begin{dmath}\label{eqn:qmLecture14:780}
\antisymmetric{\hatL_z}{\hatL_{-}}
=
\inv{2} \Hbar^2 \antisymmetric{ n_1 - n_2}{a_2^\dagger a_1 }
=
\inv{2} \Hbar^2 \antisymmetric{ a_1^\dagger a_1 - a_2^\dagger a_2 }{a_2^\dagger a_1 }
=
\inv{2} \Hbar^2 \lr{
a_2^\dagger \antisymmetric{ a_1^\dagger a_1 }{a_1 }
- a_1 \antisymmetric{ a_2^\dagger a_2 }{a_2^\dagger }
}
=
\inv{2} \Hbar^2 \lr{
a_2^\dagger (-a_1)
- a_1 a_2^\dagger
}
=
- \Hbar^2 a_2^\dagger a_1
=
- \Hbar \hatL_{-}.
\end{dmath}

With

\begin{equation}\label{eqn:qmLecture14:800}
\begin{aligned}
j &= \frac{n_1 + n_2}{2} \\
m &= \frac{n_1 - n_2}{2} \\
\end{aligned}
\end{equation}

We can make the identification

\begin{equation}\label{eqn:qmLecture14:820}
\ket{n_1, n_2} = \ket{ j+ m , j - m}.
\end{equation}

\paragraph{Another way}

With

\begin{equation}\label{eqn:qmLecture14:840}
\hatL_{+} \ket{j, m} = d_{j,m}^{+} \ket{j, m+1}
\end{equation}

or

\begin{equation}\label{eqn:qmLecture14:860}
\Hbar a_1^\dagger a_2  \ket{j + m, j-m} = d_{j,m}^{+} \ket{ j + m + 1, j- m-1},
\end{equation}

we can seek this factor \( d_{j,m}^{+} \) by operating with \( \hatL_{+} \)

\begin{dmath}\label{eqn:qmLecture14:880}
\hatL_{+} \ket{j, m}
=
\Hbar a_1^\dagger a_2 \ket{n_1, n_2}
=
\Hbar a_1^\dagger a_2 \ket{j+m,j-m}
=
\Hbar \sqrt{ n + 1 } \sqrt{n_2} \ket{j+m +1,j-m-1}
=
\Hbar \sqrt{ \lr{ j+ m + 1}\lr{ j - m } } \ket{j+m +1,j-m-1}
\end{dmath}

That gives
\begin{equation}\label{eqn:qmLecture14:900}
\begin{aligned}
d_{j,m}^{+} &= \Hbar \sqrt{\lr{ j - m } \lr{ j+ m + 1} } \\
d_{j,m}^{-} &= \Hbar \sqrt{\lr{ j + m } \lr{ j- m + 1} }.
\end{aligned}
\end{equation}

This equivalence can be used to model spin interaction in crystals as harmonic oscillators.   This equivalence of lattice vibrations and spin oscillations is called ``spin waves''.
\index{spin wave}

%\EndArticle

      %
% Copyright � 2015 Peeter Joot.  All Rights Reserved.
% Licenced as described in the file LICENSE under the root directory of this GIT repository.
%
%\input{../blogpost.tex}
%\renewcommand{\basename}{qmLecture15}
%\renewcommand{\dirname}{notes/phy1520/}
%\newcommand{\keywords}{PHY1520H}
%\input{../peeter_prologue_print2.tex}
%
%%\usepackage{phy1520}
%\usepackage{peeters_braket}
%%\usepackage{peeters_layout_exercise}
%\usepackage{peeters_figures}
%\usepackage{mathtools}
%
%\beginArtNoToc
%\generatetitle{PHY1520H Graduate Quantum Mechanics.  Lecture 15: angular momentum rotation representation, and angular momentum addition.  Taught by Prof.\ Arun Paramekanti}
%%\chapter{angular momentum rotation representation, and angular momentum addition}
%\label{chap:qmLecture15}
%
%\paragraph{Disclaimer}
%
%Peeter's lecture notes from class.  These may be incoherent and rough.
%
%These are notes for the UofT course PHY1520, Graduate Quantum Mechanics, taught by Prof. Paramekanti, covering \textchapref{{3}} \citep{sakurai2014modern} content.
%
%%%\paragraph{Angular momentum (wrap up.)}
%%%
%%%We found
%%%
%%%\begin{equation}\label{eqn:qmLecture15:20}
%%%\begin{aligned}
%%%\hat{\BL^2} \ket{j, m} &= j(j+1) \Hbar^2 \ket{j,m} \\
%%%\hatL_z \ket{j, m} &= \Hbar m \ket{j,m} \\
%%%\hatL_{\pm} \ket{j, m } &= \Hbar \sqrt{(j \mp m)(j \pm m + 1)} \ket{j, m \pm 1 }
%%%\end{aligned}
%%%\end{equation}
%%%
%%%or Schwinger
%%%
%%%\begin{equation}\label{eqn:qmLecture15:40}
%%%\begin{aligned}
%%%\hatL_z &= \inv{2} \lr{ \hatn_1 - \hatn_2 } \Hbar \\
%%%\hatL_{+} &= a_1^\dagger a_2 \Hbar \\
%%%\hatL_{-} &= a_1 a_2^\dagger \Hbar \\
%%%j &= \inv{2} \lr{ \hatn_1 + \hatn_2 },
%%%\end{aligned}
%%%\end{equation}
%%%
%%%where each of the \( a_1, a_2 \) operators obey
%%%
%%%\begin{equation}\label{eqn:qmLecture15:60}
%%%\begin{aligned}
%%%\antisymmetric{a_1}{a_1^\dagger} &= 1 \\
%%%\antisymmetric{a_2}{a_2^\dagger} &= 1
%%%\end{aligned}
%%%\end{equation}
%%%
%%%and any pair of different index \( a \) operators commute, as in
%%%
%%%\begin{equation}\label{eqn:qmLecture15:80}
%%%\antisymmetric{a_1}{a_2^\dagger} = 0.
%%%\end{equation}
%%%
\section{Representations}

\index{angular momentum!rotation operator}
It's possible to compute matrix representations of the rotation operators
%
\begin{equation}\label{eqn:qmLecture15:100}
\hatR_\ncap(\phi) = e^{i \Lcap \cdot \ncap \phi/\Hbar}.
\end{equation}

With respect to a ket it's possible to find
%
\begin{equation}\label{eqn:qmLecture15:120}
e^{i \Lcap \cdot \ncap \phi/\Hbar} \ket{j, m}
=
\sum_{m'} d^j_{m m'}(\ncap, \phi) \ket{ j, m' }.
\end{equation}
%
This has a block diagonal form that's sketched in \cref{fig:qmLecture15:qmLecture15Fig1}.

\imageFigure{../figures/phy1520-quantum/qmLecture15Fig1}{Block diagonal form for angular momentum matrix representation.}{fig:qmLecture15:qmLecture15Fig1}{0.2}

We can view \( d^j_{m m'}(\ncap, \phi) \) as a matrix, representing the rotation.  The problem of determining these matrices can be reduced to that of determining the matrix for \( \Lcap \), because once we have that we can exponentiate that.

\makeexample{Spin half}{example:qmLecture15:1}{

From the eigenvalue relationships, with basis states
%
\begin{equation}\label{eqn:qmLecture15:160}
\begin{aligned}
\ket{\uparrow} &=
\begin{bmatrix}
1 \\
0
\end{bmatrix} \\
\ket{\downarrow} &=
\begin{bmatrix}
0 \\
1 \\
\end{bmatrix}
\end{aligned}
\end{equation}

we find
%
\begin{equation}\label{eqn:qmLecture15:180}
\begin{aligned}
\hatL_z &= \frac{\Hbar}{2} \PauliZ \\
\hatL_{+} &= \frac{\Hbar}{2}
\begin{bmatrix}
0 & 1 \\
0 & 0
\end{bmatrix} \\
\hatL_{-} &= \frac{\Hbar}{2}
\begin{bmatrix}
0 & 0 \\
1 & 0
\end{bmatrix}.
\end{aligned}
\end{equation}

Rearranging we find the Pauli matrices
\index{angular momentum!Pauli matrix}
%
\begin{dmath}\label{eqn:qmLecture15:200}
\hatL_k = \inv{2} \Hbar \sigma_i.
\end{dmath}

Noting that \( \lr{ \Bsigma \cdot \ncap }^2 = 1 \), and \( \lr{\Bsigma \cdot \ncap }^3 = \Bsigma \cdot \ncap \), the rotation matrix is
%
\begin{dmath}\label{eqn:qmLecture15:220}
e^{ i \Bsigma \cdot \ncap \phi/2 } \ket{\inv{2}, m} = \lr{ \cos( \phi/2 ) + i \Bsigma \cdot \ncap \sin(\phi/2) } \ket{\inv{2}, m}.
\end{dmath}
%
} % example

The steps taken in the example above, which apply to all values of \( j \) were

\begin{enumerate}
\item Enumerate the states.
\begin{equation}\label{eqn:qmLecture15:140}
j_1 = \inv{2} \leftrightarrow\, \mbox{2 states (dimension of irreducible representation = 2)}
\end{equation}
%
\item Construct the \( \Lcap \) matrices.
\item Construct \( d_{m m'}^j(\ncap, \phi) \).
\end{enumerate}

\section{Spherical harmonics}
\index{spherical harmonics}
%Angular momentum operator in space representation}

For \( L = 1 \) it turns out that the rotation matrices turn out to be the 3D rotation matrices.  In the space representation
%
\begin{dmath}\label{eqn:qmLecture15:240}
\BL = \Br \cross \Bp,
\end{dmath}
%
the coordinates of the operator are
%
\begin{dmath}\label{eqn:qmLecture15:260}
\hatL_k = i \epsilon_{k m n} r_m \lr{ -i \Hbar \PD{r_n}{} }
\end{dmath}

We see that scaling \( \Br \rightarrow \alpha \Br \) does not change this operator, allowing for an angular representation \( \hatL_k(\theta, \phi) \) that have the form
%
\begin{dmath}\label{eqn:qmLecture15:280}
\begin{aligned}
\hatL_z &= -i \Hbar \PD{\phi}{} \\
\hatL_{\pm} &= \Hbar \lr{ \pm \PD{\theta}{} + i \cot \theta \PD{\phi}{} }.
\end{aligned}
\end{dmath}

Here \( \theta \) and \( \phi \) are the polar and azimuthal angles respectively as illustrated in \cref{fig:qmLecture15:qmLecture15Fig2}.

\imageFigure{../figures/phy1520-quantum/qmLecture15Fig2}{Spherical coordinate convention.}{fig:qmLecture15:qmLecture15Fig2}{0.2}

Introducing the \textAndIndex{spherical harmonics} \( Y_{lm} \), the equivalent wave function representation of the problem is
%
\begin{equation}\label{eqn:qmLecture15:300}
\begin{aligned}
\Lcap Y_{lm}(\theta, \phi) &= \Hbar^2 l (l + 1) Y_{lm}(\theta, \phi) \\
\hatL_z Y_{lm}(\theta, \phi) &= \Hbar m Y_{lm}(\theta, \phi) \\
\end{aligned}
\end{equation}

One can find these functions
%
\begin{dmath}\label{eqn:qmLecture15:320}
Y_{lm}(\theta, \phi) = P_{l, m}(\cos \theta) e^{i m \phi},
\end{dmath}
%
where \( P_{l, m}(\cos \theta) \) are called the \textAndIndex{associated Legendre polynomials}.  This can be applied whenever we have
%
\begin{dmath}\label{eqn:qmLecture15:340}
\antisymmetric{H}{\hatL_k} = 0.
\end{dmath}

where all the eigenfunctions will have the form
%
\begin{dmath}\label{eqn:qmLecture15:360}
\Psi(r, \theta, \phi) = R(r) Y_{lm}(\theta, \phi).
\end{dmath}
%
\section{Addition of angular momentum}
\index{angular momentum!addition}

Since \( \Lcap \) is a vector we expect to be able to add angular momentum in some way similar to the addition of classical vectors as illustrated in \cref{fig:qmLecture15:qmLecture15Fig3}.

\imageFigure{../figures/phy1520-quantum/qmLecture15Fig3}{Classical vector addition.}{fig:qmLecture15:qmLecture15Fig3}{0.15}

When we have a potential that depends only on the difference in position \( V(\Br_1 - \Br_2) \) then we know from classical problems it is effective to work in center of mass coordinates
%
\begin{equation}\label{eqn:qmLecture15:380}
\begin{aligned}
\Rcap_{\textrm{cm}} &= \frac{\rcap_1 + \rcap_2}{2} \\
\Pcap_{\textrm{cm}} &= \pcap_1 + \pcap_2
\end{aligned}
\end{equation}

where
%
\begin{dmath}\label{eqn:qmLecture15:400}
\antisymmetric{\hatR_i}{\hatP_j} = i \Hbar \delta_{ij}.
\end{dmath}

Given
%
\begin{dmath}\label{eqn:qmLecture15:420}
\Lcap_1 + \Lcap_2 = \Lcap_{\textrm{tot}},
\end{dmath}
%
do we have
\begin{dmath}\label{eqn:qmLecture15:440}
\antisymmetric{
\hatL_{\textrm{tot}, i}
}{
\hatL_{\textrm{tot}, j}
}
= i \Hbar \epsilon_{i j k} \hatL_{\textrm{tot}, k} ?
\end{dmath}
%
That is
%
\begin{dmath}\label{eqn:qmLecture15:460}
\antisymmetric{\hatL_{1,i} + \hatL_{1,j}}{\hatL_{2,i} + \hatL_{2,j}} = i \Hbar \epsilon_{i j k} \lr{ \hatL_{1,k} + \hatL_{1,k} }
\end{dmath}
%
FIXME: Right at the end of the lecture, there was a mention of something about whether or not \( \Lcap_1^2 \) and \( \hatL_{1,z} \) were sharply defined, but I missed it.  Ask about this if not covered in the next lecture.

%\EndArticle

      %
% Copyright � 2015 Peeter Joot.  All Rights Reserved.
% Licenced as described in the file LICENSE under the root directory of this GIT repository.
%
%\input{../blogpost.tex}
%\renewcommand{\basename}{qmLecture16}
%\renewcommand{\dirname}{notes/phy1520/}
%\newcommand{\keywords}{PHY1520H}
%\input{../peeter_prologue_print2.tex}
%
%%\usepackage{phy1520}
%\usepackage{peeters_braket}
%%\usepackage{peeters_layout_exercise}
%\usepackage{peeters_figures}
%\usepackage{mathtools}
%\usepackage{enumerate}
%
%\beginArtNoToc
%\generatetitle{PHY1520H Graduate Quantum Mechanics.  Lecture 16: Addition of angular momenta.  Taught by Prof.\ Arun Paramekanti}
%%\chapter{Addition of angular momenta}
%\label{chap:qmLecture16}
%
%\paragraph{Disclaimer}
%
%Peeter's lecture notes from class.  These may be incoherent and rough.
%
%These are notes for the UofT course PHY1520, Graduate Quantum Mechanics, taught by Prof. Paramekanti, covering \textchapref{{3}} \citep{sakurai2014modern} content.
%
% FIXME: merge into previous.
\section{Addition of angular momenta (cont.)}
\index{angular momentum!addition}

\begin{itemize}
\item
For orbital angular momentum
%
\begin{equation}\label{eqn:qmLecture16:20}
\begin{aligned}
\Lcap_1 &= \rcap_1 \cross \pcap_1, \\
\Lcap_1 &= \rcap_1 \cross \pcap_1.
\end{aligned}
\end{equation}
We can show that it is true that
%
\begin{equation}\label{eqn:qmLecture16:40}
\antisymmetric{L_{1i} + L_{2i}}{L_{1j} + L_{2j}} =
i \Hbar \epsilon_{i j k} \lr{ L_{1k} + L_{2k} },
\end{equation}
%
because the angular momentum of the independent particles commute.  Given this is it fair to consider that the sum
%
\begin{equation}\label{eqn:qmLecture16:60}
\Lcap_1 + \Lcap_2,
\end{equation}
is also angular momentum.
\item
Given \( \ket{l_1, m_1} \) and \( \ket{l_2, m_2} \), if a measurement is made of \( \Lcap_1 + \Lcap_2 \), what do we get?
Specifically, what do we get for
%
\begin{equation}\label{eqn:qmLecture16:80}
\lr{\Lcap_1 + \Lcap_2}^2,
\end{equation}
%
and for
\begin{equation}\label{eqn:qmLecture16:100}
\lr{\hatL_{1z} + \hatL_{2z}}?
\end{equation}
%
For the latter, we get
%
\begin{equation}\label{eqn:qmLecture16:120}
\lr{\hatL_{1z} + \hatL_{2z}}\ket{ l_1, m_1 ; l_2, m_2 }
=
\lr{ \Hbar m_1 + \Hbar m_2 } \ket{ l_1, m_1 ; l_2, m_2 }.
\end{equation}
%
Given
\begin{equation}\label{eqn:qmLecture16:140}
\hatL_{1z} + \hatL_{2z} = \hatL_z^{\textrm{tot}},
\end{equation}
%
we find
\begin{equation}\label{eqn:qmLecture16:160}
\begin{aligned}
\antisymmetric{\hatL_z^{\textrm{tot}}}{\Lcap_1^2} &= 0 \\
\antisymmetric{\hatL_z^{\textrm{tot}}}{\Lcap_2^2} &= 0 \\
\antisymmetric{\hatL_z^{\textrm{tot}}}{\Lcap_{1z}} &= 0 \\
\antisymmetric{\hatL_z^{\textrm{tot}}}{\Lcap_{1z}} &= 0.
\end{aligned}
\end{equation}
%
We also find
%
\begin{dmath}\label{eqn:qmLecture16:180}
\antisymmetric{(\Lcap_1 + \Lcap_2)^2}{\Lcap_1^2}
=
\antisymmetric{\Lcap_1^2 + \Lcap_2^2 + 2 \Lcap_1 \cdot \Lcap_2}{\Lcap_1^2}
=
0,
\end{dmath}
%
but for
\begin{dmath}\label{eqn:qmLecture16:200}
\antisymmetric{(\Lcap_1 + \Lcap_2)^2}{\hatL_{1z}}
=
\antisymmetric{\Lcap_1^2 + \Lcap_2^2 + 2 \Lcap_1 \cdot \Lcap_2}{\hatL_{1z}}
=
2 \antisymmetric{\Lcap_1 \cdot \Lcap_2}{\hatL_{1z}}
\ne 0.
\end{dmath}
%
\end{itemize}
Classically if we have measured \( \Lcap_{1} \) and \( \Lcap_{2} \) then we know the total angular momentum as sketched in \cref{fig:qmLecture16:qmLecture16Fig1}.
% FIXME: merge with previous lecture
\imageFigure{../figures/phy1520-quantum/qmLecture16Fig1}{Classical addition of angular momenta.}{fig:qmLecture16:qmLecture16Fig1}{0.15}
In QM where we don't know all the components of the angular momentum simultaneously, things get fuzzier.  For example, if the \( \hatL_{1z} \) and \( \hatL_{2z} \) components have been measured, we have the angular momentum defined within a conical region as sketched in \cref{fig:qmLecture16:qmLecture16Fig2}.
\imageFigure{../figures/phy1520-quantum/qmLecture16Fig2}{Addition of angular momenta given measured \( \hatL_z \).}{fig:qmLecture16:qmLecture16Fig2}{0.2}
Suppose we know \( \hatL_z^{\textrm{tot}} \) precisely, but have imprecise information about \( \lr{\Lcap^{\textrm{tot}}}^2 \).  Can we determine bounds for this?  Let \( \ket{\psi} = \ket{ l_1, m_2 ; l_2, m_2 } \), so
%
\begin{dmath}\label{eqn:qmLecture16:220}
\bra{\psi} \lr{ \Lcap_1 + \Lcap_2 }^2 \ket{\psi}
=
\bra{\psi} \Lcap_1^2 \ket{\psi}
+ \bra{\psi} \Lcap_2^2 \ket{\psi}
+ 2 \bra{\psi} \Lcap_1 \cdot \Lcap_2 \ket{\psi}
=
l_1 \lr{ l_1 + 1} \Hbar^2
+ l_2 \lr{ l_2 + 1} \Hbar^2
+ 2
\bra{\psi} \Lcap_1 \cdot \Lcap_2 \ket{\psi}.
\end{dmath}
%
Using the Cauchy-Schwartz inequality
%
\begin{dmath}\label{eqn:qmLecture16:240}
\Abs{\braket{\phi}{\psi}}^2 \le
\Abs{\braket{\phi}{\phi}}
\Abs{\braket{\psi}{\psi}},
\end{dmath}
%
which is the equivalent of the classical relationship
\begin{dmath}\label{eqn:qmLecture16:260}
\lr{ \BA \cdot \BB }^2 \le \BA^2 \BB^2.
\end{dmath}
%
Applying this to the last term, we have
%
\begin{dmath}\label{eqn:qmLecture16:280}
%\lr{ \underbrace{\bra{\psi} \Lcap_1}_{\phi} \cdot \underbrace{\Lcap_2 \ket{\psi}}_{\psi} }^2
\lr{ \bra{\psi} \Lcap_1 \cdot \Lcap_2 \ket{\psi} }^2
\le
\bra{ \psi} \Lcap_1 \cdot \Lcap_1 \ket{\psi}
\bra{ \psi} \Lcap_2 \cdot \Lcap_2 \ket{\psi}
=
\Hbar^4
l_1 \lr{ l_1 + 1 }
l_2 \lr{ l_2 + 2 }.
\end{dmath}
%
Thus for the max we have
%
\begin{dmath}\label{eqn:qmLecture16:300}
\bra{\psi} \lr{ \Lcap_1 + \Lcap_2 }^2 \ket{\psi}
\le
\Hbar^2 l_1 \lr{ l_1 + 1 }
+\Hbar^2 l_2 \lr{ l_2 + 1 }
+ 2 \Hbar^2 \sqrt{ l_1 \lr{ l_1 + 1 } l_2 \lr{ l_2 + 2 } },
\end{dmath}
and for the min
\begin{dmath}\label{eqn:qmLecture16:360}
\bra{\psi} \lr{ \Lcap_1 + \Lcap_2 }^2 \ket{\psi}
\ge
\Hbar^2 l_1 \lr{ l_1 + 1 }
+\Hbar^2 l_2 \lr{ l_2 + 1 }
-  2 \Hbar^2 \sqrt{ l_1 \lr{ l_1 + 1 } l_2 \lr{ l_2 + 2 } }.
\end{dmath}
%
To try to pretty up these estimate, starting with the max, note that if we replace a portion of the RHS with something bigger, we are left with a strict less than relationship.
That is
%
\begin{equation}\label{eqn:qmLecture16:320}
\begin{aligned}
l_1 \lr{ l_1 + 1 } &< \lr{ l_1 + \inv{2} }^2, \\
l_2 \lr{ l_2 + 1 } &< \lr{ l_2 + \inv{2} }^2.
\end{aligned}
\end{equation}
That is
%
\begin{dmath}\label{eqn:qmLecture16:340}
\bra{\psi} \lr{ \Lcap_1 + \Lcap_2 }^2 \ket{\psi}
<
\Hbar^2
\lr{
l_1 \lr{ l_1 + 1 }
+ l_2 \lr{ l_2 + 1 }
+ 2 \lr{ l_1 + \inv{2} } \lr{ l_2 + \inv{2} }
}
=
\Hbar^2
\lr{
l_1^2 + l_2^2 + l_1 + l_2
+ 2 l_1 l_2 + l_1 + l_2 + \inv{2}
}
=
\Hbar^2
\lr{
\lr{ l_1 + l_2 + \inv{2} }
\lr{ l_1 + l_2 + \frac{3}{2} } - \inv{4}
},
\end{dmath}
or
\begin{dmath}\label{eqn:qmLecture16:380}
l_{\textrm{tot}} \lr{ l_{\textrm{tot}} + 1 }
<
\lr{ l_1 + l_2 + \inv{2} }
\lr{ l_1 + l_2 + \frac{3}{2} }
,
\end{dmath}
%
which, gives
%
\begin{dmath}\label{eqn:qmLecture16:400}
l_{\textrm{tot}} < l_1 + l_2 + \inv{2}.
\end{dmath}
%
Finally, given a quantization requirement, that is
%
\boxedEquation{eqn:qmLecture16:420}{
l_{\textrm{tot}} \le l_1 + l_2.
}
Similarly, for the min, we find
%
\begin{dmath}\label{eqn:qmLecture16:440}
\bra{\psi} \lr{ \Lcap_1 + \Lcap_2 }^2 \ket{\psi}
>
\Hbar^2
\lr{
l_1 \lr{ l_1 + 1 }
+ l_2 \lr{ l_2 + 1 }
- 2 \lr{ l_1 + \inv{2} } \lr{ l_2 + \inv{2} }
}
=
\Hbar^2
\lr{
l_1^2 + l_2^2 %+ \cancel{l_1 + l_2}
- 2 l_1 l_2
%-\cancel{l_1 - l_2}
- \inv{2}
}
=
\Hbar^2
\lr{
\lr{ l_1 - l_2 - \inv{2}}\lr{ l_1 - l_2 + \inv{2}} - \inv{4}
}.
\end{dmath}
%
The total angular momentum quantum number must then satisfy
%
\begin{dmath}\label{eqn:qmLecture16:480}
l_{\textrm{tot}}( l_{\textrm{tot}} + 1 ) >
\lr{ l_1 - l_2 -\inv{2} } \lr{ l_1 - l_2 +\inv{2} } - \inv{4}.
\end{dmath}
Is it true that
%
\begin{dmath}\label{eqn:qmLecture16:500}
l_{\textrm{tot}}( l_{\textrm{tot}} + 1 ) >
\lr{ l_1 - l_2 -\inv{2} } \lr{ l_1 - l_2 +\inv{2} }?
\end{dmath}
This is true when \( l_{\textrm{tot}} > l_1 - l_2 - \inv{2} \), assuming that \( l_1 > l_2 \).  Suppose \( l_{\textrm{tot}} = l_1 - l_2 - \inv{2} \), then
%
\begin{dmath}\label{eqn:qmLecture16:520}
l_{\textrm{tot}}( l_{\textrm{tot}} + 1 )
= \lr{ l_1 - l_2 -\inv{2} } \lr{ l_1 - l_2 +\inv{2} }
= \lr{ l_1 - l_2 }^2 - \inv{4}.
\end{dmath}
%
So, is it true that
\begin{dmath}\label{eqn:qmLecture16:540}
\lr{ l_1 - l_2 }^2 - \inv{4} \ge l_1^2 + l_1 + l_2^2 + l_2 - 2
\sqrt{ l_1(l_1 + 1) l_2( l_2 + 1) }?
\end{dmath}
If that is the case we have
\begin{dmath}\label{eqn:qmLecture16:560}
-2 l_1 l_2 - \inv{4} \ge l_1 + l_2
 - 2 \sqrt{ l_1(l_1 + 1) l_2( l_2 + 1) },
\end{dmath}
%
\begin{dmath}\label{eqn:qmLecture16:580}
 2 \sqrt{ l_2(l_1 + 1) l_1( l_2 + 1) } \ge
l_1 + l_2
+2 l_1 l_2 + \inv{4}
=
l_1(l_2 + 1) + l_2(l_1 + 1) + \inv{4}.
\end{dmath}
%
This has the structure
%
\begin{dmath}\label{eqn:qmLecture16:600}
2 \sqrt{ x y } \ge x + y + \inv{4},
\end{dmath}
%
or
\begin{dmath}\label{eqn:qmLecture16:620}
4 x y \ge (x + y)^2 + \inv{16} + \inv{2}(x + y),
\end{dmath}
%
or
\begin{dmath}\label{eqn:qmLecture16:700}
0 \ge (x - y)^2 + \inv{16} + \inv{2}(x + y).
\end{dmath}
%
But since \( x + y \ge 0 \) this inequality is not satisfied when \( l_{\textrm{tot}} = l_1 - l_2 -\inv{2} \).  We can conclude
%
\begin{equation}\label{eqn:qmLecture16:640}
l_1 - l_2 -\inv{2} < l_{\textrm{tot}} < l_1 + l_2 + \inv{2}.
\end{equation}
%
Is it true that
\begin{equation}\label{eqn:qmLecture16:660}
l_1 - l_2 \ge l_{\textrm{tot}} \ge l_1 + l_2 ?
\end{equation}
Note that we have two separate Hilbert spaces \( l_1 \otimes l_2 \)
of dimension \( 2 l_1 + 1 \) and \( 2 l_2 + 1 \) respectively.  The total number of states is
%
\begin{dmath}\label{eqn:qmLecture16:680}
\sum_{l_{\textrm{tot}} = l_1 - l_2}^{l_1 + l_2}
\lr{ 2 l_{\textrm{tot}} + 1 }
=
2 \sum_{n = l_1 - l_2}^{l_1 + l_2} n + \cancel{l_1} + l_2 - (\cancel{l_1} - l_2) + 1
=
2 \inv{2} \lr{ l_1 + l_2 + (l_1 - l_2) } \lr{ l_1 + l_2 - (l_1 - l_2) + 1 }
+
2 l_2 + 1
=
2 l_1 \lr{ 2 l_2 + 1 }
+
2 l_2 + 1
=
(2 l_1 + 1)
(2 l_2 + 1).
\end{dmath}
%
So the end result is that given \( \ket{l_1, m_1 }, \ket{l_2, m_2} \), with \( l_1 \ge l_2 \), where, in steps of 1,
%
\boxedEquation{eqn:qmLecture16:460}{
l_1 - l_2 \le l_{\textrm{tot}} \le l_1 + l_2.
}

%\EndArticle

      %
% Copyright � 2015 Peeter Joot.  All Rights Reserved.
% Licenced as described in the file LICENSE under the root directory of this GIT repository.
%
%\input{../blogpost.tex}
%\renewcommand{\basename}{qmLecture17}
%\renewcommand{\dirname}{notes/phy1520/}
%\newcommand{\keywords}{PHY1520H}
%\input{../peeter_prologue_print2.tex}
%
%%\usepackage{phy1520}
%\usepackage{peeters_braket}
%%\usepackage{peeters_layout_exercise}
%\usepackage{peeters_figures}
%\usepackage{mathtools}
%%\usepackage{enumerate}
%
%\beginArtNoToc
%\generatetitle{PHY1520H Graduate Quantum Mechanics.  Lecture 17: Clebsch-Gordan.  Taught by Prof.\ Arun Paramekanti}
%%\chapter{Clebsch-Gordan}
%\label{chap:qmLecture17}
%
%\paragraph{Disclaimer}
%
%Peeter's lecture notes from class.  These may be incoherent and rough.
%
%These are notes for the UofT course PHY1520, Graduate Quantum Mechanics, taught by Prof. Paramekanti, covering \textchapref{{1}} \citep{sakurai2014modern} content.

\section{Clebsch-Gordan.}
\index{Clebsch-Gordan}
How can we related total momentum states to individual momentum states in the \( 1 \otimes 2 \) space?
%
\begin{equation}\label{eqn:qmLecture16:720}
\ket{l_1, l_2, l, m } = \sum_{m_1, m_2} C_{l_1 l_2 m_1 m_2}^{l_1 l_2 l m} \ket{l_1 m_1 ; l_2 m_2 }.
\end{equation}
%
The values \( C_{l_1 l_2 m_1 m_2}^{l_1 l_2 l m} \) are called the Clebsch-Gordan coefficients.
\makeexample
{Example: spin one and spin one half}
{example:qmLecture17:1}{
With individual momentum states \( \ket{l_1 m_1 ; l_2 m_2 } \)
%
\begin{equation}\label{eqn:qmLecture16:740}
\begin{aligned}
l_1 &= 1 \\
m_1 &= \pm 1, 0 \\
l_2 &= \inv{2} \\
m_2 &= \pm \inv{2}.
\end{aligned}
\end{equation}
The total angular momentum numbers are
%
\begin{equation}\label{eqn:qmLecture16:760}
l_{\textrm{tot}} \in [l_1 - l_2, l_1 + l_2] = [\ifrac{1}{2}, \ifrac{3}{2}].
\end{equation}
%
The possible states \( \ket{ l_{\textrm{tot}}, m_{\textrm{tot}} } \) are
%
\begin{equation}\label{eqn:qmLecture16:780}
\ket{\inv{2}, \inv{2} }, \ket{\inv{2}, -\inv{2} },
\end{equation}
%
and
\begin{equation}\label{eqn:qmLecture16:800}
\ket{\frac{3}{2}, \frac{3}{2} }, \ket{\frac{3}{2}, \frac{1}{2} },
\ket{\frac{3}{2}, -\frac{1}{2} }
\ket{\frac{3}{2}, -\frac{3}{2} }.
\end{equation}
%
The Clebsch-Gordan procedure is the search for an orthogonal angular momentum basis, built up from the individual momentum bases.  For the total momentum basis we want the basis states to satisfy the ladder operators, but also have them satisfy the consistent ladder operators for the individual particle angular momenta.  This procedure is sketched in \cref{fig:qmLecture17:qmLecture17Fig1}.
\imageFigure{../figures/phy1520-quantum/qmLecture17Fig1}{Spin one,one-half Clebsch-Gordan procedure.}{fig:qmLecture17:qmLecture17Fig1}{0.2}
Demonstrating by example, let the highest total momentum state be proportional to the highest product of individual momentum states
%
\begin{equation}\label{eqn:qmLecture16:820}
\ket{ \frac{3}{2} \frac{3}{2} } = \ket{ 1 1 } \otimes \ket{ \frac{1}{2} \frac{1}{2} }.
\end{equation}
%
A lowered state can be constructed in two different ways, one using the total angular momentum lowering operator
%
\begin{dmath}\label{eqn:qmLecture16:840}
\ket{ \frac{3}{2} \frac{1}{2} }
=
\hatL_{-}^{\textrm{tot}} \ket{ \frac{3}{2} \frac{3}{2} }
= \Hbar \sqrt{\lr{\frac{3}{2} + \frac{3}{2}}\lr{\frac{3}{2} - \frac{3}{2} + 1}} \ket{ \frac{3}{2} \frac{1}{2} }
= \Hbar \sqrt{3} \ket{ \frac{3}{2} \frac{1}{2} }.
\end{dmath}
%
On the other hand, the lowering operator can also be expressed as \( \hatL_{-}^{\textrm{tot}} = \hatL_{-}^{(1)} \otimes 1 + 1 \otimes \hatL_{-}^{(2)} \).  Operating with that gives
%
\begin{dmath}\label{eqn:qmLecture16:860}
\ket{ \frac{3}{2} \frac{1}{2} }
=
\hatL_{-}^{\textrm{tot}} \ket{ 1 1 } \otimes \ket{ \frac{1}{2} \frac{1}{2} }
=
\hatL_{-}^{(1)} \ket{ 1 1 } \otimes \ket{ \frac{1}{2} \frac{1}{2} }
+
\ket{ 1 1 } \otimes \hatL_{-}^2 \ket{ \frac{1}{2} \frac{1}{2} }
=
\Hbar \sqrt{\lr{1 + 1}\lr{1 - 1 + 1}} \ket{ 1 0 } \otimes \ket{ \frac{1}{2} \frac{1}{2} }
+
\Hbar \sqrt{\lr{\frac{1}{2} + \frac{1}{2}}\lr{\frac{1}{2} - \frac{1}{2} + 1}} \ket{ 1 1 } \otimes \ket{ \frac{1}{2} -\frac{1}{2} }
=
\Hbar \sqrt{2} \ket{ 1 0 } \otimes \ket{ \frac{1}{2} \frac{1}{2} }
+
\Hbar \ket{ 1 1 } \otimes \ket{ \frac{1}{2} -\frac{1}{2} }.
\end{dmath}
%%%
Equating both sides and dispensing with the direct product notation, this is
%
\begin{dmath}\label{eqn:qmLecture16:880}
\sqrt{3} \ket{ \frac{3}{2} \frac{1}{2} }
=
\sqrt{2} \ket{ 1 0 ; \frac{1}{2} \frac{1}{2} }
+
\ket{ 1 1 ; \frac{1}{2} -\frac{1}{2} },
\end{dmath}
%
or
%
%\begin{dmath}\label{eqn:qmLecture16:900}
\boxedEquation{eqn:qmLecture16:1020}{
\ket{ \frac{3}{2} \frac{1}{2} }
=
\sqrt{\frac{2}{3}} \ket{ 1 0 ; \frac{1}{2} \frac{1}{2} }
+
\inv{\sqrt{3}} \ket{ 1 1 ; \frac{1}{2} -\frac{1}{2} }.
}
%\end{dmath}
%
This is clearly both a unit ket, and normal to \( \ket{ \frac{3}{2} \frac{3}{2} } \).  We can continue operating with the lowering operator for the total angular momentum to construct all the states down to \( \ket{ \frac{3}{2} \frac{-3}{2} } \).  Working with \( \Hbar = 1 \) since we see it cancel out, the next lower state follows from
%
\begin{dmath}\label{eqn:qmLecture16:920}
\hatL_{-}^{\textrm{tot}} \ket{ \frac{3}{2} \frac{1}{2} }
=
\sqrt{2 \times 2} \ket{ \frac{3}{2} \frac{-1}{2} }
=
2 \ket{ \frac{3}{2} \frac{-1}{2} }.
\end{dmath}
%
and from the individual lowering operators on the components of \( \ket{ \frac{3}{2} \frac{1}{2} } \).
%
\begin{dmath}\label{eqn:qmLecture16:940}
\hatL_{-}^{(1)} \ket{ 1 0 ; \frac{1}{2} \frac{1}{2} }
=
\sqrt{ 1 \times 2 } \ket{ 1 -1 ; \frac{1}{2} \frac{1}{2} },
\end{dmath}
%
and
%
\begin{dmath}\label{eqn:qmLecture16:960}
\hatL_{-}^{(2)} \ket{ 1 0 ; \frac{1}{2} \frac{1}{2} }
=
\sqrt{ 1 \times 1 } \ket{ 1 0 ; \frac{1}{2} \frac{-1}{2} },
\end{dmath}
%
and
%
\begin{dmath}\label{eqn:qmLecture16:980}
\hatL_{-}^1
\ket{ 1 1 ; \frac{1}{2} -\frac{1}{2} }
=
\sqrt{ 2 \times 1 }
\ket{ 1 0 ; \frac{1}{2} -\frac{1}{2} }.
\end{dmath}
%
This gives
%
\begin{dmath}\label{eqn:qmLecture16:1000}
2 \ket{ \frac{3}{2} \frac{-1}{2} } =
\sqrt{\frac{2}{3}} \lr{
\sqrt{ 2 } \ket{ 1 -1 ; \frac{1}{2} \frac{1}{2} }
+
\ket{ 1 0 ; \frac{1}{2} \frac{-1}{2} }
}
+ \inv{\sqrt{3}}
\sqrt{ 2 }
\ket{ 1 0 ; \frac{1}{2} -\frac{1}{2} },
\end{dmath}
%
or
%
%\begin{dmath}\label{eqn:qmLecture16:1000}
\boxedEquation{eqn:qmLecture16:1040}{
\ket{ \frac{3}{2} \frac{-1}{2} } =
\inv{\sqrt{ 3 }} \ket{ 1 -1 ; \frac{1}{2} \frac{1}{2} }
+
\sqrt{\frac{2}{3}}
\ket{ 1 0 ; \frac{1}{2} \frac{-1}{2} }.
}
%\end{dmath}
%
There's one more possible state with total angular momentum \( \frac{3}{2} \).  This time
%
\begin{dmath}\label{eqn:qmLecture16:1060}
\hatL_{-}^{\textrm{tot}}
\ket{ \frac{3}{2} \frac{-1}{2} }
=
\sqrt{ 1 \times 3 }
\ket{ \frac{3}{2} \frac{-3}{2} }
=
\inv{\sqrt{ 3 }} \hatL_{-}^{(2)} \ket{ 1 -1 ; \frac{1}{2} \frac{1}{2} }
+
\sqrt{\frac{2}{3}}
\hatL_{-}^{(1)} \ket{ 1 0 ; \frac{1}{2} \frac{-1}{2} }
=
\inv{\sqrt{ 3 }} \sqrt{1 \times 1 } \ket{ 1 -1 ; \frac{1}{2} \frac{-1}{2} }
+
\sqrt{\frac{2}{3}}
\sqrt{ 1 \times 2 } \ket{ 1 -1 ; \frac{1}{2} \frac{-1}{2} },
\end{dmath}
%
or
%\begin{dmath}\label{eqn:qmLecture16:1080}
\boxedEquation{eqn:qmLecture16:1100}{
\ket{ \frac{3}{2} \frac{-3}{2} }
=
\ket{ 1 -1 ; \frac{1}{2} \frac{-1}{2} }.
}
%\end{dmath}
%
The \( \ket{ \frac{1}{2} \frac{1}{2} } \) state is constructed as normal to \( \ket{ \frac{3}{2} \frac{1}{2} } \), or
%
%\begin{dmath}\label{eqn:qmLecture17:1120}
\boxedEquation{eqn:qmLecture17:1140}{
\ket{ \frac{1}{2} \frac{1}{2} } =
\sqrt{\frac{1}{3}} \ket{ 1 0 ; \frac{1}{2} \frac{1}{2} }
-
\sqrt{ \frac{2}{3} } \ket{ 1 1 ; \frac{1}{2} -\frac{1}{2} },
}
%\end{dmath}
and \( \ket{ \frac{1}{2} -\frac{1}{2} } \) by lowering that.  With
%
\begin{equation}\label{eqn:qmLecture17:1160}
\hatL_{-}^{\textrm{tot}} \ket{ \frac{1}{2} \frac{1}{2} } = \sqrt{ 1 \times 1 } \ket{ \frac{1}{2} -\frac{1}{2} },
\end{equation}
%
we have
%
\begin{dmath}\label{eqn:qmLecture17:1180}
\ket{ \frac{1}{2} -\frac{1}{2} } =
\sqrt{\frac{1}{3}} \lr{
\sqrt{ 1 \times 2 } \ket{ 1 -1 ; \frac{1}{2} \frac{1}{2} }
+ \ket{ 1 0 ; \frac{1}{2} -\frac{1}{2} }
}
-\sqrt{ \frac{2}{3} } \sqrt{ 2 \times 1 } \ket{ 1 0 ; \frac{1}{2} -\frac{1}{2} },
\end{dmath}
%
or
%\begin{dmath}\label{eqn:qmLecture17:1200}
\boxedEquation{eqn:qmLecture17:1220}{
\ket{ \frac{1}{2} -\frac{1}{2} } =
\sqrt{ \frac{2}{3} } \ket{ 1 -1 ; \frac{1}{2} \frac{1}{2} }
- \inv{\sqrt{3}} \ket{ 1 0 ; \frac{1}{2} -\frac{1}{2} }.
}
%\end{dmath}
%
Observe that further lowering this produces zero
%
\begin{dmath}\label{eqn:qmLecture17:1240}
\hatL_{-}^{\textrm{tot}} \ket{ \frac{1}{2} -\frac{1}{2} }
=
\sqrt{ \frac{2}{3} } \sqrt{ 1 \times 1 } \ket{ 1 -1 ; \frac{1}{2} -\frac{1}{2} }
- \inv{\sqrt{3}} \sqrt{ 1 \times 2 } \ket{ 1 -1 ; \frac{1}{2} -\frac{1}{2} }.
= 0.
\end{dmath}
%
All the basis elements have been determined.  In summary
\begin{equation}\label{eqn:qmLecture17:99}
\begin{aligned}
\ket{ \frac{3}{2} \frac{3}{2} } &= \ket{ 1 1; \frac{1}{2} \frac{1}{2} } \\
%
\sqrt{3} \ket{ \frac{3}{2} \frac{1}{2} } &= \sqrt{2} \ket{ 1 0 ; \frac{1}{2} \frac{1}{2} } + \ket{ 1 1 ; \frac{1}{2} -\frac{1}{2} } \\
\sqrt{3} \ket{ \frac{1}{2} \frac{1}{2} } &= \ket{ 1 0 ; \frac{1}{2} \frac{1}{2} } - \sqrt{2} \ket{ 1 1 ; \frac{1}{2} -\frac{1}{2} } \\
%
\sqrt{3} \ket{ \frac{3}{2} \frac{-1}{2} } &= \ket{ 1 -1 ; \frac{1}{2} \frac{1}{2} } + \sqrt{2} \ket{ 1 0 ; \frac{1}{2} \frac{-1}{2} } \\
\sqrt{3} \ket{ \frac{1}{2} -\frac{1}{2} } &= \sqrt{2} \ket{ 1 -1 ; \frac{1}{2} \frac{1}{2} } - \ket{ 1 0 ; \frac{1}{2} -\frac{1}{2} } \\
%
\ket{ \frac{3}{2} \frac{-3}{2} } &= \ket{ 1 -1 ; \frac{1}{2} \frac{-1}{2} }.
\end{aligned}
\end{equation}
%, and are summarized in \cref{tab:qmLecture17:100}.
%\captionedTable{Spin one,half total angular momentum basis.}{tab:qmLecture17:100}{
%\begin{tabular}{|l|l}
%\cline{1-1}
%\( \ket{ \frac{3}{2} \frac{3}{2} } = \ket{ 1 1; \frac{1}{2} \frac{1}{2} } \)
% &                        \\ \hline
%%
%\( \sqrt{3} \ket{ \frac{3}{2} \frac{1}{2} } = \sqrt{2} \ket{ 1 0 ; \frac{1}{2} \frac{1}{2} } + \ket{ 1 1 ; \frac{1}{2} -\frac{1}{2} } \)
% & \multicolumn{1}{l|}{
%\( \sqrt{3} \ket{ \frac{1}{2} \frac{1}{2} } = \ket{ 1 0 ; \frac{1}{2} \frac{1}{2} } - \sqrt{2} \ket{ 1 1 ; \frac{1}{2} -\frac{1}{2} } \)
%} \\ \hline
%%
%\( \sqrt{3} \ket{ \frac{3}{2} \frac{-1}{2} } = \ket{ 1 -1 ; \frac{1}{2} \frac{1}{2} } + \sqrt{2} \ket{ 1 0 ; \frac{1}{2} \frac{-1}{2} } \)
% & \multicolumn{1}{l|}{
%\( \sqrt{3} \ket{ \frac{1}{2} -\frac{1}{2} } = \sqrt{2} \ket{ 1 -1 ; \frac{1}{2} \frac{1}{2} } - \ket{ 1 0 ; \frac{1}{2} -\frac{1}{2} } \)
%} \\ \hline
%%
%\( \ket{ \frac{3}{2} \frac{-3}{2} } = \ket{ 1 -1 ; \frac{1}{2} \frac{-1}{2} } \)
% &                        \\ \cline{1-1}
%\end{tabular}
%}
} % example
%
\makeexample
{Spin two, spin one.}
{example:qmLecture17:2}{
With \( j_1 = 2 \) and \( j_2 = 1 \), we have \( j \in 1,2,3 \), and can proceed the same way as sketched in \cref{fig:qmLecture17:qmLecture17Fig2}.
\imageFigure{../figures/phy1520-quantum/qmLecture17Fig2}{Spin two,one Clebsch-Gordan procedure.}{fig:qmLecture17:qmLecture17Fig2}{0.2}
Working through the details for this example is left to the problem set.
} % example
%\EndArticle

      \section{Problems.}
         %
% Copyright © 2015 Peeter Joot.  All Rights Reserved.
% Licenced as described in the file LICENSE under the root directory of this GIT repository.
%
\makeproblem{Angular momentum commutators.}{problem:qmLecture13:1}{
\index{angular momentum!commutators}

Using \( \hatL_i = \epsilon_{ijk} \hatr_j \hatp_k \), show that
%
\begin{dmath}\label{eqn:qmLecture13:1620}
\antisymmetric{\hatL_i}{\hatL_j} = i \Hbar \epsilon_{ijk} \hatL_k
\end{dmath}

} % problem

\makeanswer{problem:qmLecture13:1}{

Let's start without using abstract index expressions, computing the commutator for \( \hatL_1, \hatL_2 \), which should show the basic steps required
%
\begin{dmath}\label{eqn:qmLecture13:1640}
\antisymmetric{\hatL_1}{\hatL_2}
=
\antisymmetric{\hatr_2 \hatp_3 - \hatr_3 \hatp_2}{\hatr_3 \hatp_1 - \hatr_1 \hatp_3}
=
\antisymmetric{\hatr_2 \hatp_3}{\hatr_3 \hatp_1}
-\antisymmetric{\hatr_2 \hatp_3}{\hatr_1 \hatp_3}
-\antisymmetric{\hatr_3 \hatp_2}{\hatr_3 \hatp_1}
+\antisymmetric{\hatr_3 \hatp_2}{\hatr_1 \hatp_3}.
\end{dmath}

The first of these commutators is
%
\begin{dmath}\label{eqn:qmLecture13:1660}
\antisymmetric{\hatr_2 \hatp_3}{\hatr_3 \hatp_1}
=
{\hatr_2 \hatp_3}{\hatr_3 \hatp_1}
-
{\hatr_3 \hatp_1}
{\hatr_2 \hatp_3}
=
\hatr_2 \hatp_1 \antisymmetric{\hatp_3}{\hatr_3}
=
-i \Hbar \hatr_2 \hatp_1.
\end{dmath}

We see that any factors in the commutator don't have like indexes (i.e. \( \hatr_k, \hatp_k \)) on both position and momentum terms, can be pulled out of the commutator.  This leaves
%
\begin{dmath}\label{eqn:qmLecture13:1680}
\antisymmetric{\hatL_1}{\hatL_2}
=
\hatr_2 \hatp_1 \antisymmetric{\hatp_3}{\hatr_3}
-\cancel{\antisymmetric{\hatr_2 \hatp_3}{\hatr_1 \hatp_3}}
-\cancel{\antisymmetric{\hatr_3 \hatp_2}{\hatr_3 \hatp_1}}
+\hatr_1 \hatp_2 \antisymmetric{\hatr_3}{\hatp_3}
=
i \Hbar \lr{ \hatr_1 \hatp_2 - \hatr_2 \hatp_1 }
=
i \Hbar \hatL_3.
\end{dmath}

With cyclic permutation this is really enough to consider \cref{eqn:qmLecture13:1620} proven.  However, can we do this in the general case with the abstract index expression?  The quantity to simplify looks forbidding
%
\begin{dmath}\label{eqn:qmLecture13:1700}
\antisymmetric{\hatL_i}{\hatL_j}
=
\epsilon_{i a b }
\epsilon_{j s t }
\antisymmetric{ \hatr_a \hatp_b }{ \hatr_s \hatp_t }
\end{dmath}

Because there are no repeated indexes, this doesn't submit to any of the normal reduction identities.  Note however, since we only care about the \( i \ne j \) case, that one of the indexes \( a, b \) must be \( j \) for this quantity to be non-zero.  Therefore (for \( i \ne j \))
%
\begin{dmath}\label{eqn:qmLecture13:1720}
\antisymmetric{\hatL_i}{\hatL_j}
=
\epsilon_{i j b }
\epsilon_{j s t }
\antisymmetric{ \hatr_j \hatp_b }{ \hatr_s \hatp_t   }
+
\epsilon_{i a j }
\epsilon_{j s t }
\antisymmetric{ \hatr_a \hatp_j }{ \hatr_s \hatp_t   }
=
\epsilon_{i j b }
\epsilon_{j s t }
\lr{
\antisymmetric{ \hatr_j \hatp_b }{ \hatr_s \hatp_t   }
-
\antisymmetric{ \hatr_b \hatp_j }{ \hatr_s \hatp_t   }
}
=
-\delta^{s t}_{[i b]}
\antisymmetric{ \hatr_j \hatp_b - \hatr_b \hatp_j }{ \hatr_s \hatp_t }
=
\antisymmetric{ \hatr_j \hatp_b - \hatr_b \hatp_j }{ \hatr_b \hatp_i - \hatr_i \hatp_b }
=
  \antisymmetric{ \hatr_j \hatp_b }{ \hatr_b \hatp_i }
- \cancel{\antisymmetric{ \hatr_j \hatp_b }{ \hatr_i \hatp_b }}
- \cancel{\antisymmetric{ \hatr_b \hatp_j }{ \hatr_b \hatp_i }}
+ \antisymmetric{ \hatr_b \hatp_j }{ \hatr_i \hatp_b }
=
\hatr_j \hatp_i  \antisymmetric{ \hatp_b }{ \hatr_b }
+ \hatr_i \hatp_j \antisymmetric{ \hatr_b }{ \hatp_b }
=
 i \Hbar \lr{ \hatr_i \hatp_j - \hatr_j \hatp_i }
=
 i \Hbar \epsilon_{i j k} \hatr_i \hatp_j .
\end{dmath}

} % answer

%\EndArticle

         % p3.1, 3.2
         %
% Copyright � 2015 Peeter Joot.  All Rights Reserved.
% Licenced as described in the file LICENSE under the root directory of this GIT repository.
%
%\input{../blogpost.tex}
%\renewcommand{\basename}{someSpinProblems}
%\renewcommand{\dirname}{notes/phy1520/}
%%\newcommand{\dateintitle}{}
%%\newcommand{\keywords}{}
%
%\input{../peeter_prologue_print2.tex}
%
%\usepackage{peeters_layout_exercise}
%\usepackage{peeters_braket}
%\usepackage{peeters_figures}
%
%\beginArtNoToc
%
%\generatetitle{Some spin problems}
%%\chapter{Some spin problems}
%%\label{chap:someSpinProblems}
%
%Problems from angular momentum chapter of \citep{sakurai2014modern}.
%
%
\makeoproblem{\( S_y \) eigenvectors.}{problem:someSpinProblems:1}{\citep{sakurai2014modern} pr. 4.1}{
\index{spin half}
Find the eigenvectors of \( \sigma_y \), and then find the probability that a measurement of \( S_y \) will be \( \Hbar/2 \) when the state is initially
%
%\begin{equation}\label{eqn:someSpinProblems:20}
\(
\begin{bmatrix}
\alpha \\
\beta
\end{bmatrix}
\).
%\end{equation}
%
} % problem
%
\makeanswer{problem:someSpinProblems:1}{
%
The eigenvalues should be \( \pm 1 \), which is easily checked
%
\begin{equation}\label{eqn:someSpinProblems:40}
\begin{aligned}
0 &=
\Abs{ \sigma_y - \lambda }
\\ &=
\begin{vmatrix}
-\lambda & -i \\
i & -\lambda
\end{vmatrix}
\\ &=
\lambda^2 - 1.
\end{aligned}
\end{equation}
%
For \( \ket{+} = (a,b)^\T \) we must have
%
\begin{equation}\label{eqn:someSpinProblems:60}
-1 a - i b = 0,
\end{equation}
%
so
%
\begin{equation}\label{eqn:someSpinProblems:80}
\ket{+} \propto
\begin{bmatrix}
-i \\
1
\end{bmatrix},
\end{equation}
%
or
\begin{equation}\label{eqn:someSpinProblems:100}
\ket{+} =
\inv{\sqrt{2}}
\begin{bmatrix}
1 \\
i
\end{bmatrix}.
\end{equation}
%
For \( \ket{-} \) we must have
%
\begin{equation}\label{eqn:someSpinProblems:120}
a - i b = 0,
\end{equation}
%
so
%
\begin{equation}\label{eqn:someSpinProblems:140}
\ket{+} \propto
\begin{bmatrix}
i \\
1
\end{bmatrix},
\end{equation}
%
or
\begin{equation}\label{eqn:someSpinProblems:160}
\ket{+} =
\inv{\sqrt{2}}
\begin{bmatrix}
1 \\
-i
\end{bmatrix}.
\end{equation}
%
The normalized eigenvectors are
%
\boxedEquation{eqn:someSpinProblems:180}{
\ket{\pm} =
\inv{\sqrt{2}}
\begin{bmatrix}
1 \\
\pm i
\end{bmatrix}.
}
For the probability question we are interested in
%
\begin{equation}\label{eqn:someSpinProblems:200}
\begin{aligned}
\Abs{\bra{S_y; +}
\begin{bmatrix}
\alpha \\
\beta
\end{bmatrix}
}^2
&=
\inv{2} \Abs{
\begin{bmatrix}
1 & -i
\end{bmatrix}
\begin{bmatrix}
\alpha \\
\beta
\end{bmatrix}
}^2
\\ &=
\inv{2} \lr{ \Abs{\alpha}^2 + \Abs{\beta}^2 }
\\ &=
\inv{2}.
\end{aligned}
\end{equation}
%
There is a 50\% chance of finding the particle in the \( \ket{S_x;+} \) state, independent of the initial state.
} % answer
%
\makeoproblem{Magnetic Hamiltonian eigenvectors.}{problem:someSpinProblems:2}{\citep{sakurai2014modern} pr. 3.2}{
\index{magnetic field}

Using Pauli matrices, find the eigenvectors for the magnetic spin interaction Hamiltonian
%
\begin{equation}\label{eqn:someSpinProblems:220}
H = - \inv{\Hbar} 2 \mu \BS \cdot \BB.
\end{equation}
} % problem
%
\makeanswer{problem:someSpinProblems:2}{
%
\begin{equation}\label{eqn:someSpinProblems:240}
\begin{aligned}
H
&= - \mu \Bsigma \cdot \BB
\\ &= - \mu \lr{ B_x \PauliX + B_y \PauliY + B_z \PauliZ }
\\ &= - \mu
\begin{bmatrix}
B_z & B_x - i B_y \\
B_x + i B_y & -B_z
\end{bmatrix}.
\end{aligned}
\end{equation}
%
The characteristic equation is
\begin{equation}\label{eqn:someSpinProblems:260}
\begin{aligned}
0
&=
\begin{vmatrix}
-\mu B_z -\lambda & -\mu(B_x - i B_y) \\
-\mu(B_x + i B_y) & \mu B_z - \lambda
\end{vmatrix}
\\ &=
-\lr{ (\mu B_z)^2 - \lambda^2 }
- \mu^2\lr{ B_x^2 - (iB_y)^2 }
\\ &=
\lambda^2 - \mu^2 \BB^2.
\end{aligned}
\end{equation}
%
That is
\boxedEquation{eqn:someSpinProblems:360}{
\lambda = \pm \mu B.
}
Now for the eigenvectors.  We are looking for \( \ket{\pm} = (a,b)^\T \) such that
%
\begin{equation}\label{eqn:someSpinProblems:300}
\begin{aligned}
0
&= (-\mu B_z \mp \mu B) a -\mu(B_x - i B_y) b,
%\\ &= (B_z + B) a \\ &= (B_x - i B_y) b
\end{aligned}
\end{equation}
or
%
\begin{equation}\label{eqn:someSpinProblems:320}
\ket{\pm} \propto
\begin{bmatrix}
B_x - i B_y \\
B_z \pm B
\end{bmatrix}.
\end{equation}
%
This squares to
%
\begin{equation}\label{eqn:someSpinProblems:340}
B_x^2 + B_y^2 + B_z^2 + B^2 \pm 2 B B_z
= 2 B( B \pm B_z ),
\end{equation}
%
so the normalized eigenkets are
\boxedEquation{eqn:someSpinProblems:380}{
\ket{\pm}
=
\inv{\sqrt{2 B( B \pm B_z )}}
\begin{bmatrix}
B_x - i B_y \\
B_z \pm B
\end{bmatrix}.
}
} % answer

%\EndArticle

         % p3.3
         %
% Copyright � 2015 Peeter Joot.  All Rights Reserved.
% Licenced as described in the file LICENSE under the root directory of this GIT repository.
%
%\input{../blogpost.tex}
%\renewcommand{\basename}{unimodularAndRotation}
%\renewcommand{\dirname}{notes/phy1520/}
%%\newcommand{\dateintitle}{}
%%\newcommand{\keywords}{}
%
%\input{../peeter_prologue_print2.tex}
%
%\usepackage{peeters_layout_exercise}
%\usepackage{peeters_braket}
%\usepackage{peeters_figures}
%
%\beginArtNoToc
%
%\generatetitle{Unimodular transformation}
%%\chapter{Unimodular transformation}
%%\label{chap:unimodularAndRotation}

\makeoproblem{Unimodular transformation.}{problem:unimodularAndRotation:1}{\citep{sakurai2014modern} pr. 3.3}{
\index{unimodular transformation}
Given the matrix
%
\begin{dmath}\label{eqn:unimodularAndRotation:20}
U =
\frac
{a_0 + i \sigma \cdot \Ba}
{a_0 - i \sigma \cdot \Ba},
\end{dmath}
%
where \( a_0, \Ba \) are real valued constant and vector respectively.

\makesubproblem{}{problem:unimodularAndRotation:1:a}
Show that this is a unimodular and unitary transformation.
\makesubproblem{}{problem:unimodularAndRotation:1:b}
A unitary transformation can represent an arbitrary rotation.  Determine the rotation angle and direction in terms of \( a_0, \Ba \).
} % problem

\makeanswer{problem:unimodularAndRotation:1}{

\makeSubAnswer{}{problem:unimodularAndRotation:1:a}

Let's call these factors \( A_{\pm} \), which expand to
%
\begin{dmath}\label{eqn:unimodularAndRotation:40}
A_{\pm}
=
a_0 \pm i \sigma \cdot \Ba
=
\begin{bmatrix}
a_0 \pm i a_z  & \pm \lr{ a_y + i a_x} \\
\mp (a_y - i a_x) & a_0 \mp i a_z \\
\end{bmatrix},
\end{dmath}
%
or with \( z = a_0 + i a_z \), and \( w = a_y + i a_x \), these are
%
\begin{dmath}\label{eqn:unimodularAndRotation:120}
A_{+}
=
\begin{bmatrix}
z & w \\
-w^\conj & z^\conj
\end{bmatrix}
\end{dmath}
\begin{dmath}\label{eqn:unimodularAndRotation:180}
A_{-}
=
\begin{bmatrix}
z^\conj & -w \\
w^\conj & z
\end{bmatrix}.
\end{dmath}

These both have a determinant of
\begin{dmath}\label{eqn:unimodularAndRotation:60}
\Abs{z}^2 + \Abs{w}^2
=
\Abs{a_0 + i a_z}^2 + \Abs{a_y + i a_x}^2
= a_0^2 + \Ba^2.
\end{dmath}

The inverse of the latter is
\begin{dmath}\label{eqn:unimodularAndRotation:200}
A_{-}^{-1}
=
\inv{ a_0^2 + \Ba^2 }
\begin{bmatrix}
z & w \\
-w^\conj & z^\conj
\end{bmatrix}
\end{dmath}

Noting that the numerator and denominator commute the inverse can be applied in either order.  Picking one, the transformation of interest, after writing \( A = a_0^2 + \Ba^2 \), is
%
\begin{dmath}\label{eqn:unimodularAndRotation:100}
U
=
\inv{A}
\begin{bmatrix}
z & w \\
-w^\conj & z^\conj
\end{bmatrix}
\begin{bmatrix}
z & w \\
-w^\conj & z^\conj
\end{bmatrix}
=
\inv{A}
\begin{bmatrix}
z^2 - \Abs{w}^2          & w( z + z^\conj) \\
-w^\conj (z^\conj + z )  & (z^\conj)^2 - \Abs{w}^2
\end{bmatrix}.
\end{dmath}

Recall that a unimodular transformation is one that has the form
%
\begin{dmath}\label{eqn:unimodularAndRotation:140}
\begin{bmatrix}
z & w \\
-w^\conj & z^\conj
\end{bmatrix},
\end{dmath}
%
provided \( \Abs{z}^2 + \Abs{w}^2 = 1 \), so \cref{eqn:unimodularAndRotation:100} is unimodular if the following sum is unity, which is the case
%
\begin{dmath}\label{eqn:unimodularAndRotation:160}
\frac{\Abs{z^2 - \Abs{w}^2}^2}{\lr{ \Abs{z}^2 + \Abs{w}^2}^2 } + \Abs{w}^2 \frac{\Abs{z + z^\conj}^2 }{\lr{ \Abs{z}^2 + \Abs{w}^2}^2 }
=
\frac{
\lr{ z^2 - \Abs{w}^2 } \lr{ (z^\conj)^2 - \Abs{w}^2 }
+ \Abs{w}^2 \lr{ z + z^\conj }^2
}{
\lr{ \Abs{z}^2 + \Abs{w}^2}^2
}
=
\frac{
\Abs{z}^4 + \Abs{w}^4 - \Abs{w}^2 \lr{ \cancel{z^2 + (z^\conj)^2} }
+ \Abs{w}^2 \lr{ \cancel{z^2 + (z^\conj)^2} + 2 \Abs{z}^2 }
}{
\lr{ \Abs{z}^2 + \Abs{w}^2}^2
}
= 1.
\end{dmath}

\makeSubAnswer{}{problem:unimodularAndRotation:1:b}
The most general rotation of a vector \( \Ba \), described by Pauli matrices is
%
\begin{dmath}\label{eqn:unimodularAndRotation:220}
e^{i \Bsigma \cdot \ncap \theta/2}
\Bsigma \cdot \Ba
e^{-i \Bsigma \cdot \ncap \theta/2}
=
\Bsigma \cdot \ncap + \lr{ \Bsigma \cdot \Ba - (\Ba \cdot \ncap) \Bsigma \cdot \ncap } \cos \theta + \Bsigma \cdot (\Ba \cross \ncap) \sin\theta.
\end{dmath}

If the unimodular matrix above, applied as \( \Bsigma \cdot \Ba' = U^\dagger \Bsigma \cdot \Ba U \) is to also describe this rotation, we want the equivalence
%
\begin{dmath}\label{eqn:unimodularAndRotation:240}
U = e^{-i \Bsigma \cdot \ncap \theta/2},
\end{dmath}
%
or
%
\begin{dmath}\label{eqn:unimodularAndRotation:260}
\inv{a_0^2 + \Ba^2}
\begin{bmatrix}
a_0^2 - \Ba^2 + 2 i a_0 a_z & 2 a_0 ( a_y + i a_x ) \\
-2 a_0( a_y - i a_x )       & a_0^2 - \Ba^2 - 2 i a_0 a_z
\end{bmatrix}
=
\begin{bmatrix}
\cos(\theta/2) - i n_z \sin(\theta/2)  & (-n_y -i n_x) \sin(\theta/2) \\
-( - n_y + i n_x ) \sin(\theta/2)      & \cos(\theta/2) + i n_z \sin(\theta/2)
\end{bmatrix}.
\end{dmath}

Equating components, that is
\begin{equation}\label{eqn:unimodularAndRotation:280}
\begin{aligned}
\cos(\theta/2) &= \frac{a_0^2 - \Ba^2}{a_0^2 + \Ba^2} \\
-n_x \sin(\theta/2) &= \frac{2 a_0 a_x}{a_0^2 + \Ba^2} \\
-n_y \sin(\theta/2) &= \frac{2 a_0 a_y}{a_0^2 + \Ba^2} \\
-n_z \sin(\theta/2) &= \frac{2 a_0 a_y}{a_0^2 + \Ba^2} \\
\end{aligned}
\end{equation}

Noting that
%
\begin{dmath}\label{eqn:unimodularAndRotation:300}
\begin{aligned}
\sin(\theta/2)
&=
\sqrt{
1 - \frac{(a_0^2 - \Ba^2)^2}{(a_0^2 + \Ba^2)^2}
} \\
&=
\frac{
\sqrt{ (a_0^2 + \Ba^2)^2 - (a_0^2 - \Ba^2)^2 }
}
{
a_0^2 + \Ba^2
} \\
&=
\frac{\sqrt{ 4 a_0^2 \Ba^2 }}{a_0^2 + \Ba^2} \\
&=
\frac{2 a_0 \Abs{\Ba} }{a_0^2 + \Ba^2}
\end{aligned}
\end{dmath}

The vector normal direction can be written
%
\begin{dmath}\label{eqn:unimodularAndRotation:320}
\Bn
= - \frac{2 a_0}{(a_0^2 + \Ba^2) \sin(\theta/2)} \Ba,
\end{dmath}
%
or
%
%\begin{dmath}\label{eqn:unimodularAndRotation:340}
\boxedEquation{eqn:unimodularAndRotation:340}{
\Bn = - \frac{\Ba}{\Abs{\Ba}}.
}
%\end{dmath}

The angle of rotation is
%
%\begin{dmath}\label{eqn:unimodularAndRotation:380}
\boxedEquation{eqn:unimodularAndRotation:380}{
\theta = 2 \Atan \frac{2 a_0 \Abs{\Ba}}{ a_0^2 - \Ba^2}.
}
%\end{dmath}

} % answer

%\EndArticle

         % p3.9
         %
% Copyright � 2015 Peeter Joot.  All Rights Reserved.
% Licenced as described in the file LICENSE under the root directory of this GIT repository.
%
%\input{../blogpost.tex}
%\renewcommand{\basename}{gradQuantumProblemSet6Problem1}
%\renewcommand{\dirname}{notes/phy1520/}
%%\newcommand{\dateintitle}{}
%%\newcommand{\keywords}{}
%
%\input{../peeter_prologue_print2.tex}
%
%\usepackage{peeters_layout_exercise}
%\usepackage{peeters_braket}
%\usepackage{peeters_figures}
%
%\beginArtNoToc
%
%\generatetitle{Determining the rotation angle and normal for a rotation through Euler angles}
%%\chapter{Determining the rotation angle and normal for a rotation through Euler angles}
%%\label{chap:gradQuantumProblemSet6Problem1}
%
\makeoproblem{Rotation through Euler angles.}{problem:gradQuantumProblemSet6Problem1:1}{\citep{sakurai2014modern} pr. 3.9}
%\makeproblem{Sequence of rotations.}{problem:gradQuantumProblemSet6Problem1:1}
{
\index{Euler angle}
Consider a sequence of Euler rotations represented by
%
\begin{dmath}\label{eqn:gradQuantumProblemSet6Problem1:20}
\calD^{1/2}(\alpha, \beta, \gamma)
=
e^{-i \sigma_z \alpha/2} e^{-i \sigma_y \beta/2} e^{-i \sigma_z \gamma_2 }.
\end{dmath}
} % problem

Because rotations form a group, this sequence of rotations corresponds to a single rotation by an angle \( \theta \) about a new axis \( \ncap \). What is \(\theta\)? What is \( \ncap\)?
%
\makeanswer{problem:gradQuantumProblemSet6Problem1:1}{
\withproblemsetsParagraph{
First expand the z-axis rotations into their Pauli matrix form
%
\begin{dmath}\label{eqn:gradQuantumProblemSet6Problem1:200}
\begin{aligned}
e^{-i \sigma_z \mu/2}
&=
\cos(\mu/2) - i \PauliZ \sin(\mu/2) \\
&=
\begin{bmatrix}
\cos(\mu/2) -i \sin(\mu/2) & 0 \\
0 & \cos(\mu/2) +i \sin(\mu/2)
\end{bmatrix} \\
&=
\begin{bmatrix}
e^{-i \mu/2} & 0 \\
0 & e^{i \mu/2}
\end{bmatrix}.
\end{aligned}
\end{dmath}
%
For the y-axis rotation we have
\begin{dmath}\label{eqn:gradQuantumProblemSet6Problem1:220}
\begin{aligned}
e^{-i \sigma_y \beta/2}
&=
\cos(\beta/2) - i \PauliY \sin(\beta/2) \\
&=
\cos(\beta/2)
+
\begin{bmatrix}
0 & -1 \\
1 & 0
\end{bmatrix}
\sin(\beta/2) \\
&=
\begin{bmatrix}
\cos(\beta/2) & - \sin(\beta/2) \\
\sin(\beta/2) & \cos(\beta/2)
\end{bmatrix}.
\end{aligned}
\end{dmath}
%
The composition of rotations is therefore
\begin{dmath}\label{eqn:gradQuantumProblemSet6Problem1:240}
\begin{aligned}
\calD^{1/2}(\alpha, \beta, \gamma)
&=
\begin{bmatrix}
e^{-i \alpha/2} & 0 \\
0 & e^{i \alpha/2}
\end{bmatrix}
\begin{bmatrix}
\cos(\beta/2) & - \sin(\beta/2) \\
\sin(\beta/2) & \cos(\beta/2)
\end{bmatrix}
\begin{bmatrix}
e^{-i \gamma/2} & 0 \\
0 & e^{i \gamma/2}
\end{bmatrix} \\
&=
\begin{bmatrix}
e^{-i \alpha/2} & 0 \\
0 & e^{i \alpha/2}
\end{bmatrix}
\begin{bmatrix}
e^{-i \gamma/2} \cos(\beta/2) & - e^{i \gamma/2} \sin(\beta/2) \\
e^{-i \gamma/2} \sin(\beta/2) & e^{i \gamma/2} \cos(\beta/2)
\end{bmatrix} \\
&=
\begin{bmatrix}
e^{-i \alpha/2} e^{-i \gamma/2} \cos(\beta/2) & - e^{-i \alpha/2} e^{i \gamma/2} \sin(\beta/2) \\
e^{i \alpha/2} e^{-i \gamma/2} \sin(\beta/2) & e^{+i \alpha/2} e^{i \gamma/2} \cos(\beta/2)
\end{bmatrix} \\
&=
\begin{bmatrix}
e^{-i(\alpha+\gamma)/2} \cos \frac{\beta}{2} & -e^{-i(\alpha-\gamma)/2} \sin \frac{\beta}{2} \\
e^{i(\alpha-\gamma)/2} \sin \frac{\beta}{2} & e^{i(\alpha+\gamma)/2} \cos \frac{\beta}{2}
\end{bmatrix}.
\end{aligned}
\end{dmath}
%
Compare this to the matrix for a rotation (double sided) about a normal, given by
%
\begin{dmath}\label{eqn:gradQuantumProblemSet6Problem1:40}
\calR
= e^{-i \Bsigma \cdot \ncap \theta/2}
= \cos \frac{\theta}{2} I - i \Bsigma \cdot \ncap \sin \frac{\theta}{2}.
\end{dmath}
%
With \( \ncap = \lr{ \sin \Theta \cos\Phi, \sin \Theta \sin\Phi, \cos\Theta} \), the normal direction in its Pauli basis is
%
\begin{dmath}\label{eqn:gradQuantumProblemSet6Problem1:60}
\Bsigma \cdot \ncap
=
\begin{bmatrix}
\cos\Theta        & \sin \Theta \cos\Phi - i \sin \Theta \sin\Phi \\
\sin \Theta \cos\Phi + i \sin \Theta \sin\Phi & -\cos\Theta
\end{bmatrix}
=
\begin{bmatrix}
\cos\Theta        & \sin \Theta e^{-i \Phi} \\
\sin \Theta e^{i \Phi} & -\cos\Theta
\end{bmatrix},
\end{dmath}
%
so
%
\begin{dmath}\label{eqn:gradQuantumProblemSet6Problem1:80}
\calR =
\begin{bmatrix}
\cos \frac{\theta}{2} -i \sin \frac{\theta}{2} \cos\Theta & -i \sin \Theta e^{-i \Phi} \sin \frac{\theta}{2} \\
-i \sin \Theta e^{i \Phi} \sin \frac{\theta}{2}           & \cos \frac{\theta}{2} +i \sin \frac{\theta}{2} \cos\Theta \\
\end{bmatrix}.
\end{dmath}
%
It's not obvious how to put this into correspondence with the matrix for the Euler rotations.  Doing so certainly doesn't look fun.  To solve this problem, let's go the opposite direction, and put the matrix for the Euler rotations into the form of \cref{eqn:gradQuantumProblemSet6Problem1:40}.
That is
\begin{dmath}\label{eqn:gradQuantumProblemSet6Problem1:100}
\begin{aligned}
\calD^{1/2}(\alpha, \beta, \gamma)
&=
\begin{bmatrix}
e^{-i(\alpha+\gamma)/2} \cos \frac{\beta}{2} & -e^{-i(\alpha-\gamma)/2} \sin \frac{\beta}{2} \\
e^{i(\alpha-\gamma)/2} \sin \frac{\beta}{2} & e^{i(\alpha+\gamma)/2} \cos \frac{\beta}{2}
\end{bmatrix} \\
&=
\begin{bmatrix}
\cos\frac{\alpha+\gamma}{2} \cos \frac{\beta}{2} & - \cos\frac{\alpha-\gamma}{2} \sin \frac{\beta}{2} \\
\cos\frac{\alpha-\gamma}{2} \sin \frac{\beta}{2} & \cos\frac{\alpha+\gamma}{2} \cos \frac{\beta}{2}
\end{bmatrix} \\
&\quad +
i
\begin{bmatrix}
- \sin\frac{\alpha+\gamma}{2} \cos \frac{\beta}{2} & \sin\frac{\alpha-\gamma}{2} \sin \frac{\beta}{2} \\
 \sin\frac{\alpha-\gamma}{2} \sin \frac{\beta}{2} & \sin\frac{\alpha+\gamma}{2} \cos \frac{\beta}{2}
\end{bmatrix} \\
&=
\lr{\cos\frac{\alpha+\gamma}{2} \cos \frac{\beta}{2} }
+ \lr{i \sin\frac{\alpha-\gamma}{2} \sin \frac{\beta}{2} } \sigma_x \\
&\qquad - \lr{i \cos\frac{\alpha-\gamma}{2} \sin \frac{\beta}{2} } \sigma_y
- \lr{i \sin\frac{\alpha+\gamma}{2} \cos \frac{\beta}{2} } \sigma_z.
\end{aligned}
\end{dmath}
%
This gives us
%
\begin{equation}\label{eqn:gradQuantumProblemSet6Problem1:120}
\begin{aligned}
\cos\frac{\theta}{2} &= \cos\frac{\alpha+\gamma}{2} \cos \frac{\beta}{2} \\
\ncap \sin\frac{\theta}{2} &= \lr{ -\sin\frac{\alpha-\gamma}{2} \sin \frac{\beta}{2}, \cos\frac{\alpha-\gamma}{2} \sin \frac{\beta}{2}, \sin\frac{\alpha+\gamma}{2} \cos \frac{\beta}{2} }.
\end{aligned}
\end{equation}
%
The angle is
%
\begin{dmath}\label{eqn:gradQuantumProblemSet6Problem1:140}
\theta
= 2 \Atan \frac{
\sqrt{\sin^2\frac{\beta}{2} + \sin^2\frac{\alpha+\gamma}{2} \cos^2\frac{\beta}{2}
}
}{\cos\frac{\alpha+\gamma}{2} \cos \frac{\beta}{2}},
\end{dmath}
%
or
%\begin{dmath}\label{eqn:gradQuantumProblemSet6Problem1:180}
\boxedEquation{eqn:gradQuantumProblemSet6Problem1:180}{
\theta = 2 \Atan \frac{
\sqrt{\tan^2\frac{\beta}{2} + \sin^2\frac{\alpha+\gamma}{2}
}
}{\cos\frac{\alpha+\gamma}{2}
},
}
%\end{dmath}
and the normal direction is
%\begin{dmath}\label{eqn:gradQuantumProblemSet6Problem1:160}
\boxedEquation{eqn:gradQuantumProblemSet6Problem1:160}{
\begin{aligned}
\ncap
&=
\inv{\sqrt{1 - \cos^2\frac{\alpha+\gamma}{2} \cos^2\frac{\beta}{2} }} \\
&\qquad
\lr{ -\sin\frac{\alpha-\gamma}{2} \sin \frac{\beta}{2}, \cos\frac{\alpha-\gamma}{2} \sin \frac{\beta}{2}, \sin\frac{\alpha+\gamma}{2} \cos \frac{\beta}{2} }.
\end{aligned}
}
%\end{dmath}
}
} % answer
%\EndArticle

         %
% Copyright � 2015 Peeter Joot.  All Rights Reserved.
% Licenced as described in the file LICENSE under the root directory of this GIT repository.
%
\makeoproblem{Angular momentum addition.}{gradQuantum:problemSet6:2}{2015 ps6 p2}{
\index{angular momentum!addition}
\index{Clebsch-Gordan}
%\makesubproblem{}{gradQuantum:problemSet6:2a}
You have to add angular momenta \( j_1 = 1 \) and \( j_2 = 2 \) to form total angular momentum states with \( j = 1,2,3\).  There are a total of \( 15 \) \( \ket{j,m} \) states in the total angular momentum basis.  Express each of them in terms of the old basis \( \ket{j_1, j_2 ; m_1, m_2 } \) set.
} % makeproblem

\makeanswer{gradQuantum:problemSet6:2}{
\withproblemsetsParagraph{
%\makeSubAnswer{}{gradQuantum:problemSet6:2a}

This problem was computed using \nbref{ps6:clebschGordan2.nb}.
\index{Mathematica}
}
}

         %
% Copyright � 2015 Peeter Joot.  All Rights Reserved.
% Licenced as described in the file LICENSE under the root directory of this GIT repository.
%
\makeoproblem{Spin-1 rotations.}{gradQuantum:problemSet6:3}{2015 ps6 p3}{
\index{spin one}
\index{rotation}
%\makesubproblem{}{gradQuantum:problemSet6:3a}

Consider angular momentum \( j = 1 \).

\makesubproblem{}{gradQuantum:problemSet6:3a}

Express \( \bra{j = 1, m'} \hatJ_y \ket{ j = 1, m } \) as a \( 3 \times 3 \) matrix.

\makesubproblem{}{gradQuantum:problemSet6:3b}
Show that for \( j = 1 \) % , we can replace (units where \( \Hbar = 1 \)),
\begin{dmath}\label{eqn:gradQuantumProblemSet6Problem3:20}
e^{-i \hatJ_y \beta/\Hbar}
=
1 - i \frac{\hatJ_y}{\Hbar} \sin\beta - \frac{\hatJ_y^2}{\Hbar^2} \lr{ 1 - \cos\beta }.
\end{dmath}
} % makeproblem

\makeanswer{gradQuantum:problemSet6:3}{
\withproblemsetsParagraph{
\makeSubAnswer{}{gradQuantum:problemSet6:3a}

From \citep{sakurai2014modern} (5.41), for \( j' = j = 1 \), the matrix elements for the ladder operators can be summarized as
%
\begin{dmath}\label{eqn:gradQuantumProblemSet6Problem3:140}
\bra{j, m'} \hatJ_{\pm} \ket{j, m} = \Hbar \sqrt{(1 \mp m)(2 \pm m)} \delta_{m', m\pm 1}.
\end{dmath}
%
With
%
\begin{equation}\label{eqn:gradQuantumProblemSet6Problem3:160}
\begin{aligned}
\hatJ_{+} &= \hatJ_x + i \hatJ_y \\
\hatJ_{-} &= \hatJ_x - i \hatJ_y,
\end{aligned}
\end{equation}

we have
%
\begin{dmath}\label{eqn:gradQuantumProblemSet6Problem3:180}
\hatJ_y = \frac{\hatJ_{+} - \hatJ_{-}}{2i},
\end{dmath}
%
so
%
\begin{dmath}\label{eqn:gradQuantumProblemSet6Problem3:200}
\frac{2 i}{\Hbar} \bra{j, m'} \hatJ_y \ket{j, m} = \sqrt{(1 - m)(2 + m)} \delta_{m', m + 1} -\sqrt{(1 + m)(2 - m)} \delta_{m', m - 1}.
\end{dmath}
%
We have nine matrix elements
%
\begin{equation}\label{eqn:gradQuantumProblemSet6Problem3:240}
\begin{aligned}
\frac{2 i}{\Hbar} \bra{j, 1} \hatJ_y \ket{j, 1} &= \sqrt{(1 - 1)(2 + 1)} \delta_{1, 1 + 1} -\sqrt{(1 + 1)(2 - 1)} \delta_{1, 1 - 1} = 0 \\
\frac{2 i}{\Hbar} \bra{j, 1} \hatJ_y \ket{j, 0} &= \sqrt{(1 - 0)(2 + 0)} \delta_{1, 0 + 1} -\sqrt{(1 + 0)(2 - 0)} \delta_{1, 0 - 1} = \sqrt{2} \\
\frac{2 i}{\Hbar} \bra{j, 1} \hatJ_y \ket{j, -1} &= \sqrt{(1 - -1)(2 + -1)} \delta_{1, -1 + 1} -\sqrt{(1 + -1)(2 - -1)} \delta_{1, -1 - 1} = 0 \\
\frac{2 i}{\Hbar} \bra{j, 0} \hatJ_y \ket{j, 1} &= \sqrt{(1 - 1)(2 + 1)} \delta_{0, 1 + 1} -\sqrt{(1 + 1)(2 - 1)} \delta_{0, 1 - 1} = -\sqrt{2} \\
\frac{2 i}{\Hbar} \bra{j, 0} \hatJ_y \ket{j, 0} &= \sqrt{(1 - 0)(2 + 0)} \delta_{0, 0 + 1} -\sqrt{(1 + 0)(2 - 0)} \delta_{0, 0 - 1} = 0 \\
\frac{2 i}{\Hbar} \bra{j, 0} \hatJ_y \ket{j, -1} &= \sqrt{(1 - -1)(2 + -1)} \delta_{0, -1 + 1} -\sqrt{(1 + -1)(2 - -1)} \delta_{0, -1 - 1} = \sqrt{2} \\
\frac{2 i}{\Hbar} \bra{j, -1} \hatJ_y \ket{j, 1} &= \sqrt{(1 - 1)(2 + 1)} \delta_{-1, 1 + 1} -\sqrt{(1 + 1)(2 - 1)} \delta_{-1, 1 - 1} = 0 \\
\frac{2 i}{\Hbar} \bra{j, -1} \hatJ_y \ket{j, 0} &= \sqrt{(1 - 0)(2 + 0)} \delta_{-1, 0 + 1} -\sqrt{(1 + 0)(2 - 0)} \delta_{-1, 0 - 1} = -\sqrt{2} \\
\frac{2 i}{\Hbar} \bra{j, -1} \hatJ_y \ket{j, -1} &= \sqrt{(1 - -1)(2 + -1)} \delta_{-1, -1 + 1} -\sqrt{(1 + -1)(2 - -1)} \delta_{-1, -1 - 1} = 0.
\end{aligned}
\end{equation}

Put into matrix form, that is
%
\begin{dmath}\label{eqn:gradQuantumProblemSet6Problem3:260}
\bra{j, m'} \hatJ_y \ket{j, m}
=
\frac{\Hbar}{2 i}
\begin{bmatrix}
0 & \sqrt{2} & 0 \\
-\sqrt{2} & 0 & \sqrt{2} \\
0 & -\sqrt{2} & 0
\end{bmatrix},
\end{dmath}
%
or
%\begin{dmath}\label{eqn:gradQuantumProblemSet6Problem3:40}
\boxedEquation{eqn:gradQuantumProblemSet6Problem3:40}{
\hatJ_y
=
\frac{\Hbar i}{\sqrt{2}}
\begin{bmatrix}
0 & -1 &  0 \\
1 &  0 & -1 \\
0 &  1 &  0 \\
\end{bmatrix}.
}
%\end{dmath}

\makeSubAnswer{}{gradQuantum:problemSet6:3b}

The square of the matrix representation of \( \hatJ_y \) of \cref{eqn:gradQuantumProblemSet6Problem3:40} is
%
\begin{dmath}\label{eqn:gradQuantumProblemSet6Problem3:60}
\hatJ_y^2
=
-\frac{\Hbar^2}{2}
\begin{bmatrix}
0 & -1 &  0 \\
1 &  0 & -1 \\
0 &  1 &  0 \\
\end{bmatrix}
\begin{bmatrix}
0 & -1 &  0 \\
1 &  0 & -1 \\
0 &  1 &  0 \\
\end{bmatrix}
=
-\frac{\Hbar^2}{2}
\begin{bmatrix}
-1 & 0 & 1 \\
0 & -2 & 0 \\
1 & 0 & -1
\end{bmatrix},
\end{dmath}
%
and the cube is
\begin{dmath}\label{eqn:gradQuantumProblemSet6Problem3:80}
\hatJ_y^3
=
-\frac{\Hbar^2}{2}
\frac{\Hbar i}{\sqrt{2}}
\begin{bmatrix}
-1 & 0 & 1 \\
0 & -2 & 0 \\
1 & 0 & -1
\end{bmatrix}
\begin{bmatrix}
0 & -1 &  0 \\
1 &  0 & -1 \\
0 &  1 &  0 \\
\end{bmatrix}
=
-\frac{i\Hbar^3}{2 \sqrt{2}}
\begin{bmatrix}
0 & 2 & 0 \\
-2 & 0 & 2 \\
0 & -2 & 0
\end{bmatrix}
=
\Hbar^2 \frac{i \Hbar}{\sqrt{2}}
\begin{bmatrix}
0 & -1 &  0 \\
1 &  0 & -1 \\
0 &  1 &  0 \\
\end{bmatrix}
=
\Hbar^2 \hatJ_y.
\end{dmath}

This proves (5.55) from the text
\begin{dmath}\label{eqn:gradQuantumProblemSet6Problem3:100}
\frac{\hatJ_y^3}{\Hbar^3} = \frac{\hatJ_y}{\Hbar}.
\end{dmath}

In particular
\begin{dmath}\label{eqn:gradQuantumProblemSet6Problem3:220}
\begin{aligned}
\lr{ \frac{\hatJ_y}{\Hbar} }^0 &= 1 \\
\lr{ \frac{\hatJ_y}{\Hbar} }^1 &= \lr{ \frac{\hatJ_y}{\Hbar} }^1 \\
\lr{ \frac{\hatJ_y}{\Hbar} }^2 &= \lr{ \frac{\hatJ_y}{\Hbar} }^2 \\
\lr{ \frac{\hatJ_y}{\Hbar} }^3 &=      \frac{\hatJ_y}{\Hbar} \\
\lr{ \frac{\hatJ_y}{\Hbar} }^4 &= \lr{ \frac{\hatJ_y}{\Hbar}}^2 \\
\lr{ \frac{\hatJ_y}{\Hbar} }^5 &=      \frac{\hatJ_y}{\Hbar} \\
& \vdots
\end{aligned}
\end{dmath}

Expanding the exponential in series we have
\begin{dmath}\label{eqn:gradQuantumProblemSet6Problem3:120}
\begin{aligned}
e^{-i \beta \hatJ_y/\Hbar}
&=
1
- \frac{i \beta \hatJ_y/\Hbar}{1!}
+ \frac{\lr{-i \beta \hatJ_y/\Hbar}^2}{2!}
+ \frac{\lr{-i \beta \hatJ_y/\Hbar}^3}{3!} \\
&\quad + \frac{\lr{-i \beta \hatJ_y/\Hbar}^4}{4!}
+ \frac{\lr{-i \beta \hatJ_y/\Hbar}^5}{5!}
+ \frac{\lr{-i \beta \hatJ_y/\Hbar}^6}{6!}
+ \cdots \\
&=
1
- i \frac{\beta \hatJ_y/\Hbar}{1!}
- \frac{\lr{ \beta \hatJ_y/\Hbar}^2}{2!}
+ i \frac{\lr{ \beta \hatJ_y/\Hbar}^3}{3!} \\
&\qquad + \frac{\lr{ \beta \hatJ_y/\Hbar}^4}{4!}
- i \frac{\lr{ \beta \hatJ_y/\Hbar}^5}{5!}
- \frac{\lr{ \beta \hatJ_y/\Hbar}^6}{6!}
+ \cdots \\
&=
1
+ i \frac{\hatJ_y}{\Hbar} \lr{ - \beta + \frac{\beta^3}{3!} - \frac{\beta^5}{5!} + \cdots } \\
&+ \frac{\hatJ_y^2}{\Hbar^2} \lr{ - \frac{\beta^2}{2!} + \frac{\beta^4}{4!} - \frac{\beta^6}{6!} + \cdots } \\
&=
1 - i \frac{\hatJ_y}{\Hbar} \sin\beta + \frac{\hatJ_y^2}{\Hbar^2} \lr{ -1 + \cos\beta } \qedmarker
\end{aligned}
\end{dmath}

Note that \( J_x^3/\Hbar^3 = J_x/\Hbar \) and \( J_z^3/\Hbar^3 = J_z/\Hbar \) too, so this relation generalizes to the other spin one operators as well.

%which proves \cref{eqn:gradQuantumProblemSet6Problem3:20}.
}
}

         % p3.17
         %
% Copyright � 2015 Peeter Joot.  All Rights Reserved.
% Licenced as described in the file LICENSE under the root directory of this GIT repository.
%
%\input{../blogpost.tex}
%\renewcommand{\basename}{LsquaredLzProblem}
%\renewcommand{\dirname}{notes/phy1520/}
%%\newcommand{\dateintitle}{}
%%\newcommand{\keywords}{}
%
%\input{../peeter_prologue_print2.tex}
%
%\usepackage{peeters_layout_exercise}
%\usepackage{peeters_braket}
%\usepackage{peeters_figures}
%\usepackage{enumerate}
%
%\beginArtNoToc
%
%%\generatetitle{Lz and Ls eigenvalues and probabilities for a wave function}
%\generatetitle{\(L_z\) and \( \BL^2 \) eigenvalues and probabilities for a wave function}
%%\chapter{Lz and Ls eigenvalues and probabilities for a wave function}
%%\label{chap:LsquaredLzProblem}
%
\makeoproblem{\(L_z\) and \( \BL^2 \) eigenvalues and probabilities for a wave function.}{problem:LsquaredLzProblem:1}{\citep{sakurai2014modern} pr. 3.17}{
\index{angular momentum}

Given a wave function
%
\begin{dmath}\label{eqn:LsquaredLzProblem:20}
\psi(r,\theta, \phi) = f(r) \lr{ x + y + 3 z },
\end{dmath}
%
\makesubproblem{}{problem:LsquaredLzProblem:1:a}
Determine if this wave function is an eigenfunction of \( \BL^2 \), and the value of \( l \) if it is an eigenfunction.
\makesubproblem{}{problem:LsquaredLzProblem:1:b}
Determine the probabilities for the particle to be found in any given \( \ket{l, m} \) state,
\makesubproblem{}{problem:LsquaredLzProblem:1:c}
If it is known that \( \psi \) is an energy eigenfunction with energy \( E \) indicate how we can find \( V(r) \).
%
} % problem
%
\makeanswer{problem:LsquaredLzProblem:1}{
%
\makeSubAnswer{}{problem:LsquaredLzProblem:1:a}
%
Using
\begin{equation}\label{eqn:LsquaredLzProblem:40}
\BL^2
=
-\Hbar^2 \lr{ \inv{\sin^2\theta} \partial_{\phi\phi} + \inv{\sin\theta} \partial_\theta \lr{ \sin\theta \partial_\theta} },
\end{equation}
%
and
%
\begin{equation}\label{eqn:LsquaredLzProblem:60}
\begin{aligned}
x &= r \sin\theta \cos\phi \\
y &= r \sin\theta \sin\phi \\
z &= r \cos\theta
\end{aligned}
\end{equation}

it's a quick computation to show that
%
\begin{equation}\label{eqn:LsquaredLzProblem:80}
\BL^2 \psi = 2 \Hbar^2 \psi = 1(1 + 1) \Hbar^2 \psi,
\end{equation}
%
so this function is an eigenket of \( \BL^2 \) with an eigenvalue of \( 2 \Hbar^2 \), which corresponds to \( l = 1 \), a p-orbital state.
%
\makeSubAnswer{}{problem:LsquaredLzProblem:1:b}
%
Recall that the angular representation of \( L_z \) is
%
\begin{equation}\label{eqn:LsquaredLzProblem:100}
L_z = -i \Hbar \PD{\phi},
\end{equation}
%
so we have
%
\begin{equation}\label{eqn:LsquaredLzProblem:120}
\begin{aligned}
L_z x &= i \Hbar y \\
L_z y &= - i \Hbar x \\
L_z z &= 0,
\end{aligned}
\end{equation}

The \( L_z \) action on \( \psi \) is
%
\begin{equation}\label{eqn:LsquaredLzProblem:140}
L_z \psi = -i \Hbar r f(r) \lr{ - y + x }.
\end{equation}
%
This wave function is not an eigenket of \( L_z \).  Expressed in terms of the \( L_z \) basis states \( e^{i m \phi} \), this wave function is
%
\begin{dmath}\label{eqn:LsquaredLzProblem:160}
\psi
= r f(r) \lr{ \sin\theta \lr{ \cos\phi + \sin\phi} + \cos\theta }
= r f(r) \lr{ \frac{\sin\theta}{2} \lr{ e^{i \phi} \lr{ 1 + \inv{i}} + e^{-i\phi} \lr{ 1 - \inv{i} } } + \cos\theta }
= r f(r) \lr{
\frac{(1-i)\sin\theta}{2} e^{1 i \phi}
+
\frac{(1+i)\sin\theta}{2} e^{- 1 i \phi}
+ \cos\theta e^{0 i \phi}
}
\end{dmath}

Assuming that \( \psi \) is normalized, the probabilities for measuring \( m = 1,-1,0 \) respectively are
%
\begin{dmath}\label{eqn:LsquaredLzProblem:180}
P_{\pm 1}
= 2 \pi \rho \Abs{\frac{1\mp i}{2}}^2 \int_0^\pi \sin\theta d\theta \sin^2 \theta
= -2 \pi \rho \int_1^{-1} du (1-u^2)
= 2 \pi \rho \evalrange{ \lr{ u - \frac{u^3}{3} } }{-1}{1}
= 2 \pi \rho \lr{ 2 - \frac{2}{3}}
= \frac{ 8 \pi \rho}{3},
\end{dmath}
%
and
%
\begin{dmath}\label{eqn:LsquaredLzProblem:200}
P_{0} = 2 \pi \rho \int_0^\pi \sin\theta \cos\theta = 0,
\end{dmath}
%
where
%
\begin{dmath}\label{eqn:LsquaredLzProblem:220}
\rho = \int_0^\infty r^4 \Abs{f(r)}^2 dr.
\end{dmath}
%
Because the probabilities must sum to 1, this means the \( m = \pm 1 \) states are equiprobable with \( P_{\pm} = 1/2 \), fixing \( \rho = 3/16\pi \), even without knowing \( f(r) \).
%
\makeSubAnswer{}{problem:LsquaredLzProblem:1:c}
%
The operator \( r^2 \Bp^2 \) can be decomposed into a \( \BL^2 \) component and some other portions, from which we can write
%
\begin{dmath}\label{eqn:LsquaredLzProblem:240}
H \psi
= \lr{ \frac{\Bp^2}{2m} + V(r)  } \psi
=
\lr{
- \frac{\Hbar^2}{2m} \lr{ \partial_{rr} + \frac{2}{r} \partial_r - \inv{\Hbar^2 r^2} \BL^2 } + V(r) } \psi.
\end{dmath}
%
(See: \citep{sakurai2014modern} eq. 6.21)

In this case where \( \BL^2 \psi = 2 \Hbar^2 \psi \) we can rearrange for \( V(r) \)
%
\begin{dmath}\label{eqn:LsquaredLzProblem:260}
V(r)
= E + \inv{\psi} \frac{\Hbar^2}{2m} \lr{ \partial_{rr} + \frac{2}{r} \partial_r - \frac{2}{r^2} } \psi
= E + \inv{f(r)} \frac{\Hbar^2}{2m} \lr{ \partial_{rr} + \frac{2}{r} \partial_r - \frac{2}{r^2} } f(r).
\end{dmath}
%
See \nbref{sakuraiProblem3.17.nb} for some verifications of some of the algebra for this problem.
} % answer

%\EndArticle

         % p3.18
         %
% Copyright � 2015 Peeter Joot.  All Rights Reserved.
% Licenced as described in the file LICENSE under the root directory of this GIT repository.
%
%{
%\input{../blogpost.tex}
%\renewcommand{\basename}{angularMomentumExpectation}
%\renewcommand{\dirname}{notes/phy1520/}
%%\newcommand{\dateintitle}{}
%%\newcommand{\keywords}{}
%
%\input{../peeter_prologue_print2.tex}
%
%\usepackage{peeters_layout_exercise}
%\usepackage{peeters_braket}
%\usepackage{peeters_figures}
%
%\beginArtNoToc
%
%\generatetitle{Angular momentum expectation}
%%\chapter{Angular momentum expectation}
%%\label{chap:angularMomentumExpectation}
%
\makeoproblem{Angular momentum expectation values.}{problem:angularMomentumExpectation:180}{\citep{sakurai2014modern} pr. 3.18}{
\index{angular momentum!expectation}
Compute the expectation values for the first and second powers of the angular momentum operators with respect to states \( \ket{lm} \).
%
} % problem
%
\makeanswer{problem:angularMomentumExpectation:180}{
We can write the expectation values for the \( L_z \) powers immediately
%
\begin{dmath}\label{eqn:angularMomentumExpectation:20}
\expectation{L_z}
= m \Hbar,
\end{dmath}
%
and
%
\begin{dmath}\label{eqn:angularMomentumExpectation:40}
\expectation{L_z^2} = (m \Hbar)^2.
\end{dmath}
%
For the x and y components first express the operators in terms of the ladder operators.
%
\begin{equation}\label{eqn:angularMomentumExpectation:60}
\begin{aligned}
L_{+} &= L_x + i L_y \\
L_{-} &= L_x - i L_y.
\end{aligned}
\end{equation}
%
Rearranging gives
%
\begin{equation}\label{eqn:angularMomentumExpectation:80}
\begin{aligned}
L_x &= \inv{2} \lr{ L_{+} + L_{-} } \\
L_y &= \inv{2i} \lr{ L_{+} - L_{-} }.
\end{aligned}
\end{equation}
%
The first order expectations \( \expectation{L_x}, \expectation{L_y} \) are both zero since \( \expectation{L_{+}} = \expectation{L_{-}} \).  For the second order expectation values we have
%
\begin{dmath}\label{eqn:angularMomentumExpectation:100}
L_x^2
= \inv{4} \lr{ L_{+} + L_{-} } \lr{ L_{+} + L_{-} }
= \inv{4} \lr{ L_{+} L_{+} + L_{-} L_{-} + L_{+} L_{-} + L_{-} L_{+} }
= \inv{4} \lr{ L_{+} L_{+} + L_{-} L_{-} + 2 (L_x^2 + L_y^2) }
= \inv{4} \lr{ L_{+} L_{+} + L_{-} L_{-} + 2 (\BL^2 - L_z^2) },
\end{dmath}
%
and
\begin{dmath}\label{eqn:angularMomentumExpectation:120}
L_y^2
= -\inv{4} \lr{ L_{+} - L_{-} } \lr{ L_{+} - L_{-} }
= -\inv{4} \lr{ L_{+} L_{+} + L_{-} L_{-} - L_{+} L_{-} - L_{-} L_{+} }
= -\inv{4} \lr{ L_{+} L_{+} + L_{-} L_{-} - 2 (L_x^2 + L_y^2) }
= -\inv{4} \lr{ L_{+} L_{+} + L_{-} L_{-} - 2 (\BL^2 - L_z^2) }.
\end{dmath}
%
Any expectation value \( \bra{lm} L_{+} L_{+} \ket{lm} \) or \( \bra{lm} L_{-} L_{-} \ket{lm} \) will be zero, leaving
%
\begin{dmath}\label{eqn:angularMomentumExpectation:140}
\expectation{L_x^2}
=
\expectation{L_y^2}
=
\inv{4} \expectation{2 (\BL^2 - L_z^2) }
=
\inv{2} \lr{ \Hbar^2 l(l+1) - (\Hbar m)^2 }.
\end{dmath}
%
Observe that we have
\begin{equation}\label{eqn:angularMomentumExpectation:160}
\expectation{L_x^2}
+
\expectation{L_y^2}
+
\expectation{L_z^2}
=
\Hbar^2 l(l+1)
=
\expectation{\BL^2},
\end{equation}
%
which is the quantum mechanical analogue of the classical scalar equation \( \BL^2 = L_x^2 + L_y^2 + L_z^2 \).
} % answer

%\EndArticle

         % p3.33
         %
% Copyright � 2015 Peeter Joot.  All Rights Reserved.
% Licenced as described in the file LICENSE under the root directory of this GIT repository.
%
%{
%\input{../blogpost.tex}
%\renewcommand{\basename}{spinThreeHalvesNucleus}
%\renewcommand{\dirname}{notes/phy1520/}
%%\newcommand{\dateintitle}{}
%%\newcommand{\keywords}{}
%
%\input{../peeter_prologue_print2.tex}
%
%\usepackage{peeters_layout_exercise}
%\usepackage{peeters_braket}
%\usepackage{peeters_figures}
%
%\beginArtNoToc
%
%\generatetitle{Spin three halves spin interaction}
%%\chapter{Spin three halves spin interaction}
%%\label{chap:spinThreeHalvesNucleus}
%

\makeoproblem{Spin three halves spin interaction.}{problem:spinThreeHalvesNucleus:1}{\citep{sakurai2014modern} pr. 3.33}{
\index{spin three halves}
\index{applied electric field}

A spin \( 3/2 \) nucleus subjected to an external electric field has an interaction Hamiltonian of the form
%
\begin{dmath}\label{eqn:spinThreeHalvesNucleus:20}
H = \frac{e Q}{2 s(s-1) \Hbar^2} \lr{
\lr{\PDSq{x}{\phi}}_0 S_x^2
+\lr{\PDSq{y}{\phi}}_0 S_y^2
+\lr{\PDSq{z}{\phi}}_0 S_z^2
}.
\end{dmath}

\makesubproblem{}{problem:spinThreeHalvesNucleus:1:a}

Show that the interaction energy can be written as
%
\begin{dmath}\label{eqn:spinThreeHalvesNucleus:40}
A(3 S_z^2 - \BS^2) + B(S_{+}^2 + S_{-}^2).
\end{dmath}

\makesubproblem{}{problem:spinThreeHalvesNucleus:1:b}

Find the energy eigenvalues for such a Hamiltonian.

} % problem

\makeanswer{problem:spinThreeHalvesNucleus:1}{

\makeSubAnswer{}{problem:spinThreeHalvesNucleus:1:a}

Reordering
\begin{equation}\label{eqn:spinThreeHalvesNucleus:60}
\begin{aligned}
S_{+} &= S_x + i S_y \\
S_{-} &= S_x - i S_y,
\end{aligned}
\end{equation}

gives
\begin{equation}\label{eqn:spinThreeHalvesNucleus:80}
\begin{aligned}
S_x &= \inv{2} \lr{ S_{+} + S_{-} } \\
S_y &= \inv{2i} \lr{ S_{+} - S_{-} }.
\end{aligned}
\end{equation}

The squared spin operators are
\begin{dmath}\label{eqn:spinThreeHalvesNucleus:100}
S_x^2
=
\inv{4} \lr{ S_{+}^2 + S_{-}^2 + S_{+} S_{-} + S_{-} S_{+} }
=
\inv{4} \lr{ S_{+}^2 + S_{-}^2 + 2( S_x^2 + S_y^2 ) }
=
\inv{4} \lr{ S_{+}^2 + S_{-}^2 + 2( \BS^2 - S_z^2 ) },
\end{dmath}
%
\begin{dmath}\label{eqn:spinThreeHalvesNucleus:120}
S_y^2
=
-\inv{4} \lr{ S_{+}^2 + S_{-}^2 - S_{+} S_{-} - S_{-} S_{+} }
=
-\inv{4} \lr{ S_{+}^2 + S_{-}^2 - 2( S_x^2 + S_y^2 ) }
=
-\inv{4} \lr{ S_{+}^2 + S_{-}^2 - 2( \BS^2 - S_z^2 ) }.
\end{dmath}

This gives
\begin{dmath}\label{eqn:spinThreeHalvesNucleus:140}
\begin{aligned}
H
=
\frac{e Q}{2 s(s-1) \Hbar^2}
\biglr{ &
\inv{4} \lr{\PDSq{x}{\phi}}_0 \lr{ S_{+}^2 + S_{-}^2 + 2( \BS^2 - S_z^2 ) } \\
&
-\lr{\PDSq{y}{\phi}}_0 \lr{ S_{+}^2 + S_{-}^2 - 2( \BS^2 - S_z^2 ) } \\
&+\lr{\PDSq{z}{\phi}}_0 S_z^2
} \\
=
\frac{e Q}{2 s(s-1) \Hbar^2}
\biglr{ &
\inv{4} \lr{  \lr{\PDSq{x}{\phi}}_0 -\lr{\PDSq{y}{\phi}}_0 } \lr{ S_{+}^2 + S_{-}^2 } \\
&+
\inv{2} \lr{
\lr{\PDSq{x}{\phi}}_0 + \lr{\PDSq{y}{\phi}}_0
} \BS^2  \\
&+
\lr{
\lr{\PDSq{z}{\phi}}_0
-
\inv{2} \lr{\PDSq{x}{\phi}}_0 - \inv{2} \lr{\PDSq{y}{\phi}}_0
} S_z^2
}.
\end{aligned}
\end{dmath}

For a static electric field we have
%
\begin{dmath}\label{eqn:spinThreeHalvesNucleus:160}
\spacegrad^2 \phi = -\frac{\rho}{\epsilon_0},
\end{dmath}

but are evaluating it at a point away from the generating charge distribution, so \( \spacegrad^2 \phi = 0 \) at that point.  This gives
%
\begin{dmath}\label{eqn:spinThreeHalvesNucleus:180}
\begin{aligned}
H
=
\frac{e Q}{4 s(s-1) \Hbar^2}
\biglr{ &
\inv{2} \lr{  \lr{\PDSq{x}{\phi}}_0 -\lr{\PDSq{y}{\phi}}_0
} \lr{ S_{+}^2 + S_{-}^2 } \\
&+
\lr{
\lr{\PDSq{x}{\phi}}_0 + \lr{\PDSq{y}{\phi}}_0
} (\BS^2 - 3 S_z^2)
},
\end{aligned}
\end{dmath}

so
\begin{dmath}\label{eqn:spinThreeHalvesNucleus:200}
A =
-\frac{e Q}{4 s(s-1) \Hbar^2}  \lr{
\lr{\PDSq{x}{\phi}}_0 + \lr{\PDSq{y}{\phi}}_0
}
\end{dmath}
\begin{dmath}\label{eqn:spinThreeHalvesNucleus:220}
B =
\frac{e Q}{8 s(s-1) \Hbar^2}
\lr{ \lr{\PDSq{x}{\phi}}_0 - \lr{\PDSq{y}{\phi}}_0 }.
\end{dmath}

\makeSubAnswer{}{problem:spinThreeHalvesNucleus:1:b}

Using \nbref{sakuraiProblem3.33.nb}, matrix representations for the spin three halves operators and the Hamiltonian were constructed with respect to the basis \( \setlr{ \ket{3/2}, \ket{1/2}, \ket{-1/2}, \ket{-3/2} } \)
%
\begin{equation}\label{eqn:spinThreeHalvesNucleus:240}
\begin{aligned}
S_{+} &=
\Hbar
\begin{bmatrix}
 0 & \sqrt{3} & 0 & 0 \\
 0 & 0 & 2 & 0 \\
 0 & 0 & 0 & \sqrt{3} \\
 0 & 0 & 0 & 0 \\
\end{bmatrix} \\
S_{-} &=
\Hbar
\begin{bmatrix}
 0 & 0 & 0 & 0 \\
 \sqrt{3} & 0 & 0 & 0 \\
 0 & 2 & 0 & 0 \\
 0 & 0 & \sqrt{3}  & 0 \\
\end{bmatrix} \\
S_x &=
\Hbar
\begin{bmatrix}
 0 & \sqrt{3}/2 & 0 & 0 \\
 \sqrt{3}/2 & 0 & 1 & 0 \\
 0 & 1  & 0 & \sqrt{3}/2 \\
 0 & 0 & \sqrt{3}/2 & 0 \\
\end{bmatrix} \\
S_y &=
i \Hbar
\begin{bmatrix}
 0 & -\ifrac{\sqrt{3}}{2} & 0 & 0 \\
 \ifrac{\sqrt{3}}{2} & 0 & -1 & 0 \\
 0 & 1 & 0 & -\ifrac{\sqrt{3}}{2} \\
 0 & 0 & \ifrac{\sqrt{3}}{2} & 0 \\
\end{bmatrix} \\
S_z &=
\frac{\Hbar}{2}
\begin{bmatrix}
 3 & 0 & 0 & 0 \\
 0 & 1 & 0 & 0 \\
 0 & 0 & -1 & 0 \\
 0 & 0 & 0 & -3 \\
\end{bmatrix} \\
H &=
\begin{bmatrix}
 3 A & 0 & 2 \sqrt{3} B & 0 \\
 0 & -3 A & 0 & 2 \sqrt{3} B \\
 2 \sqrt{3} B & 0 & -3 A & 0 \\
 0 & 2 \sqrt{3} B & 0 & 3 A \\
\end{bmatrix}.
\end{aligned}
\end{equation}

The energy eigenvalues are found to be
%
\begin{equation}\label{eqn:spinThreeHalvesNucleus:260}
E = \pm \Hbar^2 \sqrt{9 A^2 + 12 B^2 },
\end{equation}

with two fold degeneracies for each eigenvalue.
} % answer

%}
%\EndArticle

   \mychapter{Approximation methods.}
      %
% Copyright � 2015 Peeter Joot.  All Rights Reserved.
% Licenced as described in the file LICENSE under the root directory of this GIT repository.
%
%\input{../blogpost.tex}
%\renewcommand{\basename}{qmLecture18}
%\renewcommand{\dirname}{notes/phy1520/}
%\newcommand{\keywords}{PHY1520H}
%\input{../peeter_prologue_print2.tex}
%
%%\usepackage{phy1520}
%\usepackage{peeters_braket}
%%\usepackage{peeters_layout_exercise}
%\usepackage{peeters_figures}
%\usepackage{mathtools}
%
%\beginArtNoToc
%\generatetitle{PHY1520H Graduate Quantum Mechanics.  Lecture 18: Approximation methods.  Taught by Prof.\ Arun Paramekanti}
%%\chapter{Approximation methods}
%\label{chap:qmLecture18}
%
%\paragraph{Disclaimer}
%
%Peeter's lecture notes from class.  These may be incoherent and rough, especially since I didn't attend this class myself, and am doing a walkthrough of notes provided by Nishant.
%
%These are notes for the UofT course PHY1520, Graduate Quantum Mechanics, taught by Prof. Paramekanti, covering \textchapref{{5}} \citep{sakurai2014modern} content.
%
\section{Approximation methods.}
\index{approximation methods}
\index{perturbation}

Suppose we have a perturbed Hamiltonian
%
\begin{equation}\label{eqn:qmLecture18:20}
H = H_0 + \lambda V,
\end{equation}
%
where \( \lambda = 0 \) represents a solvable (perhaps known) system, and \( \lambda = 1 \) is the case of interest.  There are two approaches of interest

\begin{enumerate}
\item Direct solution of \( H \) with \( \lambda = 1 \).
\item Take \( \lambda \) small, and do a series expansion.  This is perturbation theory.
\index{perturbation}
\end{enumerate}

\section{Variational methods.}
\index{variational method}

Given
%
\begin{equation}\label{eqn:qmLecture18:40}
H \ket{\phi_n} = E_n \ket{\phi_n},
\end{equation}
%
where we don't know \( \ket{\phi_n} \), we can compute the expectation with respect to an arbitrary state \( \ket{\psi} \)
%
\begin{dmath}\label{eqn:qmLecture18:60}
\bra{\psi} H \ket{\psi}
=
\bra{\psi} H \lr{ \sum_n \ket{\phi_n} \bra{\phi_n} } \ket{\psi}
=
\sum_n E_n \braket{\psi}{\phi_n} \braket{\phi_n}{\psi}
=
\sum_n E_n \Abs{\braket{\psi}{\phi_n}}^2.
\end{dmath}
%
Define
%
\begin{dmath}\label{eqn:qmLecture18:80}
\overbar{E}
= \frac{\bra{\psi} H \ket{\psi}}{\braket{\psi}{\psi}}.
\end{dmath}
%
Assuming that it is possible to express the state in the Hamiltonian energy basis
%
\begin{dmath}\label{eqn:qmLecture18:100}
\ket{\psi}
=
\sum_n a_n \ket{\phi_n},
\end{dmath}
%
this average energy is
\begin{dmath}\label{eqn:qmLecture18:120}
\overbar{E}
= \frac{ \sum_{m,n}\bra{\phi_m} a_m^\conj H a_n \ket{\phi_n}}{ \sum_n \Abs{a_n}^2 }
= \frac{ \sum_{n} \Abs{a_n}^2 E_n }{ \sum_n \Abs{a_n}^2 }.
= \sum_{n}
\frac{\Abs{a_n}^2 }{ \sum_n \Abs{a_n}^2 }
E_n
= \sum_n \frac{P_n}{\sum_m P_m} E_n,
\end{dmath}
%
where \( P_m = \Abs{a_m}^2 \), which has the structure of a probability coefficient once divided by \( \sum_m P_m \),
%FIXME: Nishant's notes showed these probabilities ordered so that \( P_1 \ge P_2 \ge P_3 \cdots \),
as sketched in \cref{fig:lecture18:lecture18Fig1}.
\imageFigure{../figures/phy1520-quantum/lecture18Fig1}{A decreasing probability distribution.}{fig:lecture18:lecture18Fig1}{0.1}
This average energy is a probability weighted average of the individual energy basis states.   One of those energies is the ground state energy \( E_1 \), so we necessarily have
%
%\begin{dmath}\label{eqn:qmLecture18:140}
\boxedEquation{eqn:qmLecture18:160}{
\overbar{E} \ge E_1.
}
%\end{dmath}
%
%Why would such an ordering necessarily exist if \( \ket{\psi} \) is an arbitrary state?  Could that state not be closer to the third energy eigenstate (say), than to the ground state?
%
%We also have
%
%\begin{dmath}\label{eqn:qmLecture18:180}
%\overbar{E} - E_1
%= \sum_n \frac{P_n}{\sum_m P_m} E_n - \sum_n \frac{P_n}{\sum_m P_m} E_1
%= \sum_n \frac{P_n}{\sum_m P_m} \lr{ E_n - E_1 },
%\end{dmath}
%
%where \( E_n - E_1 \ge 0 \).  This means that we also have
%
%\begin{equation}\label{eqn:qmLecture18:200}
%\overbar{E} - E_1 \ge 0.
%\end{equation}
%
%FIXME: not sure what this last point was about?  How is \( \overbar{E} \ge E_1 \) different than \( \overbar{E} - E_1 \ge 0 \)?
%

\makeexample{Particle in an offset box.}{example:qmLecture18:1}{

For the infinite potential box sketched in \cref{fig:lecture18:lecture18Fig2}.

\imageFigure{../figures/phy1520-quantum/lecture18Fig2}{Infinite potential \( [0,L] \) box.}{fig:lecture18:lecture18Fig2}{0.2}

The exact solutions for such a system are found to be
%
\begin{equation}\label{eqn:qmLecture18:220}
\psi(x) = \sqrt{\frac{2}{L}} \sin\lr{ \frac{n \pi}{L} x },
\end{equation}
%
where the energies are
%
\begin{equation}\label{eqn:qmLecture18:240}
E = \frac{\Hbar^2}{2m} \frac{n^2 \pi^2}{L^2}.
\end{equation}
%
The function \( \psi' = x (L-x) \) also satisfies the boundary value constraints?  How close in energy is that function to the ground state?
%
\begin{dmath}\label{eqn:qmLecture18:260}
\overbar{E}
=
-\frac{\Hbar^2}{2m} \frac{\int_0^L dx x (L-x) \frac{d^2}{dx^2} \lr{ x (L-x) }}{
\int_0^L dx x^2 (L-x)^2
}
=
\frac{\Hbar^2}{2m} \frac{\frac{2 L^3}{6}}{
\frac{L^5}{30}
}
=
\frac{\Hbar^2}{2m} \frac{10}{L^2}.
\end{dmath}
%
This average energy is quite close to the ground state energy
%
\begin{equation}\label{eqn:qmLecture18:280}
\frac{\overbar{E} }{E_1} = \frac{10}{\pi^2} = 1.014.
\end{equation}
%
} % example

\makeexample{Particle in a symmetric box.}{example:qmLecture18:2}{
\imageFigure{../figures/phy1520-quantum/lecture18Fig3}{Infinite potential \( [-L/2,L/2] \) box.}{fig:qmLecture18:lecture18Fig3}{0.15}
Shifting the boundaries, as sketched in \cref{fig:qmLecture18:lecture18Fig3} doesn't change the energy levels.  For this potential let's try a shifted trial function
%
\begin{equation}\label{eqn:qmLecture18:300}
\psi(x) = \lr{ x - \frac{L}{2} } \lr{ x + \frac{L}{2} } = x^2 - \frac{L^2}{4},
\end{equation}
%
without worrying about the form of the exact solution.  This produces the same result as above
%
\begin{dmath}\label{eqn:qmLecture18:270}
\overbar{E}
=
-\frac{\Hbar^2}{2m} \frac{\int_0^L dx \lr{ x^2 - \frac{L^2}{4} } \frac{d^2}{dx^2} \lr{ x^2 - \frac{L^2}{4} }}{
\int_0^L dx \lr{x^2 - \frac{L^2}{4} }^2
}
=
-\frac{\Hbar^2}{2m} \frac{- 2 L^3/6}{
\frac{L^5}{30}
}
=
\frac{\Hbar^2}{2m} \frac{10}{L^2}.
\end{dmath}
%
} % example

\paragraph{Summary (Nishant)}

The above example is that of a particle in a box. The actual wave function is a sin as shown. But we can
come up with a guess wave function that meets the boundary conditions and ask how accurate it is
compared to the actual one.

Basically we are assuming a wave function form and then seeing how it differs from the exact form.
We cannot do this if we have nothing to compare it against. But, we note that the variance of the
number operator in the systems eigenstate is zero. So we can still calculate the variance and try to
minimize it. This is one way of coming up with an approximate wave function. This does not necessarily
give the ground state wave function though. For this we need to minimize the energy itself.

%\EndArticle

      %
% Copyright � 2015 Peeter Joot.  All Rights Reserved.
% Licenced as described in the file LICENSE under the root directory of this GIT repository.
%
%\input{../blogpost.tex}
%\renewcommand{\basename}{qmLecture19}
%\renewcommand{\dirname}{notes/phy1520/}
%\newcommand{\keywords}{PHY1520H}
%\input{../peeter_prologue_print2.tex}
%
%%\usepackage{phy1520}
%\usepackage{peeters_braket}
%%\usepackage{peeters_layout_exercise}
%\usepackage{peeters_figures}
%\usepackage{mathtools}
%
%\beginArtNoToc
%\generatetitle{PHY1520H Graduate Quantum Mechanics.  Lecture 19: Variational method.  Taught by Prof.\ Arun Paramekanti}
%%\chapter{Variational method}
%\label{chap:qmLecture19}
%
%\paragraph{Disclaimer}
%
%Peeter's lecture notes from class.  These may be incoherent and rough.
%
%These are notes for the UofT course PHY1520, Graduate Quantum Mechanics, taught by Prof. Paramekanti, covering \textchapref{{5}} \citep{sakurai2014modern} content.
%
\section{Variational method.}
\index{variational method}

Today we want to use the variational degree of freedom to try to solve some problems that we don't have analytic solutions for.

\paragraph{Anharmonic oscillator}
\index{anharmonic oscillator}
%
\begin{equation}\label{eqn:qmLecture19:20}
V(x) = \inv{2} m \omega^2 x^2 + \lambda x^4, \qquad \lambda \ge 0.
\end{equation}
%
With the potential growing faster than the harmonic oscillator, which had a ground state solution
%
\begin{equation}\label{eqn:qmLecture19:40}
\psi(x) = \inv{\pi^{1/4}} \inv{a_0^{1/2} } e^{- x^2/2 a_0^2},
\end{equation}
%
where
\begin{equation}\label{eqn:qmLecture19:60}
a_0 = \sqrt{\frac{\Hbar}{m \omega}}.
\end{equation}
%
Let's try allowing \( a_0 \rightarrow a \), to be a variational degree of freedom
%
\begin{equation}\label{eqn:qmLecture19:80}
\psi_a(x) = \inv{\pi^{1/4}} \inv{a^{1/2} } e^{- x^2/2 a^2},
\end{equation}
%
\begin{dmath}\label{eqn:qmLecture19:100}
\bra{\psi_a} H \ket{\psi_a}
=
\bra{\psi_a} \frac{p^2}{2m} + \inv{2} m \omega^2 x^2 + \lambda x^4 \ket{\psi_a}.
\end{dmath}
We can find
\begin{equation}\label{eqn:qmLecture19:120}
\expectation{x^2} = \inv{2} a^2,
\end{equation}
\begin{equation}\label{eqn:qmLecture19:140}
\expectation{x^4} = \frac{3}{4} a^4.
\end{equation}
Define
%
\begin{equation}\label{eqn:qmLecture19:160}
\tilde{\omega} = \frac{\Hbar}{m a^2},
\end{equation}
%
so that
%
\begin{dmath}\label{eqn:qmLecture19:180}
\overbar{E}_a
=
\bra{\psi_a} \lr{ \frac{p^2}{2m} + \inv{2} m \tilde{\omega}^2 x^2 }
+ \lr{
\inv{2} m \lr{ \omega^2 - \tilde{\omega}^2 } x^2
+
\lambda x^4 }
\ket{\psi_a}
=
\inv{2} \Hbar \tilde{\omega} + \inv{2} m  \lr{ \omega^2 - \tilde{\omega}^2 } \inv{2} a^2 + \frac{3}{4} \lambda a^4.
\end{dmath}
%
Write this as
\begin{dmath}\label{eqn:qmLecture19:200}
\overbar{E}_{\tilde{\omega}}
=
\inv{2} \Hbar \tilde{\omega} + \inv{4} \frac{\Hbar}{\tilde{\omega}} \lr{ \omega^2 - \tilde{\omega}^2 } + \frac{3}{4} \lambda \frac{\Hbar^2}{m^2 \tilde{\omega}^2 }.
\end{dmath}
%
This might look something like \cref{fig:lecture19:lecture19Fig1}.
\imageFigure{../figures/phy1520-quantum/lecture19Fig1a}{Energy after perturbation.}{fig:lecture19:lecture19Fig1}{0.2}
Demand that
%
\begin{dmath}\label{eqn:qmLecture19:220}
0
= \PD{\tilde{\omega}}{ \overbar{E}_{\tilde{\omega}}}
=
\frac{\Hbar}{2} - \frac{\Hbar}{4} \frac{\omega^2}{\tilde{\omega}^2}
- \frac{\Hbar}{4}
+ \frac{3}{4} (-2) \frac{\lambda \Hbar^2}{m^2 \tilde{\omega}^3}
=
\frac{\Hbar}{4}
\lr{
1 - \frac{\omega^2}{\tilde{\omega}^2}
- 6 \frac{\lambda \Hbar}{m^2 \tilde{\omega}^3}
},
\end{dmath}
or
\begin{equation}\label{eqn:qmLecture19:260}
\tilde{\omega}^3 - \omega^2 \tilde{\omega} - \frac{6 \lambda \Hbar}{m^2} = 0.
\end{equation}
%
for \( \lambda a_0^4 \ll \Hbar \omega \), we have something like \( \tilde{\omega} = \omega + \epsilon \).  Expanding \cref{eqn:qmLecture19:260} to first order in \( \epsilon \), this gives
%
\begin{equation}\label{eqn:qmLecture19:280}
\omega^3 + 3 \omega^2 \epsilon - \omega^2 \lr{ \omega + \epsilon } - \frac{6 \lambda \Hbar}{m^2} = 0,
\end{equation}
%
so that
%
\begin{equation}\label{eqn:qmLecture19:300}
2 \omega^2 \epsilon = \frac{6 \lambda \Hbar}{m^2},
\end{equation}
%
and
%
\begin{equation}\label{eqn:qmLecture19:320}
\Hbar \epsilon = \frac{ 3 \lambda \Hbar^2}{m^2 \omega^2 } = 3 \lambda a_0^4.
\end{equation}
%
Plugging into
%
\begin{dmath}\label{eqn:qmLecture19:340}
\overbar{E}_{\omega + \epsilon}
=
\inv{2} \Hbar \lr{ \omega + \epsilon }
+ \inv{4} \frac{\Hbar}{\omega} \lr{ -2 \omega \epsilon + \epsilon^2 } + \frac{3}{4} \lambda \frac{\Hbar^2}{m^2 \omega^2 }
\approx
\inv{2} \Hbar \lr{ \omega + \epsilon }
- \inv{2} \Hbar \epsilon
+ \frac{3}{4} \lambda \frac{\Hbar^2}{m^2 \omega^2 }
=
\inv{2} \Hbar \omega + \frac{3}{4} \lambda a_0^4.
\end{dmath}
%
With \cref{eqn:qmLecture19:320}, that is
%
\begin{equation}\label{eqn:qmLecture19:540}
\overbar{E}_{\tilde{\omega} = \omega + \epsilon} \approx \inv{2} \Hbar \lr{ \omega + \frac{\epsilon}{2} }.
\end{equation}
%
The energy levels are shifted slightly for each shift in the Hamiltonian frequency.

What do we have in the extreme anharmonic limit, where \( \lambda a_0^4 \gg \Hbar \omega \)?  Now we get
%
\begin{equation}\label{eqn:qmLecture19:360}
\tilde{\omega}^\conj = \lr{ \frac{ 6 \Hbar \lambda }{m^2} }^{1/3},
\end{equation}
%
and
\begin{equation}\label{eqn:qmLecture19:380}
\overbar{E}_{\tilde{\omega}^\conj} = \frac{\Hbar^{4/3} \lambda^{1/3}}{m^{2/3}} \frac{3}{8} 6^{1/3}.
\end{equation}
%
(this last result is pulled from a web treatment somewhere of the anharmonic oscillator).  Note that the first factor in this energy, with \( \Hbar^4 \lambda/m^2 \) travelling together could have been worked out on dimensional grounds.

This variational method tends to work quite well in these limits.  For a system where \( m = \omega = \Hbar = 1 \), for this problem, we have

\captionedTable{Comparing numeric and variational solutions}{tab:1}{
\begin{tabular}{|l|l|l|}
\hline
\(\Hbar/\omega\) & numeric & variational \\ \hline
100 & 3.13 & 3.16 \\ \hline
1000 & 6.69  & 6.81 \\ \hline
\end{tabular}
}

\paragraph{Example: (sketch) double well potential}
\index{double well potential}

%\cref{fig:lecture19:lecture19Fig2}.
\imageFigure{../figures/phy1520-quantum/lecture19Fig2}{Double well potential.}{fig:lecture19:lecture19Fig2}{0.15}
%
\begin{equation}\label{eqn:qmLecture19:400}
V(x) = \frac{m \omega^2}{8 a^2} \lr{ x - a }^2\lr{ x + a}^2.
\end{equation}
%
Note that this potential, and the Hamiltonian, both commute with parity.
We are interested in the regime where \( a_0^2 = \frac{\Hbar}{m \omega} \ll a^2 \).
Near \( x = \pm a \), this will be approximately
%
\begin{equation}\label{eqn:qmLecture19:420}
V(x) = \inv{2} m \omega^2 \lr{ x \pm a }^2.
\end{equation}
%
Guessing a wave function that is an eigenstate of parity
%
\begin{equation}\label{eqn:qmLecture19:440}
\Psi_{\pm} = g_{\pm} \lr{ \phi_\txtR(x) \pm \phi_\txtL(x) }.
\end{equation}
%
Perhaps this looks like the even and odd functions sketched in \cref{fig:lecture19:lecture19Fig3}, and \cref{fig:lecture19:lecture19Fig4}.
\imageFigure{../figures/phy1520-quantum/lecture19Fig3}{Even double well function.}{fig:lecture19:lecture19Fig3}{0.2}
\imageFigure{../figures/phy1520-quantum/lecture19Fig4}{Odd double well function.}{fig:lecture19:lecture19Fig4}{0.1}
Using harmonic oscillator functions
%
\begin{equation}\label{eqn:qmLecture19:460}
\begin{aligned}
\phi_\txtL(x) &= \Psi_{\txtH.\txtO.}(x + a), \\
\phi_\txtR(x) &= \Psi_{\txtH.\txtO.}(x - a).
\end{aligned}
\end{equation}
After doing a lot of integral (i.e. in the problem set), we will see a splitting of the variational energy levels as sketched in \cref{fig:lecture19:lecture19Fig5}.
\imageFigure{../figures/phy1520-quantum/lecture19Fig5}{Splitting for double well potential.}{fig:lecture19:lecture19Fig5}{0.1}
This sort of level splitting was what was used in the very first mazers.
\section{Perturbation theory (outline).}
\index{perturbation}
Given
%
\begin{equation}\label{eqn:qmLecture19:480}
H = H_0 + \lambda V,
\end{equation}
%
where \( \lambda V \) is ``small''.  We want to figure out the eigenvalues and eigenstates of this Hamiltonian
%
\begin{equation}\label{eqn:qmLecture19:500}
H \ket{n} = E_n \ket{n}.
\end{equation}
%
We don't know what these are, but do know that
%
\begin{equation}\label{eqn:qmLecture19:520}
H_0 \ket{n^{(0)}} = E_n^{(0)} \ket{n^{(0)}}.
\end{equation}
%
We are hoping that the level transitions have adiabatic transitions between the original and perturbed levels as sketched in \cref{fig:lecture19:lecture19Fig6}.
\imageFigure{../figures/phy1520-quantum/lecture19Fig6}{Adiabatic transitions.}{fig:lecture19:lecture19Fig6}{0.1}
and not crossed level transitions as sketched in \cref{fig:lecture19:lecture19Fig7}.
\imageFigure{../figures/phy1520-quantum/lecture19Fig7}{Crossed level transitions.}{fig:lecture19:lecture19Fig7}{0.1}
If we have level crossings (which can in general occur), as opposed to adiabatic transitions, then we have no hope of using perturbation theory.
%\EndArticle

      %
% Copyright � 2015 Peeter Joot.  All Rights Reserved.
% Licenced as described in the file LICENSE under the root directory of this GIT repository.
%
%\input{../blogpost.tex}
%\renewcommand{\basename}{qmLecture20}
%\renewcommand{\dirname}{notes/phy1520/}
%\newcommand{\keywords}{PHY1520H}
%\input{../peeter_prologue_print2.tex}
%
%%\usepackage{phy1520}
%\usepackage{peeters_braket}
%%\usepackage{peeters_layout_exercise}
%\usepackage{peeters_figures}
%\usepackage{mathtools}
%
%\beginArtNoToc
%\generatetitle{PHY1520H Graduate Quantum Mechanics.  Lecture 20: Perturbation theory.  Taught by Prof.\ Arun Paramekanti}
%%\chapter{Pertubation theory}
%\label{chap:qmLecture20}
%
%\paragraph{Disclaimer}
%
%Peeter's lecture notes from class.  These may be incoherent and rough.
%
%These are notes for the UofT course PHY1520, Graduate Quantum Mechanics, taught by Prof. Paramekanti, covering \textchapref{{5}} \citep{sakurai2014modern} content.

\section{Simplest perturbation example.}
\index{perturbation!simplest}

Given a \( 2 \times 2 \) Hamiltonian \( H = H_0 + V \), where

\begin{dmath}\label{eqn:qmLecture20:20}
H =
\begin{bmatrix}
a & c \\
c^\conj & b
\end{bmatrix},
\end{dmath}

note that if \( c = 0 \) is

\begin{equation}\label{eqn:qmLecture20:60}
H = H_0 =
\begin{bmatrix}
a & 0 \\
0 & b
\end{bmatrix}.
\end{equation}

The off diagonal terms can be considered to be a perturbation

\begin{dmath}\label{eqn:qmLecture20:80}
V =
\begin{bmatrix}
0 & c \\
c^\conj & 0
\end{bmatrix},
\end{dmath}

with \( H = H_0 + V \).

\paragraph{Energy levels after perturbation}

We can solve for the eigenvalues of \( H \) easily, finding

\begin{dmath}\label{eqn:qmLecture20:40}
\lambda_\pm = \frac{a + b}{2} \pm \sqrt{ \lr{ \frac{a - b}{2}}^2 + \Abs{c}^2 }.
\end{dmath}

Plots of a few \( a,b \) variations of \( \lambda_{\pm} \) are shown in \cref{fig:lecture20:lecture20Fig3}.  The quadratic (non-degenerate) domain is found near the \( c = 0 \) points of all but the first ( \( a = b \) ) plot, and the degenerate (linear in \( \Abs{c}^2 \)) regions are visible for larger values of \( c \).

\mathImageFourFiguresTwoLines
{../phy1520-quantum-figureslecture20Fig3}
{../phy1520-quantum-figureslecture20Fig4}
{../phy1520-quantum-figureslecture20Fig5}
{../phy1520-quantum-figureslecture20Fig6}
{Plots of \( \lambda_{\pm} \) for \((a,b) \in \setlr{(1,1),(1,0),(1,5),(-8,8)}\)}{fig:lecture20:lecture20Fig3}{scale=0.4}{visualizationOfEigenvaluesOfTwoByTwoHermitianMatrix.nb}

\paragraph{Some approximations}

Suppose that \( \Abs{c} \ll \Abs{a - b} \), then

\begin{dmath}\label{eqn:qmLecture20:100}
\lambda_\pm \approx \frac{a + b}{2} \pm \Abs{ \frac{a - b}{2} } \lr{ 1 + 2 \frac{\Abs{c}^2}{\Abs{a - b}^2} }.
\end{dmath}

If  \( a > b \), then

\begin{dmath}\label{eqn:qmLecture20:120}
\lambda_\pm \approx \frac{a + b}{2} \pm \frac{a - b}{2} \lr{ 1 + 2 \frac{\Abs{c}^2}{\lr{a - b}^2} }.
\end{dmath}

\begin{dmath}\label{eqn:qmLecture20:140}
\lambda_{+}
= \frac{a + b}{2} + \frac{a - b}{2} \lr{ 1 + 2 \frac{\Abs{c}^2}{\lr{a - b}^2} }
= a + \lr{a - b} \frac{\Abs{c}^2}{\lr{a - b}^2}
= a + \frac{\Abs{c}^2}{a - b},
\end{dmath}

and
\begin{dmath}\label{eqn:qmLecture20:680}
\lambda_{-}
= \frac{a + b}{2} - \frac{a - b}{2} \lr{ 1 + 2 \frac{\Abs{c}^2}{\lr{a - b}^2} }
=
b + \lr{a - b} \frac{\Abs{c}^2}{\lr{a - b}^2}
= b + \frac{\Abs{c}^2}{a - b}.
\end{dmath}

This adiabatic evolution displays a ``level repulsion'', quadratic in \( \Abs{c} \),
% as sketched in \cref{fig:lecture20:lecture20Fig1},
and is described as a non-degenerate permutation.

%\imageFigure{../phy1520-quantum-figureslecture20Fig1}{Adiabatic (non-degenerate) pertubation}{fig:lecture20:lecture20Fig1}{0.2}

If \( \Abs{c} \gg \Abs{a -b} \), then

\begin{dmath}\label{eqn:qmLecture20:160}
\lambda_\pm
= \frac{a + b}{2} \pm \Abs{c} \sqrt{ 1 + \inv{\Abs{c}^2} \lr{ \frac{a - b}{2}}^2 }
\approx \frac{a + b}{2} \pm \Abs{c} \lr{ 1 + \inv{2 \Abs{c}^2} \lr{ \frac{a - b}{2}}^2 }
= \frac{a + b}{2} \pm \Abs{c} \pm \frac{\lr{a - b}^2}{8 \Abs{c}}.
\end{dmath}

Here we loose the adiabaticity, and have ``level repulsion'' that is linear in \( \Abs{c} \).
%, as sketched in \cref{fig:lecture20:lecture20Fig2}.
We no longer have the sign of \( a - b \) in the expansion.  This is described as a degenerate permutation.

%\imageFigure{../phy1520-quantum-figureslecture20Fig2}{Degenerate perbutation}{fig:lecture20:lecture20Fig2}{0.2}

\section{General non-degenerate perturbation}
\index{perturbation!non-degenerate}

Given an unperturbed system with solutions of the form

\begin{equation}\label{eqn:qmLecture20:180}
H_0 \ket{n^{(0)}} = E_n^{(0)} \ket{n^{(0)}},
\end{equation}

we want to solve the perturbed Hamiltonian equation

\begin{equation}\label{eqn:qmLecture20:200}
\lr{ H_0 + \lambda V } \ket{ n } = \lr{ E_n^{(0)} + \Delta n } \ket{n}.
\end{equation}

Here \( \Delta n \) is an energy shift as that goes to zero as \( \lambda \rightarrow 0 \).  We can write this as

\begin{equation}\label{eqn:qmLecture20:220}
\lr{ E_n^{(0)} - H_0 } \ket{ n } = \lr{ \lambda V - \Delta_n } \ket{n}.
\end{equation}

We are hoping to iterate with application of the inverse to an initial estimate of \( \ket{n} \)

\begin{equation}\label{eqn:qmLecture20:240}
\ket{n} = \lr{ E_n^{(0)} - H_0 }^{-1} \lr{ \lambda V - \Delta_n } \ket{n}.
\end{equation}

This gets us into trouble if \( \lambda \rightarrow 0 \), which can be fixed by using

\begin{equation}\label{eqn:qmLecture20:260}
\ket{n} = \lr{ E_n^{(0)} - H_0 }^{-1} \lr{ \lambda V - \Delta_n } \ket{n} + \ket{ n^{(0)} },
\end{equation}

which can be seen to be a solution to \cref{eqn:qmLecture20:220}.  We want to ask if

\begin{equation}\label{eqn:qmLecture20:280}
\lr{ \lambda V - \Delta_n } \ket{n} ,
\end{equation}

contains a bit of \( \ket{ n^{(0)} } \)?  To determine this act with \( \bra{n^{(0)}} \) on the left

\begin{dmath}\label{eqn:qmLecture20:300}
\bra{ n^{(0)} } \lr{ \lambda V - \Delta_n } \ket{n}
=
\bra{ n^{(0)} } \lr{ E_n^{(0)} - H_0 } \ket{n}
=
\lr{ E_n^{(0)} - E_n^{(0)} } \braket{n^{(0)}}{n}
=
0.
\end{dmath}

This shows that \( \ket{n} \) is entirely orthogonal to \( \ket{n^{(0)}} \).

Define a projection operator

\begin{dmath}\label{eqn:qmLecture20:320}
P_n = \ket{n^{(0)}}\bra{n^{(0)}},
\end{dmath}

which has the idempotent property \( P_n^2 = P_n \) that we expect of a projection operator.

Define a rejection operator
\begin{dmath}\label{eqn:qmLecture20:340}
\overbar{P}_n
= 1 -
\ket{n^{(0)}}\bra{n^{(0)}}
= \sum_{m \ne n}
\ket{m^{(0)}}\bra{m^{(0)}}.
\end{dmath}

Because \( \ket{n} \) has no component in the direction \( \ket{n^{(0)}} \), the rejection operator can be inserted much like we normally do with the identity operator, yielding

\begin{dmath}\label{eqn:qmLecture20:360}
\ket{n}' = \lr{ E_n^{(0)} - H_0 }^{-1} \overbar{P}_n \lr{ \lambda V - \Delta_n } \ket{n} + \ket{ n^{(0)} },
\end{dmath}

valid for any initial \( \ket{n} \).

\paragraph{Power series perturbation expansion}

Instead of iterating, suppose that the unknown state and unknown energy difference operator can be expanded in a \( \lambda \) power series, say

\begin{dmath}\label{eqn:qmLecture20:380}
\ket{n}
=
\ket{n_0}
+ \lambda \ket{n_1}
+ \lambda^2 \ket{n_2}
+ \lambda^3 \ket{n_3} + \cdots
\end{dmath}

and

\begin{dmath}\label{eqn:qmLecture20:400}
\Delta_{n} = \Delta_{n_0}
+ \lambda \Delta_{n_1}
+ \lambda^2 \Delta_{n_2}
+ \lambda^3 \Delta_{n_3} + \cdots
\end{dmath}

We usually interpret functions of operators in terms of power series expansions.  In the case of \( \lr{ E_n^{(0)} - H_0 }^{-1} \), we have a concrete interpretation when acting on one of the unperturbed eigenstates

\begin{dmath}\label{eqn:qmLecture20:420}
\inv{ E_n^{(0)} - H_0 } \ket{m^{(0)}} =
\inv{ E_n^{(0)} - E_m^0 } \ket{m^{(0)}}.
\end{dmath}

This gives

\begin{dmath}\label{eqn:qmLecture20:440}
\ket{n}
=
\inv{ E_n^{(0)} - H_0 }
\sum_{m \ne n}
\ket{m^{(0)}}\bra{m^{(0)}}
 \lr{ \lambda V - \Delta_n } \ket{n} + \ket{ n^{(0)} },
\end{dmath}

or

%\begin{equation}\label{eqn:qmLecture20:460}
\boxedEquation{eqn:qmLecture20:480}{
\ket{n}
=
 \ket{ n^{(0)} }
+
\sum_{m \ne n}
\frac{\ket{m^{(0)}}\bra{m^{(0)}}}
{
E_n^{(0)} - E_m^{(0)}
}
 \lr{ \lambda V - \Delta_n } \ket{n}.
}
%\end{equation}

From \cref{eqn:qmLecture20:220}, note that

\begin{equation}\label{eqn:qmLecture20:500}
\Delta_n =
\frac{\bra{n^{(0)}} \lambda V \ket{n}}{\braket{n^0}{n}},
\end{equation}

however, we will normalize by setting \( \braket{n^0}{n} = 1 \), so

%\begin{equation}\label{eqn:qmLecture20:521}
\boxedEquation{eqn:qmLecture20:521}{
\Delta_n =
\bra{n^{(0)}} \lambda V \ket{n}.
}
%\end{equation}

\paragraph{to \( O(\lambda^0) \) }

If all \( \lambda^n, n > 0 \) are zero, then we have

\begin{subequations}
\label{eqn:qmLecture20:780}
\begin{dmath}\label{eqn:qmLecture20:740}
\ket{n_0}
=
 \ket{ n^{(0)} }
+
\sum_{m \ne n}
\frac{\ket{m^{(0)}}\bra{m^{(0)}}}
{
E_n^{(0)} - E_m^{(0)}
}
 \lr{ - \Delta_{n_0} } \ket{n_0}
\end{dmath}
\begin{dmath}\label{eqn:qmLecture20:800}
\Delta_{n_0} \braket{n^{(0)}}{n_0} = 0
\end{dmath}
\end{subequations}

so

\begin{dmath}\label{eqn:qmLecture20:540}
\begin{aligned}
\ket{n_0} &= \ket{n^{(0)}} \\
\Delta_{n_0} &= 0.
\end{aligned}
\end{dmath}

\paragraph{to \( O(\lambda^1) \) }

Requiring identity for all \( \lambda^1 \) terms means

\begin{dmath}\label{eqn:qmLecture20:760}
\ket{n_1} \lambda
=
\sum_{m \ne n}
\frac{\ket{m^{(0)}}\bra{m^{(0)}}}
{
E_n^{(0)} - E_m^{(0)}
}
 \lr{ \lambda V - \Delta_{n_1} \lambda } \ket{n_0},
\end{dmath}

so

\begin{dmath}\label{eqn:qmLecture20:560}
\ket{n_1}
=
\sum_{m \ne n}
\frac{
\ket{m^{(0)}} \bra{ m^{(0)}}
}
{
E_n^{(0)} - E_m^{(0)}
}
\lr{ V - \Delta_{n_1} } \ket{n_0}.
\end{dmath}

With the assumption that \( \ket{n^{(0)}} \) is normalized, and with the shorthand

\begin{dmath}\label{eqn:qmLecture20:600}
V_{m n} = \bra{ m^{(0)}} V \ket{n^{(0)}},
\end{dmath}

that is

\begin{equation}\label{eqn:qmLecture20:580}
\begin{aligned}
\ket{n_1}
&=
\sum_{m \ne n}
\frac{
\ket{m^{(0)}}
}
{
E_n^{(0)} - E_m^{(0)}
}
V_{m n}
%\bra{ m^{(0)}} V \ket{n_0}
\\
\Delta_{n_1} &= \bra{ n^{(0)} } V \ket{ n^{(0)} } = V_{nn}.
\end{aligned}
\end{equation}

\paragraph{to \( O(\lambda^2) \) }

The second order perturbation states are found by selecting only the \( \lambda^2 \) contributions to

\begin{dmath}\label{eqn:qmLecture20:820}
\lambda^2 \ket{n_2}
=
\sum_{m \ne n}
\frac{\ket{m^{(0)}}\bra{m^{(0)}}}
{
E_n^{(0)} - E_m^{(0)}
}
 \lr{ \lambda V - (\lambda \Delta_{n_1} + \lambda^2 \Delta_{n_2}) }
\lr{
\ket{n_0}
+ \lambda \ket{n_1}
}.
\end{dmath}

Because \( \ket{n_0} = \ket{n^{(0)}} \), the \( \lambda^2 \Delta_{n_2} \) is killed, leaving

\begin{dmath}\label{eqn:qmLecture20:840}
\ket{n_2}
=
\sum_{m \ne n}
\frac{\ket{m^{(0)}}\bra{m^{(0)}}}
{
E_n^{(0)} - E_m^{(0)}
}
 \lr{ V - \Delta_{n_1} }
\ket{n_1}
=
\sum_{m \ne n}
\frac{\ket{m^{(0)}}\bra{m^{(0)}}}
{
E_n^{(0)} - E_m^{(0)}
}
 \lr{ V - \Delta_{n_1} }
\sum_{l \ne n}
\frac{
\ket{l^{(0)}}
}
{
E_n^{(0)} - E_l^{(0)}
}
V_{l n},
\end{dmath}

which can be written as

\begin{dmath}\label{eqn:qmLecture20:620}
\ket{n_2}
=
\sum_{l,m \ne n}
\ket{m^{(0)}}
\frac{V_{m l} V_{l n}}
{
\lr{ E_n^{(0)} - E_m^{(0)} }
\lr{ E_n^{(0)} - E_l^{(0)} }
}
-
\sum_{m \ne n}
\ket{m^{(0)}}
\frac{V_{n n} V_{m n}}
{
\lr{ E_n^{(0)} - E_m^{(0)} }^2
}.
\end{dmath}

For the second energy perturbation we have

\begin{dmath}\label{eqn:qmLecture20:860}
\lambda^2 \Delta_{n_2} =
\bra{n^{(0)}} \lambda V \lr{ \lambda \ket{n_1} },
\end{dmath}

or

\begin{dmath}\label{eqn:qmLecture20:880}
\Delta_{n_2}
=
\bra{n^{(0)}} V \ket{n_1}
=
\bra{n^{(0)}} V
\sum_{m \ne n}
\frac{
\ket{m^{(0)}}
}
{
E_n^{(0)} - E_m^{(0)}
}
V_{m n}.
\end{dmath}

That is

\begin{dmath}\label{eqn:qmLecture20:900}
\Delta_{n_2}
=
\sum_{m \ne n} \frac{V_{n m} V_{m n} }{E_n^{(0)} - E_m^{(0)}}.
\end{dmath}

\paragraph{to \( O(\lambda^3) \) }

Similarly, it can be shown that

\begin{dmath}\label{eqn:qmLecture20:640}
\Delta_{n_3} =
\sum_{l, m \ne n} \frac{V_{n m} V_{m l} V_{l n} }{
\lr{ E_n^{(0)} - E_m^{(0)} }
\lr{ E_n^{(0)} - E_l^{(0)} }
}
-
\sum_{ m \ne n} \frac{V_{n m} V_{n n} V_{m n} }{
\lr{ E_n^{(0)} - E_m^{(0)} }^2
}.
\end{dmath}

In general, the energy perturbation is given by

\begin{dmath}\label{eqn:qmLecture20:660}
\Delta_n^{(l)} = \bra{n^{(0)}} V \ket{n_{l-1}}.
\end{dmath}

%\EndArticle

      %
% Copyright � 2015 Peeter Joot.  All Rights Reserved.
% Licenced as described in the file LICENSE under the root directory of this GIT repository.
%
%\input{../blogpost.tex}
%\renewcommand{\basename}{qmLecture21}
%\renewcommand{\dirname}{notes/phy1520/}
%\newcommand{\keywords}{PHY1520H}
%\input{../peeter_prologue_print2.tex}
%
%%\usepackage{phy1520}
%\usepackage{peeters_braket}
%\usepackage{macros_cal}
%\usepackage{macros_bm}
%%\usepackage{peeters_layout_exercise}
%\usepackage{peeters_figures}
%\usepackage{mathtools}
%
%\beginArtNoToc
%\generatetitle{PHY1520H Graduate Quantum Mechanics.  Lecture 21: Non-degenerate perturbation.  Taught by Prof.\ Arun Paramekanti}
%%\chapter{Non-degenerate pertubation}
%\label{chap:qmLecture21}
%
%\paragraph{Disclaimer}
%
%Peeter's lecture notes from class.  These may be incoherent and rough.
%
%These are notes for the UofT course PHY1520, Graduate Quantum Mechanics, taught by Prof. Paramekanti, covering \textchapref{{5}} \citep{sakurai2014modern} content.
%
%\paragraph{Non-degenerate perturbation theory.  Recap.}
%
%\begin{dmath}\label{eqn:qmLecture21:20}
%\ket{n} = \ket{n_0}
%+ \lambda \ket{n_1}
%+ \lambda^2 \ket{n_2}
%+ \lambda^3 \ket{n_3} + \cdots
%\end{dmath}
%
%and
%
%\begin{dmath}\label{eqn:qmLecture21:40}
%\Delta_{n} = \Delta_{n_0}
%+ \lambda \Delta_{n_1}
%+ \lambda^2 \Delta_{n_2}
%+ \lambda^3 \Delta_{n_3} + \cdots
%\end{dmath}
%
%\begin{dmath}\label{eqn:qmLecture21:60}
%\begin{aligned}
%\Delta_{n_1} &= \bra{n^{(0)}} V \ket{n^{(0)}} \\
%\ket{n_0} &= \ket{n^{(0)}}
%\end{aligned}
%\end{dmath}
%
%\begin{dmath}\label{eqn:qmLecture21:80}
%\begin{aligned}
%\Delta_{n_2} &= \sum_{m \ne n} \frac{\Abs{\bra{n^{(0)}} V \ket{m^{(0)}}}^2}{E_n^{(0)} - E_m^{(0)}} \\
%\ket{n_1} &= \sum_{m \ne n} \frac{ \ket{m^{(0)}} V_{mn} }{E_n^{(0)} - E_m^{(0)}}
%\end{aligned}
%\end{dmath}
%
\section{Stark effect.}
\index{Stark effect}

%
\begin{equation}\label{eqn:qmLecture21:100}
H = H_{\textrm{atom}} + e \calE z,
\end{equation}
%
where \( H_{\textrm{atom}} \) is assumed to be Hydrogen-like with Hamiltonian
%
\begin{equation}\label{eqn:qmLecture21:120}
H_{\textrm{atom}} = \frac{\Bp^2}{2m} - \frac{e^2}{4 \pi \epsilon_0 r},
\end{equation}
%
and wave functions
%
\begin{equation}\label{eqn:qmLecture21:140}
\braket{\Br}{\psi_{n l m}} = R_{n l}(r) Y_{lm}( \theta, \phi ).
\end{equation}
%
Referring to \cref{eqn:qmLecture21:640}, the first level correction to the energy
%
\begin{dmath}\label{eqn:qmLecture21:160}
\Delta_1
= \bra{\psi_{100}} e \calE z \ket{ \psi_{100}}
= e \calE \int \frac{d\Omega}{4 \pi} \cos \theta \int dr r^2 R_{100}^2(r).
\end{dmath}
The cosine integral is obliterated, so we have \( \Delta_1 = 0 \).
How about the second order energy correction?  That is
%
\begin{dmath}\label{eqn:qmLecture21:180}
\Delta_2 = \sum_{n l m \ne 100} \frac{
\Abs{ \bra{\psi_{100}} e \calE z \ket{ n l m }}^2
}{
E_{100}^{(0)} - E_{n l m}
}.
\end{dmath}

The matrix element in the numerator is the absolute square of
%
\begin{equation}\label{eqn:qmLecture21:200}
V_{100,nlm}
=
e \calE \int d\Omega \inv{\sqrt{ 4 \pi } }
%\underbrace{
\cos\theta Y_{l m}(\theta, \phi)
%}_{non zero only when \( l = 1, m = 0\) }
\int dr r^3 R_{100}(r) R_{n l}(r).
\end{equation}
%
For all \( m \ne 0 \), \( Y_{lm} \) includes a \( e^{i m \phi} \) factor, so this cosine integral is zero.  For \( m = 0 \), each of the \( Y_{lm} \) functions appears to contain either even or odd powers of cosines (see: \cref{eqn:qmLecture21:760}).  This shows that for even \( 2k = l \), the cosine integral is zero
%
\begin{equation}\label{eqn:qmLecture21:780}
\int_0^\pi \sin\theta \cos\theta \sum_k a_k \cos^{2k}\theta d\theta
=
0,
\end{equation}
%
since \( \cos^{2k}(\theta) \) is even and \( \sin\theta \cos\theta \) is odd over the same interval.  We find zero for \( \int_0^\pi \sin\theta \cos\theta Y_{30}(\theta, \phi) d\theta \), and Mathematica appears to show that the rest of these integrals for \( l > 1 \) are also zero.

FIXME: find the property of the spherical harmonics that can be used to prove that this is true in general for \( l > 1 \).

This leaves
%
\begin{dmath}\label{eqn:qmLecture21:220}
\Delta_2
= \sum_{n \ne 1} \frac{
\Abs{ \bra{\psi_{100}} e \calE z \ket{ n 1 0 }}^2
}{
E_{100}^{(0)} - E_{n 1 0}
}
=
-e^2 \calE^2
\sum_{n \ne 1} \frac{
\Abs{ \bra{\psi_{100}} z \ket{ n 1 0 }}^2
}{
E_{n 1 0}
-E_{100}^{(0)}
}.
\end{dmath}
%
This is sometimes written in terms of a polarizability \( \alpha \)
%
\begin{equation}\label{eqn:qmLecture21:260}
\Delta_2 = -\frac{\calE^2}{2} \alpha,
\end{equation}
%
where
%
\begin{dmath}\label{eqn:qmLecture21:280}
\alpha =
2 e^2
\sum_{n \ne 1} \frac{
\Abs{ \bra{\psi_{100}} z \ket{ n 1 0 }}^2
}{
E_{n 1 0}
-E_{100}^{(0)}
}.
\end{dmath}
%
With
\begin{equation}\label{eqn:qmLecture21:840}
\BP = \alpha \bcE,
\end{equation}
%
the energy change upon turning on the electric field from \( 0 \rightarrow \calE \) is simply \( - \BP \cdot d\bcE \) integrated from \( 0 \rightarrow \calE \).  Putting \( \BP = \alpha \calE \zcap \), we have
%
\begin{dmath}\label{eqn:qmLecture21:400}
- \int_0^\calE P_z d\calE
=
- \int_0^\calE \alpha \calE d\calE
=
- \inv{2} \alpha \calE^2,
\end{dmath}
leading to an energy change \( - \alpha \calE^2/2 \), so we can directly compute \( \expectation{\BP} \) or we can compute change in energy, and both contain information about the polarization factor \( \alpha \).
There is an exact answer to the sum \cref{eqn:qmLecture21:280}, but we aren't going to try to get it here.  Instead let's look for bounds
%
\begin{equation}\label{eqn:qmLecture21:240}
\Delta_2^{\mathrm{min}} < \Delta_2 < \Delta_2^{\mathrm{max}},
\end{equation}
%
\begin{dmath}\label{eqn:qmLecture21:320}
\alpha^{\mathrm{min}} = 2 e^2 \frac{
\Abs{ \bra{\psi_{100}} z \ket{\psi_{210}} }^2
}{E_{210}^{(0)} - E_{100}^{(0)}}.
\end{dmath}
For the hydrogen atom we have
%
\begin{equation}\label{eqn:qmLecture21:820}
E_n = -\frac{ e^2}{ 2 n^2 a_0 },
\end{equation}
%
allowing any difference of energy levels to be expressed as a fraction of the ground state energy, such as
%
\begin{equation}\label{eqn:qmLecture21:340}
E_{210}^{(0)} = \inv{4} E_{100}^{(0)} = \inv{4} \frac{ -\Hbar^2 }{ 2 m a_0^2 }.
\end{equation}
So
\begin{dmath}\label{eqn:qmLecture21:360}
E_{210}^{(0)} - E_{100}^{(0)} = \frac{3}{4}
\frac{ \Hbar^2 }{ 2 m a_0^2 }
\end{dmath}
In the numerator we have
%
\begin{dmath}\label{eqn:qmLecture21:380}
\bra{\psi_{100}} z \ket{\psi_{210}}
=
\int r^2 d\Omega
\lr{ \inv{\sqrt{\pi} a_0^{3/2}} e^{-r/a_0} } r \cos\theta \lr{
\inv{4 \sqrt{2 \pi} a_0^{3/2}} \frac{r}{a_0} e^{-r/2a_0} \cos\theta
}
=
(2 \pi)
\inv{\sqrt{\pi}} \inv{4 \sqrt{2 \pi} } a_0
\int_0^\pi d\theta \sin\theta \cos^2\theta
\int_0^\infty \frac{dr}{a_0} \frac{r^4}{a_0^4} e^{-r/a_0 - r/2 a_0}
=
(2 \cancel{\pi})
\cancel{\inv{\sqrt{\pi}}} \inv{4 \sqrt{2 \cancel{\pi}} } a_0
\lr{ \evalrange{-\frac{u^3}{3}}{1}{-1} }
\int_0^\infty s^4 ds e^{- 3 s/2 }
=
\inv{2 \sqrt{2}} \frac{2}{3} a_0 \frac{256}{81}
%=
%\int \frac{d\Omega}{\sqrt{4 \pi}} \sqrt{ \frac{3}{4 \pi} } \cos\theta \cos\theta
%\int dr r^2 r \frac{2}{a_0^{3/2}} e^{-r/a_0} \inv{\sqrt{3}} \frac{r}{a_0} \inv{ (2 a_0)^{3/2} } e^{-r/2 a_0}
=
\frac{1}{3 \sqrt{2} } \frac{ 256}{81} a_0
\approx 0.75 a_0.
\end{dmath}
%
This gives
%
\begin{dmath}\label{eqn:qmLecture21:420}
\alpha^{\mathrm{min}}
= \frac{ 2 e^2 (0.75)^2 a_0^2 }{ \frac{3}{4} \frac{\Hbar^2}{2 m a_0^2} }
= \frac{6}{4} \frac{2 m e^2 a_0^4}{ \Hbar^2 }
= 3 \frac{m e^2 a_0^4}{ \Hbar^2 }
= 3 \frac{ 4 \pi \epsilon_0 }{a_0} a_0^4
\approx 4 \pi \epsilon_0 a_0^3 \times 3.
\end{dmath}
% a_0 = 4 \pi e_0 hbar^2/m e^2
% a_0 e^2 = 4 \pi e_0 hbar^2/m
% 4 \pi e_0 /(a_0) = m e^2/hbar^2
The factor \( 4 \pi \epsilon_0 a_0^3 \) are the natural units for the polarizability.
There is a neat trick that generalizes to many problems to find the upper bound.  Recall that the general polarizability was
%
\begin{dmath}\label{eqn:qmLecture21:440}
\alpha =
2 e^2
\sum_{nlm \ne 100} \frac{
\Abs{ \bra{100} z \ket{ n l m }}^2
}{
E_{n l m}
-E_{100}^{(0)}
}.
\end{dmath}
%
If we are looking for the upper bound, and replace the denominator by the smallest energy difference that will be encountered, it can be brought out of the sum, for
%
\begin{dmath}\label{eqn:qmLecture21:460}
\alpha^{\mathrm{max}} =
2 e^2
\inv{E_{2 1 0}
-E_{100}^{(0)} }
\sum_{nlm \ne 100}
\bra{100} z \ket{ n l m } \bra{nlm} z \ket{ 100 }.
\end{dmath}
Because \( \bra{nlm} z \ket{100} = 0 \), the constraint in the sum can be removed, and the identity summation evaluated
%
\begin{dmath}\label{eqn:qmLecture21:480}
\alpha^{\mathrm{max}} =
2 e^2
\inv{E_{2 1 0}
-E_{100}^{(0)} }
\sum_{nlm}
\bra{100} z \ket{ n l m } \bra{nlm} z \ket{ 100 }
=
\frac{2 e^2 }{ \frac{3}{4} \frac{\Hbar^2}{ 2 m a_0^2} }
\bra{100} z^2 \ket{ 100 }
=
\frac{16 e^2 m a_0^2 }{ 3 \Hbar^2 } \times a_0^2
= 4 \pi \epsilon_0 a_0^3 \times \frac{16}{3}.
\end{dmath}
%
The bounds are
%
%\begin{dmath}\label{eqn:qmLecture21:520}
\boxedEquation{eqn:qmLecture21:540}{
3 \ge \frac{\alpha}{\alpha^{\mathrm{at}}} < \frac{16}{3},
}
%\end{dmath}

where
%
\begin{equation}\label{eqn:qmLecture21:560}
\alpha^{\mathrm{at}} = 4 \pi \epsilon_0 a_0^3.
\end{equation}
%
The actual value is
\begin{equation}\label{eqn:qmLecture21:580}
\frac{\alpha}{\alpha^{\mathrm{at}}} = \frac{9}{2}.
\end{equation}
%
See \nbref{lecture21someSphericalHarmonicsAndTheirIntegrals.nb}, for some of the integrals above, and for spherical harmonic tables.

\paragraph{Example: Computing the dipole moment}
\index{dipole moment}
%
\begin{equation}\label{eqn:qmLecture21:600}
\expectation{P_z}
= \alpha \calE
= \bra{\psi_{100}} e z \ket{\psi_{100}}.
\end{equation}
%
Without any perturbation this is zero.  After perturbation, retaining only the terms that are first order in \( \delta \psi_{100} \) we have
%
\begin{dmath}\label{eqn:qmLecture21:620}
\bra{\psi_{100} + \delta \psi_{100}} e z \ket{\psi_{100} + \delta \psi_{100}}
\approx
\bra{\psi_{100}} e z \ket{\delta \psi_{100}}
+
\bra{\delta \psi_{100}} e z \ket{\psi_{100}}.
\end{dmath}
%
\paragraph{Next time: van der Walls}

We will look at two hyrdogenic atomic systems interacting where the pair of nuclei are supposed to be infinitely heavy and stationary.  The wave functions each set of atoms are individually known, but we can consider the problem of the interactions of atom 1's electrons with atom 2's nucleus and atom 2's electrons, and also the opposite interactions of atom 2's electrons with atom 1's nucleus and its electrons.  This leads to a result that is linear in the electric field (unlike the above result, which is called the quadratic Stark effect).

%problem set: 3pm: rm 1305

%\EndArticle

      %
% Copyright � 2015 Peeter Joot.  All Rights Reserved.
% Licenced as described in the file LICENSE under the root directory of this GIT repository.
%
%\input{../blogpost.tex}
%\renewcommand{\basename}{qmLecture22}
%\renewcommand{\dirname}{notes/phy1520/}
%\newcommand{\keywords}{PHY1520H}
%\input{../peeter_prologue_print2.tex}
%
%%\usepackage{phy1520}
%\usepackage{peeters_braket}
%\usepackage{macros_cal}
%%\usepackage{peeters_layout_exercise}
%\usepackage{peeters_figures}
%\usepackage{mathtools}
%
%\beginArtNoToc
%\generatetitle{PHY1520H Graduate Quantum Mechanics.  Lecture 22: van der Walls potential and Stark effect.  Taught by Prof.\ Arun Paramekanti}
%%\chapter{van der Walls potential and Stark effect}
%\label{chap:qmLecture22}
%
%\paragraph{Disclaimer}
%
%Peeter's lecture notes from class.  These may be incoherent and rough.
%
%These are notes for the UofT course PHY1520, Graduate Quantum Mechanics, taught by Prof. Paramekanti, covering \textchapref{{5}} \citep{sakurai2014modern} content.
%
\paragraph{Another approach (for last time?)}

Imagine we perturb a potential, say a harmonic oscillator with an electric field
%
\begin{dmath}\label{eqn:qmLecture22:20}
V_0(x) = \inv{2} k x^2,
\end{dmath}
\begin{dmath}\label{eqn:qmLecture22:40}
V(x) = \calE e x.
\end{dmath}
After minimizing the energy, using \( \PDi{x}{V} = 0 \), we get
%
\begin{dmath}\label{eqn:qmLecture22:60}
\inv{2} k x^2 + \calE e x \rightarrow k x^\conj = - e \calE,
\end{dmath}
%
\begin{dmath}\label{eqn:qmLecture22:80}
p^\conj = -e x^\conj = - \frac{e^2 \calE}{k}.
\end{dmath}
For such a system the polarizability is
%
\begin{dmath}\label{eqn:qmLecture22:100}
\alpha = \frac{e^2 }{k},
\end{dmath}
%
\begin{dmath}\label{eqn:qmLecture22:120}
\inv{2} k \lr{ -\frac{ e \calE}{k} }^2 + \calE e \lr{ - \frac{e \calE}{k} }
= - \inv{2} \lr{ \frac{e^2}{k} } \calE^2
= - \inv{2} \alpha \calE^2
\end{dmath}
%\cref{fig:lecture22:lecture22Fig1}.
%\imageFigure{../figures/phy1520-quantum/lecture22Fig1}{CAPTION: lecture22Fig1}{fig:lecture22:lecture22Fig1}{0.2}
\section{van der Walls potential.}
\index{van der Walls}
%
\begin{dmath}\label{eqn:qmLecture22:140}
H_0 =
H_{0 1} + H_{0 2},
\end{dmath}
%
where
%
\begin{equation}\label{eqn:qmLecture22:160}
H_{0 \alpha} = \frac{p_\alpha^2}{2m} - \frac{e^2}{4 \pi \epsilon_0 \Abs{ \Br_\alpha - \BR_\alpha} }, \qquad \alpha = 1,2.
\end{equation}
%
The full interaction potential is
%
\begin{dmath}\label{eqn:qmLecture22:180}
V =
\frac{e^2}{4 \pi \epsilon_0} \lr{
\inv{\Abs{\BR_1 - \BR_2}}
+
\inv{\Abs{\Br_1 - \Br_2}}
-
\inv{\Abs{\Br_1 - \BR_2}}
-
\inv{\Abs{\Br_2 - \BR_1}}
}.
\end{dmath}
Let
%
\begin{dmath}\label{eqn:qmLecture22:200}
\Bx_\alpha = \Br_\alpha - \BR_\alpha,
\end{dmath}
%
\begin{dmath}\label{eqn:qmLecture22:220}
\BR = \BR_1 - \BR_2,
\end{dmath}
%
as sketched in \cref{fig:lecture22:lecture22Fig2}.
\imageFigure{../figures/phy1520-quantum/lecture22Fig2}{Two atom interaction.}{fig:lecture22:lecture22Fig2}{0.2}
%
\begin{dmath}\label{eqn:qmLecture22:240}
H_{0 \alpha}
=
\frac{\Bp^2}{2m}
-\frac{e^2}{4 \pi \epsilon_0 \Abs{\Bx_\alpha}},
\end{dmath}
which allows the total interaction potential to be written
\begin{dmath}\label{eqn:qmLecture22:260}
V =
\frac{e^2}{4 \pi \epsilon_0 R}
\lr{
1
+
\frac{R}{\Abs{\Bx_1 - \Bx_2 + \BR}}
-
\frac{R}{\Abs{\Bx_1 + \BR}}
-
\frac{R}{\Abs{-\Bx_2 + \BR}}
}.
\end{dmath}
For \( R \gg x_1, x_2 \), this interaction potential, after a multipole expansion, is approximately
%
\begin{dmath}\label{eqn:qmLecture22:280}
V =
\frac{e^2}{4 \pi \epsilon_0} \lr{
\frac{\Bx_1 \cdot \Bx_2}{\Abs{\BR}^3}
-3 \frac{
(\Bx_1 \cdot \BR)
(\Bx_2 \cdot \BR)
}{\Abs{\BR}^5}
}
\end{dmath}
Showing this is left as a exercise.
\paragraph{1. \( O(\lambda) \) }.
With
%
\begin{dmath}\label{eqn:qmLecture22:300}
\psi_0 = \ket{ 1s, 1s },
\end{dmath}
%
\begin{dmath}\label{eqn:qmLecture22:320}
\Delta E^{(1)} = \bra{\psi_0} V \ket{\psi_0},
\end{dmath}
the two particle wave functions are of the form
%
\begin{dmath}\label{eqn:qmLecture22:340}
\braket{ \Bx_1, \Bx_2 }{\psi_0} =
\psi_{1s}(\Bx_1)
\psi_{1s}(\Bx_2),
\end{dmath}
%
so braket integrals must be evaluated over a six-fold space.  Recall that
%
\begin{dmath}\label{eqn:qmLecture22:740}
\psi_{1s} = \inv{\sqrt{\pi} a_0^{3/2} } e^{-r/a_0},
\end{dmath}
%
so
%
\begin{dmath}\label{eqn:qmLecture22:760}
\bra{\psi_{1s}} x_i \ket{\psi_{1s}}
\propto
\int_0^\pi \sin\theta d\theta \int_0^{2\pi} d\phi x_i,
\end{dmath}
where
\begin{dmath}\label{eqn:qmLecture22:780}
x_i \in \setlr{ r \sin\theta \cos\phi, r \sin\theta \sin\phi, r \cos\theta }.
\end{dmath}
%
The \( x, y \) integrals are zero because of the \( \phi \) integral, and the \( z \) integral is proportional to \( \int_0^\pi \sin(2 \theta) d\theta \), which is also zero.  This leads to zero averages
%
\begin{equation}\label{eqn:qmLecture22:360}
\expectation{\Bx_1} = 0 = \expectation{\Bx_2},
\end{equation}
so
%
\begin{dmath}\label{eqn:qmLecture22:380}
\Delta E^{(1)} = 0.
\end{dmath}
%
\paragraph{2. \( O(\lambda^2) \)}.
%
\begin{dmath}\label{eqn:qmLecture22:400}
\Delta E^{(2)}
= \sum_{n \ne 0} \frac{ \Abs{ \bra{\psi_n } V \ket{\psi_0} }^2 }{E_0 - E_n}
= \sum_{n \ne 0} \frac{ \bra{\psi_0 } V \ket{\psi_n}  \bra{\psi_n } V \ket{\psi_0} }{E_0 - E_n}.
\end{dmath}
%
This is a sum over all excited states.
%Here the \( \sum' \) indicates the sum over all excited states.
We expect that this will be of the form
%
\begin{dmath}\label{eqn:qmLecture22:420}
\Delta E^{(2)} = - \lr{ \frac{e^2}{4 \pi \epsilon_0} }^2 \frac{C_6}{R^6},
\end{dmath}
where
\( \Bx_1 \) and \( \Bx_2 \) are dipole operators.  The first time this has a non-zero expectation is when we go from the 1s to the 2p states (both 1s and 2s states are spherically symmetric).
Noting that \( E_n = -e^2/2 n^2 a_0 \), we can compute a minimum bound for the energy denominator
%
\begin{dmath}\label{eqn:qmLecture22:440}
\lr{E_n - E_0}^{\mathrm{min}}
= 2 \lr{ E_{2p} - E_{1s} }
= 2 E_{1s} \lr{ \inv{4} - 1 }
= 2 \frac{3}{4} \Abs{E_{1s}}
= \frac{3}{2} \Abs{E_{1s}}.
\end{dmath}
%
Note that the factor of two above comes from summing over the energies for both electrons.  This gives us
%
\begin{dmath}\label{eqn:qmLecture22:460}
C_6
=
\frac{3}{2} \Abs{E_{1s}}
\bra{\psi_0 } \tilde{V} \ket{\psi_0},
\end{dmath}
%
where
%
\begin{dmath}\label{eqn:qmLecture22:480}
\tilde{V} =
\lr{
\Bx_1 \cdot \Bx_2
-3
(\Bx_1 \cdot \Rcap)
(\Bx_2 \cdot \Rcap)
}.
\end{dmath}
\paragraph{What about degeneracy?}
\index{degeneracy}
%
\begin{dmath}\label{eqn:qmLecture22:500}
\Delta E^{(2)}_n
= \sum_{m \ne n} \frac{ \Abs{ \bra{\psi_n } V \ket{\psi_0} }^2 }{E_0 - E_n}.
\end{dmath}
If \( \bra{\psi_n} V \ket{\psi_m} \propto \delta_{n m} \) then it's okay.
In general the we can't expect the matrix element will be anything but fully populated, say
%
\begin{dmath}\label{eqn:qmLecture22:520}
V =
\begin{bmatrix}
V_{11} & V_{12} & V_{13} & V_{14} \\
V_{21} & V_{22} & V_{23} & V_{24} \\
V_{31} & V_{32} & V_{33} & V_{34} \\
V_{41} & V_{42} & V_{43} & V_{44} \\
\end{bmatrix}.
\end{dmath}
%
If we choose a basis so that
%
\begin{dmath}\label{eqn:qmLecture22:540}
V =
\begin{bmatrix}
V_{11} &        &        &        \\
       & V_{22} &        &        \\
       &        & V_{33} &        \\
       &        &        & V_{44} \\
\end{bmatrix}.
\end{dmath}
%
When this is the case, we have no mixing of elements in the sum of \cref{eqn:qmLecture22:500}

\paragraph{Degeneracy in the Stark effect}
\index{degeneracy!Stark effect}
%
\begin{dmath}\label{eqn:qmLecture22:560}
H = H_0 + e \calE z,
\end{dmath}
%
where
%
\begin{dmath}\label{eqn:qmLecture22:580}
H_0 = \frac{\Bp^2}{2m} - \frac{e}{4 \pi \epsilon_0} \inv{\Abs{\Bx}}.
\end{dmath}
Consider the states \( 2s, 2 p_x, 2p_y, 2p_z \), for which \( E_n^{(0)} \equiv E_{2 s} \), as sketched in \cref{fig:lecture22:lecture22Fig3}.
\imageFigure{../figures/phy1520-quantum/lecture22Fig3}{2s 2p degeneracy.}{fig:lecture22:lecture22Fig3}{0.2}
Because of spherical symmetry
%
\begin{equation}\label{eqn:qmLecture22:600}
\begin{aligned}
\bra{2 s} e \calE z \ket{ 2 s}      &= 0, \\
\bra{2 p_x} e \calE z \ket{ 2 p_x}  &= 0, \\
\bra{2 p_y} e \calE z \ket{ 2 p_y}  &= 0, \\
\bra{2 p_z} e \calE z \ket{ 2 p_z}  &= 0.
\end{aligned}
\end{equation}
Looking at odd and even properties, it turns out that the only off-diagonal matrix element is
%
\begin{equation}\label{eqn:qmLecture22:620}
\bra{2 s} e \calE z \ket{ 2 p_z } = V_1 = -3 e \calE a_0.
\end{equation}
%
With a \( \setlr{ 2s, 2p_x, 2p_y, 2p_z } \) basis the potential matrix is
%
\begin{dmath}\label{eqn:qmLecture22:640}
\begin{bmatrix}
0 & 0 & 0 & V_1 \\
0 & 0 & 0 & 0 \\
0 & 0 & 0 & 0 \\
V_1^\conj & 0 & 0 & 0 \\
\end{bmatrix}
\end{dmath}
which has the block structure
%\cref{fig:lecture22:lecture22Fig4}.
%\imageFigure{../figures/phy1520-quantum/lecture22Fig4}{CAPTION: lecture22Fig4}{fig:lecture22:lecture22Fig4}{0.2}
% 2s, 2p_z
\begin{dmath}\label{eqn:qmLecture22:660}
\begin{bmatrix}
0 & -\Abs{V_1} \\
-\Abs{V_1} & 0 \\
\end{bmatrix}.
\end{dmath}
This implies that the energy splitting goes as
%
\begin{equation}\label{eqn:qmLecture22:680}
E_{2s} \rightarrow
E_{2s} \pm \Abs{V_1},
\end{equation}
%
as sketched in \cref{fig:lecture22:lecture22Fig5}.
\imageFigure{../figures/phy1520-quantum/lecture22Fig5}{Stark effect energy level splitting.}{fig:lecture22:lecture22Fig5}{0.2}
The diagonalizing states corresponding to eigenvalues \( \pm 3 a_0 \calE \), are \( (\ket{2s} \mp \ket{2p_z})/\sqrt{2} \).
The matrix element above is calculated explicitly in \nbref{lecture22Integrals.nb}.
The degeneracy that is left unsplit here, and has to be accounted for should we attempt higher order perturbation calculations.
%\EndArticle

      \section{Problems.}
         %
% Copyright © 2015 Peeter Joot.  All Rights Reserved.
% Licenced as described in the file LICENSE under the root directory of this GIT repository.
%
\makeproblem{van der Walls multipole expansion.}{problem:qmLecture22:1}{
\index{van der Walls!multipole expansion}
Prove \cref{eqn:qmLecture22:280}.
} % problem

\makeanswer{problem:qmLecture22:1}{

Noting that

\begin{dmath}\label{eqn:qmLecture22:700}
\lr{1 + \epsilon}^{-1/2}
=
1 -\inv{2} \epsilon -\inv{2}\lr{\frac{-3}{2}}\inv{2!} \epsilon^2
=
1 -\inv{2} \epsilon + \frac{3}{8} \epsilon^2,
\end{dmath}

we have

\begin{dmath}\label{eqn:qmLecture22:720}
\frac{R}{\Abs{\Bepsilon + \BR}}
=
\frac{1}{\Abs{\frac{\Bepsilon}{R} + \Rcap}}
=
\lr{ 1 + 2 \frac{\Bepsilon}{R} \cdot \Rcap + \lr{\frac{\Bepsilon}{R}}^2 }^{-1/2}
=
1 - \frac{\Bepsilon}{R} \cdot \Rcap -\inv{2} \lr{\frac{\Bepsilon}{R}}^2
+ \frac{3}{8}
\lr{ 2 \frac{\Bepsilon}{R} \cdot \Rcap + \lr{\frac{\Bepsilon}{R}}^2 }^2
=
1 - \frac{\Bepsilon}{R} \cdot \Rcap -\inv{2} \lr{\frac{\Bepsilon}{R}}^2
+ \frac{3}{8}
\lr{ 4 \lr{ \frac{\Bepsilon}{R} \cdot \Rcap}^2 + \lr{\frac{\Bepsilon}{R}}^4
+ 4 \frac{\Bepsilon}{R} \cdot \Rcap \lr{\frac{\Bepsilon}{R}}^2
}
\approx
1 - \frac{\Bepsilon}{R} \cdot \Rcap -\inv{2} \lr{\frac{\Bepsilon}{R}}^2
+ \frac{3}{2}
\lr{ \frac{\Bepsilon}{R} \cdot \Rcap}^2 .
\end{dmath}

Inserting the values from the brackets of \cref{eqn:qmLecture22:260} we have

\begin{dmath}\label{eqn:qmLecture22:800}
\begin{aligned}
1
+
\frac{R}{\Abs{\Bx_1 - \Bx_2 + \BR}}
&-
\frac{R}{\Abs{\Bx_1 + \BR}}
-
\frac{R}{\Abs{-\Bx_2 + \BR}} \\
&=
- \frac{\lr{ \Bx_1 - \Bx_2 }}{R} \cdot \Rcap -\inv{2} \lr{\frac{\lr{ \Bx_1 - \Bx_2 }}{R}}^2
+ \frac{3}{2}
\lr{ \frac{\lr{ \Bx_1 - \Bx_2 }}{R} \cdot \Rcap}^2  \\
&\quad + \frac{\Bx_1}{R} \cdot \Rcap +\inv{2} \lr{\frac{\Bx_1}{R}}^2
- \frac{3}{2}
\lr{ \frac{\Bx_1}{R} \cdot \Rcap}^2  \\
&\quad - \frac{\Bx_2}{R} \cdot \Rcap +\inv{2} \lr{\frac{\Bx_2}{R}}^2
- \frac{3}{2}
\lr{ \frac{\Bx_2}{R} \cdot \Rcap}^2  \\
&=
\frac{\Bx_1}{R} \cdot \frac{\Bx_2 }{R}
+ \frac{3}{2}
\lr{ \frac{\lr{ \Bx_1 - \Bx_2 }}{R} \cdot \Rcap}^2  \\
&\quad
- \frac{3}{2}
\lr{ \frac{\Bx_1}{R} \cdot \Rcap}^2  \\
&\quad
- \frac{3}{2}
\lr{ \frac{\Bx_2}{R} \cdot \Rcap}^2  \\
&=
\frac{\Bx_1}{R} \cdot \frac{\Bx_2 }{R}
- 3 \frac{\Bx_1}{R} \cdot \Rcap  \frac{\Bx_2}{R} \cdot \Rcap.
\end{aligned}
\end{dmath}

This proves \cref{eqn:qmLecture22:280}.
} % answer

         % p5.1
         %
% Copyright � 2015 Peeter Joot.  All Rights Reserved.
% Licenced as described in the file LICENSE under the root directory of this GIT repository.
%
%\input{../blogpost.tex}
%\renewcommand{\basename}{harmonicOscillatorEnergyShiftPertubation}
%\renewcommand{\dirname}{notes/phy1520/}
%%\newcommand{\dateintitle}{}
%%\newcommand{\keywords}{}
%
%\input{../peeter_prologue_print2.tex}
%
%\usepackage{peeters_layout_exercise}
%\usepackage{peeters_braket}
%\usepackage{peeters_figures}
%
%\beginArtNoToc
%
%\generatetitle{Harmonic oscillator with energy shift}
%%\chapter{Harmonic oscillator with energy shift}
%%\label{chap:harmonicOscillatorEnergyShiftPertubation}
%
\makeoproblem{Harmonic oscillator with energy shift.}{problem:harmonicOscillatorEnergyShiftPertubation:160}{\citep{sakurai2014modern} pr. 5.1}{
\index{harmonic oscillator!perturbation}

Given a perturbed 1D SHO Hamiltonian
%
\begin{equation}\label{eqn:harmonicOscillatorEnergyShiftPertubation:20}
H = \inv{2m} p^2 + \inv{2} m \omega^2 x^2 + \lambda b x,
\end{equation}
%
calculate the first non-zero perturbation to the ground state energy.  Then solve for that energy directly and compare.
%
} % problem
%
\makeanswer{problem:harmonicOscillatorEnergyShiftPertubation:160}{
The first order energy shift is seen to be zero
%
\begin{dmath}\label{eqn:harmonicOscillatorEnergyShiftPertubation:40}
\Delta_0^{(0)}
= V_{00}
= \bra{0} b x \ket{0}
= \frac{x_0}{\sqrt{2}} \bra{0} a + a^\dagger \ket{0}
= \frac{x_0}{\sqrt{2}} \braket{0}{1}
= 0.
\end{dmath}
%
The first order perturbation to the ground state is
%
\begin{dmath}\label{eqn:harmonicOscillatorEnergyShiftPertubation:60}
\ket{0^{(1)}}
= \sum_{m \ne 0} \frac{ \ket{m} \bra{m} b x \ket{0} }{ \Hbar \omega/2 - \Hbar \omega (m - 1/2) }
= -b \frac{x_0}{\sqrt{2} \Hbar \omega} \sum_{m \ne 0} \frac{ \ket{m} \braket{m}{1} }{ m }
= -b \frac{x_0}{\sqrt{2} \Hbar \omega} \ket{1}.
\end{dmath}
%
The second order ground state energy perturbation is
%
\begin{dmath}\label{eqn:harmonicOscillatorEnergyShiftPertubation:80}
\Delta_0^{(2)}
=
\bra{0} b x \ket{0^{(1)}}
=
\frac{b x_0}{\sqrt{2}} \bra{0} a + a^\dagger \lr{ -b \frac{x_0}{\sqrt{2} \Hbar \omega} \ket{1} }
=
\frac{b x_0}{\sqrt{2}} \lr{ -b \frac{x_0}{\sqrt{2} \Hbar \omega} }
=
-\frac{b^2 x_0^2}{ 2 \Hbar \omega }
=
-\frac{b^2 }{ 2 \Hbar \omega } \frac{\Hbar}{m \omega}
=
-\frac{b^2 }{ 2 m \omega^2 },
\end{dmath}
%
so the total energy perturbation up to second order is
%
\begin{equation}\label{eqn:harmonicOscillatorEnergyShiftPertubation:100}
\Delta_0 = -\lambda^2 \frac{b^2 }{ 2 m \omega^2 }.
\end{equation}
%
To compare to the exact result, rewrite the Hamiltonian as
%
\begin{dmath}\label{eqn:harmonicOscillatorEnergyShiftPertubation:120}
H
= \inv{2m} p^2 + \inv{2} m \omega^2 \lr{ x^2 + \frac{2 \lambda b x}{m \omega^2} }
= \inv{2m} p^2 + \inv{2} m \omega^2 \lr{ x + \frac{\lambda b }{m \omega^2} }^2 - \inv{2} m \omega^2 \lr{ \frac{\lambda b }{m \omega^2} }^2.
\end{dmath}
%
The Hamiltonian is subject to a constant energy shift
%
\begin{dmath}\label{eqn:harmonicOscillatorEnergyShiftPertubation:140}
\Delta E
=
- \inv{2} m \omega^2 \frac{\lambda^2 b^2 }{m^2 \omega^4}
=
- \frac{\lambda^2 b^2 }{2 m \omega^2}.
\end{dmath}
%
This is an exact match with the second order perturbation result of \cref{eqn:harmonicOscillatorEnergyShiftPertubation:100}.
} % answer

%\EndArticle

         %
% Copyright � 2015 Peeter Joot.  All Rights Reserved.
% Licenced as described in the file LICENSE under the root directory of this GIT repository.
%
\makeoproblem{Double well potential.}{gradQuantum:problemSet7:1}{2015 ps7 p1}{
\index{double well potential}

Consider a particle in the double well potential
%
\begin{dmath}\label{eqn:gradQuantumProblemSet7Problem1:20}
V (x) =
\frac{m \omega^2}{ 8 a^2 }
\lr{ x + a }^2 \lr{x - a}^2.
\end{dmath}
%
Expanding \( V(x) \) around \( x = \pm a \) leads to a harmonic potential with frequency \( \omega \).
Construct variational states with even/odd parity as \( \psi_\pm(x) = g_\pm \lr{ \phi(x - a) \pm \phi(x + a) } \) where \( \phi(x) \) is the normalized ground state of the usual harmonic oscillator with frequency \( \omega \), i.e.,
%
\begin{equation}\label{eqn:gradQuantumProblemSet7Problem1:40}
\phi(x)
=
\lr{ \inv{ \pi a_0^2 }}^{1/4}
e^{
- \frac{x^2}{ 2 a_0^2 }
}
;
\qquad a_0 = \sqrt{ \frac{\Hbar}{m \omega} }.
\end{equation}
%
\makesubproblem{}{gradQuantum:problemSet7:1a}
Determine the normalization constants \( g_\pm \).
Next using these wavefunctions, determine the variational energies of these two states.
Hence determine the `tunnel splitting' between the two states, induced by the tunneling through
the barrier region.
In your calculations, you can assume \( a \gg a_0 \), so retain only the leading terms in any polynomials you might encounter when you do the integrals.
%
\makesubproblem{}{gradQuantum:problemSet7:1b}
If we pay attention to these lowest two states (left well and right well) in the full Hilbert space, we can write a phenomenological \( 2 \times 2 \) Hamiltonian
%
\begin{equation}\label{eqn:gradQuantumProblemSet7Problem1:60}
H =
\begin{bmatrix}
\epsilon_0 & -\gamma \\
-\gamma & \epsilon_0
\end{bmatrix},
\end{equation}
%
where \( \epsilon_0 \) is the energy on each side, and \( \gamma \) leads to tunneling, so if we start off in the left well, \( t \) leads to a nonzero amplitude to find it in the right well at a later time.
Find its eigenvalues and eigenvectors.
Comparing with your variational result for the energy splitting, determine the `tunnel coupling' \( \gamma \).
%
} % makeproblem
%
\makeanswer{gradQuantum:problemSet7:1}{
\withproblemsetsParagraph{
\makeSubAnswer{}{gradQuantum:problemSet7:1a}
%
The integration grunt work for this problem can be found in \nbref{ps7:doubleWellPotential.nb}.  This yields an energy difference of
%
\begin{equation}\label{eqn:gradQuantumProblemSet7Problem1:80}
\overbar{E}_{+} - \overbar{E}_{-}
=
\frac{\Hbar \omega}{8}
\lr{
\lr{
\frac{4 a^2}{a_0^2}
+
\lr{\frac{a^2}{a_0^2}-5 } \exp\lr{\frac{a^2}{a_0^2}} - 2
}
\lr{ \coth\lr{ \frac{a^2}{a_0^2} } -1 }
}.
\end{equation}
%
With \( u = a^2/a_0^2 \), in the \( a \gg a_0 \) limit, the almost zero \( \coth u - 1 \) difference can be approximated as an exponential
%
\begin{dmath}\label{eqn:gradQuantumProblemSet7Problem1:100}
\coth u -1
=
\frac{e^{2u} + 1}{e^{2u} - 1} - 1
=
\frac{e^{2u} + 1 - e^{2u} + 1 }{e^{2u} - 1}
=
\frac{ 2 }{e^{2u} - 1}
\approx
2 e^{-2u},
\end{dmath}
%
so the energy difference is approximately
%
\begin{dmath}\label{eqn:gradQuantumProblemSet7Problem1:120}
\overbar{E}_{+} - \overbar{E}_{-}
\approx
\frac{\Hbar \omega}{4} \frac{a^2}{a_0^2} \exp\lr{-\frac{a^2}{a_0^2}}.
\end{dmath}
%
\makeSubAnswer{}{gradQuantum:problemSet7:1b}
Now lets compare to the energy levels of the phenomenological Hamiltonian, which are given by
%
\begin{dmath}\label{eqn:gradQuantumProblemSet7Problem1:140}
0 = \lr{ \epsilon_0 - E }^2 - \gamma^2,
\end{dmath}
%
with eigenvalues
%
\begin{dmath}\label{eqn:gradQuantumProblemSet7Problem1:160}
E_{\pm} = \epsilon_0 \pm \gamma.
\end{dmath}
%
If the eigenvectors are proportional to the column vector given by
%
\begin{dmath}\label{eqn:gradQuantumProblemSet7Problem1:180}
\ket{\pm} =
\begin{bmatrix}
a \\
b
\end{bmatrix},
\end{dmath}
%
then we must have
%
\begin{dmath}\label{eqn:gradQuantumProblemSet7Problem1:200}
0
= \lr{ \epsilon_0 - (\epsilon_0 \pm \gamma) } a - \gamma b
= \gamma \lr{ \mp a - b },
\end{dmath}
%
or
%
\begin{dmath}\label{eqn:gradQuantumProblemSet7Problem1:220}
\ket{\pm}
=
\inv{\sqrt{2}}
\begin{bmatrix}
1 \\
\mp 1
\end{bmatrix}.
\end{dmath}
%
The energy level difference for this Hamiltonian is
%
\begin{dmath}\label{eqn:gradQuantumProblemSet7Problem1:240}
\Delta E
= E_{+} - E_{-}
= \epsilon_0 + \gamma - \lr{ \epsilon_0 - \gamma }
= 2 \gamma.
\end{dmath}
%
Equating this difference with \cref{eqn:gradQuantumProblemSet7Problem1:120}, we have
%
\begin{dmath}\label{eqn:gradQuantumProblemSet7Problem1:260}
\gamma
=
\frac{\Hbar \omega}{8} \frac{a^2}{a_0^2} \exp\lr{-\frac{a^2}{a_0^2}}.
\end{dmath}
}
}

         %
% Copyright � 2015 Peeter Joot.  All Rights Reserved.
% Licenced as described in the file LICENSE under the root directory of this GIT repository.
%
\makeoproblem{Helium-4 atom.}{gradQuantum:problemSet7:2}{2015 ps7 p2}{
\index{helium-4}

Consider the Helium atom with atomic number \( Z=2 \), which leads to the nuclear charge \( Z=2e \), and two electrons with charge \( -e \) each.

\makesubproblem{}{gradQuantum:problemSet7:2a}
Show that ignoring electron-electron interactions leads to a ground state energy \( E_{\textrm{He}} = 4 E_\txtH \)
where \( E_\txtH \) is the ground state energy of the hydrogen atom.

\makesubproblem{}{gradQuantum:problemSet7:2b}
Consider the full problem which retains the Coulomb interaction between the electrons, i.e.

\begin{dmath}\label{eqn:gradQuantumProblemSet7Problem2:20}
H
=
\inv{2m} \lr{ \Bp_1^2 + \Bp_2^2 }
-
2 e^2
\inv{4 \pi \epsilon_0 }
\lr{ \inv{r_1} + \inv{r_2} }
+
e^2
\inv{4 \pi \epsilon_0 }
\inv{ \Abs{\Br_1 - \Br_2} }.
\end{dmath}

and consider the variational wavefunction
\begin{dmath}\label{eqn:gradQuantumProblemSet7Problem2:40}
\psi(\Br_1, \Br_2)
=
N
e^{- \inv{a} \lr{ r_1 + r_2 } }.
\end{dmath}

where \( N \) is the normalization constant, and \( a \) is a variational parameter. Determine the variational ground state energy, and minimize with respect to a to find the best estimate for the ground state energy of Helium.
Compare with numerical estimates of the energy.
} % makeproblem

\makeanswer{gradQuantum:problemSet7:2}{
\withproblemsetsParagraph{
\makeSubAnswer{}{gradQuantum:problemSet7:2a}

%\paragraph{Comparing the Hydrogen and Helium ground state energies.}

%Having initially thought that I had to show that \( E_{\textrm{He}} = 4 E_\txtH\), and geting an 8 times difference above, I went looking for mistakes and tried the Helium ground state a different way.
Without the electron-electron interaction term, the Helium Hamiltonian is separable.  Assuming a wave function of the form

\begin{dmath}\label{eqn:gradQuantumProblemSet7Problem2:660}
\psi(r_1, r_2) = \psi_1(r_1) \psi_2(r_2),
\end{dmath}

The Hamiltonian action on this wave function is

\begin{dmath}\label{eqn:gradQuantumProblemSet7Problem2:680}
E \psi_1(r_1) \psi_2(r_2)
=
\lr{ \inv{2m} \Bp_1^2 \psi_1(r_1) } \psi_2(r_2) + \lr{ \inv{2m} \Bp_2^2 \psi_2(r_1) } \psi_1(r_2)
-
2 e^2
\inv{4 \pi \epsilon_0 } \inv{r_1} \psi_1(r_1) \psi_2(r_2)
-
2 e^2
\inv{4 \pi \epsilon_0 } \inv{r_1} \psi_1(r_1) \psi_2(r_2),
\end{dmath}

or
\begin{dmath}\label{eqn:gradQuantumProblemSet7Problem2:700}
E
=
\lr{\inv{\psi_1(r)} \lr{ \inv{2m} \Bp_1^2 \psi_1(r_1) }
-
2 e^2
\inv{4 \pi \epsilon_0 } \inv{r_1} }
+
\lr{\inv{\psi_2(r)} \lr{ \inv{2m} \Bp_2^2 \psi_2(r_2) }
-
2 e^2
\inv{4 \pi \epsilon_0 } \inv{r_2}}.
\end{dmath}

This can be written in separated form as

\begin{equation}\label{eqn:gradQuantumProblemSet7Problem2:720}
\begin{aligned}
E_1 \psi_1(r_1) &= \inv{2m} \Bp_1^2 \psi_1(r_1) - 2 e^2 \inv{4 \pi \epsilon_0 } \inv{r_1} \psi_1(r_1) \\
E_2 \psi_2(r_2) &= \inv{2m} \Bp_2^2 \psi_2(r_2) - 2 e^2 \inv{4 \pi \epsilon_0 } \inv{r_2} \psi_2(r_2) \\
E &= E_1 + E_2.
\end{aligned}
\end{equation}

Observe that each of these separated Hamiltonians have (with \( Z = 2 \) ) the form

\begin{equation}\label{eqn:gradQuantumProblemSet7Problem2:740}
E \psi(r) = \inv{2m} \Bp^2 \psi(r) - Z e^2 \inv{4 \pi \epsilon_0 } \inv{r} \psi(r).
\end{equation}

With \( Z = 1 \) that is precisely the Hamiltonian for the Hydrogen atom.  If the wavefunction for this Hamiltonian is assumed to be \( \psi(r) = e^{-r/a} \), we find

\begin{equation}\label{eqn:gradQuantumProblemSet7Problem2:760}
\frac{\bra{\psi} H \ket{\psi} }{\braket{\psi}{\psi}}
=
\frac{\Hbar^2}{2 m a^2} - \frac{Z e^2}{4 \pi \epsilon_0 a},
\end{equation}

which has its minimum at

\begin{equation}\label{eqn:gradQuantumProblemSet7Problem2:780}
a_{\mathrm{min}} = \frac{a_0}{Z},
\end{equation}

where
\begin{dmath}\label{eqn:gradQuantumProblemSet7Problem2:880}
a_0 = \frac{4 \pi \epsilon_0 \Hbar^2}{m e^2}.
\end{dmath}

The minimum energy is found to be

\begin{equation}\label{eqn:gradQuantumProblemSet7Problem2:800}
E_{\mathrm{min}} = -\inv{2} \frac{e^2 Z^2}{ 4 \pi \epsilon_0 a_0 }.
\end{equation}

With \( Z = 1 \), the Hydrogen ground state energy is

\begin{equation}\label{eqn:gradQuantumProblemSet7Problem2:820}
E_\txtH
= -\inv{2} \frac{e^2}{ 4 \pi \epsilon_0 a_0 },
\end{equation}

a value of about \( -13.6 \si{eV} \).  The Helium ground state energy is

\begin{equation}\label{eqn:gradQuantumProblemSet7Problem2:840}
\begin{aligned}
E_{\textrm{He}}
&= \evalbar{E_1}{Z=2} + \evalbar{E_2}{Z=2} \\
&= -\inv{2} \lr{ 2^2 + 2^2 } \frac{e^2 }{ 4 \pi \epsilon_0 a_0 } \\
&= - 4 \frac{e^2 }{ 4 \pi \epsilon_0 a_0 }.
\end{aligned}
\end{equation}

This is \( E_{\textrm{He}} = 8 E_\txtH\), a value of about \( -109 eV \).

The computations above can be found in \nbref{ps7:heliumAtomGroundStateWithInteraction.nb}.

\makeSubAnswer{}{gradQuantum:problemSet7:2b}

The Laplacian of an exponentially decreasing trial function \( e^{-r/a} \) is

\begin{dmath}\label{eqn:gradQuantumProblemSet7Problem2:340}
\begin{aligned}
\spacegrad^2 e^{-r/a}
&=
\inv{r^2} \PD{r}{} \lr{ r^2 \PD{r}{e^{-r/a}} } \\
&=
\inv{r^2} \PD{r}{} \lr{ -\frac{r^2}{a} e^{-r/a} } \\
&=
-\inv{r^2 a} \lr{ 2 r - \frac{r^2}{a} } e^{-r/a},
\end{aligned}
\end{dmath}

%To calculate \( \Bp_j^2 \psi \) first compute the Laplacian of an exponential
%
%\begin{dmath}\label{eqn:gradQuantumProblemSet7Problem2:60}
%\spacegrad^2 e^{\phi}
%=
%\spacegrad \cdot \spacegrad e^\phi
%=
%\spacegrad \cdot \lr{ e^\phi \spacegrad \phi }
%=
%e^\phi \spacegrad^2 \phi + \lr{ \spacegrad \phi }^2 e^\phi
%=
%\lr{ \spacegrad^2 \phi + \lr{ \spacegrad \phi }^2  } e^\phi.
%\end{dmath}
%
%For \( \phi = -r/a \), we have
%
%\begin{dmath}\label{eqn:gradQuantumProblemSet7Problem2:80}
%\spacegrad \phi
%=
%- \inv{a} \spacegrad \sqrt{ \Bx^2 }
%=
%- \inv{a} \spacegrad \sqrt{ x_k x_k }
%=
%- \inv{a} \inv{2 r} \Be_j ( 2 \partial_j x_k ) x_k
%=
%- \frac{\Bx}{a r},
%\end{dmath}
%
%and
%\begin{dmath}\label{eqn:gradQuantumProblemSet7Problem2:100}
%\spacegrad^2 \phi
%=
%-\inv{a} \spacegrad \cdot \frac{ \Bx}{ r}
%=
%-\inv{a} \lr{
%\inv{r} \spacegrad \cdot \Bx
%+
%\Bx \cdot \spacegrad \inv{ r }
%}
%=
%-\inv{a} \lr{
%\frac{3}{r}
%+
%\Bx \cdot \lr{ -\inv{r^3} \Bx }
%}
%=
%-\inv{a} \lr{
%\frac{3}{r}
%-
%\frac{1}{r}
%}
%=
%-\frac{2}{a r}.
%\end{dmath}
%
or

\begin{dmath}\label{eqn:gradQuantumProblemSet7Problem2:120}
\spacegrad^2 e^{-r/a} = \inv{a} \lr{ \inv{a} -\frac{2}{r} } e^{-r/a}.
\end{dmath}

%%%For the Hydrogen atom (with \( a = a_0 \)), the Hamiltonian action on the unnormalized ground state wavefunction \( \psi = e^{-r/a} \) is
%%%
%%%\begin{dmath}\label{eqn:gradQuantumProblemSet7Problem2:380}
%%%H \psi(r)
%%%=
%%%\frac{\Bp^2}{2m} \psi
%%%- e^2 \inv{4 \pi \epsilon_0 } \inv{r} \psi
%%%=
%%%-\frac{\Hbar^2}{2m} \inv{a} \lr{ \inv{a} -\frac{2}{r} } \psi
%%%- e^2 \inv{4 \pi \epsilon_0 } \inv{r} \psi
%%%=
%%%\lr{ -\frac{\Hbar^2}{2m} \inv{a^2} + \lr{ \frac{\Hbar^2}{m a} - e^2 \inv{4 \pi \epsilon_0 } \inv{r} } } e^{-r/a}.
%%%\end{dmath}
%%%
%%%The hydrogen ground state energy is
%%%\begin{dmath}\label{eqn:gradQuantumProblemSet7Problem2:160}
%%%E_\txtH
%%%%=
%%%%\frac{
%%%%   \bra{ \psi }
%%%%   \lr{ -\frac{\Hbar^2}{2m a} \lr{ \inv{a} -\frac{2}{ r} } - e^2 \inv{4 \pi \epsilon_0 r} }
%%%%   \ket{ \psi }
%%%%}
%%%%{ \braket{ \psi }{ \psi } }
%%%=
%%%   \lr{ \frac{\Hbar^2}{m a} - e^2 \inv{4 \pi \epsilon_0 } }
%%%\frac{
%%%\bra{ \psi } \inv{r} \ket{ \psi }
%%%}
%%%{
%%%   \braket{ \psi }{ \psi }
%%%}
%%%   - \frac{\Hbar^2}{2m a^2} .
%%%\end{dmath}

For Helium without electron-electron interaction the kinetic portion of the Hamiltonian action on this trial function \( \psi = e^{-(r_1 + r_2)/a} \) is

\begin{dmath}\label{eqn:gradQuantumProblemSet7Problem2:140}
H \psi(r_1, r_2)
=
\frac{\Bp_1^2}{2m} \psi
+
\frac{\Bp_2^2}{2m} \psi
- 2 e^2 \inv{4 \pi \epsilon_0 } \lr{ \inv{r_1} + \inv{r_2} } \psi
=
-\frac{\Hbar^2}{2m a} \lr{ \frac{2}{a} -\frac{2}{ r_1}  -\frac{2}{ r_2} } \psi
- 2 e^2 \inv{4 \pi \epsilon_0 } \lr{ \inv{r_1} + \inv{r_2} } \psi
=
\lr{ -\frac{\Hbar^2}{m a^2}
+
\lr{ \frac{\Hbar^2}{m a} - \frac{e^2}{2 \pi \epsilon_0} } \lr{ \inv{r_1} + \inv{r_2} }
}
e^{-(r_1 + r_2)/a}.
\end{dmath}

Now, assuming that \( \psi = e^{-(r_1 + r_2)/a} \) is the unnormalized ground state wavefunction for the Helium atom without electron-electron interaction, that ground state energy is given by

\begin{dmath}\label{eqn:gradQuantumProblemSet7Problem2:260}
E_{\textrm{He}}
%=
%\frac{
%   \bra{ \psi }
%   \lr{
%      -\frac{\Hbar^2}{m a} \lr{ \inv{a} -\frac{1}{ r_1}  -\frac{1}{ r_2} }
%      - 2 e^2 \inv{4 \pi \epsilon_0 } \lr{ \inv{r_1} + \inv{r_2} }
%   }
%   \ket{ \psi }
%}
%{ \braket{ \psi }{ \psi } }
=
   \lr{ \frac{\Hbar^2}{m a} - e^2 \inv{2 \pi \epsilon_0 } }
\frac{
\bra{ \psi } \inv{r_1} + \inv{r_2} \ket{ \psi }
}
{
   \braket{ \psi }{ \psi }
}
   - \frac{\Hbar^2}{m a^2} .
\end{dmath}

%%%\paragraph{Calculating Hydrogen ground state energy}

We'll need a couple helper integrals
%A couple helper integrals
%For the normalization factor we have

\begin{dmath}\label{eqn:gradQuantumProblemSet7Problem2:180}
%\begin{aligned}
%\braket{ \psi }{ \psi }
%&=
4 \pi \int_0^\infty r^2 dr e^{-2 r/a}
=
%&=
%{4 \pi}\frac{a^3}{2^3} \int_0^\infty r^2 dr e^{-r} \\
%&=
%\inv{2} { \pi}{a^3} \int_0^\infty 2r dr e^{-r} \\
%&=
%{\pi}{a^3} \int_0^\infty dr e^{-r} \\
%&=
{\pi}{a^3},
%\end{aligned}
\end{dmath}

and %for the inverse radial expectation we have

\begin{dmath}\label{eqn:gradQuantumProblemSet7Problem2:200}
%\bra{ \psi } \inv{r} \ket{ \psi }
%=
4 \pi \int_0^\infty r dr e^{-2 r/a}
%=
%{4 \pi}\frac{a^2}{2^2} \int_0^\infty r dr e^{-r}
=
{\pi}{a^2}.
\end{dmath}

%so
%
%%%\begin{dmath}\label{eqn:gradQuantumProblemSet7Problem2:220}
%%%E_\txtH
%%%=
%%%\lr{ \frac{\Hbar^2}{m a} - e^2 \inv{4 \pi \epsilon_0 } } \frac{\pi a^2}{\pi a^3}
%%%   - \frac{\Hbar^2}{2m a^2} ,
%%%=
%%%\frac{\Hbar^2}{2 m a^2} - \frac{e^2}{4 \pi \epsilon_0 a }.
%%%\end{dmath}
%%%
%%%A test minimization of this energy using \( a \) as a variational parameter finds
%%%
%%%\begin{dmath}\label{eqn:gradQuantumProblemSet7Problem2:620}
%%%a = \frac{4 \pi \epsilon_0 \Hbar^2}{m e^2},
%%%\end{dmath}
%%%
%%%which is the Bohr-radius as expected.  Substituting that gives
%%%
%%%%\begin{dmath}\label{eqn:gradQuantumProblemSet7Problem2:240}
%%%\boxedEquation{eqn:gradQuantumProblemSet7Problem2:240}{
%%%E_\txtH
%%%=
%%%-
%%%\frac{ m e^4 }{ 32 \pi^2 \epsilon_0^2 \Hbar^2 }
%%%=
%%%-
%%%\inv{2} \frac{e^2}{4 \pi \epsilon_0 a_0}.
%%%}
%%%%\end{dmath}
%%%
%%%Numerically this is about \( -13.6 \si{eV} \).
%
%\paragraph{Calculating Helium ground state energy}
%
To normalize the wavefunction, we need a six-fold integral over both the spatial domains.  With only radial dependence that is

\begin{dmath}\label{eqn:gradQuantumProblemSet7Problem2:280}
%\begin{aligned}
\braket{ \psi }{ \psi }
=
\lr{ 4 \pi}^2
\int_0^\infty r_1^2 dr_1 e^{-2 r_1/a}
\int_0^\infty r_2^2 dr_2 e^{-2 r_2/a}
%&=
%\lr{ 4 \pi}^2 \lr{ \frac{a}{2} }^6
%\lr{ \int_0^\infty r^2 dr e^{-r} }^2 \\
%&=
%2^{4 - 6 + 2}
%\pi^2 a^6 \\
= \pi^2 a^6.
%\end{aligned}
\end{dmath}

We also need the inverse radial expectations.  Calculating the expectation of \( 1/r_1 \) is sufficient, and is

\begin{dmath}\label{eqn:gradQuantumProblemSet7Problem2:300}
\bra{ \psi } \inv{r_1} \ket{ \psi }
=
\lr{ 4 \pi}^2
\int_0^\infty r_1 dr_1 e^{-2 r_1/a}
\int_0^\infty r_2^2 dr_2 e^{-2 r_2/a}
%=
%\lr{ 4 \pi}^2 \lr{ \frac{a}{2} }^5
%\lr{ \int_0^\infty r dr e^{-r} }
%\lr{ \int_0^\infty r^2 dr e^{-r} }
%=
%2^{4 - 5 + 1}
%\pi^2 a^5
= \pi^2 a^5.
\end{dmath}

So, without the electron-electron interaction, the ground state energy is

\begin{dmath}\label{eqn:gradQuantumProblemSet7Problem2:320}
E_{\textrm{He}}
=
   \lr{ \frac{\Hbar^2}{m a} - e^2 \inv{2 \pi \epsilon_0 } }
\frac{ 2 \pi^2 a^5 }
{
   \pi^2 a^6
}
   - \frac{\Hbar^2}{m a^2}
=
   \frac{\Hbar^2}{m a^2} - e^2 \inv{\pi \epsilon_0 a }.
\end{dmath}

%%%It appears that the value of \( a \) that minimizes this energy is not the bohr radius for this wave function, so the assumption that \( \psi(r_1, r_2) = e^{-(r_1 + r_2)/a_0} \) was an eigenfunction for the Hamiltonian was incorrect.  Performing the variation, we find that the minimum energy is found at
%%%
%%%\begin{dmath}\label{eqn:gradQuantumProblemSet7Problem2:640}
%%%a = \inv{2} \frac{4 \pi \epsilon_0 \Hbar^2}{m e^2},
%%%\end{dmath}
%%%
%%%which is half the Bohr-radius.  Substituting that into the energy above gives
%%%
%%%\boxedEquation{eqn:gradQuantumProblemSet7Problem2:400}{
%%%E_{\textrm{He}}
%%%=
%%%-\frac{e^2 m}{4 \pi^2 \epsilon_0^2 \Hbar^2}
%%%=
%%%-
%%%\frac{e^2}{\pi \epsilon_0 a_0}.
%%%}
%%%
%This is \( E_{\textrm{He}} = 8 E_\txtH\), a value of about \( -110 eV \).

To evaluate the interaction term, a Fourier transform representation of that inverse radial distance can be employed

\begin{dmath}\label{eqn:gradQuantumProblemSet7Problem2:900}
\inv{\Abs{\Br}}
= \inv{2 \pi^2} \int d^3 k \frac{e^{i \Bk \cdot \Br}}{\Bk^2},
\end{dmath}

where this is understood to be the \( \epsilon \rightarrow 0 \) limit of

\begin{dmath}\label{eqn:gradQuantumProblemSet7Problem2:420}
\inv{\Abs{\Br}} e^{-\epsilon \Abs{\Br}}
= \inv{2 \pi^2} \int d^3 k \frac{e^{i \Bk \cdot \Br}}{\Bk^2 + \epsilon^2}.
\end{dmath}

See \citep{byron1992mca} for a demonstration of this identity, and the contour used to evaluate the RHS of \cref{eqn:gradQuantumProblemSet7Problem2:420}.  Employing this inverse radial representation, the \( r_1 \) and \( r_2 \) contributions to the interaction can be decoupled

\begin{dmath}\label{eqn:gradQuantumProblemSet7Problem2:440}
\frac{e^2}{4 \pi \epsilon_0} \bra{\psi} \inv{\Abs{\Br_1 - \Br_2}} \ket{\psi}
=
\frac{e^2}{4 \pi \epsilon_0}
\int 2 \pi dr_1 d\theta_1 r_1^2 \sin(\theta_1)
\int 2 \pi dr_2 d\theta_2 r_2^2 \sin(\theta_2)
e^{ -2(r_1 + r_2)/a}
\inv{2 \pi^2}
\int d^3 k \frac{e^{i \Bk \cdot \lr{ \Br_1 - \Br_2} }}{\Bk^2}
=
\frac{e^2}{4 \pi \epsilon_0}
\inv{2 \pi^2}
\int d^3 k \inv{\Bk^2}
\int 2 \pi dr_1 d\theta_1 r_1^2 \sin(\theta_1) e^{-2r_1/a + i \Bk \cdot \Br_1}
\int 2 \pi dr_2 d\theta_2 r_2^2 \sin(\theta_2) e^{-2r_2/a - i \Bk \cdot \Br_2}.
\end{dmath}

The spatial domain integrals can now be evaluated separately.  With a coordinate system picked so that \( \Bk = \pm k \zcap \), that gives

\begin{dmath}\label{eqn:gradQuantumProblemSet7Problem2:460}
\begin{aligned}
2 \pi \int dr d\theta r^2 \sin(\theta) e^{-2r/a + i \Bk \cdot \Br}
&=
2 \pi \int_0^\infty dr r^2 e^{-2r/a}
\int_0^\pi d\theta
\frac{d}{d\theta} (-\cos(\theta))
e^{\pm i k r \cos\theta} \\
&=
2 \pi \int_0^\infty dr r^2 e^{-2r/a}
\int_{-1}^1 du
e^{\mp i k r u} \\
&=
2 \pi \int_0^\infty dr r^2 e^{-2r/a}
\frac{e^{\mp i k r} - e^{\pm i k r} }{ \mp i k r } \\
&=
2 \pi \frac{2}{k} \int_0^\infty dr r e^{-2r/a} \sin( k r ) \\
&=
2 \pi \frac{2}{k} \int_0^\infty dr r e^{-2r/a} \sin( k r ) \\
&=
\frac{16 \pi a^3}{\lr{1 + a^2 k^2 }^2}.
\end{aligned}
\end{dmath}

We see that the specific orientation used to evaluate the integral does not matter, so we have

\begin{dmath}\label{eqn:gradQuantumProblemSet7Problem2:480}
\frac{e^2}{4 \pi \epsilon_0} \bra{\psi} \inv{\Abs{\Br_1 - \Br_2}} \ket{\psi}
=
\frac{e^2}{4 \pi \epsilon_0}
\inv{2 \pi^2}
\int d^3 k \inv{\Bk^2}
\frac{(16 \pi a^3)^2}{\lr{1 + a^2 k^2 }^4}
=
\frac{e^2}{4 \pi \epsilon_0}
\inv{2 \pi^2}
(4\pi)
16^2 \pi^2 a^6
\int dk k^2 \inv{\Bk^2}
\inv{\lr{1 + a^2 k^2 }^4}
%=
%\frac{32 e^2 a^6}{\epsilon_0}
%\int dk
%\inv{\lr{1 + a^2 k^2 }^4}
%=
%\frac{32 e^2 a^6}{\epsilon_0}
%\frac{5 \pi}{32 a}
=
\frac{5 \pi e^2 a^5}{32 \epsilon_0}.
\end{dmath}

Rescaling with the normalization factor gives

\begin{dmath}\label{eqn:gradQuantumProblemSet7Problem2:500}
\frac{e^2}{4 \pi \epsilon_0} \bra{\psi} \inv{\Abs{\Br_1 - \Br_2}} \ket{\psi}/\braket{\psi}{\psi}
=
\frac{5 \pi e^2 a^5}{32 \epsilon_0 } \inv{\pi^2 a^6}
=
\frac{5 e^2 }{32 \pi \epsilon_0 a}.
\end{dmath}

Adding this electron-electron interaction to the Helium ground energy calculated in
\cref{eqn:gradQuantumProblemSet7Problem2:320} gives

\begin{dmath}\label{eqn:gradQuantumProblemSet7Problem2:520}
E_{\textrm{He}}
=
\frac{\Hbar^2}{m a^2} -\frac{27}{32} e^2 \inv{\pi \epsilon_0 a }.
\end{dmath}

For the minimum we want to solve

\begin{dmath}\label{eqn:gradQuantumProblemSet7Problem2:540}
0
=
\PD{a}{E}
=
-2 \frac{\Hbar^2}{m a^3} + \frac{27}{32} e^2 \inv{\pi \epsilon_0 a^2 },
\end{dmath}

which has the minimum at

\begin{dmath}\label{eqn:gradQuantumProblemSet7Problem2:560}
a = - \frac{64 \Hbar^2 \pi \epsilon_0}{27 m e^2}.
\end{dmath}

Note that \( m \) should really be treated as the reduced mass of the electron, but doing so isn't numerically significant.  The final result for the variational ground state energy is

%Noting that \( a_0 = \Hbar^2/m e^2 \),
%the ground state energy, after substituting this value of \( a \) is

%\begin{dmath}\label{eqn:gradQuantumProblemSet7Problem2:580}
\boxedEquation{eqn:gradQuantumProblemSet7Problem2:600}{
E_{\textrm{He}}
=
- \lr{\frac{27}{16}}^2 \frac{e^2}{4 \pi \epsilon_0 a_0} \approx -77.5 \si{eV}
}
%\end{dmath}

In atomic units this is

\begin{dmath}\label{eqn:gradQuantumProblemSet7Problem2:860}
E_{\textrm{He}}
=
- \lr{\frac{27}{16}}^2 \frac{e^2}{a_0} \approx -2.848 \frac{e^2}{a_0}.
\end{dmath}

In \citep{desai2009quantum} the measured value is stated as \( -2.90 \,\ifrac{e^2}{a_0} \).  Table 1 of \citep{aznabayev2015energy}, which lists high precision calculations of all the energy levels, has \( -2.903724 \,\ifrac{e^2}{a_0} \) for the 1s energy level.  The calculated value of \cref{eqn:gradQuantumProblemSet7Problem2:600} is about 2 \% off the mark.

See \nbref{ps7:heliumAtomGroundStateWithInteraction.nb}, for a complete end to end verification of the calculations above.
}
}

         %
% Copyright � 2015 Peeter Joot.  All Rights Reserved.
% Licenced as described in the file LICENSE under the root directory of this GIT repository.
%
% desai 24.3
\makeoproblem{Harmonic oscillator variation.}{gradQuantum:problemSet7:3}{\citep{desai2009quantum} pr. 24.3}
%\makeproblem{Harmonic oscillator variation}{gradQuantum:problemSet7:3}
{
\index{harmonic oscillator!variational method}
%\makesubproblem{}{gradQuantum:problemSet7:3a}
%
Consider a 1D harmonic oscillator with an unnormalized trial wavefunction \( \psi_v(x) = e^{-\beta \Abs{x}} \).
Minimize the ground state energy with respect to \( \beta \), thus obtaining the optimal \( \beta \) as well as the variational ground state energy.
Compare with the exact result.
Note that you need to be careful evaluating derivatives since the wavefunction has a `cusp' at \( x = 0\).
} % makeproblem
%
\makeanswer{gradQuantum:problemSet7:3}{
\withproblemsetsParagraph{
%\makeSubAnswer{}{gradQuantum:problemSet7:3a}
%
In order to make the derivatives of the trial function better behaved at the origin, we can treat it as a distribution, writing
%
\begin{equation}\label{eqn:gradQuantumProblemSet7Problem3:20}
\psi(x) = \Theta(x) e^{-\beta x} + \Theta(-x) e^{\beta x},
\end{dmath}
%
This has a derivative
%
\begin{dmath}\label{eqn:gradQuantumProblemSet7Problem3:40}
\psi'(x)
=
\delta(x) e^{-\beta x} - \delta(-x) e^{\beta x}
+ \beta \lr{ -\Theta(x) e^{-\beta x} + \Theta(-x) e^{\beta x} }
=
-2 \delta(x) \sinh( \beta x )
+ \beta \lr{ -\Theta(x) e^{-\beta x} + \Theta(-x) e^{\beta x} }.
\end{dmath}
%
Using \( \delta'(x) = - \delta(x)/x \), the second derivative is
%
\begin{dmath}\label{eqn:gradQuantumProblemSet7Problem3:60}
\psi''(x)
=
+2 \delta(x) \sinh( \beta x )/x
-2 \beta \delta(x) \cosh( \beta x )
+ \beta \lr{ -\delta(x) e^{-\beta x} - \delta(-x) e^{\beta x} }
+ \beta^2 \lr{ \Theta(x) e^{-\beta x} + \Theta(-x) e^{\beta x} }
=
+2 \delta(x) \beta \lr{ \frac{\sinh( \beta x )}{\beta x} - \cosh( \beta x ) }
- 2 \beta \delta(x) \cosh( \beta x )
+ \beta^2 e^{-\beta \Abs{x} }.
\end{dmath}
%
Because
%
\begin{equation}\label{eqn:gradQuantumProblemSet7Problem3:80}
\int \delta(x) \cosh( \beta x ) f(x) dx = f(0),
\end{dmath}
%
and
%
\begin{dmath}\label{eqn:gradQuantumProblemSet7Problem3:100}
\int \delta(x) \lr{ \frac{\sinh( \beta x )}{\beta x} - \cosh( \beta x ) } f(x) dx
= \lr{ 1 - 1 } f(0)
= 0,
\end{dmath}
%
this second derivative can be simplified to
\begin{dmath}\label{eqn:gradQuantumProblemSet7Problem3:120}
\psi''(x)
=
- 2 \beta \delta(x) + \beta^2 e^{-\beta \Abs{x} }.
\end{dmath}
%
This has the \( \beta^2 \psi(x) \) value that we expect at points away from the origin.  All the expectations can now be computed.  The normalization is
%
\begin{dmath}\label{eqn:gradQuantumProblemSet7Problem3:140}
\braket{\psi}{\psi}
=
2 \int_0^\infty e^{-2 \beta x} dx
=
2 \evalrange{\frac{e^{-2 \beta x}}{-2 \beta}}{0}{\infty}
=
\inv{\beta}.
\end{dmath}
%
Observe that
%
\begin{dmath}\label{eqn:gradQuantumProblemSet7Problem3:160}
\frac{d^2}{d\beta^2} \int_0^\infty e^{-2 \beta x} dx
=
(-2)^2 \int_0^\infty x^2 e^{-2 \beta x} dx
=
\frac{d}{d\beta} \lr{ -\inv{2 \beta^2}}
=
\frac{1}{\beta^3},
\end{dmath}
%
so
\begin{equation}\label{eqn:gradQuantumProblemSet7Problem3:180}
\int_0^\infty x^2 e^{-2 \beta x} dx = \frac{1}{4 \beta^3},
\end{dmath}
%
a result we will need later.  The kinetic portion of the energy expectation is
%
\begin{dmath}\label{eqn:gradQuantumProblemSet7Problem3:200}
\bra{\psi} \frac{p^2}{2m} \ket{\psi}
=
-\frac{\Hbar^2}{2m}
\int_{-\infty}^\infty e^{-\beta x} \lr{ - 2 \beta \delta(x) + \beta^2 e^{-\beta \Abs{x} } }
=
-\frac{\Hbar^2}{2m} \beta^2 \inv{\beta} -\frac{\Hbar^2}{2m} \lr{ - 2 \beta }
= \frac{\Hbar^2}{2m} \beta.
\end{dmath}
%
The potential portion of the energy expectation is
\begin{dmath}\label{eqn:gradQuantumProblemSet7Problem3:220}
\bra{\psi} \inv{2} m \omega^2 x^2 \ket{\psi}
=
m \omega^2 \int_0^\infty x^2 e^{-2 \beta x} dx
=
m \omega^2 \frac{1}{4 \beta^3}.
\end{dmath}
%
Adding things up we have
\begin{dmath}\label{eqn:gradQuantumProblemSet7Problem3:240}
\overbar{E}(\beta)
=
\frac{\bra{\psi} H \ket{\psi}}{\braket{\psi}{\psi}}
=
\frac{\frac{\Hbar^2}{2m} \beta
+ \frac{ m \omega^2}{4 \beta^3}}{\inv{\beta}}
=
\frac{\Hbar^2}{2m} \beta^2
+ \frac{ m \omega^2}{4 \beta^2}
=
\frac{\Hbar \omega}{2}
\lr{
\frac{\Hbar}{m \omega} \beta^2 +
\frac{ m \omega}{2 \Hbar \beta^2}
}
=
\frac{\Hbar \omega}{2}
\lr{
x_0^2 \beta^2 + \inv{2 x_0^2 \beta^2}
}.
\end{dmath}
%
Minimizing gives
%
\begin{dmath}\label{eqn:gradQuantumProblemSet7Problem3:260}
0
=
\frac{d}{d\beta}
\lr{
x_0^2 \beta^2 + \inv{2 x_0^2 \beta^2}
}
=
2 x_0^2 \beta - \inv{x_0^2 \beta^3},
\end{dmath}
%
or
%
\begin{equation}\label{eqn:gradQuantumProblemSet7Problem3:280}
\beta^4 = \inv{2 x_0^4},
\end{dmath}
%
which gives
\boxedEquation{eqn:gradQuantumProblemSet7Problem3:300}{
\beta = \inv{2^{1/4} x_0}.
}

The energy at this value of \( \beta \) is
%
\begin{dmath}\label{eqn:gradQuantumProblemSet7Problem3:320}
\overbar{E}_{\textrm{min}}
=
\frac{\Hbar \omega}{2}
\lr{
x_0^2 \inv{\sqrt{2} x_0^2} + \frac{ \sqrt{2} x_0^2}{2 x_0^2 }
}
=
\frac{\Hbar \omega}{2}
\frac{2}{ \sqrt{2} },
\end{dmath}
%
or
\boxedEquation{eqn:gradQuantumProblemSet7Problem3:340}{
\overbar{E}_{\textrm{min}}
=
\frac{\Hbar \omega}{2} \sqrt{2} > \Hbar \omega \lr{ 0 + \inv{2} }.
}
We find that the trial function that minimizes the average energy is
%
\begin{equation}\label{eqn:gradQuantumProblemSet7Problem3:360}
\psi(x) = 2^{1/4} x_0 \exp\lr{ -2^{-1/4} \Abs{x}/x_0 },
\end{dmath}
%
with an average energy that is \( 1.41 \times \) the actual ground state energy for the harmonic oscillator.
}
}

         % p5.21
         %
% Copyright � 2015 Peeter Joot.  All Rights Reserved.
% Licenced as described in the file LICENSE under the root directory of this GIT repository.
%
%\input{../blogpost.tex}
%\renewcommand{\basename}{absolutePotentialVariation}
%\renewcommand{\dirname}{notes/phy1520/}
%%\newcommand{\dateintitle}{}
%%\newcommand{\keywords}{}
%
%\input{../peeter_prologue_print2.tex}
%
%\usepackage{peeters_layout_exercise}
%\usepackage{peeters_braket}
%\usepackage{peeters_figures}
%
%\beginArtNoToc
%
%\generatetitle{Energy estimate for an absolute value potential}
%%\chapter{Energy estimate for an absolute value potential}
%%\label{chap:absolutePotentialVariation}
%
%Here's a simple problem, a lot like the problem set 6 variational calculation.
%
\makeoproblem{Energy of absolute value potential.}{problem:absolutePotentialVariation:1}{\citep{sakurai2014modern} pr. 5.21}{
\index{variational method}

Estimate the lowest eigenvalue \( \lambda \) of the differential equation
%
\begin{dmath}\label{eqn:absolutePotentialVariation:20}
\frac{d^2}{dx^2}\psi + \lr{ \lambda - \Abs{x} } \psi = 0.
\end{dmath}
%
Using \( \alpha \) variation with the trial function
%
\begin{dmath}\label{eqn:absolutePotentialVariation:40}
\psi =
\left\{
\begin{array}{l l}
c(\alpha - \Abs{x}) & \quad \mbox{\(\Abs{x} < \alpha \) } \\
0 & \quad \mbox{\(\Abs{x} > \alpha \) }
\end{array}
\right.
\end{dmath}
%
} % problem
%
\makeanswer{problem:absolutePotentialVariation:1}{
First rewrite the differential equation in a Hamiltonian like fashion
%
\begin{equation}\label{eqn:absolutePotentialVariation:60}
H \psi = -\frac{d^2}{dx^2}\psi + \Abs{x} \psi = \lambda \psi.
\end{equation}
%
We need the derivatives of the trial distribution.  The first derivative is
%
\begin{dmath}\label{eqn:absolutePotentialVariation:80}
\frac{d}{dx} \psi
=
-c \frac{d}{dx} \Abs{x}
=
-c \frac{d}{dx} \lr{ x \theta(x) - x \theta(-x) }
=
-c \lr{
 \theta(x) - \theta(-x)
+
x \delta(x) + x \delta(-x)
}
=
-c \lr{
 \theta(x) - \theta(-x)
+
2 x \delta(x)
}.
\end{dmath}
%
The second derivative is
\begin{dmath}\label{eqn:absolutePotentialVariation:100}
\frac{d^2}{dx^2} \psi
=
-c \frac{d}{dx} \lr{
 \theta(x) - \theta(-x)
+
2 x \delta(x)
}
=
-c \lr{
 \delta(x) + \delta(-x)
+
2 \delta(x)
+
2 x \delta'(x)
}
=
-c \lr{
4 \delta(x)
+
2 x \frac{-\delta(x) }{x}
}
=
-2 c \delta(x).
\end{dmath}
%
This gives
%
\begin{dmath}\label{eqn:absolutePotentialVariation:120}
H \psi = -2 c \delta(x) + \Abs{x} c \lr{ \alpha - \Abs{x} }.
\end{dmath}
%
We are now set to compute some of the inner products.  The normalization is the simplest
%
\begin{dmath}\label{eqn:absolutePotentialVariation:140}
\begin{aligned}
\braket{\psi}{\psi}
&= c^2 \int_{-\alpha}^\alpha ( \alpha - \Abs{x} )^2 dx \\
&= 2 c^2 \int_{0}^\alpha ( x - \alpha )^2 dx \\
&= 2 c^2 \int_{-\alpha}^0 u^2 du \\
&= 2 c^2 \lr{ -\frac{(-\alpha)^3}{3} } \\
&= \frac{2}{3} c^2 \alpha^3.
\end{aligned}
\end{dmath}
%
For the energy
\begin{dmath}\label{eqn:absolutePotentialVariation:160}
\begin{aligned}
\braket{\psi}{H \psi}
&=
c^2 \int dx \lr{ \alpha - \Abs{x} } \lr{ -2 \delta(x) + \Abs{x} \lr{ \alpha - \Abs{x} } } \\
&=
c^2 \lr{ - 2 \alpha + \int_{-\alpha}^\alpha dx \lr{ \alpha - \Abs{x} }^2 \Abs{x} } \\
&=
c^2 \lr{ - 2 \alpha + 2 \int_{-\alpha}^0 du u^2 \lr{ u + \alpha } } \\
&=
c^2 \lr{ - 2 \alpha + 2 \evalrange{\lr{ \frac{u^4}{4} + \alpha \frac{u^3}{3} }}{-\alpha}{0} } \\
&=
c^2 \lr{ - 2 \alpha - 2 \lr{ \frac{\alpha^4}{4} - \frac{\alpha^4}{3} } } \\
&=
c^2 \lr{ - 2 \alpha + \inv{6} \alpha^4 }.
\end{aligned}
\end{dmath}
%
The energy estimate is
%
\begin{dmath}\label{eqn:absolutePotentialVariation:180}
\overbar{E}
=
\frac{\braket{\psi}{H \psi}}{\braket{\psi}{\psi}}
=
\frac{ - 2 \alpha + \inv{6} \alpha^4 }{ \frac{2}{3} \alpha^3}
=
- \frac{3}{\alpha^2} + \inv{4} \alpha.
\end{dmath}
%
This has its minimum at
\begin{dmath}\label{eqn:absolutePotentialVariation:200}
0 = -\frac{6}{\alpha^3} + \inv{4},
\end{dmath}
%
or
\begin{dmath}\label{eqn:absolutePotentialVariation:220}
\alpha = 2 \times 3^{1/3}.
\end{dmath}
%
Back substitution into the energy gives
%
\begin{dmath}\label{eqn:absolutePotentialVariation:240}
\overbar{E}
=
- \frac{3}{4 \times 3^{2/3}} + \inv{2} 3^{1/3}
= \frac{3^{4/3}}{4}
\approx 1.08.
\end{dmath}
%
The problem says the exact answer is 1.019, so the variation gets within 6 \%.
%
} % answer

%\EndArticle

         %
% Copyright � 2015 Peeter Joot.  All Rights Reserved.
% Licenced as described in the file LICENSE under the root directory of this GIT repository.
%
\makeoproblem{Anharmonic oscillator.}{gradQuantum:problemSet8:1}{2015 ps8 p1}{
\index{anharmonic oscillator}

Consider a quantum particle in the ground state of a 1D anharmonic oscillator potential
%
\begin{equation}\label{eqn:gradQuantumProblemSet8Problem1:20}
V(x) = \inv{2} m \omega^2 x^2 + \lambda x^4 = V_0 + \lambda V'.
\end{equation}
%
Compute the first and second order energy shift of this oscillator perturbatively in \( \lambda \).
%
%\makesubproblem{}{gradQuantum:problemSet8:1a}
} % makeproblem
%
\makeanswer{gradQuantum:problemSet8:1}{
\withproblemsetsParagraph{

Using \nbref{ps8:harmonicOscillatorRaiseAndLoweringOperators.nb} the action of the potential on the ground state is
%
\begin{equation}\label{eqn:gradQuantumProblemSet8Problem1:40}
\begin{aligned}
V' \ket{0}
&= x^4 \ket{0}
\\ &=
x_0^4 \lr{ \frac{3}{4} \ket{0}
+ \frac{3}{\sqrt{2}} \ket{2}
+ \sqrt{\frac{3}{2}} \ket{4}
}.
\end{aligned}
\end{equation}
%
That allows us to compute the first order energy shift
%
\begin{equation}\label{eqn:gradQuantumProblemSet8Problem1:60}
\begin{aligned}
\Delta^{(1)}
&= \bra{0} V' \ket{0}
\\ &= \frac{3}{4} x_0^4.
\end{aligned}
\end{equation}
%
Writing the perturbed state as
%
\begin{equation}\label{eqn:gradQuantumProblemSet8Problem1:80}
\ket{n} = \ket{0} + \lambda \ket{0}' + \lambda^2 \ket{0}'' + \cdots,
\end{equation}
%
the first order perturbation \( \ket{0}' \) of the ground state is
%
\begin{equation}\label{eqn:gradQuantumProblemSet8Problem1:100}
\begin{aligned}
\ket{0}'
&= \sum_{m \ne 0} \frac{\ket{m}\bra{m} x^4 \ket{0} }{\Hbar \omega/2 - \Hbar \omega( m + 1/2 ) }
\\ &=
- \frac{ x_0^4}{\Hbar \omega} \sum_{m \ne 0} \frac{\ket{m}\bra{m} }{m}
\lr{ \frac{3}{4} \ket{0}
+ \frac{3}{\sqrt{2}} \ket{2}
+ \sqrt{\frac{3}{2}} \ket{4}
}
\\ &=
- \frac{ x_0^4}{\Hbar \omega}
\lr{
\inv{2}
  \frac{3}{\sqrt{2}} \ket{2}
+
\inv{4}\sqrt{\frac{3}{2}} \ket{4}
}.
\end{aligned}
\end{equation}
%
The second order energy shift can now be calculated, and is
%
\begin{equation}\label{eqn:gradQuantumProblemSet8Problem1:120}
\begin{aligned}
\Delta^{(2)}
&=
\bra{0} V' \ket{0}'
\\ &=
- \frac{ x_0^8}{\Hbar \omega}
\lr{ \frac{3}{4} \bra{0}
+ \frac{3}{\sqrt{2}} \bra{2}
+ \sqrt{\frac{3}{2}} \bra{4}
}
\lr{
  \inv{2}\frac{3}{\sqrt{2}} \ket{2}
+ \inv{4}\sqrt{\frac{3}{2}} \ket{4}
}
\\ &=
- \frac{ x_0^8}{\Hbar \omega} \frac{21}{8}.
\end{aligned}
\end{equation}
%
To second order the total energy shift is
%\begin{dmath}\label{eqn:gradQuantumProblemSet8Problem1:140}
\boxedEquation{eqn:gradQuantumProblemSet8Problem1:160}{
\Delta
= \frac{3}{4} \lambda x_0^4
- \frac{21 x_0^8 \lambda^2}{8 \Hbar \omega}.
}
%\end{dmath}
%
%\makeSubAnswer{}{gradQuantum:problemSet8:1a}
}
}

         %
% Copyright � 2015 Peeter Joot.  All Rights Reserved.
% Licenced as described in the file LICENSE under the root directory of this GIT repository.
%
\makeoproblem{Quadrupolar potential.}{gradQuantum:problemSet8:2}{2015 ps8 p2}{
\index{quadrupolar potential}

Consider a p-orbital electron of hydrogen with \( \ket{ n,l = 1, m } \), with \( m = 0, \pm 1 \), subject to an external potential
%
\begin{equation}\label{eqn:gradQuantumProblemSet8Problem2:20}
V(x, y, z) = \lambda(x^2 - y^2),
\end{dmath}
%
with \( \lambda \) being a constant. For fixed \( n \), obtain the correct eigenstates which diagonalize
the perturbation, without worrying about doing radial integrals explicitly. Show that the three-fold degeneracy of the
p-orbital is completely broken by the perturbation to linear order in \( \lambda \).
%
%\makesubproblem{}{gradQuantum:problemSet8:2a}
} % makeproblem
%
\makeanswer{gradQuantum:problemSet8:2}{
\withproblemsetsParagraph{
%\makeSubAnswer{}{gradQuantum:problemSet8:2a}
%
The potential in spherical coordinates is
%
\begin{equation}\label{eqn:gradQuantumProblemSet8Problem2:40}
V = \lambda r^2 \sin^2\theta \lr{ \cos^2\phi - \sin^2\phi } = \lambda r^2 \sin^2\theta \cos(2 \phi).
\end{dmath}
%
The p-orbital wave functions are
%
\begin{equation}\label{eqn:gradQuantumProblemSet8Problem2:60}
\psi_{n1m}(r, \theta, \phi) = R_n(r) Y_{1,m}(\theta, \phi),
\end{dmath}
%
where
\begin{equation}\label{eqn:gradQuantumProblemSet8Problem2:80}
\begin{aligned}
Y_{1,1}(\theta, \phi) &= -\frac{1}{2} \sqrt{\frac{3}{2 \pi }} e^{i \phi } \sin  \theta, \\
Y_{1,0}(\theta, \phi) &= \frac{1}{2} \sqrt{\frac{3}{\pi }} \cos  \theta, \\
Y_{1,-1}(\theta, \phi) &= \frac{1}{2} \sqrt{\frac{3}{2 \pi }} e^{-i \phi } \sin  \theta.
\end{aligned}
\end{equation}
%
That is enough information to construct the matrix element of the perturbing potential with respect to these states.  Those are
%
\begin{equation}\label{eqn:gradQuantumProblemSet8Problem2:100}
\begin{aligned}
&\bra{n' 1 m'} V \ket{n 1 m} \\
&=
\int_0^\infty r^2 dr \int_0^\pi \sin\theta d\theta 
\,\times \\ &\qquad
\int_0^{2 \pi} d\phi R_n(r) Y^\conj_{1, m'}(\theta, \phi) \lambda r^2 \sin^2\theta \cos(2 \phi) R_n(r) Y_{1, m}(\theta, \phi) \\
&=
\lambda \int_0^\infty r^4 dr R^2_n(r)
\int_0^\pi \sin^3\theta \cos(2\phi) d\theta \int_0^{2 \pi} d\phi Y^\conj_{1, m'}(\theta, \phi) Y_{1, m}(\theta, \phi) \\
&=
\lambda \int_0^\infty r^4 dr R^2_n(r)
\begin{bmatrix}
0 & 0 & -\frac{2}{5} \\
0 & 0 & 0 \\
-\frac{2}{5} & 0 & 0 \\
\end{bmatrix}.
\end{aligned}
\end{equation}
%
See
\nbref{ps8:quadrupolarPotentialPorbitalSplitting.nb}
for a computation of the matrix.
It has eigenvalues
%
\begin{dmath}\label{eqn:gradQuantumProblemSet8Problem2:120}
\lambda \int_0^\infty r^4 dr R^2_n(r)
\setlr{
 -\frac{2}{5},
 \frac{2}{5},
 0
}.
\end{dmath}
%
We see that the radial factor \( R_n(r) \) of these wave function provides only a constant adjustment to the energy levels splitting that breaks the degeneracy.   That degeneracy is completely broken by this perturbation.

The respective eigenvectors for this matrix are
\begin{dmath}\label{eqn:gradQuantumProblemSet8Problem2:140}
\setlr{
\inv{\sqrt{2}}
\begin{bmatrix}
1 \\
0 \\
1
\end{bmatrix},
\inv{\sqrt{2}}
\begin{bmatrix}
1 \\
0 \\
-1
\end{bmatrix},
\begin{bmatrix}
0 \\
1 \\
0
\end{bmatrix}
},
\end{dmath}
%
so the wave functions, say \( \setlr{\psi_{n,-1},\psi_{n,1}, \psi_{n,0}} \), that diagonalize this perturbation potential are
%
\begin{dmath}\label{eqn:gradQuantumProblemSet8Problem2:160}
\begin{aligned}
\psi_{n,-1}(r, \theta, \phi) &= \frac{R_n(r)}{\sqrt{2}} \lr{ Y_{-1,1}(\theta, \phi) + Y_{1,1}(\theta, \phi) } = \frac{R_n(r)}{2 i} \sqrt{\frac{3}{\pi }} \sin\phi \sin\theta, \\
\psi_{n,1}(r, \theta, \phi) &= \frac{R_n(r)}{\sqrt{2}} \lr{ Y_{-1,1}(\theta, \phi) - Y_{1,1}(\theta, \phi) } = -\frac{R_n(r)}{2} \sqrt{\frac{3}{\pi }} \cos\phi \sin\theta, \\
\psi_{n,0}(r, \theta, \phi) &= R_n(r) Y_{1,0}(\theta, \phi) = \frac{R_n(r)}{2} \sqrt{\frac{3}{\pi }} \cos\theta.
\end{aligned}
\end{dmath}
%
It is natural to adjust the phases above, forming an alternate basis
%
\begin{dmath}\label{eqn:gradQuantumProblemSet8Problem2:180}
\setlr{-\psi_{n,1}, i\psi_{n,-1}, \psi_{n,0}}
=
\frac{R_n(r)}{2} \sqrt{\frac{3}{\pi }} \rcap,
\end{dmath}
%
where \( \rcap = \setlr{ \sin\theta \cos\phi, \sin\theta \sin\phi, \cos\theta } \), the set of components of the unit vector parameterized by \( \theta, \phi \).
In this basis that level splitting is \( \lambda \int_0^\infty r^4 dr R^2_n(r) \setlr{ \frac{2}{5}, -\frac{2}{5}, 0 } \) respectively.
}
}

         %
% Copyright � 2015 Peeter Joot.  All Rights Reserved.
% Licenced as described in the file LICENSE under the root directory of this GIT repository.
%
\makeoproblem{Harmonic oscillator.}{gradQuantum:problemSet8:3}{2015 ps8 p3}{
\index{harmonic oscillator!anharmonic perturbation}

Consider a 2D harmonic oscillator with
%
\begin{dmath}\label{eqn:gradQuantumProblemSet8Problem3:20}
H =
\frac{p_x^2}{2m}
+\frac{p_y^2}{2m}
+ \inv{2} m \omega^2 \lr{ x^2 + y^2 }
\end{dmath}

Turn on an anharmonic perturbation
%
\begin{dmath}\label{eqn:gradQuantumProblemSet8Problem3:40}
V = \lambda g_1 \frac{m^2 \omega^3}{\Hbar} \lr{ x^4 + y^4 } + \lambda^2 g_2 m \omega^2 x y,
\end{dmath}
%
Note that the potentials have been altered from the original problem statement to have dimensions of energy with dimensionless scale factors \( g_1, g_2, \lambda \).

\makesubproblem{}{gradQuantum:problemSet8:3a}
%
Derive the equations for the energy shifts and perturbed states for a second order perturbing potential of the form above.

\makesubproblem{}{gradQuantum:problemSet8:3b}
%
Find the perturbed eigenstate and the corresponding energy shifts up to \( O(\lambda^2) \) for the ground state.  Ignore terms of \( O(\lambda^3) \).

\makesubproblem{}{gradQuantum:problemSet8:3c}
%
Do the same for the first two degenerate states.

} % makeproblem
%
\makeanswer{gradQuantum:problemSet8:3}{
\withproblemsetsParagraph{
\makeSubAnswer{}{gradQuantum:problemSet8:3a}
%
With a \( \lambda^2 \) perturbation we have to step back and revisit the derivation of the energy level and perturbed state formulas.  Given
%
\begin{dmath}\label{eqn:gradQuantumProblemSet8Problem3:60}
H = H_0 + \lambda V_1 + \lambda^2 V_2,
\end{dmath}
%
with known solution \( H_0 \ket{n^{(0)}} = E^{(0)} \ket{n^{(0)}} \), we seek the a power series representation of the perturbed ket and an energy shift \( \Delta \)
%
\begin{dmath}\label{eqn:gradQuantumProblemSet8Problem3:80}
\ket{n} = \ket{n_0} + \lambda \ket{n_1} + \lambda^2 \ket{n_2} + \cdots
\end{dmath}
\begin{dmath}\label{eqn:gradQuantumProblemSet8Problem3:100}
\Delta = \lambda \Delta^{(1)} + \lambda^2 \Delta^{(2)} + \cdots
\end{dmath}

where
%
\begin{dmath}\label{eqn:gradQuantumProblemSet8Problem3:120}
H \ket{n} = \lr{ E^{(0)} + \Delta } \ket{n}.
\end{dmath}
%
We can assume that the we have the same sort of representation of the perturbed state
%
\begin{dmath}\label{eqn:gradQuantumProblemSet8Problem3:140}
\ket{n} = \ket{n^{(0)}} + \frac{\overbar{P}_n}{E^{(0)} - H_0} \lr{ \lambda_1 V_1 + \lambda^2 V_2 - \Delta } \ket{n},
\end{dmath}
%
where
%
\begin{dmath}\label{eqn:gradQuantumProblemSet8Problem3:160}
\overbar{P}_n = 1 - \ket{n^{(0)}}\bra{n^{(0)}} = \sum_{m \ne n} \ket{m^{(0)}}\bra{m^{(0)}}.
\end{dmath}
%
To check this, operating with \( E^{(0)} - H_0 \), we have
%
\begin{dmath}\label{eqn:gradQuantumProblemSet8Problem3:180}
\lr{ E^{(0)} - H_0 } \ket{n}
=
\lr{ E^{(0)} - H_0 } \ket{n^{(0)}} +
\overbar{P}_n \lr{ \lambda_1 V_1 + \lambda^2 V_2 - \Delta } \ket{n}
=
\lr{ 1 - \ket{n^{(0)}}\bra{n^{(0)}} } \lr{ H - H_0 - \Delta } \ket{n},
\end{dmath}
%
or
\begin{dmath}\label{eqn:gradQuantumProblemSet8Problem3:200}
\lr{ E^{(0)} - H + \Delta} \ket{n}
=
-\ket{n^{(0)}}\bra{n^{(0)}} \lr{ H - H_0 - \Delta } \ket{n}
=
-\ket{n^{(0)}} \lr{
\lr{ E^{(0)} + \Delta} - E^{(0)} -\Delta } \braket{n^{(0)}}{n}
= 0.
\end{dmath}
%
The LHS is also zero as desired, showing that \cref{eqn:gradQuantumProblemSet8Problem3:140} is the desired perturbation relationship.
For the perturbed state we are looking for just the \( \lambda^1 \) terms of \cref{eqn:gradQuantumProblemSet8Problem3:140}, which after dropping all second order and higher terms is
%
\begin{equation}\label{eqn:gradQuantumProblemSet8Problem3:420}
\ket{n_0} + \lambda \ket{n_1} = \ket{n_0} + \sum_{m \ne n} \frac{\ket{m^{(0)}} \bra{m^{(0)}}}{E^{(0)} - E_m} \lr{ \lambda V_1 - \lambda \Delta^{(1)} } \lr{ \ket{n_0} + \lambda \ket{n_1} },
\end{equation}
%
so the first order state perturbation is
%
\begin{equation}\label{eqn:gradQuantumProblemSet8Problem3:440}
\ket{n_1} = \sum_{m \ne n} \frac{\ket{m^{(0)}} \bra{m^{(0)}}}{E^{(0)} - E_m} \lr{ V_1 - \Delta^{(1)} } \ket{n_0}.
\end{equation}
%
The \( \Delta^{(1)} \) contribution drops out, leaving
%
\boxedEquation{eqn:gradQuantumProblemSet8Problem3:460}{
\ket{n_1} = \sum_{m \ne n} \frac{\ket{m^{(0)}} \bra{m^{(0)}}}{E^{(0)} - E_m} V_1 \ket{n_0},
}

just as we had for a strictly first order perturbing potential.

For the energy shifts consider the braket
%
\begin{dmath}\label{eqn:gradQuantumProblemSet8Problem3:220}
\bra{n^{(0)}} H -H_0 \ket{n}
=
\bra{n^{(0)}} { V_1 \Delta^{(1)} + V_2 \Delta^{(2)} } \ket{n}
=
\lr{ \lr{ E^{(0)} + \Delta } -E^{(0)} } \braket{n^{(0)}}{n}
=
\Delta \braket{n^{(0)}}{n},
\end{dmath}
%
or
%As with the a first order \( \lambda \) perturbation, we can impose a requirement that \( \braket{0^{(0)}}{n} = 1 \), so
%
\begin{dmath}\label{eqn:gradQuantumProblemSet8Problem3:240}
\Delta \braket{n^{(0)}}{n} = \bra{n^{(0)}} { V_1 \Delta^{(1)} + V_2 \Delta^{(2)} } \ket{n}.
\end{dmath}
%
Expanding both sides in powers of \( \lambda \) we have
%
\begin{dmath}\label{eqn:gradQuantumProblemSet8Problem3:260}
\sum_{r = 1, s = 0} \lambda^{r+s} \Delta^{(r)} \braket{n^{(0)}}{n_s}
=
\sum_{m = 0} \lambda^{m+1} \bra{n^{(0)}} V_1 \ket{n_m} + \lambda^{m+2} \bra{n^{(0)}} V_2 \ket{n_m}
\end{dmath}

With \( \ket{n^{(0)}} = \ket{n_0} \) as required in the \( \lambda \rightarrow 0 \) limit, the \( \lambda = 1 \) contribution to these sums is
%
%\begin{dmath}\label{eqn:gradQuantumProblemSet8Problem3:280}
\boxedEquation{eqn:gradQuantumProblemSet8Problem3:300}{
\Delta^{(1)} = \bra{n_0} V_1 \ket{n_0}.
}
%\end{dmath}
%
The second order contribution is
%
\begin{dmath}\label{eqn:gradQuantumProblemSet8Problem3:320}
\Delta^{(1)} \braket{n^{(0)}}{n_1} + \Delta^{(2)} \braket{n^{(0)}}{n_0}
=
\bra{n^{(0)}} V_1 \ket{n_1} + \bra{n^{(0)}} V_2 \ket{n_0},
\end{dmath}
%
or
\begin{equation}\label{eqn:gradQuantumProblemSet8Problem3:380}
\Delta^{(2)} = \bra{n_0} V_1 \ket{n_1} + \bra{n_0} V_2 \ket{n_0} - \bra{n_0} V_1 \ket{n_0} \braket{n_0}{n_1}.
\end{equation}
%
%We can write this as
%\begin{equation}\label{eqn:gradQuantumProblemSet8Problem3:360}
%%\boxedEquation{eqn:gradQuantumProblemSet8Problem3:400}{
%\begin{aligned}
%\Delta^{(2)} &= \bra{n_0} V_1 \ket{n_1}_\perp + \bra{n_0} V_2 \ket{n_0} \\
%\ket{n_1}_\perp &= \biglr{ 1 - \ket{n_0}\bra{n_0} } \ket{n_1},
%\end{aligned}
%%}
%\end{equation}
%
%where \( \ket{n_1}_\perp \) is the rejection of \( \ket{n_0} \) from the first order perturbed state \( \ket{n_1} \), the portions of \( \ket{n_1} \) that are orthogonal to \( \ket{n_0} \).
However, from \cref{eqn:gradQuantumProblemSet8Problem3:460} we see that \( \ket{n_1} \) has no \( \ket{n_0} \) component, this means the second order shift is just
%
\boxedEquation{eqn:gradQuantumProblemSet8Problem3:500}{
\Delta^{(2)} = \bra{n_0} V_1 \ket{n_1} + \bra{n_0} V_2 \ket{n_0}.
}

\makeSubAnswer{}{gradQuantum:problemSet8:3b}
%
In \nbref{ps8:2dHarmonicOscillatorOperators.nb}, for
an initial state \( \ket{n_0} = \ket{0,0} \), it is calculated that
%
\begin{dmath}\label{eqn:gradQuantumProblemSet8Problem3:480}
V_1 \ket{0,0}
=
g_1 \frac{\Hbar \omega}{x_0^4} \lr{ x^4 + y^4 } \ket{0,0}
=
g_1 \Hbar \omega
\lr{
\frac{3 }{2} \ket{0,0}
+
\frac{3 }{\sqrt{2}} \lr{ \ket{2,0} + \ket{0,2} }
+
\sqrt{\frac{3}{2}} \lr{ \ket{4,0} + \ket{0,4} }
}.
\end{dmath}
%
The first energy shift is
%
\begin{dmath}\label{eqn:gradQuantumProblemSet8Problem3:520}
\Delta^{(1)} = \bra{0,0} V_1 \ket{0,0} = \frac{3}{2} g_1 \Hbar \omega,
\end{dmath}
%
and the first order perturbation of the state is
%
\begin{dmath}\label{eqn:gradQuantumProblemSet8Problem3:540}
\ket{n_1} = g_1 \Hbar \omega
\lr{
\frac{3}{\sqrt{2}} \frac{\ket{2,0} + \ket{0,2} }{ \Hbar \omega \lr{ 1 + 0 + 0 } - \Hbar \omega \lr{ 1 + 2 + 0} }
+
\sqrt{\frac{3}{2}} \frac{ \ket{4,0} + \ket{0,4} }{ \Hbar \omega \lr{ 1 + 0 + 0 } - \Hbar \omega \lr{ 1 + 4 + 0} }
},
\end{dmath}
%
or
\begin{dmath}\label{eqn:gradQuantumProblemSet8Problem3:560}
\ket{n_1} = -g_1
\lr{
\frac{3}{2 \sqrt{2}} \lr{\ket{2,0} + \ket{0,2} }
+
\inv{4} \sqrt{\frac{3}{2}} \lr{ \ket{4,0} + \ket{0,4} }
},
\end{dmath}
%
We can calculate
\begin{dmath}\label{eqn:gradQuantumProblemSet8Problem3:580}
\begin{aligned}
\bra{0,0} V_2 \ket{0,0} &= 0 \\
\bra{0,0} V_1 \ket{n_1} &= -\frac{21}{4} \Hbar \omega g_1^2,
\end{aligned}
\end{dmath}

so the second order energy shift is
\begin{dmath}\label{eqn:gradQuantumProblemSet8Problem3:600}
\Delta^{(2)} = -\frac{21}{4} \Hbar \omega g_1^2,
\end{dmath}
%
so the ground state energy shift, to second order in \( \lambda \), is
%
%\begin{dmath}\label{eqn:gradQuantumProblemSet8Problem3:620}
\boxedEquation{eqn:gradQuantumProblemSet8Problem3:680}{
\Hbar \omega \rightarrow \Hbar \omega + \frac{3}{2} \Hbar \omega g_1 \lambda - \frac{21}{4} \Hbar \omega g_1^2 \lambda^2.
}
%\end{dmath}
%
For the ground state, there is no contribution from the second order potential \( \lambda^2 V_2 \).

\makeSubAnswer{}{gradQuantum:problemSet8:3c}
%
The next two highest states are \( \ket{1,0}, \ket{0,1} \) both with unperturbed energy eigenvalues \( 2 \Hbar \omega \).  For a basis spanning the \( \setlr{ \ket{1,0}, \ket{0,1} } \) subspace, the matrix element of the perturbed Hamiltonian is
%
\begin{dmath}\label{eqn:gradQuantumProblemSet8Problem3:640}
H_0 + \lambda V_1 + \lambda^2 V_2
=
2 \Hbar \omega I
+ \frac{9}{2} g_1 \lambda \Hbar \omega I
+ \frac{1}{2} g_2 \lambda^2 \Hbar \omega \sigma_1
=
\Hbar \omega
\begin{bmatrix}
2 + \frac{9}{2} g_1 \lambda & \frac{1}{2} g_2 \lambda^2 \\
\frac{1}{2} g_2 \lambda^2 & 2 + \frac{9}{2} g_1 \lambda
\end{bmatrix}.
\end{dmath}
%
Since the eigenvalues of \(
\begin{bmatrix}
a & b \\
b & a
\end{bmatrix} \) are just \( a \pm b \), the energy splitting to first order for these first two degenerate states is
%
%\begin{equation}\label{eqn:gradQuantumProblemSet8Problem3:660}
\boxedEquation{eqn:gradQuantumProblemSet8Problem3:700}{
2 \Hbar \omega \rightarrow \Hbar \omega \lr{ 2 + \frac{9}{2} g_1 \lambda \pm \frac{g_2 \lambda^2 }{2} }.
}
%\end{equation}
%
While the \( x y \) perturbation potential left the ground state untouched, it is responsible for the energy level splitting for the degenerate states \( \ket{1,0}, \ket{1,0} \).
}
}

         %
% Copyright � 2015 Peeter Joot.  All Rights Reserved.
% Licenced as described in the file LICENSE under the root directory of this GIT repository.
%
\makeoproblem{Hyperfine levels.}{gradQuantum:problemSet8:4}{2015 ps8 p4}{
\index{hyperfine levels}

We can schematically model the hyperfine interaction between the electron and proton spins as \( A \BS_\txte \cdot \BS_\txtp \) where \( A \) is the hyperfine interaction energy.

\makesubproblem{}{gradQuantum:problemSet8:4a}
Consider the spin-1/2 proton interacting with a spin-1/2 electron.
What are the spin eigenstates and eigenvalues?

\makesubproblem{}{gradQuantum:problemSet8:4b}
Now consider applying a magnetic field which leads to an extra term
%
\begin{dmath}\label{eqn:gradQuantumProblemSet8Problem4:20}
-B \lr{ g_\txte \mu_\txte S_\txte^z + g_\txtp \mu_\txtp S_\txtp^z }
\end{dmath}

with gyromagnetic ratios \( g_\txte \approx -2 \) and \( g_\txtp \approx 5.5 \), with magnetic moments \( \mu_\txte = e/2m_\txte \) and
\( \mu_\txtp = e/2m_\txtp \). The large nuclear mass ensures \( \mu_\txte/\mu_\txtp \sim 2000 \), so let us simply set \( \mu_\txtp = 0\). For convenience, set \( B g_\txte \mu_\txte \rightarrow B_{\textrm{eff}} \) so the Hamiltonian becomes
%
\begin{dmath}\label{eqn:gradQuantumProblemSet8Problem4:40}
H = A \BS_\txte \cdot \BS_\txtp - B_{\textrm{eff}} S_\txte^z,
\end{dmath}
%
so the only dimensionless parameter is \( B_{\textrm{eff}}/A \).

Using perturbation theory (degenerate or non-degenerate as appropriate) find how the coupled hyperfine levels split
for weak field \( B_{\textrm{eff}}/A \ll 1 \).
Also consider the strong field limit \( B_{\textrm{eff}}/A \gg 1 \).

\makesubproblem{}{gradQuantum:problemSet8:4c}
Compute the full field evolution of the levels and compare with the perturbative low field regime result and the high field regime result.

} % makeproblem

\makeanswer{gradQuantum:problemSet8:4}{
\withproblemsetsParagraph{
\makeSubAnswer{}{gradQuantum:problemSet8:4a}
%What are the spin eigenstates and eigenvalues?

With respect to the basis \( \beta = \ket{++}, \ket{-+}, \ket{+-}, \ket{--} \), where \( \ket{\pm, \pm} = \ket{\pm}_\txte \otimes \ket{\pm}_\txtp \) are the direct products of the eigenkets of the \( \BS_\txte \) and \( \BS_\txtp \) operators (not of the respective \( S^z \) operators), the unperturbed interaction Hamiltonian is
%
\begin{dmath}\label{eqn:gradQuantumProblemSet8Problem4:60}
A \BS_\txte \cdot \BS_\txtp
=
A
\begin{bmatrix}
\bra{++} \BS_\txte \cdot \BS_\txtp \ket{++} & \bra{++} \BS_\txte \cdot \BS_\txtp \ket{-+} & \bra{++} \BS_\txte \cdot \BS_\txtp \ket{+-} & \bra{++} \BS_\txte \cdot \BS_\txtp \ket{--} \\
\bra{-+} \BS_\txte \cdot \BS_\txtp \ket{++} & \bra{-+} \BS_\txte \cdot \BS_\txtp \ket{-+} & \bra{-+} \BS_\txte \cdot \BS_\txtp \ket{+-} & \bra{-+} \BS_\txte \cdot \BS_\txtp \ket{--} \\
\bra{+-} \BS_\txte \cdot \BS_\txtp \ket{++} & \bra{+-} \BS_\txte \cdot \BS_\txtp \ket{-+} & \bra{+-} \BS_\txte \cdot \BS_\txtp \ket{+-} & \bra{+-} \BS_\txte \cdot \BS_\txtp \ket{--} \\
\bra{--} \BS_\txte \cdot \BS_\txtp \ket{++} & \bra{--} \BS_\txte \cdot \BS_\txtp \ket{-+} & \bra{--} \BS_\txte \cdot \BS_\txtp \ket{+-} & \bra{--} \BS_\txte \cdot \BS_\txtp \ket{--} \\
\end{bmatrix}
=
\frac{A \Hbar^2}{4}
\begin{bmatrix}
\bra{+} \sigma_\txte \ket{+} \bra{+} \sigma_\txtp \ket{+} & \bra{+} \sigma_\txte \ket{-} \bra{+} \sigma_\txtp \ket{+} & \bra{+} \sigma_\txte \ket{+} \bra{+} \sigma_\txtp \ket{-} & \bra{+} \sigma_\txte \ket{-} \bra{+} \sigma_\txtp \ket{-} \\
\bra{-} \sigma_\txte \ket{+} \bra{+} \sigma_\txtp \ket{+} & \bra{-} \sigma_\txte \ket{-} \bra{+} \sigma_\txtp \ket{+} & \bra{-} \sigma_\txte \ket{+} \bra{+} \sigma_\txtp \ket{-} & \bra{-} \sigma_\txte \ket{-} \bra{+} \sigma_\txtp \ket{-} \\
\bra{+} \sigma_\txte \ket{+} \bra{-} \sigma_\txtp \ket{+} & \bra{+} \sigma_\txte \ket{-} \bra{-} \sigma_\txtp \ket{+} & \bra{+} \sigma_\txte \ket{+} \bra{-} \sigma_\txtp \ket{-} & \bra{+} \sigma_\txte \ket{-} \bra{-} \sigma_\txtp \ket{-} \\
\bra{-} \sigma_\txte \ket{+} \bra{-} \sigma_\txtp \ket{+} & \bra{-} \sigma_\txte \ket{-} \bra{-} \sigma_\txtp \ket{+} & \bra{-} \sigma_\txte \ket{+} \bra{-} \sigma_\txtp \ket{-} & \bra{-} \sigma_\txte \ket{-} \bra{-} \sigma_\txtp \ket{-} \\
\end{bmatrix}
=
\frac{A \Hbar^2}{4}
\begin{bmatrix}
(1) (1) & (0) (1) & (1) (0) & (0) (0) \\
(0) (1) & (-1) (1) & (0) (0) & (-1) (0) \\
(1) (0) & (0) (0) & (1) (-1) & (0) (-1) \\
(0) (0) & (-1) (0) & (0) (-1) & (-1) (-1) \\
\end{bmatrix}
=
\frac{A \Hbar^2}{4}
\begin{bmatrix}
\sigma_3 & 0 \\
0 & -\sigma_3
\end{bmatrix}.
\end{dmath}

The spin eigenstates are the basis elements of \( \beta \) above, with respective eigenvalues
%
\begin{dmath}\label{eqn:gradQuantumProblemSet8Problem4:80}
\setlr{ A \Hbar^2/4, -A \Hbar^2/4, -A \Hbar^2/4, A \Hbar^2/4}
\end{dmath}
%
\makeSubAnswer{}{gradQuantum:problemSet8:4b}

The matrix representation of the perturbation potential is
%
\begin{dmath}\label{eqn:gradQuantumProblemSet8Problem4:100}
-B_{\textrm{eff}} S^z_\txte
=
-\frac{B_{\textrm{eff}} \Hbar}{2}
\begin{bmatrix}
\bra{+} \sigma^z_\txte \ket{+} \braket{+}{+} & \bra{+} \sigma^z_\txte \ket{-} \braket{+}{+} & \bra{+} \sigma^z_\txte \ket{+} \braket{+}{-} & \bra{+} \sigma^z_\txte \ket{-} \braket{+}{-} \\
\bra{-} \sigma^z_\txte \ket{+} \braket{+}{+} & \bra{-} \sigma^z_\txte \ket{-} \braket{+}{+} & \bra{-} \sigma^z_\txte \ket{+} \braket{+}{-} & \bra{-} \sigma^z_\txte \ket{-} \braket{+}{-} \\
\bra{+} \sigma^z_\txte \ket{+} \braket{-}{+} & \bra{+} \sigma^z_\txte \ket{-} \braket{-}{+} & \bra{+} \sigma^z_\txte \ket{+} \braket{-}{-} & \bra{+} \sigma^z_\txte \ket{-} \braket{-}{-} \\
\bra{-} \sigma^z_\txte \ket{+} \braket{-}{+} & \bra{-} \sigma^z_\txte \ket{-} \braket{-}{+} & \bra{-} \sigma^z_\txte \ket{+} \braket{-}{-} & \bra{-} \sigma^z_\txte \ket{-} \braket{-}{-} \\
\end{bmatrix}
=
-\frac{B_{\textrm{eff}} \Hbar}{2}
\begin{bmatrix}
\sigma^z_\txte & 0 \\
0 & \sigma^z_\txte
\end{bmatrix},
\end{dmath}
%
Assuming the \( \BS_\txte \) operator is directed along \( \ncap = (\sin\theta \cos\phi, \sin\theta \sin\phi, \cos\theta) \) with eigenkets
\begin{dmath}\label{eqn:gradQuantumProblemSet8Problem4:120}
\ket{+} =
\begin{bmatrix}
e^{-i\phi} \cos(\theta/2) \\
\sin(\theta/2) \\
\end{bmatrix}
\end{dmath}
\begin{dmath}\label{eqn:gradQuantumProblemSet8Problem4:140}
\ket{-} =
\begin{bmatrix}
-e^{-i\phi} \sin(\theta/2) \\
\cos(\theta/2) \\
\end{bmatrix},
\end{dmath}
%
the representation of the \( \sigma^z_\txte \) operator is
%
\begin{dmath}\label{eqn:gradQuantumProblemSet8Problem4:160}
\sigma^z_\txte
=
\begin{bmatrix}
\cos\theta & -\sin\theta \\
-\sin\theta & -\cos\theta \\
\end{bmatrix}
=
U \PauliZ U^{-1},
\end{dmath}
%
where
\begin{equation}\label{eqn:gradQuantumProblemSet8Problem4:180}
U =
\begin{bmatrix}
-\cos(\theta/2) & \sin(\theta/2) \\
\sin(\theta/2) & \cos(\theta/2)
\end{bmatrix}.
\end{equation}

This is demonstrated in \nbref{ps8:PauliMatrixSpinOperators.nb}.

The full Hamiltonian can now be written in block matrix form
%
\begin{dmath}\label{eqn:gradQuantumProblemSet8Problem4:200}
H
=
\frac{A \Hbar^2}{4}
\begin{bmatrix}
\sigma_z & 0 \\
0 & -\sigma_z
\end{bmatrix}
-\frac{B_{\textrm{eff}} \Hbar}{2}
\begin{bmatrix}
U \sigma_z U^{-1} & 0 \\
0 & U \sigma_z U^{-1}
\end{bmatrix}
\end{dmath}

Transforming the Hamiltonian to the \( S^z_\txte \) basis we have
%
\begin{dmath}\label{eqn:gradQuantumProblemSet8Problem4:220}
H' =
\frac{A \Hbar^2}{4}
\begin{bmatrix}
U^{-1} \sigma_z U & 0 \\
0 & -U^{-1} \sigma_z U
\end{bmatrix}
-\frac{B_{\textrm{eff}} \Hbar}{2}
\begin{bmatrix}
\sigma_z & 0 \\
0 & \sigma_z
\end{bmatrix}
\end{dmath}

%With \( C = \cos(\theta/2), S = \sin(\theta/2) \) these \( U^{-1} \sigma_z U \) block matrices are
%
%\begin{dmath}\label{eqn:gradQuantumProblemSet8Problem4:240}
%U^{-1} \sigma_z U
%=
%\inv{-C^2 - S^2}
%\begin{bmatrix}
%C & -S \\
%-S & -C
%\end{bmatrix}
%\begin{bmatrix}
%1 & 0 \\
%0 & -1
%\end{bmatrix}
%\begin{bmatrix}
%-C & S \\
%S & C
%\end{bmatrix}
%=
%\begin{bmatrix}
%-C & S \\
%S & C
%\end{bmatrix}
%\begin{bmatrix}
%-C & S \\
%-S & -C
%\end{bmatrix}
%=
%\begin{bmatrix}
%C^2 - S^2 & -2 S C \\
%-2 C S & S^2 - C^2
%\end{bmatrix}
%=
%\begin{bmatrix}
%\cos\theta & - \sin\theta \\
%-\sin\theta & -\cos\theta
%\end{bmatrix}.
%\end{dmath}
%
%We don't need this for the zeroth order energy split.

For \( B_{\textrm{eff}} \ll A \), the first order energy splitting can be read off by inspection
%
\begin{equation}\label{eqn:gradQuantumProblemSet8Problem4:260}
\begin{aligned}
\frac{\Hbar^2 A}{4} &\rightarrow \frac{\Hbar^2 A}{4} -\frac{B_{\textrm{eff}} \Hbar}{2} \\
-\frac{\Hbar^2 A}{4} &\rightarrow -\frac{\Hbar^2 A}{4} +\frac{B_{\textrm{eff}} \Hbar}{2} \\
-\frac{\Hbar^2 A}{4} &\rightarrow -\frac{\Hbar^2 A}{4} -\frac{B_{\textrm{eff}} \Hbar}{2} \\
\frac{\Hbar^2 A}{4} &\rightarrow \frac{\Hbar^2 A}{4} +\frac{B_{\textrm{eff}} \Hbar}{2} \\
\end{aligned}
\end{equation}

For the strong field limit, we can flip the problem, and consider \( A \BS_\txte \cdot \BS_\txtp \) to be a perturbation of an initial Hamiltonian \( H_0 = -B_{\textrm{eff}} S^z_\txte \).  The diagonalization of that perturbation is just \cref{eqn:gradQuantumProblemSet8Problem4:200} so the first order energy shifts are
%
\begin{equation}\label{eqn:gradQuantumProblemSet8Problem4:280}
\begin{aligned}
-\frac{\Hbar B_{\textrm{eff}}}{2} &\rightarrow -\frac{B_{\textrm{eff}} \Hbar}{2} + \frac{\Hbar^2 A}{4} \\
\frac{\Hbar B_{\textrm{eff}}}{2} &\rightarrow \frac{B_{\textrm{eff}} \Hbar}{2} - \frac{\Hbar^2 A}{4} \\
-\frac{\Hbar B_{\textrm{eff}}}{2} &\rightarrow -\frac{B_{\textrm{eff}} \Hbar}{2} - \frac{\Hbar^2 A}{4} \\
\frac{\Hbar B_{\textrm{eff}}}{2} &\rightarrow \frac{B_{\textrm{eff}} \Hbar}{2} + \frac{\Hbar^2 A}{4}.
\end{aligned}
\end{equation}

The splitting is the same to first order, but the starting energies are different.

\makeSubAnswer{}{gradQuantum:problemSet8:4a}

The full field solutions (as found in \nbref{ps8:PauliMatrixSpinOperators.nb}) are
%
\begin{equation}\label{eqn:gradQuantumProblemSet8Problem4:300}
\begin{array}{c}
 -\frac{\Hbar}{4} \sqrt{4 B_{\textrm{eff}}^2 - 4 A \Hbar \cos\theta B_{\textrm{eff}} + A^2 \Hbar^2}, \\
  \frac{\Hbar}{4} \sqrt{4 B_{\textrm{eff}}^2 - 4 A \Hbar \cos\theta B_{\textrm{eff}} + A^2 \Hbar^2}, \\
 -\frac{\Hbar}{4} \sqrt{4 B_{\textrm{eff}}^2 + 4 A \Hbar \cos\theta B_{\textrm{eff}} + A^2 \Hbar^2}, \\
  \frac{\Hbar}{4} \sqrt{4 B_{\textrm{eff}}^2 + 4 A \Hbar \cos\theta B_{\textrm{eff}} + A^2 \Hbar^2}.
\end{array}
\end{equation}

For the weak field \( B_{\textrm{eff}} \ll A \) the respective approximations of these energies are
%
\begin{equation}\label{eqn:gradQuantumProblemSet8Problem4:320}
\begin{array}{c}
 -\frac{A \Hbar^2}{4} - \frac{B_{\textrm{eff}} \Hbar \cos\theta}{2}, \\
  \frac{A \Hbar^2}{4} - \frac{B_{\textrm{eff}} \Hbar \cos\theta}{2}, \\
 -\frac{A \Hbar^2}{4} + \frac{B_{\textrm{eff}} \Hbar \cos\theta}{2}, \\
  \frac{A \Hbar^2}{4} + \frac{B_{\textrm{eff}} \Hbar \cos\theta}{2},
\end{array}
\end{equation}

whereas for the strong field \( B_{\textrm{eff}} \gg A \) the respective approximations of these energies are
%
\begin{equation}\label{eqn:gradQuantumProblemSet8Problem4:340}
\begin{array}{c}
 -\frac{\Hbar B_{\textrm{eff}}}{2} - \frac{A \Hbar^2 \cos\theta}{4}, \\
  \frac{\Hbar B_{\textrm{eff}}}{2} - \frac{A \Hbar^2 \cos\theta}{4}, \\
 -\frac{\Hbar B_{\textrm{eff}}}{2} + \frac{A \Hbar^2 \cos\theta}{4}, \\
  \frac{\Hbar B_{\textrm{eff}}}{2} + \frac{A \Hbar^2 \cos\theta}{4}.
\end{array}
\end{equation}

The full field solution has an orientation specific coupling that the first order perturbative solution does not find, so the perturbation is most accurate when the electron spin orientation is close to the z-axis (\( \ncap \cdot \zcap = \cos\theta \approx 1 \)).
}
}

         % p5.4
         %
% Copyright � 2015 Peeter Joot.  All Rights Reserved.
% Licenced as described in the file LICENSE under the root directory of this GIT repository.
%
%\input{../blogpost.tex}
%\renewcommand{\basename}{2dHarmonicOscillatorXYPerturbation}
%\renewcommand{\dirname}{notes/phy1520/}
%%\newcommand{\dateintitle}{}
%%\newcommand{\keywords}{}
%
%\input{../peeter_prologue_print2.tex}
%
%\usepackage{peeters_layout_exercise}
%\usepackage{peeters_braket}
%\usepackage{peeters_figures}
%\usepackage{enumerate}
%
%\beginArtNoToc
%
%\generatetitle{2D SHO xy perturbation}
%\chapter{2D SHO xy perturbation}
%\label{chap:2dHarmonicOscillatorXYPerturbation}

\makeoproblem{2D SHO xy perturbation.}{problem:2dHarmonicOscillatorXYPerturbation:1}{\citep{sakurai2014modern} pr. 5.4}{
\index{harmonic oscillator!xy perturbation}

Given a 2D SHO with Hamiltonian
%
\begin{dmath}\label{eqn:2dHarmonicOscillatorXYPerturbation:20}
H_0 = \inv{2m} \lr{ p_x^2 + p_y^2 } + \frac{m \omega^2}{2} \lr{ x^2 + y^2 },
\end{dmath}
%
\makesubproblem{}{problem:2dHarmonicOscillatorXYPerturbation:1:a}
What are the energies and degeneracies of the three lowest states?

\makesubproblem{}{problem:2dHarmonicOscillatorXYPerturbation:1:b}
With perturbation
%
\begin{dmath}\label{eqn:2dHarmonicOscillatorXYPerturbation:40}
V = m \omega^2 x y,
\end{dmath}
%
calculate the first order energy perturbations and the zeroth order perturbed states.

\makesubproblem{}{problem:2dHarmonicOscillatorXYPerturbation:1:c}
%
Solve the \( H_0 + \delta V \) problem exactly, and compare.
} % problem

\makeanswer{problem:2dHarmonicOscillatorXYPerturbation:1}{
%
\makeSubAnswer{}{problem:2dHarmonicOscillatorXYPerturbation:1:a}
%

Recall that we have
%
\begin{dmath}\label{eqn:2dHarmonicOscillatorXYPerturbation:60}
H \ket{n_1, n_2} =
\Hbar\omega
\lr{
n_1 + n_2 + 1
}
\ket{n_1, n_2},
\end{dmath}
%
So the three lowest energy states are \( \ket{0,0}, \ket{1,0}, \ket{0,1} \) with energies \( \Hbar \omega, 2 \Hbar \omega, 2 \Hbar \omega \) respectively (with a two fold degeneracy for the second two energy eigenkets).

\makeSubAnswer{}{problem:2dHarmonicOscillatorXYPerturbation:1:b}
Consider the action of \( x y \) on the \( \beta = \setlr{ \ket{0,0}, \ket{1,0}, \ket{0,1} } \) subspace.  Those are
%
\begin{dmath}\label{eqn:2dHarmonicOscillatorXYPerturbation:200}
x y \ket{0,0}
=
\frac{x_0^2}{2} \lr{ a + a^\dagger } \lr{ b + b^\dagger } \ket{0,0}
=
\frac{x_0^2}{2} \lr{ b + b^\dagger } \ket{1,0}
=
\frac{x_0^2}{2} \ket{1,1}.
\end{dmath}
%
\begin{dmath}\label{eqn:2dHarmonicOscillatorXYPerturbation:220}
x y \ket{1, 0}
=
\frac{x_0^2}{2} \lr{ a + a^\dagger } \lr{ b + b^\dagger } \ket{1,0}
=
\frac{x_0^2}{2} \lr{ a + a^\dagger } \ket{1,1}
=
\frac{x_0^2}{2} \lr{ \ket{0,1} + \sqrt{2} \ket{2,1} } .
\end{dmath}
%
\begin{dmath}\label{eqn:2dHarmonicOscillatorXYPerturbation:240}
x y \ket{0, 1}
=
\frac{x_0^2}{2} \lr{ a + a^\dagger } \lr{ b + b^\dagger } \ket{0,1}
=
\frac{x_0^2}{2} \lr{ b + b^\dagger } \ket{1,1}
=
\frac{x_0^2}{2} \lr{ \ket{1,0} + \sqrt{2} \ket{1,2} }.
\end{dmath}
%
%<row|A|column>
The matrix representation of \( m \omega^2 x y \) with respect to the subspace spanned by basis \( \beta \) above is
%
\begin{dmath}\label{eqn:2dHarmonicOscillatorXYPerturbation:260}
x y
\sim
\inv{2} \Hbar \omega
\begin{bmatrix}
0 & 0 & 0 \\
0 & 0 & 1 \\
0 & 1 & 0 \\
\end{bmatrix}.
\end{dmath}
%
This diagonalizes with
\begin{subequations}
\label{eqn:2dHarmonicOscillatorXYPerturbation:280}
\begin{dmath}\label{eqn:2dHarmonicOscillatorXYPerturbation:300}
U =
\begin{bmatrix}
1 & 0  \\
0 & \tilde{U}
\end{bmatrix}
\end{dmath}
\begin{dmath}\label{eqn:2dHarmonicOscillatorXYPerturbation:320}
\tilde{U}
=
\inv{\sqrt{2}}
\begin{bmatrix}
1 & 1 \\
1 & -1 \\
\end{bmatrix}
\end{dmath}
\begin{dmath}\label{eqn:2dHarmonicOscillatorXYPerturbation:340}
D =
\inv{2} \Hbar \omega
\begin{bmatrix}
0 & 0 & 0 \\
0 & 1 & 0 \\
0 & 0 & -1 \\
\end{bmatrix}
\end{dmath}
\begin{equation}\label{eqn:2dHarmonicOscillatorXYPerturbation:360}
x y = U D U^\dagger = U D U.
\end{equation}
\end{subequations}

The unperturbed Hamiltonian in the original basis is
%
\begin{dmath}\label{eqn:2dHarmonicOscillatorXYPerturbation:380}
H_0
=
\Hbar \omega
\begin{bmatrix}
1 & 0 \\
0 & 2 I
\end{bmatrix},
\end{dmath}
%
So the transformation to the diagonal \( x y \) basis leaves the initial Hamiltonian unaltered
%
\begin{dmath}\label{eqn:2dHarmonicOscillatorXYPerturbation:400}
H_0'
= U^\dagger H_0 U
=
\Hbar \omega
\begin{bmatrix}
1 & 0  \\
0 & \tilde{U} 2 I \tilde{U}
\end{bmatrix}
=
\Hbar \omega
\begin{bmatrix}
1 & 0  \\
0 & 2 I
\end{bmatrix}.
\end{dmath}
%
Now we can compute the first order energy shifts almost by inspection.  Writing the new basis as \( \beta' = \setlr{ \ket{0}, \ket{1}, \ket{2} } \) those energy shifts are just the diagonal elements from the \( x y \) operators matrix representation
%
\begin{dmath}\label{eqn:2dHarmonicOscillatorXYPerturbation:420}
\begin{aligned}
E^{{(1)}}_0 &= \bra{0} V \ket{0} = 0 \\
E^{{(1)}}_1 &= \bra{1} V \ket{1} = \inv{2} \Hbar \omega \\
E^{{(1)}}_2 &= \bra{2} V \ket{2} = -\inv{2} \Hbar \omega.
\end{aligned}
\end{dmath}
%
The new energies are
%
\begin{dmath}\label{eqn:2dHarmonicOscillatorXYPerturbation:440}
\begin{aligned}
E_0 &\rightarrow \Hbar \omega \\
E_1 &\rightarrow \Hbar \omega \lr{ 2 + \delta/2 } \\
E_2 &\rightarrow \Hbar \omega \lr{ 2 - \delta/2 }.
\end{aligned}
\end{dmath}
%
\makeSubAnswer{}{problem:2dHarmonicOscillatorXYPerturbation:1:c}
%
For the exact solution, it's possible to rotate the coordinate system in a way that kills the explicit \( x y \) term of the perturbation.  That we could do this for \( x, y \) operators wasn't obvious to me, but after doing so (and rotating the momentum operators the same way) the new operators still have the required commutators.  Let
%
\begin{dmath}\label{eqn:2dHarmonicOscillatorXYPerturbation:80}
\begin{bmatrix}
u \\
v
\end{bmatrix}
=
\begin{bmatrix}
\cos\theta & \sin\theta \\
-\sin\theta & \cos\theta
\end{bmatrix}
\begin{bmatrix}
x \\
y
\end{bmatrix}
=
\begin{bmatrix}
x \cos\theta + y \sin\theta \\
-x \sin\theta + y \cos\theta
\end{bmatrix}.
\end{dmath}
%
Similarly, for the momentum operators, let
\begin{dmath}\label{eqn:2dHarmonicOscillatorXYPerturbation:100}
\begin{bmatrix}
p_u \\
p_v
\end{bmatrix}
=
\begin{bmatrix}
\cos\theta & \sin\theta \\
-\sin\theta & \cos\theta
\end{bmatrix}
\begin{bmatrix}
p_x \\
p_y
\end{bmatrix}
=
\begin{bmatrix}
p_x \cos\theta + p_y \sin\theta \\
-p_x \sin\theta + p_y \cos\theta
\end{bmatrix}.
\end{dmath}
%
For the commutators of the new operators we have
%
\begin{dmath}\label{eqn:2dHarmonicOscillatorXYPerturbation:120}
\antisymmetric{u}{p_u}
=
\antisymmetric{x \cos\theta + y \sin\theta}{p_x \cos\theta + p_y \sin\theta}
=
\antisymmetric{x}{p_x} \cos^2\theta + \antisymmetric{y}{p_y} \sin^2\theta
=
i \Hbar \lr{ \cos^2\theta + \sin^2\theta }
=
i\Hbar.
\end{dmath}
%
\begin{dmath}\label{eqn:2dHarmonicOscillatorXYPerturbation:140}
\antisymmetric{v}{p_v}
=
\antisymmetric{-x \sin\theta + y \cos\theta}{-p_x \sin\theta + p_y \cos\theta}
=
\antisymmetric{x}{p_x} \sin^2\theta + \antisymmetric{y}{p_y} \cos^2\theta
=
i \Hbar.
\end{dmath}
%
\begin{dmath}\label{eqn:2dHarmonicOscillatorXYPerturbation:160}
\antisymmetric{u}{p_v}
=
\antisymmetric{x \cos\theta + y \sin\theta}{-p_x \sin\theta + p_y \cos\theta}
= \cos\theta \sin\theta \lr{ -\antisymmetric{x}{p_x} + \antisymmetric{y}{p_p} }
=
0.
\end{dmath}
%
\begin{dmath}\label{eqn:2dHarmonicOscillatorXYPerturbation:180}
\antisymmetric{v}{p_u}
=
\antisymmetric{-x \sin\theta + y \cos\theta}{p_x \cos\theta + p_y \sin\theta}
= \cos\theta \sin\theta \lr{ -\antisymmetric{x}{p_x} + \antisymmetric{y}{p_p} }
=
0.
\end{dmath}
%
We see that the new operators are canonical conjugate as required.  For this problem, we just want a 45 degree rotation, with
%
\begin{equation}\label{eqn:2dHarmonicOscillatorXYPerturbation:460}
\begin{aligned}
x &= \inv{\sqrt{2}} \lr{ u + v } \\
y &= \inv{\sqrt{2}} \lr{ u - v }.
\end{aligned}
\end{equation}
%
We have
\begin{dmath}\label{eqn:2dHarmonicOscillatorXYPerturbation:480}
x^2 + y^2
=
\inv{2} \lr{ (u+v)^2 + (u-v)^2 }
=
\inv{2} \lr{ 2 u^2 + 2 v^2 + 2 u v - 2 u v }
=
u^2 + v^2,
\end{dmath}
%
\begin{dmath}\label{eqn:2dHarmonicOscillatorXYPerturbation:500}
p_x^2 + p_y^2
=
\inv{2} \lr{ (p_u+p_v)^2 + (p_u-p_v)^2 }
=
\inv{2} \lr{ 2 p_u^2 + 2 p_v^2 + 2 p_u p_v - 2 p_u p_v }
=
p_u^2 + p_v^2,
\end{dmath}
%
and
\begin{dmath}\label{eqn:2dHarmonicOscillatorXYPerturbation:520}
x y
=
\inv{2} \lr{ (u+v)(u-v) }
=
\inv{2} \lr{ u^2 - v^2 }.
\end{dmath}
%
The perturbed Hamiltonian is
%
\begin{dmath}\label{eqn:2dHarmonicOscillatorXYPerturbation:540}
H_0 + \delta V
=
\inv{2m} \lr{ p_u^2 + p_v^2 }
+ \inv{2} m \omega^2 \lr{ u^2 + v^2 + \delta u^2 - \delta v^2 }
=
\inv{2m} \lr{ p_u^2 + p_v^2 }
+ \inv{2} m \omega^2 \lr{ u^2(1 + \delta) + v^2 (1 - \delta) }.
\end{dmath}
%
In this coordinate system, the corresponding eigensystem is
%
\begin{dmath}\label{eqn:2dHarmonicOscillatorXYPerturbation:560}
H \ket{n_1, n_2}
= \Hbar \omega \lr{ 1 + n_1 \sqrt{1 + \delta} + n_2 \sqrt{ 1 - \delta } } \ket{n_1, n_2}.
\end{dmath}
%
For small \( \delta \)
%
\begin{dmath}\label{eqn:2dHarmonicOscillatorXYPerturbation:580}
n_1 \sqrt{1 + \delta} + n_2 \sqrt{ 1 - \delta }
\approx
n_1 + n_2
+ \inv{2} n_1 \delta
- \inv{2} n_2 \delta,
\end{dmath}
%
so
\begin{dmath}\label{eqn:2dHarmonicOscillatorXYPerturbation:600}
H \ket{n_1, n_2}
\approx \Hbar \omega \lr{ 1 + n_1 + n_2 + \inv{2} n_1 \delta - \inv{2} n_2 \delta
} \ket{n_1, n_2}.
\end{dmath}
%
The lowest order perturbed energy levels are
%
\begin{dmath}\label{eqn:2dHarmonicOscillatorXYPerturbation:620}
\ket{0,0} \rightarrow \Hbar \omega
\end{dmath}
\begin{dmath}\label{eqn:2dHarmonicOscillatorXYPerturbation:640}
\ket{1,0} \rightarrow \Hbar \omega \lr{ 2 + \inv{2} \delta }
\end{dmath}
\begin{dmath}\label{eqn:2dHarmonicOscillatorXYPerturbation:660}
\ket{0,1} \rightarrow \Hbar \omega \lr{ 2 - \inv{2} \delta }
\end{dmath}
%
The degeneracy of the \( \ket{0,1}, \ket{1,0} \) states has been split, and to first order match the zeroth order perturbation result.
} % answer

%\EndArticle

         % p5.11
         %
% Copyright � 2015 Peeter Joot.  All Rights Reserved.
% Licenced as described in the file LICENSE under the root directory of this GIT repository.
%
%\input{../blogpost.tex}
%\renewcommand{\basename}{simplestTwoByTwoPerturbation}
%\renewcommand{\dirname}{notes/phy1520/}
%%\newcommand{\dateintitle}{}
%%\newcommand{\keywords}{}
%
%\input{../peeter_prologue_print2.tex}
%
%\usepackage{peeters_layout_exercise}
%\usepackage{peeters_braket}
%\usepackage{peeters_figures}
%\usepackage{enumerate}
%
%\beginArtNoToc
%
%\generatetitle{Simplest perturbation two by two Hamiltonian}
%%\chapter{Simplest perturbation two by two Hamiltonian}
%%\label{chap:simplestTwoByTwoPerturbation}
%
\makeoproblem{Perturbation of two state Hamiltonian.}{problem:simplestTwoByTwoPerturbation:1}{\citep{sakurai2014modern} pr. 5.11}{
\index{two state!perturbation}

%\paragraph{Q: two state Hamiltonian.}

Given a two-state system
%
\begin{dmath}\label{eqn:simplestTwoByTwoPerturbation:20}
H = H_0 + \lambda V
=
\begin{bmatrix}
E_1 & \lambda \Delta \\
\lambda \Delta & E_2
\end{bmatrix}
\end{dmath}

%
\makesubproblem{}{problem:simplestTwoByTwoPerturbation:1:a}
%\begin{enumerate}[a]
%\item
Solve the system exactly.
%\item
\makesubproblem{}{problem:simplestTwoByTwoPerturbation:1:b}
Find the first order perturbed states and second order energy shifts, and compare to the exact solution.
%\item
\makesubproblem{}{problem:simplestTwoByTwoPerturbation:1:c}
%
Solve the degenerate case for \( E_1 = E_2 \), and compare to the exact solution.
%\end{enumerate}
} % problem
%
\makeanswer{problem:simplestTwoByTwoPerturbation:1}{
%
\makeSubAnswer{}{problem:simplestTwoByTwoPerturbation:1:a}
%\paragraph{A: part (a)}

The energy eigenvalues \( \epsilon \) are given by
%
\begin{dmath}\label{eqn:simplestTwoByTwoPerturbation:40}
0
=
\lr{ E_1 - \epsilon }
\lr{ E_2 - \epsilon }
- (\lambda \Delta)^2,
\end{dmath}
%
or
%
\begin{dmath}\label{eqn:simplestTwoByTwoPerturbation:60}
\epsilon^2 - \epsilon\lr{ E_1 + E_2 } + E_1 E_2 = (\lambda \Delta)^2.
\end{dmath}
%
After rearranging this is
\begin{dmath}\label{eqn:simplestTwoByTwoPerturbation:80}
\epsilon = \frac{ E_1 + E_2 }{2} \pm \sqrt{ \lr{ \frac{ E_1 - E_2 }{2} }^2 + (\lambda \Delta)^2 }.
\end{dmath}
%
Notice that for \( E_2 = E_1 \) we have
%
\begin{dmath}\label{eqn:simplestTwoByTwoPerturbation:100}
\epsilon = E_1 \pm \lambda \Delta.
\end{dmath}
%
Since a change of basis can always put the problem in a form so that \( E_1 > E_2 \), let's assume that to make an approximation of the energy eigenvalues for \( \Abs{\lambda \Delta} \ll \ifrac{ (E_1 - E_2) }{2} \)
%
\begin{dmath}\label{eqn:simplestTwoByTwoPerturbation:120}
\epsilon
=
\frac{ E_1 + E_2 }{2} \pm \frac{ E_1 - E_2 }{2} \sqrt{ 1 + \frac{(2 \lambda \Delta)^2}{(E_1 - E_2)^2} }
\approx
\frac{ E_1 + E_2 }{2} \pm \frac{ E_1 - E_2 }{2} \lr{ 1 + 2 \frac{(\lambda \Delta)^2}{(E_1 - E_2)^2} }
=
\frac{ E_1 + E_2 }{2} \pm \frac{ E_1 - E_2 }{2}
\pm
\frac{(\lambda \Delta)^2}{E_1 - E_2}
=
E_1 + \frac{(\lambda \Delta)^2}{E_1 - E_2}, E_2 + \frac{(\lambda \Delta)^2}{E_2 - E_1}.
\end{dmath}
%
For the perturbed states, starting with the plus case, if
%
\begin{dmath}\label{eqn:simplestTwoByTwoPerturbation:140}
\ket{+} \propto
\begin{bmatrix}
a \\
b
\end{bmatrix},
\end{dmath}
%
we must have
\begin{dmath}\label{eqn:simplestTwoByTwoPerturbation:160}
0
=
\biglr{ E_1 - \lr{ E_1 + \frac{(\lambda \Delta)^2}{E_1 - E_2} } } a + \lambda \Delta b
=
\biglr{ - \frac{(\lambda \Delta)^2}{E_1 - E_2} } a + \lambda \Delta b,
\end{dmath}
%
so
%
\begin{dmath}\label{eqn:simplestTwoByTwoPerturbation:180}
\ket{+} \rightarrow
\begin{bmatrix}
1 \\
\frac{\lambda \Delta}{E_1 - E_2}
\end{bmatrix}
= \ket{+} + \frac{\lambda \Delta}{E_1 - E_2} \ket{-}.
\end{dmath}
%
Similarly for the minus case we must have
%
\begin{dmath}\label{eqn:simplestTwoByTwoPerturbation:200}
0
=
\lambda \Delta a + \biglr{ E_2 - \lr{ E_2 + \frac{(\lambda \Delta)^2}{E_2 - E_1} } } b
=
\lambda \Delta b + \biglr{ - \frac{(\lambda \Delta)^2}{E_2 - E_1} } b,
\end{dmath}
%
for
\begin{dmath}\label{eqn:simplestTwoByTwoPerturbation:220}
\ket{-} \rightarrow
\ket{-} + \frac{\lambda \Delta}{E_2 - E_1} \ket{+}.
\end{dmath}
%
\makeSubAnswer{}{problem:simplestTwoByTwoPerturbation:1:b}
%\paragraph{A: part (b)}

For the perturbation the first energy shift for perturbation of the \( \ket{+} \) state is
%
\begin{dmath}\label{eqn:simplestTwoByTwoPerturbation:240}
E_{+}^{(1)}
= \ket{+} V \ket{+}
=
\lambda \Delta
\begin{bmatrix}
1 & 0
\end{bmatrix}
\begin{bmatrix}
0 & 1 \\
1 & 0
\end{bmatrix}
\begin{bmatrix}
1 \\
0
\end{bmatrix}
=\lambda \Delta
\begin{bmatrix}
1 & 0
\end{bmatrix}
\begin{bmatrix}
0 \\
1
\end{bmatrix}
=
0.
\end{dmath}
%
The first order energy shift for the perturbation of the \( \ket{-} \) state is also zero.  The perturbed \( \ket{+} \) state is
%
\begin{dmath}\label{eqn:simplestTwoByTwoPerturbation:260}
\ket{+}^{(1)}
= \frac{\overbar{P}_{+}}{E_1 - H_0} V \ket{+}
= \frac{\ket{-}\bra{-}}{E_1 - E_2} V \ket{+}
\end{dmath}

The numerator matrix element is
%
\begin{dmath}\label{eqn:simplestTwoByTwoPerturbation:280}
\bra{-} V \ket{+}
=
\begin{bmatrix}
0 & 1
\end{bmatrix}
\begin{bmatrix}
0 & \Delta \\
\Delta & 0
\end{bmatrix}
\begin{bmatrix}
1 \\
0
\end{bmatrix}
=
\begin{bmatrix}
0 & 1
\end{bmatrix}
\begin{bmatrix}
0 \\
\Delta
\end{bmatrix}
=
\Delta,
\end{dmath}
%
so
%
\begin{dmath}\label{eqn:simplestTwoByTwoPerturbation:300}
\ket{+} \rightarrow \ket{+} + \ket{-} \frac{\Delta}{E_1 - E_2}.
\end{dmath}
%
Observe that this matches the first order series expansion of the exact value above.

For the perturbation of \( \ket{-} \) we need the matrix element
%
\begin{dmath}\label{eqn:simplestTwoByTwoPerturbation:320}
\bra{+} V \ket{-}
=
\begin{bmatrix}
1 & 0
\end{bmatrix}
\begin{bmatrix}
0 & \Delta \\
\Delta & 0
\end{bmatrix}
\begin{bmatrix}
0 \\
1
\end{bmatrix}
=
\begin{bmatrix}
1 & 0
\end{bmatrix}
\begin{bmatrix}
\Delta \\
0 \\
\end{bmatrix}
=
\Delta,
\end{dmath}
%
so it's clear that the perturbed ket is
%
\begin{dmath}\label{eqn:simplestTwoByTwoPerturbation:340}
\ket{-} \rightarrow \ket{-} + \ket{+} \frac{\Delta}{E_2 - E_1},
\end{dmath}
%
also matching the approximation found from the exact computation.  The second order energy shifts can now be calculated
%
\begin{dmath}\label{eqn:simplestTwoByTwoPerturbation:360}
\bra{+} V \ket{+}'
=
\begin{bmatrix}
1 & 0
\end{bmatrix}
\begin{bmatrix}
0 & \Delta \\
\Delta & 0
\end{bmatrix}
\begin{bmatrix}
1 \\
\frac{\Delta}{E_1 - E_2}
\end{bmatrix}
=
\begin{bmatrix}
1 & 0
\end{bmatrix}
\begin{bmatrix}
\frac{\Delta^2}{E_1 - E_2} \\
\Delta
\end{bmatrix}
=
\frac{\Delta^2}{E_1 - E_2},
\end{dmath}
%
and
%
\begin{dmath}\label{eqn:simplestTwoByTwoPerturbation:380}
\bra{-} V \ket{-}'
=
\begin{bmatrix}
0 & 1
\end{bmatrix}
\begin{bmatrix}
0 & \Delta \\
\Delta & 0
\end{bmatrix}
\begin{bmatrix}
\frac{\Delta}{E_2 - E_1} \\
1 \\
\end{bmatrix}
=
\begin{bmatrix}
0 & 1
\end{bmatrix}
\begin{bmatrix}
\Delta \\
\frac{\Delta^2}{E_2 - E_1} \\
\end{bmatrix}
=
\frac{\Delta^2}{E_2 - E_1},
\end{dmath}
%
The energy perturbations are therefore
\begin{dmath}\label{eqn:simplestTwoByTwoPerturbation:400}
\begin{aligned}
E_1 &\rightarrow E_1 + \frac{(\lambda \Delta)^2}{E_1 - E_2} \\
E_2 &\rightarrow E_2 + \frac{(\lambda \Delta)^2}{E_2 - E_1}.
\end{aligned}
\end{dmath}
%
This is what we had by approximating the exact case.
%
\makeSubAnswer{}{problem:simplestTwoByTwoPerturbation:1:c}
%\paragraph{A: part (c)}

For the \( E_2 = E_1 \) case, we'll have to diagonalize the perturbation potential.  That is
%
\begin{dmath}\label{eqn:simplestTwoByTwoPerturbation:420}
\begin{aligned}
V &= U \bigwedge U^\dagger \\
\bigwedge &=
\begin{bmatrix}
\Delta & 0 \\
0 & -\Delta
\end{bmatrix} \\
U &= U^\dagger = \inv{\sqrt{2}}
\begin{bmatrix}
1 & 1 \\
1 & -1
\end{bmatrix}.
\end{aligned}
\end{dmath}
%
A change of basis for the Hamiltonian is
%
\begin{dmath}\label{eqn:simplestTwoByTwoPerturbation:440}
H'
=
U^\dagger H U
=
U^\dagger H_0 U + \lambda U^\dagger V U
=
E_1 U^\dagger + \lambda U^\dagger V U
=
H_0 + \lambda \bigwedge.
\end{dmath}
%
We can now calculate the perturbation energy with respect to the new basis, say \( \setlr{ \ket{1}, \ket{2} } \).  Those energy shifts are
%
\begin{dmath}\label{eqn:simplestTwoByTwoPerturbation:460}
\begin{aligned}
\Delta^{(1)} &= \bra{1} V \ket{1} = \Delta \\
\Delta^{(2)} &= \bra{2} V \ket{2} = -\Delta.
\end{aligned}
\end{dmath}
%
The perturbed energies are therefore
%
\begin{equation}\label{eqn:simplestTwoByTwoPerturbation:480}
\begin{aligned}
E_1 &\rightarrow E_1 + \lambda \Delta \\
E_2 &\rightarrow E_2 - \lambda \Delta,
\end{aligned}
\end{equation}

which matches \cref{eqn:simplestTwoByTwoPerturbation:100}, the exact result.

} % answer

%\EndArticle

         % p5.16
         %
% Copyright � 2015 Peeter Joot.  All Rights Reserved.
% Licenced as described in the file LICENSE under the root directory of this GIT repository.
%
%\input{../blogpost.tex}
%\renewcommand{\basename}{symmetricPotentialDerivativeExpectation}
%\renewcommand{\dirname}{notes/phy1520/}
%%\newcommand{\dateintitle}{}
%%\newcommand{\keywords}{}
%
%\input{../peeter_prologue_print2.tex}
%
%\usepackage{peeters_layout_exercise}
%\usepackage{peeters_braket}
%\usepackage{peeters_figures}
%
%\beginArtNoToc
%
%\generatetitle{Expectation of spherically symmetric 3D potential derivative}
%%\chapter{Expectation of spherically symmetric 3D potential derivative}
%%\label{chap:symmetricPotentialDerivativeExpectation}
%
\makeoproblem{Spherically symmetric 3D potential derivative.}{problem:symmetricPotentialDerivativeExpectation:1}{\citep{sakurai2014modern} pr. 5.16}{
\index{spherically symmetric potential!derivative}
%
\makesubproblem{}{problem:symmetricPotentialDerivativeExpectation:1:a}
For a particle in a spherically symmetric potential \( V(r) \) show that
%
\begin{equation}\label{eqn:symmetricPotentialDerivativeExpectation:20}
\Abs{\psi(0)}^2 = \frac{m}{2 \pi \Hbar^2} \expectation{ \frac{dV}{dr} },
\end{equation}
%
for all s-states, ground or excited.
%
\makesubproblem{}{problem:symmetricPotentialDerivativeExpectation:1:b}
%
Show this is the case for the 3D SHO and hydrogen wave functions.
%
} % problem
%
\makeanswer{problem:symmetricPotentialDerivativeExpectation:1}{
%
\makeSubAnswer{}{problem:symmetricPotentialDerivativeExpectation:1:a}
%
The text works a problem that looks similar to this by considering the commutator of an operator \( A \), later set to \( A = p_r = -i \Hbar \PDi{r}{} \) the radial momentum operator.  First it is noted that
%
\begin{equation}\label{eqn:symmetricPotentialDerivativeExpectation:40}
0 = \bra{nlm} \antisymmetric{H}{A} \ket{nlm},
\end{dmath}
%
since \( H \) operating to either the right or the left is the energy eigenvalue \( E_n \).  Next it appears the author uses an angular momentum factoring of the squared momentum operator.  Looking earlier in the text that factoring is found to be
%
\begin{dmath}\label{eqn:symmetricPotentialDerivativeExpectation:60}
\frac{\Bp^2}{2m}
= \inv{2 m r^2} \BL^2 - \frac{\Hbar^2}{2m} \lr{ \PDSq{r}{} + \frac{2}{r} \PD{r}{} }.
\end{dmath}
%
With
\begin{equation}\label{eqn:symmetricPotentialDerivativeExpectation:80}
R = - \frac{\Hbar^2}{2m} \lr{ \PDSq{r}{} + \frac{2}{r} \PD{r}{} },
\end{dmath}
%
we have
%
\begin{dmath}\label{eqn:symmetricPotentialDerivativeExpectation:100}
0
= \bra{nlm} \antisymmetric{H}{p_r} \ket{nlm}
= \bra{nlm} \antisymmetric{\frac{\Bp^2}{2m} + V(r)}{p_r} \ket{nlm}
= \bra{nlm} \antisymmetric{\inv{2 m r^2} \BL^2 + R + V(r)}{p_r} \ket{nlm}
= \bra{nlm} \antisymmetric{\frac{-\Hbar^2 l (l+1)}{2 m r^2} + R + V(r)}{p_r} \ket{nlm}.
\end{dmath}
%
Let's consider the commutator of each term separately.  First
%
\begin{dmath}\label{eqn:symmetricPotentialDerivativeExpectation:120}
\antisymmetric{V}{p_r} \psi
=
V p_r \psi
-
p_r V \psi
=
V p_r \psi
-
(p_r V) \psi
-
V p_r \psi
=
-
(p_r V) \psi
=
i \Hbar \PD{r}{V} \psi.
\end{dmath}
%
Setting \( V(r) = 1/r^2 \), we also have
%
\begin{dmath}\label{eqn:symmetricPotentialDerivativeExpectation:160}
\antisymmetric{\inv{r^2}}{p_r} \psi
=
-\frac{2 i \Hbar}{r^3} \psi.
\end{dmath}
%
Finally
\begin{dmath}\label{eqn:symmetricPotentialDerivativeExpectation:180}
\antisymmetric{\PDSq{r}{} + \frac{2}{r} \PD{r}{} }{ \PD{r}{}}
=
\lr{ \partial_{rr} + \frac{2}{r} \partial_r } \partial_r
-
\partial_r \lr{ \partial_{rr} + \frac{2}{r} \partial_r }
=
\partial_{rrr} + \frac{2}{r} \partial_{rr}
-
\lr{
\partial_{rrr} -\frac{2}{r^2} \partial_r + \frac{2}{r} \partial_{rr}
}
=
-\frac{2}{r^2} \partial_r,
\end{dmath}
%
so
\begin{dmath}\label{eqn:symmetricPotentialDerivativeExpectation:200}
\antisymmetric{R}{p_r}
=-\frac{2}{r^2} \frac{-\Hbar^2}{2m} p_r
=\frac{\Hbar^2}{m r^2} p_r.
\end{dmath}
%
Putting all the pieces back together, we've got
\begin{dmath}\label{eqn:symmetricPotentialDerivativeExpectation:220}
0
= \bra{nlm} \antisymmetric{\frac{-\Hbar^2 l (l+1)}{2 m r^2} + R + V(r)}{p_r} \ket{nlm}
=
i \Hbar
\bra{nlm} \lr{
\frac{\Hbar^2 l (l+1)}{m r^3} - \frac{i\Hbar}{m r^2} p_r +
\PD{r}{V}
}
\ket{nlm}.
\end{dmath}
%
Since s-states are those for which \( l = 0 \), this means
%
\begin{dmath}\label{eqn:symmetricPotentialDerivativeExpectation:240}
\expectation{\PD{r}{V}}
= \frac{i\Hbar}{m } \expectation{ \inv{r^2} p_r }
= \frac{\Hbar^2}{m } \expectation{ \inv{r^2} \PD{r}{} }
= \frac{\Hbar^2}{m } \int_0^\infty dr \int_0^\pi d\theta \int_0^{2 \pi} d\phi r^2 \sin\theta \psi^\conj(r,\theta, \phi) \inv{r^2} \PD{r}{\psi(r,\theta,\phi)}.
\end{dmath}
%
Since s-states are spherically symmetric, this is
\begin{dmath}\label{eqn:symmetricPotentialDerivativeExpectation:260}
\expectation{\PD{r}{V}}
= \frac{4 \pi \Hbar^2}{m } \int_0^\infty dr \psi^\conj \PD{r}{\psi}.
\end{dmath}
%
That integral is
%
\begin{dmath}\label{eqn:symmetricPotentialDerivativeExpectation:280}
\int_0^\infty dr \psi^\conj \PD{r}{\psi}
=
\evalrange{\Abs{\psi}^2}{0}{\infty} - \int_0^\infty dr \PD{r}{\psi^\conj} \psi.
\end{dmath}
%
With the hydrogen atom, our radial wave functions are real valued.  It's reasonable to assume that we can do the same for other real-valued spherical potentials.  If that is the case, we have
%
\begin{dmath}\label{eqn:symmetricPotentialDerivativeExpectation:300}
2 \int_0^\infty dr \psi^\conj \PD{r}{\psi}
=
\Abs{\psi(0)}^2,
\end{dmath}
%
and
%
%\begin{equation}\label{eqn:symmetricPotentialDerivativeExpectation:320}
\boxedEquation{eqn:symmetricPotentialDerivativeExpectation:340}{
\expectation{\PD{r}{V}}
= \frac{2 \pi \Hbar^2}{m } \Abs{\psi(0)}^2,
}
%\end{equation}
which completes this part of the problem.
%
\makeSubAnswer{}{problem:symmetricPotentialDerivativeExpectation:1:b}
%
For a hydrogen like atom, in atomic units, we have
%
\begin{dmath}\label{eqn:symmetricPotentialDerivativeExpectation:360}
\expectation{
\PD{r}{V}
}
=
\expectation{
\PD{r}{} \lr{ -\frac{Z e^2}{r} }
}
=
Z e^2
\expectation
{
\inv{r^2}
}
=
Z e^2 \frac{Z^2}{n^3 a_0^2 \lr{ l + 1/2 }}
=
\frac{\Hbar^2}{m a_0} \frac{2 Z^3}{n^3 a_0^2}
=
\frac{2 \Hbar^2 Z^3}{m n^3 a_0^3}.
\end{dmath}
%
On the other hand for \( n = 1 \), we have
%
\begin{dmath}\label{eqn:symmetricPotentialDerivativeExpectation:380}
\frac{2 \pi \Hbar^2}{m} \Abs{R_{10}(0)}^2 \Abs{Y_{00}}^2
=
\frac{2 \pi \Hbar^2}{m} \frac{Z^3}{a_0^3} 4 \inv{4 \pi}
=
\frac{2 \Hbar^2 Z^3}{m a_0^3},
\end{dmath}
%
and for \( n = 2 \), we have
%
\begin{dmath}\label{eqn:symmetricPotentialDerivativeExpectation:400}
\frac{2 \pi \Hbar^2}{m} \Abs{R_{20}(0)}^2 \Abs{Y_{00}}^2
=
\frac{2 \pi \Hbar^2}{m} \frac{Z^3}{8 a_0^3} 4 \inv{4 \pi}
=
\frac{\Hbar^2 Z^3}{4 m a_0^3}.
\end{dmath}
%
These both match the potential derivative expectation when evaluated for the s-orbital (\( l = 0 \)).

In \nbref{sakuraiProblem5.16bSHO.nb} is a verification for the 3D SHO ground state.  There it was found that
%
\begin{dmath}\label{eqn:symmetricPotentialDerivativeExpectation:420}
\expectation{\PD{r}{V}}
= \frac{2 \pi \Hbar^2}{m } \Abs{\psi(0)}^2
= 2 \sqrt{\frac{m \omega ^3 \Hbar}{ \pi }}.
\end{dmath}
%
} % answer
%\EndArticle

         % p5.17(a)
         %
% Copyright � 2015 Peeter Joot.  All Rights Reserved.
% Licenced as described in the file LICENSE under the root directory of this GIT repository.
%
%\input{../blogpost.tex}
%\renewcommand{\basename}{LyPerturbation}
%\renewcommand{\dirname}{notes/phy1520/}
%%\newcommand{\dateintitle}{}
%%\newcommand{\keywords}{}
%
%\input{../peeter_prologue_print2.tex}
%
%\usepackage{peeters_layout_exercise}
%\usepackage{peeters_braket}
%\usepackage{peeters_figures}
%
%\beginArtNoToc
%
%\generatetitle{\(L_y\) perturbation}
%%\chapter{\(L_y\) perturbation}
%%\label{chap:LyPerturbation}
%
\makeoproblem{\( L_y \) perturbation.}{problem:LyPerturbation:1}{\citep{sakurai2014modern} pr. 5.17(a)}{
\index{angular momentum!perturbation}

Find the first non-zero energy shift for the perturbed Hamiltonian
%
\begin{dmath}\label{eqn:LyPerturbation:20}
H = A \BL^2 + B L_z + C L_y = H_0 + V.
\end{dmath}
%
} % problem
%
\makeanswer{problem:LyPerturbation:1}{
%
The energy eigenvalues for state \( \ket{l, m} \) prior to perturbation are
%
\begin{dmath}\label{eqn:LyPerturbation:40}
A \Hbar^2 l(l+1) + B \Hbar m.
\end{dmath}
%
The first order energy shift is zero

% Lp = Lx + iLy
% Lm = Lx - iLy
% 2i Ly = Lp - Lm
\begin{dmath}\label{eqn:LyPerturbation:60}
\Delta^1
=
\bra{l, m} C L_y \ket{l, m}
=
\frac{C}{2 i}
\bra{l, m} \lr{ L_{+} - L_{-} } \ket{l, m}
=
0,
\end{dmath}
%
so we need the second order shift.  Assuming no degeneracy to start, the perturbed state is
%
\begin{dmath}\label{eqn:LyPerturbation:80}
\ket{l, m}' = \sum' \frac{\ket{l', m'} \bra{l', m'}}{E_{l,m} - E_{l', m'}} V \ket{l, m},
\end{dmath}
%
and the next order energy shift is
\begin{dmath}\label{eqn:LyPerturbation:100}
\Delta^2
=
\bra{l m} V
\sum' \frac{\ket{l', m'} \bra{l', m'}}{E_{l,m} - E_{l', m'}} V \ket{l, m}
=
\sum' \frac{\bra{l, m} V \ket{l', m'} \bra{l', m'}}{E_{l,m} - E_{l', m'}} V \ket{l, m}
=
\sum' \frac{ \Abs{ \bra{l', m'} V \ket{l, m} }^2 }{E_{l,m} - E_{l', m'}}
=
\sum_{m' \ne m} \frac{ \Abs{ \bra{l, m'} V \ket{l, m} }^2 }{E_{l,m} - E_{l, m'}}
=
\sum_{m' \ne m} \frac{ \Abs{ \bra{l, m'} V \ket{l, m} }^2 }{
\lr{ A \Hbar^2 l(l+1) + B \Hbar m }
-\lr{ A \Hbar^2 l(l+1) + B \Hbar m' }
}
=
\inv{B \Hbar} \sum_{m' \ne m} \frac{ \Abs{ \bra{l, m'} V \ket{l, m} }^2 }{
m - m'
}.
\end{dmath}
%
The sum over \( l' \) was eliminated because \( V \) only changes the \( m \) of any state \( \ket{l,m} \), so the matrix element \( \bra{l',m'} V \ket{l, m} \) must includes a \( \delta_{l', l} \) factor.
Since we are now summing over \( m' \ne m \), some of the matrix elements in the numerator should now be non-zero, unlike the case when the zero first order energy shift was calculated in \cref{eqn:LyPerturbation:60}.
%
\begin{equation}\label{eqn:LyPerturbation:120}
\begin{aligned}
&\bra{l, m'} C L_y \ket{l, m} \\
&\qquad=
\frac{C}{2 i}
\bra{l, m'} \lr{ L_{+} - L_{-} } \ket{l, m} \\
&\qquad=
\frac{C}{2 i}
\bra{l, m'}
\lr{
L_{+}
\ket{l, m}
- L_{-}
\ket{l, m}
} \\
&\qquad=
\frac{C \Hbar}{2 i}
\bra{l, m'}
\biglr{
\sqrt{(l - m)(l + m + 1)} \ket{l, m + 1} \\
&\qquad\qquad-
\sqrt{(l + m)(l - m + 1)} \ket{l, m - 1}
} \\
&\qquad=
\frac{C \Hbar}{2 i}
\lr{
\sqrt{(l - m)(l + m + 1)} \delta_{m', m + 1}
-
\sqrt{(l + m)(l - m + 1)} \delta_{m', m - 1}
}.
\end{aligned}
\end{equation}
%
After squaring and summing, the cross terms will be zero since they involve products of delta functions with different indices.  That leaves
%
\begin{dmath}\label{eqn:LyPerturbation:140}
\Delta^2
=
\frac{C^2 \Hbar}{4 B} \sum_{m' \ne m} \frac{
(l - m)(l + m + 1) \delta_{m', m + 1}
-
(l + m)(l - m + 1) \delta_{m', m - 1}
}{
m - m'
}
=
\frac{C^2 \Hbar}{4 B}
\lr{
\frac{ (l - m)(l + m + 1) }{ m - (m+1) }
-
\frac{ (l + m)(l - m + 1) }{ m - (m-1)}
}
=
\frac{C^2 \Hbar}{4 B}
\lr{
-
(l^2 - m^2 + l - m)
-
(l^2 - m^2 + l + m)
}
=
-\frac{C^2 \Hbar}{2 B} (l^2 - m^2 + l ),
\end{dmath}
%
so to first order the energy shift is
%
\boxedEquation{eqn:LyPerturbation:160}{
A \Hbar^2 l(l+1) + B \Hbar m \rightarrow
\Hbar l(l+1)
\lr{
A \Hbar
-\frac{C^2}{2 B}
}
+ B \Hbar m
+\frac{C^2 m^2 \Hbar}{2 B}.
}

\paragraph{Exact perturbation equation}

If we wanted to solve the Hamiltonian exactly, we've have to diagonalize the \( 2 m + 1 \) dimensional Hamiltonian
%
\begin{dmath}\label{eqn:LyPerturbation:180}
\bra{l, m'} H \ket{l, m}
=
\lr{ A \Hbar^2 l(l+1) + B \Hbar m } \delta_{m', m}
+
\frac{C \Hbar}{2 i}
\lr{
\sqrt{(l - m)(l + m + 1)} \delta_{m', m + 1}
-
\sqrt{(l + m)(l - m + 1)} \delta_{m', m - 1}
}.
\end{dmath}
%
This Hamiltonian matrix has a very regular structure
%
\begin{equation}\label{eqn:LyPerturbation:200}
\begin{aligned}
H &=
(A l(l+1) \Hbar^2 - B \Hbar (l+1)) I \\
&+ B \Hbar
\begin{bmatrix}
1 &   &   &        &          \\
  & 2 &   &        &          \\
  &   & 3 &        &          \\
  &   &   & \ddots &          \\
  &   &   &        & 2 l + 1
\end{bmatrix} \\
&+
\frac{C \Hbar}{i}
%\begin{bmatrix}
%0                & -\sqrt{(2l-1)(1)} &                  &                    & \\
%\sqrt{(2l-1)(1)} & 0                 & -\sqrt{(2l-2)(2)}&                    & \\
%                 & \sqrt{(2l-2)(2)}  &                  &                    & \\
%                 &                   & \ddots           &                    & \\
%&                 &                   &  0               & - \sqrt{(1)(2l-1)}  \\
%&                 &                   & \sqrt{(1)(2l-1)} & 0
%\end{bmatrix}
\begin{bmatrix}
0                & -c_{2l-1,1} &                  &                    & \\
c_{2l-1,1} & 0                 & -c_{2l-2,2}&                    & \\
                 & c_{2l-2,2}  &                  &                    & \\
                 &                   & \ddots           &                    & \\
&                 &                   &  0               & - c_{1,2l-1}  \\
&                 &                   & c_{1,2l-1} & 0
\end{bmatrix},
\end{aligned}
\end{equation}
where \( c_{a, b} = \sqrt{ a b } \).

Solving for the eigenvalues of this Hamiltonian for increasing \( l \) in Mathematica (\nbref{sakuraiProblem5.17a.nb}), it appears that the eigenvalues are
%
\begin{dmath}\label{eqn:LyPerturbation:220}
\lambda_m = A \Hbar^2 (l)(l+1) + \Hbar m B \sqrt{ 1 + \frac{4 C^2}{B^2} },
\end{dmath}
%
so to first order in \( C^2 \), these are
%
\begin{dmath}\label{eqn:LyPerturbation:221}
\lambda_m = A \Hbar^2 (l)(l+1) + \Hbar m B \lr{ 1 + \frac{2 C^2}{B^2} }.
\end{dmath}
%
We have a \( C^2 \Hbar/B \) term in both the perturbative energy shift
\cref{eqn:LyPerturbation:140}, and the first order expansion of the exact solution
\cref{eqn:LyPerturbation:220}.
Comparing this for the \( l = 5 \) case, the coefficients of \( C^2 \Hbar/B \) in \cref{eqn:LyPerturbation:140} are all negative
\begin{dmath}\label{eqn:LyPerturbation:n}
-17.5, -17., -16.5, -16., -15.5, -15., -14.5, -14., -13.5, -13., -12.5,
\end{dmath}
whereas the coefficient of \( C^2 \Hbar/B \) in the first order expansion of the exact solution \cref{eqn:LyPerturbation:220} are \( 2 m \), ranging from \( [-10, 10] \).
} % answer

%\EndArticle

         % p5.18
         %
% Copyright � 2015 Peeter Joot.  All Rights Reserved.
% Licenced as described in the file LICENSE under the root directory of this GIT repository.
%
%\input{../blogpost.tex}
%\renewcommand{\basename}{quadraticZeeman}
%\renewcommand{\dirname}{notes/phy1520/}
%%\newcommand{\dateintitle}{}
%%\newcommand{\keywords}{}
%
%\input{../peeter_prologue_print2.tex}
%
%\usepackage{peeters_layout_exercise}
%\usepackage{peeters_braket}
%\usepackage{peeters_figures}
%
%\beginArtNoToc
%
%\generatetitle{Quadratic Zeeman effect}
%%\chapter{Quadradic Zeeman effect}
%%\label{chap:quadraticZeeman}

\makeoproblem{Quadratic Zeeman effect.}{problem:quadraticZeeman:1}{\citep{sakurai2014modern} pr. 5.18}{
\index{Zeeman effect!quadratic}

Work out the quadratic Zeeman effect for the ground state hydrogen atom due to the usually neglected \( e^2 \BA^2/2 m_e c^2 \) term in the Hamiltonian.

} % problem

\makeanswer{problem:quadraticZeeman:1}{

The first order energy shift is

For a z-oriented magnetic field we can use
%
\begin{dmath}\label{eqn:quadraticZeeman:20}
\BA = \frac{B}{2} \lr{ -y, x, 0 },
\end{dmath}

so the perturbation potential is
%
\begin{dmath}\label{eqn:quadraticZeeman:40}
V
= \frac{e^2 \BA^2}{2 m_e c^2}
= \frac{e^2 \BB^2 (x^2 + y^2)}{8 m_e c^2}
= \frac{ e^2 \BB^2 r^2 \sin^2\theta }{8 m_e c^2}
\end{dmath}

The ground state wave function is
%
\begin{dmath}\label{eqn:quadraticZeeman:60}
\psi_0
= \braket{\Bx}{0}
= \inv{\sqrt{\pi a_0^3}} e^{-r/a_0},
\end{dmath}

so the energy shift is
%
\begin{dmath}\label{eqn:quadraticZeeman:80}
\Delta
= \bra{0} V \ket{0}
= \inv{ \pi a_0^3 } 2 \pi \frac{ e^2 \BB^2 }{8 m_e c^2} \int_0^\infty r^2 \sin\theta e^{-2r/a_0} r^2 \sin^2\theta dr d\theta
=
\frac{ e^2 \BB^2 }{4 a_0^3 m_e c^2}
\int_0^\infty r^4 e^{-2r/a_0} dr \int_0^\pi \sin^3\theta d\theta
= -
\frac{ e^2 \BB^2 }{4 a_0^3 m_e c^2}
\frac{4!}{(2/a_0)^{4+1} } \evalrange{\lr{u - \frac{u^3}{3}}}{1}{-1}
=
\frac{ e^2 a_0^2 \BB^2 }{4 m_e c^2}.
\end{dmath}

If this energy shift is written in terms of a diamagnetic susceptibility \( \chi \) defined by
%
\begin{dmath}\label{eqn:quadraticZeeman:100}
\Delta = -\inv{2} \chi \BB^2,
\end{dmath}

the diamagnetic susceptibility is
%
\begin{dmath}\label{eqn:quadraticZeeman:120}
\chi = -\frac{ e^2 a_0^2 }{2 m_e c^2}.
\end{dmath}
} % answer

%\EndArticle

         % final p6:
         %
% Copyright � 2015 Peeter Joot.  All Rights Reserved.
% Licenced as described in the file LICENSE under the root directory of this GIT repository.
%
%{
%\input{../blogpost.tex}
%\renewcommand{\basename}{hyperfineElectronAndBosonNucleus}
%\renewcommand{\dirname}{notes/phy1520/}
%%\newcommand{\dateintitle}{}
%%\newcommand{\keywords}{}
%
%\input{../peeter_prologue_print2.tex}
%
%\usepackage{peeters_layout_exercise}
%\usepackage{peeters_braket}
%\usepackage{peeters_figures}
%
%\beginArtNoToc
%
%\generatetitle{Electron and Boson hyperfine spin interaction}
%%\chapter{Electron and Boson hyperfine spin interaction}
%%\label{chap:hyperfineElectronAndBosonNucleus}

\makeoproblem{Electron and Boson hyperfine spin interaction}{problem:hyperfineElectronAndBosonNucleus:1}{2015 final exam}{

This problem is a variation of problem set 8 problem 4, but instead of the spin interaction of a Fermionic nucleus with an electron, this problem was to look at the spin interaction of an electron with a Bosonic nucleus.

The interaction Hamiltonian in this problem includes a Zeeman field
%
\begin{equation}\label{eqn:hyperfineElectronAndBosonNucleus:20}
H = J \BS_e \cdot \BS_n + B S_e^z,
\end{equation}
%
\makesubproblem{}{problem:hyperfineElectronAndBosonNucleus:1:a}
Solve the problem exactly.

\makesubproblem{}{problem:hyperfineElectronAndBosonNucleus:1:b}
Do a first order perturbative solution.

} % problem

\makeanswer{problem:hyperfineElectronAndBosonNucleus:1}{
\withproblemsetsParagraph{

\makeSubAnswer{}{problem:hyperfineElectronAndBosonNucleus:1:a}
With two states for the electron spin and three for the nuclear spin, the state space for the spin system is six dimensional.  Without the Zeeman contribution the action of Hamiltonian is
%
\begin{equation}\label{eqn:hyperfineElectronAndBosonNucleus:40}
\begin{aligned}
H \ket{\ncap; + } \ket{1} &= \frac{J \Hbar^2}{2} \ket{\ncap; + } \ket{1} \\
H \ket{\ncap; - } \ket{1} &= -\frac{J \Hbar^2}{2} \ket{\ncap; - } \ket{1} \\
H \ket{\ncap; + } \ket{0} &= 0 \ket{\ncap; + } \ket{0} \\
H \ket{\ncap; - } \ket{0} &= 0 \ket{\ncap; - } \ket{0} \\
H \ket{\ncap; + } \ket{-1} &= -\frac{J \Hbar^2}{2} \ket{\ncap; + } \ket{-1} \\
H \ket{\ncap; - } \ket{-1} &= \frac{J \Hbar^2}{2} \ket{\ncap; - } \ket{-1} \\
\end{aligned}
\end{equation}

The can be put into block matrix form
%
\begin{equation}\label{eqn:hyperfineElectronAndBosonNucleus:60}
J \BS_e \cdot \BS_n
=
\frac{J \Hbar^2}{2}
\begin{bmatrix}
\sigma_z & 0 & 0 \\
0        & 0 & 0 \\
0        & 0 & -\sigma_z \\
\end{bmatrix}.
\end{equation}
%
The Zeeman field in the basis above is
%
\begin{equation}\label{eqn:hyperfineElectronAndBosonNucleus:80}
B S_e^z
=
\frac{B \Hbar}{2}
\begin{bmatrix}
R & 0 & 0 \\
0 & R & 0 \\
0 & 0 & R \\
\end{bmatrix},
\end{equation}
%
where \( R \) is the representation of \( \sigma_z \) in a \( \ket{\ncap; + }, \ket{\ncap; - } \) basis.  Representing the electron spin states as
%
\begin{dmath}\label{eqn:hyperfineElectronAndBosonNucleus:100}
\begin{aligned}
\ket{\ncap; +} &=
\begin{bmatrix}
\cos(\theta/2) e^{-i\phi} \\
\sin(\theta/2)
\end{bmatrix}
=
\cos(\theta/2) e^{-i\phi} \ket{\zcap;+}
+
\sin(\theta/2) \ket{\zcap;-}
\\
\ket{\ncap; -} &=
\begin{bmatrix}
-\sin(\theta/2) e^{-i\phi} \\
\cos(\theta/2)
\end{bmatrix}
=
-\sin(\theta/2) e^{-i\phi} \ket{\zcap;+}
+ \cos(\theta/2) \ket{\zcap;-},
\end{aligned}
\end{dmath}

so the \( S_e^z \) action on the spin states is
\begin{dmath}\label{eqn:hyperfineElectronAndBosonNucleus:120}
S_e^z \ket{\ncap;+}
=
\frac{\Hbar}{2} \lr{
\cos(\theta/2) e^{-i\phi} \ket{\zcap;+}
-
\sin(\theta/2) \ket{\zcap;-}
}
=
\frac{\Hbar}{2}
\begin{bmatrix}
\cos(\theta/2) e^{-i\phi} \\
-\sin(\theta/2)
\end{bmatrix},
\end{dmath}
%
and
\begin{dmath}\label{eqn:hyperfineElectronAndBosonNucleus:140}
S_e^z \ket{\ncap;-}
=
\frac{\Hbar}{2} \lr{
-\sin(\theta/2) e^{-i\phi} \ket{\zcap;+}
-
\cos(\theta/2) \ket{\zcap;-}
}
=
\frac{\Hbar}{2}
\begin{bmatrix}
-\sin(\theta/2) e^{-i\phi} \\
-\cos(\theta/2)
\end{bmatrix}.
\end{dmath}
%
The matrix elements for \( S_e^z \) are
%
\begin{dmath}\label{eqn:hyperfineElectronAndBosonNucleus:200}
\bra{\ncap;+} S_e^z \ket{\ncap;+}
=
\frac{\Hbar}{2}
\begin{bmatrix}
\cos(\theta/2) e^{i\phi} &
\sin(\theta/2)
\end{bmatrix}
\begin{bmatrix}
\cos(\theta/2) e^{-i\phi} \\
-\sin(\theta/2)
\end{bmatrix}
=
\frac{\Hbar}{2} \cos\theta,
\end{dmath}
%
\begin{dmath}\label{eqn:hyperfineElectronAndBosonNucleus:160}
\bra{\ncap;+} S_e^z \ket{\ncap;-}
=
\frac{\Hbar}{2}
\begin{bmatrix}
\cos(\theta/2) e^{i\phi} &
\sin(\theta/2)
\end{bmatrix}
\begin{bmatrix}
-\sin(\theta/2) e^{-i\phi} \\
-\cos(\theta/2)
\end{bmatrix}
=
-\frac{\Hbar}{2} \sin\theta,
\end{dmath}
%
\begin{dmath}\label{eqn:hyperfineElectronAndBosonNucleus:220}
\bra{\ncap;-} S_e^z \ket{\ncap;+}
=
\frac{\Hbar}{2}
\begin{bmatrix}
-\sin(\theta/2) e^{i\phi} &
\cos(\theta/2)
\end{bmatrix}
\begin{bmatrix}
\cos(\theta/2) e^{-i\phi} \\
-\sin(\theta/2)
\end{bmatrix}
=
-\frac{\Hbar}{2} \sin\theta,
\end{dmath}
%
\begin{dmath}\label{eqn:hyperfineElectronAndBosonNucleus:180}
\bra{\ncap;-} S_e^z \ket{\ncap;-}
=
\frac{\Hbar}{2}
\begin{bmatrix}
-\sin(\theta/2) e^{i\phi} &
\cos(\theta/2)
\end{bmatrix}
\begin{bmatrix}
-\sin(\theta/2) e^{-i\phi} \\
-\cos(\theta/2)
\end{bmatrix}
=
-\frac{\Hbar}{2} \cos\theta,
\end{dmath}
%
or
\begin{dmath}\label{eqn:hyperfineElectronAndBosonNucleus:240}
R =
\begin{bmatrix}
\cos\theta & -\sin\theta \\
- \sin\theta & -\cos\theta
\end{bmatrix}.
\end{dmath}
%
%\paragraph{Exact solution, and strong and weak field approximations.}
\makeSubAnswer{}{problem:hyperfineElectronAndBosonNucleus:1:b}


In \nbref{finalExamProblem6HyperfineInteraction.nb} the exact solution to the system is found to be
%
\begin{dmath}\label{eqn:hyperfineElectronAndBosonNucleus:260}
E \in \setlr{ \pm \frac{B\Hbar}{2}, \pm \frac{\Hbar}{2} \sqrt{ B^2 \pm 2 B J \Hbar \cos\theta + J^2 \Hbar^2 } },
\end{dmath}
%
or with \( m \in \setlr{-1, 0, 1} \)
%
\begin{dmath}\label{eqn:hyperfineElectronAndBosonNucleus:280}
E \in \pm \frac{\Hbar}{2} \sqrt{ B^2 + 2 B J m \Hbar \cos\theta + m^2 J^2 \Hbar^2 }.
\end{dmath}
%
In the strong field case where \( B \gg \Hbar J \), we have
\begin{dmath}\label{eqn:hyperfineElectronAndBosonNucleus:300}
E \approx \pm \frac{\Hbar B}{2} \lr{ 1 + \frac{J}{B} m \Hbar \cos\theta + \inv{2} m^2 \frac{J^2}{B^2} \Hbar^2 },
\end{dmath}
%
whereas in the weak field case where \( B \ll \Hbar J \), we have
\begin{dmath}\label{eqn:hyperfineElectronAndBosonNucleus:320}
E \approx
\left\{
\begin{array}{l l}
\pm \frac{\Hbar B}{2} & \quad \mbox{if \( m = 0\) } \\
\pm \frac{\Hbar^2 J}{2} \lr{
1
+ \frac{B}{\Hbar J} m \cos\theta
+ \inv{2} \frac{B^2}{J^2 \Hbar^2} }
 & \quad \mbox{if \( m \ne 0\),}
\end{array}
\right.
\end{dmath}
%
Note that the result above is exact for the \( m = 0 \) case.

\paragraph{Perturbative solutions}

The degenerate perturbation requires diagonalizing \( B S_e^z \).  Since \( R \) is a representation of \( S_e^z \), the eigenvalues of \( \frac{ B \Hbar R}{2} \) are \( \pm B \Hbar /2 \).

This means that the first order shifts to the energy eigenvalues are
%
\begin{equation}\label{eqn:hyperfineElectronAndBosonNucleus:340}
\begin{aligned}
\frac{J \Hbar^2}{2} &\rightarrow \frac{J \Hbar^2}{2} + \frac{B \Hbar}{2} \\
-\frac{J \Hbar^2}{2} &\rightarrow -\frac{J \Hbar^2}{2} - \frac{B \Hbar}{2} \\
0 &\rightarrow + \frac{B \Hbar}{2} \\
0 &\rightarrow - \frac{B \Hbar}{2} \\
-\frac{J \Hbar^2}{2} &\rightarrow -\frac{J \Hbar^2}{2} + \frac{B \Hbar}{2} \\
\frac{J \Hbar^2}{2} &\rightarrow \frac{J \Hbar^2}{2} - \frac{B \Hbar}{2}.
\end{aligned}
\end{equation}
}
} % answer

%}
%\EndNoBibArticle

