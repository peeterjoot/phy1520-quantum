%
% Copyright � 2015 Peeter Joot.  All Rights Reserved.
% Licenced as described in the file LICENSE under the root directory of this GIT repository.
%
%\input{../blogpost.tex}
%\renewcommand{\basename}{magneticLorentzForceLagrangian}
%\renewcommand{\dirname}{notes/phy1520/}
%%\newcommand{\dateintitle}{}
%%\newcommand{\keywords}{}
%
%\input{../peeter_prologue_print2.tex}
%
%\usepackage{peeters_layout_exercise}
%\usepackage{peeters_braket}
%\usepackage{peeters_figures}
%
%\beginArtNoToc

%\generatetitle{Lagrangian for magnetic portion of Lorentz force}
%\label{chap:magneticLorentzForceLagrangian}

In \citep{sakurai2014modern} it is claimed in an Aharonov-Bohm discussion that a Lagrangian modification to include electromagnetism is

\begin{dmath}\label{eqn:magneticLorentzForceLagrangian:20}
\LL \rightarrow \LL + \frac{e}{c} \Bv \cdot \BA.
\end{dmath}

That can't be the full Lagrangian since there is no \( \phi \) term, so what exactly do we get?

If you have somehow, like I did, forgot the exact form of the Euler-Lagrange equations (i.e. where do the dots go), then the derivation of those equations can come to your rescue.  The starting point is the action

\begin{dmath}\label{eqn:magneticLorentzForceLagrangian:40}
S = \int \LL(x, \xdot, t) dt,
\end{dmath}

where the end points of the integral are fixed, and we assume we have no variation at the end points.  The variational calculation is

\begin{dmath}\label{eqn:magneticLorentzForceLagrangian:60}
\delta S
= \int \delta \LL(x, \xdot, t) dt
= \int \lr{ \PD{x}{\LL} \delta x + \PD{\xdot}{\LL} \delta \xdot } dt
= \int \lr{ \PD{x}{\LL} \delta x + \PD{\xdot}{\LL} \delta \ddt{x} } dt
= \int \lr{ \PD{x}{\LL} - \ddt{}\lr{\PD{\xdot}{\LL}} } \delta x dt
+ \delta x \PD{\xdot}{\LL}.
\end{dmath}

The boundary term is killed after evaluation at the end points where the variation is zero.  For the result to hold for all variations \( \delta x \), we must have

%\begin{dmath}\label{eqn:magneticLorentzForceLagrangian:80}
\boxedEquation{eqn:magneticLorentzForceLagrangian:80}{
\PD{x}{\LL} = \ddt{}\lr{\PD{\xdot}{\LL}}.
}
%\end{dmath}

Now lets apply this to the Lagrangian at hand.  For the position derivative we have

\begin{dmath}\label{eqn:magneticLorentzForceLagrangian:100}
\PD{x_i}{\LL}
=
\frac{e}{c} v_j \PD{x_i}{A_j}.
\end{dmath}

For the canonical momentum term, assuming \( \BA = \BA(\Bx) \) we have

\begin{dmath}\label{eqn:magneticLorentzForceLagrangian:120}
\ddt{} \PD{\xdot_i}{\LL}
=
\ddt{}
\lr{ m \xdot_i
+
\frac{e}{c} A_i
}
=
m \ddot{x}_i
+
\frac{e}{c}
\ddt{A_i}
=
m \ddot{x}_i
+
\frac{e}{c}
\PD{x_j}{A_i} \frac{dx_j}{dt}.
\end{dmath}

Assembling the results, we've got

\begin{dmath}\label{eqn:magneticLorentzForceLagrangian:140}
0
=
\ddt{} \PD{\xdot_i}{\LL}
-
\PD{x_i}{\LL}
=
m \ddot{x}_i
+
\frac{e}{c}
\PD{x_j}{A_i} \frac{dx_j}{dt}
-
\frac{e}{c} v_j \PD{x_i}{A_j},
\end{dmath}

or
\begin{dmath}\label{eqn:magneticLorentzForceLagrangian:160}
m \ddot{x}_i
=
\frac{e}{c} v_j \PD{x_i}{A_j}
-
\frac{e}{c}
\PD{x_j}{A_i} v_j
=
\frac{e}{c} v_j
\lr{
\PD{x_i}{A_j}
-
\PD{x_j}{A_i}
}
=
\frac{e}{c} v_j B_k \epsilon_{i j k}.
\end{dmath}

In vector form that is

%\begin{dmath}\label{eqn:magneticLorentzForceLagrangian:180}
\boxedEquation{eqn:magneticLorentzForceLagrangian:180}{
m \ddot{\Bx}
=
\frac{e}{c} \Bv \cross \BB.
}
%\end{dmath}

So, we get the magnetic term of the Lorentz force.  Also note that this shows the Lagrangian (and the end result), was not in SI units.  The \( 1/c \) term would have to be dropped for SI.

%\EndArticle
