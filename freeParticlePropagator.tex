%
% Copyright � 2015 Peeter Joot.  All Rights Reserved.
% Licenced as described in the file LICENSE under the root directory of this GIT repository.
%
%\input{../blogpost.tex}
%\renewcommand{\basename}{freeParticlePropagator}
%\renewcommand{\dirname}{notes/phy1520/}
%%\newcommand{\dateintitle}{}
%%\newcommand{\keywords}{}
%
%\input{../peeter_prologue_print2.tex}
%
%\usepackage{peeters_layout_exercise}
%\usepackage{peeters_braket}
%\usepackage{peeters_figures}
%
%\beginArtNoToc
%
%\generatetitle{Free particle propagator}
%%\chapter{Free particle propagator}
%\label{chap:freeParticlePropagator}
%
\makeoproblem{Free particle propagator.}{problem:freeParticlePropagator:31}{\citep{sakurai2014modern} pr. 2.31}{
\index{propagator!free particle}
Derive the free particle propagator in one and three dimensions.
} % problem
%
\makeanswer{problem:freeParticlePropagator:31}{
%
I found the description in the text confusing, so let's start from scratch with the definition of the propagator.  This is the kernel of the spatial convolution integral that encodes time evolution, and can be expressed by expanding a general time state with two sets of identity operators.  Let the position relative state at time \( t \), relative to an initial time \( t_0 \) be given by \( \braket{\Bx}{\alpha, t ; t_0 } \), and expand this in terms of a complete basis of energy eigenstates \( \ket{a'} \) and the time evolution operator
%
\begin{dmath}\label{eqn:freeParticlePropagator:20}
\begin{aligned}
\braket{\Bx''}{\alpha, t ; t_0 }
&= \bra{\Bx''} U \ket{\alpha, t_0 } \\
&= \bra{\Bx''} e^{-i H (t -t_0)/\Hbar} \ket{\alpha, t_0 } \\
&= \bra{\Bx''} e^{-i H (t -t_0)/\Hbar} \lr{ \sum_{a'} \ket{a'} \bra{a' }} \ket{\alpha, t_0 } \\
&= \bra{\Bx''} \sum_{a'} e^{-i E_{a'} (t -t_0)/\Hbar} \ket{a'} \braket{a' }{\alpha, t_0 } \\
&=
\bra{\Bx''} \sum_{a'} e^{-i E_{a'} (t -t_0)/\Hbar} \ket{a'} \bra{a' }
\lr{ \int d^3 \Bx'
\ket{\Bx'}\bra{\Bx'}
}
\ket{\alpha, t_0 } \\
&=
\int d^3 \Bx'
\lr{
\bra{\Bx''} \sum_{a'} e^{-i E_{a'} (t -t_0)/\Hbar} \ket{a'} \braket{a' }{\Bx'}
}
\braket{\Bx'}{\alpha, t_0 } \\
&=
\int d^3 \Bx' K(\Bx'', t ; \Bx', t_0) \braket{\Bx'}{\alpha, t_0 },
\end{aligned}
\end{dmath}
where
%
\begin{dmath}\label{eqn:freeParticlePropagator:40}
K(\Bx'', t ; \Bx', t_0) =
\sum_{a'}
\braket{\Bx''}{a'}\braket{a' }{\Bx'}
e^{-i E_{a'} (t -t_0)/\Hbar},
\end{dmath}
%
the propagator, is the kernel of the convolution integral that takes the state \( \ket{\alpha, t_0} \) to state \( \ket{\alpha, t ; t_0} \).  Evaluating this over the momentum states (where integration and not plain summation is required), we have
%
\begin{dmath}\label{eqn:freeParticlePropagator:60}
\begin{aligned}
K(\Bx'', t ; \Bx', t_0)
&=
\int d^3 \Bp'
\braket{\Bx''}{\Bp'}\braket{\Bp' }{\Bx'}
e^{-i E_{\Bp'} (t -t_0)/\Hbar} \\
&=
\int d^3 \Bp'
\braket{\Bx''}{\Bp'}\braket{\Bp' }{\Bx'}
\exp\lr{-i \frac{(\Bp')^2 (t -t_0)}{2 m \Hbar}} \\
&=
\int d^3 \Bp'
\frac{e^{i \Bx'' \cdot \Bp'/\Hbar}}{(\sqrt{2 \pi \Hbar})^3}
\frac{e^{-i \Bx' \cdot \Bp'/\Hbar}}{(\sqrt{2 \pi \Hbar})^3}
\exp\lr{-i \frac{(\Bp')^2 (t -t_0)}{2 m \Hbar}} \\
&=
\inv{(2 \pi \Hbar)^3}
\int d^3 \Bp'
e^{i (\Bx'' -\Bx') \cdot \Bp'/\Hbar}
\exp\lr{-i \frac{(\Bp')^2 (t -t_0)}{2 m \Hbar}} \\
&=
\inv{ 2 \pi \Hbar }
\int_{-\infty}^\infty dp_1'
e^{i (x_1'' -x_1') p_1'/\Hbar}
\exp\lr{-i \frac{(p_1')^2 (t -t_0)}{2 m \Hbar}} \times \\
&\quad \inv{ 2 \pi \Hbar }
\int_{-\infty}^\infty dp_2'
e^{i (x_2'' -x_2') p_2'/\Hbar}
\exp\lr{-i \frac{(p_2')^2 (t -t_0)}{2 m \Hbar}} \times \\
&\quad \inv{ 2 \pi \Hbar }
\int_{-\infty}^\infty dp_3'
e^{i (x_3'' -x_3') p_3'/\Hbar}
\exp\lr{-i \frac{(p_3')^2 (t -t_0)}{2 m \Hbar}}.
\end{aligned}
\end{dmath}
With \( a = \ifrac{(t -t_0)}{2 m \Hbar} \), each of these three integral factors is of the form
%
\begin{equation}\label{eqn:freeParticlePropagator:80}
\begin{aligned}
\inv{ 2 \pi \Hbar } &
\int_{-\infty}^\infty dp
e^{i \Delta x p/\Hbar }
\exp\lr{-i a p^2} \\
&=
\inv{2 \pi \Hbar \sqrt{a}}
\int_{-\infty}^\infty du
e^{i \Delta x u/(\sqrt{a}\Hbar) }
\exp\lr{-i u^2} \\
&=
\inv{2 \pi \Hbar \sqrt{a}}
\int_{-\infty}^\infty du
e^{i \Delta x u/(\sqrt{a} \Hbar) }
\exp\lr{-i (u - \Delta x /(2\sqrt{a}\Hbar))^2 + i(\Delta x/(2\sqrt{a}\Hbar))^2} \\
&=
\inv{2 \pi \Hbar \sqrt{a}}
\exp\lr{ \frac{i(\Delta x)^2 2 m \Hbar}{4 (t -t_0) \Hbar^2} }
\int_{-\infty}^\infty dz
e^{-i z^2} \\
&= \sqrt{ \frac{ -i \pi 2 m \Hbar}{ 4 \pi^2 \Hbar^2 (t -t_0)} }
\exp\lr{ \frac{i(\Delta x)^2 m}{2 (t -t_0) \Hbar} } \\
&= \sqrt{ \frac{ m }{ 2 \pi i \Hbar (t -t_0)} }
\exp\lr{ \frac{i(\Delta x)^2 m}{2 (t -t_0) \Hbar} }.
\end{aligned}
\end{equation}
%
Note that the integral above has value \( \sqrt{-i\pi} \) which can be found by integrating over the contour of \cref{fig:contour:contourFig1}, letting \( R \rightarrow \infty \).

\imageFigure{../figures/phy1520-quantum/contourFig1}{Integration contour for \( \int e^{-i z^2} \).}{fig:contour:contourFig1}{0.2}

Multiplying out each of the spatial direction factors gives the propagator in its closed form
\boxedEquation{eqn:freeParticlePropagator:120}{
K(\Bx'', t ; \Bx', t_0)
= \lr{ \sqrt{ \frac{ m }{ 2 \pi i \Hbar (t -t_0)} } }^3
\exp\lr{ \frac{i(\Bx'' - \Bx')^2 m}{2 (t -t_0) \Hbar} }.
}
%\begin{equation}\label{eqn:freeParticlePropagator:120}
%\boxed{
%}
%\end{equation}

In one or two dimensions the exponential power \( 3 \) need only be adjusted appropriately.
} % answer
