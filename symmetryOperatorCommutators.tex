%
% Copyright � 2015 Peeter Joot.  All Rights Reserved.
% Licenced as described in the file LICENSE under the root directory of this GIT repository.
%
%{
%\input{../blogpost.tex}
%\renewcommand{\basename}{symmetryOperatorCommutators}
%\renewcommand{\dirname}{notes/phy1520/}
%%\newcommand{\dateintitle}{}
%%\newcommand{\keywords}{}
%
%\input{../peeter_prologue_print2.tex}
%
%\usepackage{peeters_layout_exercise}
%\usepackage{peeters_braket}
%\usepackage{peeters_figures}
%\usepackage{enumerate}
%\usepackage{macros_cal}
%
%\beginArtNoToc
%
%\generatetitle{Commutators for some symmetry operators}
%\chapter{Commutators for some symmetry operators}
%\label{chap:symmetryOperatorCommutators}

\makeoproblem{Commutators for some symmetry operators.}{problem:symmetryOperatorCommutators:1}{\citep{sakurai2014modern} pr. 4.2}{

If \( \calT_\Bd \), \( \calD(\ncap, \phi) \), and \( \pi \) denote the translation, rotation, and parity operators respectively.  Which of the following commute and why

\makesubproblem{}{problem:symmetryOperatorCommutators:1:a}
\( \calT_\Bd \) and \( \calT_{\Bd'} \), translations in different directions.
\makesubproblem{}{problem:symmetryOperatorCommutators:1:b}
\( \calD(\ncap, \phi) \) and \( \calD(\ncap', \phi') \), rotations in different directions.
\makesubproblem{}{problem:symmetryOperatorCommutators:1:c}
\( \calT_\Bd \) and \( \pi \).
\makesubproblem{}{problem:symmetryOperatorCommutators:1:d}
\( \calD(\ncap,\phi)\) and \( \pi \).

} % problem

\makeanswer{problem:symmetryOperatorCommutators:1}{

\makeSubAnswer{}{problem:symmetryOperatorCommutators:1:a}

Consider
\begin{dmath}\label{eqn:symmetryOperatorCommutators:20}
\calT_\Bd \calT_{\Bd'} \ket{\Bx}
=
\calT_\Bd \ket{\Bx + \Bd'}
=
\ket{\Bx + \Bd' + \Bd},
\end{dmath}
%
and the reverse application of the translation operators
\begin{dmath}\label{eqn:symmetryOperatorCommutators:40}
\calT_{\Bd'} \calT_{\Bd} \ket{\Bx}
=
\calT_{\Bd'} \ket{\Bx + \Bd}
=
\ket{\Bx + \Bd + \Bd'}
=
\ket{\Bx + \Bd' + \Bd}.
\end{dmath}
%
so we see that
%
\begin{dmath}\label{eqn:symmetryOperatorCommutators:60}
\antisymmetric{\calT_\Bd}{\calT_{\Bd'}} \ket{\Bx} = 0,
\end{dmath}
%
for any position state \( \ket{\Bx} \), and therefore in general they commute.

\makeSubAnswer{}{problem:symmetryOperatorCommutators:1:b}

That rotations do not commute when they are in different directions (like any two orthogonal directions) need not be belaboured.

\makeSubAnswer{}{problem:symmetryOperatorCommutators:1:c}
We have
\begin{dmath}\label{eqn:symmetryOperatorCommutators:80}
\calT_\Bd \pi \ket{\Bx}
=
\calT_\Bd \ket{-\Bx}
=
\ket{-\Bx + \Bd},
\end{dmath}
%
yet
\begin{dmath}\label{eqn:symmetryOperatorCommutators:100}
\pi \calT_\Bd \ket{\Bx}
=
\pi \ket{\Bx + \Bd}
=
\ket{-\Bx - \Bd}
\ne
\ket{-\Bx + \Bd}.
\end{dmath}
%
so, in general \( \antisymmetric{\calT_\Bd}{\pi} \ne 0 \).

\makeSubAnswer{}{problem:symmetryOperatorCommutators:1:d}

We have
%
\begin{dmath}\label{eqn:symmetryOperatorCommutators:120}
\pi \calD(\ncap, \phi) \ket{\Bx}
=
\pi \calD(\ncap, \phi) \pi^\dagger \pi \ket{\Bx}
=
\pi \calD(\ncap, \phi) \pi^\dagger \pi \ket{\Bx}
=
\pi \lr{ \sum_{k=0}^\infty \frac{(-i \BJ \cdot \ncap)^k}{k!} } \pi^\dagger \pi \ket{\Bx}
=
\sum_{k=0}^\infty \frac{(-i (\pi \BJ \pi^\dagger) \cdot (\pi \ncap \pi^\dagger) )^k}{k!} \pi \ket{\Bx}
=
\sum_{k=0}^\infty \frac{(-i \BJ \cdot \ncap)^k}{k!} \pi \ket{\Bx}
=
\calD(\ncap, \phi) \pi \ket{\Bx},
\end{dmath}
%
so \( \antisymmetric{\calD(\ncap, \phi)}{\pi} \ket{\Bx} = 0 \), for any position state \( \ket{\Bx} \), and therefore these operators commute in general.
} % answer

%}
%\EndArticle
