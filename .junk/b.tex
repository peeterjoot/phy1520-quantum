%
% Copyright � 2021 Peeter Joot.  All Rights Reserved.
% Licenced as described in the file LICENSE under the root directory of this GIT repository.
%
%{
\input{../latex/blogpost.tex}
\renewcommand{\basename}{b}
%\renewcommand{\dirname}{notes/phy1520/}
\renewcommand{\dirname}{notes/ece1228-electromagnetic-theory/}
%\newcommand{\dateintitle}{}
%\newcommand{\keywords}{}

\input{../latex/peeter_prologue_print2.tex}

\usepackage{peeters_layout_exercise}
\usepackage{peeters_braket}
\usepackage{peeters_figures}
\usepackage{siunitx}
\usepackage{verbatim}
%\usepackage{mhchem} % \ce{}
%\usepackage{macros_bm} % \bcM
%\usepackage{macros_qed} % \qedmarker
%\usepackage{txfonts} % \ointclockwise

\beginArtNoToc
\generatetitle{XXX}
The concrete examples above give some intuition for solving the more abstract problem.  Suppose that we are working in a basis that simultaneously diagonalizes operator \( A \) and the Hamiltonian \( H \).  To make life easy consider the simplest case where this basis is also an eigenbasis for the second operator \( B \) for all but two of that operators eigenvectors.  For such a system let's write
%
\begin{equation}\label{eqn:angularMomentumAndCentralForceCommutators:160}
\begin{aligned}
H \ket{1} &= \epsilon_1 \ket{1} \\
H \ket{2} &= \epsilon_2 \ket{2} \\
A \ket{1} &= a_1 \ket{1} \\
A \ket{2} &= a_2 \ket{2},
\end{aligned}
\end{equation}
where \( \ket{1}\), and \( \ket{2} \) are not eigenkets of \( B \).  Because \( B \) also commutes with \( H \), we must have
%
\begin{dmath}\label{eqn:angularMomentumAndCentralForceCommutators:180}
H B \ket{1}
= H \sum_n \ket{n}\bra{n} B \ket{1}
= \sum_n \epsilon_n \ket{n} B_{n 1},
\end{dmath}
%
and
\begin{dmath}\label{eqn:angularMomentumAndCentralForceCommutators:200}
B H \ket{1}
= B \epsilon_1 \ket{1}
= \epsilon_1 \sum_n \ket{n}\bra{n} B \ket{1}
= \epsilon_1 \sum_n \ket{n} B_{n 1}.
\end{dmath}
%
We can now compute the action of the commutators on \( \ket{1}, \ket{2} \), 
\begin{dmath}\label{eqn:angularMomentumAndCentralForceCommutators:220}
\antisymmetric{B}{H} \ket{1}
=
\sum_n \lr{ \epsilon_1 - \epsilon_n } \ket{n} B_{n 1}.
\end{dmath}
%
Similarly
\begin{dmath}\label{eqn:angularMomentumAndCentralForceCommutators:240}
\antisymmetric{B}{H} \ket{2}
=
\sum_n \lr{ \epsilon_2 - \epsilon_n } \ket{n} B_{n 2}.
\end{dmath}
%
However, for those kets \( \ket{m} \in \setlr{ \ket{3}, \ket{4}, \cdots } \) that are eigenkets of \( B \), with \( B \ket{m} = b_m \ket{m} \), we have
%
\begin{dmath}\label{eqn:angularMomentumAndCentralForceCommutators:280}
\antisymmetric{B}{H} \ket{m}
=
B \epsilon_m \ket{m} - H b_m \ket{m}
=
b_m \epsilon_m \ket{m} - \epsilon_m b_m \ket{m}
=
0,
\end{dmath}
%
The sums in 
\cref{eqn:angularMomentumAndCentralForceCommutators:220}
and 
\cref{eqn:angularMomentumAndCentralForceCommutators:240} reduce to
\begin{dmath}\label{eqn:angularMomentumAndCentralForceCommutators:500}
\antisymmetric{B}{H} \ket{1}
=
\sum_{n=1}^2 \lr{ \epsilon_1 - \epsilon_n } \ket{n} B_{n 1}
=
\lr{ \epsilon_1 - \epsilon_2 } \ket{2} B_{2 1},
\end{dmath}
and
\begin{dmath}\label{eqn:angularMomentumAndCentralForceCommutators:520}
\antisymmetric{B}{H} \ket{2}
=
\sum_{n=1}^2 \lr{ \epsilon_2 - \epsilon_n } \ket{n} B_{n 2}
=
\lr{ \epsilon_2 - \epsilon_1 } \ket{1} B_{1 2}.
\end{dmath}
Since the commutator is zero, the matrix elements of the commutator must all be zero, in particular
\begin{equation}\label{eqn:angularMomentumAndCentralForceCommutators:260}
\begin{aligned}
   \bra{1} \antisymmetric{B}{H} \ket{1} &= \lr{ \epsilon_1 - \epsilon_2 } B_{2 1} \braket{1}{2} = 0 \\
   \bra{2} \antisymmetric{B}{H} \ket{1} &= \lr{ \epsilon_1 - \epsilon_2 } B_{2 1} \braket{1}{1} \\
   \bra{1} \antisymmetric{B}{H} \ket{2} &= \lr{ \epsilon_2 - \epsilon_1 } B_{1 2} \braket{1}{2} = 0 \\
   \bra{2} \antisymmetric{B}{H} \ket{2} &= \lr{ \epsilon_2 - \epsilon_1 } B_{1 2} \braket{2}{2} = 0.
\end{aligned}
\end{equation}
We must either have
\begin{itemize}
\item \( B_{2 1} = B_{1 2} = 0 \), or
\item \( \epsilon_1 = \epsilon_2 \).
\end{itemize}
If the first condition were true we would have
%
\begin{dmath}\label{eqn:angularMomentumAndCentralForceCommutators:300}
B \ket{1}
=
\ket{n}\bra{n} B \ket{1}
=
\ket{n} B_{n 1}
=
\ket{1} B_{1 1},
\end{dmath}
%
and \( B \ket{2} = B_{2 2} \ket{2} \).  This contradicts the requirement that \( \ket{1}, \ket{2} \) not be eigenkets of \( B \), leaving only the second option.  That second option means there must be a degeneracy in the system.
\EndNoBibArticle
