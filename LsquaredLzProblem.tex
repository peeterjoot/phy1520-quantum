%
% Copyright � 2015 Peeter Joot.  All Rights Reserved.
% Licenced as described in the file LICENSE under the root directory of this GIT repository.
%
%\input{../blogpost.tex}
%\renewcommand{\basename}{LsquaredLzProblem}
%\renewcommand{\dirname}{notes/phy1520/}
%%\newcommand{\dateintitle}{}
%%\newcommand{\keywords}{}
%
%\input{../peeter_prologue_print2.tex}
%
%\usepackage{peeters_layout_exercise}
%\usepackage{peeters_braket}
%\usepackage{peeters_figures}
%\usepackage{enumerate}
%
%\beginArtNoToc
%
%%\generatetitle{Lz and Ls eigenvalues and probabilities for a wave function}
%\generatetitle{\(L_z\) and \( \BL^2 \) eigenvalues and probabilities for a wave function}
%%\chapter{Lz and Ls eigenvalues and probabilities for a wave function}
%%\label{chap:LsquaredLzProblem}

\makeoproblem{\(L_z\) and \( \BL^2 \) eigenvalues and probabilities for a wave function.}{problem:LsquaredLzProblem:1}{\citep{sakurai2014modern} pr. 3.17}{
\index{angular momentum}

Given a wave function
%
\begin{dmath}\label{eqn:LsquaredLzProblem:20}
\psi(r,\theta, \phi) = f(r) \lr{ x + y + 3 z },
\end{dmath}

\makesubproblem{}{problem:LsquaredLzProblem:1:a}
Determine if this wave function is an eigenfunction of \( \BL^2 \), and the value of \( l \) if it is an eigenfunction.
\makesubproblem{}{problem:LsquaredLzProblem:1:b}
Determine the probabilities for the particle to be found in any given \( \ket{l, m} \) state,
\makesubproblem{}{problem:LsquaredLzProblem:1:c}
If it is known that \( \psi \) is an energy eigenfunction with energy \( E \) indicate how we can find \( V(r) \).

} % problem

\makeanswer{problem:LsquaredLzProblem:1}{

\makeSubAnswer{}{problem:LsquaredLzProblem:1:a}

Using
\begin{equation}\label{eqn:LsquaredLzProblem:40}
\BL^2
=
-\Hbar^2 \lr{ \inv{\sin^2\theta} \partial_{\phi\phi} + \inv{\sin\theta} \partial_\theta \lr{ \sin\theta \partial_\theta} },
\end{equation}

and
%
\begin{equation}\label{eqn:LsquaredLzProblem:60}
\begin{aligned}
x &= r \sin\theta \cos\phi \\
y &= r \sin\theta \sin\phi \\
z &= r \cos\theta
\end{aligned}
\end{equation}

it's a quick computation to show that
%
\begin{equation}\label{eqn:LsquaredLzProblem:80}
\BL^2 \psi = 2 \Hbar^2 \psi = 1(1 + 1) \Hbar^2 \psi,
\end{equation}

so this function is an eigenket of \( \BL^2 \) with an eigenvalue of \( 2 \Hbar^2 \), which corresponds to \( l = 1 \), a p-orbital state.

\makeSubAnswer{}{problem:LsquaredLzProblem:1:b}

Recall that the angular representation of \( L_z \) is
%
\begin{equation}\label{eqn:LsquaredLzProblem:100}
L_z = -i \Hbar \PD{\phi},
\end{equation}

so we have
%
\begin{equation}\label{eqn:LsquaredLzProblem:120}
\begin{aligned}
L_z x &= i \Hbar y \\
L_z y &= - i \Hbar x \\
L_z z &= 0,
\end{aligned}
\end{equation}

The \( L_z \) action on \( \psi \) is
%
\begin{equation}\label{eqn:LsquaredLzProblem:140}
L_z \psi = -i \Hbar r f(r) \lr{ - y + x }.
\end{equation}

This wave function is not an eigenket of \( L_z \).  Expressed in terms of the \( L_z \) basis states \( e^{i m \phi} \), this wave function is
%
\begin{dmath}\label{eqn:LsquaredLzProblem:160}
\psi
= r f(r) \lr{ \sin\theta \lr{ \cos\phi + \sin\phi} + \cos\theta }
= r f(r) \lr{ \frac{\sin\theta}{2} \lr{ e^{i \phi} \lr{ 1 + \inv{i}} + e^{-i\phi} \lr{ 1 - \inv{i} } } + \cos\theta }
= r f(r) \lr{
\frac{(1-i)\sin\theta}{2} e^{1 i \phi}
+
\frac{(1+i)\sin\theta}{2} e^{- 1 i \phi}
+ \cos\theta e^{0 i \phi}
}
\end{dmath}

Assuming that \( \psi \) is normalized, the probabilities for measuring \( m = 1,-1,0 \) respectively are
%
\begin{dmath}\label{eqn:LsquaredLzProblem:180}
P_{\pm 1}
= 2 \pi \rho \Abs{\frac{1\mp i}{2}}^2 \int_0^\pi \sin\theta d\theta \sin^2 \theta
= -2 \pi \rho \int_1^{-1} du (1-u^2)
= 2 \pi \rho \evalrange{ \lr{ u - \frac{u^3}{3} } }{-1}{1}
= 2 \pi \rho \lr{ 2 - \frac{2}{3}}
= \frac{ 8 \pi \rho}{3},
\end{dmath}

and
%
\begin{dmath}\label{eqn:LsquaredLzProblem:200}
P_{0} = 2 \pi \rho \int_0^\pi \sin\theta \cos\theta = 0,
\end{dmath}

where
%
\begin{dmath}\label{eqn:LsquaredLzProblem:220}
\rho = \int_0^\infty r^4 \Abs{f(r)}^2 dr.
\end{dmath}

Because the probabilities must sum to 1, this means the \( m = \pm 1 \) states are equiprobable with \( P_{\pm} = 1/2 \), fixing \( \rho = 3/16\pi \), even without knowing \( f(r) \).

\makeSubAnswer{}{problem:LsquaredLzProblem:1:c}

The operator \( r^2 \Bp^2 \) can be decomposed into a \( \BL^2 \) component and some other portions, from which we can write
%
\begin{dmath}\label{eqn:LsquaredLzProblem:240}
H \psi
= \lr{ \frac{\Bp^2}{2m} + V(r)  } \psi
=
\lr{
- \frac{\Hbar^2}{2m} \lr{ \partial_{rr} + \frac{2}{r} \partial_r - \inv{\Hbar^2 r^2} \BL^2 } + V(r) } \psi.
\end{dmath}

(See: \citep{sakurai2014modern} eq. 6.21)

In this case where \( \BL^2 \psi = 2 \Hbar^2 \psi \) we can rearrange for \( V(r) \)
%
\begin{dmath}\label{eqn:LsquaredLzProblem:260}
V(r)
= E + \inv{\psi} \frac{\Hbar^2}{2m} \lr{ \partial_{rr} + \frac{2}{r} \partial_r - \frac{2}{r^2} } \psi
= E + \inv{f(r)} \frac{\Hbar^2}{2m} \lr{ \partial_{rr} + \frac{2}{r} \partial_r - \frac{2}{r^2} } f(r).
\end{dmath}

See \nbref{sakuraiProblem3.17.nb} for some verifications of some of the algebra for this problem.
} % answer

%\EndArticle
