%
% Copyright � 2015 Peeter Joot.  All Rights Reserved.
% Licenced as described in the file LICENSE under the root directory of this GIT repository.
%
\makeoproblem{Reduced density matrix.}{gradQuantum:problemSet1:2}{phy1520 2015 ps1.2}{
\index{reduced density!matrix}

\index{spin half}
Consider two spin-1/2 particles, the Hilbert space is now 4-dimensional,
with states
\( \ket{ \uparrow \uparrow } \),
\( \ket{ \uparrow \downarrow } \),
\( \ket{ \downarrow \uparrow } \),
\( \ket{ \downarrow \downarrow } \).
Let us consider the following pure states:
\begin{enumerate}[(i)]
\item \( \inv{2} \lr{
 \ket{ \uparrow \uparrow }
- \ket{ \uparrow \downarrow }
- \ket{ \downarrow \downarrow }
+ \ket{ \downarrow \uparrow }
} \)
\item \( \inv{\sqrt{2}} \lr{
 \ket{ \uparrow \uparrow }
+ \ket{ \downarrow \downarrow }
} \)
\item \( \inv{\sqrt{5}} \lr{
 \ket{ \uparrow \uparrow }
+ 2  \ket{ \downarrow \downarrow }
} \)
\end{enumerate}
In each case, obtain the reduced \( 2 \times 2 \) density matrix which describes the first spin, when we trace over the
second spin. The von Neumann entanglement entropy is defined via
\( S_{\txtv \txtN} = -\tr(\rho_\txtR \ln \rho_\txtR ) \) where \( \rho_\txtR \) is the reduced
density matrix you have obtained above and the \( \tr \) now refers to tracing over the first spin. Using the reduced
density matrices you have obtained above, compute the corresponding
\( S_{\txtv \txtN} \),
and simply explain your result in words.
Consider the Renyi entropy
\( S_n =
\inv{1-n}
\ln \lr{ \tr( \rho_\txtR^n ) } \).
Prove that
\( S_{n \rightarrow 1} = S_{\txtv \txtN} \),
and compute \( S_{n=2} \) for the above \( \rho_\txtR \).
} % makeproblem
%
\makeanswer{gradQuantum:problemSet1:2}{
\withproblemsetsParagraph{

\paragraph{reduced density operator matrix representation}
\index{reduced density!operator}

Some notation is useful to start.  Let \( \ket{a}_1 \) represent a spin state for the first particle and \( \ket{a}_2 \) represent a spin state for the second particle.  If the state under consideration is \( \ket{\psi} \), the reduced density operator matrix representation, after summing over all the second particle states, is
%
\begin{dmath}\label{eqn:gradQuantumProblemSet1Problem2:20}
\rho_\txtR
= \sum_a \prescript{}{2}{\bra{a}} \lr{ \ket{\psi} \bra{\psi} } \ket{a}_2.
\end{dmath}
%
Computing these \( {}_2\braket{a}{\psi} \) brackets for each state will allow the each of reduced density matrix to be expressed easily.

\begin{enumerate}[(i)]
\item For this state we have
%
\begin{dmath}\label{eqn:gradQuantumProblemSet1Problem2:40}
\prescript{}{2}{\braket{\uparrow}{\psi}} =
\inv{2}
\prescript{}{2}{\bra{\uparrow}}
\lr{
 \ket{ \uparrow \uparrow }
- \ket{ \uparrow \downarrow }
- \ket{ \downarrow \downarrow }
+ \ket{ \downarrow \uparrow }
}
=
\inv{2}
\lr{
 \ket{ \uparrow }_1
+ \ket{ \downarrow }_1
}
=
\inv{2}
\begin{bmatrix}
1 \\
1
\end{bmatrix},
\end{dmath}
%
and
%
\begin{dmath}\label{eqn:gradQuantumProblemSet1Problem2:60}
\prescript{}{2}{\braket{\downarrow}{\psi}} =
\inv{2}
\prescript{}{2}{\bra{\downarrow}}
\lr{
 \ket{ \uparrow \uparrow }
- \ket{ \uparrow \downarrow }
- \ket{ \downarrow \downarrow }
+ \ket{ \downarrow \uparrow }
}
=
\inv{2}
\lr{
- \ket{ \uparrow }_1
- \ket{ \downarrow }_1
}
=
\inv{2}
\begin{bmatrix}
-1 \\
-1
\end{bmatrix},
\end{dmath}
%
so the reduced density operator matrix representation is
%
\begin{dmath}\label{eqn:gradQuantumProblemSet1Problem2:80}
\rho_\txtR
=
\inv{4}
\begin{bmatrix}
1 \\
1
\end{bmatrix}
\begin{bmatrix}
1 & 1
\end{bmatrix}
+
\inv{4}
\begin{bmatrix}
-1 \\
-1
\end{bmatrix}
\begin{bmatrix}
-1 & -1
\end{bmatrix}
=
\inv{2}
\begin{bmatrix}
1 & 1 \\
1 & 1
\end{bmatrix}.
\end{dmath}
%
\item For this state we have
%
\begin{equation}\label{eqn:gradQuantumProblemSet1Problem2:100}
\prescript{}{2}{\braket{\uparrow}{\psi}} =
\inv{\sqrt{2}}
\prescript{}{2}{\bra{\uparrow}}
\lr{
 \ket{ \uparrow \uparrow }
+ \ket{ \downarrow \downarrow }
}
=
\inv{\sqrt{2}}
 \ket{ \uparrow }_1
=
\inv{\sqrt{2}}
\begin{bmatrix}
1 \\
0
\end{bmatrix},
\end{equation}
%
and
%
\begin{equation}\label{eqn:gradQuantumProblemSet1Problem2:120}
\prescript{}{2}{\braket{\downarrow}{\psi}} =
\inv{\sqrt{2}}
\prescript{}{2}{\bra{\downarrow}}
\lr{
 \ket{ \uparrow \uparrow }
+ \ket{ \downarrow \downarrow }
}
=
\inv{\sqrt{2}}
 \ket{ \downarrow }_1
=
\inv{\sqrt{2}}
\begin{bmatrix}
0 \\
1
\end{bmatrix},
\end{equation}
%
so the reduced density operator matrix representation is
%
\begin{dmath}\label{eqn:gradQuantumProblemSet1Problem2:140}
\rho_\txtR
=
\inv{2}
\begin{bmatrix}
1 \\
0
\end{bmatrix}
\begin{bmatrix}
1 & 0
\end{bmatrix}
+
\inv{2}
\begin{bmatrix}
0 \\
1
\end{bmatrix}
\begin{bmatrix}
0 & 1
\end{bmatrix}
=
\inv{2}
\begin{bmatrix}
1 & 0 \\
0 & 1
\end{bmatrix}.
\end{dmath}
%
\item For this state we have
%
\begin{equation}\label{eqn:gradQuantumProblemSet1Problem2:160}
\prescript{}{2}{\braket{\uparrow}{\psi}} =
\inv{\sqrt{5}}
\prescript{}{2}{\bra{\uparrow}}
\lr{
 \ket{ \uparrow \uparrow }
+ 2 \ket{ \downarrow \downarrow }
}
=
\inv{\sqrt{5}}
 \ket{ \uparrow }_1
=
\inv{\sqrt{5}}
\begin{bmatrix}
1 \\
0
\end{bmatrix},
\end{equation}
%
and
%
\begin{equation}\label{eqn:gradQuantumProblemSet1Problem2:180}
\prescript{}{2}{\braket{\downarrow}{\psi}} =
\inv{\sqrt{5}}
\prescript{}{2}{\bra{\downarrow}}
\lr{
 \ket{ \uparrow \uparrow }
+ 2 \ket{ \downarrow \downarrow }
}
=
\frac{2}{\sqrt{5}}
 \ket{ \downarrow }_1
=
\frac{2}{\sqrt{5}}
\begin{bmatrix}
0 \\
1
\end{bmatrix},
\end{equation}
%
so the reduced density operator matrix representation is
%
\begin{dmath}\label{eqn:gradQuantumProblemSet1Problem2:200}
\rho_\txtR
=
\inv{5}
\begin{bmatrix}
1 \\
0
\end{bmatrix}
\begin{bmatrix}
1 & 0
\end{bmatrix}
+
\frac{4}{5}
\begin{bmatrix}
0 \\
1
\end{bmatrix}
\begin{bmatrix}
0 & 1
\end{bmatrix}
=
\inv{5}
\begin{bmatrix}
1 & 0 \\
0 & 4
\end{bmatrix}.
\end{dmath}
%
\end{enumerate}

\paragraph{von Neumann entropy.}
\index{von Neumann entropy}

Here is the computation of the von Neumann entropy for each of these states.
\begin{enumerate}[(i)]
\item For this case, the eigenvalues of the reduced density matrix are \( \setlr{0,1} \), so the von Neumann entropy is
%
\begin{dmath}\label{eqn:gradQuantumProblemSet1Problem2:220}
S_{\txtv \txtN} = - 1 \ln 1 - 0 \ln 0 = 0,
\end{dmath}
%
where the zero logarithm is interpreted in the limit \( \lim_{x \rightarrow 0} -x \ln x = 0 \).

\item Since this reduced density matrix is diagonal, the von Neumann entropy can be calculated easily
%
\begin{dmath}\label{eqn:gradQuantumProblemSet1Problem2:240}
S_{\txtv \txtN} = 2 \lr{ - \inv{2} \ln \inv{2} } = \ln 2 \approx 0.69
\end{dmath}

\item This is also diagonal with von Neumann entropy of
%
\begin{dmath}\label{eqn:gradQuantumProblemSet1Problem2:260}
S_{\txtv \txtN}
= - \inv{5} \ln \inv{5} - \frac{4}{5} \ln \frac{4}{5}
= \inv{5} \lr{ \ln 5 + 4 \ln (5/4) }
\approx 0.50
\end{dmath}

In the diagonal representation the entropy of the reduced density operator for case (i) shows that it is the most ordered system, with a unit probability amplitude for one state of that representation, and zero amplitude for the other.  Case (iii) of the reduced system is the next most ordered, with the more of the probability amplitude weighting associated with the first particle in the spin down state.  This results in a greater than zero entropy.  Case (ii) is the least ordered of the reduced systems with equal probability amplitudes for each first particle spin state, and has the maximum von Neumann entropy possible for a two dimensional Hilbert space.

\paragraph{Renyi entropy.}
\index{Renyi entropy}

Here is the computation of the Renyi entropy for \( n = 2 \) for each of these states.  That is
%
\begin{equation}\label{eqn:gradQuantumProblemSet1Problem2:280}
S_{n=2} = -\ln \tr \lr{ \rho_\txtR^2 }.
\end{equation}
\end{enumerate}

\begin{enumerate}[(i)]
\item
For this case the reduced density matrix is idempotent
\index{idempotent}
%
\begin{equation}\label{eqn:gradQuantumProblemSet1Problem2:300}
\rho_\txtR^2 = \inv{4}
\begin{bmatrix}
1 & 1  \\
1 & 1
\end{bmatrix}
\begin{bmatrix}
1 & 1  \\
1 & 1
\end{bmatrix}
=
\inv{2}
\begin{bmatrix}
1 & 1  \\
1 & 1
\end{bmatrix}.
\end{equation}
%
So the trace is the sum of the eigenvalues of \( \rho_\txtR \), and the \( n = 2 \) Renyi entropy is
%
\begin{equation}\label{eqn:gradQuantumProblemSet1Problem2:320}
S_{n=2} = -\ln (1 + 0) = 0.
\end{equation}
%
\item
For this case, the \( n = 2 \) Renyi entropy is
%
\begin{dmath}\label{eqn:gradQuantumProblemSet1Problem2:340}
S_{n=2} = -\ln \lr{ \inv{2^2} + \inv{2^2}} = \ln 2 \approx 0.69.
\end{dmath}
%
\item

Finally, for this case, the \( n = 2 \) Renyi entropy is
%
\begin{dmath}\label{eqn:gradQuantumProblemSet1Problem2:360}
S_{n=2} = -\ln \lr{ \frac{1}{25} + \frac{16}{25} } = \ln \frac{17}{25} \approx 0.39.
\end{dmath}
%
\end{enumerate}

\paragraph{Limit of Renyi entropy}

In a diagonal representation of the reduced density matrix, the \( n \rightarrow 1 \) summation in the numerator of the Renyi entropy
%
\begin{dmath}\label{eqn:gradQuantumProblemSet1Problem2:380}
S_{n \rightarrow 1}
= \lim_{n \rightarrow 1} \frac{\ln \sum_{k} \rho_{kk}^n }{1 - n},
\end{dmath}
%
tends to unity, so that logarithm tends to zero.  This results in division of two nearly zero quantities, which can be evaluated using l'H\^{o}pital's rule.  That is
%
\begin{dmath}\label{eqn:gradQuantumProblemSet1Problem2:400}
S_{n \rightarrow 1}
= \frac{\evalbar{\frac{d}{dn} \ln \sum_{k} \rho_{kk}^n}{n=1} }{-1}
=
\frac{
\sum_k \evalbar{\frac{d}{dn} \rho_{kk}^n}{n=1}
}{-
\evalbar{\sum_{k} \rho_{kk}^n}{n=1}
}
=
- \sum_k \evalbar{\frac{d}{dn} \rho_{kk}^n}{n=1}.
\end{dmath}
%
To evaluate the derivatives set \( a^n = e^y \), or \( y = n \ln a \).  That gives
%
\begin{dmath}\label{eqn:gradQuantumProblemSet1Problem2:420}
\frac{d}{dn} a^n
=
\frac{d}{dn} e^{ n \ln a }
=
a^n \ln a,
\end{dmath}
%
which gives
%
\begin{dmath}\label{eqn:gradQuantumProblemSet1Problem2:440}
S_{n \rightarrow 1}
=
-
\evalbar{
\sum_k \rho_{kk}^n \ln \rho_{kk}
}{n=1}
=
\sum_k \rho_{kk} \ln \rho_{kk},
\end{dmath}
%
which is precisely the von Neumann entropy.
}
}
