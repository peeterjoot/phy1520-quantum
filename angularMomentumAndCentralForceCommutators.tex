%
% Copyright � 2015 Peeter Joot.  All Rights Reserved.
% Licenced as described in the file LICENSE under the root directory of this GIT repository.
%
%\input{../blogpost.tex}
%\renewcommand{\basename}{angularMomentumAndCentralForceCommutators}
%\renewcommand{\dirname}{notes/phy1520/}
%%\newcommand{\dateintitle}{}
%%\newcommand{\keywords}{}
%
%\input{../peeter_prologue_print2.tex}
%
%\usepackage{peeters_layout_exercise}
%\usepackage{peeters_braket}
%\usepackage{peeters_figures}
%
%\beginArtNoToc
%
%\generatetitle{Commutators of angular momentum and a central force Hamiltonian}
%%\chapter{Commutators of angular momentum and a central force Hamiltonian}
%%\label{chap:angularMomentumAndCentralForceCommutators}
%
\makeoproblem{Non-commuting observables, degeneracy.}{problem:angularMomentumAndCentralForceCommutators:1}{\citep{sakurai2014modern} pr. 1.17}{
\index{degeneracy}
\index{non-commuting observables}
Show that non-commuting operators that both commute with the Hamiltonian, have, in general, degenerate energy eigenvalues.  That is
%
\begin{equation}\label{eqn:angularMomentumAndCentralForceCommutators:320}
[A,H] = [B,H] = 0,
\end{equation}
%
but
%
\begin{dmath}\label{eqn:angularMomentumAndCentralForceCommutators:340}
[A,B] \ne 0.
\end{dmath}
%
\makesubproblem{}{problem:angularMomentumAndCentralForceCommutators:1:a}
%
Consider \( L_x, L_z \) and a central force Hamiltonian \( H = \Bp^2/2m + V(r) \) as examples.
%
\makesubproblem{}{problem:angularMomentumAndCentralForceCommutators:1:b}
%
Construct some simple matrix examples that illustrate the degeneracy conditions.
%
\makesubproblem{}{problem:angularMomentumAndCentralForceCommutators:1:c}
%
Prove the general case.
} % problem
%
\makeanswer{problem:angularMomentumAndCentralForceCommutators:1}{
%
\makeSubAnswer{}{problem:angularMomentumAndCentralForceCommutators:1:a}
%
Let's start with demonstrate these commutators act as expected in these cases.
With \( \BL = \Bx \cross \Bp \), we have
%
\begin{equation}\label{eqn:angularMomentumAndCentralForceCommutators:20}
\begin{aligned}
L_x &= y p_z - z p_y \\
L_y &= z p_x - x p_z \\
L_z &= x p_y - y p_x.
\end{aligned}
\end{equation}
%
The \( L_x, L_z \) commutator is
%
\begin{equation}\label{eqn:angularMomentumAndCentralForceCommutators:40}
\begin{aligned}
\antisymmetric{L_x}{L_z}
&=
\antisymmetric{y p_z - z p_y }{x p_y - y p_x} \\
&=
\antisymmetric{y p_z}{x p_y}
-\antisymmetric{y p_z}{y p_x}
-\antisymmetric{z p_y }{x p_y}
+\antisymmetric{z p_y }{y p_x} \\
&=
x p_z \antisymmetric{y}{p_y}
+ z p_x \antisymmetric{p_y }{y} \\
&=
i \Hbar \lr{ x p_z - z p_x } \\
&=
- i \Hbar L_y
\end{aligned}.
\end{equation}
Cyclically permuting the indexes shows that no pairs of different \( \BL \) components commute.  For \( L_y, L_x \) that is
%
\begin{equation}\label{eqn:angularMomentumAndCentralForceCommutators:60}
\begin{aligned}
\antisymmetric{L_y}{L_x}
&=
\antisymmetric{z p_x - x p_z }{y p_z - z p_y} \\
&=
\antisymmetric{z p_x}{y p_z}
-\antisymmetric{z p_x}{z p_y}
-\antisymmetric{x p_z }{y p_z}
+\antisymmetric{x p_z }{z p_y} \\
&=
y p_x \antisymmetric{z}{p_z}
+ x p_y \antisymmetric{p_z }{z} \\
&=
i \Hbar \lr{ y p_x - x p_y } \\
&=
- i \Hbar L_z,
\end{aligned}
\end{equation}

and for \( L_z, L_y \)
%
\begin{equation}\label{eqn:angularMomentumAndCentralForceCommutators:80}
\begin{aligned}
\antisymmetric{L_z}{L_y}
&=
\antisymmetric{x p_y - y p_x }{z p_x - x p_z} \\
&=
\antisymmetric{x p_y}{z p_x}
-\antisymmetric{x p_y}{x p_z}
-\antisymmetric{y p_x }{z p_x}
+\antisymmetric{y p_x }{x p_z} \\
&=
z p_y \antisymmetric{x}{p_x}
+ y p_z \antisymmetric{p_x }{x} \\
&=
i \Hbar \lr{ z p_y - y p_z } \\
&=
- i \Hbar L_x.
\end{aligned}
\end{equation}
%
If these angular momentum components are also shown to commute with themselves (which they do), the commutator relations above can be summarized as

\index{central force potential}
\begin{equation}\label{eqn:angularMomentumAndCentralForceCommutators:100}
\antisymmetric{L_a}{L_b} = i \Hbar \epsilon_{a b c} L_c.
\end{equation}
%
In the example to consider, we'll have to consider the commutators with \( \Bp^2 \) and \( V(r) \).  Picking any one component of \( \BL \) is sufficient due to the symmetries of the problem.  For example
%
\begin{equation}\label{eqn:angularMomentumAndCentralForceCommutators:120}
\begin{aligned}
\antisymmetric{L_x}{\Bp^2}
&=
\antisymmetric{y p_z - z p_y}{p_x^2 + p_y^2 + p_z^2} \\
&=
\antisymmetric{y p_z}{\cancel{p_x^2} + p_y^2 + \cancel{p_z^2}}
-\antisymmetric{z p_y}{\cancel{p_x^2} + \cancel{p_y^2} + p_z^2} \\
&=
p_z \antisymmetric{y}{p_y^2}
-p_y \antisymmetric{z}{p_z^2} \\
&=
p_z 2 i \Hbar p_y
-p_y 2 i \Hbar p_z  \\
&=
0.
\end{aligned}
\end{equation}
%
How about the commutator of \( \BL \) with the potential?  It is sufficient to consider one component again, for example
%
\begin{equation}\label{eqn:angularMomentumAndCentralForceCommutators:140}
\begin{aligned}
\antisymmetric{L_x}{V}
&=
\antisymmetric{y p_z - z p_y}{V} \\
&=
y \antisymmetric{p_z}{V} - z \antisymmetric{p_y}{V} \\
&=
-i \Hbar y \PD{z}{V(r)} + i \Hbar z \PD{y}{V(r)} \\
&=
-i \Hbar y \PD{r}{V}\PD{z}{r} + i \Hbar z \PD{r}{V}\PD{y}{r}  \\
&=
-i \Hbar y \PD{r}{V} \frac{z}{r} + i \Hbar z \PD{r}{V}\frac{y}{r}  \\
&=
0.
\end{aligned}
\end{equation}
%
This has shown that all the components of \( \BL \) commute with a central force Hamiltonian, and each different component of \( \BL \) do not commute.  It does not demonstrate the degeneracy, but I do recall that exists for this system.

%\paragraph{Matrix example of non-commuting commutators}
\makeSubAnswer{}{problem:angularMomentumAndCentralForceCommutators:1:b}
%
I thought perhaps the problem at hand would be easier if I were to construct some example matrices representing operators that did not commute, but did commuted with a Hamiltonian.  I came up with
%
\begin{equation}\label{eqn:angularMomentumAndCentralForceCommutators:360}
\begin{aligned}
A &=
\begin{bmatrix}
\sigma_z & 0 \\
0 & 1
\end{bmatrix}
=
\begin{bmatrix}
 1 & 0 & 0 \\
 0 & -1 & 0 \\
 0 & 0 & 1 \\
\end{bmatrix} \\
B &=
\begin{bmatrix}
\sigma_x & 0 \\
0 & 1
\end{bmatrix}
=
\begin{bmatrix}
 0 & 1 & 0 \\
 1 & 0 & 0 \\
 0 & 0 & 1 \\
\end{bmatrix} \\
H &=
\begin{bmatrix}
 0 & 0 & 0 \\
 0 & 0 & 0 \\
 0 & 0 & 1 \\
\end{bmatrix}
\end{aligned}.
\end{equation}
This system has \( \antisymmetric{A}{H} = \antisymmetric{B}{H} = 0 \), and
%
\begin{dmath}\label{eqn:angularMomentumAndCentralForceCommutators:380}
\antisymmetric{A}{B}
=
\begin{bmatrix}
 0 & 2 & 0 \\
-2 & 0 & 0 \\
 0 & 0 & 0 \\
\end{bmatrix}.
\end{dmath}
There is one shared eigenvector between all of \( A, B, H \)
%
\begin{dmath}\label{eqn:angularMomentumAndCentralForceCommutators:400}
\ket{3} =
\begin{bmatrix}
0 \\
0 \\
1
\end{bmatrix}.
\end{dmath}
%
The other eigenvectors for \( A \) are
\begin{equation}\label{eqn:angularMomentumAndCentralForceCommutators:420}
\ket{a_1} =
\begin{bmatrix}
1 \\
0 \\
0
\end{bmatrix},\qquad
\ket{a_2} =
\begin{bmatrix}
0 \\
1 \\
0
\end{bmatrix},
\end{equation}
and for \( B \)
\begin{equation}\label{eqn:angularMomentumAndCentralForceCommutators:440}
\ket{b_1} =
\inv{\sqrt{2}}
\begin{bmatrix}
1 \\
1 \\
0
\end{bmatrix},\qquad
\ket{b_2} =
\inv{\sqrt{2}}
\begin{bmatrix}
1 \\
-1 \\
0
\end{bmatrix}.
\end{equation}
This clearly has the degeneracy sought.

Looking to \citep{commutingMatrices}, it appears that it is possible to construct an even simpler example.  Let
%
\begin{equation}\label{eqn:angularMomentumAndCentralForceCommutators:460}
\begin{aligned}
A &=
\begin{bmatrix}
0 & 1 \\
0 & 0
\end{bmatrix} \\
B &=
\begin{bmatrix}
1 & 0 \\
0 & 0
\end{bmatrix} \\
H &=
\begin{bmatrix}
0 & 0 \\
0 & 0
\end{bmatrix}.
\end{aligned}
\end{equation}
%
Here \( \antisymmetric{A}{B} = -A \), and \( \antisymmetric{A}{H} = \antisymmetric{B}{H} = 0 \), but the Hamiltonian isn't interesting at all physically.

A less boring example builds on this.  Let
%
\begin{equation}\label{eqn:angularMomentumAndCentralForceCommutators:480}
\begin{aligned}
A &=
\begin{bmatrix}
0 & 1 & 0 \\
0 & 0 & 0 \\
0 & 0 & 1
\end{bmatrix} \\
B &=
\begin{bmatrix}
1 & 0 & 0 \\
0 & 0 & 0 \\
0 & 0 & 1
\end{bmatrix} \\
H &=
\begin{bmatrix}
0 & 0 & 0 \\
0 & 0 & 0 \\
0 & 0 & 1 \\
\end{bmatrix}.
\end{aligned}
\end{equation}
%
Here \( \antisymmetric{A}{B} \ne 0 \), and \( \antisymmetric{A}{H} = \antisymmetric{B}{H} = 0 \).  I don't see a way for any exception to be constructed.

%\paragraph{The problem}
\makeSubAnswer{}{problem:angularMomentumAndCentralForceCommutators:1:c}
%
The concrete examples above give some intuition for solving the more abstract problem.  Suppose that we are working in a basis that simultaneously diagonalizes operator \( A \) and the Hamiltonian \( H \).  To make life easy consider the simplest case where this basis is also an eigenbasis for the second operator \( B \) for all but two of that operators eigenvectors.  For such a system let's write
%
\begin{equation}\label{eqn:angularMomentumAndCentralForceCommutators:160}
\begin{aligned}
H \ket{1} &= \epsilon_1 \ket{1} \\
H \ket{2} &= \epsilon_2 \ket{2} \\
A \ket{1} &= a_1 \ket{1} \\
A \ket{2} &= a_2 \ket{2},
\end{aligned}
\end{equation}
where \( \ket{1}\), and \( \ket{2} \) are not eigenkets of \( B \).  Because \( B \) also commutes with \( H \), we must have
%
\begin{dmath}\label{eqn:angularMomentumAndCentralForceCommutators:180}
H B \ket{1}
= H \sum_n \ket{n}\bra{n} B \ket{1}
= \sum_n \epsilon_n \ket{n} B_{n 1},
\end{dmath}
%
and
\begin{dmath}\label{eqn:angularMomentumAndCentralForceCommutators:200}
B H \ket{1}
= B \epsilon_1 \ket{1}
= \epsilon_1 \sum_n \ket{n}\bra{n} B \ket{1}
= \epsilon_1 \sum_n \ket{n} B_{n 1}.
\end{dmath}
%
We can now compute the action of the commutators on \( \ket{1}, \ket{2} \),
\begin{dmath}\label{eqn:angularMomentumAndCentralForceCommutators:220}
\antisymmetric{B}{H} \ket{1}
=
\sum_n \lr{ \epsilon_1 - \epsilon_n } \ket{n} B_{n 1}.
\end{dmath}
%
Similarly
\begin{dmath}\label{eqn:angularMomentumAndCentralForceCommutators:240}
\antisymmetric{B}{H} \ket{2}
=
\sum_n \lr{ \epsilon_2 - \epsilon_n } \ket{n} B_{n 2}.
\end{dmath}
%
However, for those kets \( \ket{m} \in \setlr{ \ket{3}, \ket{4}, \cdots } \) that are eigenkets of \( B \), with \( B \ket{m} = b_m \ket{m} \), we have
%
\begin{dmath}\label{eqn:angularMomentumAndCentralForceCommutators:280}
\antisymmetric{B}{H} \ket{m}
=
B \epsilon_m \ket{m} - H b_m \ket{m}
=
b_m \epsilon_m \ket{m} - \epsilon_m b_m \ket{m}
=
0,
\end{dmath}
%
The sums in
\cref{eqn:angularMomentumAndCentralForceCommutators:220}
and
\cref{eqn:angularMomentumAndCentralForceCommutators:240} reduce to
\begin{dmath}\label{eqn:angularMomentumAndCentralForceCommutators:500}
\antisymmetric{B}{H} \ket{1}
=
\sum_{n=1}^2 \lr{ \epsilon_1 - \epsilon_n } \ket{n} B_{n 1}
=
\lr{ \epsilon_1 - \epsilon_2 } \ket{2} B_{2 1},
\end{dmath}
and
\begin{dmath}\label{eqn:angularMomentumAndCentralForceCommutators:520}
\antisymmetric{B}{H} \ket{2}
=
\sum_{n=1}^2 \lr{ \epsilon_2 - \epsilon_n } \ket{n} B_{n 2}
=
\lr{ \epsilon_2 - \epsilon_1 } \ket{1} B_{1 2}.
\end{dmath}
Since the commutator is zero, the matrix elements of the commutator must all be zero, in particular
\begin{equation}\label{eqn:angularMomentumAndCentralForceCommutators:260}
\begin{aligned}
   \bra{1} \antisymmetric{B}{H} \ket{1} &= \lr{ \epsilon_1 - \epsilon_2 } B_{2 1} \braket{1}{2} = 0 \\
   \bra{2} \antisymmetric{B}{H} \ket{1} &= \lr{ \epsilon_1 - \epsilon_2 } B_{2 1} \braket{1}{1} \\
   \bra{1} \antisymmetric{B}{H} \ket{2} &= \lr{ \epsilon_2 - \epsilon_1 } B_{1 2} \braket{1}{2} = 0 \\
   \bra{2} \antisymmetric{B}{H} \ket{2} &= \lr{ \epsilon_2 - \epsilon_1 } B_{1 2} \braket{2}{2}.
\end{aligned}
\end{equation}
We must either have
\begin{itemize}
\item \( B_{2 1} = B_{1 2} = 0 \), or
\item \( \epsilon_1 = \epsilon_2 \).
\end{itemize}
If the first condition were true we would have
%
\begin{dmath}\label{eqn:angularMomentumAndCentralForceCommutators:300}
B \ket{1}
=
\ket{n}\bra{n} B \ket{1}
=
\ket{n} B_{n 1}
=
\ket{1} B_{1 1},
\end{dmath}
%
and \( B \ket{2} = B_{2 2} \ket{2} \).  This contradicts the requirement that \( \ket{1}, \ket{2} \) not be eigenkets of \( B \), leaving only the second option.  That second option means there must be a degeneracy in the system.
} % answer
%\EndArticle
