%
% Copyright � 2015 Peeter Joot.  All Rights Reserved.
% Licenced as described in the file LICENSE under the root directory of this GIT repository.
%
%\input{../blogpost.tex}
%\renewcommand{\basename}{qmLecture12}
%\renewcommand{\dirname}{notes/phy1520/}
%\newcommand{\keywords}{PHY1520H}
%\input{../peeter_prologue_print2.tex}
%
%%\usepackage{phy1520}
%\usepackage{peeters_braket}
%\usepackage{peeters_layout_exercise}
%\usepackage{peeters_figures}
%\usepackage{mathtools}
%
%\beginArtNoToc
%\generatetitle{PHY1520H Graduate Quantum Mechanics.  Lecture 12: Symmetry (cont.).  Taught by Prof.\ Arun Paramekanti}
%%\chapter{Symmetry (cont.)}
%\label{chap:qmLecture12}
%
%\paragraph{Disclaimer}
%
%Peeter's lecture notes from class.  These may be incoherent and rough.
%
%These are notes for the UofT course PHY1520, Graduate Quantum Mechanics, taught by Prof. Paramekanti, covering \textchapref{{1}} \citep{sakurai2014modern} content.
%
\paragraph{Parity (review)}

\begin{dmath}\label{eqn:qmLecture12:20}
\hat\Pi \hatx \hat\Pi = - \hatx
\end{dmath}
\begin{dmath}\label{eqn:qmLecture12:40}
\hat\Pi \hatp \hat\Pi = - \hatp
\end{dmath}

These are \underline{polar} vectors, in contrast to an \underline{axial} vector such as \( \BL = \Br \cross \Bp \).

\begin{dmath}\label{eqn:qmLecture12:60}
\hat\Pi^2 = 1
\end{dmath}

\begin{dmath}\label{eqn:qmLecture12:80}
\Psi(x) \rightarrow \Psi(-x)
\end{dmath}

If \( \antisymmetric{\hat\Pi}{\hatH} = 0 \) then all the eigenstates are either

\begin{itemize}
\item even: \( \hat\Pi \) eigenvalue is \( + 1 \).
\item odd: \( \hat\Pi \) eigenvalue is \( - 1 \).
\end{itemize}

\paragraph{Note on parity in multiple dimensions}

A Hamiltonian can be constructed with parity symmetries in one or more directions.  For example, given a potential

\begin{dmath}\label{eqn:qmLecture12:81}
V(x,y) = a x + b x^2 + c y^2,
\end{dmath}

We don't have parity symmetry for \( \Bx = (x,y) \), but do have parity symmetry in the \( y \) direction.  Assuming a separated variables form for the wave function, say \( \psi(x,y) = X(x)Y(y) \), we can't say much about \( X \) on the grounds of symmetry considerations only, but know that \( Y \) has to be either an even or odd function.

%We are done with discrete symmetry operators for now.

\section{Translations}
\index{translation!symmetry}
\index{translation!operator}

Define a (continuous) translation operator

\begin{dmath}\label{eqn:qmLecture12:100}
\hatT_\epsilon \ket{x} = \ket{x + \epsilon}
\end{dmath}

The action of this operator is sketched in \cref{fig:lecture12:lecture12Fig1}.

\imageFigure{../../figures/phy1520/lecture12Fig1}{Translation operation.}{fig:lecture12:lecture12Fig1}{0.15}

This is a unitary operator

\begin{equation}\label{eqn:qmLecture12:120}
\hatT_{-\epsilon} = \hatT_{\epsilon}^\dagger = \hatT_{\epsilon}^{-1}
\end{equation}

In a position basis, the action of this operator is

\begin{dmath}\label{eqn:qmLecture12:140}
\bra{x} \hatT_{\epsilon} \ket{\psi} = \braket{x-\epsilon}{\psi} = \psi(x - \epsilon)
\end{dmath}

\begin{dmath}\label{eqn:qmLecture12:160}
\Psi(x - \epsilon) \approx \Psi(x) - \epsilon \PD{x}{\Psi}
\end{dmath}

\begin{dmath}\label{eqn:qmLecture12:180}
\bra{x} \hatT_{\epsilon} \ket{\Psi}
= \braket{x}{\Psi} - \frac{\epsilon}{\Hbar} \bra{ x} i \hatp \ket{\Psi}
\end{dmath}

\begin{dmath}\label{eqn:qmLecture12:200}
\hatT_{\epsilon} \approx \lr{ 1 - i \frac{\epsilon}{\Hbar} \hatp }
\end{dmath}

A non-infinitesimal translation can be composed of many small translations, as sketched in \cref{fig:lecture12:lecture12Fig2}.

\imageFigure{../../figures/phy1520/lecture12Fig2}{Composition of small translations.}{fig:lecture12:lecture12Fig2}{0.15}

For \( \epsilon \rightarrow 0, N \rightarrow \infty, N \epsilon = a \), the total translation operator is

\begin{dmath}\label{eqn:qmLecture12:220}
\hatT_{a}
= \hatT_{\epsilon}^N
= \lim_{\epsilon \rightarrow 0, N \rightarrow \infty, N \epsilon = a }
\lr{ 1 - \frac{\epsilon}{\Hbar} \hatp }^N
= e^{-i a \hatp/\Hbar}
\end{dmath}

The momentum \( \hatp \) is called a ``Generator'' \index{generator!translation} of translations.  If a Hamiltonian \( H \) is translationally invariant, then

\begin{dmath}\label{eqn:qmLecture12:240}
\antisymmetric{\hatT_{a}}{H} = 0, \qquad \forall a.
\end{dmath}

This means that momentum will be a good quantum number

\begin{dmath}\label{eqn:qmLecture12:260}
\antisymmetric{\hatp}{H} = 0.
\end{dmath}

\section{Rotations}
\index{infitesimal rotation}

Rotations form a non-Abelian group \index{Abelian group}, since the order of rotations \( \hatR_1 \hatR_2 \ne \hatR_2 \hatR_1 \).

Given a rotation acting on a ket

\begin{dmath}\label{eqn:qmLecture12:280}
\hatR \ket{\Br} = \ket{R \Br},
\end{dmath}

observe that the action of the rotation operator on a wave function is inverted

\begin{dmath}\label{eqn:qmLecture12:300}
\bra{\Br} \hatR \ket{\Psi}
=
\bra{R^{-1} \Br} \ket{\Psi}
= \Psi(R^{-1} \Br).
\end{dmath}

\makeexample{Z axis normal rotation}{example:qmLecture12:1}{

Consider an infinitesimal rotation about the z-axis as sketched in \cref{fig:lecture12:lecture12Fig3}.

\imageTwoFigures
{../../figures/phy1520/lecture12Fig3}
{../../figures/phy1520/lecture12Fig4}
{Rotation about z-axis.}{fig:lecture12:lecture12Fig3}{scale=0.1}

\begin{equation}\label{eqn:qmLecture12:320}
\begin{aligned}
x' &= x - \epsilon y \\
y' &= y + \epsilon y \\
z' &= z
\end{aligned}
\end{equation}

The rotated wave function is

\begin{dmath}\label{eqn:qmLecture12:340}
\tilde{\Psi}(x,y,z)
= \Psi( x + \epsilon y, y - \epsilon x, z )
=
 \Psi( x, y, z )
+
\epsilon y
\mathLabelBox
[ labelstyle={below of=m\themathLableNode, below of=m\themathLableNode} ]
{\PD{x}{\Psi}}{\(i \hatp_x/\Hbar\)}
-
\epsilon x
\mathLabelBox
[ labelstyle={below of=m\themathLableNode, below of=m\themathLableNode} ]
{\PD{y}{\Psi}}{\(i \hatp_y/\Hbar\)}.
\end{dmath}

The state must then transform as

\begin{dmath}\label{eqn:qmLecture12:360}
\ket{\tilde{\Psi}}
=
\lr{
1
+ i \frac{\epsilon}{\Hbar} \haty \hatp_x
- i \frac{\epsilon}{\Hbar} \hatx \hatp_y
}
\ket{\Psi}.
\end{dmath}

Observe that the combination \( \hatx \hatp_y - \haty \hatp_x \) is the \( \hatL_z \) component of angular momentum \( \Lcap = \rcap \cross \pcap \), so the infinitesimal rotation can be written

%\begin{dmath}\label{eqn:qmLecture12:380}
\boxedEquation{eqn:qmLecture12:400}{
\hatR_z(\epsilon) \ket{\Psi}
=
\lr{ 1 - i \frac{\epsilon}{\Hbar} \hatL_z } \ket{\Psi}.
}
%\end{dmath}

For a finite rotation \( \epsilon \rightarrow 0, N \rightarrow \infty, \phi = \epsilon N \), the total rotation is

\begin{dmath}\label{eqn:qmLecture12:420}
\hatR_z(\phi)
=
\lr{ 1 - \frac{i \epsilon}{\Hbar} \hatL_z }^N,
\end{dmath}

or
%\begin{dmath}\label{eqn:qmLecture12:440}
\boxedEquation{eqn:qmLecture12:460}{
\hatR_z(\phi)
=
e^{-i \frac{\phi}{\Hbar} \hatL_z}.
}
%\end{dmath}

Note that \( \antisymmetric{\hatL_x}{\hatL_y} \ne 0 \).
} % example

By construction using Euler angles or any other method, a general rotation will include contributions from components of all the angular momentum operator, and will have the structure

%\begin{dmath}\label{eqn:qmLecture12:480}
\boxedEquation{eqn:qmLecture12:500}{
\hatR_\ncap(\phi)
=
e^{-i \frac{\phi}{\Hbar} \lr{ \Lcap \cdot \ncap }}.
}
%\end{dmath}

\paragraph{Rotationally invariant \( \hatH \).}
\index{rotation invariance}

Given a rotationally invariant Hamiltonian

\begin{dmath}\label{eqn:qmLecture12:520}
\antisymmetric{\hatR_\ncap(\phi)}{\hatH} = 0 \qquad \forall \ncap, \phi,
\end{dmath}

then every

\begin{dmath}\label{eqn:qmLecture12:540}
\antisymmetric{\BL \cdot \ncap}{\hatH} = 0,
\end{dmath}

or
\begin{dmath}\label{eqn:qmLecture12:560}
\antisymmetric{L_i}{\hatH} = 0,
\end{dmath}

Non-Abelian implies degeneracies in the spectrum.

\section{Time-reversal}
\index{time reversal}

Imagine that we have something moving along a curve at time \( t = 0 \), and ending up at the final position at time \( t = t_f \), as sketched in \cref{fig:lecture12:lecture12Fig5}.

\imageFigure{../../figures/phy1520/lecture12Fig5}{Time reversal trajectory.}{fig:lecture12:lecture12Fig5}{0.15}

Now imagine that we flip the direction of motion (i.e. flipping the velocity) and run time backwards so the final-time state becomes the initial state.

If the time reversal operator is designated \( \hat\Theta \), with operation

\begin{dmath}\label{eqn:qmLecture12:580}
\hat\Theta \ket{\Psi} = \ket{\tilde{\Psi}},
\end{dmath}

so that

\begin{dmath}\label{eqn:qmLecture12:600}
\hat\Theta^{-1} e^{-i \hatH t/\Hbar} \hat\Theta \ket{\Psi(t)} = \ket{\Psi(0)},
\end{dmath}

or

\begin{dmath}\label{eqn:qmLecture12:620}
\hat\Theta^{-1} e^{-i \hatH t/\Hbar} \hat\Theta \ket{\Psi(0)} = \ket{\Psi(-t)}.
\end{dmath}

%\EndArticle
