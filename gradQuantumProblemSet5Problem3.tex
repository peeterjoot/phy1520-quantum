%
% Copyright � 2015 Peeter Joot.  All Rights Reserved.
% Licenced as described in the file LICENSE under the root directory of this GIT repository.
%
\makeoproblem{Simultaneous eigenstates.}{gradQuantum:problemSet5:3}{2015 ps5 p3}{
\index{simultaneous eigenstate}
\index{anticommuting}
\index{parity}
%\makesubproblem{}{gradQuantum:problemSet5:3a}

A quantum state \( \ket{\Psi} \) is a simultaneous eigenstate of two anticommuting Hermitian operators \( A, B \), with \( A B + B A = 0\).
What can you say about the eigenvalues of \( A, B \) for the state \(\ket{\Psi}\)? Illustrate your point using the parity operator and the momentum operator.
} % makeproblem

\makeanswer{gradQuantum:problemSet5:3}{
\withproblemsetsParagraph{

For the eigenvalues of the respective states, let
%
\begin{equation}\label{eqn:gradQuantumProblemSet5Problem3:20}
A \ket{\Psi} = a \ket{\Psi},
\end{equation}
%
and
\begin{equation}\label{eqn:gradQuantumProblemSet5Problem3:40}
B \ket{\Psi} = b \ket{\Psi}.
\end{equation}
%
The action of the anticommutator on this state is
%
\begin{dmath}\label{eqn:gradQuantumProblemSet5Problem3:60}
\lr{ A B + B A } \ket{\Psi}
=
\lr{ a b + b a } \ket{\Psi}
=
0.
\end{dmath}
%
This means the product \( 2 a b \), is zero, and that at least one of \( a \) or \( b \) must be zero.

The concrete example of the parity operator \( \hat\Pi \) and the momentum operator \( \hatp \) demonstrates this.
Because the eigenvalues of the parity operator can only be \( \pm 1 \), if such a simultaneous eigenstate exists in the momentum basis, it must have a \( p = 0 \) eigenvalue.  To see this explicitly, assume there is some value \( p \) for which \( \ket{p} \) is a simultaneous eigenstate of \( \hat\Pi \) and \( \hatp \).  First consider
%
\begin{dmath}\label{eqn:gradQuantumProblemSet5Problem3:80}
\hat\Pi \hatp \ket{p}
=
p \hat\Pi \ket{p}
=
- p \ket{p}.
\end{dmath}
%
Since these operators anticommute
%
\begin{dmath}\label{eqn:gradQuantumProblemSet5Problem3:100}
\hat\Pi \hatp = - \hatp \hat\Pi,
\end{dmath}
%
we must also have
%
\begin{dmath}\label{eqn:gradQuantumProblemSet5Problem3:120}
\hat\Pi \hatp \ket{p}
= - \hatp \hat\Pi \ket{p}
= \hatp \ket{p}
= p \ket{p}.
\end{dmath}
%
The only way this can be satisfied is when the eigenvalue of this state is \( p = 0 \).
}
}
