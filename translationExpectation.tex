%
% Copyright � 2015 Peeter Joot.  All Rights Reserved.
% Licenced as described in the file LICENSE under the root directory of this GIT repository.
%
%\input{../blogpost.tex}
%\renewcommand{\basename}{translationExpectation}
%\renewcommand{\dirname}{notes/phy1520/}
%%\newcommand{\dateintitle}{}
%%\newcommand{\keywords}{}
%
%\input{../peeter_prologue_print2.tex}
%
%\usepackage{peeters_layout_exercise}
%\usepackage{peeters_braket}
%\usepackage{peeters_figures}
%
%\beginArtNoToc
%
%\generatetitle{SHO translation operator expectation}
%\chapter{SHO translation operator expectation}
%\label{chap:translationExpectation}
%
\makeoproblem{SHO translation operator expectation.}{problem:translationExpectation:2.12}{\citep{sakurai2014modern} pr. 2.12}{
%
\index{translation!expectation}
\index{harmonic oscillator!translation operator}
Using the Heisenberg picture evaluate the expectation of the position operator \( \expectation{x} \) with respect to the initial time state
%
\begin{equation}\label{eqn:translationExpectation:20}
\ket{\alpha, 0} = e^{-i p_0 a/\Hbar} \ket{0},
\end{equation}
%
where \( p_0 \) is the initial time position operator, and \( a \) is a constant with dimensions of position.
%
} % problem
%
\makeanswer{problem:translationExpectation:2.12}{
%
Recall that the Heisenberg picture position operator expands to
%
\begin{equation}\label{eqn:translationExpectation:40}
\begin{aligned}
x^\txtH(t)
&= U^\dagger x U
\\ &= x_0 \cos(\omega t) + \frac{p_0}{m \omega} \sin(\omega t),
\end{aligned}
\end{equation}
%
so the expectation of the position operator is
\begin{equation}\label{eqn:translationExpectation:60}
\begin{aligned}
\expectation{x}
&=
\bra{0} e^{i p_0 a/\Hbar} \lr{ x_0 \cos(\omega t) + \frac{p_0}{m \omega} \sin(\omega t) } e^{-i p_0 a/\Hbar} \ket{0}
\\ &=
\bra{0} \lr{ e^{i p_0 a/\Hbar} x_0 \cos(\omega t) e^{-i p_0 a/\Hbar} \cos(\omega t) + \frac{p_0}{m \omega} \sin(\omega t) } \ket{0}.
\end{aligned}
\end{equation}
%
The exponential sandwich above can be expanded using the Baker-Campbell-Hausdorff \citep{wiki:bakercampbellHausdorff} formula
%
\begin{equation}\label{eqn:translationExpectation:80}
\begin{aligned}
e^{i p_0 a/\Hbar} x_0 e^{-i p_0 a/\Hbar}
&=
x_0
+ \frac{i a}{\Hbar} \antisymmetric{p_0}{x_0}
+ \inv{2!} \lr{\frac{i a}{\Hbar}}^2 \antisymmetric{p_0}{\antisymmetric{p_0}{x_0}}
+ \cdots \\
&=
x_0
+ \frac{i a}{\Hbar} \lr{ -i \Hbar }
+ \inv{2!} \lr{\frac{i a}{\Hbar}}^2 \antisymmetric{p_0}{-i \Hbar}
+ \cdots \\
&=
x_0 + a.
\end{aligned}
\end{equation}
%
The position expectation with respect to this translated state is
%
\begin{equation}\label{eqn:translationExpectation:100}
\begin{aligned}
\expectation{x}
&= \bra{0} \lr{ (x_0 + a)\cos(\omega t) + \frac{p_0}{m \omega} \sin(\omega t) }\ket{0}
\\ &= a \cos(\omega t).
\end{aligned}
\end{equation}
%
The final simplification above follows from \( \bra{n} x \ket{n} = \bra{n} p \ket{n} = 0 \).
%
} % answer

%\EndArticle
