%
% Copyright � 2015 Peeter Joot.  All Rights Reserved.
% Licenced as described in the file LICENSE under the root directory of this GIT repository.
%
%\input{../blogpost.tex}
%\renewcommand{\basename}{qmLecture15}
%\renewcommand{\dirname}{notes/phy1520/}
%\newcommand{\keywords}{PHY1520H}
%\input{../peeter_prologue_print2.tex}
%
%%\usepackage{phy1520}
%\usepackage{peeters_braket}
%%\usepackage{peeters_layout_exercise}
%\usepackage{peeters_figures}
%\usepackage{mathtools}
%
%\beginArtNoToc
%\generatetitle{PHY1520H Graduate Quantum Mechanics.  Lecture 15: angular momentum rotation representation, and angular momentum addition.  Taught by Prof.\ Arun Paramekanti}
%%\chapter{angular momentum rotation representation, and angular momentum addition}
%\label{chap:qmLecture15}
%
%\paragraph{Disclaimer}
%
%Peeter's lecture notes from class.  These may be incoherent and rough.
%
%These are notes for the UofT course PHY1520, Graduate Quantum Mechanics, taught by Prof. Paramekanti, covering \textchapref{{3}} \citep{sakurai2014modern} content.
%
%%%\paragraph{Angular momentum (wrap up.)}
%%%
%%%We found
%%%
%%%\begin{equation}\label{eqn:qmLecture15:20}
%%%\begin{aligned}
%%%\hat{\BL^2} \ket{j, m} &= j(j+1) \Hbar^2 \ket{j,m} \\
%%%\hatL_z \ket{j, m} &= \Hbar m \ket{j,m} \\
%%%\hatL_{\pm} \ket{j, m } &= \Hbar \sqrt{(j \mp m)(j \pm m + 1)} \ket{j, m \pm 1 }
%%%\end{aligned}
%%%\end{equation}
%%%
%%%or Schwinger
%%%
%%%\begin{equation}\label{eqn:qmLecture15:40}
%%%\begin{aligned}
%%%\hatL_z &= \inv{2} \lr{ \hatn_1 - \hatn_2 } \Hbar \\
%%%\hatL_{+} &= a_1^\dagger a_2 \Hbar \\
%%%\hatL_{-} &= a_1 a_2^\dagger \Hbar \\
%%%j &= \inv{2} \lr{ \hatn_1 + \hatn_2 },
%%%\end{aligned}
%%%\end{equation}
%%%
%%%where each of the \( a_1, a_2 \) operators obey
%%%
%%%\begin{equation}\label{eqn:qmLecture15:60}
%%%\begin{aligned}
%%%\antisymmetric{a_1}{a_1^\dagger} &= 1 \\
%%%\antisymmetric{a_2}{a_2^\dagger} &= 1
%%%\end{aligned}
%%%\end{equation}
%%%
%%%and any pair of different index \( a \) operators commute, as in
%%%
%%%\begin{equation}\label{eqn:qmLecture15:80}
%%%\antisymmetric{a_1}{a_2^\dagger} = 0.
%%%\end{equation}
%%%
\section{Representations}

\index{angular momentum!rotation operator}
It's possible to compute matrix representations of the rotation operators

\begin{equation}\label{eqn:qmLecture15:100}
\hatR_\ncap(\phi) = e^{i \Lcap \cdot \ncap \phi/\Hbar}.
\end{equation}

With respect to a ket it's possible to find

\begin{equation}\label{eqn:qmLecture15:120}
e^{i \Lcap \cdot \ncap \phi/\Hbar} \ket{j, m}
=
\sum_{m'} d^j_{m m'}(\ncap, \phi) \ket{ j, m' }.
\end{equation}

This has a block diagonal form that's sketched in \cref{fig:qmLecture15:qmLecture15Fig1}.

\imageFigure{../phy1520-quantum-figuresqmLecture15Fig1}{Block diagonal form for angular momentum matrix representation.}{fig:qmLecture15:qmLecture15Fig1}{0.2}

We can view \( d^j_{m m'}(\ncap, \phi) \) as a matrix, representing the rotation.  The problem of determining these matrices can be reduced to that of determining the matrix for \( \Lcap \), because once we have that we can exponentiate that.

\makeexample{Spin half}{example:qmLecture15:1}{

From the eigenvalue relationships, with basis states

\begin{equation}\label{eqn:qmLecture15:160}
\begin{aligned}
\ket{\uparrow} &=
\begin{bmatrix}
1 \\
0
\end{bmatrix} \\
\ket{\downarrow} &=
\begin{bmatrix}
0 \\
1 \\
\end{bmatrix}
\end{aligned}
\end{equation}

we find

\begin{equation}\label{eqn:qmLecture15:180}
\begin{aligned}
\hatL_z &= \frac{\Hbar}{2} \PauliZ \\
\hatL_{+} &= \frac{\Hbar}{2}
\begin{bmatrix}
0 & 1 \\
0 & 0
\end{bmatrix} \\
\hatL_{-} &= \frac{\Hbar}{2}
\begin{bmatrix}
0 & 0 \\
1 & 0
\end{bmatrix}.
\end{aligned}
\end{equation}

Rearranging we find the Pauli matrices
\index{angular momentum!Pauli matrix}

\begin{dmath}\label{eqn:qmLecture15:200}
\hatL_k = \inv{2} \Hbar \sigma_i.
\end{dmath}

Noting that \( \lr{ \Bsigma \cdot \ncap }^2 = 1 \), and \( \lr{\Bsigma \cdot \ncap }^3 = \Bsigma \cdot \ncap \), the rotation matrix is

\begin{dmath}\label{eqn:qmLecture15:220}
e^{ i \Bsigma \cdot \ncap \phi/2 } \ket{\inv{2}, m} = \lr{ \cos( \phi/2 ) + i \Bsigma \cdot \ncap \sin(\phi/2) } \ket{\inv{2}, m}.
\end{dmath}

} % example

The steps taken in the example above, which apply to all values of \( j \) were

\begin{enumerate}
\item Enumerate the states.
\begin{equation}\label{eqn:qmLecture15:140}
j_1 = \inv{2} \leftrightarrow\, \mbox{2 states (dimension of irreducible representation = 2)}
\end{equation}

\item Construct the \( \Lcap \) matrices.
\item Construct \( d_{m m'}^j(\ncap, \phi) \).
\end{enumerate}

\section{Spherical harmonics}
\index{spherical harmonics}
%Angular momentum operator in space representation}

For \( L = 1 \) it turns out that the rotation matrices turn out to be the 3D rotation matrices.  In the space representation

\begin{dmath}\label{eqn:qmLecture15:240}
\BL = \Br \cross \Bp,
\end{dmath}

the coordinates of the operator are

\begin{dmath}\label{eqn:qmLecture15:260}
\hatL_k = i \epsilon_{k m n} r_m \lr{ -i \Hbar \PD{r_n}{} }
\end{dmath}

We see that scaling \( \Br \rightarrow \alpha \Br \) does not change this operator, allowing for an angular representation \( \hatL_k(\theta, \phi) \) that have the form

\begin{dmath}\label{eqn:qmLecture15:280}
\begin{aligned}
\hatL_z &= -i \Hbar \PD{\phi}{} \\
\hatL_{\pm} &= \Hbar \lr{ \pm \PD{\theta}{} + i \cot \theta \PD{\phi}{} }.
\end{aligned}
\end{dmath}

Here \( \theta \) and \( \phi \) are the polar and azimuthal angles respectively as illustrated in \cref{fig:qmLecture15:qmLecture15Fig2}.

\imageFigure{../phy1520-quantum-figuresqmLecture15Fig2}{Spherical coordinate convention.}{fig:qmLecture15:qmLecture15Fig2}{0.2}

Introducing the \textAndIndex{spherical harmonics} \( Y_{lm} \), the equivalent wave function representation of the problem is

\begin{equation}\label{eqn:qmLecture15:300}
\begin{aligned}
\Lcap Y_{lm}(\theta, \phi) &= \Hbar^2 l (l + 1) Y_{lm}(\theta, \phi) \\
\hatL_z Y_{lm}(\theta, \phi) &= \Hbar m Y_{lm}(\theta, \phi) \\
\end{aligned}
\end{equation}

One can find these functions

\begin{dmath}\label{eqn:qmLecture15:320}
Y_{lm}(\theta, \phi) = P_{l, m}(\cos \theta) e^{i m \phi},
\end{dmath}

where \( P_{l, m}(\cos \theta) \) are called the \textAndIndex{associated Legendre polynomials}.  This can be applied whenever we have

\begin{dmath}\label{eqn:qmLecture15:340}
\antisymmetric{H}{\hatL_k} = 0.
\end{dmath}

where all the eigenfunctions will have the form

\begin{dmath}\label{eqn:qmLecture15:360}
\Psi(r, \theta, \phi) = R(r) Y_{lm}(\theta, \phi).
\end{dmath}

\section{Addition of angular momentum}
\index{angular momentum!addition}

Since \( \Lcap \) is a vector we expect to be able to add angular momentum in some way similar to the addition of classical vectors as illustrated in \cref{fig:qmLecture15:qmLecture15Fig3}.

\imageFigure{../phy1520-quantum-figuresqmLecture15Fig3}{Classical vector addition.}{fig:qmLecture15:qmLecture15Fig3}{0.15}

When we have a potential that depends only on the difference in position \( V(\Br_1 - \Br_2) \) then we know from classical problems it is effective to work in center of mass coordinates

\begin{equation}\label{eqn:qmLecture15:380}
\begin{aligned}
\Rcap_{\textrm{cm}} &= \frac{\rcap_1 + \rcap_2}{2} \\
\Pcap_{\textrm{cm}} &= \pcap_1 + \pcap_2
\end{aligned}
\end{equation}

where

\begin{dmath}\label{eqn:qmLecture15:400}
\antisymmetric{\hatR_i}{\hatP_j} = i \Hbar \delta_{ij}.
\end{dmath}

Given

\begin{dmath}\label{eqn:qmLecture15:420}
\Lcap_1 + \Lcap_2 = \Lcap_{\textrm{tot}},
\end{dmath}

do we have
\begin{dmath}\label{eqn:qmLecture15:440}
\antisymmetric{
\hatL_{\textrm{tot}, i}
}{
\hatL_{\textrm{tot}, j}
}
= i \Hbar \epsilon_{i j k} \hatL_{\textrm{tot}, k} ?
\end{dmath}

That is

\begin{dmath}\label{eqn:qmLecture15:460}
\antisymmetric{\hatL_{1,i} + \hatL_{1,j}}{\hatL_{2,i} + \hatL_{2,j}} = i \Hbar \epsilon_{i j k} \lr{ \hatL_{1,k} + \hatL_{1,k} }
\end{dmath}

FIXME: Right at the end of the lecture, there was a mention of something about whether or not \( \Lcap_1^2 \) and \( \hatL_{1,z} \) were sharply defined, but I missed it.  Ask about this if not covered in the next lecture.

%\EndArticle
