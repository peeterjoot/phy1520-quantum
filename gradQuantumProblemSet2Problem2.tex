%
% Copyright � 2015 Peeter Joot.  All Rights Reserved.
% Licenced as described in the file LICENSE under the root directory of this GIT repository.
%

% Augmented version of problem 2.7 from the text.
\makeoproblem{Virial theorem.}{gradQuantum:problemSet2:2}{phy1520 2015 ps2.2}{
\index{virial theorem}

Consider a three-dimensional Hamiltonian
%
\begin{dmath}\label{eqn:gradQuantumProblemSet2Problem2:21}
H = \frac{\Bp^2}{2m} + V(\Bx).
\end{dmath}
%
\makesubproblem{}{gradQuantum:problemSet2:2a}

Calculate \( \antisymmetric{\Bx \cdot \Bp}{H} \) and show that
%
\begin{dmath}\label{eqn:gradQuantumProblemSet2Problem2:41}
\ddt{} \expectation{ \Bx \cdot \Bp } = \expectation{ \frac{\Bp^2}{m} } - \expectation{ \Bx \cdot \spacegrad V }.
\end{dmath}
%
When the l.h.s. vanishes, the result that the r.h.s. is zero is called the quantum virial theorem.

\makesubproblem{}{gradQuantum:problemSet2:2c}
Consider the 3D isotropic harmonic oscillator, and show explicitly that its eigenstates obey the virial theorem.

\makesubproblem{}{gradQuantum:problemSet2:2d}
Evaluate the r.h.s. for the superposition state \( \ket{0,0,0} + \ket{0,0,2} \) where the notation stands for \( \ket{ n_x, n_y, n_z } \) occupation numbers.

\makesubproblem{}{gradQuantum:problemSet2:2b}
Under what conditions does the left-hand side vanish?
} % makeproblem

%%%%%%%%%%%%%%
%
% from: ../phy1520/gradQuantumProblemSet2Problem2.tex
%
%
\makeanswer{gradQuantum:problemSet2:2}{
\withproblemsetsParagraph{

\makeSubAnswer{}{gradQuantum:problemSet2:2a}

We'll need various commutators to evaluate the Heisenberg equation of motion.
\begin{dmath}\label{eqn:gradQuantumProblemSet2Problem2:40}
\antisymmetric{\Bx \cdot \Bp}{H}
=
\inv{2 m} \antisymmetric{\Bx \cdot \Bp}{\Bp^2} + \antisymmetric{\Bx \cdot \Bp}{V(\Bx)}
=
\inv{2 m} \lr{ x_r p_r \Bp^2 - \Bp^2 x_r p_r}
+
\lr{ x_r p_r V(\Bx) - V(\Bx) x_r p_r }
=
\inv{2 m} \antisymmetric{ x_r }{\Bp^2} p_r
+
x_r \antisymmetric{ p_r}{ V(\Bx)},
\end{dmath}
%
Evaluating those commutators separately, gives
%
\begin{dmath}\label{eqn:gradQuantumProblemSet2Problem2:60}
\begin{aligned}
\antisymmetric{ x_r }{\Bp^2}
&=
\antisymmetric{ x_r }{p_r^2}\qquad \text{(no sum)} \\
&=
2 i \Hbar p_r,
\end{aligned}
\end{dmath}

and
%
\begin{dmath}\label{eqn:gradQuantumProblemSet2Problem2:80}
\antisymmetric{ p_r}{ V(\Bx)}
= -i \Hbar \PD{x_r}{V(\Bx)},
\end{dmath}
%
so
\begin{dmath}\label{eqn:gradQuantumProblemSet2Problem2:100}
\ddt{}\lr{\Bx \cdot \Bp}
=
\inv{i \Hbar}
\antisymmetric{\Bx \cdot \Bp}{H}
=
\inv{2 m} 2 p_r p_r - x_r \PD{x_r}{V(\Bx)}
=
\frac{\Bp^2}{m} - \Bx \cdot \spacegrad V(\Bx).
\end{dmath}
%
Evaluating the expectation of this identity with respect to a stationary state \( \ket{\psi} \) (i.e. a state independent of time), we have
%
\begin{dmath}\label{eqn:gradQuantumProblemSet2Problem2:120}
\bra{\psi} \ddt{} \Bx \cdot \Bp \ket{\psi}
=
\ddt{} \bra{\psi} \Bx \cdot \Bp \ket{\psi}
=
\ddt{} \expectation{ \Bx \cdot \Bp }
= \expectation{\frac{\Bp^2}{m}} - \expectation{\Bx \cdot \spacegrad V(\Bx)}.
\end{dmath}
%
Because the expectation was with respect to a stationary state, the time derivative could be moved outside of the expectation operation.

\makeSubAnswer{}{gradQuantum:problemSet2:2c}
When \cref{eqn:gradQuantumProblemSet2Problem2:120} is zero, we have the quantum equivalent of the virial theorem, relating the average kinetic energy to the potential
%
\begin{dmath}\label{eqn:gradQuantumProblemSet2Problem2:140}
2 \expectation{T} = \expectation{\Bx \cdot \spacegrad V(\Bx)}
\end{dmath}

To evaluate these expectations operations with respect to the 3D SHO eigenstates, let
%
\begin{dmath}\label{eqn:gradQuantumProblemSet2Problem2:240}
\begin{aligned}
a_x(t) &= a_x e^{-i \omega t} \\
a_y(t) &= a_y e^{-i \omega t} \\
a_z(t) &= a_z e^{-i \omega t},
\end{aligned}
\end{dmath}

Note that
%
\begin{dmath}\label{eqn:gradQuantumProblemSet2Problem2:480}
\expectation{ \Bx \cdot \spacegrad V }
=
m \omega^2 \expectation{ \Bx^2(t) }
=
m \omega^2 \frac{x_0^2}{2}
\sum_{k=1}^3
\expectation{ \lr{a_k(t) + a_k^\dagger(t)}^2 }
=
\frac{ \Hbar \omega }{2}
\sum_{k=1}^3
\bra{\psi} \lr{a_k(t) + a_k^\dagger(t)}^2 \ket{\psi}
=
\frac{ \Hbar \omega }{2}
\sum_{k=1}^3
\bra{\psi} \lr{a_k  e^{-i \omega t} + a_k^\dagger e^{i \omega t}}^2 \ket{\psi},
\end{dmath}
%
and
%
\begin{dmath}\label{eqn:gradQuantumProblemSet2Problem2:500}
\expectation{ \frac{\Bp^2}{m} }
=
\frac{-\Hbar^2}{2 m x_0^2}
\sum_{k=1}^3
\expectation{ \lr{a_k^\dagger(t) - a_k(t) }^2 }
=
\frac{ \Hbar \omega }{2}
\sum_{k=1}^3
\bra{\psi}
\lr{ a_k(t) - a_k^\dagger(t) }
\lr{ a_k^\dagger(t) - a_k(t) }
\ket{\psi}
=
\frac{ \Hbar \omega }{2}
\sum_{k=1}^3
\bra{\psi}
\lr{ a_k e^{-i \omega t}- a_k^\dagger e^{i \omega t}}
\lr{ a_k^\dagger e^{i \omega t} - a_k e^{-i \omega t}}
\ket{\psi}.
\end{dmath}
%
In both cases the pairs of raising and lowering operators have been factored into conjugate pairs so that only the action of the latter on \( \ket{\psi} \) need be considered.  Considering the \( x \) component for example, we've got
%
\begin{dmath}\label{eqn:gradQuantumProblemSet2Problem2:540}
\lr{ a_x^\dagger e^{i \omega t} \pm a_x e^{-i \omega t}}
\ket{n_x, n_y, n_z}
=
e^{i \omega t} \sqrt{n_x + 1} \ket{n_x + 1, n_y, n_z}
\pm
e^{-i \omega t} \sqrt{n_x} \ket{n_x - 1, n_y, n_z}.
\end{dmath}
%
This gives
%
\begin{equation}\label{eqn:gradQuantumProblemSet2Problem2:560}
\expectation{ \Bx \cdot \spacegrad V } = \expectation{ \frac{\Bp^2}{m} }
=
\frac{\Hbar \omega}{2}
\lr{
(n_x + 1) + n_x
+(n_y + 1) + n_y
+(n_z + 1) + n_z
},
\end{equation}
%
or
%
\boxedEquation{eqn:gradQuantumProblemSet2Problem2:580}{
\expectation{ \Bx \cdot \spacegrad V } = \expectation{ \frac{\Bp^2}{m} }
=
\Hbar \omega
\lr{ n_x + n_y + n_z + \frac{3}{2} }.
}

Neither expectation has any time dependence, and the virial theorem has been confirmed.

\makeSubAnswer{}{gradQuantum:problemSet2:2d}

With \( \ket{\psi} = \ket{0,0,0} + \ket{0,0,2} \), the gradient portion of the RHS is
%
\begin{dmath}\label{eqn:gradQuantumProblemSet2Problem2:600}
\expectation{ \Bx \cdot \spacegrad V }
=
m \omega^2 \expectation{ \Bx^2(t) }
=
m \omega^2 \frac{x_0^2}{2}
\sum_{k=1}^3 \bra{\psi} \lr{ a_k e^{-i \omega t} + a_k^\dagger e^{i \omega t} }^2 \ket{\psi}
=
\frac{\Hbar \omega}{2}
\sum_{k=1}^3 \bra{\psi} \lr{ a_k e^{-i \omega t} + a_k^\dagger e^{i \omega t} }^2 \ket{\psi}
\end{dmath}
%
%\begin{dmath}\label{eqn:gradQuantumProblemSet2Problem2:260}
%\begin{aligned}
%\expectation{ \Bx \cdot \spacegrad V }
%&=
%%m \omega^2 \expectation{ \Bx^2(t) } \\
%%&=
%%m \omega^2 \frac{x_0^2}{2}
%\frac{\Hbar \omega}{2}
%\Biglr{
%\bra{\psi} \lr{ a_x e^{-i \omega t} + a_x^\dagger e^{i \omega t} }^2 \ket{\psi} \\
%&\qquad  +\bra{\psi} \lr{ a_y e^{-i \omega t} + a_y^\dagger e^{i \omega t} }^2 \ket{\psi} \\
%&\qquad  +\bra{\psi} \lr{ a_z e^{-i \omega t} + a_z^\dagger e^{i \omega t} }^2 \ket{\psi} } \\
%&=
%\frac{\Hbar \omega}{2}
%\bra{0_x} \lr{ a_x e^{-i \omega t} + a_x^\dagger e^{i \omega t} }^2 \ket{0_x}
%\braket{0_y}{0_y}\lr{\bra{0_z} + \bra{2_z}}\lr{\ket{0_z} + \ket{2_z}} \\
%&+
%\frac{\Hbar \omega}{2}
%\braket{0_x}{0_x}
%\bra{0_y} \lr{ a_y e^{-i \omega t} + a_y^\dagger e^{i \omega t} }^2 \ket{0_y}
%\lr{\bra{0_z} + \bra{2_z}}\lr{\ket{0_z} + \ket{2_z}} \\
%&+
%\frac{\Hbar \omega}{2}
%\braket{0_x}{0_x}\braket{0_y}{0_y}
%\lr{\bra{0_z} + \bra{2_z}}\lr{ a_z e^{-i \omega t} + a_z^\dagger e^{i \omega t} }^2
%\lr{\ket{0_z} + \ket{2_z}}
%\end{aligned}
%\end{dmath}

We require the following for each \( k \)
%
\begin{dmath}\label{eqn:gradQuantumProblemSet2Problem2:620}
\lr{ a_k e^{-i \omega t} + a_k^\dagger e^{i \omega t} } \lr{ \ket{0,0,0} + \ket{0,0,2} }.
\end{dmath}
%
For \( k = 1 \) this is
%
\begin{dmath}\label{eqn:gradQuantumProblemSet2Problem2:640}
\ket{1,0,0} + \ket{1,0,2},
\end{dmath}
%
for \( k = 2 \) this is
%
\begin{dmath}\label{eqn:gradQuantumProblemSet2Problem2:660}
\ket{0,1,0} + \ket{0,1,2},
\end{dmath}
%
and for \( k = 3 \)
%
\begin{dmath}\label{eqn:gradQuantumProblemSet2Problem2:680}
\begin{aligned}
e^{-i \omega t} &\sqrt{2} \ket{0,0,1}
+
e^{i \omega t}
\lr{
\ket{0,0,1} + \sqrt{3} \ket{0,0,3}
} \\
&=
\ket{0,0,1}
\lr{
e^{-i \omega t} \sqrt{2} + e^{i \omega t}
}
+
\sqrt{3} e^{i \omega t}
\ket{0,0,3}.
\end{aligned}
\end{dmath}
%
This gives
%
\begin{dmath}\label{eqn:gradQuantumProblemSet2Problem2:700}
\expectation{ \Bx \cdot \spacegrad V }
=
\frac{\Hbar\omega}{2}
\lr{
2 + 2 + 3 + \Abs{ e^{-i \omega t} \sqrt{2} + e^{i \omega t} }^2
}
=
\frac{\Hbar\omega}{2}
\lr{
2 + 2 + 3 + 1 + 2 + 2 \sqrt{2} \cos(2 \omega t)
}
\end{dmath}

or
%
%\begin{dmath}\label{eqn:gradQuantumProblemSet2Problem2:340}
\boxedEquation{eqn:gradQuantumProblemSet2Problem2:340}{
\expectation{ \Bx \cdot \spacegrad V }
=
\Hbar \omega
\lr{ 5 + \sqrt{2} \cos( 2 \omega t ) }.
}
%\end{dmath}
%
For the kinetic portion, we've got
%
\begin{dmath}\label{eqn:gradQuantumProblemSet2Problem2:360}
\expectation{\frac{\Bp^2}{m}}
=
%\frac{\Hbar^2}{2 x_0^2 m}
\frac{\Hbar \omega}{2}
\sum_{k=1}^3
\bra{\psi}
\lr{ a_k e^{-i\omega t} - a_k^\dagger e^{i\omega t} }
\lr{ a_k^\dagger e^{i\omega t} - a_k e^{-i\omega t} } \ket{\psi}.
\end{dmath}
%
The squared momentum operator has been factored into conjugate pairs, so only the action on \( \ket{\psi} \) need be computed.  For the \( x \) component of that operation we have
%
\begin{dmath}\label{eqn:gradQuantumProblemSet2Problem2:380}
\lr{ a_x^\dagger e^{i\omega t} - a_x e^{-i\omega t} } \lr{ \ket{0,0,0} + \ket{0,0,2} }
=
a_x^\dagger e^{i\omega t} \lr{ \ket{0,0,0} + \ket{0,0,2} }
=
e^{i\omega t}
\lr{ \ket{1,0,0} + \ket{1,0,2} }.
\end{dmath}
%
The \( y \) component, by inspection, must be
%
\begin{dmath}\label{eqn:gradQuantumProblemSet2Problem2:400}
\lr{ a_x^\dagger e^{i\omega t} - a_x e^{-i\omega t} } \lr{ \ket{0,0,0} + \ket{0,0,2} }
=
e^{i\omega t}
\lr{ \ket{0,1,0} + \ket{0,1,2} }.
\end{dmath}
%
The \( z \) component is slightly messier to compute
%
\begin{dmath}\label{eqn:gradQuantumProblemSet2Problem2:420}
\lr{ a_z^\dagger e^{i\omega t} - a_z e^{-i\omega t} } \lr{ \ket{0,0,0} + \ket{0,0,2} }
=
e^{i\omega t} \lr{ \ket{0,0,1} + \sqrt{3} \ket{0,0,3} }
- e^{-i\omega t} \sqrt{2} \ket{0,0,1}
=
\lr{ e^{i\omega t} - e^{-i\omega t} \sqrt{2} } \ket{0,0,1}
+ e^{i\omega t} \sqrt{3} \ket{0,0,3}.
\end{dmath}
%
Summation of all the component contributions gives
%
\begin{dmath}\label{eqn:gradQuantumProblemSet2Problem2:440}
\expectation{\frac{\Bp^2}{m}}
=
\frac{\Hbar \omega}{2} \lr{
2 + 2 + 3 +
+ \Abs{e^{i\omega t} - e^{-i\omega t} \sqrt{2}}^2
}
=
\frac{\Hbar \omega}{2} \lr{ 4 + 3 + 1 + 2 - 2 \sqrt{2} \cos(2 \omega t)},
\end{dmath}
%
or
\boxedEquation{eqn:gradQuantumProblemSet2Problem2:460}{
\expectation{\frac{\Bp^2}{m}}
=
\Hbar \omega \lr{ 5 - \sqrt{2} \cos(2 \omega t) }.
}

This does not match \cref{eqn:gradQuantumProblemSet2Problem2:340}.

\makeSubAnswer{}{gradQuantum:problemSet2:2b}

To derive \cref{eqn:gradQuantumProblemSet2Problem2:41} the expectation had to be calculated with respect to stationary (non-time dependent) states.  As an example, we confirmed this by showing the r.h.s was zero with respect to the energy eigenstates.  For the superposition state \( \ket{0,0,0} + \ket{0,0,2} \) this was observed to be insufficient.  It is clear that the perfect cancellation of the time dependence that was required so that \( \expectation{\frac{\Bp^2}{m}} = \expectation{ \Bx \cdot \spacegrad V } \) will not, in general, be possible for such superposition states.  The virial theorem requires not only expectations with respect to stationary states, but requires those stationary states to also be energy eigenstates.

%%%\paragraph{Junk?}
%%%We've seen above for the harmonic oscillator that the r.h.s vanished when the expectations were with respect to energy eigenstates.
%%%
%%%This derivative will be equal zero when
%%%\begin{dmath}\label{eqn:gradQuantumProblemSet2Problem2:160}
%%%0
%%%=
%%%\ddt{} \expectation{ \Bx \cdot \Bp }
%%%=
%%%\bra{\psi} \ddt{} \Bx \cdot \Bp \ket{\psi}
%%%=
%%%\bra{\psi} \ddt{\Bx} \cdot \Bp + \Bx \cdot \ddt{\Bp} \ket{\psi}
%%%=
%%%\bra{\psi} \PD{\Bp}{H} \cdot \Bp - \Bx \cdot \PD{\Bx}{H} \ket{\psi},
%%%\end{dmath}
%%%
%%%or when
%%%
%%%\begin{dmath}\label{eqn:gradQuantumProblemSet2Problem2:180}
%%%\expectation{ \PD{\Bp}{H} \cdot \Bp } = \expectation{\Bx \cdot \PD{\Bx}{H}}.
%%%\end{dmath}
%%%
%%%Consider the SHO Hamiltonian for example.  For that we have
%%%
%%%\begin{dmath}\label{eqn:gradQuantumProblemSet2Problem2:200}
%%%\expectation{ \PD{\Bp}{H} \cdot \Bp }
%%%=
%%%\expectation{\frac{\Bp}{m} \cdot \Bp}
%%%=
%%%\expectation{\frac{\Bp^2}{m}},
%%%\end{dmath}
%%%
%%%and
%%%\begin{dmath}\label{eqn:gradQuantumProblemSet2Problem2:220}
%%%\expectation{ \Bx \cdot \PD{\Bx}{H} }
%%%=
%%%\expectation{ \Bx \cdot m \omega^2 \Bx }
%%%=
%%%m \omega^2 \expectation{ \Bx^2 }.
%%%\end{dmath}
%%%
%%%In one dimension, with expectation relative to state \( \ket{n} \) these both equal \( \Hbar \omega (n + 1/2) \), the energy of state \( \ket{n} \).  It seems plausible that there is an additional argument that could be used to show that is the case more generally.
}
}
