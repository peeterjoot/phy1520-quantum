%
% Copyright � 2015 Peeter Joot.  All Rights Reserved.
% Licenced as described in the file LICENSE under the root directory of this GIT repository.
%
%\input{../blogpost.tex}
%\renewcommand{\basename}{gmLecture4}
%\renewcommand{\dirname}{notes/phy1520/}
%%\newcommand{\dateintitle}{}
%\newcommand{\keywords}{PHY1520H}
%
%\input{../peeter_prologue_print2.tex}
%%\usepackage{phy1520}
%\usepackage{peeters_braket}
%%\usepackage{peeters_layout_exercise}
%\usepackage{peeters_figures}
%\usepackage{mathtools}
%
%\beginArtNoToc
%\generatetitle{PHY1520H Graduate Quantum Mechanics.  Lecture 4: Quantum Harmonic oscillator and coherent states.  Taught by Prof.\ Arun Paramekanti}
%%\chapter{Quantum Harmonic oscillator and coherent states}
%\label{chap:lecture4}
%
%\paragraph{Disclaimer}
%
%Peeter's lecture notes from class.  These may be incoherent and rough.  This lecture reviewed a lot of quantum harmonic oscillator theory, and wouldn't make sense without having seen raising and lowering operators (ladder operators), number operators, and the like.
%
%These are notes for the UofT course PHY1520, Graduate Quantum Mechanics, taught by Prof. Paramekanti, covering \textchapref{{2}} \citep{sakurai2014modern} content.
%
\section{Classical Harmonic Oscillator.}
Recall the classical Harmonic oscillator equations in their Hamiltonian form

\begin{subequations}
\label{eqn:qmLecture4:20}
\begin{equation}\label{eqn:qmLecture4:40}
\ddt{x} = \frac{p}{m}
\end{equation}
\begin{equation}\label{eqn:qmLecture4:60}
\ddt{p} = -k x.
\end{equation}
\end{subequations}

With
%
\begin{equation}\label{eqn:qmLecture4:140}
\begin{aligned}
x(t = 0) &= x_0 \\
p(t = 0) &= p_0 \\
k &= m \omega^2,
\end{aligned}
\end{equation}

\index{classical harmonic oscillator}
the solutions are ellipses in phase space

\begin{subequations}
\label{eqn:qmLecture4:80}
\begin{equation}\label{eqn:qmLecture4:100}
x(t) = x_0 \cos(\omega t) + \frac{p_0}{m \omega} \sin(\omega t)
\end{equation}
\begin{equation}\label{eqn:qmLecture4:120}
p(t) = p_0 \cos(\omega t) - m \omega x_0 \sin(\omega t).
\end{equation}
\end{subequations}

After a suitable scaling of the variables, these elliptical orbits can be transformed into circular trajectories.

\section{Quantum Harmonic Oscillator.}
\index{harmonic oscillator}
%
Starting with the Hamiltonian for the harmonic oscillator
\begin{equation}\label{eqn:qmLecture4:160}
\hatH = \frac{\hatp^2}{2 m} + \inv{2} k \hatx^2,
\end{equation}
set
%\begin{subequations}
%\label{eqn:qmLecture4:180}
\begin{equation}\label{eqn:qmLecture4:200}
\hatX = \sqrt{\frac{m \omega}{\Hbar}} \hatx,\qquad
%\end{equation}
%\begin{equation}\label{eqn:qmLecture4:220}
\hatP = \sqrt{\inv{m \omega \Hbar}} \hatp
\end{equation}
%\end{subequations}
With this change of variables the commutators transform from
%
\begin{equation}\label{eqn:qmLecture4:240}
\antisymmetric{ \hatx}{\hatp} = i \Hbar,
\end{equation}
%
to
\begin{equation}\label{eqn:qmLecture4:260}
\antisymmetric{ \hatX}{\hatP} = i,
\end{equation}
%
and the Hamiltonian can be reexpressed in a factorized form
%
\begin{equation}\label{eqn:qmLecture4:280}
\begin{aligned}
\hatH
&= \frac{\Hbar \omega}{2} \lr{ \hatX^2 + \hatP^2 }
\\ &= \Hbar \omega \lr{ \lr{ \frac{\hatX -i \hatP}{\sqrt{2}} } \lr{ \frac{\hatX +i \hatP}{\sqrt{2}}} + \inv{2} }.
\end{aligned}
\end{equation}
%
\index{ladder operator}
\index{raising operator}
\index{lowering operator}
Define ladder operators (raising and lowering operators respectively)
\begin{subequations}
\label{eqn:qmLecture4:300}
\begin{equation}\label{eqn:qmLecture4:320}
\hata^\dagger = \frac{\hatX -i \hatP}{\sqrt{2}}
\end{equation}
\begin{equation}\label{eqn:qmLecture4:340}
\hata = \frac{\hatX +i \hatP}{\sqrt{2}},
\end{equation}
\end{subequations}
so
%
\begin{equation}\label{eqn:qmLecture4:360}
\hatH = \Hbar \omega \lr{ \hata^\dagger \hata + \inv{2} }.
\end{equation}
%
We can show
%
\begin{equation}\label{eqn:qmLecture4:380}
\antisymmetric{\hata}{\hata^\dagger} = 1,
\end{equation}
%
and
\index{number operator}
\begin{equation}\label{eqn:qmLecture4:400}
N \ket{n} \equiv \hata^\dagger a = n \ket{n},
\end{equation}
%
where \( n \ge 0 \) is an integer.

Recall that
%
\begin{equation}\label{eqn:qmLecture4:420}
\hata \ket{0} = 0,
\end{equation}
%
and
%
\begin{equation}\label{eqn:qmLecture4:440}
\bra{X} X + i P \ket{0} = 0.
\end{equation}
%
With
%
\begin{equation}\label{eqn:qmLecture4:460}
\braket{x}{0} = \Psi_0(x),
\end{equation}
%
we can show
%
\begin{equation}\label{eqn:qmLecture4:480}
\inv{\sqrt{2}} \lr{ X + \PD{X}{} } \Psi_0(X) = 0.
\end{equation}
%
Also recall that
\begin{subequations}
\label{eqn:qmLecture4:500}
\begin{equation}\label{eqn:qmLecture4:520}
\hata \ket{n} = \sqrt{n} \ket{n-1}
\end{equation}
\begin{equation}\label{eqn:qmLecture4:540}
\hata^\dagger \ket{n} = \sqrt{n + 1} \ket{n+1}
\end{equation}
\end{subequations}
\section{Coherent states.}
\index{coherent state}
Coherent states for the quantum harmonic oscillator are the eigenkets for the creation and annihilation operators
\begin{subequations}
\label{eqn:qmLecture4:560}
\begin{equation}\label{eqn:qmLecture4:580}
\hata \ket{z} = z \ket{z}
\end{equation}
\begin{equation}\label{eqn:qmLecture4:600}
\hata^\dagger \ket{\tilde{z}} = \tilde{z} \ket{\tilde{z}},
\end{equation}
\end{subequations}
where
%
\begin{equation}\label{eqn:qmLecture4:620}
\ket{z} = \sum_{n = 0}^\infty c_n \ket{n},
\end{equation}
%
and \( z \) is allowed to be a complex number.

Looking for such a state, we compute
%
\begin{equation}\label{eqn:qmLecture4:640}
\begin{aligned}
\hata \ket{z}
&= \sum_{n =1}^\infty c_n \hata \ket{n}
\\ &= 
\sum_{n=1}^\infty c_n \sqrt{n} \ket{n-1}.
\end{aligned}
\end{equation}
Compare this to
%
\begin{equation}\label{eqn:qmLecture4:660}
\begin{aligned}
z \ket{z}
&=
z \sum_{n =0}^\infty c_n \ket{n}
\\ &=
\sum_{n=1}^\infty c_n \sqrt{n} \ket{n-1}
\\ &=
\sum_{n=0}^\infty c_{n+1} \sqrt{n+1} \ket{n},
\end{aligned}
\end{equation}
%
so
%
\begin{equation}\label{eqn:qmLecture4:680}
c_{n+1} \sqrt{n+1} = z c_n
\end{equation}
This gives
%
\begin{equation}\label{eqn:qmLecture4:700}
c_{n+1} = \frac{z c_n}{\sqrt{n+1}}
\end{equation}
%
\begin{equation}\label{eqn:qmLecture4:720}
\begin{aligned}
c_1 &= c_0 z \\
c_2 &= \frac{z c_1}{\sqrt{2}} = \frac{z^2 c_0}{\sqrt{2}} \\
\vdots &
\end{aligned}
\end{equation}
or
%
\begin{equation}\label{eqn:qmLecture4:740}
c_n = \frac{z^n}{\sqrt{n!}}.
\end{equation}
%
So the desired state is
%
\begin{equation}\label{eqn:qmLecture4:760}
\ket{z} = c_0 \sum_{n=0}^\infty \frac{z^n}{\sqrt{n!}} \ket{n}.
\end{equation}
%
Also recall that
%
\begin{equation}\label{eqn:qmLecture4:780}
\ket{n} = \frac{\lr{ \hata^\dagger }^n}{\sqrt{n!}} \ket{0},
\end{equation}
%
which gives
%
\begin{equation}\label{eqn:qmLecture4:800}
\begin{aligned}
\ket{z}
&= c_0 \sum_{n =0}^\infty \frac{\lr{z \hata^\dagger}^n }{n!} \ket{0}
\\ &= c_0 e^{z \hata^\dagger}  \ket{0}.
\end{aligned}
\end{equation}
%
The normalization is
%
\begin{equation}\label{eqn:qmLecture4:820}
c_0 = e^{-\Abs{z}^2/2}.
\end{equation}
%
While we have \( \braket{n_1}{n_2} = \delta_{n_1, n_2} \), these \( \ket{z} \) states are not orthonormal.  Figuring out that this overlap
%
\begin{equation}\label{eqn:qmLecture4:840}
\braket{z_1}{z_2} \ne 0,
\end{equation}
%
will be left for homework.

\section{Coherent state time evolution.}
\index{coherent state!time evolution}

We don't know much about these coherent states.  For example does a coherent state at time zero evolve to a coherent state?
%
\begin{equation}\label{eqn:qmLecture4:860}
\ket{z} \overset{?}{\rightarrow} \ket{z(t)}.
\end{equation}
\index{Heisenberg picture!coherent state}
It turns out that these questions are best tackled in the Heisenberg picture, considering
%
\begin{equation}\label{eqn:qmLecture4:880}
e^{-i \hatH t/\Hbar } \ket{z}.
\end{equation}
%
For example, what is the average of the position operator
%
\begin{equation}\label{eqn:qmLecture4:900}
\bra{z} e^{i \hatH t/\Hbar } \hatx e^{-i \hatH t/\Hbar } \ket{z}
=
\sum_{n, n' = 0}^\infty
\bra{n} c_n^\conj e^{i E_n t/\Hbar}
\lr{ a + a^\dagger} \sqrt{ \frac{\Hbar}{m \omega} }
c_{n'} e^{i E_{n'} t/\Hbar}
\ket{n}.
\end{equation}
%
This is very messy to attempt.  Instead if we know how the operator evolves we can calculate
%
\begin{equation}\label{eqn:qmLecture4:920}
\bra{z} \hatx_\txtH(t) \ket{z},
\end{equation}
%
that is
%
\begin{equation}\label{eqn:qmLecture4:940}
\expectation{\hatx}(t) = \bra{z} \hatx_\txtH(t) \ket{z},
\end{equation}
%
and for momentum
%
\begin{equation}\label{eqn:qmLecture4:960}
\expectation{\hatp}(t) = \bra{z} \hatp_\txtH(t) \ket{z}.
\end{equation}
%
The question to ask is what are the expansions of

\begin{subequations}
\label{eqn:qmLecture4:980}
\begin{equation}\label{eqn:qmLecture4:1000}
\hata_\txtH(t) = e^{i \hatH t/\Hbar} \hata e^{-i \hatH t/\Hbar}.
\end{equation}
\begin{equation}\label{eqn:qmLecture4:1020}
\hata^\dagger_\txtH(t) = e^{i \hatH t/\Hbar} \hata^\dagger e^{-i \hatH t/\Hbar}.
\end{equation}
\end{subequations}
We want to understand how these operators act on the basis states
%
\begin{equation}\label{eqn:qmLecture4:1040}
\begin{aligned}
\hata_\txtH(t) \ket{n}
&= e^{i \hatH t/\Hbar} \hata e^{-i \hatH t/\Hbar} \ket{n}
\\ &= e^{i \hatH t/\Hbar} \hata e^{-i t \omega (n + 1/2)} \ket{n}
\\ &=
e^{-i t \omega (n + 1/2)}
e^{i \hatH t/\Hbar}
\sqrt{n} \ket{n-1}
\\ &=
\sqrt{n}
e^{-i t \omega (n + 1/2)}
e^{i t \omega (n - 1/2)}
\ket{n-1}
\\ &=
\sqrt{n}  e^{-i \omega t} \ket{n-1}
\\ &=
e^{-i \omega t} \ket{n}.
\end{aligned}
\end{equation}
%
So we have found
%
\begin{equation}\label{eqn:qmLecture4:1060}
\begin{aligned}
\hata_\txtH(t) &= a e^{-i\omega t} \\
\hata^\dagger_\txtH(t) &= a^\dagger e^{i\omega t}.
\end{aligned}
\end{equation}
%\paragraph{Position and momentum operator time evolution}
%\EndArticle
