%
% Copyright � 2015 Peeter Joot.  All Rights Reserved.
% Licenced as described in the file LICENSE under the root directory of this GIT repository.
%
%\input{../blogpost.tex}
%\renewcommand{\basename}{gmLecture1}
%\renewcommand{\dirname}{notes/phy1520/}
%%\newcommand{\dateintitle}{}
%%\newcommand{\keywords}{}
%
%\input{../peeter_prologue_print2.tex}
%
%\usepackage{peeters_layout_exercise}
%\usepackage{peeters_braket}
%\usepackage{peeters_figures}
%
%\beginArtNoToc
%
%\generatetitle{PHY1520H Graduate Quantum Mechanics.  Lecture 1: Lighting review.  Taught by Prof.\ Arun Paramekanti}
%\chapter{Lighting review}
%\label{chap:gmLecture1}

%\paragraph{Disclaimer}
%
%Peeter's lecture notes from class.  These may be incoherent and rough.
%
%These are notes for the UofT course PHY1520, Graduate Quantum Mechanics, taught by Prof. Paramekanti, covering \textchapref{{1}} \citep{sakurai2014modern} content.
%%
%
%Text \citep{sakurai2014modern} (revised edition).
%
\section{Classical mechanics.}

We'll be talking about one body physics for most of this course.  In classical mechanics we can figure out the particle trajectories using both of \( (\Br, \Bp \), where
%
\begin{dmath}\label{eqn:qmLecture1:20}
\begin{aligned}
\ddt{\Br} &= \inv{m} \Bp, \\
\ddt{\Bp} &= \spacegrad V.
\end{aligned}
\end{dmath}
A two dimensional phase space as sketched in \cref{fig:lectureOnePhaseSpaceClassical:lectureOnePhaseSpaceClassicalFig1} shows the trajectory of a point particle subject to some equations of motion

\imageFigure{../figures/phy1520-quantum/lectureOnePhaseSpaceClassicalFig1}{One dimensional classical phase space example.}{fig:lectureOnePhaseSpaceClassical:lectureOnePhaseSpaceClassicalFig1}{0.3}

\section{Quantum mechanics.}

For this lecture, we'll work with natural units, setting
%
%\begin{dmath}\label{eqn:qmLecture1:480}
\boxedEquation{eqn:qmLecture1:480}{
\Hbar = 1.
}
%\end{dmath}
%
\index{state vector}
In QM we are no longer allowed to think of position and momentum, but have to start asking about state vectors \( \ket{\Psi} \).

\index{basis}
\index{braket}
We'll consider the state vector with respect to some basis, for example, in a position basis, we write
%
\begin{equation}\label{eqn:qmLecture1:40}
\braket{ x }{\Psi } = \Psi(x),
\end{dmath}
%
\index{wave function}
\index{probability!amplitude}
a complex numbered ``wave function'', the probability amplitude for a particle in \( \ket{\Psi} \) to be in the vicinity of \( x \).

We could also consider the state in a momentum basis
\index{momentum!basis}
%
\begin{equation}\label{eqn:qmLecture1:60}
\braket{ p }{\Psi } = \Psi(p),
\end{dmath}
%
a probability amplitude with respect to momentum \( p \).
More precisely,
%
\begin{equation}\label{eqn:qmLecture1:80}
\Abs{\Psi(x)}^2 dx \ge 0,
\end{dmath}
is the probability of finding the particle in the range \( (x, x + dx ) \).  To have meaning as a probability, we require
\index{normalization}
\begin{equation}\label{eqn:qmLecture1:100}
\int_{-\infty}^\infty \Abs{\Psi(x)}^2 dx = 1.
\end{dmath}
%
\index{probability!density function}
The average position can be calculated using this probability density function.  For example
%
\begin{equation}\label{eqn:qmLecture1:120}
\expectation{x} = \int_{-\infty}^\infty \Abs{\Psi(x)}^2 x dx,
\end{dmath}
%
or
\begin{equation}\label{eqn:qmLecture1:140}
\expectation{f(x)} = \int_{-\infty}^\infty \Abs{\Psi(x)}^2 f(x) dx.
\end{dmath}
%
Similarly, calculation of an average of a function of momentum can be expressed as

\index{expectation}
\begin{equation}\label{eqn:qmLecture1:160}
\expectation{f(p)} = \int_{-\infty}^\infty \Abs{\Psi(p)}^2 f(p) dp.
\end{dmath}
%
\section{Transformation from a position to momentum basis.}
\index{momentum basis}
We have a problem, if we which to compute an average in momentum space such as \( \expectation{p} \), when given a wavefunction \( \Psi(x) \).

How do we convert
%
\begin{dmath}\label{eqn:qmLecture1:180}
\Psi(p)
\overset{?}{\leftrightarrow}
\Psi(x),
\end{dmath}
%
or equivalently
\begin{dmath}\label{eqn:qmLecture1:200}
\braket{p}{\Psi}
\overset{?}{\leftrightarrow}
\braket{x}{\Psi}.
\end{dmath}
%
Such a conversion can be performed by virtue of an the assumption that we have a complete orthonormal basis, for which we can introduce identity operations such as
%
\begin{equation}\label{eqn:qmLecture1:220}
\int_{-\infty}^\infty dp \ket{p}\bra{p} = 1,
\end{dmath}
%
or
\begin{equation}\label{eqn:qmLecture1:240}
\int_{-\infty}^\infty dx \ket{x}\bra{x} = 1
\end{dmath}

Some interpretations:

\begin{enumerate}
\item \( \ket{x_0} \leftrightarrow \text{sits at} x = x_0 \)
\item \( \braket{x}{x'} \leftrightarrow \delta(x - x') \)
\item \( \braket{p}{p'} \leftrightarrow \delta(p - p') \)
\item \( \braket{x}{p'} = \frac{e^{i p x}}{\sqrt{V}} \), where \( V \) is the volume of the box containing the particle.  We'll define the appropriate normalization for an infinite box volume later.
\end{enumerate}

The delta function interpretation of the braket \( \braket{p}{p'} \) justifies the identity operator, since we recover any state in the basis when operating with it.  For example, in momentum space
%
\begin{dmath}\label{eqn:qmLecture1:260}
1 \ket{p}
=
\lr{ \int_{-\infty}^\infty dp'
\ket{p'}\bra{p'} }
\ket{p}
=
\int_{-\infty}^\infty dp'
\ket{p'}
\braket{p'}{p}
=
\int_{-\infty}^\infty dp'
\ket{p'}
\delta(p - p')
=
\ket{p}.
\end{dmath}
%
This also the determination of an integral operator representation for the delta function
%
\begin{dmath}\label{eqn:qmLecture1:500}
\delta(x - x')
=
\braket{x}{x'}
=
\int dp \braket{x}{p} \braket{p}{x'}
=
\inv{V} \int dp e^{i p x} e^{-i p x'},
\end{dmath}
%
or
%
\begin{dmath}\label{eqn:qmLecture1:520}
\delta(x - x')
=
\inv{V} \int dp e^{i p (x- x')}.
\end{dmath}
%
Here we used the fact that \( \braket{p}{x} = \braket{x}{p}^\conj \).

FIXME: do we have a justification for that conjugation with what was defined here so far?

The conversion from a position basis to momentum space is now possible
%
\begin{dmath}\label{eqn:qmLecture1:280}
\braket{p}{\Psi}
=
\Psi(p)
= \int_{-\infty}^\infty \braket{p}{x} \braket{x}{\Psi} dx
= \int_{-\infty}^\infty \frac{e^{-ip x}}{\sqrt{V}} \Psi(x) dx.
\end{dmath}
%
The momentum space to position space conversion can be written as
%
\begin{dmath}\label{eqn:qmLecture1:300}
\Psi(x)
= \int_{-\infty}^\infty \frac{e^{ip x}}{\sqrt{V}} \Psi(p) dp.
\end{dmath}
%
Now we can go back and figure out the an expectation
%
\begin{dmath}\label{eqn:qmLecture1:320}
\expectation{p}
=
\int \Psi^\conj(p) \Psi(p) p d p
=
\int dp
\lr{
\int_{-\infty}^\infty \frac{e^{ip x}}{\sqrt{V}} \Psi^\conj(x) dx
}
\lr{
\int_{-\infty}^\infty \frac{e^{-ip x'}}{\sqrt{V}} \Psi(x') dx'
}
p
=\int dp dx dx'
\Psi^\conj(x)
\inv{V} e^{ip (x-x')} \Psi(x') p
=
\int dp dx dx'
\Psi^\conj(x)
\inv{V} \lr{ -i\PD{x}{e^{ip (x-x')}} }\Psi(x')
=
\int dp dx
\Psi^\conj(x) \lr{ -i \PD{x}{} }
\inv{V} \int dx' e^{ip (x-x')} \Psi(x')
=
\int dx
\Psi^\conj(x) \lr{ -i \PD{x}{} }
\int dx' \lr{ \inv{V} \int dp e^{ip (x-x')} } \Psi(x')
=
\int dx
\Psi^\conj(x) \lr{ -i \PD{x}{} }
\int dx' \delta(x - x') \Psi(x')
=
\int dx
\Psi^\conj(x) \lr{ -i \PD{x}{} }
\Psi(x).
\end{dmath}
%FIXME : Performing this integral we find
%
%\begin{equation}\label{eqn:qmLecture1:340}
%p e^{i p x} \leftrightarrow -i \PD{x}{} e^{i p x},
%\end{dmath}
%
%so
%
%\begin{dmath}\label{eqn:qmLecture1:360}
%\expectation{p}
%=
%\int dp dx dx'
%\lr{ \frac{e^{ip x}}{\sqrt{V}} \Psi^\conj(x)  }
%\lr{ i \frac{\Psi(x')}{\sqrt{V}} \PD{x'}{} e^{-ip x'} }
%= \int \frac{dx dx'}{V} \Psi^\conj(x) \Psi(x) i \PD{x'}{} \lr{ \int dp e^{i p x - i p x'} }
%= \int \frac{dx dx'}{V} \Psi^\conj(x) \Psi(x) i \PD{x'} V \delta(x - x')
%= \int dx dx' \Psi^\conj(x) \Psi(x) i \PD{x'}{} \delta(x - x')
%= \int dx dx' \Psi^\conj(x) \lr{ -i \PD{x'}{\Psi(x')}} \delta(x - x')
%= \int dx \Psi^\conj(x) \lr{ -i \PD{x}{} } \Psi(x).
%\end{dmath}
%
Here we've essentially calculated the position space representation of the momentum operator, allowing identifications of the following form
\index{momentum!operator}
%
\begin{equation}\label{eqn:qmLecture1:380}
p \leftrightarrow -i \PD{x}{}
\end{dmath}
\begin{equation}\label{eqn:qmLecture1:400}
p^2 \leftrightarrow - \PDSq{x}{}.
\end{dmath}
%
\paragraph{Alternate starting point.}

Most of the above results followed from the claim that \( \braket{x}{p} = e^{i p x} \).  Note that this position space representation of the momentum operator can also be taken as the starting point.  Given that, the exponential representation of the position-momentum braket follows
%
\begin{dmath}\label{eqn:qmLecture1:540}
\bra{x} P \ket{p}
=
-i \Hbar \PD{x}{} \braket{x}{p},
\end{dmath}
%
but \( \bra{x} P \ket{p} = p \braket{x}{p} \), providing a differential equation for \( \braket{x}{p} \)
%
\begin{equation}\label{eqn:qmLecture1:560}
p \braket{x}{p} = -i \Hbar \PD{x}{} \braket{x}{p},
\end{dmath}
%
with solution
%
\begin{equation}\label{eqn:qmLecture1:580}
i p x/\Hbar = \ln \braket{x}{p} + \text{const},
\end{dmath}
%
or
\begin{equation}\label{eqn:qmLecture1:600}
\braket{x}{p} \propto e^{i p x/\Hbar}.
\end{dmath}
%
\section{Matrix interpretation.}

\begin{enumerate}
\item Ket's \( \ket{\Psi} \leftrightarrow \text{column vector} \)
\item Bra's \( \bra{\Psi} \leftrightarrow {(\text{row vector})}^\conj \)
\item Operators \( \leftrightarrow \) matrices that act on vectors.
\end{enumerate}
%
\begin{equation}\label{eqn:qmLecture1:420}
\hatp \ket{\Psi} \rightarrow \ket{\Psi'}.
\end{dmath}
\section{Time evolution.}
\index{time evolution}
\index{Hamiltonian}
For a state subject to the equations of motion given by the Hamiltonian operator \( \hatH \)
%
\begin{equation}\label{eqn:qmLecture1:440}
i \PD{t}{} \ket{\Psi} = \hatH \ket{\Psi},
\end{dmath}
%
the time evolution is given by
\begin{equation}\label{eqn:qmLecture1:460}
\ket{\Psi(t)} = e^{-i \hatH t} \ket{\Psi(0)}.
\end{dmath}
%
%\section{Density matrix and incomplete information}
%

%\EndArticle
