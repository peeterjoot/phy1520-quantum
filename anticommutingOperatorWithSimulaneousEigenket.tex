%
% Copyright � 2015 Peeter Joot.  All Rights Reserved.
% Licenced as described in the file LICENSE under the root directory of this GIT repository.
%
%\input{../blogpost.tex}
%\renewcommand{\basename}{anticommutingOperatorWithSimulaneousEigenket}
%\renewcommand{\dirname}{notes/phy1520/}
%%\newcommand{\dateintitle}{}
%%\newcommand{\keywords}{}
%
%\input{../peeter_prologue_print2.tex}
%
%\usepackage{peeters_layout_exercise}
%\usepackage{peeters_braket}
%\usepackage{peeters_figures}
%
%\beginArtNoToc
%
%\generatetitle{Can anticommuting operators have a simultaneous eigenket?}
%\chapter{Can anticommuting operators have a simulaneous eigenket?}
%\label{chap:anticommutingOperatorWithSimulaneousEigenket}

\makeoproblem{Can anticommuting operators have a simultaneous eigenket?}{problem:anticommutingOperatorWithSimulaneousEigenket:1}{\citep{sakurai2014modern} pr. 1.16}{
\index{simultaneous eigenstate}

Two Hermitian operators anticommute
%
\begin{dmath}\label{eqn:anticommutingOperatorWithSimulaneousEigenket:20}
\symmetric{A}{B} = A B + B A = 0.
\end{dmath}

Is it possible to have a simultaneous eigenket of \( A \) and \( B \)?  Prove or illustrate your assertion.
} % problem

\makeanswer{problem:anticommutingOperatorWithSimulaneousEigenket:1}{

Suppose that such a simultaneous non-zero eigenket \( \ket{\alpha} \) exists, then
%
\begin{dmath}\label{eqn:anticommutingOperatorWithSimulaneousEigenket:40}
A \ket{\alpha} = a \ket{\alpha},
\end{dmath}

and
%
\begin{dmath}\label{eqn:anticommutingOperatorWithSimulaneousEigenket:60}
B \ket{\alpha} = b \ket{\alpha}
\end{dmath}

This gives
%
\begin{dmath}\label{eqn:anticommutingOperatorWithSimulaneousEigenket:80}
\lr{ A B + B A } \ket{\alpha}
=
\lr{A b + B a} \ket{\alpha}
= 2 a b \ket{\alpha}.
\end{dmath}

If this is zero, one of the operators must have a zero eigenvalue.  Knowing that we can construct an example of such operators.  In matrix form, let

\begin{subequations}
\label{eqn:anticommutingOperatorWithSimulaneousEigenket:100}
\begin{dmath}\label{eqn:anticommutingOperatorWithSimulaneousEigenket:120}
A =
\begin{bmatrix}
1 & 0 & 0 \\
0 & -1 & 0 \\
0 & 0 & a \\
\end{bmatrix}
\end{dmath}
\begin{dmath}\label{eqn:anticommutingOperatorWithSimulaneousEigenket:140}
B =
\begin{bmatrix}
0 & 1 & 0 \\
1 & 0 & 0 \\
0 & 0 & b \\
\end{bmatrix}.
\end{dmath}
\end{subequations}

These are both Hermitian, and anticommute provided at least one of \( a, b\) is zero.  These have a common eigenket
%
\begin{dmath}\label{eqn:anticommutingOperatorWithSimulaneousEigenket:160}
\ket{\alpha} =
\begin{bmatrix}
0 \\
0 \\
1
\end{bmatrix}.
\end{dmath}

A zero eigenvalue of one of the commuting operators may not be a sufficient condition for such anticommutation.
%It also appears that not all Hermitian matrices that anticommute, where one has a zero eigenvalue, necessarily have a common eigenket.  An example is
%
%\begin{subequations}
%\label{eqn:anticommutingOperatorWithSimulaneousEigenket:180}
%\begin{dmath}\label{eqn:anticommutingOperatorWithSimulaneousEigenket:200}
%A =
%\begin{bmatrix}
%1 & 0 & 0 & 0 \\
%0 & -1 & 0  & 0\\
%0 & 0 & 1  & 0\\
%0 & 0 & 0  & 0\\
%\end{bmatrix}
%\end{dmath}
%\begin{dmath}\label{eqn:anticommutingOperatorWithSimulaneousEigenket:220}
%B =
%\begin{bmatrix}
%0 & 1 & 0 & 0 \\
%1 & 0 & 0 & 0 \\
%0 & 0 & 0 & 0 \\
%0 & 0 & 0 & 1 \\
%\end{bmatrix}.
%\end{dmath}
%\end{subequations}
%
%The eigenkets for the zero eigenvalues for \( A \) and \( B \) are \( (0,0,0,1) \) and \( (0,0,1,0) \) respectively, but neither of these are a common eigenket.
} % answer

%\EndArticle
