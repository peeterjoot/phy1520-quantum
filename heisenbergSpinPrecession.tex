%
% Copyright � 2015 Peeter Joot.  All Rights Reserved.
% Licenced as described in the file LICENSE under the root directory of this GIT repository.
%
%\input{../blogpost.tex}
%\renewcommand{\basename}{heisenbergSpinPrecession}
%\renewcommand{\dirname}{notes/phy1520/}
%%\newcommand{\dateintitle}{}
%%\newcommand{\keywords}{}
%
%\input{../peeter_prologue_print2.tex}
%
%\usepackage{peeters_layout_exercise}
%\usepackage{peeters_braket}
%\usepackage{peeters_figures}
%
%\beginArtNoToc
%
%\generatetitle{Heisenberg picture spin precession}
%\chapter{Heisenberg picture spin precession}
%\label{chap:heisenbergSpinPrecession}

\makeoproblem{Heisenberg picture spin precession.}{problem:heisenbergSpinPrecession:2.1}{\citep{sakurai2014modern} pr. 2.1}{
For the spin Hamiltonian
\index{spin precession}

\begin{dmath}\label{eqn:heisenbergSpinPrecession:20}
H = -\frac{e B}{m c} S_z = \omega S_z,
\end{dmath}

express and solve the Heisenberg equations of motion for \( S_x(t), S_y(t) \), and \( S_z(t) \).
} % problem

\makeanswer{problem:heisenbergSpinPrecession:2.1}{

The equations of motion are of the form

\begin{dmath}\label{eqn:heisenbergSpinPrecession:40}
\frac{dS_i^\txtH}{dt}
= \inv{i \Hbar} \antisymmetric{S_i^\txtH}{H}
= \inv{i \Hbar} \antisymmetric{U^\dagger S_i U}{H}
= \inv{i \Hbar} \lr{U^\dagger S_i U H - H U^\dagger S_i U }
= \inv{i \Hbar} U^\dagger \lr{ S_i H - H S_i } U
= \frac{\omega}{i \Hbar} U^\dagger \antisymmetric{ S_i}{S_z } U.
\end{dmath}

These are

\begin{dmath}\label{eqn:heisenbergSpinPrecession:60}
\begin{aligned}
\frac{dS_x^\txtH}{dt} &= -\omega U^\dagger S_y U \\
\frac{dS_y^\txtH}{dt} &= \omega U^\dagger S_x U \\
\frac{dS_z^\txtH}{dt} &= 0.
\end{aligned}
\end{dmath}

To completely specify these equations, we need to expand \( U(t) \), which is

\begin{dmath}\label{eqn:heisenbergSpinPrecession:80}
U(t)
= e^{-i H t /\Hbar}
= e^{-i \omega S_z t /\Hbar}
= e^{-i \omega \sigma_z t /2}
= \cos\lr{ \omega t/2 } -i \sigma_z \sin\lr{ \omega t/2 }
=
\begin{bmatrix}
\cos\lr{ \omega t/2 } -i \sin\lr{ \omega t/2 } & 0 \\
0 & \cos\lr{ \omega t/2 } + i \sin\lr{ \omega t/2 }
\end{bmatrix}
=
\begin{bmatrix}
e^{-i\omega t/2} & 0 \\
0 & e^{i\omega t/2}
\end{bmatrix}.
\end{dmath}

The equations of motion can now be written out in full.  To do so seems a bit silly since we also know that \( S_x^\txtH = U^\dagger S_x U, S_y^\txtH U^\dagger S_x U \).  However, if that is temporarily forgotten, we can show that the Heisenberg equations of motion can be solved for these too.

\begin{dmath}\label{eqn:heisenbergSpinPrecession:100}
U^\dagger S_x U
=
\frac{\Hbar}{2}
\begin{bmatrix}
e^{i\omega t/2} & 0 \\
0 & e^{-i\omega t/2}
\end{bmatrix}
\PauliX
\begin{bmatrix}
e^{-i\omega t/2} & 0 \\
0 & e^{i\omega t/2}
\end{bmatrix}
=
\frac{\Hbar}{2}
\begin{bmatrix}
0 & e^{i\omega t/2} \\
e^{-i\omega t/2} & 0
\end{bmatrix}
\begin{bmatrix}
e^{-i\omega t/2} & 0 \\
0 & e^{i\omega t/2}
\end{bmatrix}
=
\frac{\Hbar}{2}
\begin{bmatrix}
0 & e^{i\omega t} \\
e^{-i\omega t} & 0
\end{bmatrix},
\end{dmath}

and
\begin{dmath}\label{eqn:heisenbergSpinPrecession:120}
U^\dagger S_y U
=
\frac{\Hbar}{2}
\begin{bmatrix}
e^{i\omega t/2} & 0 \\
0 & e^{-i\omega t/2}
\end{bmatrix}
\PauliY
\begin{bmatrix}
e^{-i\omega t/2} & 0 \\
0 & e^{i\omega t/2}
\end{bmatrix}
=
\frac{i\Hbar}{2}
\begin{bmatrix}
0 & -e^{i\omega t/2} \\
e^{-i\omega t/2} & 0
\end{bmatrix}
\begin{bmatrix}
e^{-i\omega t/2} & 0 \\
0 & e^{i\omega t/2}
\end{bmatrix}
=
\frac{i \Hbar}{2}
\begin{bmatrix}
0 & -e^{i\omega t} \\
e^{-i\omega t} & 0
\end{bmatrix}.
\end{dmath}

The equations of motion are now fully specified

\begin{dmath}\label{eqn:heisenbergSpinPrecession:140}
\begin{aligned}
\frac{dS_x^\txtH}{dt} &=
-\frac{i \Hbar \omega}{2}
\begin{bmatrix}
0 & -e^{i\omega t} \\
e^{-i\omega t} & 0
\end{bmatrix} \\
\frac{dS_y^\txtH}{dt} &=
\frac{\Hbar \omega}{2}
\begin{bmatrix}
0 & e^{i\omega t} \\
e^{-i\omega t} & 0
\end{bmatrix} \\
\frac{dS_z^\txtH}{dt} &= 0.
\end{aligned}
\end{dmath}

Integration gives

\begin{dmath}\label{eqn:heisenbergSpinPrecession:160}
\begin{aligned}
S_x^\txtH &=
\frac{\Hbar}{2}
\begin{bmatrix}
0 & e^{i\omega t} \\
e^{-i\omega t} & 0
\end{bmatrix} + C \\
S_y^\txtH &=
\frac{\Hbar}{2}
\begin{bmatrix}
0 & -i e^{i\omega t} \\
i e^{-i\omega t} & 0
\end{bmatrix} + C \\
S_z^\txtH &= C.
\end{aligned}
\end{dmath}

The integration constants are fixed by the boundary condition \( S_i^\txtH(0) = S_i \), so

\begin{dmath}\label{eqn:heisenbergSpinPrecession:180}
\begin{aligned}
S_x^\txtH &=
\frac{\Hbar}{2}
\begin{bmatrix}
0 & e^{i\omega t} \\
e^{-i\omega t} & 0
\end{bmatrix} \\
S_y^\txtH &=
\frac{i \Hbar}{2}
\begin{bmatrix}
0 & - e^{i\omega t} \\
 e^{-i\omega t} & 0
\end{bmatrix} \\
S_z^\txtH &= S_z.
\end{aligned}
\end{dmath}

Observe that these integrated values \( S_x^\txtH, S_y^\txtH \) match \cref{eqn:heisenbergSpinPrecession:100}, and \cref{eqn:heisenbergSpinPrecession:120} as expected.

} % answer

%\EndArticle
