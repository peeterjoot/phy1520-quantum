%
% Copyright � 2015 Peeter Joot.  All Rights Reserved.
% Licenced as described in the file LICENSE under the root directory of this GIT repository.
%
%\input{../blogpost.tex}
%\renewcommand{\basename}{chapter3Notes}
%\renewcommand{\dirname}{notes/phy1520/}
%\newcommand{\keywords}{PHY1520H}
%\input{../peeter_prologue_print2.tex}
%
%%\usepackage{phy1520}
%\usepackage{peeters_braket}
%\usepackage{peeters_layout_exercise}
%\usepackage{peeters_figures}
%\usepackage{mathtools}
%
%\beginArtNoToc
%\generatetitle{PHY1520H Graduate Quantum Mechanics.  Lecture 3: Density matrix (cont.).  Taught by Prof.\ Arun Paramekanti}
%%\generatetitle{Density matrix (cont.)}
%%\chapter{Density matrix (cont.)}
%\label{chap:chapter3Notes}
%
%\paragraph{Disclaimer}
%
%Peeter's lecture notes from class.  These may be incoherent and rough.
%
%These are notes for the UofT course PHY1520, Graduate Quantum Mechanics, taught by Prof. Paramekanti, covering \textchapref{{1}} \citep{sakurai2014modern} content.
%
%\paragraph{Density matrix (cont.)}

An example of a partitioned system with four total states (two spin 1/2 particles) is sketched in \cref{fig:twoSpins:twoSpinsFig1}.

\imageFigure{../figures/phy1520-quantum/twoSpinsFig1}{Two spins.}{fig:twoSpins:twoSpinsFig1}{0.2}

An example of a partitioned system with eight total states (three spin 1/2 particles) is sketched in \cref{fig:threeSpins:threeSpinsFig2}.

\imageFigure{../figures/phy1520-quantum/threeSpinsFig2}{Three spins.}{fig:threeSpins:threeSpinsFig2}{0.3}

The density matrix
%
\begin{dmath}\label{eqn:qmLecture3:20}
\hat\rho = \ket{\Psi}\bra{\Psi},
\end{dmath}
is clearly an operator as can be seen by applying it to a state
%
\begin{dmath}\label{eqn:qmLecture3:40}
\hat\rho \ket{\phi} = \ket{\Psi} \lr{ \braket{ \Psi }{\phi} }.
\end{dmath}
%
The quantity in braces is just a complex number.

After expanding the pure state \( \ket{\Psi} \) in terms of basis states for each of the two partitions
%
\begin{dmath}\label{eqn:qmLecture3:60}
\ket{\Psi}
= \sum_{m,n} C_{m, n} \ket{m}_\txtL \ket{n}_\txtR.
\end{dmath}
%
With \( \txtL \) and \( \txtR \) implied for \( \ket{m}, \ket{n} \) indexed states respectively, this can be written
%
\begin{dmath}\label{eqn:qmLecture3:460}
\ket{\Psi}
= \sum_{m,n} C_{m, n} \ket{m} \ket{n}.
\end{dmath}
%
The density operator is
%
\begin{dmath}\label{eqn:qmLecture3:80}
\hat\rho =
\sum_{m,n}
C_{m, n}
C_{m', n'}^\conj
\ket{m} \ket{n}
\sum_{m',n'}
\bra{m'} \bra{n'}.
\end{dmath}
%
Suppose we trace over the right partition of the state space, defining such a trace as the reduced density operator \( \hat\rho_{\textrm{red}} \)
%
\begin{dmath}\label{eqn:qmLecture3:100}
\hat\rho_{\textrm{red}}
\equiv
\tr_\txtR(\hat\rho)
= \sum_{\tilde{n}} \bra{\tilde{n}} \hat\rho \ket{ \tilde{n}}
= \sum_{\tilde{n}}
\bra{\tilde{n} }
\lr{
\sum_{m,n}
C_{m, n}
\ket{m} \ket{n}
}
\lr{
\sum_{m',n'}
C_{m', n'}^\conj
\bra{m'} \bra{n'}
}
\ket{ \tilde{n} }
=
\sum_{\tilde{n}}
\sum_{m,n}
\sum_{m',n'}
C_{m, n}
C_{m', n'}^\conj
\ket{m} \delta_{\tilde{n} n}
\bra{m' }
\delta_{ \tilde{n} n' }
=
\sum_{\tilde{n}, m, m'}
C_{m, \tilde{n}}
C_{m', \tilde{n}}^\conj
\ket{m} \bra{m' }
\end{dmath}

Computing the matrix element of \( \hat\rho_{\textrm{red}} \), we have
%
\begin{dmath}\label{eqn:qmLecture3:120}
\bra{\tilde{m}} \hat\rho_{\textrm{red}} \ket{\tilde{m}}
=
\sum_{m, m', \tilde{n}} C_{m, \tilde{n}} C_{m', \tilde{n}}^\conj \braket{ \tilde{m}}{m} \braket{m'}{\tilde{m}}
=
\sum_{\tilde{n}} \Abs{C_{\tilde{m}, \tilde{n}} }^2.
\end{dmath}
%
This is the probability that the left partition is in state \( \tilde{m} \).

\section{Average of an observable.}
\index{density operator!observable expectation}

Suppose we have two spin half particles.  For such a system the total magnetization is
%
\begin{dmath}\label{eqn:qmLecture3:140}
S_{\textrm{Total}} =
S_1^z
+
S_1^z,
\end{dmath}
%
as sketched in \cref{fig:magneticMomentTwoSpins:magneticMomentTwoSpinsFig3}.

\imageFigure{../figures/phy1520-quantum/magneticMomentTwoSpinsFig3}{Magnetic moments from two spins.}{fig:magneticMomentTwoSpins:magneticMomentTwoSpinsFig3}{0.1}

The average of some observable is
%
\begin{dmath}\label{eqn:qmLecture3:160}
\expectation{\hatA}
= \sum_{m, n, m', n'} C_{m, n}^\conj C_{m', n'}
\bra{m}\bra{n} \hatA \ket{n'} \ket{m'}.
\end{dmath}
%
\index{trace}
Consider the trace of the density operator observable product
%
\begin{dmath}\label{eqn:qmLecture3:180}
\tr( \hat\rho \hatA )
= \sum_{m, n} \braket{m n}{\Psi} \bra{\Psi} \hatA \ket{m, n}.
\end{dmath}
%
Let
%
\begin{dmath}\label{eqn:qmLecture3:200}
\ket{\Psi} = \sum_{m, n} C_{m n} \ket{m, n},
\end{dmath}
%
so that
%
\begin{dmath}\label{eqn:qmLecture3:220}
\tr( \hat\rho \hatA )
= \sum_{m, n, m', n', m'', n''} C_{m', n'} C_{m'', n''}^\conj
\braket{m n}{m', n'} \bra{m'', n''} \hatA \ket{m, n}
= \sum_{m, n, m'', n''} C_{m, n} C_{m'', n''}^\conj
\bra{m'', n''} \hatA \ket{m, n}.
\end{dmath}
%
This is just
%
%\begin{dmath}\label{eqn:qmLecture3:240}
\boxedEquation{eqn:qmLecture3:240}{
\bra{\Psi} \hatA \ket{\Psi} = \tr( \hat\rho \hatA ).
}
%\end{dmath}
%
\section{Left observables.}
\index{density operator!left observable}

Consider
%
\begin{dmath}\label{eqn:qmLecture3:260}
\bra{\Psi} \hatA_\txtL \ket{\Psi}
= \tr(\hat\rho \hatA_\txtL)
=
\tr_\txtL
\tr_\txtR
(\hat\rho \hatA_\txtL)
=
\tr_\txtL
\lr{
\lr{
\tr_\txtR \hat\rho
}
\hatA_\txtL)
}
=
\tr_\txtL
\lr{
\hat\rho_{\textrm{red}}
\hatA_\txtL)
}.
\end{dmath}
%
We see
%
\begin{dmath}\label{eqn:qmLecture3:280}
\bra{\Psi} \hatA_\txtL \ket{\Psi}
=
\tr_\txtL \lr{ \hat\rho_{\textrm{red}, \txtL} \hatA_\txtL }.
\end{dmath}
%
We find that we don't need to know the state of the complete system to answer questions about portions of the system, but instead just need \( \hat\rho \), a ``probability operator'' that provides all the required information about the partitioning of the system.

\section{Pure states vs. mixed states.}

For pure states we can assign a state vector and talk about reduced scenarios.  For mixed states we must work with reduced density matrices.

\index{spin half!two particles}
\makeexample{Two particle spin half pure states}{example:qmLecture3:1}{

Consider
%
\begin{dmath}\label{eqn:qmLecture3:300}
\ket{\psi_1} = \inv{\sqrt{2}} \lr{ \ket{ \uparrow \downarrow } - \ket{ \downarrow \uparrow } }
\end{dmath}
%
\begin{dmath}\label{eqn:qmLecture3:320}
\ket{\psi_2} = \inv{\sqrt{2}} \lr{ \ket{ \uparrow \downarrow } + \ket{ \uparrow \uparrow } }.
\end{dmath}
%
For the first pure state the density operator is
\begin{dmath}\label{eqn:qmLecture3:360}
\hat\rho = \inv{2}
\lr{ \ket{ \uparrow \downarrow } - \ket{ \downarrow \uparrow } }
\lr{ \bra{ \uparrow \downarrow } - \bra{ \downarrow \uparrow } }.
\end{dmath}

What are the reduced density matrices?
\index{reduced density!operator}
%
\begin{dmath}\label{eqn:qmLecture3:340}
\hat\rho_\txtL
= \tr_\txtR \lr{ \hat\rho }
=
\inv{2} (-1)(-1) \ket{\downarrow}\bra{\downarrow}
+\inv{2} (+1)(+1) \ket{\uparrow}\bra{\uparrow},
\end{dmath}
%
so the matrix representation of this reduced density operator is
%
\begin{dmath}\label{eqn:qmLecture3:380}
\hat\rho_\txtL
=
\inv{2}
\begin{bmatrix}
1 & 0 \\
0 & 1
\end{bmatrix}.
\end{dmath}
%
For the second pure state the density operator is
\begin{dmath}\label{eqn:qmLecture3:400}
\hat\rho = \inv{2}
\lr{ \ket{ \uparrow \downarrow } + \ket{ \uparrow \uparrow } }
\lr{ \bra{ \uparrow \downarrow } + \bra{ \uparrow \uparrow } }.
\end{dmath}
%
This has a reduced density matrix
\begin{dmath}\label{eqn:qmLecture3:420}
\hat\rho_\txtL
= \tr_\txtR \lr{ \hat\rho }
=
\inv{2} \ket{\uparrow}\bra{\uparrow}
+\inv{2} \ket{\uparrow}\bra{\uparrow}
=
\ket{\uparrow}\bra{\uparrow}.
\end{dmath}
%
This has a matrix representation
\begin{dmath}\label{eqn:qmLecture3:440}
\hat\rho_\txtL
=
\begin{bmatrix}
1 & 0 \\
0 & 0
\end{bmatrix}.
\end{dmath}
\index{entanglement entropy}
In this second example, we have more information about the left partition.  That will be seen as a zero entanglement entropy in the problem set.  In contrast we have less information about the first state, and will find a non-zero positive entanglement entropy in that case.
} % example

%\EndArticle
