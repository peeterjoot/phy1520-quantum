%
% Copyright � 2015 Peeter Joot.  All Rights Reserved.
% Licenced as described in the file LICENSE under the root directory of this GIT repository.
%
\section{Angular momentum}

In classical mechanics the (orbital) angular momentum is
\index{orbital angular momentum}
\index{angular momentum operator}
%
\begin{dmath}\label{eqn:qmLecture13:880}
\BL = \Br \cross \Bp.
\end{dmath}
%
Here ``orbital'' is to distinguish from spin angular momentum.

In quantum mechanics, the mapping to operators, in component form, is
%
\begin{dmath}\label{eqn:qmLecture13:900}
\hatL_i = \epsilon_{ijk} \hatr_j \hatp_k.
\end{dmath}
%
These operators do not commute
\begin{dmath}\label{eqn:qmLecture13:920}
\antisymmetric{\hatL_i}{\hatL_j}
%= \epsilon_{i m n} \epsilon_{ijk}
%\antisymmetric{\hatr_m \hatp_n}{\hatr_k \hatp_l}
%=
%\delta^{[mn]}_{jk}
%\lr{
%\hatr_m \hatp_n \hatr_k \hatp_l
%-
%\hatr_k \hatp_l
%\hatr_m \hatp_n
%}
%=
%\lr{
%\hatr_j \hatp_k \hatr_k \hatp_l
%-
%\hatr_k \hatp_j \hatr_k \hatp_l
%-
%\hatr_k \hatp_l
%\hatr_j \hatp_k
%+
%\hatr_k \hatp_l
%\hatr_k \hatp_j
%}
=
i \Hbar \epsilon_{ijk} \hatL_k.
\end{dmath}
%
%FIXME: fill in the details.

\index{vector operator}
which means that we can't simultaneously determine \( \hatL_i \) for all \( i \).

Aside: In quantum mechanics, we define an operator \( \Vcap \) to be a vector operator if
%
\begin{dmath}\label{eqn:qmLecture13:940}
\antisymmetric{\hatL_i}{\hatV_j}
=
i \Hbar \epsilon_{ijk} \hatV_k.
\end{dmath}
%
The commutator of the squared angular momentum operator with any \( \hatL_i \), say \( \hatL_x \) is zero
%
\begin{dmath}\label{eqn:qmLecture13:960}
\begin{aligned}
\antisymmetric{
\hatL_x^2 +
\hatL_y^2 +
\hatL_z^2
}
{\hatL_x}
&=
\hatL_y \hatL_y \hatL_x
- \hatL_x \hatL_y \hatL_y
+
\hatL_z \hatL_z \hatL_x
- \hatL_x \hatL_z \hatL_z \\
&=
\hatL_y \lr{ \antisymmetric{\hatL_y}{\hatL_x} + \cancel{\hatL_x \hatL_y} }
-\lr{ \antisymmetric{\hatL_x}{\hatL_y} + \cancel{\hatL_y \hatL_x} } \hatL_y \\
&\quad +\hatL_z \lr{ \antisymmetric{\hatL_z}{\hatL_x} + \cancel{\hatL_x \hatL_z} }
-\lr{ \antisymmetric{\hatL_x}{\hatL_z} + \cancel{\hatL_z \hatL_x} } \hatL_z \\
&=
\hatL_y \antisymmetric{\hatL_y}{\hatL_x}
-\antisymmetric{\hatL_x}{\hatL_y} \hatL_y
+\hatL_z \antisymmetric{\hatL_z}{\hatL_x}
-\antisymmetric{\hatL_x}{\hatL_z} \hatL_z \\
&=
i \Hbar \lr{
-\hatL_y \hatL_z
- \hatL_z \hatL_y
+\hatL_z \hatL_y
+ \hatL_y \hatL_z
} \\
&=
0.
\end{aligned}
\end{dmath}
%
%In fact
%\begin{dmath}\label{eqn:qmLecture13:980}
%\antisymmetric{\Lcap^2 }{\hatL_i} = 0.
%\end{dmath}
%
Suppose we have a state \( \ket{\Psi} \) with a well defined \( \hatL_z \) eigenvalue and well defined \( \hat{\BL^2} \) eigenvalue, written as
%
\begin{dmath}\label{eqn:qmLecture13:1000}
\ket{\Psi} = \ket{a, b},
\end{dmath}
%
where the label \( a \) is used for the eigenvalue of \( \Lcap^2 \) and \( b \) labels the eigenvalue of \( \hatL_z \).  Then
%
\begin{equation}\label{eqn:qmLecture13:1020}
\begin{aligned}
\Lcap^2 \ket{a , b} &= \Hbar^2 a \ket{a ,b} \\
\hatL_z \ket{a , b} &= \Hbar b \ket{a ,b}.
\end{aligned}
\end{equation}
%
Things aren't so nice when we act with other angular momentum operators, producing a scrambled mess
%
\begin{equation}\label{eqn:qmLecture13:1040}
\begin{aligned}
\hatL_x \ket{a , b} &= \sum_{a', b'} \calA^x_{a, b, a', b'} \ket{a', b'} \\
\hatL_y \ket{a , b} &= \sum_{a', b'} \calA^y_{a, b, a', b'} \ket{a', b'} \\
\end{aligned}
\end{equation}

With this representation, we have
%
\begin{dmath}\label{eqn:qmLecture13:1060}
\hatL_x \Lcap^2 \ket{a, b}
=
\hatL_x \Hbar^2 a
\sum_{a', b'} \calA^x_{a, b, a', b'} \ket{a', b'}.
\end{dmath}
%
\begin{dmath}\label{eqn:qmLecture13:1080}
\Lcap^2 \hatL_x \ket{a, b}
=
\Hbar^2
\sum_{a', b'} a' \calA^x_{a, b, a', b'} \ket{a', b'}.
\end{dmath}
%
Since \( \Lcap^2, \hatL_x \) commute, we must have
%
\begin{dmath}\label{eqn:qmLecture13:1100}
\calA^x_{a, b, a', b'} = \delta_{a, a'} \calA^x_{a'; b, b'},
\end{dmath}
%
and similarly
\begin{dmath}\label{eqn:qmLecture13:1120}
\calA^y_{a, b, a', b'} = \delta_{a, a'} \calA^y_{a'; b, b'}.
\end{dmath}
%
Simplifying things we can write the action of \( \hatL_x, \hatL_y \) on the state as
%
\begin{dmath}\label{eqn:qmLecture13:1140}
\begin{aligned}
\hatL_x \ket{a , b} &= \sum_{ b'} \calA^x_{a; b, b'} \ket{a, b'} \\
\hatL_y \ket{a , b} &= \sum_{ b'} \calA^y_{a; b, b'} \ket{a, b'} \\
\end{aligned}
\end{dmath}

Let's define
\begin{dmath}\label{eqn:qmLecture13:1160}
\begin{aligned}
\hatL_{+} &\equiv \hatL_x + i \hatL_y \\
\hatL_{-} &\equiv \hatL_x - i \hatL_y \\
\end{aligned}
\end{dmath}

Because these are sums of \( \hatL_x, \hatL_y \) they must also commute with \( \Lcap^2 \)
%
\begin{dmath}\label{eqn:qmLecture13:1180}
\antisymmetric{\Lcap^2}{\hatL_{\pm}} = 0.
\end{dmath}
%
The commutators with \( \hatL_z \) are non-zero
%
\begin{dmath}\label{eqn:qmLecture13:1740}
\antisymmetric{\hatL_z}{\hatL_{\pm}}
=
\hatL_z \lr{ \hatL_x \pm i \hatL_y }
- \lr{ \hatL_x \pm i \hatL_y } \hatL_z
=
\antisymmetric{\hatL_z}{\hatL_x}
\pm i
\antisymmetric{\hatL_z}{\hatL_y}
=
i \Hbar \lr{
\hatL_y \mp i \hatL_x
}
=
\Hbar \lr{ i \hatL_y \pm \hatL_x }
=
\pm \Hbar \lr{ \hatL_x \pm i \hatL_y }
=
\pm \Hbar \hatL_{\pm}.
\end{dmath}
%
%\begin{dmath}\label{eqn:qmLecture13:1200}
%\antisymmetric{\hatL_z}{\hatL_{\pm}} = \pm \Hbar \hatL_{\pm}.
%\end{dmath}
%
Explicitly, that is
%
\begin{dmath}\label{eqn:qmLecture13:1220}
\begin{aligned}
\hatL_z \hatL_{+} - \hatL_{+} \hatL_z &= \Hbar \hatL_{+} \\
\hatL_z \hatL_{-} - \hatL_{-} \hatL_z &= -\Hbar \hatL_{-}
\end{aligned}
\end{dmath}

Now we are set to compute actions of these (assumed) raising and lowering operators on the eigenstate of \( \hatL_z, \Lcap^2 \)
%
\begin{dmath}\label{eqn:qmLecture13:1240}
\hatL_z \hatL_{\pm} \ket{a, b}
=
\Hbar \hatL_{\pm} \ket{a,b} \pm \hatL_{\pm} \hatL_z \ket{a,b}
=
\Hbar \hatL_{\pm} \ket{a,b} \pm \Hbar b \hatL_{\pm} \ket{a,b}
=
\Hbar \lr{ b \pm 1 } \hatL_{\pm} \ket{a, b} .
\end{dmath}
%
There must be a proportionality of the form
%
\begin{dmath}\label{eqn:qmLecture13:1260}
\ket{\hatL_{\pm}} \propto \ket{a, b \pm 1},
\end{dmath}
%
The products of the raising and lowering operators are
%
\begin{dmath}\label{eqn:qmLecture13:1280}
\hatL_{-} \hatL_{+}
=
\lr{ \hatL_x - i \hatL_y }
\lr{ \hatL_x + i \hatL_y }
=
\hatL_x^2 + \hatL_y^2 + i \hatL_x \hatL_y - i \hatL_y \hatL_x
=
\lr{ \Lcap^2 - \hatL_z^2 } + i \antisymmetric{\hatL_x}{\hatL_y}
=
\Lcap^2 - \hatL_z^2 - \Hbar \hatL_z,
\end{dmath}
%
and
\begin{dmath}\label{eqn:qmLecture13:1300}
\hatL_{+} \hatL_{-}
=
\lr{ \hatL_x + i \hatL_y }
\lr{ \hatL_x - i \hatL_y }
=
\hatL_x^2 + \hatL_y^2 - i \hatL_x \hatL_y + i \hatL_y \hatL_x
=
\lr{ \Lcap^2 - \hatL_z^2 } - i \antisymmetric{\hatL_x}{\hatL_y}
=
\Lcap^2 - \hatL_z^2 + \Hbar \hatL_z,
\end{dmath}
%
So we must have
%
\begin{equation}\label{eqn:qmLecture13:1320}
0
\le \bra{a, b} \hatL_{-} \hatL_{+} \ket{a, b}
=
\bra{a, b}
\lr{ \Lcap^2 - \hatL_z^2 - \Hbar \hatL_z }
\ket{a, b}
=
\Hbar^2 a - \Hbar^2 b^2 - \Hbar^2 b,
\end{equation}
%
and
%
\begin{equation}\label{eqn:qmLecture13:1340}
0
\le \bra{a, b} \hatL_{+} \hatL_{-} \ket{a, b}
=
\bra{a, b}
\lr{ \Lcap^2 - \hatL_z^2 + \Hbar \hatL_z }
\ket{a, b}
=
\Hbar^2 a - \Hbar^2 b^2 + \Hbar^2 b.
\end{equation}
%
This puts constraints on \( a, b \), roughly of the form

\begin{enumerate}
\item
\begin{equation}\label{eqn:qmLecture13:1360}
a - b( b + 1) \ge 0
\end{equation}

With \( b_{\textrm{max}} > 0 \), \( b_{\textrm{max}} \approx \sqrt{a} \).

\item
\begin{equation}\label{eqn:qmLecture13:1380}
a - b( b - 1) \ge 0
\end{equation}

With \( b_{\textrm{min}} < 0 \), \( b_{\textrm{min}} \approx -\sqrt{a} \).

\end{enumerate}

%\paragraph{I Love and Desire Sofia Always!}

