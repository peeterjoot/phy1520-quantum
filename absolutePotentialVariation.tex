%
% Copyright � 2015 Peeter Joot.  All Rights Reserved.
% Licenced as described in the file LICENSE under the root directory of this GIT repository.
%
%\input{../blogpost.tex}
%\renewcommand{\basename}{absolutePotentialVariation}
%\renewcommand{\dirname}{notes/phy1520/}
%%\newcommand{\dateintitle}{}
%%\newcommand{\keywords}{}
%
%\input{../peeter_prologue_print2.tex}
%
%\usepackage{peeters_layout_exercise}
%\usepackage{peeters_braket}
%\usepackage{peeters_figures}
%
%\beginArtNoToc
%
%\generatetitle{Energy estimate for an absolute value potential}
%%\chapter{Energy estimate for an absolute value potential}
%%\label{chap:absolutePotentialVariation}
%
%Here's a simple problem, a lot like the problem set 6 variational calculation.

\makeoproblem{Energy estimate for an absolute value potential.}{problem:absolutePotentialVariation:1}{\citep{sakurai2014modern} pr. 5.21}{
\index{variational method}

Estimate the lowest eigenvalue \( \lambda \) of the differential equation
%
\begin{dmath}\label{eqn:absolutePotentialVariation:20}
\frac{d^2}{dx^2}\psi + \lr{ \lambda - \Abs{x} } \psi = 0.
\end{dmath}

Using \( \alpha \) variation with the trial function
%
\begin{dmath}\label{eqn:absolutePotentialVariation:40}
\psi =
\left\{
\begin{array}{l l}
c(\alpha - \Abs{x}) & \quad \mbox{\(\Abs{x} < \alpha \) } \\
0 & \quad \mbox{\(\Abs{x} > \alpha \) }
\end{array}
\right.
\end{dmath}

} % problem

\makeanswer{problem:absolutePotentialVariation:1}{
First rewrite the differential equation in a Hamiltonian like fashion
%
\begin{equation}\label{eqn:absolutePotentialVariation:60}
H \psi = -\frac{d^2}{dx^2}\psi + \Abs{x} \psi = \lambda \psi.
\end{equation}

We need the derivatives of the trial distribution.  The first derivative is
%
\begin{dmath}\label{eqn:absolutePotentialVariation:80}
\frac{d}{dx} \psi
=
-c \frac{d}{dx} \Abs{x}
=
-c \frac{d}{dx} \lr{ x \theta(x) - x \theta(-x) }
=
-c \lr{
 \theta(x) - \theta(-x)
+
x \delta(x) + x \delta(-x)
}
=
-c \lr{
 \theta(x) - \theta(-x)
+
2 x \delta(x)
}.
\end{dmath}

The second derivative is
\begin{dmath}\label{eqn:absolutePotentialVariation:100}
\frac{d^2}{dx^2} \psi
=
-c \frac{d}{dx} \lr{
 \theta(x) - \theta(-x)
+
2 x \delta(x)
}
=
-c \lr{
 \delta(x) + \delta(-x)
+
2 \delta(x)
+
2 x \delta'(x)
}
=
-c \lr{
4 \delta(x)
+
2 x \frac{-\delta(x) }{x}
}
=
-2 c \delta(x).
\end{dmath}

This gives
%
\begin{dmath}\label{eqn:absolutePotentialVariation:120}
H \psi = -2 c \delta(x) + \Abs{x} c \lr{ \alpha - \Abs{x} }.
\end{dmath}

We are now set to compute some of the inner products.  The normalization is the simplest
%
\begin{dmath}\label{eqn:absolutePotentialVariation:140}
\begin{aligned}
\braket{\psi}{\psi}
&= c^2 \int_{-\alpha}^\alpha ( \alpha - \Abs{x} )^2 dx \\
&= 2 c^2 \int_{0}^\alpha ( x - \alpha )^2 dx \\
&= 2 c^2 \int_{-\alpha}^0 u^2 du \\
&= 2 c^2 \lr{ -\frac{(-\alpha)^3}{3} } \\
&= \frac{2}{3} c^2 \alpha^3.
\end{aligned}
\end{dmath}

For the energy
\begin{dmath}\label{eqn:absolutePotentialVariation:160}
\begin{aligned}
\braket{\psi}{H \psi}
&=
c^2 \int dx \lr{ \alpha - \Abs{x} } \lr{ -2 \delta(x) + \Abs{x} \lr{ \alpha - \Abs{x} } } \\
&=
c^2 \lr{ - 2 \alpha + \int_{-\alpha}^\alpha dx \lr{ \alpha - \Abs{x} }^2 \Abs{x} } \\
&=
c^2 \lr{ - 2 \alpha + 2 \int_{-\alpha}^0 du u^2 \lr{ u + \alpha } } \\
&=
c^2 \lr{ - 2 \alpha + 2 \evalrange{\lr{ \frac{u^4}{4} + \alpha \frac{u^3}{3} }}{-\alpha}{0} } \\
&=
c^2 \lr{ - 2 \alpha - 2 \lr{ \frac{\alpha^4}{4} - \frac{\alpha^4}{3} } } \\
&=
c^2 \lr{ - 2 \alpha + \inv{6} \alpha^4 }.
\end{aligned}
\end{dmath}

The energy estimate is
%
\begin{dmath}\label{eqn:absolutePotentialVariation:180}
\overbar{E}
=
\frac{\braket{\psi}{H \psi}}{\braket{\psi}{\psi}}
=
\frac{ - 2 \alpha + \inv{6} \alpha^4 }{ \frac{2}{3} \alpha^3}
=
- \frac{3}{\alpha^2} + \inv{4} \alpha.
\end{dmath}

This has its minimum at
\begin{dmath}\label{eqn:absolutePotentialVariation:200}
0 = -\frac{6}{\alpha^3} + \inv{4},
\end{dmath}
%
or
\begin{dmath}\label{eqn:absolutePotentialVariation:220}
\alpha = 2 \times 3^{1/3}.
\end{dmath}

Back substitution into the energy gives
%
\begin{dmath}\label{eqn:absolutePotentialVariation:240}
\overbar{E}
=
- \frac{3}{4 \times 3^{2/3}} + \inv{2} 3^{1/3}
= \frac{3^{4/3}}{4}
\approx 1.08.
\end{dmath}

The problem says the exact answer is 1.019, so the variation gets within 6 \%.

} % answer

%\EndArticle
