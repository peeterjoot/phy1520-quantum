%
% Copyright � 2015 Peeter Joot.  All Rights Reserved.
% Licenced as described in the file LICENSE under the root directory of this GIT repository.
%
%{
%\input{../blogpost.tex}
%\renewcommand{\basename}{particleInUniformElectricAndMagneticField}
%\renewcommand{\dirname}{notes/phy1520/}
%%\newcommand{\dateintitle}{}
%%\newcommand{\keywords}{}
%
%\input{../peeter_prologue_print2.tex}
%
%\usepackage{peeters_layout_exercise}
%\usepackage{peeters_braket}
%\usepackage{peeters_figures}
%
%\beginArtNoToc
%
%\generatetitle{Particle in charged uniform electric and magnetic fields}
%%\chapter{Particle in charged uniform electric and magnetic fields}
%%\label{chap:particleInUniformElectricAndMagneticField}

\makeoproblem{Particle in uniform electric and magnetic fields.}{problem:particleInUniformElectricAndMagneticField:1}{2015 midterm p2}{

Find the energy eigenvalues and states for a charged particle moving the in the \( x, y \) plane in a uniform magnetic field \( B \zcap \) and a uniform electric field \( E \ycap \).
} % problem

\makeanswer{problem:particleInUniformElectricAndMagneticField:1}{

The Hamiltonian for such a problem has the form
%
\begin{dmath}\label{eqn:particleInUniformElectricAndMagneticField:20}
H =
\frac{(p_x - q A_x/c)^2}{2m} +
\frac{(p_y - q A_y/c)^2}{2m} + q \phi,
\end{dmath}
%
where \( \BA, \phi \) are the potentials for the electromagnetic field.  Since we don't want a time dependent electric potential, our only choice is
%
\begin{equation}\label{eqn:particleInUniformElectricAndMagneticField:40}
E \ycap = -\spacegrad \phi = -\spacegrad (-E y).
\end{equation}

We have many choices for the magnetic field, but it will have to be of the form \( \BA = B(-a y, b x,0) \), for example \( a = b = 1/2 \).  The Hamiltonian becomes
%
\begin{dmath}\label{eqn:particleInUniformElectricAndMagneticField:60}
H =
\frac{(p_x + q B a y /c)^2}{2m} +
\frac{(p_y - q B b x /c)^2}{2m} - q E y,
\end{dmath}
%
We seek a wave function that makes this separable.  If we try \( \psi = e^{i k y} \phi(x) \) we get
%
\begin{equation}\label{eqn:particleInUniformElectricAndMagneticField:80}
\frac{H \psi}{e^{iky}}
=
\lr{ \frac{(p_x + q B a y /c)^2}{2m} +
\frac{(\Hbar k - q B b x /c)^2}{2m} - q E y
} \phi(x),
\end{equation}
%
and if we try \( \psi = e^{i k y} \phi(x) \) we get
%
\begin{equation}\label{eqn:particleInUniformElectricAndMagneticField:100}
\frac{H \psi}{e^{ikx}}
=
\lr{ \frac{(\Hbar k + q B a y /c)^2}{2m} +
\frac{(p_y - q B b x /c)^2}{2m} - q E y
} \phi(y).
\end{equation}

The latter is separable if we set \( b = 0 \), which requires \( a = 1 \), leaving an eigenvalue equation for \( \phi(y) \)
%
\begin{dmath}\label{eqn:particleInUniformElectricAndMagneticField:120}
H'
=
\frac{(\Hbar k + q B y /c)^2}{2m} +
\frac{p_y^2}{2m}
- q E y
=
\frac{p_y^2}{2m}
- \lr{ q E - \Hbar k q B /m c } y
+ \inv{2 m }\lr{\frac{q B y}{c}}^2
+ \frac{(\Hbar k)^2}{2m}
=
\frac{p_y^2}{2m}
+ \inv{2m} \lr{ \frac{q B}{c} }^2
\lr{
- \frac{2}{m} \lr{ \frac{m c}{q B} }^2 \lr{ q E - \Hbar k q B /m c } y
+ y^2
}
+ \frac{(\Hbar k)^2}{2m}
=
\frac{p_y^2}{2m}
+
\inv{2} m
\lr{ \frac{q B}{m c} }^2
\lr{
y - \inv{m} \lr{ \frac{m c}{q B} }^2 \lr{ q E - \Hbar k q B /m c }
}^2
-
\inv{2 m} \lr{ \frac{m c}{q B} }^2 \lr{ q E - \Hbar k q B /m c }^2
+ \frac{(\Hbar k)^2}{2m}.
\end{dmath}

Let
\begin{equation}\label{eqn:particleInUniformElectricAndMagneticField:140}
\begin{aligned}
\omega &= \frac{q B}{m c} \\
y_0 &= \frac{1}{m \omega^2} \lr{ q E - \Hbar k \omega } \\
E_0 &= \frac{(\Hbar k)^2}{2m} - \inv{2} m \omega^2 y_0^2,
\end{aligned}
\end{equation}

leaving
%
\begin{equation}\label{eqn:particleInUniformElectricAndMagneticField:160}
H' = \frac{p_y^2}{2m} + \inv{2} m\omega^2 (y -y_0)^2 + E_0.
\end{equation}

The energy eigenvalues are therefore
%
\begin{equation}\label{eqn:particleInUniformElectricAndMagneticField:180}
E = \Hbar \omega \lr{ n + \inv{2} } + E_0,
\end{equation}
%
and the eigenfunctions are
%
\begin{equation}\label{eqn:particleInUniformElectricAndMagneticField:200}
\psi = e^{i k x} \phi_n(y - y_0),
\end{equation}
%
where \( \phi_n(y) \) is the n-th Harmonic oscillator wavefunction.
} % answer

%}
%\EndNoBibArticle
