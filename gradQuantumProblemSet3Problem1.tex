%
% Copyright � 2015 Peeter Joot.  All Rights Reserved.
% Licenced as described in the file LICENSE under the root directory of this GIT repository.
%
\makeoproblem{Aharonov Bohm effect.}{gradQuantum:problemSet3:1}{phy1520 2015 ps3.1}
{
\index{Aharonov-Bohm effect}

\makesubproblem{}{gradQuantum:problemSet3:1a}
Consider Young's double slit experiment with electrons, having a monoenergetic source of electrons hitting a double slit with slit spacing \( d \), with the electrons then landing on a screen at a distance \( D \) away from the double slit.
For electrons with energy E, find the de Broglie wavelength \( \lambda \), and hence the spacing between the fringes on the screen.
You can ignore the drop in intensity as the electron beam `spreads' when it travels from the slits to the screen (recall that the slits act as effective point sources), so just take phase changes into account along the travel path.

\makesubproblem{}{gradQuantum:problemSet3:1b}
Next, imagine a thin solenoidal flux \( \Phi \) being placed between the two slits, so that electron paths which encircle the flux once will pick up an Aharonov Bohm phase \( e \Phi/\Hbar c \).
Compute the resulting shift in the interference pattern on the screen.
Show that when the flux \( \Phi \) is increased from \( 0 \rightarrow h c/e \), the interference pattern shifts by exactly one fringe, so the new pattern appears the same as the old.
This is the same flux periodicity we saw in class for the energy levels versus flux for a particle on a ring.

} % makeproblem

\makeanswer{gradQuantum:problemSet3:1}{
\withproblemsetsParagraph{
\makeSubAnswer{}{gradQuantum:problemSet3:1a}

In general, the superposition of two equal amplitude wave packets with the same wavelength can be factored into a phase and amplitude
%
\begin{dmath}\label{eqn:gradQuantumProblemSet3Problem1:20}
e^{i (\omega t - k L_1)} + e^{i (\omega t - k L_1)}
=
e^{i (\omega t - k L_1/2 - k L_2/2)}
\lr{
e^{-i k L_1/2 + i k L_2/2}
+
e^{i k L_1/2 - i k L_2/2}
}
=
2 e^{i(\omega t - k L_1/2 - k L_2/2)} \cos\lr{ k (L_1 - L_2)/2 }.
\end{dmath}
%
Now consider the geometry of this screen configuration as sketched in \cref{fig:ps3p1:ps3p1Fig1}.

\imageFigure{../figures/phy1520-quantum/ps3p1Fig1}{Double slit interference.}{fig:ps3p1:ps3p1Fig1}{0.2}

The upper and lower path lengths are
%
\begin{dmath}\label{eqn:gradQuantumProblemSet3Problem1:40}
L_{1,2}
= \sqrt{ D^2 + (y \mp d/2)^2 }
= D \sqrt{ 1 + \frac{(y \mp d/2)^2}{D^2} }
\approx D \lr{ 1 + \inv{2} \frac{(y \mp d/2)^2}{D^2} }
= D + \inv{2 D} (y \mp d/2)^2.
\end{dmath}
%
To first order the length difference is
%
\begin{dmath}\label{eqn:gradQuantumProblemSet3Problem1:60}
L_1 - L_2
=
\inv{2 D} (y - d/2)^2
-\inv{2 D} (y + d/2)^2
=
-\frac{y d}{D}.
\end{dmath}
%
The amplitude of the interference pattern, at height \( y \) on the screen is gated by the cosine
%
\begin{dmath}\label{eqn:gradQuantumProblemSet3Problem1:80}
\cos\lr{ \frac{k y d}{ 2 D} }.
\end{dmath}
%
This has peaks and zeros separated by
%
\begin{dmath}\label{eqn:gradQuantumProblemSet3Problem1:100}
\frac{k \Delta y d}{ 2 D} = \pi,
\end{dmath}
%
or
\begin{dmath}\label{eqn:gradQuantumProblemSet3Problem1:120}
\Delta y = \frac{2 \pi D}{k d}.
\end{dmath}
%
The electron wave number ( \( k = 2 \pi/\lambda \) ) is
%
\begin{dmath}\label{eqn:gradQuantumProblemSet3Problem1:140}
k = \frac{\sqrt{2 m E}}{\Hbar},
\end{dmath}
%
so the peak separation is
%
\boxedEquation{eqn:gradQuantumProblemSet3Problem1:160}{
\Delta y
%= \frac{2 \pi D \Hbar}{d \sqrt{2 m E}}
= \frac{D h}{d \sqrt{2 m E}}.
}

\makeSubAnswer{}{gradQuantum:problemSet3:1b}

Suppose the upper electron path has a positive orientation with respect to the vector potential direction, while the lower electron path has a negative orientation.

The sum of the wave packets will have the form
%
\begin{dmath}\label{eqn:gradQuantumProblemSet3Problem1:180}
\begin{aligned}
e^{i \omega t - k L_1 - e \Phi/\Hbar c}
&+
e^{i \omega t - k L_2 + e \Phi/\Hbar c} \\
&=
2
e^{i (\omega t - k L_1/2 - k L_2/2)}
\lr{
e^{-i k L_1/2 + i k L_2/2 - e \Phi/\Hbar c}
+
e^{i k L_1/2 - i k L_2/2 + e \Phi/\Hbar c}
} \\
&=
2 e^{i(\omega t - k L_1/2 - k L_2/2)} \cos\lr{ k (L_1 - L_2)/2 + \frac{e \Phi}{\Hbar c}}.
\end{aligned}
\end{dmath}
%
As \( \Phi \rightarrow c h/e \) the additional phase term approaches
%
\begin{dmath}\label{eqn:gradQuantumProblemSet3Problem1:200}
\frac{h}{\Hbar} = 2 \pi,
\end{dmath}
%
so the entire interference pattern is shifted exactly one full cycle.
}
}
