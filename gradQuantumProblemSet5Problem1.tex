%
% Copyright � 2015 Peeter Joot.  All Rights Reserved.
% Licenced as described in the file LICENSE under the root directory of this GIT repository.
%
\makeoproblem{No Kramers theorem for spin-1.}{gradQuantum:problemSet5:1}{2015 ps5 p1}{
\index{Kramer's theorem}
\index{spin one}
\index{time reversal}
%\makesubproblem{}{gradQuantum:problemSet5:1a}
%
Consider a spin-1 particle. Even with time-reversal invariance, there is no Kramers theorem, so each eigenvalue of
a generic spin-1 Hamiltonian will be non-degenerate. Systematically construct a Hamiltonian which is time-reversal
invariant and obviously also Hermitian to illustrate this point, making clear the logic of your construction (i.e., why you are including terms which you are including).
} % makeproblem
%
\makeanswer{gradQuantum:problemSet5:1}{
\withproblemsetsParagraph{
One representation of the spin-one \( J_z \) operator is
%
\begin{dmath}\label{eqn:gradQuantumProblemSet5Problem1:20}
J_z
=
\Hbar
\begin{bmatrix}
1 & 0 & 0 \\
0 & 0 & 0 \\
0 & 0 & -1 \\
\end{bmatrix}.
\end{dmath}
This operator changes sign under time reversal.  The \( L_z \) operator also changes sign under time reversal, and can be written
%
\begin{equation}\label{eqn:gradQuantumProblemSet5Problem1:40}
L_z = -i \Hbar \PD{\phi}{}.
\end{equation}
A product of these will be time reversal invariant, but not Hermitian.  To make it Hermitian we can scale with an imaginary constant
%
\begin{dmath}\label{eqn:gradQuantumProblemSet5Problem1:60}
H
= -\frac{i}{2 m \rho^2} J_z L_z
= -\frac{i (-i \Hbar) \Hbar}{2m \rho^2}
\begin{bmatrix}
1 & 0 & 0 \\
0 & 0 & 0 \\
0 & 0 & -1 \\
\end{bmatrix} \PD{\phi}{}
=
-\frac{\Hbar^2}{2m \rho^2}
\begin{bmatrix}
\partial_\phi & 0 & 0 \\
0 & 0 & 0 \\
0 & 0 & -\partial_\phi \\
\end{bmatrix}.
\end{dmath}
%
A \( \ifrac{1}{2 m \rho^2} \) factor has been included to ensure the dimensions of this spin-one time-reversal invariant Hamiltonian also has the dimensions of energy.

This Hamiltonian can be solved directly using the trial function \( \psi = e^{-i E t/\Hbar} \tilde{\psi} \), which provides an eigenvalue equation
%
\begin{equation}\label{eqn:gradQuantumProblemSet5Problem1:80}
E \tilde{\psi} = H \tilde{\psi},
\end{equation}
%
leading to solutions that have a hyperbolic form
%
\begin{dmath}\label{eqn:gradQuantumProblemSet5Problem1:100}
\psi \propto
\begin{bmatrix}
C e^{-2 m \rho^2 E \phi/\Hbar^2} \\
0 \\
C' e^{2 m \rho^2 E \phi/\Hbar^2} \\
\end{bmatrix}
e^{-i E t/\Hbar}
.
\end{dmath}
%
Consistent with the no-Kramer's theorem result for a spin-one Hamiltonian, there is a continuum of energy solutions for this Hamiltonian, with nothing to constrain them.
}
}
