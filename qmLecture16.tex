%
% Copyright � 2015 Peeter Joot.  All Rights Reserved.
% Licenced as described in the file LICENSE under the root directory of this GIT repository.
%
%\input{../blogpost.tex}
%\renewcommand{\basename}{qmLecture16}
%\renewcommand{\dirname}{notes/phy1520/}
%\newcommand{\keywords}{PHY1520H}
%\input{../peeter_prologue_print2.tex}
%
%%\usepackage{phy1520}
%\usepackage{peeters_braket}
%%\usepackage{peeters_layout_exercise}
%\usepackage{peeters_figures}
%\usepackage{mathtools}
%\usepackage{enumerate}
%
%\beginArtNoToc
%\generatetitle{PHY1520H Graduate Quantum Mechanics.  Lecture 16: Addition of angular momenta.  Taught by Prof.\ Arun Paramekanti}
%%\chapter{Addition of angular momenta}
%\label{chap:qmLecture16}
%
%\paragraph{Disclaimer}
%
%Peeter's lecture notes from class.  These may be incoherent and rough.
%
%These are notes for the UofT course PHY1520, Graduate Quantum Mechanics, taught by Prof. Paramekanti, covering \textchapref{{3}} \citep{sakurai2014modern} content.
%
% FIXME: merge into previous.
\section{Addition of angular momenta (cont.)}
\index{angular momentum!addition}

\begin{itemize}
\item
For orbital angular momentum

\begin{dmath}\label{eqn:qmLecture16:20}
\begin{aligned}
\Lcap_1 &= \rcap_1 \cross \pcap_1 \\
\Lcap_1 &= \rcap_1 \cross \pcap_1,
\end{aligned}
\end{dmath}

We can show that it is true that

\begin{dmath}\label{eqn:qmLecture16:40}
\antisymmetric{L_{1i} + L_{2i}}{L_{1j} + L_{2j}} =
i \Hbar \epsilon_{i j k} \lr{ L_{1k} + L_{2k} },
\end{dmath}

because the angular momentum of the independent particles commute.  Given this is it fair to consider that the sum

\begin{dmath}\label{eqn:qmLecture16:60}
\Lcap_1 + \Lcap_2
\end{dmath}

is also angular momentum.
\item

Given \( \ket{l_1, m_1} \) and \( \ket{l_2, m_2} \), if a measurement is made of \( \Lcap_1 + \Lcap_2 \), what do we get?

Specifically, what do we get for

\begin{dmath}\label{eqn:qmLecture16:80}
\lr{\Lcap_1 + \Lcap_2}^2,
\end{dmath}

and for
\begin{dmath}\label{eqn:qmLecture16:100}
\lr{\hatL_{1z} + \hatL_{2z}}.
\end{dmath}

For the latter, we get

\begin{dmath}\label{eqn:qmLecture16:120}
\lr{\hatL_{1z} + \hatL_{2z}}\ket{ l_1, m_1 ; l_2, m_2 }
=
\lr{ \Hbar m_1 + \Hbar m_2 } \ket{ l_1, m_1 ; l_2, m_2 }
\end{dmath}

Given
\begin{dmath}\label{eqn:qmLecture16:140}
\hatL_{1z} + \hatL_{2z} = \hatL_z^{\textrm{tot}},
\end{dmath}

we find
\begin{dmath}\label{eqn:qmLecture16:160}
\begin{aligned}
\antisymmetric{\hatL_z^{\textrm{tot}}}{\Lcap_1^2} &= 0 \\
\antisymmetric{\hatL_z^{\textrm{tot}}}{\Lcap_2^2} &= 0 \\
\antisymmetric{\hatL_z^{\textrm{tot}}}{\Lcap_{1z}} &= 0 \\
\antisymmetric{\hatL_z^{\textrm{tot}}}{\Lcap_{1z}} &= 0.
\end{aligned}
\end{dmath}

We also find

\begin{dmath}\label{eqn:qmLecture16:180}
\antisymmetric{(\Lcap_1 + \Lcap_2)^2}{\Lcap_1^2}
=
\antisymmetric{\Lcap_1^2 + \Lcap_2^2 + 2 \Lcap_1 \cdot \Lcap_2}{\Lcap_1^2}
=
0,
\end{dmath}

but for
\begin{dmath}\label{eqn:qmLecture16:200}
\antisymmetric{(\Lcap_1 + \Lcap_2)^2}{\hatL_{1z}}
=
\antisymmetric{\Lcap_1^2 + \Lcap_2^2 + 2 \Lcap_1 \cdot \Lcap_2}{\hatL_{1z}}
=
2 \antisymmetric{\Lcap_1 \cdot \Lcap_2}{\hatL_{1z}}
\ne 0.
\end{dmath}

\end{itemize}

Classically if we have measured \( \Lcap_{1} \) and \( \Lcap_{2} \) then we know the total angular momentum as sketched in \cref{fig:qmLecture16:qmLecture16Fig1}.

% FIXME: merge with previous lecture
\imageFigure{../figures/phy1520-quantum/qmLecture16Fig1}{Classical addition of angular momenta.}{fig:qmLecture16:qmLecture16Fig1}{0.2}

In QM where we don't know all the components of the angular momentum simultaneously, things get fuzzier.  For example, if the \( \hatL_{1z} \) and \( \hatL_{2z} \) components have been measured, we have the angular momentum defined within a conical region as sketched in \cref{fig:qmLecture16:qmLecture16Fig2}.

\imageFigure{../figures/phy1520-quantum/qmLecture16Fig2}{Addition of angular momenta given measured \( \hatL_z \).}{fig:qmLecture16:qmLecture16Fig2}{0.2}

Suppose we know \( \hatL_z^{\textrm{tot}} \) precisely, but have imprecise information about \( \lr{\Lcap^{\textrm{tot}}}^2 \).  Can we determine bounds for this?  Let \( \ket{\psi} = \ket{ l_1, m_2 ; l_2, m_2 } \), so

\begin{dmath}\label{eqn:qmLecture16:220}
\bra{\psi} \lr{ \Lcap_1 + \Lcap_2 }^2 \ket{\psi}
=
\bra{\psi} \Lcap_1^2 \ket{\psi}
+ \bra{\psi} \Lcap_2^2 \ket{\psi}
+ 2 \bra{\psi} \Lcap_1 \cdot \Lcap_2 \ket{\psi}
=
l_1 \lr{ l_1 + 1} \Hbar^2
+ l_2 \lr{ l_2 + 1} \Hbar^2
+ 2
\bra{\psi} \Lcap_1 \cdot \Lcap_2 \ket{\psi}.
\end{dmath}

Using the Cauchy-Schwartz inequality

\begin{dmath}\label{eqn:qmLecture16:240}
\Abs{\braket{\phi}{\psi}}^2 \le
\Abs{\braket{\phi}{\phi}}
\Abs{\braket{\psi}{\psi}},
\end{dmath}

which is the equivalent of the classical relationship
\begin{dmath}\label{eqn:qmLecture16:260}
\lr{ \BA \cdot \BB }^2 \le \BA^2 \BB^2.
\end{dmath}

Applying this to the last term, we have

\begin{dmath}\label{eqn:qmLecture16:280}
%\lr{ \underbrace{\bra{\psi} \Lcap_1}_{\phi} \cdot \underbrace{\Lcap_2 \ket{\psi}}_{\psi} }^2
\lr{ \bra{\psi} \Lcap_1 \cdot \Lcap_2 \ket{\psi} }^2
\le
\bra{ \psi} \Lcap_1 \cdot \Lcap_1 \ket{\psi}
\bra{ \psi} \Lcap_2 \cdot \Lcap_2 \ket{\psi}
=
\Hbar^4
l_1 \lr{ l_1 + 1 }
l_2 \lr{ l_2 + 2 }.
\end{dmath}

Thus for the max we have

\begin{dmath}\label{eqn:qmLecture16:300}
\bra{\psi} \lr{ \Lcap_1 + \Lcap_2 }^2 \ket{\psi}
\le
\Hbar^2 l_1 \lr{ l_1 + 1 }
+\Hbar^2 l_2 \lr{ l_2 + 1 }
+ 2 \Hbar^2 \sqrt{ l_1 \lr{ l_1 + 1 } l_2 \lr{ l_2 + 2 } }
\end{dmath}

and for the min
\begin{dmath}\label{eqn:qmLecture16:360}
\bra{\psi} \lr{ \Lcap_1 + \Lcap_2 }^2 \ket{\psi}
\ge
\Hbar^2 l_1 \lr{ l_1 + 1 }
+\Hbar^2 l_2 \lr{ l_2 + 1 }
-  2 \Hbar^2 \sqrt{ l_1 \lr{ l_1 + 1 } l_2 \lr{ l_2 + 2 } }.
\end{dmath}

To try to pretty up these estimate, starting with the max, note that if we replace a portion of the RHS with something bigger, we are left with a strict less than relationship.

That is

\begin{equation}\label{eqn:qmLecture16:320}
\begin{aligned}
l_1 \lr{ l_1 + 1 } &< \lr{ l_1 + \inv{2} }^2 \\
l_2 \lr{ l_2 + 1 } &< \lr{ l_2 + \inv{2} }^2
\end{aligned}
\end{equation}

That is

\begin{dmath}\label{eqn:qmLecture16:340}
\bra{\psi} \lr{ \Lcap_1 + \Lcap_2 }^2 \ket{\psi}
<
\Hbar^2
\lr{
l_1 \lr{ l_1 + 1 }
+ l_2 \lr{ l_2 + 1 }
+ 2 \lr{ l_1 + \inv{2} } \lr{ l_2 + \inv{2} }
}
=
\Hbar^2
\lr{
l_1^2 + l_2^2 + l_1 + l_2
+ 2 l_1 l_2 + l_1 + l_2 + \inv{2}
}
=
\Hbar^2
\lr{
\lr{ l_1 + l_2 + \inv{2} }
\lr{ l_1 + l_2 + \frac{3}{2} } - \inv{4}
}
\end{dmath}

or
\begin{dmath}\label{eqn:qmLecture16:380}
l_{\textrm{tot}} \lr{ l_{\textrm{tot}} + 1 }
<
\lr{ l_1 + l_2 + \inv{2} }
\lr{ l_1 + l_2 + \frac{3}{2} }
,
\end{dmath}

which, gives

\begin{dmath}\label{eqn:qmLecture16:400}
l_{\textrm{tot}} < l_1 + l_2 + \inv{2}.
\end{dmath}

Finally, given a quantization requirement, that is

\boxedEquation{eqn:qmLecture16:420}{
l_{\textrm{tot}} \le l_1 + l_2.
}

Similarly, for the min, we find

\begin{dmath}\label{eqn:qmLecture16:440}
\bra{\psi} \lr{ \Lcap_1 + \Lcap_2 }^2 \ket{\psi}
>
\Hbar^2
\lr{
l_1 \lr{ l_1 + 1 }
+ l_2 \lr{ l_2 + 1 }
- 2 \lr{ l_1 + \inv{2} } \lr{ l_2 + \inv{2} }
}
=
\Hbar^2
\lr{
l_1^2 + l_2^2 %+ \cancel{l_1 + l_2}
- 2 l_1 l_2
%-\cancel{l_1 - l_2}
- \inv{2}
}
=
\Hbar^2
\lr{
\lr{ l_1 - l_2 - \inv{2}}\lr{ l_1 - l_2 + \inv{2}} - \inv{4}
}.
\end{dmath}

The total angular momentum quantum number must then satisfy

\begin{dmath}\label{eqn:qmLecture16:480}
l_{\textrm{tot}}( l_{\textrm{tot}} + 1 ) >
\lr{ l_1 - l_2 -\inv{2} } \lr{ l_1 - l_2 +\inv{2} } - \inv{4}
\end{dmath}

Is it true that

\begin{dmath}\label{eqn:qmLecture16:500}
l_{\textrm{tot}}( l_{\textrm{tot}} + 1 ) >
\lr{ l_1 - l_2 -\inv{2} } \lr{ l_1 - l_2 +\inv{2} }?
\end{dmath}

This is true when \( l_{\textrm{tot}} > l_1 - l_2 - \inv{2} \), assuming that \( l_1 > l_2 \).  Suppose \( l_{\textrm{tot}} = l_1 - l_2 - \inv{2} \), then

\begin{dmath}\label{eqn:qmLecture16:520}
l_{\textrm{tot}}( l_{\textrm{tot}} + 1 )
= \lr{ l_1 - l_2 -\inv{2} } \lr{ l_1 - l_2 +\inv{2} }
= \lr{ l_1 - l_2 }^2 - \inv{4}.
\end{dmath}

So, is it true that
\begin{dmath}\label{eqn:qmLecture16:540}
\lr{ l_1 - l_2 }^2 - \inv{4} \ge l_1^2 + l_1 + l_2^2 + l_2 - 2
\sqrt{ l_1(l_1 + 1) l_2( l_2 + 1) }?
\end{dmath}

If that is the case we have
\begin{dmath}\label{eqn:qmLecture16:560}
-2 l_1 l_2 - \inv{4} \ge l_1 + l_2
 - 2 \sqrt{ l_1(l_1 + 1) l_2( l_2 + 1) },
\end{dmath}

\begin{dmath}\label{eqn:qmLecture16:580}
 2 \sqrt{ l_2(l_1 + 1) l_1( l_2 + 1) } \ge
l_1 + l_2
+2 l_1 l_2 + \inv{4}
=
l_1(l_2 + 1) + l_2(l_1 + 1) + \inv{4}.
\end{dmath}

This has the structure

\begin{dmath}\label{eqn:qmLecture16:600}
2 \sqrt{ x y } \ge x + y + \inv{4},
\end{dmath}

or
\begin{dmath}\label{eqn:qmLecture16:620}
4 x y \ge (x + y)^2 + \inv{16} + \inv{2}(x + y),
\end{dmath}

or
\begin{dmath}\label{eqn:qmLecture16:700}
0 \ge (x - y)^2 + \inv{16} + \inv{2}(x + y),
\end{dmath}

But since \( x + y \ge 0 \) this inequality is not satisfied when \( l_{\textrm{tot}} = l_1 - l_2 -\inv{2} \).  We can conclude

\begin{equation}\label{eqn:qmLecture16:640}
l_1 - l_2 -\inv{2} < l_{\textrm{tot}} < l_1 + l_2 + \inv{2}.
\end{equation}

Is it true that
\begin{equation}\label{eqn:qmLecture16:660}
l_1 - l_2 \ge l_{\textrm{tot}} \ge l_1 + l_2 ?
\end{equation}

Note that we have two separate Hilbert spaces \( l_1 \otimes l_2 \)
of dimension \( 2 l_1 + 1 \) and \( 2 l_2 + 1 \) respectively.  The total number of states is

\begin{dmath}\label{eqn:qmLecture16:680}
\sum_{l_{\textrm{tot}} = l_1 - l_2}^{l_1 + l_2}
\lr{ 2 l_{\textrm{tot}} + 1 }
=
2 \sum_{n = l_1 - l_2}^{l_1 + l_2} n + \cancel{l_1} + l_2 - (\cancel{l_1} - l_2) + 1
=
2 \inv{2} \lr{ l_1 + l_2 + (l_1 - l_2) } \lr{ l_1 + l_2 - (l_1 - l_2) + 1 }
+
2 l_2 + 1
=
2 l_1 \lr{ 2 l_2 + 1 }
+
2 l_2 + 1
=
(2 l_1 + 1)
(2 l_2 + 1).
\end{dmath}

So the end result is that given \( \ket{l_1, m_1 }, \ket{l_2, m_2} \), with \( l_1 \ge l_2 \), where, in steps of 1,

\boxedEquation{eqn:qmLecture16:460}{
l_1 - l_2 \le l_{\textrm{tot}} \le l_1 + l_2.
}

%\EndArticle
