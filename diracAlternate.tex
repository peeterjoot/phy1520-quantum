%
% Copyright � 2015 Peeter Joot.  All Rights Reserved.
% Licenced as described in the file LICENSE under the root directory of this GIT repository.
%
%\input{../blogpost.tex}
%\renewcommand{\basename}{diracAlternate}
%\renewcommand{\dirname}{notes/phy1520/}
%%\newcommand{\dateintitle}{}
%%\newcommand{\keywords}{}
%
%\input{../peeter_prologue_print2.tex}
%
%\usepackage{peeters_layout_exercise}
%\usepackage{peeters_braket}
%\usepackage{peeters_figures}
%
%\beginArtNoToc
%
%\generatetitle{Alternate Dirac equation representation}
%\chapter{AlternateDiracEquation}
%\label{chap:diracAlternate}
%
\makeoproblem
%{Alternate Dirac equation representation.}
{Dirac equation representations.}
{problem:diracAlternate:1}{2015 midterm pr. 2}{
%
Given an alternate representation of the Dirac equation
%
\begin{dmath}\label{eqn:diracAlternate:20}
H =
\begin{bmatrix}
m c^2 + V_0 & c \hatp \\
c \hatp & - m c^2 + V_0
\end{bmatrix},
\end{dmath}
%
calculate
%
\makesubproblem{}{problem:diracAlternate:1:a}
the constant momentum plane wave solutions,
%
\makesubproblem{}{problem:diracAlternate:1:b}
the constant momentum hyperbolic solutions,
%
\makesubproblem{}{problem:diracAlternate:1:c}
the Heisenberg velocity operator \( \hatv \), and
%
\makesubproblem{}{problem:diracAlternate:1:d}
find the form of the probability density current.
} % problem
%
\makeanswer{problem:diracAlternate:1}{
%
\makeSubAnswer{}{problem:diracAlternate:1:a}
%
The action of the Hamiltonian on
%
\begin{dmath}\label{eqn:diracAlternate:40}
\psi =
e^{i k x - i E t/\Hbar}
\begin{bmatrix}
\psi_1 \\
\psi_2
\end{bmatrix},
\end{dmath}
is
\begin{dmath}\label{eqn:diracAlternate:60}
H \psi
=
\begin{bmatrix}
m c^2 + V_0 & c (-i \Hbar) i k \\
c (-i \Hbar) i k & - m c^2 + V_0
\end{bmatrix}
\begin{bmatrix}
\psi_1 \\
\psi_2
\end{bmatrix}
e^{i k x - i E t/\Hbar}
=
\begin{bmatrix}
m c^2 + V_0 & c \Hbar k \\
c \Hbar k & - m c^2 + V_0
\end{bmatrix}
\psi.
\end{dmath}
%
Writing
%
\begin{dmath}\label{eqn:diracAlternate:80}
H_k
=
\begin{bmatrix}
m c^2 + V_0 & c \Hbar k \\
c \Hbar k & - m c^2 + V_0
\end{bmatrix},
\end{dmath}
the characteristic equation is
%
\begin{dmath}\label{eqn:diracAlternate:100}
0 =
(m c^2 + V_0 - \lambda)
(-m c^2 + V_0 - \lambda)
%(-m c^2 - V_0 + \lambda)
%(m c^2 - V_0 + \lambda)
- (c \Hbar k)^2
=
\lr{ (\lambda - V_0)^2 - (m c^2)^2 } - (c \Hbar k)^2,
\end{dmath}
%
so
%
\begin{equation}\label{eqn:diracAlternate:120}
\lambda = V_0 \pm \epsilon,
\end{equation}
%
where
\begin{equation}\label{eqn:diracAlternate:140}
\epsilon^2 = (m c^2)^2 + (c \Hbar k)^2.
\end{equation}
%
We've got
%
\begin{equation}\label{eqn:diracAlternate:160}
\begin{aligned}
H - ( V_0 + \epsilon )
&=
\begin{bmatrix}
m c^2 - \epsilon & c \Hbar k \\
c \Hbar k & - m c^2 - \epsilon
\end{bmatrix} \\
H - ( V_0 - \epsilon )
&=
\begin{bmatrix}
m c^2 + \epsilon & c \Hbar k \\
c \Hbar k & - m c^2 + \epsilon
\end{bmatrix},
\end{aligned}
\end{equation}

so the eigenkets are
%
\begin{equation}\label{eqn:diracAlternate:180}
\begin{aligned}
\ket{V_0+\epsilon}
&\propto
\begin{bmatrix}
-c \Hbar k \\
m c^2 - \epsilon
\end{bmatrix} \\
\ket{V_0-\epsilon}
&\propto
\begin{bmatrix}
-c \Hbar k \\
m c^2 + \epsilon
\end{bmatrix}.
\end{aligned}
\end{equation}
%
% (c \Hbar k)^2 + (m c^2 - \epsilon)^2 = 2 \epsilon^2 - 2 m c^2 \epsilon = 2 \epsilon ( \epsilon - m c^2)
% (c \Hbar k)^2 + (m c^2 + \epsilon)^2 = 2 \epsilon^2 + 2 m c^2 \epsilon = 2 \epsilon ( \epsilon + m c^2)

Up to an arbitrary phase for each, these are
%
\begin{equation}\label{eqn:diracAlternate:200}
\begin{aligned}
\ket{V_0 + \epsilon}
&=
\inv{\sqrt{ 2 \epsilon ( \epsilon - m c^2) }}
\begin{bmatrix}
c \Hbar k \\
\epsilon -m c^2
\end{bmatrix}, \\
\ket{V_0 - \epsilon}
&=
\inv{\sqrt{ 2 \epsilon ( \epsilon + m c^2) }}
\begin{bmatrix}
-c \Hbar k \\
\epsilon + m c^2
\end{bmatrix}.
\end{aligned}
\end{equation}

We can now write
%
\begin{dmath}\label{eqn:diracAlternate:220}
H_k =
E
\begin{bmatrix}
V_0 + \epsilon & 0 \\
0         & V_0 - \epsilon
\end{bmatrix}
E^{-1},
\end{dmath}
%
where
\begin{equation}\label{eqn:diracAlternate:240}
\begin{aligned}
E &=
\inv{\sqrt{2 \epsilon} }
\begin{bmatrix}
\frac{c \Hbar k}{ \sqrt{ \epsilon - m c^2 } } & -\frac{c \Hbar k}{ \sqrt{ \epsilon + m c^2 } } \\
\sqrt{ \epsilon - m c^2 } & \sqrt{ \epsilon + m c^2 }
\end{bmatrix}, \qquad k > 0 \\
E &=
\inv{\sqrt{2 \epsilon} }
\begin{bmatrix}
-\frac{c \Hbar k}{ \sqrt{ \epsilon - m c^2 } } & -\frac{c \Hbar k}{ \sqrt{ \epsilon + m c^2 } } \\
-\sqrt{ \epsilon - m c^2 } & \sqrt{ \epsilon + m c^2 }
\end{bmatrix}, \qquad k < 0.
\end{aligned}
\end{equation}
%
Here the signs have been adjusted to ensure the transformation matrix has a unit determinant.

Observe that there's redundancy in this matrix since \( \ifrac{c \Hbar \Abs{k}}{ \sqrt{ \epsilon - m c^2 } } = \sqrt{ \epsilon + m c^2 } \), and \( \ifrac{c \Hbar \Abs{k}}{ \sqrt{ \epsilon + m c^2 } } = \sqrt{ \epsilon - m c^2 } \), which allows the transformation matrix to be written in the form of a rotation matrix
%
\begin{equation}\label{eqn:diracAlternate:260}
\begin{aligned}
E &=
\inv{\sqrt{2 \epsilon} }
\begin{bmatrix}
\frac{c \Hbar k}{ \sqrt{ \epsilon - m c^2 } } & -\frac{c \Hbar k}{ \sqrt{ \epsilon + m c^2 } } \\
\frac{c \Hbar k}{ \sqrt{ \epsilon + m c^2 } } & \frac{c \Hbar k}{ \sqrt{ \epsilon - m c^2 } }
\end{bmatrix}, \qquad k > 0 \\
E &=
\inv{\sqrt{2 \epsilon} }
\begin{bmatrix}
-\frac{c \Hbar k}{ \sqrt{ \epsilon - m c^2 } } & -\frac{c \Hbar k}{ \sqrt{ \epsilon + m c^2 } } \\
\frac{c \Hbar k}{ \sqrt{ \epsilon + m c^2 } }  & -\frac{c \Hbar k}{ \sqrt{ \epsilon - m c^2 } }
\end{bmatrix}, \qquad k < 0.
\end{aligned}
\end{equation}
With
%
\begin{equation}\label{eqn:diracAlternate:280}
\begin{aligned}
\cos\theta &= \frac{c \Hbar \Abs{k}}{ \sqrt{ 2 \epsilon( \epsilon - m c^2) } } = \frac{\sqrt{ \epsilon + m c^2} }{ \sqrt{ 2 \epsilon}}\\
\sin\theta &= \frac{c \Hbar k}{ \sqrt{ 2 \epsilon( \epsilon + m c^2) } } = \frac{\sgn(k) \sqrt{ \epsilon - m c^2}}{ \sqrt{ 2 \epsilon } },
\end{aligned}
\end{equation}
the transformation matrix (and eigenkets) is
%\begin{equation}\label{eqn:diracAlternate:300}
\boxedEquation{eqn:diracAlternate:300}{
E =
\begin{bmatrix}
\ket{V_0 + \epsilon} & \ket{V_0 - \epsilon}
\end{bmatrix}
=
\begin{bmatrix}
\cos\theta & -\sin\theta \\
\sin\theta & \cos\theta
\end{bmatrix}.
}
%\end{equation}
%
Observe that \cref{eqn:diracAlternate:280} can be simplified by using double angle formulas
%
\begin{dmath}\label{eqn:diracAlternate:320}
\cos(2 \theta)
=
\frac{\lr{ \epsilon + m c^2} }{ 2 \epsilon }
-
\frac{\lr{ \epsilon - m c^2}}{ 2 \epsilon }
=
\frac{1}{ 2 \epsilon } \lr{ \epsilon + m c^2 - \epsilon + m c^2 }
=
\frac{m c^2 }{ \epsilon },
\end{dmath}
%
and
\begin{dmath}\label{eqn:diracAlternate:340}
\sin(2\theta)
=
2 \frac{1}{2 \epsilon} \sgn(k ) \sqrt{ \epsilon^2 - (m c^2)^2 }
=
\frac{\Hbar k c}{\epsilon}.
\end{dmath}
%
This allows all the \( \theta \) dependence on \( \Hbar k c \) and \( m c^2 \) to be expressed as a ratio of momenta
%
%\begin{dmath}\label{eqn:diracAlternate:360}
\boxedEquation{eqn:diracAlternate:360}{
\tan(2\theta) = \frac{\Hbar k}{m c}.
}
%\end{dmath}
%
\makeSubAnswer{}{problem:diracAlternate:1:b}
For a wave function of the form
%
\begin{dmath}\label{eqn:diracAlternate:380}
\psi =
e^{k x - i E t/\Hbar}
\begin{bmatrix}
\psi_1 \\
\psi_2
\end{bmatrix},
\end{dmath}
%
some of the work above can be recycled if we substitute \( k \rightarrow -i k \), which yields unnormalized eigenfunctions
%
\begin{equation}\label{eqn:diracAlternate:400}
\begin{aligned}
\ket{V_0+\epsilon}
&\propto
\begin{bmatrix}
i c \Hbar k \\
m c^2 - \epsilon
\end{bmatrix} \\
\ket{V_0-\epsilon}
&\propto
\begin{bmatrix}
i c \Hbar k \\
m c^2 + \epsilon
\end{bmatrix},
\end{aligned}
\end{equation}

where
%
\begin{equation}\label{eqn:diracAlternate:420}
\epsilon^2 = (m c^2)^2 - (c \Hbar k)^2.
\end{equation}
%
The squared magnitude of these wavefunctions are
%
\begin{dmath}\label{eqn:diracAlternate:440}
(c \Hbar k)^2 + (m c^2 \mp \epsilon)^2
=
(c \Hbar k)^2 + (m c^2)^2 + \epsilon^2 \mp 2 m c^2 \epsilon
=
(c \Hbar k)^2 + (m c^2)^2 + (m c^2)^2 \mp (c \Hbar k)^2 - 2 m c^2 \epsilon
= 2 (m c^2)^2 \mp 2 m c^2 \epsilon
= 2 m c^2 ( m c^2 \mp \epsilon ),
\end{dmath}
%
so, up to a constant phase for each, the normalized kets are
%
\begin{equation}\label{eqn:diracAlternate:460}
\begin{aligned}
\ket{V_0+\epsilon}
&=
\inv{\sqrt{ 2 m c^2 ( m c^2 - \epsilon ) }}
\begin{bmatrix}
i c \Hbar k \\
m c^2 - \epsilon
\end{bmatrix} \\
\ket{V_0-\epsilon}
&=
\inv{\sqrt{ 2 m c^2 ( m c^2 + \epsilon ) }}
\begin{bmatrix}
i c \Hbar k \\
m c^2 + \epsilon
\end{bmatrix}.
\end{aligned}
\end{equation}
After the \( k \rightarrow -i k \) substitution, \( H_k \) is not Hermitian, so these kets aren't expected to be orthonormal, which is readily verified
%
\begin{equation}\label{eqn:diracAlternate:480}
\begin{aligned}
\braket{V_0+\epsilon}{V_0-\epsilon}
&=
\inv{\sqrt{ 2 m c^2 ( m c^2 - \epsilon ) }}
\inv{\sqrt{ 2 m c^2 ( m c^2 + \epsilon ) }} ,\times \\
&\qquad \begin{bmatrix}
-i c \Hbar k &
m c^2 - \epsilon
\end{bmatrix}
\begin{bmatrix}
i c \Hbar k \\
m c^2 + \epsilon
\end{bmatrix} \\
&=
\frac{ 2 ( c \Hbar k )^2 }{2 m c^2 \sqrt{(\Hbar k c)^2} }  \\
&=
\sgn(k)
\frac{
\Hbar k }{m c } .
\end{aligned}
\end{equation}
%
\makeSubAnswer{}{problem:diracAlternate:1:c}
\begin{dmath}\label{eqn:diracAlternate:500}
\hatv
= \inv{i \Hbar} \antisymmetric{ \hatx }{ H}
= \inv{i \Hbar} \antisymmetric{ \hatx }{ m c^2 \sigma_z + V_0 + c \hatp \sigma_x }
= \frac{c \sigma_x}{i \Hbar} \antisymmetric{ \hatx }{ \hatp }
= c \sigma_x.
\end{dmath}
%
\makeSubAnswer{}{problem:diracAlternate:1:d}
%
Acting against a completely general wavefunction the Hamiltonian action \( H \psi \) is
%
\begin{dmath}\label{eqn:diracAlternate:520}
i \Hbar \PD{t}{\psi}
= m c^2 \sigma_z \psi + V_0 \psi + c \hatp \sigma_x \psi
= m c^2 \sigma_z \psi + V_0 \psi -i \Hbar c \sigma_x \PD{x}{\psi}.
\end{dmath}
%
Conversely, the conjugate \( (H \psi)^\dagger \) is
%
\begin{equation}\label{eqn:diracAlternate:540}
-i \Hbar \PD{t}{\psi^\dagger}
= m c^2 \psi^\dagger \sigma_z + V_0 \psi^\dagger +i \Hbar c \PD{x}{\psi^\dagger} \sigma_x.
\end{equation}
%
These give
%
\begin{equation}\label{eqn:diracAlternate:560}
\begin{aligned}
i \Hbar \psi^\dagger \PD{t}{\psi}
&=
m c^2 \psi^\dagger \sigma_z \psi + V_0 \psi^\dagger \psi -i \Hbar c \psi^\dagger \sigma_x \PD{x}{\psi} \\
-i \Hbar \PD{t}{\psi^\dagger} \psi
&= m c^2 \psi^\dagger \sigma_z \psi + V_0 \psi^\dagger \psi +i \Hbar c \PD{x}{\psi^\dagger} \sigma_x \psi.
\end{aligned}
\end{equation}
%
Taking differences
\begin{dmath}\label{eqn:diracAlternate:580}
\psi^\dagger \PD{t}{\psi} + \PD{t}{\psi^\dagger} \psi
=
- c \psi^\dagger \sigma_x \PD{x}{\psi} - c \PD{x}{\psi^\dagger} \sigma_x \psi,
\end{dmath}
%
or
%
\begin{dmath}\label{eqn:diracAlternate:600}
0
=
\PD{t}{}
\lr{
\psi^\dagger \psi
}
+
\PD{x}{}
\lr{
c \psi^\dagger \sigma_x \psi
}.
\end{dmath}
%
The probability current still has the usual form \( \rho = \psi^\dagger \psi = \psi_1^\conj \psi_1 + \psi_2^\conj \psi_2 \), but the probability current with this representation of the Dirac Hamiltonian is
%
\begin{dmath}\label{eqn:diracAlternate:620}
j
= c \psi^\dagger \sigma_x \psi
= c
\begin{bmatrix}
\psi_1^\conj &
\psi_2^\conj
\end{bmatrix}
\begin{bmatrix}
\psi_2 \\
\psi_1
\end{bmatrix}
= c \lr{ \psi_1^\conj \psi_2 + \psi_2^\conj \psi_1 }.
\end{dmath}
} % answer

%\EndNoBibArticle
