%
% Copyright � 2015 Peeter Joot.  All Rights Reserved.
% Licenced as described in the file LICENSE under the root directory of this GIT repository.
%
%\input{../blogpost.tex}
%\renewcommand{\basename}{lecture2}
%\renewcommand{\dirname}{notes/phy1520/}
%%\newcommand{\dateintitle}{}
%%\newcommand{\keywords}{}
%
%\input{../peeter_prologue_print2.tex}
%
%\usepackage{peeters_layout_exercise}
%\usepackage{peeters_braket}
%\usepackage{peeters_figures}
%\usepackage{enumerate}
%
%\beginArtNoToc
%
%\generatetitle{PHY1520H Graduate Quantum Mechanics.  Lecture 2: Basic concepts, time evolution, and density operators.  Taught by Prof.\ Arun Paramekanti}
%%\label{chap:gmLecture1}
%
%\paragraph{Disclaimer}
%
%Peeter's lecture notes from class.  These may be incoherent and rough.
%
%These are notes for the UofT course PHY1520, Graduate Quantum Mechanics, taught by Prof. Paramekanti, covering \textchapref{{1}} \citep{sakurai2014modern} content.
%%

\section{Review: Basic concepts}

We've reviewed the basic concepts that we will encounter in Quantum Mechanics.

\begin{enumerate}
\item Abstract state vector.  \( \ket{ \psi} \)
\item Basis states.  \( \ket{ x } \)
\item Observables, special Hermitian operators.  We'll only deal with linear observables.
\item Measurement.
\end{enumerate}

We can either express the wave functions \( \psi(x) = \braket{x}{\psi} \) in terms of a basis for the observable, or can express the observable in terms of the basis of the wave function (position or momentum for example).

We saw that the position space representation of a momentum operator (also an observable) was

\begin{dmath}\label{eqn:qmLecture2:20}
\hatp \rightarrow -i \Hbar \PD{x}{}.
\end{dmath}

In general we can find the matrix element representation of any operator by considering its representation in a given basis.  For example, in a position basis, that would be

\begin{dmath}\label{eqn:qmLecture2:40}
\bra{x'} \hatA \ket{x} \leftrightarrow A_{x x'}
\end{dmath}

The Hermitian property of the observable means that \( A_{x x'} = A_{x' x}^\conj \)

\begin{dmath}\label{eqn:qmLecture2:60}
\int dx \bra{x'} \hatA \ket{x} \braket{x }{\psi} = \braket{x'}{\phi}
\leftrightarrow
A_{x' x} \psi_x = \phi_{x'}.
\end{dmath}

\index{measurement}
\makeexample{Measurement example}{example:qmLecture2:1}{

Consider a polarization apparatus as sketched in \cref{fig:polarizerMeasurement:polarizerMeasurementFig1}, where the output is of the form \( I_{\textrm{out}} = I_{\textrm{in}} \cos^2 \theta \).

\imageFigure{../figures/phy1520-quantum/polarizerMeasurementFig1}{Polarizer apparatus.}{fig:polarizerMeasurement:polarizerMeasurementFig1}{0.3}

A general input state can be written in terms of each of the possible polarizations

\begin{dmath}\label{eqn:qmLecture2:80}
\alpha \ket{ \updownarrow } + \beta \ket{ \leftrightarrow } \sim
\cos\theta \ket{ \updownarrow } + \sin\theta \ket{ \leftrightarrow }
\end{dmath}

Here \( \abs{\alpha}^2 \) is the probability that the input state is in the upwards polarization state, and \( \abs{\beta}^2 \) is the probability that the input state is in the downwards polarization state.

The measurement of the polarization results in an output state that has a specific polarization.  That measurement is said to collapse the wavefunction.
} % example

When attempting a measurement, looking for a specific value, effects the state of the system, and is call a strong or projective measurement.  Such a measurement is

\begin{enumerate}[(i)]
\item Probabilistic.
\item Requires many measurements.
\end{enumerate}

This measurement process results a determination of the eigenvalue of the operator.  The eigenvalue production of measurement is why we demand that operators be Hermitian.

It is also possible to try to do a weaker (perturbative) measurement, where some information is extracted from the input state without completely altering it.

\paragraph{Time evolution}

\begin{enumerate}
\item Schr\"{o}dinger picture.
\index{Schr\"{o}dinger picture}
\index{time evolution}
The time evolution process is governed by a Schr\"{o}dinger equation of the following form

\begin{dmath}\label{eqn:qmLecture2:100}
i \Hbar \PD{t}{} \ket{\Psi(t)} = \hatH \ket{\Psi(t)}.
\end{dmath}

This Hamiltonian could be, for example,

\begin{dmath}\label{eqn:qmLecture2:120}
\hatH = \frac{\hatp^2}{2m} + V(x),
\end{dmath}

Such a representation of time evolution is expressed in terms of operators \( \hatx, \hatp, \hatH, \cdots \) that are independent of time.
\item Heisenberg picture.
\index{Heisenberg picture}

Suppose we have a state \( \ket{\Psi(t)} \) and operate on this with an operator

\begin{dmath}\label{eqn:qmLecture2:140}
\hatA \ket{\Psi(t)}.
\end{dmath}

This will have time evolution of the form

\begin{dmath}\label{eqn:qmLecture2:160}
\hatA e^{-i \hatH t/\Hbar} \ket{\Psi(0)},
\end{dmath}

or in matrix element form

\begin{dmath}\label{eqn:qmLecture2:180}
\bra{\phi(t)} \hatA \ket{\Psi(t)}
=
\bra{\phi(0)}
e^{i \hatH t/\Hbar}
\hatA e^{-i \hatH t/\Hbar} \ket{\Psi(0)}.
\end{dmath}

We work with states that do not evolve in time \( \ket{\phi(0)}, \ket{\Psi(0)}, \cdots \), but operators do evolve in time according to

\begin{dmath}\label{eqn:qmLecture2:200}
\hatA(t) =
e^{i \hatH t/\Hbar}
\hatA e^{-i \hatH t/\Hbar}.
\end{dmath}

\end{enumerate}

\paragraph{Density operator}
\index{density operator}

We can have situations where it is impossible to determine a single state that describes the system.  For example, given the gas in the room that you are sitting in, there are things that we can measure, but it is impossible to describe the state that describes all the particles and also impossible to construct a Hamiltonian that governs all the interactions of those many many particles.

We need a probabilistic description to even describe such a complex system, and to be able to deal with concepts like entanglement.

Suppose we have a complex system that can be partitioned into two subsets, left and right, as sketched in \cref{fig:partitions:partitionsFig2}.

\imageFigure{../figures/phy1520-quantum/partitionsFig2}{System partitioned into separate set of states.}{fig:partitions:partitionsFig2}{0.3}

If the states in each partition can be enumerated separately, we can write the state of the system as sums over the probability amplitudes that for the combined states.

\begin{dmath}\label{eqn:qmLecture2:220}
\ket{\Psi}
=
\sum_{m, n} C_{m,n} \ket{m} \ket{n}
%\equiv
%\sum_{m, n} C_{m,n} \ket{m} \directproduct \ket{n}.
\end{dmath}

Here \( C_{m, n} \) is the probability amplitude to find the state in the combined state \( \ket{m} \ket{n} \).

As an example of such a system, we could investigate a two particle configuration where spin up or spin down can be separately measured for each particle.

\begin{dmath}\label{eqn:qmLecture2:240}
\ket{\psi} = \inv{\sqrt{2}} \lr{
\ket{\uparrow}\ket{\downarrow}
+
\ket{\downarrow}\ket{\uparrow}
}
\end{dmath}

Considering such a system we could ask questions such as

\begin{itemize}
\item What is the probability that the left half is in state \( m \)?  This would be

\begin{dmath}\label{eqn:qmLecture2:260}
\sum_n \Abs{C_{m, n}}^2
\end{dmath}

\item Probability that the left half is in state \( m \), and the
probability that the right half is in state \( n \)?  That is

\begin{dmath}\label{eqn:qmLecture2:280}
\Abs{C_{m, n}}^2
\end{dmath}
\end{itemize}

We define the density operator

\begin{dmath}\label{eqn:qmLecture2:300}
\hat\rho = \ket{\Psi} \bra{\Psi}.
\end{dmath}

This is \textAndIndex{idempotent}

\begin{dmath}\label{eqn:qmLecture2:320}
\hat\rho^2 =
\lr{ \ket{\Psi} \bra{\Psi} }
\lr{ \ket{\Psi} \bra{\Psi} }
=
\ket{\Psi} \bra{\Psi}
\end{dmath}

%\EndArticle
