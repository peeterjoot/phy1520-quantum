%
% Copyright � 2015 Peeter Joot.  All Rights Reserved.
% Licenced as described in the file LICENSE under the root directory of this GIT repository.
%
%%%\input{../blogpost.tex}
%%%\renewcommand{\basename}{qmLecture14}
%%%\renewcommand{\dirname}{notes/phy1520/}
%%%\newcommand{\keywords}{PHY1520H}
%%%\input{../peeter_prologue_print2.tex}
%%%
%%%%\usepackage{phy1520}
%%%\usepackage{peeters_braket}
%%%\usepackage{macros_cal}
%%%%\usepackage{peeters_layout_exercise}
%%%\usepackage{peeters_figures}
%%%\usepackage{mathtools}
%%%
%%%\beginArtNoToc
%%%\generatetitle{PHY1520H Graduate Quantum Mechanics.  Lecture 14: Angular momentum (cont.).  Taught by Prof.\ Arun Paramekanti}
%%%%\chapter{Angular momentum (cont.)}
%%%\label{chap:qmLecture14}
%%%
%%%\paragraph{Disclaimer}
%%%
%%%Peeter's lecture notes from class.  These may be incoherent and rough.
%%%
%%%These are notes for the UofT course PHY1520, Graduate Quantum Mechanics, taught by Prof. Paramekanti, covering \textchapref{{3}} \citep{sakurai2014modern} content.
%%%
%%\paragraph{Review: Angular momentum}
%%
%%Given eigenket \( \ket{a, b} \), where
%%
%%\begin{equation}\label{eqn:qmLecture14:20}
%%\begin{aligned}
%%\Lcap^2 \ket{a, b} &= \Hbar^2 a \ket{a,b} \\
%%\hatL_z \ket{a, b} &= \Hbar b \ket{a,b}
%%\end{aligned}
%%\end{equation}
%%
%%We were looking for
%%
%%\begin{dmath}\label{eqn:qmLecture14:40}
%%\hatL_{x,y} \ket{a,b} = \sum_{b'} \calA^{x,y}_{a; b, b'} \ket{a,b'},
%%\end{dmath}
%%
%%by applying
%%
%%\begin{dmath}\label{eqn:qmLecture14:60}
%%\hatL_{\pm} = \hatL_x \pm i \hatL_y.
%%\end{dmath}
%%
%%We found
%%
%%\begin{dmath}\label{eqn:qmLecture14:80}
%%\hatL_{\pm} \propto \ket{a, b \pm 1}.
%%\end{dmath}
%%
%%Let
%%
%%\begin{dmath}\label{eqn:qmLecture14:100}
%%\ket{\phi_\pm} = \hatL_{\pm} \ket{a, b}.
%%\end{dmath}
%%
%%We want
%%
%%\begin{dmath}\label{eqn:qmLecture14:120}
%%\braket{\phi_\pm}{\phi_\pm} \ge 0,
%%\end{dmath}
%%
%%or
%%\begin{dmath}\label{eqn:qmLecture14:140}
%%\begin{aligned}
%%\bra{a,b} \hatL_{+} \hatL_{-} \ket{a, b} &\ge 0 \\
%%\bra{a,b} \hatL_{-} \hatL_{+} \ket{a, b} &\ge 0
%%\end{aligned}
%%\end{dmath}
%%
%%We found
%%
%%\begin{dmath}\label{eqn:qmLecture14:160}
%%\hatL_{+} \hatL_{-} =
%%\lr{ \hatL_x + i \hatL_y } \lr{ \hatL_x - i \hatL_y }
%%= \lr{ \Lcap^2 - \hatL_z^2 } -i \antisymmetric{\hatL_x}{\hatL_y}
%%= \lr{ \Lcap^2 - \hatL_z^2 } -i \lr{ i \Hbar \hatL_z }
%%= \lr{ \Lcap^2 - \hatL_z^2 } + \Hbar \hatL_z,
%%\end{dmath}
%%
%%so
%%
%%\begin{dmath}\label{eqn:qmLecture14:180}
%%\bra{a,b} \hatL_{+} \hatL_{-} \ket{a, b}
%%=
%%\expectation{ \Lcap^2 - \hatL_z^2 + \Hbar \hatL_z }.
%%\end{dmath}
%%
%%Similarly
%%\begin{dmath}\label{eqn:qmLecture14:200}
%%\bra{a,b} \hatL_{-} \hatL_{+} \ket{a, b}
%%=
%%\expectation{ \Lcap^2 - \hatL_z^2 - \Hbar \hatL_z }.
%%\end{dmath}
%%
%%\paragraph{Constraints}
%%
%%\begin{equation}\label{eqn:qmLecture14:220}
%%\begin{aligned}
%%a - b^2 + b &\ge 0 \\
%%a - b^2 - b &\ge 0
%%\end{aligned}
%%\end{equation}

Assuming that the \( b_{\textrm{max}} \) and \( b_{\textrm{min}} \) values are
satisfied at the equality extremes we have
%
\begin{equation}\label{eqn:qmLecture14:240}
\begin{aligned}
b_{\textrm{max}} \lr{ b_{\textrm{max}} + 1 } &= a \\
b_{\textrm{min}} \lr{ b_{\textrm{min}} - 1 } &= a.
\end{aligned}
\end{equation}
%
Equating these pair of equations and rearranging, we have
%
\begin{dmath}\label{eqn:qmLecture14:680}
\lr{ b_{\textrm{max}} + \inv{2} }^2 - \inv{4} = \lr{ b_{\textrm{min}} - \inv{2} }^2 - \inv{4},
\end{dmath}
%
which has solutions at
%
\begin{dmath}\label{eqn:qmLecture14:700}
b_{\textrm{max}} + \inv{2} = \pm \lr{ b_{\textrm{min}} - \inv{2} }.
\end{dmath}
%
One of the solutions is
%
\begin{dmath}\label{eqn:qmLecture14:260}
-b_{\textrm{min}} = b_{\textrm{max}},
\end{dmath}
%
as desired.  The other solution is \( b_{\textrm{max}} = b_{\textrm{min}} - 1 \), which we discard.

The final constraint is therefore
%
%\begin{equation}\label{eqn:qmLecture14:280}
\boxedEquation{eqn:qmLecture14:300}{
- b_{\textrm{max}} \le b \le b_{\textrm{max}},
}
%\end{equation}

and
%
\begin{equation}\label{eqn:qmLecture14:320}
\begin{aligned}
\hatL_{+} \ket{a, b_{\textrm{max}}} &= 0 \\
\hatL_{-} \ket{a, b_{\textrm{min}}} &= 0
\end{aligned}
\end{equation}

If we had the sequence, which must terminate at \( b_{\textrm{min}} \) or else it will go on forever
%
\begin{dmath}\label{eqn:qmLecture14:340}
\ket{a, b_{\textrm{max}}}
\overset{\hatL_{-}}{\rightarrow}
\ket{a, b_{\textrm{max}} - 1}
\overset{\hatL_{-}}{\rightarrow}
\ket{a, b_{\textrm{max}} - 2}
\cdots
\overset{\hatL_{-}}{\rightarrow}
\ket{a, b_{\textrm{min}}},
\end{dmath}
%
then we know that \( b_{\textrm{max}} - b_{\textrm{min}} \in \bbZ \), or
%
\begin{equation}\label{eqn:qmLecture14:360}
b_{\textrm{max}} - n = b_{\textrm{min}} = -b_{\textrm{max}}
\end{equation}

or
%
\begin{equation}\label{eqn:qmLecture14:380}
b_{\textrm{max}} = \frac{n}{2},
\end{equation}
%
this is either an integer or a \( 1/2 \) odd integer, depending on whether \( n \) is even or odd.  These are called ``orbital'' or ``spin'' respectively.

The convention is to write \( m \) for the \( \hatL_z \) eigenvalue (the magnetic quantum number), and \( j \), the azimthual quantum number, to describe the \( j(j+1) \) eigenvalue of the \( \Lcap^2 \) operator.
%
\begin{equation}\label{eqn:qmLecture14:400}
\begin{aligned}
b_{\textrm{max}} &= j \\
a &= j(j + 1).
\end{aligned}
\end{equation}
\index{angular momentum!magnetic quantum number}
\index{angular momentum!azimthual quantum number}

so for \( m \in -j, -j + 1, \cdots, +j \)
%
%\begin{equation}\label{eqn:qmLecture14:420}
\boxedEquation{eqn:qmLecture14:440}{
\begin{aligned}
\Lcap^2 \ket{j, m} &= \Hbar^2 j (j + 1) \ket{j, m} \\
L_z \ket{j, m} &= \Hbar m \ket{j, m}.
\end{aligned}
}
%\end{equation}

\section{Schwinger's SHO angular momentum operators.}
\index{angular momentum!Schwinger}

In \citep{schwinger1955quantum} a powerful method for describing angular momentum with harmonic oscillators was introduced, which will be outlined here.  The question is whether we can construct a set of harmonic oscillators that allows a mapping from
%
\begin{dmath}\label{eqn:qmLecture14:460}
\hatL_{+} \leftrightarrow a^{+}?
\end{dmath}

Picture two harmonic oscillators, one with states counted from one zero towards \( \infty \) and another with states counted from a different zero towards \( -\infty \), as pictured in \cref{fig:qmLecture14:qmLecture14Fig1}.

\imageFigure{../figures/phy1520-quantum/qmLecture14Fig1}{Overlapping SHO domains.}{fig:qmLecture14:qmLecture14Fig1}{0.2}

Is it possible that such an overlapping set of harmonic oscillators can provide the properties of the angular momentum operators?  Let's relabel the counting so that we have two sets of positive counted SHO systems, each counted in a positive direction as sketched in \cref{fig:qmLecture14:qmLecture14Fig2}.

\imageFigure{../figures/phy1520-quantum/qmLecture14Fig2}{Relabeling the counting for overlapping SHO systems.}{fig:qmLecture14:qmLecture14Fig2}{0.2}

It turns out that given a constraint there the number of ways to distribute particles between a pair of SHO systems, the process that can be viewed as reproducing the angular momentum action is a transfer of particles from one harmonic oscillator to the other.  For \( \hatL_z = +j \)
%
\begin{equation}\label{eqn:qmLecture14:480}
\begin{aligned}
n_1 &= n_{\textrm{max}} \\
n_2 &= 0,
\end{aligned}
\end{equation}

and for \( \hatL_z = -j \)
%
\begin{equation}\label{eqn:qmLecture14:500}
\begin{aligned}
n_1 &= 0 \\
n_2 &= n_{\textrm{max}}.
\end{aligned}
\end{equation}
%
We can make the identifications
%
\begin{dmath}\label{eqn:qmLecture14:520}
\hatL_z = \lr{ n_1 - n_2 } \frac{\Hbar}{2},
\end{dmath}
%
and
\begin{equation}\label{eqn:qmLecture14:540}
j = \inv{2} n_{\textrm{max}},
\end{equation}
%
or
%
\begin{equation}\label{eqn:qmLecture14:560}
n_1 + n_2 = \text{fixed} = n_{\textrm{max}}
\end{equation}

Changes that keep \( n_1 + n_2 \) fixed are those that change \( n_1 \), \( n_2 \) by \( +1 \) or \( -1 \) respectively, as sketched in \cref{fig:qmLecture14:qmLecture14Fig3}.

\imageFigure{../figures/phy1520-quantum/qmLecture14Fig3}{Number conservation constraint.}{fig:qmLecture14:qmLecture14Fig3}{0.15}

Can we make an identification that takes
%
\begin{equation}\label{eqn:qmLecture14:580}
\ket{n_1, n_2} \overset{\hatL_{-}}{\rightarrow} \ket{n_1 - 1, n_2 + 1}?
\end{equation}
%
What operator in the SHO problem has this effect?  Let's try
%
%\begin{equation}\label{eqn:qmLecture14:600}
\boxedEquation{eqn:qmLecture14:620}{
\begin{aligned}
\hatL_{-} &= \Hbar a_2^\dagger a_1  \\
\hatL_{+} &= \Hbar a_1^\dagger a_2  \\
\hatL_z &= \frac{\Hbar}{2} \lr{ n_1 - n_2 }
\end{aligned}
}
%\end{equation}

Is this correct?  Do we need to make any scalar adjustments?  We want
%
\begin{equation}\label{eqn:qmLecture14:640}
\antisymmetric{\hatL_z}{\hatL_{\pm}} = \pm \Hbar \hatL_{\pm}.
\end{equation}
%
First check this with the \( \hatL_{+} \) commutator
%
\begin{dmath}\label{eqn:qmLecture14:660}
\antisymmetric{\hatL_z}{\hatL_{+}}
=
\inv{2} \Hbar^2 \antisymmetric{ n_1 - n_2}{a_1^\dagger a_2 }
=
\inv{2} \Hbar^2 \antisymmetric{ a_1^\dagger a_1 - a_2^\dagger a_2 }{a_1^\dagger a_2 }
=
\inv{2} \Hbar^2
\lr{
\antisymmetric{ a_1^\dagger a_1 }{a_1^\dagger a_2 }
-\antisymmetric{ a_2^\dagger a_2 }{a_1^\dagger a_2 }
}
=
\inv{2} \Hbar^2
\lr{
a_2 \antisymmetric{ a_1^\dagger a_1 }{a_1^\dagger }
-a_1^\dagger \antisymmetric{ a_2^\dagger a_2 }{a_2 }
}.
\end{dmath}
%
But
%
\begin{dmath}\label{eqn:qmLecture14:720}
\antisymmetric{ a^\dagger a }{a^\dagger }
=
a^\dagger a
a^\dagger
-
a^\dagger
a^\dagger a
=
a^\dagger \lr{ 1 +
a^\dagger a}
-
a^\dagger
a^\dagger a
=
a^\dagger,
\end{dmath}
%
and
\begin{dmath}\label{eqn:qmLecture14:740}
\antisymmetric{ a^\dagger a }{a}
=
a^\dagger a a
-a a^\dagger a
=
a^\dagger a a
-\lr{ 1 + a^\dagger a } a
=
-a,
\end{dmath}
%
so
\begin{equation}\label{eqn:qmLecture14:760}
\antisymmetric{\hatL_z}{\hatL_{+}} = \Hbar^2 a_2 a_1^\dagger = \Hbar \hatL_{+},
\end{equation}
%
as desired.  Similarly
%
\begin{dmath}\label{eqn:qmLecture14:780}
\antisymmetric{\hatL_z}{\hatL_{-}}
=
\inv{2} \Hbar^2 \antisymmetric{ n_1 - n_2}{a_2^\dagger a_1 }
=
\inv{2} \Hbar^2 \antisymmetric{ a_1^\dagger a_1 - a_2^\dagger a_2 }{a_2^\dagger a_1 }
=
\inv{2} \Hbar^2 \lr{
a_2^\dagger \antisymmetric{ a_1^\dagger a_1 }{a_1 }
- a_1 \antisymmetric{ a_2^\dagger a_2 }{a_2^\dagger }
}
=
\inv{2} \Hbar^2 \lr{
a_2^\dagger (-a_1)
- a_1 a_2^\dagger
}
=
- \Hbar^2 a_2^\dagger a_1
=
- \Hbar \hatL_{-}.
\end{dmath}
%
With
%
\begin{equation}\label{eqn:qmLecture14:800}
\begin{aligned}
j &= \frac{n_1 + n_2}{2} \\
m &= \frac{n_1 - n_2}{2} \\
\end{aligned}
\end{equation}

We can make the identification
%
\begin{equation}\label{eqn:qmLecture14:820}
\ket{n_1, n_2} = \ket{ j+ m , j - m}.
\end{equation}
%
\paragraph{Another way}

With
%
\begin{equation}\label{eqn:qmLecture14:840}
\hatL_{+} \ket{j, m} = d_{j,m}^{+} \ket{j, m+1}
\end{equation}
%
or
%
\begin{equation}\label{eqn:qmLecture14:860}
\Hbar a_1^\dagger a_2  \ket{j + m, j-m} = d_{j,m}^{+} \ket{ j + m + 1, j- m-1},
\end{equation}
%
we can seek this factor \( d_{j,m}^{+} \) by operating with \( \hatL_{+} \)
%
\begin{dmath}\label{eqn:qmLecture14:880}
\hatL_{+} \ket{j, m}
=
\Hbar a_1^\dagger a_2 \ket{n_1, n_2}
=
\Hbar a_1^\dagger a_2 \ket{j+m,j-m}
=
\Hbar \sqrt{ n + 1 } \sqrt{n_2} \ket{j+m +1,j-m-1}
=
\Hbar \sqrt{ \lr{ j+ m + 1}\lr{ j - m } } \ket{j+m +1,j-m-1}
\end{dmath}
%
That gives
\begin{equation}\label{eqn:qmLecture14:900}
\begin{aligned}
d_{j,m}^{+} &= \Hbar \sqrt{\lr{ j - m } \lr{ j+ m + 1} } \\
d_{j,m}^{-} &= \Hbar \sqrt{\lr{ j + m } \lr{ j- m + 1} }.
\end{aligned}
\end{equation}
%
This equivalence can be used to model spin interaction in crystals as harmonic oscillators.   This equivalence of lattice vibrations and spin oscillations is called ``spin waves''.
\index{spin wave}

%\EndArticle
