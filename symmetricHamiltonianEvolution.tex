%
% Copyright � 2015 Peeter Joot.  All Rights Reserved.
% Licenced as described in the file LICENSE under the root directory of this GIT repository.
%
%\input{../blogpost.tex}
%\renewcommand{\basename}{symmetricHamiltonianEvolution}
%\renewcommand{\dirname}{notes/phy1520/}
%%\newcommand{\dateintitle}{}
%%\newcommand{\keywords}{}
%
%\input{../peeter_prologue_print2.tex}
%
%\usepackage{peeters_layout_exercise}
%\usepackage{peeters_braket}
%\usepackage{peeters_figures}
%
%\beginArtNoToc
%
%\generatetitle{A symmetric real Hamiltonian}
%\chapter{A symmetric real Hamiltonian}
%\label{chap:symmetricHamiltonianEvolution}

\makeoproblem{A symmetric real Hamiltonian.}{problem:symmetricHamiltonianEvolution:9}{\citep{sakurai2014modern} pr. 2.9}{
\index{Hamiltonian!symmetric real}

Find the time evolution for the state \( \ket{a'} \) for a Hamiltonian of the form
%
\begin{dmath}\label{eqn:symmetricHamiltonianEvolution:20}
H = \delta \lr{ \ket{a'}\bra{a'} + \ket{a''}\bra{a''} }
\end{dmath}
} % problem

\makeanswer{problem:symmetricHamiltonianEvolution:9}{

This Hamiltonian has the matrix representation
%
\begin{dmath}\label{eqn:symmetricHamiltonianEvolution:40}
H =
\begin{bmatrix}
0 & \delta \\
\delta & 0
\end{bmatrix},
\end{dmath}
%
which has a characteristic equation of
%
\begin{dmath}\label{eqn:symmetricHamiltonianEvolution:60}
\lambda^2 -\delta^2 = 0,
\end{dmath}
%
so the energy eigenvalues are \( \pm \delta \).

The diagonal basis states are respectively
%
\begin{dmath}\label{eqn:symmetricHamiltonianEvolution:80}
\ket{\pm\delta} =
\inv{\sqrt{2}}
\begin{bmatrix}
\pm 1 \\
1
\end{bmatrix}.
\end{dmath}
%
The time evolution operator is
%
\begin{dmath}\label{eqn:symmetricHamiltonianEvolution:100}
U
= e^{-i H t/\Hbar}
=
  e^{-i \delta t/\Hbar} \ket{+\delta}\bra{+\delta}
+ e^{i \delta t/\Hbar} \ket{-\delta}\bra{-\delta}
=
\frac{e^{-i \delta t/\Hbar} }{2}
\begin{bmatrix}
1 & 1
\end{bmatrix}
\begin{bmatrix}
1  \\
1
\end{bmatrix}
+ \frac{e^{i \delta t/\Hbar} }{2}
\begin{bmatrix}
-1 & 1
\end{bmatrix}
\begin{bmatrix}
-1  \\
1
\end{bmatrix}
=
\frac{e^{-i \delta t/\Hbar} }{2}
\begin{bmatrix}
1 & 1 \\
1 & 1
\end{bmatrix}
+\frac{e^{i \delta t/\Hbar} }{2}
\begin{bmatrix}
1 & -1 \\
-1 & 1
\end{bmatrix}
=
\begin{bmatrix}
\cos(\delta t/\Hbar) & -i\sin(\delta t/\Hbar) \\
-i \sin(\delta t/\Hbar) & \cos(\delta t/\Hbar) \\
\end{bmatrix}.
\end{dmath}
%
%The non-diagonal states have the matrix representation
%
%\begin{dmath}\label{eqn:symmetricHamiltonianEvolution:120}
%\begin{aligned}
%\ket{a'} &= \inv{\sqrt{2}} \lr{ \ket{+\delta} - \ket{-\delta} } \\
%\ket{a''} &= \inv{\sqrt{2}} \lr{ \ket{+\delta} + \ket{-\delta} },
%\end{aligned}
%\end{dmath}
%
%so
The desired time evolution in the original basis is
%
\begin{dmath}\label{eqn:symmetricHamiltonianEvolution:140}
\ket{a', t}
=
e^{-i H t/\Hbar}
\ket{a', 0}
=
\begin{bmatrix}
\cos(\delta t/\Hbar) & -i\sin(\delta t/\Hbar) \\
-i \sin(\delta t/\Hbar) & \cos(\delta t/\Hbar) \\
\end{bmatrix}
\begin{bmatrix}
1 \\
0
\end{bmatrix}
=
\begin{bmatrix}
\cos(\delta t/\Hbar) \\
-i \sin(\delta t/\Hbar)
\end{bmatrix}
=
\cos(\delta t/\Hbar) \ket{a',0} -i \sin(\delta t/\Hbar) \ket{a'',0}.
\end{dmath}
%
This evolution has the same structure as left circularly polarized light.

The probability of finding the system in state \( \ket{a''} \) given an initial state of \( \ket{a',0} \) is
%
\begin{dmath}\label{eqn:symmetricHamiltonianEvolution:160}
P
=
\Abs{\braket{a''}{a',t}}^2
=
\sin^2 \lr{ \delta t/\Hbar }.
\end{dmath}
} % answer

%\EndArticle
