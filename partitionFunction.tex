%
% Copyright � 2015 Peeter Joot.  All Rights Reserved.
% Licenced as described in the file LICENSE under the root directory of this GIT repository.
%
%\input{../blogpost.tex}
%\renewcommand{\basename}{partitionFunction}
%\renewcommand{\dirname}{notes/phy1520/}
%%\newcommand{\dateintitle}{}
%%\newcommand{\keywords}{}
%
%\input{../peeter_prologue_print2.tex}
%
%\usepackage{peeters_layout_exercise}
%\usepackage{peeters_braket}
%\usepackage{peeters_figures}
%
%\beginArtNoToc
%
%\generatetitle{Partition function and ground state energy}
%\chapter{Partition function and ground state energy}
%\label{chap:partitionFunction}
%
\makeoproblem{Partition function, ground state energy.}{problem:partitionFunction:32}{\citep{sakurai2014modern} pr. 2.32}{
\index{partition function}
\index{ground state}

Define the partition function as
%
\begin{equation}\label{eqn:partitionFunction:20}
Z = \int d^3 x' \evalbar{ K( \Bx', t ; \Bx', 0 ) }{\beta = i t/\Hbar},
\end{dmath}
%
Show that the ground state energy is given by
%
\begin{equation}\label{eqn:partitionFunction:40}
-\inv{Z} \PD{\beta}{Z}, \qquad \beta \rightarrow \infty.
\end{dmath}
%
} % problem
%
\makeanswer{problem:partitionFunction:32}{
%
The propagator evaluated at the same point is
%
\begin{dmath}\label{eqn:partitionFunction:60}
K( \Bx', t ; \Bx', 0 )
=
\sum_{a'} \braket{\Bx'}{a'} \ket{a'}{\Bx'} \exp\lr{ -\frac{i E_{a'} t}{\Hbar}}
=
\sum_{a'} \Abs{\braket{\Bx'}{a'}}^2 \exp\lr{ -\frac{i E_{a'} t}{\Hbar}}
=
\sum_{a'} \Abs{\braket{\Bx'}{a'}}^2 \exp\lr{ -E_{a'} \beta}.
\end{dmath}
%
The derivative is
\begin{dmath}\label{eqn:partitionFunction:80}
\PD{\beta}{Z}
=
-\int d^3 x' \sum_{a'} E_{a'} \Abs{\braket{\Bx'}{a'}}^2 \exp\lr{ -E_{a'} \beta}.
\end{dmath}
%
In the \( \beta \rightarrow \infty \) this sum will be dominated by the term with the lowest value of \( E_{a'} \).  Suppose that state is \( a' = 0 \), then
%
\begin{dmath}\label{eqn:partitionFunction:100}
\lim_{ \beta \rightarrow \infty }
-\inv{Z} \PD{\beta}{Z}
= \frac{
\int d^3 x' E_{0} \Abs{\braket{\Bx'}{0}}^2 \exp\lr{ -E_{0} \beta}
}
{
\int d^3 x' \Abs{\braket{\Bx'}{0}}^2 \exp\lr{ -E_{0} \beta}
}
= E_0.
\end{dmath}
%
This stat mech like result seems very striking and profound, and makes me want to go off and study the QM formulation of stat mech that I recall seeing in \citep{pathriastatistical}, but not covered back in phy452.
} % answer

%\EndArticle
