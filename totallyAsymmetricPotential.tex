%
% Copyright � 2015 Peeter Joot.  All Rights Reserved.
% Licenced as described in the file LICENSE under the root directory of this GIT repository.
%
%{
%\input{../blogpost.tex}
%\renewcommand{\basename}{totallyAsymmetricPotential}
%\renewcommand{\dirname}{notes/phy1520/}
%%\newcommand{\dateintitle}{}
%%\newcommand{\keywords}{}
%
%\input{../peeter_prologue_print2.tex}
%
%\usepackage{peeters_layout_exercise}
%\usepackage{peeters_braket}
%\usepackage{peeters_figures}
%
%\beginArtNoToc
%
%\generatetitle{Totally asymmetric potential}
%%\chapter{Totally asymmetric potential}
%%\label{chap:totallyAsymmetricPotential}
%
\makeoproblem{Totally asymmetric potential.}{problem:totallyAsymmetricPotential:1}{\citep{sakurai2014modern} pr. 4.11}{
\index{spherical harmonics}
\index{time reversal}
%
\makesubproblem{}{problem:totallyAsymmetricPotential:1:a}
Given a time reversal invariant Hamiltonian, show that for any energy eigenket
%
\begin{dmath}\label{eqn:totallyAsymmetricPotential:20}
\expectation{\BL} = 0.
\end{dmath}
%
\makesubproblem{}{problem:totallyAsymmetricPotential:1:b}
%
If the wave function of such a state is expanded as
%
\begin{dmath}\label{eqn:totallyAsymmetricPotential:40}
\sum_{l,m} F_{l m} Y_{l m}(\theta, \phi),
\end{dmath}
%
what are the phase restrictions on \( F_{lm} \)?
%
} % problem
%
\makeanswer{problem:totallyAsymmetricPotential:1}{
%
\makeSubAnswer{}{problem:totallyAsymmetricPotential:1:a}
%
For a time reversal invariant Hamiltonian \( H \) we have
%
\begin{dmath}\label{eqn:totallyAsymmetricPotential:60}
H \Theta = \Theta H.
\end{dmath}
%
If \( \ket{\psi} \) is an energy eigenstate with eigenvalue \( E \), we have
%
\begin{dmath}\label{eqn:totallyAsymmetricPotential:80}
H \Theta \ket{\psi}
= \Theta H \ket{\psi}
= \lambda \Theta \ket{\psi},
\end{dmath}
%
so \( \Theta \ket{\psi} \) is also an eigenvalue of \( H \), so can only differ from \( \ket{\psi} \) by a phase factor.  That is
%
\begin{dmath}\label{eqn:totallyAsymmetricPotential:100}
\ket{\psi'}
=
\Theta \ket{\psi}
= e^{i\delta} \ket{\psi}.
\end{dmath}
%
Now consider the expectation of \( \BL \) with respect to a time reversed state
%
\begin{dmath}\label{eqn:totallyAsymmetricPotential:120}
\bra{ \psi'} \BL \ket{\psi'}
=
\bra{ \psi} \Theta^{-1} \BL \Theta \ket{\psi}
=
\bra{ \psi} (-\BL) \ket{\psi},
\end{dmath}
%
however, we also have
%
\begin{dmath}\label{eqn:totallyAsymmetricPotential:140}
\bra{ \psi'} \BL \ket{\psi'}
=
\lr{ \bra{ \psi} e^{-i\delta} } \BL \lr{ e^{i\delta} \ket{\psi} }
=
\bra{\psi} \BL \ket{\psi},
\end{dmath}
%
so we have \( \bra{\psi} \BL \ket{\psi} = -\bra{\psi} \BL \ket{\psi} \) which is only possible if \( \expectation{\BL} = \bra{\psi} \BL \ket{\psi} = 0\).
%
\makeSubAnswer{}{problem:totallyAsymmetricPotential:1:b}
%
Consider the expansion of the wave function of a time reversed energy eigenstate
%
\begin{dmath}\label{eqn:totallyAsymmetricPotential:160}
\bra{\Bx} \Theta \ket{\psi}
=
\bra{\Bx} e^{i\delta} \ket{\psi}
=
e^{i\delta} \braket{\Bx}{\psi},
\end{dmath}
%
and then consider the same state expanded in the position basis
%
\begin{dmath}\label{eqn:totallyAsymmetricPotential:180}
\bra{\Bx} \Theta \ket{\psi}
=
\bra{\Bx} \Theta \int d^3 \Bx' \lr{ \ket{\Bx'}\bra{\Bx'} } \ket{\psi}
=
\bra{\Bx} \Theta \int d^3 \Bx' \lr{ \braket{\Bx'}{\psi} } \ket{\Bx'}
=
\bra{\Bx} \int d^3 \Bx' \lr{ \braket{\Bx'}{\psi} }^\conj \Theta \ket{\Bx'}
=
\bra{\Bx} \int d^3 \Bx' \lr{ \braket{\Bx'}{\psi} }^\conj \ket{\Bx'}
=
\int d^3 \Bx' \lr{ \braket{\Bx'}{\psi} }^\conj \braket{\Bx}{\Bx'}
=
\int d^3 \Bx' \braket{\psi}{\Bx'} \delta(\Bx- \Bx')
=
\braket{\psi}{\Bx}.
\end{dmath}
%
This demonstrates a relationship between the wave function and its complex conjugate
%
\begin{dmath}\label{eqn:totallyAsymmetricPotential:200}
\braket{\Bx}{\psi} = e^{-i\delta} \braket{\psi}{\Bx}.
\end{dmath}
%
Now expand the wave function in the spherical harmonic basis
%
\begin{dmath}\label{eqn:totallyAsymmetricPotential:220}
\begin{aligned}
\braket{\Bx}{\psi}
&=
\int d\Omega \braket{\Bx}{\ncap}\braket{\ncap}{\psi} \\
&=
\sum_{lm} F_{lm}(r) Y_{lm}(\theta, \phi) \\
&=
e^{-i\delta}
\lr{
\sum_{lm} F_{lm}(r) Y_{lm}(\theta, \phi) }^\conj \\
&=
e^{-i\delta}
\sum_{lm} \lr{ F_{lm}(r)}^\conj Y_{lm}^\conj(\theta, \phi)  \\
&=
e^{-i\delta}
\sum_{lm} \lr{ F_{lm}(r)}^\conj (-1)^m Y_{l,-m}(\theta, \phi)  \\
&=
e^{-i\delta}
\sum_{lm} \lr{ F_{l,-m}(r)}^\conj (-1)^m Y_{l,m}(\theta, \phi),
\end{aligned}
\end{dmath}

so the \( F_{lm} \) functions are constrained by
%
\begin{dmath}\label{eqn:totallyAsymmetricPotential:240}
F_{lm}(r) = e^{-i\delta} \lr{ F_{l,-m}(r)}^\conj (-1)^m.
\end{dmath}
} % answer

%}
%\EndArticle
