%
% Copyright � 2015 Peeter Joot.  All Rights Reserved.
% Licenced as described in the file LICENSE under the root directory of this GIT repository.
%
%\input{../blogpost.tex}
%\renewcommand{\basename}{qmLecture18}
%\renewcommand{\dirname}{notes/phy1520/}
%\newcommand{\keywords}{PHY1520H}
%\input{../peeter_prologue_print2.tex}
%
%%\usepackage{phy1520}
%\usepackage{peeters_braket}
%%\usepackage{peeters_layout_exercise}
%\usepackage{peeters_figures}
%\usepackage{mathtools}
%
%\beginArtNoToc
%\generatetitle{PHY1520H Graduate Quantum Mechanics.  Lecture 18: Approximation methods.  Taught by Prof.\ Arun Paramekanti}
%%\chapter{Approximation methods}
%\label{chap:qmLecture18}
%
%\paragraph{Disclaimer}
%
%Peeter's lecture notes from class.  These may be incoherent and rough, especially since I didn't attend this class myself, and am doing a walkthrough of notes provided by Nishant.
%
%These are notes for the UofT course PHY1520, Graduate Quantum Mechanics, taught by Prof. Paramekanti, covering \textchapref{{5}} \citep{sakurai2014modern} content.
%
\section{Approximation methods}
\index{approximation methods}
\index{perturbation}

Suppose we have a perturbed Hamiltonian

\begin{equation}\label{eqn:qmLecture18:20}
H = H_0 + \lambda V,
\end{equation}

where \( \lambda = 0 \) represents a solvable (perhaps known) system, and \( \lambda = 1 \) is the case of interest.  There are two approaches of interest

\begin{enumerate}
\item Direct solution of \( H \) with \( \lambda = 1 \).
\item Take \( \lambda \) small, and do a series expansion.  This is perturbation theory.
\index{perturbation}
\end{enumerate}

\section{Variational methods}
\index{variational method}

Given

\begin{dmath}\label{eqn:qmLecture18:40}
H \ket{\phi_n} = E_n \ket{\phi_n},
\end{dmath}

where we don't know \( \ket{\phi_n} \), we can compute the expectation with respect to an arbitrary state \( \ket{\psi} \)

\begin{dmath}\label{eqn:qmLecture18:60}
\bra{\psi} H \ket{\psi}
=
\bra{\psi} H \lr{ \sum_n \ket{\phi_n} \bra{\phi_n} } \ket{\psi}
=
\sum_n E_n \braket{\psi}{\phi_n} \braket{\phi_n}{\psi}
=
\sum_n E_n \Abs{\braket{\psi}{\phi_n}}^2.
\end{dmath}

Define

\begin{dmath}\label{eqn:qmLecture18:80}
\overbar{E}
= \frac{\bra{\psi} H \ket{\psi}}{\braket{\psi}{\psi}}.
\end{dmath}

Assuming that it is possible to express the state in the Hamiltonian energy basis

\begin{dmath}\label{eqn:qmLecture18:100}
\ket{\psi}
=
\sum_n a_n \ket{\phi_n},
\end{dmath}

this average energy is
\begin{dmath}\label{eqn:qmLecture18:120}
\overbar{E}
= \frac{ \sum_{m,n}\bra{\phi_m} a_m^\conj H a_n \ket{\phi_n}}{ \sum_n \Abs{a_n}^2 }
= \frac{ \sum_{n} \Abs{a_n}^2 E_n }{ \sum_n \Abs{a_n}^2 }.
= \sum_{n}
\frac{\Abs{a_n}^2 }{ \sum_n \Abs{a_n}^2 }
E_n
= \sum_n \frac{P_n}{\sum_m P_m} E_n,
\end{dmath}

where \( P_m = \Abs{a_m}^2 \), which has the structure of a probability coefficient once divided by \( \sum_m P_m \),
%FIXME: Nishant's notes showed these probabilities ordered so that \( P_1 \ge P_2 \ge P_3 \cdots \),
as sketched in \cref{fig:lecture18:lecture18Fig1}.

\imageFigure{../phy1520-quantum-figures/lecture18Fig1}{A decreasing probability distribution.}{fig:lecture18:lecture18Fig1}{0.1}

This average energy is a probability weighted average of the individual energy basis states.   One of those energies is the ground state energy \( E_1 \), so we necessarily have

%\begin{dmath}\label{eqn:qmLecture18:140}
\boxedEquation{eqn:qmLecture18:160}{
\overbar{E} \ge E_1.
}
%\end{dmath}

%Why would such an ordering necessarily exist if \( \ket{\psi} \) is an arbitrary state?  Could that state not be closer to the third energy eigenstate (say), than to the ground state?
%
%We also have
%
%\begin{dmath}\label{eqn:qmLecture18:180}
%\overbar{E} - E_1
%= \sum_n \frac{P_n}{\sum_m P_m} E_n - \sum_n \frac{P_n}{\sum_m P_m} E_1
%= \sum_n \frac{P_n}{\sum_m P_m} \lr{ E_n - E_1 },
%\end{dmath}
%
%where \( E_n - E_1 \ge 0 \).  This means that we also have
%
%\begin{dmath}\label{eqn:qmLecture18:200}
%\overbar{E} - E_1 \ge 0.
%\end{dmath}
%
%FIXME: not sure what this last point was about?  How is \( \overbar{E} \ge E_1 \) different than \( \overbar{E} - E_1 \ge 0 \)?
%

\makeexample{Particle in an offset box.}{example:qmLecture18:1}{

For the infinite potential box sketched in \cref{fig:lecture18:lecture18Fig2}.

\imageFigure{../phy1520-quantum-figures/lecture18Fig2}{Infinite potential \( [0,L] \) box.}{fig:lecture18:lecture18Fig2}{0.2}

The exact solutions for such a system are found to be

\begin{dmath}\label{eqn:qmLecture18:220}
\psi(x) = \sqrt{\frac{2}{L}} \sin\lr{ \frac{n \pi}{L} x },
\end{dmath}

where the energies are

\begin{dmath}\label{eqn:qmLecture18:240}
E = \frac{\Hbar^2}{2m} \frac{n^2 \pi^2}{L^2}.
\end{dmath}

The function \( \psi' = x (L-x) \) also satisfies the boundary value constraints?  How close in energy is that function to the ground state?

\begin{dmath}\label{eqn:qmLecture18:260}
\overbar{E}
=
-\frac{\Hbar^2}{2m} \frac{\int_0^L dx x (L-x) \frac{d^2}{dx^2} \lr{ x (L-x) }}{
\int_0^L dx x^2 (L-x)^2
}
=
\frac{\Hbar^2}{2m} \frac{\frac{2 L^3}{6}}{
\frac{L^5}{30}
}
=
\frac{\Hbar^2}{2m} \frac{10}{L^2}.
\end{dmath}

This average energy is quite close to the ground state energy

\begin{dmath}\label{eqn:qmLecture18:280}
\frac{\overbar{E} }{E_1} = \frac{10}{\pi^2} = 1.014.
\end{dmath}

} % example

\makeexample{Particle in a symmetric box.}{example:qmLecture18:2}{

\imageFigure{../phy1520-quantum-figures/lecture18Fig3}{Infinite potential \( [-L/2,L/2] \) box.}{fig:qmLecture18:lecture18Fig3}{0.15}

Shifting the boundaries, as sketched in \cref{fig:qmLecture18:lecture18Fig3} doesn't change the energy levels.  For this potential let's try a shifted trial function

\begin{dmath}\label{eqn:qmLecture18:300}
\psi(x) = \lr{ x - \frac{L}{2} } \lr{ x + \frac{L}{2} } = x^2 - \frac{L^2}{4},
\end{dmath}

without worrying about the form of the exact solution.  This produces the same result as above

\begin{dmath}\label{eqn:qmLecture18:270}
\overbar{E}
=
-\frac{\Hbar^2}{2m} \frac{\int_0^L dx \lr{ x^2 - \frac{L^2}{4} } \frac{d^2}{dx^2} \lr{ x^2 - \frac{L^2}{4} }}{
\int_0^L dx \lr{x^2 - \frac{L^2}{4} }^2
}
=
-\frac{\Hbar^2}{2m} \frac{- 2 L^3/6}{
\frac{L^5}{30}
}
=
\frac{\Hbar^2}{2m} \frac{10}{L^2}.
\end{dmath}

} % example

\paragraph{Summary (Nishant)}

The above example is that of a particle in a box. The actual wave function is a sin as shown. But we can
come up with a guess wave function that meets the boundary conditions and ask how accurate it is
compared to the actual one.

Basically we are assuming a wave function form and then seeing how it differs from the exact form.
We cannot do this if we have nothing to compare it against. But, we note that the variance of the
number operator in the systems eigenstate is zero. So we can still calculate the variance and try to
minimize it. This is one way of coming up with an approximate wave function. This does not necessarily
give the ground state wave function though. For this we need to minimize the energy itself.

%\EndArticle
