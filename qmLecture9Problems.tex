%
% Copyright © 2015 Peeter Joot.  All Rights Reserved.
% Licenced as described in the file LICENSE under the root directory of this GIT repository.
%
%
\makeproblem{Calculate the right going diagonalization.}{problem:qmLecture9Problems:1}{
\index{Dirac equation!right going solution}
%
\makesubproblem{}{problem:qmLecture9Problems:1:a}
%
Prove \cref{eqn:qmLecture9:160}.
%
% \makesubproblem{}{problem:qmLecture9Problems:1:b}
%
% Do the same thing for a left going wave solution.
%
} % problem
%
\makeanswer{problem:qmLecture9Problems:1}{
%
\makeSubAnswer{}{problem:qmLecture9Problems:1:a}
To determine the relations for \( \theta_k \) we have to solve
%
\begin{dmath}\label{eqn:qmLecture9Problems:280}
\begin{bmatrix}
E_k & 0 \\
0 & -E_k
\end{bmatrix}
= R^{-1} H R.
\end{dmath}
%
Working with \( \Hbar = c = 1 \) temporarily, and \( C = \cos\theta_k, S = \sin\theta_k \), that is
%
\begin{dmath}\label{eqn:qmLecture9Problems:300}
\begin{bmatrix}
E_k & 0 \\
0 & -E_k
\end{bmatrix}
=
\begin{bmatrix}
C & S \\
-S & C
\end{bmatrix}
\begin{bmatrix}
k & m \\
m & -k
\end{bmatrix}
\begin{bmatrix}
C & -S \\
S & C
\end{bmatrix}
=
\begin{bmatrix}
C & S \\
-S & C
\end{bmatrix}
\begin{bmatrix}
k C + m S & -k S + m C \\
m C - k S & -m S - k C
\end{bmatrix}
=
\begin{bmatrix}
k C^2 + m S C + m C S - k S^2   & -k S C + m C^2 -m S^2 - k C S \\
-k C S - m S^2 + m C^2 - k S C & k S^2 - m C S -m S C - k C^2
\end{bmatrix}
=
\begin{bmatrix}
k \cos(2 \theta_k) + m \sin(2 \theta_k) & m \cos(2 \theta_k) - k \sin(2 \theta_k) \\
m \cos(2 \theta_k) - k \sin(2 \theta_k) & -k \cos(2 \theta_k) - m \sin(2 \theta_k) \\
\end{bmatrix}.
\end{dmath}
%
This gives
%
\begin{dmath}\label{eqn:qmLecture9Problems:320}
E_k
\begin{bmatrix}
1 \\
0
\end{bmatrix}
=
\begin{bmatrix}
k \cos(2 \theta_k) + m \sin(2 \theta_k) \\
m \cos(2 \theta_k) - k \sin(2 \theta_k) \\
\end{bmatrix}
=
\begin{bmatrix}
k & m \\
m & -k
\end{bmatrix}
\begin{bmatrix}
\cos(2 \theta_k) \\
\sin(2 \theta_k) \\
\end{bmatrix}.
\end{dmath}
%
Adding back in the \(\Hbar\)'s and \(c\)'s this is
%
\begin{dmath}\label{eqn:qmLecture9Problems:340}
\begin{bmatrix}
\cos(2 \theta_k) \\
\sin(2 \theta_k) \\
\end{bmatrix}
=
\frac{E_k}{-(\Hbar k c)^2 -(m c^2)^2}
\begin{bmatrix}
- \Hbar k c & - m c^2 \\
- m c^2     & \Hbar k c
\end{bmatrix}
\begin{bmatrix}
1 \\
0
\end{bmatrix}
=
\inv{E_k}
\begin{bmatrix}
\Hbar k c \\
m c^2
\end{bmatrix}.
\end{dmath}
%
% \makeSubAnswer{}{problem:qmLecture9Problems:1:b}
%
% For a wave function of the form
%
% \begin{dmath}\label{eqn:qmLecture9Problems:440}
% \Phi =
% e^{-i k x}
% \begin{bmatrix}
% f_1 \\
% f_2
% \end{bmatrix},
% \end{dmath}
%
% the Dirac equation has the form
% \begin{dmath}\label{eqn:qmLecture9Problems:460}
% H \Phi =
% \begin{bmatrix}
% \hatp c & m c^2 \\
% m c^2 & - \hatp c
% \end{bmatrix}
% e^{-i k x}
% \begin{bmatrix}
% f_1 \\
% f_2
% \end{bmatrix}
% =
% \begin{bmatrix}
% -\Hbar k c & m c^2 \\
% m c^2 & \Hbar k c
% \end{bmatrix}
% \begin{bmatrix}
% f_1 \\
% f_2
% \end{bmatrix}.
% \end{dmath}
%
% Again introducing
%
% \begin{dmath}\label{eqn:qmLecture9Problems:480}
% \begin{bmatrix}
% f_1 \\
% f_2
% \end{bmatrix}
% =
% R
% \begin{bmatrix}
% f_{+} \\
% f_{-} \\
% \end{bmatrix},
% \end{dmath}
%
% so that
%
% \begin{dmath}\label{eqn:qmLecture9Problems:500}
% H R
% \begin{bmatrix}
% f_{+} \\
% f_{-} \\
% \end{bmatrix}
% =
% \begin{bmatrix}
% -\Hbar k c & m c^2 \\
% m c^2 & \Hbar k c
% \end{bmatrix}
% R
% \begin{bmatrix}
% f_{+} \\
% f_{-} \\
% \end{bmatrix}.
% \end{dmath}
%
% With
%
% \begin{dmath}\label{eqn:qmLecture9Problems:520}
% H_k
% =
% \begin{bmatrix}
% -\Hbar k c & m c^2 \\
% m c^2 & \Hbar k c
% \end{bmatrix},
% \end{dmath}
%
% We either wish to solve
% \begin{dmath}\label{eqn:qmLecture9Problems:540}
% \begin{bmatrix}
% E_k & 0 \\
% 0 & -E_k
% \end{bmatrix}
% =
% R^{-1} H_k R,
% \end{dmath}
%
% which was the approach used above (the hard way).  A better idea is to solve
%
% \begin{dmath}\label{eqn:qmLecture9Problems:560}
% R
% \begin{bmatrix}
% E_k & 0 \\
% 0 & -E_k
% \end{bmatrix}
% R^{-1}
% =
% H_k.
% \end{dmath}
%
% Expanding this we have
% \begin{dmath}\label{eqn:qmLecture9Problems:580}
% H_k
% =
% \begin{bmatrix}
% C & -S \\
% S & C
% \end{bmatrix}
% \begin{bmatrix}
% E_k & 0 \\
% 0 & -E_k
% \end{bmatrix}
% \begin{bmatrix}
% C & S \\
% -S & C
% \end{bmatrix}
% =
% E_k
% \begin{bmatrix}
% C & S \\
% S & -C
% \end{bmatrix}
% \begin{bmatrix}
% C & S \\
% -S & C
% \end{bmatrix}
% =
% \begin{bmatrix}
% C^2 - S^2 & 2 C S \\
% 2 C S & S^2 - C^2
% \end{bmatrix},
% \end{dmath}
%
% or
% \begin{dmath}\label{eqn:qmLecture9Problems:600}
% \begin{aligned}
% \cos(2 \theta_k) &= - \frac{\Hbar k c}{E_k} \\
% \sin(2 \theta_k) &= \frac{m c^2}{E_k}.
% \end{aligned}
% \end{dmath}
} % answer
%
\makeproblem{Verify the plane wave eigenstate.}{problem:qmLecture9Problems:3}{
%
\makesubproblem{}{problem:qmLecture9Problems:3:a}
%
Verify \cref{eqn:qmLecture9:261}.
%
\makesubproblem{}{problem:qmLecture9Problems:3:b}
%
Find the form of the reflected wave.
%
} % problem
%
\makeanswer{problem:qmLecture9Problems:3}{
%
\makeSubAnswer{}{problem:qmLecture9Problems:3:a}
With
%
\begin{dmath}\label{eqn:qmLecture9Problems:620}
H_k
=
\begin{bmatrix}
\Hbar k c & m c^2 \\
m c^2 & -\Hbar k c
\end{bmatrix}
=
R
\begin{bmatrix}
E_k & 0 \\
0 & -E_k
\end{bmatrix}
R^{-1}
\end{dmath},
%
We wish to show that
%
\begin{dmath}\label{eqn:qmLecture9Problems:640}
H_k
\begin{bmatrix}
\cos\theta_k \\
\sin\theta_k
\end{bmatrix}
e^{i k x}
=
E_k
\begin{bmatrix}
\cos\theta_k \\
\sin\theta_k
\end{bmatrix}
e^{i k x}.
\end{dmath}
%
The LHS side expands to
\begin{dmath}\label{eqn:qmLecture9Problems:660}
\begin{bmatrix}
C & - S \\
S & C
\end{bmatrix}
\begin{bmatrix}
E_k & 0 \\
0 & -E_k
\end{bmatrix}
\begin{bmatrix}
C & S \\
-S & C
\end{bmatrix}
\begin{bmatrix}
C \\
S \\
\end{bmatrix}
e^{i k x}
=
\begin{bmatrix}
C & - S \\
S & C
\end{bmatrix}
\begin{bmatrix}
E_k & 0 \\
0 & -E_k
\end{bmatrix}
\begin{bmatrix}
C^2 + S^2 \\
0 \\
\end{bmatrix}
e^{i k x}
=
\begin{bmatrix}
C & - S \\
S & C
\end{bmatrix}
\begin{bmatrix}
E_k \\
0
\end{bmatrix}
e^{i k x}
=
E_k
\begin{bmatrix}
C \\
S
\end{bmatrix}
e^{i k x}. \qquad \qedmarker
\end{dmath}
%
\makeSubAnswer{}{problem:qmLecture9Problems:3:b}
%
For the reflected wave, let's assume that the reflected wave has the form
%
\begin{dmath}\label{eqn:qmLecture9Problems:680}
\Psi =
\begin{bmatrix}
\sin\theta_k \\
-\cos\theta_k \\
\end{bmatrix}
e^{-i k x }.
\end{dmath}
%
Let's verify this
%
\begin{dmath}\label{eqn:qmLecture9Problems:700}
\begin{bmatrix}
C & - S \\
S & C
\end{bmatrix}
\begin{bmatrix}
E_k & 0 \\
0 & -E_k
\end{bmatrix}
\begin{bmatrix}
C & S \\
-S & C
\end{bmatrix}
\begin{bmatrix}
S \\
-C \\
\end{bmatrix}
e^{-i k x}
=
\begin{bmatrix}
C & - S \\
S & C
\end{bmatrix}
\begin{bmatrix}
E_k & 0 \\
0 & -E_k
\end{bmatrix}
\begin{bmatrix}
C S - S C \\
-S^2 - C^2
\end{bmatrix}
e^{-i k x}
=
\begin{bmatrix}
C & - S \\
S & C
\end{bmatrix}
\begin{bmatrix}
0 \\
E_k
\end{bmatrix}
e^{-i k x}
=
E_k
\begin{bmatrix}
- S \\
C
\end{bmatrix}
e^{-i k x}
=
-E_k
\begin{bmatrix}
S \\
-C
\end{bmatrix}
e^{-i k x}.
\end{dmath}
%
However, note that we have a different rotation angle \( \theta_k \) for the forward going and reverse going waves.

For the incident wave, \( k > 0 \), we have
%
\begin{dmath}\label{eqn:qmLecture9Problems:720}
\tan 2 \theta_k = \frac{ \Hbar k c }{ m c^2 },
\end{dmath}
%
and for the reflected wave, \( k < 0 \), we have
%
\begin{dmath}\label{eqn:qmLecture9Problems:740}
\tan 2 \theta_k = \frac{ -\Hbar k c }{ m c^2 }.
\end{dmath}
%
The rotation angle for both cases can therefore be expressed as
%
\begin{dmath}\label{eqn:qmLecture9Problems:760}
\theta_k = \inv{2} \Atan\lr{ \Abs{\frac{\Hbar k }{m c}} }.
\end{dmath}
%
} % answer
%
\makeproblem{Verify the Dirac current relationship.}{problem:qmLecture9Problems:2}{
\index{Dirac equation!current}
Prove \cref{eqn:qmLecture9:240}.
} % problem
%
\makeanswer{problem:qmLecture9Problems:2}{
%
The components of the Schr\"{o}dinger equation are
%
\begin{equation}\label{eqn:qmLecture9Problems:360}
\begin{aligned}
-i \Hbar \PD{t}{\psi_1} &= -i \Hbar c \PD{x}{\psi_1} + m c^2 \psi_2  \\
-i \Hbar \PD{t}{\psi_2} &= m c^2 \psi_1 + i \Hbar c \PD{x}{\psi_2},
\end{aligned}
\end{equation}

The conjugates of these are
\begin{equation}\label{eqn:qmLecture9Problems:380}
\begin{aligned}
i \Hbar \PD{t}{\psi_1^\conj} &= i \Hbar c \PD{x}{\psi_1^\conj} + m c^2 \psi_2^\conj \\
i \Hbar \PD{t}{\psi_2^\conj} &= m c^2 \psi_1^\conj - i \Hbar c \PD{x}{\psi_2^\conj}.
\end{aligned}
\end{equation}
%
This gives
\begin{dmath}\label{eqn:qmLecture9Problems:400}
\begin{aligned}
i \Hbar \PD{t}{\rho}
&=
\lr{ i \Hbar c \PD{x}{\psi_1^\conj} + m c^2 \psi_2^\conj } \psi_1 \\
&+ \psi_1^\conj \lr{ i \Hbar c \PD{x}{\psi_1} - m c^2 \psi_2 } \\
&+ \lr{ m c^2 \psi_1^\conj - i \Hbar c \PD{x}{\psi_2^\conj} } \psi_2 \\
&+ \psi_2^\conj \lr{ -m c^2 \psi_1 - i \Hbar c \PD{x}{\psi_2} }.
\end{aligned}
\end{dmath}
%
All the non-derivative terms cancel leaving
%
\begin{dmath}\label{eqn:qmLecture9Problems:420}
\inv{c} \PD{t}{\rho}
=
\PD{x}{\psi_1^\conj} \psi_1
+ \psi_1^\conj \PD{x}{\psi_1}
- \PD{x}{\psi_2^\conj} \psi_2
- \psi_2^\conj \PD{x}{\psi_2}
=
\PD{x}{}
\lr{
\psi_1^\conj \psi_1 -
\psi_2^\conj \psi_2
}.
\end{dmath}
%
} % answer
