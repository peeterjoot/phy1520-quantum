%
% Copyright � 2015 Peeter Joot.  All Rights Reserved.
% Licenced as described in the file LICENSE under the root directory of this GIT repository.
%
\makeoproblem{Spin-1 rotations.}{gradQuantum:problemSet6:3}{2015 ps6 p3}{
\index{spin one}
\index{rotation}
%\makesubproblem{}{gradQuantum:problemSet6:3a}
%
Consider angular momentum \( j = 1 \).
%
\makesubproblem{}{gradQuantum:problemSet6:3a}
%
Express \( \bra{j = 1, m'} \hatJ_y \ket{ j = 1, m } \) as a \( 3 \times 3 \) matrix.
%
\makesubproblem{}{gradQuantum:problemSet6:3b}
Show that for \( j = 1 \) % , we can replace (units where \( \Hbar = 1 \)),
\begin{dmath}\label{eqn:gradQuantumProblemSet6Problem3:20}
e^{-i \hatJ_y \beta/\Hbar}
=
1 - i \frac{\hatJ_y}{\Hbar} \sin\beta - \frac{\hatJ_y^2}{\Hbar^2} \lr{ 1 - \cos\beta }.
\end{dmath}
} % makeproblem
%
\makeanswer{gradQuantum:problemSet6:3}{
\withproblemsetsParagraph{
\makeSubAnswer{}{gradQuantum:problemSet6:3a}
%
From \citep{sakurai2014modern} (5.41), for \( j' = j = 1 \), the matrix elements for the ladder operators can be summarized as
%
\begin{equation}\label{eqn:gradQuantumProblemSet6Problem3:140}
\bra{j, m'} \hatJ_{\pm} \ket{j, m} = \Hbar \sqrt{(1 \mp m)(2 \pm m)} \delta_{m', m\pm 1}.
\end{dmath}
%
With
%
\begin{equation}\label{eqn:gradQuantumProblemSet6Problem3:160}
\begin{aligned}
\hatJ_{+} &= \hatJ_x + i \hatJ_y \\
\hatJ_{-} &= \hatJ_x - i \hatJ_y,
\end{aligned}
\end{equation}
we have
%
\begin{equation}\label{eqn:gradQuantumProblemSet6Problem3:180}
\hatJ_y = \frac{\hatJ_{+} - \hatJ_{-}}{2i},
\end{dmath}
%
so
%
\begin{equation}\label{eqn:gradQuantumProblemSet6Problem3:200}
\frac{2 i}{\Hbar} \bra{j, m'} \hatJ_y \ket{j, m} = \sqrt{(1 - m)(2 + m)} \delta_{m', m + 1} -\sqrt{(1 + m)(2 - m)} \delta_{m', m - 1}.
\end{dmath}
%
We have nine matrix elements
%
\begin{equation}\label{eqn:gradQuantumProblemSet6Problem3:240}
\begin{aligned}
\frac{2 i}{\Hbar} &\bra{j, 1} \hatJ_y \ket{j, 1} = \\&\qquad \sqrt{(1 - 1)(2 + 1)} \delta_{1, 1 + 1} -\sqrt{(1 + 1)(2 - 1)} \delta_{1, 1 - 1} = 0 \\
\frac{2 i}{\Hbar} &\bra{j, 1} \hatJ_y \ket{j, 0} = \\&\qquad \sqrt{(1 - 0)(2 + 0)} \delta_{1, 0 + 1} -\sqrt{(1 + 0)(2 - 0)} \delta_{1, 0 - 1} = \sqrt{2} \\
\frac{2 i}{\Hbar} &\bra{j, 1} \hatJ_y \ket{j, -1} = \\&\qquad \sqrt{(1 - -1)(2 + -1)} \delta_{1, -1 + 1} -\sqrt{(1 + -1)(2 - -1)} \delta_{1, -1 - 1} = 0 \\
\frac{2 i}{\Hbar} &\bra{j, 0} \hatJ_y \ket{j, 1} = \\&\qquad \sqrt{(1 - 1)(2 + 1)} \delta_{0, 1 + 1} -\sqrt{(1 + 1)(2 - 1)} \delta_{0, 1 - 1} = -\sqrt{2} \\
\frac{2 i}{\Hbar} &\bra{j, 0} \hatJ_y \ket{j, 0} = \\&\qquad \sqrt{(1 - 0)(2 + 0)} \delta_{0, 0 + 1} -\sqrt{(1 + 0)(2 - 0)} \delta_{0, 0 - 1} = 0 \\
\frac{2 i}{\Hbar} &\bra{j, 0} \hatJ_y \ket{j, -1} = \\&\qquad \sqrt{(1 - -1)(2 + -1)} \delta_{0, -1 + 1} -\sqrt{(1 + -1)(2 - -1)} \delta_{0, -1 - 1} = \sqrt{2} \\
\frac{2 i}{\Hbar} &\bra{j, -1} \hatJ_y \ket{j, 1} = \\&\qquad \sqrt{(1 - 1)(2 + 1)} \delta_{-1, 1 + 1} -\sqrt{(1 + 1)(2 - 1)} \delta_{-1, 1 - 1} = 0 \\
\frac{2 i}{\Hbar} &\bra{j, -1} \hatJ_y \ket{j, 0} = \\&\qquad \sqrt{(1 - 0)(2 + 0)} \delta_{-1, 0 + 1} -\sqrt{(1 + 0)(2 - 0)} \delta_{-1, 0 - 1} = -\sqrt{2} \\
\frac{2 i}{\Hbar} &\bra{j, -1} \hatJ_y \ket{j, -1} = \\&\qquad \sqrt{(1 - -1)(2 + -1)} \delta_{-1, -1 + 1} -\sqrt{(1 + -1)(2 - -1)} \delta_{-1, -1 - 1} = 0.
\end{aligned}
\end{equation}
%
Put into matrix form, that is
%
\begin{dmath}\label{eqn:gradQuantumProblemSet6Problem3:260}
\bra{j, m'} \hatJ_y \ket{j, m}
=
\frac{\Hbar}{2 i}
\begin{bmatrix}
0 & \sqrt{2} & 0 \\
-\sqrt{2} & 0 & \sqrt{2} \\
0 & -\sqrt{2} & 0
\end{bmatrix},
\end{dmath}
%
or
%\begin{dmath}\label{eqn:gradQuantumProblemSet6Problem3:40}
\boxedEquation{eqn:gradQuantumProblemSet6Problem3:40}{
\hatJ_y
=
\frac{\Hbar i}{\sqrt{2}}
\begin{bmatrix}
0 & -1 &  0 \\
1 &  0 & -1 \\
0 &  1 &  0 \\
\end{bmatrix}.
}
%\end{dmath}
%
\makeSubAnswer{}{gradQuantum:problemSet6:3b}
%
The square of the matrix representation of \( \hatJ_y \) of \cref{eqn:gradQuantumProblemSet6Problem3:40} is
%
\begin{dmath}\label{eqn:gradQuantumProblemSet6Problem3:60}
\hatJ_y^2
=
-\frac{\Hbar^2}{2}
\begin{bmatrix}
0 & -1 &  0 \\
1 &  0 & -1 \\
0 &  1 &  0 \\
\end{bmatrix}
\begin{bmatrix}
0 & -1 &  0 \\
1 &  0 & -1 \\
0 &  1 &  0 \\
\end{bmatrix}
=
-\frac{\Hbar^2}{2}
\begin{bmatrix}
-1 & 0 & 1 \\
0 & -2 & 0 \\
1 & 0 & -1
\end{bmatrix},
\end{dmath}
%
and the cube is
\begin{dmath}\label{eqn:gradQuantumProblemSet6Problem3:80}
\hatJ_y^3
=
-\frac{\Hbar^2}{2}
\frac{\Hbar i}{\sqrt{2}}
\begin{bmatrix}
-1 & 0 & 1 \\
0 & -2 & 0 \\
1 & 0 & -1
\end{bmatrix}
\begin{bmatrix}
0 & -1 &  0 \\
1 &  0 & -1 \\
0 &  1 &  0 \\
\end{bmatrix}
=
-\frac{i\Hbar^3}{2 \sqrt{2}}
\begin{bmatrix}
0 & 2 & 0 \\
-2 & 0 & 2 \\
0 & -2 & 0
\end{bmatrix}
=
\Hbar^2 \frac{i \Hbar}{\sqrt{2}}
\begin{bmatrix}
0 & -1 &  0 \\
1 &  0 & -1 \\
0 &  1 &  0 \\
\end{bmatrix}
=
\Hbar^2 \hatJ_y.
\end{dmath}
%
This proves (5.55) from the text
\begin{equation}\label{eqn:gradQuantumProblemSet6Problem3:100}
\frac{\hatJ_y^3}{\Hbar^3} = \frac{\hatJ_y}{\Hbar}.
\end{dmath}
%
In particular
\begin{dmath}\label{eqn:gradQuantumProblemSet6Problem3:220}
\begin{aligned}
\lr{ \frac{\hatJ_y}{\Hbar} }^0 &= 1 \\
\lr{ \frac{\hatJ_y}{\Hbar} }^1 &= \lr{ \frac{\hatJ_y}{\Hbar} }^1 \\
\lr{ \frac{\hatJ_y}{\Hbar} }^2 &= \lr{ \frac{\hatJ_y}{\Hbar} }^2 \\
\lr{ \frac{\hatJ_y}{\Hbar} }^3 &=      \frac{\hatJ_y}{\Hbar} \\
\lr{ \frac{\hatJ_y}{\Hbar} }^4 &= \lr{ \frac{\hatJ_y}{\Hbar}}^2 \\
\lr{ \frac{\hatJ_y}{\Hbar} }^5 &=      \frac{\hatJ_y}{\Hbar} \\
& \vdots
\end{aligned}
\end{dmath}

Expanding the exponential in series we have
\begin{dmath}\label{eqn:gradQuantumProblemSet6Problem3:120}
\begin{aligned}
e^{-i \beta \hatJ_y/\Hbar}
&=
1
- \frac{i \beta \hatJ_y/\Hbar}{1!}
+ \frac{\lr{-i \beta \hatJ_y/\Hbar}^2}{2!}
+ \frac{\lr{-i \beta \hatJ_y/\Hbar}^3}{3!} \\
&\quad + \frac{\lr{-i \beta \hatJ_y/\Hbar}^4}{4!}
+ \frac{\lr{-i \beta \hatJ_y/\Hbar}^5}{5!}
+ \frac{\lr{-i \beta \hatJ_y/\Hbar}^6}{6!}
+ \cdots \\
&=
1
- i \frac{\beta \hatJ_y/\Hbar}{1!}
- \frac{\lr{ \beta \hatJ_y/\Hbar}^2}{2!}
+ i \frac{\lr{ \beta \hatJ_y/\Hbar}^3}{3!} \\
&\qquad + \frac{\lr{ \beta \hatJ_y/\Hbar}^4}{4!}
- i \frac{\lr{ \beta \hatJ_y/\Hbar}^5}{5!}
- \frac{\lr{ \beta \hatJ_y/\Hbar}^6}{6!}
+ \cdots \\
&=
1
+ i \frac{\hatJ_y}{\Hbar} \lr{ - \beta + \frac{\beta^3}{3!} - \frac{\beta^5}{5!} + \cdots } \\
&+ \frac{\hatJ_y^2}{\Hbar^2} \lr{ - \frac{\beta^2}{2!} + \frac{\beta^4}{4!} - \frac{\beta^6}{6!} + \cdots } \\
&=
1 - i \frac{\hatJ_y}{\Hbar} \sin\beta + \frac{\hatJ_y^2}{\Hbar^2} \lr{ -1 + \cos\beta } \qedmarker
\end{aligned}
\end{dmath}
Note that \( J_x^3/\Hbar^3 = J_x/\Hbar \) and \( J_z^3/\Hbar^3 = J_z/\Hbar \) too, so this relation generalizes to the other spin one operators as well.
%which proves \cref{eqn:gradQuantumProblemSet6Problem3:20}.
}
}
