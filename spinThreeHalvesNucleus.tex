%
% Copyright � 2015 Peeter Joot.  All Rights Reserved.
% Licenced as described in the file LICENSE under the root directory of this GIT repository.
%
%{
%\input{../blogpost.tex}
%\renewcommand{\basename}{spinThreeHalvesNucleus}
%\renewcommand{\dirname}{notes/phy1520/}
%%\newcommand{\dateintitle}{}
%%\newcommand{\keywords}{}
%
%\input{../peeter_prologue_print2.tex}
%
%\usepackage{peeters_layout_exercise}
%\usepackage{peeters_braket}
%\usepackage{peeters_figures}
%
%\beginArtNoToc
%
%\generatetitle{Spin three halves spin interaction}
%%\chapter{Spin three halves spin interaction}
%%\label{chap:spinThreeHalvesNucleus}
%
%
\makeoproblem{Spin three halves spin interaction.}{problem:spinThreeHalvesNucleus:1}{\citep{sakurai2014modern} pr. 3.33}{
\index{spin three halves}
\index{applied electric field}

A spin \( 3/2 \) nucleus subjected to an external electric field has an interaction Hamiltonian of the form
%
\begin{dmath}\label{eqn:spinThreeHalvesNucleus:20}
H = \frac{e Q}{2 s(s-1) \Hbar^2} \lr{
\lr{\PDSq{x}{\phi}}_0 S_x^2
+\lr{\PDSq{y}{\phi}}_0 S_y^2
+\lr{\PDSq{z}{\phi}}_0 S_z^2
}.
\end{dmath}
%
\makesubproblem{}{problem:spinThreeHalvesNucleus:1:a}
%
Show that the interaction energy can be written as
%
\begin{dmath}\label{eqn:spinThreeHalvesNucleus:40}
A(3 S_z^2 - \BS^2) + B(S_{+}^2 + S_{-}^2).
\end{dmath}
%
\makesubproblem{}{problem:spinThreeHalvesNucleus:1:b}
%
Find the energy eigenvalues for such a Hamiltonian.

} % problem
%
\makeanswer{problem:spinThreeHalvesNucleus:1}{
%
\makeSubAnswer{}{problem:spinThreeHalvesNucleus:1:a}
%
Reordering
\begin{equation}\label{eqn:spinThreeHalvesNucleus:60}
\begin{aligned}
S_{+} &= S_x + i S_y \\
S_{-} &= S_x - i S_y,
\end{aligned}
\end{equation}

gives
\begin{equation}\label{eqn:spinThreeHalvesNucleus:80}
\begin{aligned}
S_x &= \inv{2} \lr{ S_{+} + S_{-} } \\
S_y &= \inv{2i} \lr{ S_{+} - S_{-} }.
\end{aligned}
\end{equation}
%
The squared spin operators are
\begin{dmath}\label{eqn:spinThreeHalvesNucleus:100}
S_x^2
=
\inv{4} \lr{ S_{+}^2 + S_{-}^2 + S_{+} S_{-} + S_{-} S_{+} }
=
\inv{4} \lr{ S_{+}^2 + S_{-}^2 + 2( S_x^2 + S_y^2 ) }
=
\inv{4} \lr{ S_{+}^2 + S_{-}^2 + 2( \BS^2 - S_z^2 ) },
\end{dmath}
%
\begin{dmath}\label{eqn:spinThreeHalvesNucleus:120}
S_y^2
=
-\inv{4} \lr{ S_{+}^2 + S_{-}^2 - S_{+} S_{-} - S_{-} S_{+} }
=
-\inv{4} \lr{ S_{+}^2 + S_{-}^2 - 2( S_x^2 + S_y^2 ) }
=
-\inv{4} \lr{ S_{+}^2 + S_{-}^2 - 2( \BS^2 - S_z^2 ) }.
\end{dmath}
%
This gives
\begin{dmath}\label{eqn:spinThreeHalvesNucleus:140}
\begin{aligned}
H
=
\frac{e Q}{2 s(s-1) \Hbar^2}
\biglr{ &
\inv{4} \lr{\PDSq{x}{\phi}}_0 \lr{ S_{+}^2 + S_{-}^2 + 2( \BS^2 - S_z^2 ) } \\
&
-\lr{\PDSq{y}{\phi}}_0 \lr{ S_{+}^2 + S_{-}^2 - 2( \BS^2 - S_z^2 ) } \\
&+\lr{\PDSq{z}{\phi}}_0 S_z^2
} \\
=
\frac{e Q}{2 s(s-1) \Hbar^2}
\biglr{ &
\inv{4} \lr{  \lr{\PDSq{x}{\phi}}_0 -\lr{\PDSq{y}{\phi}}_0 } \lr{ S_{+}^2 + S_{-}^2 } \\
&+
\inv{2} \lr{
\lr{\PDSq{x}{\phi}}_0 + \lr{\PDSq{y}{\phi}}_0
} \BS^2  \\
&+
\lr{
\lr{\PDSq{z}{\phi}}_0
-
\inv{2} \lr{\PDSq{x}{\phi}}_0 - \inv{2} \lr{\PDSq{y}{\phi}}_0
} S_z^2
}.
\end{aligned}
\end{dmath}
%
For a static electric field we have
%
\begin{dmath}\label{eqn:spinThreeHalvesNucleus:160}
\spacegrad^2 \phi = -\frac{\rho}{\epsilon_0},
\end{dmath}
%
but are evaluating it at a point away from the generating charge distribution, so \( \spacegrad^2 \phi = 0 \) at that point.  This gives
%
\begin{dmath}\label{eqn:spinThreeHalvesNucleus:180}
\begin{aligned}
H
=
\frac{e Q}{4 s(s-1) \Hbar^2}
\biglr{ &
\inv{2} \lr{  \lr{\PDSq{x}{\phi}}_0 -\lr{\PDSq{y}{\phi}}_0
} \lr{ S_{+}^2 + S_{-}^2 } \\
&+
\lr{
\lr{\PDSq{x}{\phi}}_0 + \lr{\PDSq{y}{\phi}}_0
} (\BS^2 - 3 S_z^2)
},
\end{aligned}
\end{dmath}

so
\begin{dmath}\label{eqn:spinThreeHalvesNucleus:200}
A =
-\frac{e Q}{4 s(s-1) \Hbar^2}  \lr{
\lr{\PDSq{x}{\phi}}_0 + \lr{\PDSq{y}{\phi}}_0
}
\end{dmath}
\begin{dmath}\label{eqn:spinThreeHalvesNucleus:220}
B =
\frac{e Q}{8 s(s-1) \Hbar^2}
\lr{ \lr{\PDSq{x}{\phi}}_0 - \lr{\PDSq{y}{\phi}}_0 }.
\end{dmath}
%
\makeSubAnswer{}{problem:spinThreeHalvesNucleus:1:b}
%
Using \nbref{sakuraiProblem3.33.nb}, matrix representations for the spin three halves operators and the Hamiltonian were constructed with respect to the basis \( \setlr{ \ket{3/2}, \ket{1/2}, \ket{-1/2}, \ket{-3/2} } \)
%
\begin{equation}\label{eqn:spinThreeHalvesNucleus:240}
\begin{aligned}
S_{+} &=
\Hbar
\begin{bmatrix}
 0 & \sqrt{3} & 0 & 0 \\
 0 & 0 & 2 & 0 \\
 0 & 0 & 0 & \sqrt{3} \\
 0 & 0 & 0 & 0 \\
\end{bmatrix} \\
S_{-} &=
\Hbar
\begin{bmatrix}
 0 & 0 & 0 & 0 \\
 \sqrt{3} & 0 & 0 & 0 \\
 0 & 2 & 0 & 0 \\
 0 & 0 & \sqrt{3}  & 0 \\
\end{bmatrix} \\
S_x &=
\Hbar
\begin{bmatrix}
 0 & \sqrt{3}/2 & 0 & 0 \\
 \sqrt{3}/2 & 0 & 1 & 0 \\
 0 & 1  & 0 & \sqrt{3}/2 \\
 0 & 0 & \sqrt{3}/2 & 0 \\
\end{bmatrix} \\
S_y &=
i \Hbar
\begin{bmatrix}
 0 & -\ifrac{\sqrt{3}}{2} & 0 & 0 \\
 \ifrac{\sqrt{3}}{2} & 0 & -1 & 0 \\
 0 & 1 & 0 & -\ifrac{\sqrt{3}}{2} \\
 0 & 0 & \ifrac{\sqrt{3}}{2} & 0 \\
\end{bmatrix} \\
S_z &=
\frac{\Hbar}{2}
\begin{bmatrix}
 3 & 0 & 0 & 0 \\
 0 & 1 & 0 & 0 \\
 0 & 0 & -1 & 0 \\
 0 & 0 & 0 & -3 \\
\end{bmatrix} \\
H &=
\begin{bmatrix}
 3 A & 0 & 2 \sqrt{3} B & 0 \\
 0 & -3 A & 0 & 2 \sqrt{3} B \\
 2 \sqrt{3} B & 0 & -3 A & 0 \\
 0 & 2 \sqrt{3} B & 0 & 3 A \\
\end{bmatrix}.
\end{aligned}
\end{equation}
%
The energy eigenvalues are found to be
%
\begin{equation}\label{eqn:spinThreeHalvesNucleus:260}
E = \pm \Hbar^2 \sqrt{9 A^2 + 12 B^2 },
\end{equation}
%
with two fold degeneracies for each eigenvalue.
} % answer

%}
%\EndArticle
