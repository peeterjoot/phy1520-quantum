%
% Copyright � 2015 Peeter Joot.  All Rights Reserved.
% Licenced as described in the file LICENSE under the root directory of this GIT repository.
%
%\input{../blogpost.tex}
%\renewcommand{\basename}{dynamicsNonHermitian}
%\renewcommand{\dirname}{notes/phy1520/}
%\newcommand{\dateintitle}{}
%\newcommand{\keywords}{}

%\input{../peeter_prologue_print2.tex}
%
%\usepackage{peeters_layout_exercise}
%\usepackage{peeters_braket}
%\usepackage{peeters_figures}
%\usepackage{peeters_qed}
%
%\beginArtNoToc

%\generatetitle{Dynamics of non-Hermitian Hamiltonian}
%\chapter{Dynamics of non-Hermitian Hamiltonian}
%\label{chap:dynamicsNonHermitian}
%
\makeoproblem{Dynamics of non-Hermitian Hamiltonian.}{problem:dynamicsNonHermitian:2.2}{\citep{sakurai2014modern} pr. 2.2}{
\index{Hamiltonian!non-Hermitian}
Revisiting an earlier Hamiltonian, but assuming it was entered incorrectly as
%
\begin{dmath}\label{eqn:dynamicsNonHermitian:20}
H = H_{11} \ket{1}\bra{1}
  + H_{22} \ket{2}\bra{2}
  + H_{12} \ket{1}\bra{2}.
\end{dmath}
%
What principle is now violated?  Illustrate your point explicitly by attempting to solve the most general time-dependent problem using an illegal Hamiltonian of this kind.  You may assume that \( H_{11} = H_{22} \) for simplicity.
} % problem
%
\makeanswer{problem:dynamicsNonHermitian:2.2}{
%
In matrix form this Hamiltonian is
%
\begin{dmath}\label{eqn:dynamicsNonHermitian:40}
H =
\begin{bmatrix}
\bra{1} H \ket{1} & \bra{1} H \ket{2} \\
\bra{2} H \ket{1} & \bra{2} H \ket{2} \\
\end{bmatrix}
=
\begin{bmatrix}
H_{11} & H_{12} \\
0      & H_{22} \\
\end{bmatrix}.
\end{dmath}
%
This is not a Hermitian operator.  What is the physical implication of this non-Hermicity?  Consider the simpler case where \( H_{11} = H_{22} \).  Such a Hamiltonian has the form
%
\begin{dmath}\label{eqn:dynamicsNonHermitian:60}
H =
\begin{bmatrix}
a & b \\
0 & a
\end{bmatrix}.
\end{dmath}
%
This has only one unique eigenvector ( \( (1,0) \), but we can still solve the time evolution equation
%
\begin{equation}\label{eqn:dynamicsNonHermitian:80}
i \Hbar \PD{t}{U} = H U,
\end{equation}
%
since for constant \( H \), we have
%
\begin{equation}\label{eqn:dynamicsNonHermitian:100}
U = e^{-i H t/\Hbar}.
\end{equation}
%
To exponentiate, note that we have
%
\begin{dmath}\label{eqn:dynamicsNonHermitian:120}
{\begin{bmatrix}
a & b \\
0 & a
\end{bmatrix}}^n
=
\begin{bmatrix}
a^n & n a^{n-1} b \\
0 & a^n
\end{bmatrix}.
\end{dmath}
%
To prove the induction, the \( n = 2 \) case follows easily
%
\begin{dmath}\label{eqn:dynamicsNonHermitian:140}
\begin{bmatrix}
a & b \\
0 & a
\end{bmatrix}
\begin{bmatrix}
a & b \\
0 & a
\end{bmatrix}
=
\begin{bmatrix}
a^2 & 2 a b \\
0 & a^2
\end{bmatrix},
\end{dmath}
%
as does the general case
%
\begin{dmath}\label{eqn:dynamicsNonHermitian:160}
\begin{bmatrix}
a^n & n a^{n-1} b \\
0 & a^n
\end{bmatrix}
\begin{bmatrix}
a & b \\
0 & a
\end{bmatrix}
=
\begin{bmatrix}
a^{n+1} & (n +1 ) a^{n} b \\
0 & a^{n+1}
\end{bmatrix}.
\end{dmath}
%
The exponential sum is thus
\begin{dmath}\label{eqn:dynamicsNonHermitian:180}
e^{H \tau}
=
\begin{bmatrix}
e^{a \tau} & 0 + \frac{b \tau}{1!} + \frac{2 a b \tau^2}{2!} + \frac{3 a^2 b \tau^3}{3!} + \cdots \\
0 & e^{a \tau}
\end{bmatrix}.
\end{dmath}
%
That sum simplifies to
%
\begin{dmath}\label{eqn:dynamicsNonHermitian:200}
\frac{b \tau}{0!} + \frac{a b \tau^2}{1!} + \frac{a^2 b \tau^3}{2!} + \cdots \\
=
b \tau \lr{ 1 + \frac{a \tau}{1!} + \frac{(a \tau)^2}{2!} + \cdots }
=
b \tau e^{a \tau}.
\end{dmath}
%
The exponential is thus
\begin{dmath}\label{eqn:dynamicsNonHermitian:220}
e^{H \tau} =
\begin{bmatrix}
e^{a\tau} & b \tau e^{a\tau} \\
0 & e^{a\tau}
\end{bmatrix}
=
\begin{bmatrix}
1 & b \tau \\
0 & 1
\end{bmatrix}
e^{a\tau}.
\end{dmath}
%
In particular
%
\begin{dmath}\label{eqn:dynamicsNonHermitian:240}
U = e^{-i H t/\Hbar} =
\begin{bmatrix}
1 & -i b t/\Hbar \\
0 & 1
\end{bmatrix}
e^{-i a t /\Hbar }.
\end{dmath}
%
We can verify that this is a solution to \cref{eqn:dynamicsNonHermitian:80}.  The left hand side is
%
\begin{dmath}\label{eqn:dynamicsNonHermitian:260}
i \Hbar \PD{t}{U}
=
i \Hbar
\begin{bmatrix}
-i a/\Hbar & -i b /\Hbar + (-i b t/\Hbar)(-i a/\Hbar) \\
0 & -i a /\Hbar
\end{bmatrix}
e^{-i a t /\Hbar }
=
\begin{bmatrix}
a & b - i a b t/\Hbar \\
0 & a
\end{bmatrix}
e^{-i a t /\Hbar },
\end{dmath}
%
and for the right hand side
\begin{dmath}\label{eqn:dynamicsNonHermitian:280}
H U
=
\begin{bmatrix}
a & b \\
0 & a
\end{bmatrix}
\begin{bmatrix}
1 & -i b t/\Hbar \\
0 & 1
\end{bmatrix}
e^{-i a t /\Hbar }
=
\begin{bmatrix}
a & b - i a b t/\Hbar \\
0 & a
\end{bmatrix}
e^{-i a t /\Hbar }
=
i \Hbar \PD{t}{U}. \qedmarker
\end{dmath}

While the Schr\"{o}dinger is satisfied, we don't have the unitary inversion physical property that is desired for the time evolution operator \( U \).  Namely
%
\begin{dmath}\label{eqn:dynamicsNonHermitian:300}
U^\dagger U
=
\begin{bmatrix}
1 & 0 \\
i b t/\Hbar & 1
\end{bmatrix}
e^{i a t /\Hbar }
\begin{bmatrix}
1 & -i b t/\Hbar \\
0 & 1
\end{bmatrix}
e^{-i a t /\Hbar }
=
\begin{bmatrix}
1 & -i b t/\Hbar \\
i b t/\Hbar & (b t)^2/\Hbar^2
\end{bmatrix}
\ne I.
\end{dmath}
%
We required \( U^\dagger U = I \) for the time evolution operator, but don't have that property for this non-Hermitian Hamiltonian.
} % answer

%\EndArticle
