%
% Copyright © 2016 Peeter Joot.  All Rights Reserved.
% Licenced as described in the file LICENSE under the root directory of this GIT repository.
%

\paragraph{Mostly background.}
% (really for myself).  Grading can probably skip to around \cref{eqn:gradQuantumProblemSet4Problem3:560}

We talked about diagonalization of the Dirac Hamiltonian by introducing a rotation, and then figuring out the rotation angle required.

To understand the form of the eigenkets for particles and antiparticles in both forward and backwards moving configurations, lets do this diagonalization explicitly for both forwards and backwards solutions.

For the forward solution, given \( \Psi = \Psi_0 e^{i(k x - E t/\Hbar) } \), the Dirac equation is
%
\begin{equation}\label{eqn:gradQuantumProblemSet4Problem3:20}
i \Hbar (-i E/\Hbar) \Psi =
\begin{bmatrix}
-i \Hbar (i k) + V_0 & m c^2 \\
m c^2 & i \Hbar (i k)
\end{bmatrix}
\Psi,
\end{equation}
or
\begin{equation}\label{eqn:gradQuantumProblemSet4Problem3:40}
\begin{bmatrix}
E - V_0 & 0 \\
0 & E - V_0
\end{bmatrix}
\Psi
=
\begin{bmatrix}
\Hbar k & m c^2 \\
m c^2 &  - \Hbar k
\end{bmatrix}
\Psi.
\end{equation}
%
Similarly, for the backwards moving wave \( \Psi = e^{i(-k x - E t/\Hbar)} \), we have
%
\begin{equation}\label{eqn:gradQuantumProblemSet4Problem3:60}
\begin{bmatrix}
E - V_0 & 0 \\
0 & E - V_0
\end{bmatrix}
\Psi
=
\begin{bmatrix}
-\Hbar k & m c^2 \\
m c^2 & \Hbar k
\end{bmatrix}
\Psi.
\end{equation}
%
Working with \( \Hbar = c = 1 \) temporarily, we want to compute the eigensolutions for the matrix
%
\begin{equation}\label{eqn:gradQuantumProblemSet4Problem3:80}
H_{\pm k}
=
\begin{bmatrix}
\pm k & m \\
m & \mp k
\end{bmatrix}.
\end{equation}
%
The eigenvalues \( \epsilon \) of both are the same
%
\begin{equation}\label{eqn:gradQuantumProblemSet4Problem3:100}
\begin{aligned}
0
&=
\Abs{ H_{\pm k} - \epsilon }
\\ &=
(\pm k - \epsilon)(\mp k - \epsilon) - m^2
\\ &=
(\mp k + \epsilon)(\pm k + \epsilon) - m^2
\\ &=
\epsilon^2 - k^2 - m^2,
\end{aligned}
\end{equation}
%
or
%
\begin{equation}\label{eqn:gradQuantumProblemSet4Problem3:120}
\epsilon = \pm \sqrt{k^2 + m^2}.
\end{equation}
%
\paragraph{Eigenkets for \( H_k \)}

For the positive(negative) energy eigenvalues, we have
%
\begin{equation}\label{eqn:gradQuantumProblemSet4Problem3:140}
0
=
\begin{bmatrix}
k \mp \epsilon & m \\
m & -k \mp \epsilon
\end{bmatrix}
\begin{bmatrix}
a \\
b
\end{bmatrix},
\end{equation}
%
for some \( a, b\).  That is
%
\begin{equation}\label{eqn:gradQuantumProblemSet4Problem3:160}
(k \mp \epsilon) a + m b = 0,
\end{equation}
%
or
%
\begin{equation}\label{eqn:gradQuantumProblemSet4Problem3:180}
\begin{bmatrix}
a \\
b
\end{bmatrix}
\propto
\begin{bmatrix}
- m \\
k \mp \epsilon
\end{bmatrix}.
\end{equation}
%
For the normalization note that
%
\begin{equation}\label{eqn:gradQuantumProblemSet4Problem3:200}
\begin{aligned}
m^2 + \lr{ k \mp \epsilon }^2
&=
m^2 + k^2 + \epsilon^2 \mp 2 k \epsilon
\\ &=
2 \epsilon^2 \mp 2 k \epsilon
\\ &=
2 \epsilon (\epsilon \mp k),
\end{aligned}
\end{equation}
%
so the normalized kets are
%
\begin{equation}\label{eqn:gradQuantumProblemSet4Problem3:220}
\ket{k; \pm\epsilon} =
\inv{\sqrt{2 \epsilon(\epsilon \mp k)}}
\begin{bmatrix}
\pm m \\
\epsilon \mp k
\end{bmatrix}.
\end{equation}
%
\paragraph{Eigenkets for \( H_{-k} \)}
This time, for the positive(negative) energy eigenvalues, we have
%
\begin{equation}\label{eqn:gradQuantumProblemSet4Problem3:141}
0
=
\begin{bmatrix}
-k \mp \epsilon & m \\
m & k \mp \epsilon
\end{bmatrix}
\begin{bmatrix}
a \\
b
\end{bmatrix},
\end{equation}
%
for some \( a, b\).  That is
%
\begin{equation}\label{eqn:gradQuantumProblemSet4Problem3:161}
(-k \mp \epsilon) a + m b = 0,
\end{equation}
%
or
%
\begin{equation}\label{eqn:gradQuantumProblemSet4Problem3:181}
\begin{aligned}
\begin{bmatrix}
a \\
b
\end{bmatrix}
&\propto
\begin{bmatrix}
- m \\
-k \mp \epsilon
\end{bmatrix}
\\ &\propto
\begin{bmatrix}
m \\
k \pm \epsilon
\end{bmatrix}.
\end{aligned}
\end{equation}
%
For the normalization note that
%
\begin{equation}\label{eqn:gradQuantumProblemSet4Problem3:201}
\begin{aligned}
m^2 + \lr{ k \pm \epsilon }^2
&=
m^2 + k^2 + \epsilon^2 \pm 2 k \epsilon
\\ &=
2 \epsilon^2 \pm 2 k \epsilon
\\ &=
2 \epsilon (\epsilon \pm k),
\end{aligned}
\end{equation}
%
so the normalized kets are
%
\begin{equation}\label{eqn:gradQuantumProblemSet4Problem3:221}
\ket{-k; \pm\epsilon} =
\inv{\sqrt{2 \epsilon(\epsilon \mp k)}}
\begin{bmatrix}
\pm m \\
\epsilon \pm k
\end{bmatrix}.
\end{equation}
%
\paragraph{Simplification of the rotation matrices}
The eigenvalue equations have the form
%
\begin{equation}\label{eqn:gradQuantumProblemSet4Problem3:240}
H
\begin{bmatrix}
\ket{+} & \ket{-}
\end{bmatrix}
=
\begin{bmatrix}
\ket{+} & \ket{-}
\end{bmatrix}
\begin{bmatrix}
\epsilon & 0 \\
0 & -\epsilon
\end{bmatrix}.
\end{equation}
With \( R = \begin{bmatrix} \ket{+} & \ket{-} \end{bmatrix} \), this has the form \( H R = R \Omega \), or \( H = R \Omega R^{-1} \).  The rotation matrices have been found to be
%
\begin{equation}\label{eqn:gradQuantumProblemSet4Problem3:260}
R_{+k}
=
\inv{\sqrt{2\epsilon}}
\begin{bmatrix}
\frac{m}{\sqrt{\epsilon - k}} & \frac{-m}{\sqrt{\epsilon + k}} \\
\sqrt{\epsilon - k} & \sqrt{\epsilon + k}
\end{bmatrix},
\end{equation}
%
and
\begin{equation}\label{eqn:gradQuantumProblemSet4Problem3:280}
R_{-k}
=
\inv{\sqrt{2\epsilon}}
\begin{bmatrix}
\frac{m}{\sqrt{\epsilon + k}} & \frac{-m}{\sqrt{\epsilon - k}} \\
\sqrt{\epsilon + k} & \sqrt{\epsilon - k}
\end{bmatrix}.
\end{equation}
%
These don't look very much like rotation matrices as is, but not all the terms are independent.  Writing
%
\begin{equation}\label{eqn:gradQuantumProblemSet4Problem3:300}
\begin{aligned}
a &= \frac{m}{\sqrt{2\epsilon(\epsilon - k)}} \\
b &= \frac{m}{\sqrt{2\epsilon(\epsilon + k)}} \\
\alpha &= \sqrt{\frac{\epsilon + k}{2 \epsilon}} \\
\beta &= \sqrt{\frac{\epsilon - k}{2 \epsilon}},
\end{aligned}
\end{equation}
%
the rotation matrices are
%
\begin{equation}\label{eqn:gradQuantumProblemSet4Problem3:320}
R_{+k}
=
\begin{bmatrix}
a & - b \\
\beta & \alpha
\end{bmatrix},
\end{equation}
%
and
\begin{equation}\label{eqn:gradQuantumProblemSet4Problem3:340}
R_{-k}
=
\begin{bmatrix}
b & -a \\
\alpha & \beta
\end{bmatrix}.
\end{equation}
%
Note that
\begin{equation}\label{eqn:gradQuantumProblemSet4Problem3:360}
\begin{aligned}
\frac{a}{\alpha} &= \frac{b}{\beta} \\
&= \frac{m}{\sqrt{\epsilon^2 - k^2}}  \\
&= \frac{m}{\sqrt{m^2}}  \\
&= 1.
\end{aligned}
\end{equation}
%
So
\begin{equation}\label{eqn:gradQuantumProblemSet4Problem3:380}
R_{+k}
=
\begin{bmatrix}
a & - b \\
b & a
\end{bmatrix},
\end{equation}
%
and
\begin{equation}\label{eqn:gradQuantumProblemSet4Problem3:400}
R_{-k}
=
\begin{bmatrix}
b & -a \\
a & b
\end{bmatrix}.
\end{equation}
%
These are expected to have a unit determinant, which is verified easily
%
\begin{equation}\label{eqn:gradQuantumProblemSet4Problem3:420}
\begin{aligned}
a^2 + b^2
&=
\frac{m^2}{2 \epsilon}
\lr{
\frac{1}{\epsilon + k}
+
\frac{1}{\epsilon - k}
}
\\ &=
\frac{m^2}{2 \epsilon}
\frac{2 \epsilon}{\epsilon_2 - k^2}
\\ &= 1.
\end{aligned}
\end{equation}
%
We see that both sets of matrices invert by transposition, so we are free to make a trigonometric identification
%
\begin{equation}\label{eqn:gradQuantumProblemSet4Problem3:440}
a = \cos\theta
=
\frac{m}{\sqrt{2\epsilon(\epsilon - k)}}
\end{equation}
%
\begin{equation}\label{eqn:gradQuantumProblemSet4Problem3:460}
b = \sin\theta
=
\frac{m}{\sqrt{2\epsilon(\epsilon + k)}}.
\end{equation}
%
Using the double angle formulation used in class these are
%
\begin{equation}\label{eqn:gradQuantumProblemSet4Problem3:480}
\begin{aligned}
\cos(2 \theta)
&=
\frac{m^2}{2\epsilon(\epsilon - k)} -
\frac{m^2}{2\epsilon(\epsilon + k)}
\\ &=
\frac{m^2}{2 \epsilon} \lr{ \inv{\epsilon - k} - \inv{\epsilon + k} }
\\ &=
\frac{2 k m^2}{2 \epsilon (\epsilon^2 - k^2)}
\\ &=
\frac{k}{\epsilon},
\end{aligned}
\end{equation}
%
and
\begin{equation}\label{eqn:gradQuantumProblemSet4Problem3:500}
\begin{aligned}
\sin(2 \theta)
&=
2 \frac{m^2}{2 \epsilon} \inv{\sqrt{\epsilon^2 - k^2}}
\\ &=
\frac{m}{\epsilon}.
\end{aligned}
\end{equation}
%
Putting back in the \( \Hbar \), and \( c\) factors, the diagonalizing transformation is now fully specified
%
\begin{equation}\label{eqn:gradQuantumProblemSet4Problem3:520}
\begin{aligned}
H_{\pm k} &=
\begin{bmatrix}
\pm \Hbar k c & m c^2 \\
m c^2 & \mp \Hbar k c
\end{bmatrix}
=
R_{\pm k} \Omega R_{\pm k}^\T \\
R_{+ k} &=
\begin{bmatrix}
\cos\theta & -\sin\theta  \\
\sin\theta & \cos\theta
\end{bmatrix}
=
\begin{bmatrix}
\ket{+k;+\epsilon} &
\ket{+k;-\epsilon}
\end{bmatrix}
\\
R_{- k} &=
\begin{bmatrix}
\sin\theta & -\cos\theta  \\
\cos\theta & \sin\theta
\end{bmatrix}
=
\begin{bmatrix}
\ket{-k;+\epsilon} &
\ket{-k;-\epsilon}
\end{bmatrix}
\\
\tan(2 \theta) &= \frac{m c}{\Hbar \Abs{k}} \\
\Omega &=
\begin{bmatrix}
\epsilon & 0 \\
0 & -\epsilon
\end{bmatrix} \\
\epsilon &= \sqrt{(\Hbar k c)^2 + (m c^2)^2}.
\end{aligned}
\end{equation}
%
The functions \( \Psi = \ket{\pm k; \pm \epsilon} e^{\pm i k x - i E t/\Hbar} \) are eigenfunctions of the Hamiltonian.  For example, for a non-antiparticle forward moving state
%
\begin{equation}\label{eqn:gradQuantumProblemSet4Problem3:540}
\begin{aligned}
(E - V_0) \ket{k; +\epsilon}
&=
H_k \ket{k; +\epsilon}
\\ &=
\begin{bmatrix}
\ket{k;+\epsilon} &
\ket{k;-\epsilon}
\end{bmatrix}
\begin{bmatrix}
\epsilon & 0 \\
0 & -\epsilon
\end{bmatrix}
\begin{bmatrix}
\bra{k;+\epsilon} \\
\bra{k;-\epsilon}
\end{bmatrix}
\ket{k; \epsilon}
\\ &=
\begin{bmatrix}
\ket{k;+\epsilon} &
\ket{k;-\epsilon}
\end{bmatrix}
\begin{bmatrix}
\epsilon & 0 \\
0 & -\epsilon
\end{bmatrix}
\begin{bmatrix}
\braket{k;+\epsilon}{k; \epsilon} \\
\braket{k;-\epsilon}{k; \epsilon}
\end{bmatrix}
\\ &=
\begin{bmatrix}
\ket{k;+\epsilon} &
\ket{k;-\epsilon}
\end{bmatrix}
\begin{bmatrix}
\epsilon & 0 \\
0 & -\epsilon
\end{bmatrix}
\begin{bmatrix}
1 \\
0
\end{bmatrix}
\\ &=
\begin{bmatrix}
\ket{k;+\epsilon} &
\ket{k;-\epsilon}
\end{bmatrix}
\begin{bmatrix}
\epsilon \\
0
\end{bmatrix}
\\ &=
\epsilon
\ket{k;+\epsilon}.
\end{aligned}
\end{equation}
%
\paragraph{Back to the core problem.}

The total energy for this particle is
%
\begin{equation}\label{eqn:gradQuantumProblemSet4Problem3:560}
E = V_0 \pm \sqrt{(\Hbar k c)^2 + (m c^2)^2}.
\end{equation}
%
This can be plotted nicely, as \( \frac{E}{m c^2} \) vs. \( \frac{\Hbar k}{m c} \) if non-dimensionalized as
%
\begin{equation}\label{eqn:gradQuantumProblemSet4Problem3:580}
\frac{E}{m c^2} = \frac{V_0}{m c^2} + \sqrt{1 + \frac{(\Hbar k)^2}{(m c)^2} }.
\end{equation}
%
For the step potential we have
%
\begin{equation}\label{eqn:gradQuantumProblemSet4Problem3:600}
\begin{aligned}
\frac{E}{m c^2}
&= \sqrt{ 1 + {\frac{ \Hbar k_1}{m c} }^2 }
\\ &= \pm \sqrt{ 1 + {\frac{ \Hbar k_2}{m c} }^2 } + \frac{V_0}{m c^2},
\end{aligned}
\end{equation}
%
or
\begin{equation}\label{eqn:gradQuantumProblemSet4Problem3:620}
\frac{ \Hbar k_2}{m c}
=
\lr{ \lr{ \pm \sqrt{ 1 + {\frac{ \Hbar k_1}{m c} }^2 } - \frac{V_0}{m c^2} }^2 - 1 }^{1/2}.
\end{equation}
%
When this is real valued, there is transmission in the barrier region.  There are a few cases of interest, the \( V_0 < 2 m c^2 \) cases are plotted in \cref{fig:ps4DiracStepPotential:ps4DiracStepPotentialFig1} showing the transmission (\( V_0 = 0.7 m c^2 \)) and decaying cases (\( V_0 = 1.4 m c^2 \)) respectively.
Three \( V_0 > 2 m c^2 \) cases are plotted in \cref{fig:ps4DiracStepPotential:ps4DiracStepPotentialFig3},
showing the
anti-particle transmission,
decaying solution,
and ordinary-particle transmission
(\( (V_0/m c^2,k \Hbar/m c) = (3.4,1.7), (2.6,1.9), (2.7,6.3) \)).  At low enough momentum there is antiparticle transmission, then reflection as momentum is increased, and finally at high enough momentum ordinary particle transmission.

\mathImageTwoFigures
{../figures/phy1520-quantum/ps4DiracStepPotentialFig2}
{../figures/phy1520-quantum/ps4DiracStepPotentialFig1}
{\( V_0 < 2 m c^2 \).}{fig:ps4DiracStepPotential:ps4DiracStepPotentialFig1}{scale=0.3}
{energyVsMomentumForDiracStepPotentialWaveFunctions.nb}
%%
% v_0, k
% 3.4, 1.7
% 2.6, 1.9
% 2.7, 6.3
\mathImageThreeFiguresOneLine
{../figures/phy1520-quantum/ps4DiracStepPotentialFig3}
{../figures/phy1520-quantum/ps4DiracStepPotentialFig4}
{../figures/phy1520-quantum/ps4DiracStepPotentialFig5}
{\( V_0 > 2 m c^2 \).}{fig:ps4DiracStepPotential:ps4DiracStepPotentialFig3}{scale=0.3}
{energyVsMomentumForDiracStepPotentialWaveFunctions.nb}

We want to find the wave incident, reflected, and transmitted wave function.  From the diagonalization above, these are
%
\begin{equation}\label{eqn:gradQuantumProblemSet4Problem3:640}
\Psi_{\textrm{inc}} =
\begin{bmatrix}
\cos\theta_{k_1} \\
\sin\theta_{k_1} \\
\end{bmatrix}
e^{i (k_1 x - E t/\Hbar)},
\end{equation}
\begin{equation}\label{eqn:gradQuantumProblemSet4Problem3:660}
\Psi_{\textrm{ref}} =
\begin{bmatrix}
\sin\theta_{k_1} \\
\cos\theta_{k_1} \\
\end{bmatrix}
e^{i (-k_1 x - E t/\Hbar)}.
\end{equation}
The currents \( j = c (\psi_1^\conj \psi_1 - \psi_2^\conj \psi_2) \) for these wave functions are
%
\begin{equation}\label{eqn:gradQuantumProblemSet4Problem3:960}
j_{\textrm{inc}} = c (\cos^2 \theta_{k_1} - \sin^2 \theta_{k_1}) = c \cos(2 \theta_{k_1} ),
\end{equation}
%
and
%
\begin{equation}\label{eqn:gradQuantumProblemSet4Problem3:980}
j_{\textrm{ref}} = c (\sin^2 \theta_{k_1} - \cos^2 \theta_{k_1}) = -c \cos(2 \theta_{k_1} ).
\end{equation}
%
When there is a transmitted particle (anti-particle) in region II, the transmitted wave function are respectively
%
\begin{equation}\label{eqn:gradQuantumProblemSet4Problem3:680}
\Psi_{\textrm{trans}} =
\begin{bmatrix}
\cos\theta_{k_2} \\
\sin\theta_{k_2} \\
\end{bmatrix}
e^{i (k_2 x - E t/\Hbar)},
\end{equation}
\begin{equation}\label{eqn:gradQuantumProblemSet4Problem3:700}
\Psi_{\textrm{trans}} =
\begin{bmatrix}
-\sin\theta_{k_2} \\
\cos\theta_{k_2} \\
\end{bmatrix}
e^{i (k_2 x - E t/\Hbar)}.
\end{equation}
%
The currents for these are respectively
\begin{equation}\label{eqn:gradQuantumProblemSet4Problem3:1321}
j = c \lr{ \cos^2\theta_{k_2} - \sin^2\theta_{k_2} } = c \cos(2\theta_{k_2}),
\end{equation}
%
and
%
\begin{equation}\label{eqn:gradQuantumProblemSet4Problem3:1341}
j = c \lr{ \sin^2\theta_{k_2} - \cos^2\theta_{k_2} } = -c \cos(2\theta_{k_2}).
\end{equation}
%
To determine the weighting of the wave functions, we require matching at the boundary.  There are a few cases to consider

\begin{enumerate}[(i)]
\item Total reflection.  Are there any special values of momentum that allow for total reflection?  If so, at the boundary both components must be zero
%
\begin{equation}\label{eqn:gradQuantumProblemSet4Problem3:720}
\begin{bmatrix}
\cos\theta_{k_1} \\
\sin\theta_{k_1} \\
\end{bmatrix}
+
\frac{B}{A}
\begin{bmatrix}
\sin\theta_{k_1} \\
\cos\theta_{k_1} \\
\end{bmatrix}
=
\begin{bmatrix}
0 \\
0
\end{bmatrix}.
\end{equation}
This is possible if \( -B/A = \cot\theta_{k_1} = \tan\theta_{k_1} \), which requires
%
\begin{equation}\label{eqn:gradQuantumProblemSet4Problem3:740}
\theta_{k_1} = \frac{\pi}{4} \lr{ 1 + 2 n }, \qquad n \in \bbZ.
\end{equation}
%
However, we must also have
%
\begin{equation}\label{eqn:gradQuantumProblemSet4Problem3:760}
\tan 2 \theta_1 = \frac{m c}{\Hbar k},
\end{equation}
%
so
%
\begin{equation}\label{eqn:gradQuantumProblemSet4Problem3:780}
\theta_1 = \inv{2} \Atan\lr{ \frac{m c}{\Hbar k} }.
\end{equation}
%
Simultaneous solutions of \cref{eqn:gradQuantumProblemSet4Problem3:740}, \cref{eqn:gradQuantumProblemSet4Problem3:780} only occur at \( k = \pm 0 \), where \( \tan( (1 + 2 n)\pi/2 ) = \pm \infty \).  Since a particle at rest is not at interest in a reflection scenario, this shows that a decaying solution in region II must be introduced to match the boundary value constraints.
\item Ordinary matter transmission.
For this case at \( x = 0 \), we have
%
\begin{equation}\label{eqn:gradQuantumProblemSet4Problem3:920}
A
\begin{bmatrix}
\cos\theta_{k_1} \\
\sin\theta_{k_1} \\
\end{bmatrix}
+
B
\begin{bmatrix}
\sin\theta_{k_1} \\
\cos\theta_{k_1} \\
\end{bmatrix}
=
D
\begin{bmatrix}
\cos \theta_2 \\
\sin \theta_2
\end{bmatrix}.
\end{equation}
%
With \( a = B/A \) and \( b = D/A \), \( S_{1,2} = \sin\theta_{k_{1,2}}, C_{1,2} = \cos\theta_{k_{1,2}} \), this is
%
\begin{equation}\label{eqn:gradQuantumProblemSet4Problem3:800}
\begin{bmatrix}
C_1 \\
S_1
\end{bmatrix}
=
\begin{bmatrix}
- S_1 & C_2 \\
- C_1 & S_2
\end{bmatrix}
\begin{bmatrix}
a \\
b
\end{bmatrix},
\end{equation}
%
or
\begin{equation}\label{eqn:gradQuantumProblemSet4Problem3:820}
\begin{aligned}
\begin{bmatrix}
a \\
b
\end{bmatrix}
&=
\inv{- S_1 S_2 + C_1 C_2 }
\begin{bmatrix}
S_2 & -C_2 \\
C_1 & -S_1
\end{bmatrix}
\begin{bmatrix}
C_1 \\
S_1
\end{bmatrix}
\\ &=
\inv{\cos(\theta_{k_1} + \theta_{k_2})}
\begin{bmatrix}
C_1 S_2 - S_1 C_2 \\
C_1^2 - S_1^2
\end{bmatrix},
\end{aligned}
\end{equation}
%
which is
\boxedEquation{eqn:gradQuantumProblemSet4Problem3:840}{
\begin{aligned}
\frac{B}{A} &= \frac{ \cos(\theta_{k_2} - \theta_{k_1}) }{\cos(\theta_{k_1} + \theta_{k_2})} \\
\frac{D}{A} &= \frac{ \cos(2 \theta_{k_1}) }{\cos(\theta_{k_1} + \theta_{k_2})}.
\end{aligned}
}

Let's verify that the currents in both regions match.  With \( A = 1 \), the region I current sum is
%
\begin{equation}\label{eqn:gradQuantumProblemSet4Problem3:1000}
\begin{aligned}
j_{\textrm{inc}}
+ j_{\textrm{ref}}
&=
c \cos( 2 \theta_{k_1} ) - B^2 c \cos( 2 \theta_{k_1} )
\\ &=
c \cos( 2 \theta_{k_1} )
\lr{ 1 -
\frac{ \sin^2(\theta_{k_2} - \theta_{k_1}) }{\cos^2(\theta_{k_1} + \theta_{k_2})} }
\\ &=
c \cos( 2 \theta_{k_1} )
\frac{ \cos^2(\theta_{k_1} + \theta_{k_2}) - \sin^2(\theta_{k_2} - \theta_{k_1}) }{\cos^2(\theta_{k_1} + \theta_{k_2})}
\\ &=
c
\frac{ \cos( 2 \theta_{k_1} ) \cos(2 \theta_{k_1}) \cos(2 \theta_{k_2})}
{\cos^2(\theta_{k_1} + \theta_{k_2})}.
\end{aligned}
\end{equation}
%
Whereas, the transmitted (region II) current is
\begin{equation}\label{eqn:gradQuantumProblemSet4Problem3:1020}
\begin{aligned}
j_{\textrm{trans}}
&= c D^2 \cos( 2 \theta_{k_2} )
\\ &= c \cos( 2 \theta_{k_2} )
\frac{ \cos^2(2 \theta_{k_1}) }{\cos^2(\theta_{k_1} + \theta_{k_2})},
\end{aligned}
\end{equation}
%
so we see \( j_{\textrm{inc}} + j_{\textrm{ref}} = j_{\textrm{trans}} \), as expected.

\item Anti-matter transmission.
\index{anti-matter transmission}

For this case at \( x = 0 \), we have
%
\begin{equation}\label{eqn:gradQuantumProblemSet4Problem3:940}
A
\begin{bmatrix}
\cos\theta_{k_1} \\
\sin\theta_{k_1} \\
\end{bmatrix}
+
B
\begin{bmatrix}
\sin\theta_{k_1} \\
\cos\theta_{k_1} \\
\end{bmatrix}
=
D
\begin{bmatrix}
-\sin \theta_2 \\
\cos \theta_2 \\
\end{bmatrix}.
\end{equation}
%
With the same substitutions as above, this is
%
\begin{equation}\label{eqn:gradQuantumProblemSet4Problem3:860}
\begin{bmatrix}
C_1 \\
S_1
\end{bmatrix}
=
\begin{bmatrix}
- S_1 & -S_2 \\
- C_1 & C_2
\end{bmatrix}
\begin{bmatrix}
a \\
b
\end{bmatrix},
\end{equation}
%
or
\begin{equation}\label{eqn:gradQuantumProblemSet4Problem3:880}
\begin{aligned}
\begin{bmatrix}
a \\
b
\end{bmatrix}
&=
\inv{- (S_1 C_2 + S_2 C_1) }
\begin{bmatrix}
C_2 & -S_2 \\
C_1 & -S_1
\end{bmatrix}
\begin{bmatrix}
C_1 \\
S_1
\end{bmatrix}
\\ &=
\inv{-\sin(\theta_{k_1} + \theta_{k_2})}
\begin{bmatrix}
C_1 C_2 + S_1 S_2 \\
C_1^2 - S_1^2
\end{bmatrix},
\end{aligned}
\end{equation}
%
which is
\boxedEquation{eqn:gradQuantumProblemSet4Problem3:900}{
\begin{aligned}
\frac{B}{A} &= \frac{ \cos(\theta_{k_1} - \theta_{k_2}) }{\sin(\theta_{k_1} + \theta_{k_2})} \\
\frac{D}{A} &= -\frac{ \cos(2 \theta_{k_1}) }{\sin(\theta_{k_1} + \theta_{k_2})}.
\end{aligned}
}

For the anti-particle transmission, the region I current is
%
\begin{equation}\label{eqn:gradQuantumProblemSet4Problem3:1001}
\begin{aligned}
j_{\textrm{inc}}
+ j_{\textrm{ref}}
&=
c \cos( 2 \theta_{k_1} ) - B^2 c \cos( 2 \theta_{k_1} )
\\ &=
c \cos( 2 \theta_{k_1} )
\lr{ 1 -
\frac{ \cos^2(\theta_{k_1} - \theta_{k_2}) }{\sin^2(\theta_{k_1} + \theta_{k_2})} }
\\ &=
c \cos( 2 \theta_{k_1} )
\frac{ \sin^2(\theta_{k_1} + \theta_{k_2}) - \cos^2(\theta_{k_1} - \theta_{k_2}) }{\cos^2(\theta_{k_1} + \theta_{k_2})}
\\ &=
-c
\frac{ \cos( 2 \theta_{k_1} ) \cos(2 \theta_{k_1}) \cos(2 \theta_{k_2})}
{\sin^2(\theta_{k_1} + \theta_{k_2})}.
\end{aligned}
\end{equation}
%
Whereas, the transmitted (region II) current is
\begin{equation}\label{eqn:gradQuantumProblemSet4Problem3:1021}
\begin{aligned}
j_{\textrm{trans}}
&= -c D^2 \cos( 2 \theta_{k_2} )
\\ &= -c \cos( 2 \theta_{k_2} )
\frac{ \cos^2(2 \theta_{k_1}) }{\sin^2(\theta_{k_1} + \theta_{k_2})},
\end{aligned}
\end{equation}
%
and again we see \( j_{\textrm{inc}} + j_{\textrm{ref}} = j_{\textrm{trans}} \), as expected.
\item Decaying transmission.
In units where \( \Hbar = c = 1 \), the eigenkets found for the Dirac Hamiltonian, before normalization, were found to be
%
\begin{equation}\label{eqn:gradQuantumProblemSet4Problem3:1041}
\ket{\pm} \propto
\begin{bmatrix}
\mp m \\
\epsilon_1 \pm k
\end{bmatrix}
e^{\pm i k x - i E t/\Hbar},
\end{equation}
%
where \( \epsilon_1^2 = m^2 + k^2 \).  Letting \( k \rightarrow i k \), this provides the form of the non-oscillatory wavefunctions in region II when there is no ordinary nor anti-particle transmission.
%
\begin{equation}\label{eqn:gradQuantumProblemSet4Problem3:1061}
\ket{\pm} \propto
\begin{bmatrix}
\mp m \\
\epsilon_2 \pm i k
\end{bmatrix}
e^{\mp k x - i E t/\Hbar},
\end{equation}
%
where
%
\begin{equation}\label{eqn:gradQuantumProblemSet4Problem3:1081}
\epsilon_2^2 = m^2 - k^2.
\end{equation}

Observe that the \( \psi_2 \) component of this wave function sits on a circle in the complex plane
%
\begin{equation}\label{eqn:gradQuantumProblemSet4Problem3:1101}
\begin{aligned}
\Abs{\epsilon_2 \pm i k}^2
&= \epsilon^2 + k^2
\\ &= m^2 - k^2 + k^2
\\ &= m^2.
\end{aligned}
\end{equation}
%
Putting back in the factors of \( \Hbar \), \( c\), this allows the identification
%
\begin{equation}\label{eqn:gradQuantumProblemSet4Problem3:1121}
\epsilon_2 \pm i \Hbar k c = m c^2 e^{i\phi}.
\end{equation}
%
So, like the trigonometric representation of the oscillatory wave function, the normalized wave functions for exponential decay(increase) can be written
%
\begin{equation}\label{eqn:gradQuantumProblemSet4Problem3:1141}
\ket{\pm}
=
\inv{\sqrt{2}}
\begin{bmatrix}
\pm e^{\pm i \phi/2} \\
e^{\mp i \phi/2}
\end{bmatrix}
e^{\mp k x - i E t/\Hbar}
,
\end{equation}
%
where the respective eigenvalues are \( \epsilon_2 = \pm \sqrt{ (m c^2)^2 - (\Hbar k c)^2} \), and \( \Hbar k c < m c^2 \).

Observe that the currents for the exponential wave functions are both light-like
%
\begin{equation}\label{eqn:gradQuantumProblemSet4Problem3:1161}
j_{\pm} = c \lr{ \Abs{\pm e^{\pm i \phi/2}}^2 - \Abs{e^{\mp i \phi/2}}^2 } = 0,
\end{equation}
%
which makes some sense since the particle is not able to propagate freely in the barrier region.

At the interface, we wish to solve
%
\begin{equation}\label{eqn:gradQuantumProblemSet4Problem3:1181}
A
\begin{bmatrix}
C_1 \\
S_1
\end{bmatrix}
+
B
\begin{bmatrix}
S_1 \\
C_1 \\
\end{bmatrix}
=
\frac{D}{\sqrt{2}}
\begin{bmatrix}
e^{i \phi/2} \\
e^{-i \phi/2} \\
\end{bmatrix}.
\end{equation}
%
With \( a = A/D \), and \( b = B/D \), this has solution
%
\begin{equation}\label{eqn:gradQuantumProblemSet4Problem3:1201}
\begin{aligned}
\begin{bmatrix}
a \\
b
\end{bmatrix}
&=
\inv{\sqrt{2}}
{\begin{bmatrix}
C_1 & S_1 \\
S_1 & C_1
\end{bmatrix}}^{-1}
\begin{bmatrix}
e^{i \phi/2} \\
e^{-i \phi/2} \\
\end{bmatrix}
\\ &=
\inv{\sqrt{2} \cos(2 \theta_{k_1})}
\begin{bmatrix}
C_1 & -S_1 \\
-S_1 & C_1
\end{bmatrix}
\begin{bmatrix}
e^{i \phi/2} \\
e^{-i \phi/2} \\
\end{bmatrix}
\\ &=
\inv{\sqrt{2} \cos(2 \theta_{k_1})}
\begin{bmatrix}
C_1 e^{i \phi/2} -S_1 e^{-i \phi/2} \\
-S_1 e^{i \phi/2} + C_1 e^{-i \phi/2}
\end{bmatrix}.
\end{aligned}
\end{equation}
%
The reflection coefficient is unity
\begin{equation}\label{eqn:gradQuantumProblemSet4Problem3:1221}
\begin{aligned}
R
&= \Abs{\frac{B}{A}}^2
\\ &= \Abs{\frac{B/D}{A/D}}^2
\\ &= \frac{\Abs{-S_1 e^{i \phi/2} + C_1 e^{-i \phi/2}}^2}
{\Abs{C_1 e^{i \phi/2} -S_1 e^{-i \phi/2} }^2}
\\ &=
\frac
{ C_1^2 + S_1^2 - 2 S_1 C_1 \Real e^{-i\phi} }
{ C_1^2 + S_1^2 - 2 S_1 C_1 \Real e^{i\phi} }
\\ &=
\frac
{ 1 - \sin( 2 \theta_{k_1}) \cos \phi }
{ 1 - \sin( 2 \theta_{k_1}) \cos \phi }
\\ &=
1.
\end{aligned}
\end{equation}
%
%%%\paragraph{Confusion:}
%%%Completely counter to (my) intuition, the transmission coefficient is not 0, so we don't appear to have \( R + T = 1 \).  Instead
%%%
%%%\begin{dmath}\label{eqn:gradQuantumProblemSet4Problem3:1241}
%%%T
%%%= \Abs{\frac{D}{A}}^2
%%%=
%%%\frac{2 \cos^2(2 \theta_{k_1})}
%%%{
%%%1 - \sin( 2 \theta_{k_1}) \cos \phi
%%%}
%%%\ne 0.
%%%\end{dmath}
%%%
%%%
%%%Temporarily reverting to natural units, note that
%%%
%%%\begin{dmath}\label{eqn:gradQuantumProblemSet4Problem3:1261}
%%%\cos\phi
%%%= \Real e^{i \phi}
%%%= \Real \lr{ \frac{\epsilon_2}{m} + i \frac{k_2}{m} }
%%%%=
%%%%\frac{
%%%%\sqrt{
%%%%m^2 - k_2^2
%%%%}
%%%%}{m}
%%%\frac{\epsilon_2}{m},
%%%\end{dmath}
%%%
%%%so
%%%\begin{dmath}\label{eqn:gradQuantumProblemSet4Problem3:1281}
%%%T
%%%=
%%%\frac{
%%%   2 \lr{\frac{k_1}{\epsilon_1}}^2
%%%}{
%%%1 - \frac{m}{\epsilon_1} \frac{\epsilon_2}{m}
%%%}
%%%=
%%%\frac{
%%%   2 k_1^2
%%%}{
%%%\epsilon_1^2 - \epsilon_1 \epsilon_2
%%%}
%%%\end{dmath}
%%%
%%%In terms of \(k_1, k_2 \) after putting back \( \Hbar, c \)'s this is
%%%
%%%\begin{dmath}\label{eqn:gradQuantumProblemSet4Problem3:1301}
%%%T
%%%=
%%%\frac{
%%%   2 (\Hbar k_1 c)^2
%%%}{
%%%\lr{ m c^2 }^2 + \lr{ \Hbar k_1 c }^2
%%%- \sqrt{
%%%\lr{(m c^2)^2 + \lr{ \Hbar k_1 c}^2 }
%%%\lr{(m c^2)^2 - \lr{ \Hbar k_2 c}^2 }
%%%}
%%%}.
%%%\end{dmath}
%
\end{enumerate}
