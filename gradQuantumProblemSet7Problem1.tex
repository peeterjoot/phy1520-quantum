%
% Copyright � 2015 Peeter Joot.  All Rights Reserved.
% Licenced as described in the file LICENSE under the root directory of this GIT repository.
%
\makeoproblem{Double well potential.}{gradQuantum:problemSet7:1}{2015 ps7 p1}{
\index{double well potential}

Consider a particle in the double well potential

\begin{dmath}\label{eqn:gradQuantumProblemSet7Problem1:20}
V (x) =
\frac{m \omega^2}{ 8 a^2 }
\lr{ x + a }^2 \lr{x - a}^2.
\end{dmath}

Expanding \( V(x) \) around \( x = \pm a \) leads to a harmonic potential with frequency \( \omega \).
Construct variational states with even/odd parity as \( \psi_\pm(x) = g_\pm \lr{ \phi(x - a) \pm \phi(x + a) } \) where \( \phi(x) \) is the normalized ground state of the usual harmonic oscillator with frequency \( \omega \), i.e.,

\begin{equation}\label{eqn:gradQuantumProblemSet7Problem1:40}
\phi(x)
=
\lr{ \inv{ \pi a_0^2 }}^{1/4}
e^{
- \frac{x^2}{ 2 a_0^2 }
}
;
\qquad a_0 = \sqrt{ \frac{\Hbar}{m \omega} }.
\end{equation}

\makesubproblem{}{gradQuantum:problemSet7:1a}
Determine the normalization constants \( g_\pm \).
Next using these wavefunctions, determine the variational energies of these two states.
Hence determine the `tunnel splitting' between the two states, induced by the tunneling through
the barrier region.
In your calculations, you can assume \( a \gg a_0 \), so retain only the leading terms in any polynomials you might encounter when you do the integrals.

\makesubproblem{}{gradQuantum:problemSet7:1b}
If we pay attention to these lowest two states (left well and right well) in the full Hilbert space, we can write a phenomenological \( 2 \times 2 \) Hamiltonian

\begin{dmath}\label{eqn:gradQuantumProblemSet7Problem1:60}
H =
\begin{bmatrix}
\epsilon_0 & -\gamma \\
-\gamma & \epsilon_0
\end{bmatrix},
\end{dmath}

where \( \epsilon_0 \) is the energy on each side, and \( \gamma \) leads to tunneling, so if we start off in the left well, \( t \) leads to a nonzero amplitude to find it in the right well at a later time.
Find its eigenvalues and eigenvectors.
Comparing with your variational result for the energy splitting, determine the `tunnel coupling' \( \gamma \).

} % makeproblem

\makeanswer{gradQuantum:problemSet7:1}{
\withproblemsetsParagraph{
\makeSubAnswer{}{gradQuantum:problemSet7:1a}

The integration grunt work for this problem can be found in \nbref{ps7:doubleWellPotential.nb}.  This yields an energy difference of

\begin{dmath}\label{eqn:gradQuantumProblemSet7Problem1:80}
\overbar{E}_{+} - \overbar{E}_{-}
=
\frac{\Hbar \omega}{8}
\lr{
\lr{
\frac{4 a^2}{a_0^2}
+
\lr{\frac{a^2}{a_0^2}-5 } \exp\lr{\frac{a^2}{a_0^2}} - 2
}
\lr{ \coth\lr{ \frac{a^2}{a_0^2} } -1 }
}.
\end{dmath}

With \( u = a^2/a_0^2 \), in the \( a \gg a_0 \) limit, the almost zero \( \coth u - 1 \) difference can be approximated as an exponential

\begin{dmath}\label{eqn:gradQuantumProblemSet7Problem1:100}
\coth u -1
=
\frac{e^{2u} + 1}{e^{2u} - 1} - 1
=
\frac{e^{2u} + 1 - e^{2u} + 1 }{e^{2u} - 1}
=
\frac{ 2 }{e^{2u} - 1}
\approx
2 e^{-2u},
\end{dmath}

so the energy difference is approximately

\begin{dmath}\label{eqn:gradQuantumProblemSet7Problem1:120}
\overbar{E}_{+} - \overbar{E}_{-}
\approx
\frac{\Hbar \omega}{4} \frac{a^2}{a_0^2} \exp\lr{-\frac{a^2}{a_0^2}}.
\end{dmath}

\makeSubAnswer{}{gradQuantum:problemSet7:1b}
Now lets compare to the energy levels of the phenomenological Hamiltonian, which are given by

\begin{dmath}\label{eqn:gradQuantumProblemSet7Problem1:140}
0 = \lr{ \epsilon_0 - E }^2 - \gamma^2,
\end{dmath}

with eigenvalues

\begin{dmath}\label{eqn:gradQuantumProblemSet7Problem1:160}
E_{\pm} = \epsilon_0 \pm \gamma.
\end{dmath}

If the eigenvectors are proportional to the column vector given by

\begin{dmath}\label{eqn:gradQuantumProblemSet7Problem1:180}
\ket{\pm} =
\begin{bmatrix}
a \\
b
\end{bmatrix},
\end{dmath}

then we must have

\begin{dmath}\label{eqn:gradQuantumProblemSet7Problem1:200}
0
= \lr{ \epsilon_0 - (\epsilon_0 \pm \gamma) } a - \gamma b
= \gamma \lr{ \mp a - b },
\end{dmath}

or

\begin{dmath}\label{eqn:gradQuantumProblemSet7Problem1:220}
\ket{\pm}
=
\inv{\sqrt{2}}
\begin{bmatrix}
1 \\
\mp 1
\end{bmatrix}.
\end{dmath}

The energy level difference for this Hamiltonian is

\begin{dmath}\label{eqn:gradQuantumProblemSet7Problem1:240}
\Delta E
= E_{+} - E_{-}
= \epsilon_0 + \gamma - \lr{ \epsilon_0 - \gamma }
= 2 \gamma.
\end{dmath}

Equating this difference with \cref{eqn:gradQuantumProblemSet7Problem1:120}, we have

\begin{dmath}\label{eqn:gradQuantumProblemSet7Problem1:260}
\gamma
=
\frac{\Hbar \omega}{8} \frac{a^2}{a_0^2} \exp\lr{-\frac{a^2}{a_0^2}}.
\end{dmath}
}
}
