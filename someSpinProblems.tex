%
% Copyright � 2015 Peeter Joot.  All Rights Reserved.
% Licenced as described in the file LICENSE under the root directory of this GIT repository.
%
%\input{../blogpost.tex}
%\renewcommand{\basename}{someSpinProblems}
%\renewcommand{\dirname}{notes/phy1520/}
%%\newcommand{\dateintitle}{}
%%\newcommand{\keywords}{}
%
%\input{../peeter_prologue_print2.tex}
%
%\usepackage{peeters_layout_exercise}
%\usepackage{peeters_braket}
%\usepackage{peeters_figures}
%
%\beginArtNoToc
%
%\generatetitle{Some spin problems}
%%\chapter{Some spin problems}
%%\label{chap:someSpinProblems}
%
%Problems from angular momentum chapter of \citep{sakurai2014modern}.
%
%
\makeoproblem{\( S_y \) eigenvectors.}{problem:someSpinProblems:1}{\citep{sakurai2014modern} pr. 4.1}{
\index{spin half}
Find the eigenvectors of \( \sigma_y \), and then find the probability that a measurement of \( S_y \) will be \( \Hbar/2 \) when the state is initially
%
%\begin{equation}\label{eqn:someSpinProblems:20}
\(
\begin{bmatrix}
\alpha \\
\beta
\end{bmatrix}
\).
%\end{equation}
%
} % problem
%
\makeanswer{problem:someSpinProblems:1}{
%
The eigenvalues should be \( \pm 1 \), which is easily checked
%
\begin{equation}\label{eqn:someSpinProblems:40}
\begin{aligned}
0 &=
\Abs{ \sigma_y - \lambda }
\\ &=
\begin{vmatrix}
-\lambda & -i \\
i & -\lambda
\end{vmatrix}
\\ &=
\lambda^2 - 1.
\end{aligned}
\end{equation}
%
For \( \ket{+} = (a,b)^\T \) we must have
%
\begin{equation}\label{eqn:someSpinProblems:60}
-1 a - i b = 0,
\end{equation}
%
so
%
\begin{equation}\label{eqn:someSpinProblems:80}
\ket{+} \propto
\begin{bmatrix}
-i \\
1
\end{bmatrix},
\end{equation}
%
or
\begin{equation}\label{eqn:someSpinProblems:100}
\ket{+} =
\inv{\sqrt{2}}
\begin{bmatrix}
1 \\
i
\end{bmatrix}.
\end{equation}
%
For \( \ket{-} \) we must have
%
\begin{equation}\label{eqn:someSpinProblems:120}
a - i b = 0,
\end{equation}
%
so
%
\begin{equation}\label{eqn:someSpinProblems:140}
\ket{+} \propto
\begin{bmatrix}
i \\
1
\end{bmatrix},
\end{equation}
%
or
\begin{equation}\label{eqn:someSpinProblems:160}
\ket{+} =
\inv{\sqrt{2}}
\begin{bmatrix}
1 \\
-i
\end{bmatrix}.
\end{equation}
%
The normalized eigenvectors are
%
\boxedEquation{eqn:someSpinProblems:180}{
\ket{\pm} =
\inv{\sqrt{2}}
\begin{bmatrix}
1 \\
\pm i
\end{bmatrix}.
}
For the probability question we are interested in
%
\begin{equation}\label{eqn:someSpinProblems:200}
\begin{aligned}
\Abs{\bra{S_y; +}
\begin{bmatrix}
\alpha \\
\beta
\end{bmatrix}
}^2
&=
\inv{2} \Abs{
\begin{bmatrix}
1 & -i
\end{bmatrix}
\begin{bmatrix}
\alpha \\
\beta
\end{bmatrix}
}^2
\\ &=
\inv{2} \lr{ \Abs{\alpha}^2 + \Abs{\beta}^2 }
\\ &=
\inv{2}.
\end{aligned}
\end{equation}
%
There is a 50\% chance of finding the particle in the \( \ket{S_x;+} \) state, independent of the initial state.
} % answer
%
\makeoproblem{Magnetic Hamiltonian eigenvectors.}{problem:someSpinProblems:2}{\citep{sakurai2014modern} pr. 3.2}{
\index{magnetic field}

Using Pauli matrices, find the eigenvectors for the magnetic spin interaction Hamiltonian
%
\begin{equation}\label{eqn:someSpinProblems:220}
H = - \inv{\Hbar} 2 \mu \BS \cdot \BB.
\end{equation}
} % problem
%
\makeanswer{problem:someSpinProblems:2}{
%
\begin{equation}\label{eqn:someSpinProblems:240}
\begin{aligned}
H
&= - \mu \Bsigma \cdot \BB
\\ &= - \mu \lr{ B_x \PauliX + B_y \PauliY + B_z \PauliZ }
\\ &= - \mu
\begin{bmatrix}
B_z & B_x - i B_y \\
B_x + i B_y & -B_z
\end{bmatrix}.
\end{aligned}
\end{equation}
%
The characteristic equation is
\begin{equation}\label{eqn:someSpinProblems:260}
\begin{aligned}
0
&=
\begin{vmatrix}
-\mu B_z -\lambda & -\mu(B_x - i B_y) \\
-\mu(B_x + i B_y) & \mu B_z - \lambda
\end{vmatrix}
\\ &=
-\lr{ (\mu B_z)^2 - \lambda^2 }
- \mu^2\lr{ B_x^2 - (iB_y)^2 }
\\ &=
\lambda^2 - \mu^2 \BB^2.
\end{aligned}
\end{equation}
%
That is
\boxedEquation{eqn:someSpinProblems:360}{
\lambda = \pm \mu B.
}
Now for the eigenvectors.  We are looking for \( \ket{\pm} = (a,b)^\T \) such that
%
\begin{equation}\label{eqn:someSpinProblems:300}
\begin{aligned}
0
&= (-\mu B_z \mp \mu B) a -\mu(B_x - i B_y) b,
%\\ &= (B_z + B) a \\ &= (B_x - i B_y) b
\end{aligned}
\end{equation}
or
%
\begin{equation}\label{eqn:someSpinProblems:320}
\ket{\pm} \propto
\begin{bmatrix}
B_x - i B_y \\
B_z \pm B
\end{bmatrix}.
\end{equation}
%
This squares to
%
\begin{equation}\label{eqn:someSpinProblems:340}
B_x^2 + B_y^2 + B_z^2 + B^2 \pm 2 B B_z
= 2 B( B \pm B_z ),
\end{equation}
%
so the normalized eigenkets are
\boxedEquation{eqn:someSpinProblems:380}{
\ket{\pm}
=
\inv{\sqrt{2 B( B \pm B_z )}}
\begin{bmatrix}
B_x - i B_y \\
B_z \pm B
\end{bmatrix}.
}
} % answer

%\EndArticle
