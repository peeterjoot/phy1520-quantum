%
% Copyright � 2015 Peeter Joot.  All Rights Reserved.
% Licenced as described in the file LICENSE under the root directory of this GIT repository.
%
\makeoproblem{Hyperfine levels.}{gradQuantum:problemSet8:4}{2015 ps8 p4}{
\index{hyperfine levels}

We can schematically model the hyperfine interaction between the electron and proton spins as \( A \BS_\txte \cdot \BS_\txtp \) where \( A \) is the hyperfine interaction energy.

\makesubproblem{}{gradQuantum:problemSet8:4a}
Consider the spin-1/2 proton interacting with a spin-1/2 electron.
What are the spin eigenstates and eigenvalues?

\makesubproblem{}{gradQuantum:problemSet8:4b}
Now consider applying a magnetic field which leads to an extra term

\begin{dmath}\label{eqn:gradQuantumProblemSet8Problem4:20}
-B \lr{ g_\txte \mu_\txte S_\txte^z + g_\txtp \mu_\txtp S_\txtp^z }
\end{dmath}

with gyromagnetic ratios \( g_\txte \approx -2 \) and \( g_\txtp \approx 5.5 \), with magnetic moments \( \mu_\txte = e/2m_\txte \) and
\( \mu_\txtp = e/2m_\txtp \). The large nuclear mass ensures \( \mu_\txte/\mu_\txtp \sim 2000 \), so let us simply set \( \mu_\txtp = 0\). For convenience, set \( B g_\txte \mu_\txte \rightarrow B_{\textrm{eff}} \) so the Hamiltonian becomes

\begin{dmath}\label{eqn:gradQuantumProblemSet8Problem4:40}
H = A \BS_\txte \cdot \BS_\txtp - B_{\textrm{eff}} S_\txte^z,
\end{dmath}

so the only dimensionless parameter is \( B_{\textrm{eff}}/A \).

Using perturbation theory (degenerate or non-degenerate as appropriate) find how the coupled hyperfine levels split
for weak field \( B_{\textrm{eff}}/A \ll 1 \).
Also consider the strong field limit \( B_{\textrm{eff}}/A \gg 1 \).

\makesubproblem{}{gradQuantum:problemSet8:4c}
Compute the full field evolution of the levels and compare with the perturbative low field regime result and the high field regime result.

} % makeproblem

\makeanswer{gradQuantum:problemSet8:4}{
\withproblemsetsParagraph{
\makeSubAnswer{}{gradQuantum:problemSet8:4a}
%What are the spin eigenstates and eigenvalues?

With respect to the basis \( \beta = \ket{++}, \ket{-+}, \ket{+-}, \ket{--} \), where \( \ket{\pm, \pm} = \ket{\pm}_\txte \otimes \ket{\pm}_\txtp \) are the direct products of the eigenkets of the \( \BS_\txte \) and \( \BS_\txtp \) operators (not of the respective \( S^z \) operators), the unperturbed interaction Hamiltonian is

\begin{dmath}\label{eqn:gradQuantumProblemSet8Problem4:60}
A \BS_\txte \cdot \BS_\txtp
=
A
\begin{bmatrix}
\bra{++} \BS_\txte \cdot \BS_\txtp \ket{++} & \bra{++} \BS_\txte \cdot \BS_\txtp \ket{-+} & \bra{++} \BS_\txte \cdot \BS_\txtp \ket{+-} & \bra{++} \BS_\txte \cdot \BS_\txtp \ket{--} \\
\bra{-+} \BS_\txte \cdot \BS_\txtp \ket{++} & \bra{-+} \BS_\txte \cdot \BS_\txtp \ket{-+} & \bra{-+} \BS_\txte \cdot \BS_\txtp \ket{+-} & \bra{-+} \BS_\txte \cdot \BS_\txtp \ket{--} \\
\bra{+-} \BS_\txte \cdot \BS_\txtp \ket{++} & \bra{+-} \BS_\txte \cdot \BS_\txtp \ket{-+} & \bra{+-} \BS_\txte \cdot \BS_\txtp \ket{+-} & \bra{+-} \BS_\txte \cdot \BS_\txtp \ket{--} \\
\bra{--} \BS_\txte \cdot \BS_\txtp \ket{++} & \bra{--} \BS_\txte \cdot \BS_\txtp \ket{-+} & \bra{--} \BS_\txte \cdot \BS_\txtp \ket{+-} & \bra{--} \BS_\txte \cdot \BS_\txtp \ket{--} \\
\end{bmatrix}
=
\frac{A \Hbar^2}{4}
\begin{bmatrix}
\bra{+} \sigma_\txte \ket{+} \bra{+} \sigma_\txtp \ket{+} & \bra{+} \sigma_\txte \ket{-} \bra{+} \sigma_\txtp \ket{+} & \bra{+} \sigma_\txte \ket{+} \bra{+} \sigma_\txtp \ket{-} & \bra{+} \sigma_\txte \ket{-} \bra{+} \sigma_\txtp \ket{-} \\
\bra{-} \sigma_\txte \ket{+} \bra{+} \sigma_\txtp \ket{+} & \bra{-} \sigma_\txte \ket{-} \bra{+} \sigma_\txtp \ket{+} & \bra{-} \sigma_\txte \ket{+} \bra{+} \sigma_\txtp \ket{-} & \bra{-} \sigma_\txte \ket{-} \bra{+} \sigma_\txtp \ket{-} \\
\bra{+} \sigma_\txte \ket{+} \bra{-} \sigma_\txtp \ket{+} & \bra{+} \sigma_\txte \ket{-} \bra{-} \sigma_\txtp \ket{+} & \bra{+} \sigma_\txte \ket{+} \bra{-} \sigma_\txtp \ket{-} & \bra{+} \sigma_\txte \ket{-} \bra{-} \sigma_\txtp \ket{-} \\
\bra{-} \sigma_\txte \ket{+} \bra{-} \sigma_\txtp \ket{+} & \bra{-} \sigma_\txte \ket{-} \bra{-} \sigma_\txtp \ket{+} & \bra{-} \sigma_\txte \ket{+} \bra{-} \sigma_\txtp \ket{-} & \bra{-} \sigma_\txte \ket{-} \bra{-} \sigma_\txtp \ket{-} \\
\end{bmatrix}
=
\frac{A \Hbar^2}{4}
\begin{bmatrix}
(1) (1) & (0) (1) & (1) (0) & (0) (0) \\
(0) (1) & (-1) (1) & (0) (0) & (-1) (0) \\
(1) (0) & (0) (0) & (1) (-1) & (0) (-1) \\
(0) (0) & (-1) (0) & (0) (-1) & (-1) (-1) \\
\end{bmatrix}
=
\frac{A \Hbar^2}{4}
\begin{bmatrix}
\sigma_3 & 0 \\
0 & -\sigma_3
\end{bmatrix}.
\end{dmath}

The spin eigenstates are the basis elements of \( \beta \) above, with respective eigenvalues

\begin{dmath}\label{eqn:gradQuantumProblemSet8Problem4:80}
\setlr{ A \Hbar^2/4, -A \Hbar^2/4, -A \Hbar^2/4, A \Hbar^2/4}
\end{dmath}

\makeSubAnswer{}{gradQuantum:problemSet8:4b}

The matrix representation of the perturbation potential is

\begin{dmath}\label{eqn:gradQuantumProblemSet8Problem4:100}
-B_{\textrm{eff}} S^z_\txte
=
-\frac{B_{\textrm{eff}} \Hbar}{2}
\begin{bmatrix}
\bra{+} \sigma^z_\txte \ket{+} \braket{+}{+} & \bra{+} \sigma^z_\txte \ket{-} \braket{+}{+} & \bra{+} \sigma^z_\txte \ket{+} \braket{+}{-} & \bra{+} \sigma^z_\txte \ket{-} \braket{+}{-} \\
\bra{-} \sigma^z_\txte \ket{+} \braket{+}{+} & \bra{-} \sigma^z_\txte \ket{-} \braket{+}{+} & \bra{-} \sigma^z_\txte \ket{+} \braket{+}{-} & \bra{-} \sigma^z_\txte \ket{-} \braket{+}{-} \\
\bra{+} \sigma^z_\txte \ket{+} \braket{-}{+} & \bra{+} \sigma^z_\txte \ket{-} \braket{-}{+} & \bra{+} \sigma^z_\txte \ket{+} \braket{-}{-} & \bra{+} \sigma^z_\txte \ket{-} \braket{-}{-} \\
\bra{-} \sigma^z_\txte \ket{+} \braket{-}{+} & \bra{-} \sigma^z_\txte \ket{-} \braket{-}{+} & \bra{-} \sigma^z_\txte \ket{+} \braket{-}{-} & \bra{-} \sigma^z_\txte \ket{-} \braket{-}{-} \\
\end{bmatrix}
=
-\frac{B_{\textrm{eff}} \Hbar}{2}
\begin{bmatrix}
\sigma^z_\txte & 0 \\
0 & \sigma^z_\txte
\end{bmatrix},
\end{dmath}

Assuming the \( \BS_\txte \) operator is directed along \( \ncap = (\sin\theta \cos\phi, \sin\theta \sin\phi, \cos\theta) \) with eigenkets
\begin{dmath}\label{eqn:gradQuantumProblemSet8Problem4:120}
\ket{+} =
\begin{bmatrix}
e^{-i\phi} \cos(\theta/2) \\
\sin(\theta/2) \\
\end{bmatrix}
\end{dmath}
\begin{dmath}\label{eqn:gradQuantumProblemSet8Problem4:140}
\ket{-} =
\begin{bmatrix}
-e^{-i\phi} \sin(\theta/2) \\
\cos(\theta/2) \\
\end{bmatrix},
\end{dmath}

the representation of the \( \sigma^z_\txte \) operator is

\begin{dmath}\label{eqn:gradQuantumProblemSet8Problem4:160}
\sigma^z_\txte
=
\begin{bmatrix}
\cos\theta & -\sin\theta \\
-\sin\theta & -\cos\theta \\
\end{bmatrix}
=
U \PauliZ U^{-1},
\end{dmath}

where
\begin{equation}\label{eqn:gradQuantumProblemSet8Problem4:180}
U =
\begin{bmatrix}
-\cos(\theta/2) & \sin(\theta/2) \\
\sin(\theta/2) & \cos(\theta/2)
\end{bmatrix}.
\end{equation}

This is demonstrated in \nbref{ps8:PauliMatrixSpinOperators.nb}.

The full Hamiltonian can now be written in block matrix form

\begin{dmath}\label{eqn:gradQuantumProblemSet8Problem4:200}
H
=
\frac{A \Hbar^2}{4}
\begin{bmatrix}
\sigma_z & 0 \\
0 & -\sigma_z
\end{bmatrix}
-\frac{B_{\textrm{eff}} \Hbar}{2}
\begin{bmatrix}
U \sigma_z U^{-1} & 0 \\
0 & U \sigma_z U^{-1}
\end{bmatrix}
\end{dmath}

Transforming the Hamiltonian to the \( S^z_\txte \) basis we have

\begin{dmath}\label{eqn:gradQuantumProblemSet8Problem4:220}
H' =
\frac{A \Hbar^2}{4}
\begin{bmatrix}
U^{-1} \sigma_z U & 0 \\
0 & -U^{-1} \sigma_z U
\end{bmatrix}
-\frac{B_{\textrm{eff}} \Hbar}{2}
\begin{bmatrix}
\sigma_z & 0 \\
0 & \sigma_z
\end{bmatrix}
\end{dmath}

%With \( C = \cos(\theta/2), S = \sin(\theta/2) \) these \( U^{-1} \sigma_z U \) block matrices are
%
%\begin{dmath}\label{eqn:gradQuantumProblemSet8Problem4:240}
%U^{-1} \sigma_z U
%=
%\inv{-C^2 - S^2}
%\begin{bmatrix}
%C & -S \\
%-S & -C
%\end{bmatrix}
%\begin{bmatrix}
%1 & 0 \\
%0 & -1
%\end{bmatrix}
%\begin{bmatrix}
%-C & S \\
%S & C
%\end{bmatrix}
%=
%\begin{bmatrix}
%-C & S \\
%S & C
%\end{bmatrix}
%\begin{bmatrix}
%-C & S \\
%-S & -C
%\end{bmatrix}
%=
%\begin{bmatrix}
%C^2 - S^2 & -2 S C \\
%-2 C S & S^2 - C^2
%\end{bmatrix}
%=
%\begin{bmatrix}
%\cos\theta & - \sin\theta \\
%-\sin\theta & -\cos\theta
%\end{bmatrix}.
%\end{dmath}
%
%We don't need this for the zeroth order energy split.

For \( B_{\textrm{eff}} \ll A \), the first order energy splitting can be read off by inspection

\begin{equation}\label{eqn:gradQuantumProblemSet8Problem4:260}
\begin{aligned}
\frac{\Hbar^2 A}{4} &\rightarrow \frac{\Hbar^2 A}{4} -\frac{B_{\textrm{eff}} \Hbar}{2} \\
-\frac{\Hbar^2 A}{4} &\rightarrow -\frac{\Hbar^2 A}{4} +\frac{B_{\textrm{eff}} \Hbar}{2} \\
-\frac{\Hbar^2 A}{4} &\rightarrow -\frac{\Hbar^2 A}{4} -\frac{B_{\textrm{eff}} \Hbar}{2} \\
\frac{\Hbar^2 A}{4} &\rightarrow \frac{\Hbar^2 A}{4} +\frac{B_{\textrm{eff}} \Hbar}{2} \\
\end{aligned}
\end{equation}

For the strong field limit, we can flip the problem, and consider \( A \BS_\txte \cdot \BS_\txtp \) to be a perturbation of an initial Hamiltonian \( H_0 = -B_{\textrm{eff}} S^z_\txte \).  The diagonalization of that perturbation is just \cref{eqn:gradQuantumProblemSet8Problem4:200} so the first order energy shifts are

\begin{equation}\label{eqn:gradQuantumProblemSet8Problem4:280}
\begin{aligned}
-\frac{\Hbar B_{\textrm{eff}}}{2} &\rightarrow -\frac{B_{\textrm{eff}} \Hbar}{2} + \frac{\Hbar^2 A}{4} \\
\frac{\Hbar B_{\textrm{eff}}}{2} &\rightarrow \frac{B_{\textrm{eff}} \Hbar}{2} - \frac{\Hbar^2 A}{4} \\
-\frac{\Hbar B_{\textrm{eff}}}{2} &\rightarrow -\frac{B_{\textrm{eff}} \Hbar}{2} - \frac{\Hbar^2 A}{4} \\
\frac{\Hbar B_{\textrm{eff}}}{2} &\rightarrow \frac{B_{\textrm{eff}} \Hbar}{2} + \frac{\Hbar^2 A}{4}.
\end{aligned}
\end{equation}

The splitting is the same to first order, but the starting energies are different.

\makeSubAnswer{}{gradQuantum:problemSet8:4a}

The full field solutions (as found in \nbref{ps8:PauliMatrixSpinOperators.nb}) are

\begin{equation}\label{eqn:gradQuantumProblemSet8Problem4:300}
\begin{array}{c}
 -\frac{\Hbar}{4} \sqrt{4 B_{\textrm{eff}}^2 - 4 A \Hbar \cos\theta B_{\textrm{eff}} + A^2 \Hbar^2}, \\
  \frac{\Hbar}{4} \sqrt{4 B_{\textrm{eff}}^2 - 4 A \Hbar \cos\theta B_{\textrm{eff}} + A^2 \Hbar^2}, \\
 -\frac{\Hbar}{4} \sqrt{4 B_{\textrm{eff}}^2 + 4 A \Hbar \cos\theta B_{\textrm{eff}} + A^2 \Hbar^2}, \\
  \frac{\Hbar}{4} \sqrt{4 B_{\textrm{eff}}^2 + 4 A \Hbar \cos\theta B_{\textrm{eff}} + A^2 \Hbar^2}.
\end{array}
\end{equation}

For the weak field \( B_{\textrm{eff}} \ll A \) the respective approximations of these energies are

\begin{equation}\label{eqn:gradQuantumProblemSet8Problem4:320}
\begin{array}{c}
 -\frac{A \Hbar^2}{4} - \frac{B_{\textrm{eff}} \Hbar \cos\theta}{2}, \\
  \frac{A \Hbar^2}{4} - \frac{B_{\textrm{eff}} \Hbar \cos\theta}{2}, \\
 -\frac{A \Hbar^2}{4} + \frac{B_{\textrm{eff}} \Hbar \cos\theta}{2}, \\
  \frac{A \Hbar^2}{4} + \frac{B_{\textrm{eff}} \Hbar \cos\theta}{2},
\end{array}
\end{equation}

whereas for the strong field \( B_{\textrm{eff}} \gg A \) the respective approximations of these energies are

\begin{equation}\label{eqn:gradQuantumProblemSet8Problem4:340}
\begin{array}{c}
 -\frac{\Hbar B_{\textrm{eff}}}{2} - \frac{A \Hbar^2 \cos\theta}{4}, \\
  \frac{\Hbar B_{\textrm{eff}}}{2} - \frac{A \Hbar^2 \cos\theta}{4}, \\
 -\frac{\Hbar B_{\textrm{eff}}}{2} + \frac{A \Hbar^2 \cos\theta}{4}, \\
  \frac{\Hbar B_{\textrm{eff}}}{2} + \frac{A \Hbar^2 \cos\theta}{4}.
\end{array}
\end{equation}

The full field solution has an orientation specific coupling that the first order perturbative solution does not find, so the perturbation is most accurate when the electron spin orientation is close to the z-axis (\( \ncap \cdot \zcap = \cos\theta \approx 1 \)).
}
}
