%
% Copyright � 2015 Peeter Joot.  All Rights Reserved.
% Licenced as described in the file LICENSE under the root directory of this GIT repository.
%
%\input{../blogpost.tex}
%\renewcommand{\basename}{solenoidConstantField}
%\renewcommand{\dirname}{notes/phy1520/}
%%\newcommand{\dateintitle}{}
%%\newcommand{\keywords}{}
%
%\input{../peeter_prologue_print2.tex}
%
%\usepackage{peeters_layout_exercise}
%%\usepackage{peeters_braket}
%\usepackage{peeters_figures}
%
%\beginArtNoToc
%
%\generatetitle{Constant magnetic solenoid field}
%\label{chap:solenoidConstantField}

In \citep{sakurai2014modern} the following vector potential
%
\begin{dmath}\label{eqn:solenoidConstantField:20}
\BA = \frac{B \rho_a^2}{2 \rho} \phicap,
\end{dmath}
%
is introduced in a discussion on the Aharonov-Bohm effect, for configurations where the interior field of a solenoid is either a constant \( \BB \) or zero.

I wasn't able to make sense of this since the field I was calculating was zero for all \( \rho \ne 0 \)
%
\begin{dmath}\label{eqn:solenoidConstantField:40}
\BB
= \spacegrad \cross \BA
= \lr{ \rhocap \partial_\rho + \zcap \partial_z + \frac{\phicap}{\rho} \partial_\phi } \cross \frac{B \rho_a^2}{2 \rho} \phicap
= \lr{ \rhocap \partial_\rho + \frac{\phicap}{\rho} \partial_\phi } \cross \frac{B \rho_a^2}{2 \rho} \phicap
=
\frac{B \rho_a^2}{2}
\rhocap \cross \phicap \partial_\rho \lr{ \inv{\rho} }
+
\frac{B \rho_a^2}{2 \rho}
\frac{\phicap}{\rho} \cross \partial_\phi \phicap
=
\frac{B \rho_a^2}{2 \rho^2} \lr{ -\zcap + \phicap \cross \partial_\phi \phicap}.
\end{dmath}
%
Note that the \( \rho \) partial requires that \( \rho \ne 0 \).  To expand the cross product in the second term let \( j = \Be_1 \Be_2 \), and expand using a Geometric Algebra representation of the unit vector
%
\begin{dmath}\label{eqn:solenoidConstantField:60}
\phicap \cross \partial_\phi \phicap
=
\Be_2 e^{j \phi} \cross \lr{ \Be_2 \Be_1 \Be_2 e^{j \phi} }
=
- \Be_1 \Be_2 \Be_3
\gpgradetwo{
\Be_2 e^{j \phi} (-\Be_1) e^{j \phi}
}
=
\Be_1 \Be_2 \Be_3 \Be_2 \Be_1
= \Be_3
= \zcap.
\end{dmath}
%
So, provided \( \rho \ne 0 \), \( \BB = 0 \).

The errata \citep{sakurai2014modernErrata} provides the clarification, showing that a \( \rho > \rho_a \) constraint is required for this potential to produce the desired results.  Continuity at \( \rho = \rho_a \) means that in the interior (or at least on the boundary) we must have one of
%
\begin{dmath}\label{eqn:solenoidConstantField:80}
\BA = \frac{B \rho_a}{2} \phicap,
\end{dmath}
%
or
%
\begin{dmath}\label{eqn:solenoidConstantField:100}
\BA = \frac{B \rho}{2} \phicap.
\end{dmath}
%
The first doesn't work, but the second does
%
\begin{dmath}\label{eqn:solenoidConstantField:120}
\BB
= \spacegrad \cross \BA
= \lr{ \rhocap \partial_\rho + \zcap \partial_z + \frac{\phicap}{\rho} \partial_\phi } \cross \frac{B \rho}{2 } \phicap
=
\frac{B }{2 } \rhocap \cross \phicap
+
\frac{B \rho}{2 }
\frac{\phicap}{\rho} \cross \partial_\phi \phicap
= B \zcap.
\end{dmath}
%
So the vector potential that we want for a constant \( B \zcap \) field in the interior \( \rho < \rho_a \) of a cylindrical space, we need
%
\begin{dmath}\label{eqn:solenoidConstantField:140}
\BA =
\left\{
\begin{array}{l l}
\frac{B \rho_a^2}{2 \rho} \phicap & \quad \mbox{if \( \rho \ge \rho_a \) } \\
\frac{B \rho}{2} \phicap & \quad \mbox{if \( \rho < \rho_a \).}
\end{array}
\right.
\end{dmath}
%
An example of the magnitude of potential is graphed in \cref{fig:solenoidPotential:solenoidPotentialFig1}.

\mathImageFigure{../figures/phy1520-quantum/solenoidPotentialFig1}{Vector potential for constant field in cylindrical region.}{fig:solenoidPotential:solenoidPotentialFig1}{0.3}{vectorSolenoid.jl}

%\EndArticle
