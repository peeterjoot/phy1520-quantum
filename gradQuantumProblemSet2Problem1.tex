%
% Copyright � 2015 Peeter Joot.  All Rights Reserved.
% Licenced as described in the file LICENSE under the root directory of this GIT repository.
%
%
\makeoproblem{Coherent States.}{gradQuantum:problemSet2:1}{phy1520 2015 ps2.1, and \citep{sakurai2014modern} pr. 2.19(c)}{
\index{coherent state}

Consider the harmonic oscillator Hamiltonian \( H = p^2/2m + m \omega^2 x^2/2\). Define the coherent state \( \ket{z} \) as the eigenfunction of the annihilation operator, via \( a \ket{z} = z \ket{z} \), where \( a \) is the oscillator annihilation operator and \( z \) is some complex number which characterizes the coherent state.
%
\makesubproblem{}{gradQuantum:problemSet2:1a}
Expanding \( \ket{ z } \) in terms of oscillator energy eigenstates \( \ket{n} \), show that \( \ket{z} = C e^{z a^\dagger} \ket{0} \). Find the normalization constant C.
%
\makesubproblem{}{gradQuantum:problemSet2:1b}
Calculate the overlap \( \braket{z}{z'} \) for normalized coherent states \( \ket{z} \).
%
\makesubproblem{}{gradQuantum:problemSet2:1c}
Using the wavefunction \( \ket{z} \), compute \( \expectation{x}, \expectation{p}, \expectation{x^2}\), and \( \expectation{p^2} \) by defining \( x,p \) in terms of \( a, a^\dagger \).
%
\makesubproblem{}{gradQuantum:problemSet2:1d}
%
The time evolution of any observed quantity in quantum mechanics can be described in two ways:

\begin{enumerate}[(a)]
\item Schr\"{o}dinger: the wavefunction evolves as \( \ket{\psi(t)} = e^{-i H t/\Hbar} \ket{\psi(0)} \) and the operator \( A \) is time-independent, or
\item Heisenberg: the wavefunction is fixed to its value at \( t = 0 \), say \(  \ket{\psi} \) , and operators evolve as \( A(t) = e^{i H t/\Hbar} A e^{-i H t/\Hbar}\) .
\end{enumerate}

Show that both prescriptions yield the same result for any matrix elements or measured quantities.
%
\makesubproblem{}{gradQuantum:problemSet2:1e}
Using the Heisenberg picture, compute the time evolution of \( \expectation{x(t)}, \expectation{p(t)}, \expectation{x^2(t)}\), and \( \expectation{p^2(t)} \) in the coherent state \( \ket{z} \).  Comment on connections to classical dynamics of the oscillator in phase space.
%
\makesubproblem{}{gradQuantum:problemSet2:1f}
%
Show that \( \Abs{f(n)}^2 \) for a coherent state written as
%
\begin{dmath}\label{eqn:gradQuantumProblemSet2Problem1:561}
\ket{z} = \sum_{n=0}^\infty f(n) \ket{n}
\end{dmath}

has the form of a Poisson distribution, and find the most probable value of \( n\), and thus the most probable energy.
%
} % makeproblem
%
\makeanswer{gradQuantum:problemSet2:1}{
\withproblemsetsParagraph{
\makeSubAnswer{}{gradQuantum:problemSet2:1a}
%
Let
%
\begin{dmath}\label{eqn:gradQuantumProblemSet2Problem1:20}
\ket{z} = \sum_{n=0}^\infty c_n \ket{n},
\end{dmath}
%
The defining identity for \( \ket{z} \) becomes
%
\begin{dmath}\label{eqn:gradQuantumProblemSet2Problem1:40}
a \ket{z}
=
\sum_{n=0}^\infty c_n a \ket{n}
=
\sum_{n=1}^\infty c_n \sqrt{n} \ket{n-1}
=
\sum_{n=0}^\infty c_{n+1} \sqrt{n+1} \ket{n}
=
\sum_{n=0}^\infty c_{n} z \ket{n}.
\end{dmath}
%
Equating like terms provides a recurrence relation for \( c_n \)
%
\begin{dmath}\label{eqn:gradQuantumProblemSet2Problem1:60}
c_{n} = \frac{c_{n-1} z}{\sqrt{n}},
\end{dmath}
%
or
%
\begin{dmath}\label{eqn:gradQuantumProblemSet2Problem1:80}
\begin{aligned}
c_1 &= \frac{c_0 z}{\sqrt{1}} \\
c_2 &= \frac{c_1 z}{\sqrt{2}} = \frac{c_0 z^2}{\sqrt{2 \times 1}}  \\
c_3 &= \frac{c_2 z}{\sqrt{3}} = \frac{c_0 z^3}{\sqrt{3!}},
\end{aligned}
\end{dmath}

or more generally
%
\boxedEquation{eqn:gradQuantumProblemSet2Problem1:100}{
c_n = \frac{c_0 z^n}{\sqrt{n!}}.
}

or
\begin{dmath}\label{eqn:gradQuantumProblemSet2Problem1:120}
\ket{z}
= c_0
\sum_{n=0}^\infty \frac{z^n}{\sqrt{n!}} \ket{n}.
\end{dmath}
%
A similar recurrence relation can be constructed for \( \ket{n} \)
%
\begin{dmath}\label{eqn:gradQuantumProblemSet2Problem1:140}
\ket{n}
= a^\dagger \frac{\ket{ n - 1 }}{\sqrt{n}}
= (a^\dagger)^2 \frac{\ket{ n - 2 }}{\sqrt{(n)(n-1)}}
= (a^\dagger)^{n-1} \frac{\ket{ n - (n-1) }}{\sqrt{n(n-1)(n-2)...(n- (n-2))}}
= (a^\dagger)^{n} \frac{\ket{ 0 }}{\sqrt{n!}},
\end{dmath}
%
so that
%
\begin{dmath}\label{eqn:gradQuantumProblemSet2Problem1:160}
\ket{z}
= c_0
\sum_{n=0}^\infty \frac{(a^\dagger z)^n}{n!} \ket{0},
\end{dmath}
%
or
\boxedEquation{eqn:gradQuantumProblemSet2Problem1:180}{
\ket{z}
= c_0 e^{a^\dagger z} \ket{0}.
}

The normalization follows nicely from the \cref{eqn:gradQuantumProblemSet2Problem1:120} representation
%
\begin{dmath}\label{eqn:gradQuantumProblemSet2Problem1:200}
\begin{aligned}
\braket{z}{z}
&=
\Abs{c_0}^2
\sum_{n,m=0}^\infty
\bra{m} \frac{(z^\conj)^m}{\sqrt{m!}}
\frac{z^n}{\sqrt{n!}} \ket{n} \\
&= \Abs{c_0}^2
\sum_{n=0}^\infty
\bra{n} \frac{\Abs{z}^{2n}}{n!} \ket{n} \\
&= \Abs{c_0}^2
\sum_{n=0}^\infty
\frac{\Abs{z}^{2n}}{n!}  \\
&= \Abs{c_0}^2 e^{\Abs{z}^2} \\
&= 1.
\end{aligned}
\end{dmath}
%
Picking a real value for the constant provides a z-dependent normalization for the state
%
\boxedEquation{eqn:gradQuantumProblemSet2Problem1:220}{
c_0 = e^{-\Abs{z}^2/2},
}

or
\begin{dmath}\label{eqn:gradQuantumProblemSet2Problem1:240}
\ket{z} = e^{-\Abs{z}^2/2 + a^\dagger z} \ket{0}.
\end{dmath}
%
\makeSubAnswer{}{gradQuantum:problemSet2:1b}
%
\begin{dmath}\label{eqn:gradQuantumProblemSet2Problem1:260}
\begin{aligned}
\braket{z}{z'}
&=
e^{-\Abs{z}^2/2 } e^{-\Abs{z'}^2/2 }
\sum_{n,m=0}^\infty
\bra{m} \frac{(z^\conj)^m}{\sqrt{m!}}
\frac{(z')^n}{\sqrt{n!}} \ket{n}  \\
&=
e^{-\Abs{z}^2/2 -\Abs{z'}^2/2 }
\sum_{n=0}^\infty
\bra{n} \frac{(z^\conj)^n}{\sqrt{n!}}
\frac{(z')^n}{\sqrt{n!}} \ket{n}  \\
&=
\exp\lr{ -\Abs{z}^2/2 -\Abs{z'}^2/2 + z^\conj z' }.
\end{aligned}
\end{dmath}
%
This can be rewritten in terms of the absolute difference between the two z values
%
\begin{dmath}\label{eqn:gradQuantumProblemSet2Problem1:280}
\braket{z}{z'} =
\exp\lr{ -\inv{2} \lr{ \Abs{z - z'}^2 - 2 i \Imag\lr{ z' z^\conj } } },
\end{dmath}
%
however I'm not sure that's any prettier.
%
\makeSubAnswer{}{gradQuantum:problemSet2:1c}
%
First note that
%
\begin{dmath}\label{eqn:gradQuantumProblemSet2Problem1:300}
\bra{z} a^\dagger
=
\lr{a \ket{z}}^\dagger
=
\lr{z \ket{z}}^\dagger
=
\bra{z} z^\conj,
\end{dmath}
%
so
%
\begin{dmath}\label{eqn:gradQuantumProblemSet2Problem1:320}
\expectation{x}
=
\frac{x_0}{\sqrt{2}} \bra{z} a + a^\dagger \ket{z}
=
\frac{x_0}{\sqrt{2}} \bra{z} z + z^\conj \ket{z}
=
\frac{x_0}{\sqrt{2}} \lr{ z + z^\conj }
%=
%\frac{2 x_0}{\sqrt{2}} \Real z
%=
%\sqrt{2} x_0 \Real z
%=
%\sqrt{\frac{2 \Hbar}{ m \omega}} \Real z,
=
\sqrt{\frac{2 \Hbar}{ m \omega}} \frac{ z + z^\conj }{2},
\end{dmath}
%
\begin{dmath}\label{eqn:gradQuantumProblemSet2Problem1:340}
\expectation{p}
=
\frac{i \Hbar}{x_0 \sqrt{2}} \bra{z} a^\dagger - a\ket{z}
=
\frac{-i \Hbar}{x_0 \sqrt{2}} \bra{z} z - z^\conj \ket{z}
%=
%\frac{\sqrt{2} \Hbar}{x_0} \Imag z
%=
%\sqrt{ 2 m \Hbar \omega } \Imag z,
=
\sqrt{ 2 m \Hbar \omega } \frac{z - z^\conj}{2i}
\end{dmath}
%
\begin{dmath}\label{eqn:gradQuantumProblemSet2Problem1:360}
\expectation{x^2}
=
\frac{x_0^2}{2} \bra{z} \lr{a + a^\dagger}^2 \ket{z}
=
\frac{x_0^2}{2} \bra{z} \lr{a^2 + (a^\dagger)^2 + a a^\dagger + a^\dagger a} \ket{z}
=
\frac{x_0^2}{2} \bra{z} \lr{a^2 + (a^\dagger)^2 + \antisymmetric{a}{a^\dagger} + 2 a^\dagger a} \ket{z}
=
\frac{x_0^2}{2} \lr{z^2 + (z^\conj)^2 + 1 + 2 z^\conj z}
=
\frac{x_0^2}{2} \lr{(z + z^\conj)^2 + 1}
=
\frac{\Hbar}{2 m \omega} \lr{(z + z^\conj)^2 + 1},
\end{dmath}
%
and
\begin{dmath}\label{eqn:gradQuantumProblemSet2Problem1:380}
\expectation{p^2}
=
\frac{- \Hbar^2}{2 x_0^2} \bra{z} \lr{a^\dagger - a}^2 \ket{z}
=
\frac{- \Hbar^2}{2 x_0^2} \bra{z} \lr{(a^\dagger)^2 + a^2 - a a^\dagger - a^\dagger a} \ket{z}
=
\frac{- \Hbar^2}{2 x_0^2} \bra{z} \lr{(a^\dagger)^2 + a^2 - \antisymmetric{a}{a^\dagger} - 2 a^\dagger a} \ket{z}
=
\frac{- \Hbar^2}{2 x_0^2} \lr{(z^\conj)^2 + z^2 - 1 - 2 z^\conj z}
=
\frac{m \Hbar \omega}{2} \lr{ 1 - (z - z^\conj)^2 }.
\end{dmath}
%
As a check against what was stated in class, observe that the minimum uncertainty are satisfied
%
\begin{dmath}\label{eqn:gradQuantumProblemSet2Problem1:400}
\expectation{x^2} - \expectation{x}^2
=
\frac{x_0^2}{2} \lr{ (z + z^\conj)^2 + 1 - (z + z^\conj)^2 }
= \frac{x_0^2}{2},
\end{dmath}
%
and
\begin{dmath}\label{eqn:gradQuantumProblemSet2Problem1:420}
\expectation{p^2} - \expectation{p}^2
=
\frac{\Hbar^2}{2 x_0^2} \lr{ 1 - (z + z^\conj)^2 - -(z - z^\conj)^2 }
=
\frac{\Hbar^2}{2 x_0^2},
\end{dmath}
%
so we have
\begin{dmath}\label{eqn:gradQuantumProblemSet2Problem1:440}
\Delta x \Delta p = \frac{\Hbar}{2}.
\end{dmath}
%
\makeSubAnswer{}{gradQuantum:problemSet2:1d}
%
Suppose that \( \setlr{ \ket{\psi} } \) is a basis for the observable \( A \).  In the Heisenberg picture the matrix element for that operator is
%
\begin{dmath}\label{eqn:gradQuantumProblemSet2Problem1:501}
\bra{\psi} A(t) \ket{\psi'}
=
\bra{\psi} e^{i H t/\Hbar} A e^{-i H t/\Hbar} \ket{\psi'}
=
\sum_{\psi'',\psi'''}
\bra{\psi} e^{i H t/\Hbar} \ket{\psi''} \bra{\psi''} A \ket{\psi'''} \bra{\psi'''} e^{-i H t/\Hbar} \ket{\psi'}.
\end{dmath}
%
This product of three matrix elements has the structure of a similarity transformation \( \tilde{U}^\dagger \tilde{A} \tilde{U} \), where \( \tilde{U} \) is the matrix element of the time evolution operator and \( \tilde{A} \) is the matrix element of the observable \( A \).

Compare this to the Schr\"{o}dinger picture matrix element with respect to time evolved states
%
\begin{dmath}\label{eqn:gradQuantumProblemSet2Problem1:521}
\bra{\psi(t)} A \ket{\psi'(t)}
=
\lr{ \bra{\psi} e^{i H t/\Hbar} } A \lr{ e^{-i H t/\Hbar} \ket{\psi'} }
=
\sum_{\psi'',\psi'''}
\bra{\psi} e^{i H t/\Hbar} \ket{\psi''} \bra{\psi''} A \ket{\psi'''} \bra{\psi'''} e^{-i H t/\Hbar} \ket{\psi'}.
\end{dmath}
%
This has exactly the same structure as in the Heisenberg picture.  Since average quantities are matrix elements with respect to the same pair of states, this shows that measurements are independent of whether the Heisenberg or Schr\"{o}dinger picture is used to describe those measurements.
%
\makeSubAnswer{}{gradQuantum:problemSet2:1e}
%
With time evolution in the mix using the Heisenberg representation of the annihilation operator \( a(t) = a e^{-i \omega t} \), the \( x \) expectation is
%
\begin{dmath}\label{eqn:gradQuantumProblemSet2Problem1:460}
\expectation{x(t)}
=
\frac{x_0}{\sqrt{2}} \bra{z} \lr{ a e^{-i \omega t} + a^\dagger e^{i \omega t} } \ket{z}
=
\frac{x_0}{\sqrt{2}} \lr{ z e^{-i \omega t} + z^\conj e^{i \omega t} } .
\end{dmath}
%
It's clear how to generalize the stationary state calculations in \partref{gradQuantum:problemSet2:1c}, and can do so by inspection
%
\begin{dmath}\label{eqn:gradQuantumProblemSet2Problem1:480}
\begin{aligned}
\expectation{x} &= \sqrt{ \frac{\Hbar}{2 m \omega} } \lr{ z e^{-i \omega t} + z^\conj e^{i \omega t} } \\
\expectation{xp} &= \frac{i \Hbar}{2} \lr{ z e^{-i \omega t} + z^\conj e^{i \omega t} } \lr{ z e^{-i \omega t} - z^\conj e^{i \omega t} } \\
\expectation{p} &= -i \sqrt{ \frac{m \Hbar \omega}{2} } \lr{ z e^{-i \omega t} - z^\conj e^{i \omega t} } \\
\expectation{x^2} &= \frac{\Hbar}{2 m \omega} \lr{(z e^{-i \omega t} + z^\conj e^{i \omega t} )^2 + 1} = \expectation{x}^2 + \frac{\Hbar}{2 m \omega} \\
\expectation{p^2} &= \frac{m \Hbar \omega}{2} \lr{ 1 - (z e^{-i \omega t} - z^\conj e^{i \omega t})^2 } = \expectation{p}^2 + \frac{m \Hbar \omega}{2}.
\end{aligned}
\end{dmath}
%
In class, symmetric and antisymmetric conjugate sums of \( z \) were identified as position and momentum, with
%
\begin{dmath}\label{eqn:gradQuantumProblemSet2Problem1:541}
\begin{aligned}
x_0 &\equiv \sqrt{\frac{2 \Hbar}{m \omega}} \Real z = \sqrt{\frac{\Hbar}{2 m \omega}} \lr{ z + z^\conj } \\
p_0 &\equiv \sqrt{2 m \Hbar \omega} \Imag z = -i \sqrt{\frac{m \Hbar \omega}{2}} \lr{ z - z^\conj }.
\end{aligned}
\end{dmath}
%
With that identification the expectation values above are
%
\begin{dmath}\label{eqn:gradQuantumProblemSet2Problem1:481}
\begin{aligned}
\expectation{x} &= x_0 \cos(\omega t) + \frac{p_0}{m \omega} \sin(\omega t) \\
\expectation{p} &= p_0 \cos(\omega t) - m \omega x_0 \sin(\omega t) \\
\end{aligned}
\end{dmath}

These expectations are analogous to the phase space trajectories of classical particles.
%
\makeSubAnswer{}{gradQuantum:problemSet2:1f}
%
The Poisson distribution has the form
%
\begin{dmath}\label{eqn:gradQuantumProblemSet2Problem1:581}
P(n) = \frac{\mu^{n} e^{-\mu}}{n!}.
\end{dmath}
%
Here \( \mu \) is the mean of the distribution
%
\begin{dmath}\label{eqn:gradQuantumProblemSet2Problem1:601}
\expectation{n}
= \sum_{n=0}^\infty n P(n)
= \sum_{n=1}^\infty n \frac{\mu^{n} e^{-\mu}}{n!}
= \mu e^{-\mu} \sum_{n=1}^\infty \frac{\mu^{n-1}}{(n-1)!}
= \mu e^{-\mu} e^{\mu}
= \mu.
\end{dmath}
%
We found that the coherent state had the form
%
\begin{dmath}\label{eqn:gradQuantumProblemSet2Problem1:621}
\ket{z} = c_0 \sum_{n=0} \frac{z^n}{\sqrt{n!}} \ket{n},
\end{dmath}
%
so the probability coefficients for \( \ket{n} \) are
%
\begin{dmath}\label{eqn:gradQuantumProblemSet2Problem1:641}
P(n)
= c_0^2 \frac{\Abs{z^n}^2}{n!}
= e^{-\Abs{z}^2} \frac{\Abs{z^n}^2}{n!}.
\end{dmath}
%
This has the structure of the Poisson distribution with mean \( \mu = \Abs{z}^2 \).  The most probable value of \( n \) is that for which \( \Abs{f(n)}^2 \) is the largest.  This is, in general, hard to compute, since we have a maximization problem in the integer domain that falls outside the normal toolbox.  If we assume that \( n \) is large, so that Stirling's approximation can be used to approximate the factorial, and also seek a non-integer value that maximizes the distribution, the most probable value will be the closest integer to that, and this can be computed.  Let
%
\begin{dmath}\label{eqn:gradQuantumProblemSet2Problem1:661}
g(n)
= \Abs{f(n)}^2
= \frac{e^{-\mu} \mu^n}{n!}
= \frac{e^{-\mu} \mu^n}{e^{\ln n!}}
\approx e^{-\mu - n \ln n + n } \mu^n.
= e^{-\mu - n \ln n + n + n \ln \mu }
\end{dmath}
%
This is maximized when
%
\begin{dmath}\label{eqn:gradQuantumProblemSet2Problem1:681}
0
= \frac{dg}{dn}
= \lr{ - \ln n - 1 + 1 + \ln \mu } g(n),
\end{dmath}
%
which is maximized at \( n = \mu \).  One of the integers \( n = \lfloor \mu \rfloor \) or \( n = \lceil \mu \rceil \) that brackets this value \( \mu = \Abs{z}^2 \) is the most probable.  So, if an energy measurement is made of a coherent state \( \ket{z} \), the most probable value will be one of
%
\begin{dmath}\label{eqn:gradQuantumProblemSet2Problem1:701}
E = \Hbar \lr{
\largestIntLessThan{\Abs{z}^2}
 + \inv{2} },
\end{dmath}
%
or
%
\begin{dmath}\label{eqn:gradQuantumProblemSet2Problem1:721}
E = \Hbar \lr{
\largestIntGreaterThan{\Abs{z}^2}
 + \inv{2} },
\end{dmath}
%
}
}
