%
% Copyright � 2015 Peeter Joot.  All Rights Reserved.
% Licenced as described in the file LICENSE under the root directory of this GIT repository.
%
%\input{../blogpost.tex}
%\renewcommand{\basename}{qmLecture10}
%\renewcommand{\dirname}{notes/phy1520/}
%\newcommand{\keywords}{PHY1520H}
%\input{../peeter_prologue_print2.tex}
%
%%\usepackage{phy1520}
%\usepackage{peeters_braket}
%%\usepackage{peeters_layout_exercise}
%\usepackage{peeters_figures}
%\usepackage{mathtools}
%
%\beginArtNoToc
%\generatetitle{PHY1520H Graduate Quantum Mechanics.  Lecture 10: 1D Dirac scattering off potential step.  Taught by Prof.\ Arun Paramekanti}
%%\chapter{1D Dirac scattering off potential step}
%\label{chap:qmLecture10}
%
%\paragraph{Disclaimer}
%
%Peeter's lecture notes from class.  These may be incoherent and rough.
%
%These are notes for the UofT course PHY1520, Graduate Quantum Mechanics, taught by Prof. Paramekanti.

\section{Dirac scattering off a potential step}
\index{Dirac equation!scattering}

For the non-relativistic case we have

\begin{equation}\label{eqn:qmLecture10:20}
\begin{aligned}
E < V_0 &\implies T = 0, R = 1 \\
E > V_0 &\implies T > 0, R < 1.
\end{aligned}
\end{equation}

What happens for a relativistic 1D particle?

Referring to \cref{fig:lecture10:lecture10Fig1}.

\imageFigure{../phy1520-quantum-figures/lecture10Fig1}{Potential step.}{fig:lecture10:lecture10Fig1}{0.1}

the region I Hamiltonian is

\begin{equation}\label{eqn:qmLecture10:40}
H =
\begin{bmatrix}
\hatp c & m c^2 \\
m c^2 & - \hatp c
\end{bmatrix},
\end{equation}

for which the solution is

\begin{dmath}\label{eqn:qmLecture10:60}
\Phi = e^{i k_1 x }
\begin{bmatrix}
\cos \theta_1 \\
\sin \theta_1
\end{bmatrix},
\end{dmath}

where
\begin{dmath}\label{eqn:qmLecture10:80}
\begin{aligned}
\cos 2 \theta_1 &= \frac{ \Hbar c k_1 }{E_{k_1}} \\
\sin 2 \theta_1 &= \frac{ m c^2 }{E_{k_1}} \\
\end{aligned}
\end{dmath}

To consider the \( k_1 < 0 \) case, note that

\begin{equation}\label{eqn:qmLecture10:100}
\begin{aligned}
\cos^2 \theta_1 - \sin^2 \theta_1 &= \cos 2 \theta_1 \\
2 \sin\theta_1 \cos\theta_1 &= \sin 2 \theta_1
\end{aligned}
\end{equation}

so after flipping the signs on all the \( k_1 \) terms we find for the reflected wave

\begin{dmath}\label{eqn:qmLecture10:120}
\Phi = e^{-i k_1 x}
\begin{bmatrix}
\sin\theta_1 \\
\cos\theta_1
\end{bmatrix}.
\end{dmath}

FIXME: this reasoning doesn't entirely make sense to me.  Make sense of this by trying this solution as was done for the form of the incident wave solution.

The region I wave has the form

\begin{dmath}\label{eqn:qmLecture10:140}
\Phi_I
=
A e^{i k_1 x}
\begin{bmatrix}
\cos\theta_1 \\
\sin\theta_1 \\
\end{bmatrix}
+
B e^{-i k_1 x}
\begin{bmatrix}
\sin\theta_1 \\
\cos\theta_1 \\
\end{bmatrix}.
\end{dmath}

By the time we are done we want to have computed the reflection coefficient

\begin{dmath}\label{eqn:qmLecture10:160}
R =
\frac{\Abs{B}^2}{\Abs{A}^2}.
\end{dmath}

The region I energy is

\begin{dmath}\label{eqn:qmLecture10:180}
E = \sqrt{ \lr{ m c^2}^2 + \lr{ \Hbar c k_1 }^2 }.
\end{dmath}

We must have
\begin{equation}\label{eqn:qmLecture10:200}
E
=
\sqrt{ \lr{ m c^2}^2 + \lr{ \Hbar c k_2 }^2 } + V_0
=
\sqrt{ \lr{ m c^2}^2 + \lr{ \Hbar c k_1 }^2 },
\end{equation}

so

\begin{dmath}\label{eqn:qmLecture10:220}
\lr{ \Hbar c k_2 }^2
=
\lr{ E - V_0 }^2 - \lr{ m c^2}^2
=
\mathLabelBox
[ labelstyle={below of=m\themathLableNode, below of=m\themathLableNode} ]
{\lr{ E - V_0 + m c^2 }}{\(r_1\)}
\mathLabelBox
[ labelstyle={below of=m\themathLableNode, below of=m\themathLableNode} ]
{\lr{ E - V_0 - m c^2 }}{\(r_2\)}.
\end{dmath}

The \( r_1 \) and \( r_2 \) branches are sketched in \cref{fig:lecture10:lecture10Fig2}.

\imageFigure{../phy1520-quantum-figures/lecture10Fig2}{Energy signs.}{fig:lecture10:lecture10Fig2}{0.2}

For low energies, we have a set of potentials for which we will have propagation, despite having a potential barrier.  For still higher values of the potential barrier the product \( r_1 r_2 \) will be negative, so the solutions will be decaying.  Finally, for even higher energies, there will again be propagation.

The non-relativistic case is sketched in \cref{fig:lecture10:lecture10Fig3}.

\imageFigure{../phy1520-quantum-figures/lecture10Fig3}{Effects of increasing potential for non-relativistic case.}{fig:lecture10:lecture10Fig3}{0.1}

For the relativistic case we must consider three different cases, sketched in
\cref{fig:lecture10:lecture10Fig4a},
\cref{fig:lecture10:lecture10Fig4b}, and
\cref{fig:lecture10:lecture10Fig4c} respectively.  For the low potential energy, a particle with positive group velocity (what we've called right moving) can be matched to an equal energy portion of the potential shifted parabola in region II.  This is a case where we have transmission, but no antiparticle creation.  There will be an energy region where the region II wave function has only a dissipative term, since there is no region of either of the region II parabolic branches available at the incident energy.  When the potential is shifted still higher so that \( V_0 > E + m c^2 \), a positive group velocity in region I with a given energy can be matched to an antiparticle branch in the region II parabolic energy curve.

\imageFigure{../phy1520-quantum-figures/lecture10Fig4a}{Low potential energy.}{fig:lecture10:lecture10Fig4a}{0.1}
\imageFigure{../phy1520-quantum-figures/lecture10Fig4b}{High enough potential energy for no propagation.}{fig:lecture10:lecture10Fig4b}{0.1}
\imageFigure{../phy1520-quantum-figures/lecture10Fig4c}{High potential energy.}{fig:lecture10:lecture10Fig4c}{0.1}

\paragraph{Boundary value conditions}
\index{Dirac equation!boundary conditions}
We want to ensure that the current across the barrier is conserved (no particles are lost), as sketched in \cref{fig:lecture10:lecture10Fig5}.

\imageFigure{../phy1520-quantum-figures/lecture10Fig5}{Transmitted, reflected and incident components.}{fig:lecture10:lecture10Fig5}{0.1}

Recall that given a wave function

\begin{dmath}\label{eqn:qmLecture10:240}
\Psi =
\begin{bmatrix}
\psi_1 \\
\psi_2
\end{bmatrix},
\end{dmath}

the density and currents are respectively

\begin{equation}\label{eqn:qmLecture10:260}
\begin{aligned}
\rho &= \psi_1^\conj \psi_1 + \psi_2^\conj \psi_2 \\
j &= \psi_1^\conj \psi_1 - \psi_2^\conj \psi_2
\end{aligned}
\end{equation}

Matching boundary value conditions requires

\begin{enumerate}
\item For both the relativistic and non-relativistic cases we must have

\begin{equation}\label{eqn:qmLecture10:280}
\Psi_\txtL = \Psi_\txtR, \qquad \mbox{at \( x = 0 \).}
\end{equation}
\item For the non-relativistic case we want
\begin{equation}\label{eqn:qmLecture10:300}
\int_{-\epsilon}^\epsilon -\frac{\Hbar^2}{2m} \PDSq{x}{\Psi} =
\cancel{\int_{-\epsilon}^\epsilon \lr{ E - V(x) } \Psi(x)}
\end{equation}

\begin{equation}\label{eqn:qmLecture10:320}
-\frac{\Hbar^2}{2m} \lr{ \evalbar{\PD{x}{\Psi}}{\txtR} - \evalbar{\PD{x}{\Psi}}{\txtL} }  = 0.
\end{equation}

We have to match

For the relativistic case

\begin{dmath}\label{eqn:qmLecture10:460}
-i \Hbar \sigma_z \int_{-\epsilon}^\epsilon \PD{x}{\Psi} +
\cancel{m c^2 \sigma_x \int_{-\epsilon}^\epsilon \psi}
=
\cancel{\int_{-\epsilon}^\epsilon \lr{ E - V_0 } \psi},
\end{dmath}
\end{enumerate}

so

\begin{equation}\label{eqn:qmLecture10:340}
-i \Hbar c \sigma_z \lr{ \psi(\epsilon) - \psi(-\epsilon) }
=
-i \Hbar c \sigma_z \lr{ \psi_\txtR - \psi_\txtL }.
\end{equation}

so we must match

\begin{equation}\label{eqn:qmLecture10:360}
\sigma_z \psi_\txtR = \sigma_z \psi_\txtL .
\end{equation}

It appears that things are simpler, because we only have to match the wave function values at the boundary, and don't have to match the derivatives too.  However, we have a two component wave function, so there are still two tasks.

\paragraph{Solving the system}

Let's look for a solution for the \( E + m c^2 > V_0 \) case on the right branch, as sketched in \cref{fig:lecture10:lecture10Fig6}.

\imageFigure{../phy1520-quantum-figures/lecture10Fig6}{High potential region.  Anti-particle transmission.}{fig:lecture10:lecture10Fig6}{0.15}

While the right branch in this case is left going, this might work out since that is an antiparticle.  We could try both.

Try

\begin{dmath}\label{eqn:qmLecture10:480}
\Psi_{II} = D e^{i k_2 x}
\begin{bmatrix}
-\sin\theta_2 \\
\cos\theta_2
\end{bmatrix}.
\end{dmath}

This is justified by

\begin{dmath}\label{eqn:qmLecture10:500}
+E \rightarrow
\begin{bmatrix}
\cos\theta \\
\sin\theta
\end{bmatrix},
\end{dmath}

so

\begin{dmath}\label{eqn:qmLecture10:520}
-E \rightarrow
\begin{bmatrix}
-\sin\theta \\
\cos\theta \\
\end{bmatrix}
\end{dmath}

At \( x = 0 \) the exponentials vanish, so equating the waves at that point means

\begin{dmath}\label{eqn:qmLecture10:380}
\begin{bmatrix}
\cos\theta_1 \\
\sin\theta_1 \\
\end{bmatrix}
+
\frac{B}{A}
\begin{bmatrix}
\sin\theta_1 \\
\cos\theta_1 \\
\end{bmatrix}
=
\frac{D}{A}
\begin{bmatrix}
-\sin\theta_2 \\
\cos\theta_2
\end{bmatrix}.
\end{dmath}

Solving this yields

\begin{dmath}\label{eqn:qmLecture10:400}
\frac{B}{A} = - \frac{\cos(\theta_1 - \theta_2)}{\sin(\theta_1 + \theta_2)}.
\end{dmath}

This yields

\boxedEquation{eqn:qmLecture10:420}{
R = \frac{1 + \cos( 2 \theta_1 - 2 \theta_2) }{1 - \cos( 2 \theta_1 - 2 \theta_2)}.
}

As \( V_0 \rightarrow \infty \) this simplifies to

\begin{dmath}\label{eqn:qmLecture10:440}
R = \frac{ E - \sqrt{ E^2 - \lr{ m c^2 }^2 } }{ E + \sqrt{ E^2 - \lr{ m c^2 }^2 } }.
\end{dmath}

Filling in the details for these results part of problem set 4.

%\EndNoBibArticle
