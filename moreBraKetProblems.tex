%
% Copyright � 2015 Peeter Joot.  All Rights Reserved.
% Licenced as described in the file LICENSE under the root directory of this GIT repository.
%
%\input{../blogpost.tex}
%\renewcommand{\basename}{moreBraKetProblems}
%\renewcommand{\dirname}{notes/phy1520/}
%%\newcommand{\dateintitle}{}
%%\newcommand{\keywords}{}
%
%\input{../peeter_prologue_print2.tex}
%
%\usepackage{peeters_layout_exercise}
%\usepackage{peeters_braket}
%
%\beginArtNoToc
%
%\generatetitle{Bra-ket and spin one-half problems}
%\chapter{Bra-ket and spin one-half problems}
%\label{chap:moreBraKetProblems}


\makeoproblem{Operator matrix representation.}{problem:moreBraKetProblems:1.5}{\citep{sakurai2014modern} pr. 1.5}{
\index{matrix representation}

\makesubproblem{}{problem:moreBraKetProblems:1.5:a}

Determine the matrix representation of \( \ket{\alpha}\bra{\beta} \) given a complete set of eigenvectors \( \ket{a^r} \).

\makesubproblem{}{problem:moreBraKetProblems:1.5:b}

Verify with \( \ket{\alpha} = \ket{s_z = \Hbar/2}, \ket{s_x = \Hbar/2} \).

} % problem

\makeanswer{problem:moreBraKetProblems:1.5}{

\makeSubAnswer{}{problem:moreBraKetProblems:1.5:a}

Forming the matrix element
%
\begin{dmath}\label{eqn:moreBraKetProblems:20}
\bra{a^r} \lr{ \ket{\alpha}\bra{\beta} } \ket{a^s}
=
\braket{a^r}{\alpha}\braket{\beta}{a^s}
=
\braket{a^r}{\alpha}
\braket{a^s}{\beta}^\conj,
\end{dmath}
%
the matrix representation is seen to be
%
\begin{dmath}\label{eqn:moreBraKetProblems:40}
\ket{\alpha}\bra{\beta}
\sim
\begin{bmatrix}
\bra{a^1} \lr{ \ket{\alpha}\bra{\beta} } \ket{a^1} & \bra{a^1} \lr{ \ket{\alpha}\bra{\beta} } \ket{a^2} & \cdots \\
\bra{a^2} \lr{ \ket{\alpha}\bra{\beta} } \ket{a^1} & \bra{a^2} \lr{ \ket{\alpha}\bra{\beta} } \ket{a^2} & \cdots \\
\vdots & \vdots & \ddots \\
\end{bmatrix}
=
\begin{bmatrix}
\braket{a^1}{\alpha} \braket{a^1}{\beta}^\conj & \braket{a^1}{\alpha} \braket{a^2}{\beta}^\conj & \cdots \\
\braket{a^2}{\alpha} \braket{a^1}{\beta}^\conj & \braket{a^2}{\alpha} \braket{a^2}{\beta}^\conj & \cdots \\
\vdots & \vdots & \ddots \\
\end{bmatrix}.
\end{dmath}
%
\makeSubAnswer{}{problem:moreBraKetProblems:1.5:b}
First compute the spin-z representation of \( \ket{s_x = \Hbar/2 } \).
%
\begin{dmath}\label{eqn:moreBraKetProblems:60}
\begin{aligned}
\lr{ S_x - \Hbar/2 I }
\begin{bmatrix}
a \\
b
\end{bmatrix}
&=
\lr{
\begin{bmatrix}
0 & \Hbar/2 \\
\Hbar/2 & 0 \\
\end{bmatrix}
-
\begin{bmatrix}
\Hbar/2 & 0 \\
0 & \Hbar/2 \\
\end{bmatrix}
} \\
&=
\begin{bmatrix}
a \\
b
\end{bmatrix} \\
&=
\frac{\Hbar}{2}
\begin{bmatrix}
-1 & 1 \\
1 & -1 \\
\end{bmatrix}
\begin{bmatrix}
a \\
b
\end{bmatrix},
\end{aligned}
\end{dmath}
so \( \ket{s_x = \Hbar/2 } \propto (1,1) \).  Normalized we have
%
\begin{equation}\label{eqn:moreBraKetProblems:80}
\begin{aligned}
\ket{\alpha} &= \ket{s_z = \Hbar/2 } =
\begin{bmatrix}
1 \\
0
\end{bmatrix} \\
\ket{\beta} &= \ket{s_z = \Hbar/2 }
\inv{\sqrt{2}}
\begin{bmatrix}
1 \\
1
\end{bmatrix}.
\end{aligned}
\end{equation}
%
Using \cref{eqn:moreBraKetProblems:40} the matrix representation is
%
\begin{dmath}\label{eqn:moreBraKetProblems:100}
\ket{\alpha}\bra{\beta}
\sim
\begin{bmatrix}
(1) (1/\sqrt{2})^\conj & (1) (1/\sqrt{2})^\conj \\
(0) (1/\sqrt{2})^\conj & (0) (1/\sqrt{2})^\conj \\
\end{bmatrix}
=
\inv{\sqrt{2}}
\begin{bmatrix}
1 & 1 \\
0 & 0
\end{bmatrix}.
\end{dmath}
%
This can be confirmed with direct computation
\begin{dmath}\label{eqn:moreBraKetProblems:120}
\ket{\alpha}\bra{\beta}
=
\begin{bmatrix}
1 \\
0
\end{bmatrix}
\inv{\sqrt{2}}
\begin{bmatrix}
1 & 1
\end{bmatrix}
=
\inv{\sqrt{2}}
\begin{bmatrix}
1 & 1 \\
0 & 0
\end{bmatrix}.
\end{dmath}
%
} % answer
%
\makeoproblem{Eigenvalue of sum of kets.}{problem:moreBraKetProblems:6}{\citep{sakurai2014modern} pr. 1.6}{
Given eigenkets \( \ket{i}, \ket{j} \) of an operator \( A \), what are the conditions that \( \ket{i} + \ket{j} \) is also an eigenvector?
} % problem

\makeanswer{problem:moreBraKetProblems:6}{
Let \( A \ket{i} = i \ket{i}, A \ket{j} = j \ket{j} \), and suppose that the sum is an eigenket.  Then there must be a value \( a \) such that
%
\begin{dmath}\label{eqn:moreBraKetProblems:140}
A \lr{ \ket{i} + \ket{j} } = a \lr{ \ket{i} + \ket{j} },
\end{dmath}
%
so
%
\begin{dmath}\label{eqn:moreBraKetProblems:160}
i \ket{i} + j \ket{j} = a \lr{ \ket{i} + \ket{j} }.
\end{dmath}
%
Operating with \( \bra{i}, \bra{j} \) respectively, gives
%
\begin{equation}\label{eqn:moreBraKetProblems:180}
\begin{aligned}
i &= a \\
j &= a,
\end{aligned}
\end{equation}
so for the sum to be an eigenket, both of the corresponding energy eigenvalues must be identical (i.e. linear combinations of degenerate eigenkets are also eigenkets).
} % answer

\makeoproblem{Null operator.}{problem:moreBraKetProblems:7}{\citep{sakurai2014modern} pr. 1.7}{
\index{null operator}
Given eigenkets \( \ket{a'} \) of operator \( A \)

\makesubproblem{}{problem:moreBraKetProblems:7:a}
show that
%
\begin{equation}\label{eqn:moreBraKetProblems:200}
\prod_{a'} \lr{ A - a' }
\end{equation}
is the null operator.

\makesubproblem{}{problem:moreBraKetProblems:7:b}
%
\begin{equation}\label{eqn:moreBraKetProblems:220}
\prod_{a'' \ne a'} \frac{\lr{ A - a'' }}{a' - a''}
\end{equation}

\makesubproblem{}{problem:moreBraKetProblems:7:c}
Illustrate using \( S_z \) for a spin 1/2 system.
} % problem

\makeanswer{problem:moreBraKetProblems:7}{
\makeSubAnswer{}{problem:moreBraKetProblems:7:a}
Application of \( \ket{a} \), the eigenket of \( A \) with eigenvalue \( a \) to any term \( A - a' \) scales \( \ket{a} \) by \( a - a' \), so the product operating on \( \ket{a} \) is
%
\begin{equation}\label{eqn:moreBraKetProblems:240}
\prod_{a'} \lr{ A - a' } \ket{a} = \prod_{a'} \lr{ a - a' } \ket{a}.
\end{equation}
%
Since \( \ket{a} \) is one of the \( \setlr{\ket{a'}} \) eigenkets of \( A \), one of these terms must be zero.

\makeSubAnswer{}{problem:moreBraKetProblems:7:b}

Again, consider the action of the operator on \( \ket{a} \),
%
\begin{equation}\label{eqn:moreBraKetProblems:260}
\prod_{a'' \ne a'} \frac{\lr{ A - a'' }}{a' - a''} \ket{a}
=
\prod_{a'' \ne a'} \frac{\lr{ a - a'' }}{a' - a''} \ket{a}.
\end{equation}
%
If \( \ket{a} = \ket{a'} \), then \( \prod_{a'' \ne a'} \frac{\lr{ A - a'' }}{a' - a''} \ket{a} = \ket{a} \), whereas if it does not, then it equals one of the \( a'' \) energy eigenvalues.  This is a representation of the Kronecker delta function
%
\begin{dmath}\label{eqn:moreBraKetProblems:300}
\prod_{a'' \ne a'} \frac{\lr{ A - a'' }}{a' - a''} \ket{a} \equiv \delta_{a', a} \ket{a}
\end{dmath}
%
\makeSubAnswer{}{problem:moreBraKetProblems:7:c}

For operator \( S_z \) the eigenvalues are \( \setlr{ \Hbar/2, -\Hbar/2 } \), so the null operator must be
%
\begin{dmath}\label{eqn:moreBraKetProblems:280}
\prod_{a'} \lr{ A - a' }
=
\lr{ \frac{\Hbar}{2} }^2 \lr{ \PauliZ - \PauliI } \lr{ \PauliZ + \PauliI }
=
\begin{bmatrix}
0 & 0 \\
0 & -2
\end{bmatrix}
\begin{bmatrix}
2 & 0  \\
0 & 0 \\
\end{bmatrix}
=
\begin{bmatrix}
0 & 0  \\
0 & 0 \\
\end{bmatrix}
\end{dmath}

For the delta representation, consider the \( \ket{\pm} \) states and their eigenvalue.  The delta operators are
%
\begin{dmath}\label{eqn:moreBraKetProblems:320}
\begin{aligned}
\prod_{a'' \ne \Hbar/2} \frac{\lr{ A - a'' }}{\Hbar/2 - a''}
&=
\frac{S_z - (-\Hbar/2) I}{\Hbar/2 - (-\Hbar/2)} \\
&=
\inv{2} \lr{ \sigma_z + I } \\
&=
\inv{2} \lr{ \PauliZ + \PauliI } \\
&=
\inv{2}
\begin{bmatrix}
2 & 0 \\
0 & 0
\end{bmatrix} \\
&=
\begin{bmatrix}
1 & 0 \\
0 & 0
\end{bmatrix}.
\end{aligned}
\end{dmath}
\begin{dmath}\label{eqn:moreBraKetProblems:340}
\begin{aligned}
\prod_{a'' \ne -\Hbar/2} \frac{\lr{ A - a'' }}{-\Hbar/2 - a''}
&=
\frac{S_z - (\Hbar/2) I}{-\Hbar/2 - \Hbar/2} \\
&=
\inv{2} \lr{ \sigma_z - I } \\
&=
\inv{2} \lr{ \PauliZ - \PauliI } \\
&=
\inv{2}
\begin{bmatrix}
0 & 0 \\
0 & -2
\end{bmatrix} \\
&=
\begin{bmatrix}
0 & 0 \\
0 & 1
\end{bmatrix}.
\end{aligned}
\end{dmath}
%
These clearly have the expected delta function property acting on kets \( \ket{+} = (1,0)^\T, \ket{-} = (0, 1)^\T \).

} % answer


\makeoproblem{Spin half general normal.}{problem:moreBraKetProblems:9}{\citep{sakurai2014modern} pr. 1.9}{

\index{spin half!states}
Construct \( \ket{\BS \cdot \ncap ; + } \), where \( \ncap = ( \cos\alpha \sin\beta, \sin\alpha \sin\beta, \cos\beta )^\T \)  such that
%
\begin{dmath}\label{eqn:moreBraKetProblems:360}
\BS \cdot \ncap \ket{\BS \cdot \ncap ; + } =
\frac{\Hbar}{2} \ket{\BS \cdot \ncap ; + },
\end{dmath}
%
Solve this as an eigenvalue problem.
} % problem

\makeanswer{problem:moreBraKetProblems:9}{

The spin operator for this direction is
%
\begin{dmath}\label{eqn:moreBraKetProblems:380}
\BS \cdot \ncap
= \frac{\Hbar}{2} \Bsigma \cdot \ncap
= \frac{\Hbar}{2}
\lr{
\cos\alpha \sin\beta \PauliX + \sin\alpha \sin\beta \PauliY + \cos\beta \PauliZ
}
=
\frac{\Hbar}{2}
\begin{bmatrix}
\cos\beta &
e^{-i\alpha}
\sin\beta
\\
e^{i\alpha}
\sin\beta
& -\cos\beta
\end{bmatrix}.
\end{dmath}
%
Observed that this is traceless and has a \( -\Hbar/2 \) determinant like any of the \( x,y,z \) spin operators.

Assuming that this has an \( \Hbar/2 \) eigenvalue (to be verified later), the eigenvalue problem is
%
\begin{dmath}\label{eqn:moreBraKetProblems:400}
0 =
\BS \cdot \ncap - \Hbar/2 I
=
\frac{\Hbar}{2}
\begin{bmatrix}
\cos\beta -1 &
e^{-i\alpha}
\sin\beta
\\
e^{i\alpha}
\sin\beta
& -\cos\beta -1
\end{bmatrix}
=
\Hbar
\begin{bmatrix}
- \sin^2 \frac{\beta}{2} &
e^{-i\alpha}
\sin\frac{\beta}{2} \cos\frac{\beta}{2}
\\
e^{i\alpha}
\sin\frac{\beta}{2} \cos\frac{\beta}{2}
& -\cos^2 \frac{\beta}{2}
\end{bmatrix}
\end{dmath}

This has a zero determinant as expected, and the eigenvector \( (a,b) \) will satisfy
%
\begin{dmath}\label{eqn:moreBraKetProblems:420}
0
= - \sin^2 \frac{\beta}{2} a +
e^{-i\alpha}
\sin\frac{\beta}{2} \cos\frac{\beta}{2}
b
= \sin\frac{\beta}{2} \lr{ - \sin \frac{\beta}{2} a +
e^{-i\alpha} b
\cos\frac{\beta}{2}
}
\end{dmath}
%
\begin{dmath}\label{eqn:moreBraKetProblems:440}
\begin{bmatrix}
a \\
b
\end{bmatrix}
\propto
\begin{bmatrix}
\cos\frac{\beta}{2} \\
e^{i\alpha}
\sin\frac{\beta}{2}
\end{bmatrix}.
\end{dmath}
%
This is appropriately normalized, so the ket for \( \BS \cdot \ncap \) is
%
\begin{dmath}\label{eqn:moreBraKetProblems:460}
\ket{ \BS \cdot \ncap ; + } =
\cos\frac{\beta}{2} \ket{+} +
e^{i\alpha}
\sin\frac{\beta}{2}
\ket{-}.
\end{dmath}
%
Note that the other eigenvalue is
%
\begin{dmath}\label{eqn:moreBraKetProblems:480}
\ket{ \BS \cdot \ncap ; - } =
-\sin\frac{\beta}{2} \ket{+} +
e^{i\alpha}
\cos\frac{\beta}{2}
\ket{-}.
\end{dmath}
%
It is straightforward to show that these are orthogonal and that this has the \( -\Hbar/2 \) eigenvalue.

} % answer

\makeoproblem{Two state Hamiltonian.}{problem:moreBraKetProblems:10}{\citep{sakurai2014modern} pr. 1.10}{
\index{Hamiltonian!two state}

Solve the eigenproblem for
%
\begin{dmath}\label{eqn:moreBraKetProblems:500}
H = a \biglr{
\ket{1}\bra{1}
-\ket{2}\bra{2}
+\ket{1}\bra{2}
+\ket{2}\bra{1}
}
\end{dmath}

} % problem

\makeanswer{problem:moreBraKetProblems:10}{

In matrix form the Hamiltonian is
%
\begin{dmath}\label{eqn:moreBraKetProblems:520}
H = a
\begin{bmatrix}
1 & 1 \\
1 & -1
\end{bmatrix}.
\end{dmath}
%
The eigenvalue problem is
%
\begin{dmath}\label{eqn:moreBraKetProblems:540}
0
= \Abs{ H - \lambda I }
= (a - \lambda)(-a - \lambda) - a^2
= (-a + \lambda)(a + \lambda) - a^2
= \lambda^2 - a^2 - a^2,
\end{dmath}
%
or
%
\begin{dmath}\label{eqn:moreBraKetProblems:560}
\lambda = \pm \sqrt{2} a.
\end{dmath}
%
An eigenket proportional to \( (\alpha,\beta) \) must satisfy
%
\begin{dmath}\label{eqn:moreBraKetProblems:580}
0
= ( 1 \mp \sqrt{2} ) \alpha + \beta,
\end{dmath}
%
so
%
\begin{dmath}\label{eqn:moreBraKetProblems:600}
\ket{\pm} \propto
\begin{bmatrix}
-1 \\
1 \mp \sqrt{2}
\end{bmatrix},
\end{dmath}
%
or
%
\begin{dmath}\label{eqn:moreBraKetProblems:620}
\ket{\pm}
=
\inv{2(2 - \sqrt{2})}
\begin{bmatrix}
-1 \\
1 \mp \sqrt{2}
\end{bmatrix}
=
\frac{2 + \sqrt{2}}{4}
\begin{bmatrix}
-1 \\
1 \mp \sqrt{2}
\end{bmatrix}.
\end{dmath}
%
That is
\begin{dmath}\label{eqn:moreBraKetProblems:640}
\ket{\pm} =
\frac{2 + \sqrt{2}}{4} \lr{
-\ket{1} + (1 \mp \sqrt{2}) \ket{2}
}.
\end{dmath}
%
} % answer

%
%\EndArticle
