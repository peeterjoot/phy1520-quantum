%
% Copyright � 2015 Peeter Joot.  All Rights Reserved.
% Licenced as described in the file LICENSE under the root directory of this GIT repository.
%
%\input{../blogpost.tex}
%\renewcommand{\basename}{qmLecture20}
%\renewcommand{\dirname}{notes/phy1520/}
%\newcommand{\keywords}{PHY1520H}
%\input{../peeter_prologue_print2.tex}
%
%%\usepackage{phy1520}
%\usepackage{peeters_braket}
%%\usepackage{peeters_layout_exercise}
%\usepackage{peeters_figures}
%\usepackage{mathtools}
%
%\beginArtNoToc
%\generatetitle{PHY1520H Graduate Quantum Mechanics.  Lecture 20: Perturbation theory.  Taught by Prof.\ Arun Paramekanti}
%%\chapter{Pertubation theory}
%\label{chap:qmLecture20}
%
%\paragraph{Disclaimer}
%
%Peeter's lecture notes from class.  These may be incoherent and rough.
%
%These are notes for the UofT course PHY1520, Graduate Quantum Mechanics, taught by Prof. Paramekanti, covering \textchapref{{5}} \citep{sakurai2014modern} content.

\section{Simplest perturbation example.}
\index{perturbation!simplest}

Given a \( 2 \times 2 \) Hamiltonian \( H = H_0 + V \), where
%
\begin{dmath}\label{eqn:qmLecture20:20}
H =
\begin{bmatrix}
a & c \\
c^\conj & b
\end{bmatrix},
\end{dmath}
%
note that if \( c = 0 \) is
%
\begin{equation}\label{eqn:qmLecture20:60}
H = H_0 =
\begin{bmatrix}
a & 0 \\
0 & b
\end{bmatrix}.
\end{equation}
%
The off diagonal terms can be considered to be a perturbation
%
\begin{dmath}\label{eqn:qmLecture20:80}
V =
\begin{bmatrix}
0 & c \\
c^\conj & 0
\end{bmatrix},
\end{dmath}
%
with \( H = H_0 + V \).

\paragraph{Energy levels after perturbation}

We can solve for the eigenvalues of \( H \) easily, finding
%
\begin{dmath}\label{eqn:qmLecture20:40}
\lambda_\pm = \frac{a + b}{2} \pm \sqrt{ \lr{ \frac{a - b}{2}}^2 + \Abs{c}^2 }.
\end{dmath}
%
Plots of a few \( a,b \) variations of \( \lambda_{\pm} \) are shown in \cref{fig:lecture20:lecture20Fig3}.  The quadratic (non-degenerate) domain is found near the \( c = 0 \) points of all but the first ( \( a = b \) ) plot, and the degenerate (linear in \( \Abs{c}^2 \)) regions are visible for larger values of \( c \).

\mathImageFourFiguresTwoLines
{../figures/phy1520-quantum/lecture20Fig3}
{../figures/phy1520-quantum/lecture20Fig4}
{../figures/phy1520-quantum/lecture20Fig5}
{../figures/phy1520-quantum/lecture20Fig6}
{Plots of \( \lambda_{\pm} \) for \((a,b) \in \setlr{(1,1),(1,0),(1,5),(-8,8)}\).}{fig:lecture20:lecture20Fig3}{scale=0.4}{visualizationOfEigenvaluesOfTwoByTwoHermitianMatrix.nb}

\paragraph{Some approximations}

Suppose that \( \Abs{c} \ll \Abs{a - b} \), then
%
\begin{dmath}\label{eqn:qmLecture20:100}
\lambda_\pm \approx \frac{a + b}{2} \pm \Abs{ \frac{a - b}{2} } \lr{ 1 + 2 \frac{\Abs{c}^2}{\Abs{a - b}^2} }.
\end{dmath}
%
If  \( a > b \), then
%
\begin{dmath}\label{eqn:qmLecture20:120}
\lambda_\pm \approx \frac{a + b}{2} \pm \frac{a - b}{2} \lr{ 1 + 2 \frac{\Abs{c}^2}{\lr{a - b}^2} }.
\end{dmath}
%
\begin{dmath}\label{eqn:qmLecture20:140}
\lambda_{+}
= \frac{a + b}{2} + \frac{a - b}{2} \lr{ 1 + 2 \frac{\Abs{c}^2}{\lr{a - b}^2} }
= a + \lr{a - b} \frac{\Abs{c}^2}{\lr{a - b}^2}
= a + \frac{\Abs{c}^2}{a - b},
\end{dmath}
%
and
\begin{dmath}\label{eqn:qmLecture20:680}
\lambda_{-}
= \frac{a + b}{2} - \frac{a - b}{2} \lr{ 1 + 2 \frac{\Abs{c}^2}{\lr{a - b}^2} }
=
b + \lr{a - b} \frac{\Abs{c}^2}{\lr{a - b}^2}
= b + \frac{\Abs{c}^2}{a - b}.
\end{dmath}
%
This adiabatic evolution displays a ``level repulsion'', quadratic in \( \Abs{c} \),
% as sketched in \cref{fig:lecture20:lecture20Fig1},
and is described as a non-degenerate permutation.

%\imageFigure{../figures/phy1520-quantum/lecture20Fig1}{Adiabatic (non-degenerate) pertubation}{fig:lecture20:lecture20Fig1}{0.2}

If \( \Abs{c} \gg \Abs{a -b} \), then
%
\begin{dmath}\label{eqn:qmLecture20:160}
\lambda_\pm
= \frac{a + b}{2} \pm \Abs{c} \sqrt{ 1 + \inv{\Abs{c}^2} \lr{ \frac{a - b}{2}}^2 }
\approx \frac{a + b}{2} \pm \Abs{c} \lr{ 1 + \inv{2 \Abs{c}^2} \lr{ \frac{a - b}{2}}^2 }
= \frac{a + b}{2} \pm \Abs{c} \pm \frac{\lr{a - b}^2}{8 \Abs{c}}.
\end{dmath}
%
Here we loose the adiabaticity, and have ``level repulsion'' that is linear in \( \Abs{c} \).
%, as sketched in \cref{fig:lecture20:lecture20Fig2}.
We no longer have the sign of \( a - b \) in the expansion.  This is described as a degenerate permutation.

%\imageFigure{../figures/phy1520-quantum/lecture20Fig2}{Degenerate perbutation}{fig:lecture20:lecture20Fig2}{0.2}

\section{General non-degenerate perturbation.}
\index{perturbation!non-degenerate}

Given an unperturbed system with solutions of the form
%
\begin{equation}\label{eqn:qmLecture20:180}
H_0 \ket{n^{(0)}} = E_n^{(0)} \ket{n^{(0)}},
\end{equation}
%
we want to solve the perturbed Hamiltonian equation
%
\begin{equation}\label{eqn:qmLecture20:200}
\lr{ H_0 + \lambda V } \ket{ n } = \lr{ E_n^{(0)} + \Delta n } \ket{n}.
\end{equation}
%
Here \( \Delta n \) is an energy shift as that goes to zero as \( \lambda \rightarrow 0 \).  We can write this as
%
\begin{equation}\label{eqn:qmLecture20:220}
\lr{ E_n^{(0)} - H_0 } \ket{ n } = \lr{ \lambda V - \Delta_n } \ket{n}.
\end{equation}
%
We are hoping to iterate with application of the inverse to an initial estimate of \( \ket{n} \)
%
\begin{equation}\label{eqn:qmLecture20:240}
\ket{n} = \lr{ E_n^{(0)} - H_0 }^{-1} \lr{ \lambda V - \Delta_n } \ket{n}.
\end{equation}
%
This gets us into trouble if \( \lambda \rightarrow 0 \), which can be fixed by using
%
\begin{equation}\label{eqn:qmLecture20:260}
\ket{n} = \lr{ E_n^{(0)} - H_0 }^{-1} \lr{ \lambda V - \Delta_n } \ket{n} + \ket{ n^{(0)} },
\end{equation}
%
which can be seen to be a solution to \cref{eqn:qmLecture20:220}.  We want to ask if
%
\begin{equation}\label{eqn:qmLecture20:280}
\lr{ \lambda V - \Delta_n } \ket{n} ,
\end{equation}
%
contains a bit of \( \ket{ n^{(0)} } \)?  To determine this act with \( \bra{n^{(0)}} \) on the left
%
\begin{dmath}\label{eqn:qmLecture20:300}
\bra{ n^{(0)} } \lr{ \lambda V - \Delta_n } \ket{n}
=
\bra{ n^{(0)} } \lr{ E_n^{(0)} - H_0 } \ket{n}
=
\lr{ E_n^{(0)} - E_n^{(0)} } \braket{n^{(0)}}{n}
=
0.
\end{dmath}
%
This shows that \( \ket{n} \) is entirely orthogonal to \( \ket{n^{(0)}} \).
First define a projection operator
%
\begin{dmath}\label{eqn:qmLecture20:320}
P_n = \ket{n^{(0)}}\bra{n^{(0)}},
\end{dmath}
%
which has the idempotent property \( P_n^2 = P_n \) that we expect of a projection operator.
Now define a rejection operator
\begin{dmath}\label{eqn:qmLecture20:340}
\overbar{P}_n
= 1 -
\ket{n^{(0)}}\bra{n^{(0)}}
= \sum_{m \ne n}
\ket{m^{(0)}}\bra{m^{(0)}}.
\end{dmath}
%
Because \( \ket{n} \) has no component in the direction \( \ket{n^{(0)}} \), the rejection operator can be inserted much like we normally do with the identity operator, yielding
%
\begin{dmath}\label{eqn:qmLecture20:360}
\ket{n}' = \lr{ E_n^{(0)} - H_0 }^{-1} \overbar{P}_n \lr{ \lambda V - \Delta_n } \ket{n} + \ket{ n^{(0)} },
\end{dmath}
%
valid for any initial \( \ket{n} \).
\paragraph{Power series perturbation expansion}
Instead of iterating, suppose that the unknown state and unknown energy difference operator can be expanded in a \( \lambda \) power series, say
%
\begin{dmath}\label{eqn:qmLecture20:380}
\ket{n}
=
\ket{n_0}
+ \lambda \ket{n_1}
+ \lambda^2 \ket{n_2}
+ \lambda^3 \ket{n_3} + \cdots
\end{dmath}
and
%
\begin{dmath}\label{eqn:qmLecture20:400}
\Delta_{n} = \Delta_{n_0}
+ \lambda \Delta_{n_1}
+ \lambda^2 \Delta_{n_2}
+ \lambda^3 \Delta_{n_3} + \cdots
\end{dmath}
We usually interpret functions of operators in terms of power series expansions.  In the case of \( \lr{ E_n^{(0)} - H_0 }^{-1} \), we have a concrete interpretation when acting on one of the unperturbed eigenstates
%
\begin{dmath}\label{eqn:qmLecture20:420}
\inv{ E_n^{(0)} - H_0 } \ket{m^{(0)}} =
\inv{ E_n^{(0)} - E_m^0 } \ket{m^{(0)}}.
\end{dmath}
%
This gives
%
\begin{dmath}\label{eqn:qmLecture20:440}
\ket{n}
=
\inv{ E_n^{(0)} - H_0 }
\sum_{m \ne n}
\ket{m^{(0)}}\bra{m^{(0)}}
 \lr{ \lambda V - \Delta_n } \ket{n} + \ket{ n^{(0)} },
\end{dmath}
%
or
%
%\begin{equation}\label{eqn:qmLecture20:460}
\boxedEquation{eqn:qmLecture20:480}{
\ket{n}
=
 \ket{ n^{(0)} }
+
\sum_{m \ne n}
\frac{\ket{m^{(0)}}\bra{m^{(0)}}}
{
E_n^{(0)} - E_m^{(0)}
}
 \lr{ \lambda V - \Delta_n } \ket{n}.
}
%\end{equation}
%
From \cref{eqn:qmLecture20:220}, note that
%
\begin{equation}\label{eqn:qmLecture20:500}
\Delta_n =
\frac{\bra{n^{(0)}} \lambda V \ket{n}}{\braket{n^0}{n}},
\end{equation}
%
however, we will normalize by setting \( \braket{n^0}{n} = 1 \), so
%
%\begin{equation}\label{eqn:qmLecture20:521}
\boxedEquation{eqn:qmLecture20:521}{
\Delta_n =
\bra{n^{(0)}} \lambda V \ket{n}.
}
%\end{equation}
%
\paragraph{to \( O(\lambda^0) \) }
If all \( \lambda^n, n > 0 \) are zero, then we have
\begin{subequations}
\label{eqn:qmLecture20:780}
\begin{dmath}\label{eqn:qmLecture20:740}
\ket{n_0}
=
 \ket{ n^{(0)} }
+
\sum_{m \ne n}
\frac{\ket{m^{(0)}}\bra{m^{(0)}}}
{
E_n^{(0)} - E_m^{(0)}
}
 \lr{ - \Delta_{n_0} } \ket{n_0}
\end{dmath}
\begin{dmath}\label{eqn:qmLecture20:800}
\Delta_{n_0} \braket{n^{(0)}}{n_0} = 0,
\end{dmath}
\end{subequations}
so
%
\begin{dmath}\label{eqn:qmLecture20:540}
\begin{aligned}
\ket{n_0} &= \ket{n^{(0)}} \\
\Delta_{n_0} &= 0.
\end{aligned}
\end{dmath}
%
\paragraph{to \( O(\lambda^1) \) }
Requiring identity for all \( \lambda^1 \) terms means
%
\begin{dmath}\label{eqn:qmLecture20:760}
\ket{n_1} \lambda
=
\sum_{m \ne n}
\frac{\ket{m^{(0)}}\bra{m^{(0)}}}
{
E_n^{(0)} - E_m^{(0)}
}
 \lr{ \lambda V - \Delta_{n_1} \lambda } \ket{n_0},
\end{dmath}
%
so
%
\begin{dmath}\label{eqn:qmLecture20:560}
\ket{n_1}
=
\sum_{m \ne n}
\frac{
\ket{m^{(0)}} \bra{ m^{(0)}}
}
{
E_n^{(0)} - E_m^{(0)}
}
\lr{ V - \Delta_{n_1} } \ket{n_0}.
\end{dmath}
%
With the assumption that \( \ket{n^{(0)}} \) is normalized, and with the shorthand
%
\begin{dmath}\label{eqn:qmLecture20:600}
V_{m n} = \bra{ m^{(0)}} V \ket{n^{(0)}},
\end{dmath}
%
that is
%
\begin{equation}\label{eqn:qmLecture20:580}
\begin{aligned}
\ket{n_1}
&=
\sum_{m \ne n}
\frac{
\ket{m^{(0)}}
}
{
E_n^{(0)} - E_m^{(0)}
}
V_{m n}
%\bra{ m^{(0)}} V \ket{n_0}
, \\
\Delta_{n_1} &= \bra{ n^{(0)} } V \ket{ n^{(0)} } = V_{nn}.
\end{aligned}
\end{equation}
%
\paragraph{to \( O(\lambda^2) \) }
The second order perturbation states are found by selecting only the \( \lambda^2 \) contributions to
%
\begin{equation}\label{eqn:qmLecture20:820}
\lambda^2 \ket{n_2}
=
\sum_{m \ne n}
\frac{\ket{m^{(0)}}\bra{m^{(0)}}}
{
E_n^{(0)} - E_m^{(0)}
}
 \lr{ \lambda V - (\lambda \Delta_{n_1} + \lambda^2 \Delta_{n_2}) }
\lr{
\ket{n_0}
+ \lambda \ket{n_1}
}.
\end{equation}
%
Because \( \ket{n_0} = \ket{n^{(0)}} \), the \( \lambda^2 \Delta_{n_2} \) is killed, leaving
%
\begin{dmath}\label{eqn:qmLecture20:840}
\ket{n_2}
=
\sum_{m \ne n}
\frac{\ket{m^{(0)}}\bra{m^{(0)}}}
{
E_n^{(0)} - E_m^{(0)}
}
 \lr{ V - \Delta_{n_1} }
\ket{n_1}
=
\sum_{m \ne n}
\frac{\ket{m^{(0)}}\bra{m^{(0)}}}
{
E_n^{(0)} - E_m^{(0)}
}
 \lr{ V - \Delta_{n_1} }
\sum_{l \ne n}
\frac{
\ket{l^{(0)}}
}
{
E_n^{(0)} - E_l^{(0)}
}
V_{l n},
\end{dmath}
%
which can be written as
%
\begin{dmath}\label{eqn:qmLecture20:620}
\ket{n_2}
=
\sum_{l,m \ne n}
\ket{m^{(0)}}
\frac{V_{m l} V_{l n}}
{
\lr{ E_n^{(0)} - E_m^{(0)} }
\lr{ E_n^{(0)} - E_l^{(0)} }
}
-
\sum_{m \ne n}
\ket{m^{(0)}}
\frac{V_{n n} V_{m n}}
{
\lr{ E_n^{(0)} - E_m^{(0)} }^2
}.
\end{dmath}
%
For the second energy perturbation we have
%
\begin{dmath}\label{eqn:qmLecture20:860}
\lambda^2 \Delta_{n_2} =
\bra{n^{(0)}} \lambda V \lr{ \lambda \ket{n_1} },
\end{dmath}
%
or
%
\begin{dmath}\label{eqn:qmLecture20:880}
\Delta_{n_2}
=
\bra{n^{(0)}} V \ket{n_1}
=
\bra{n^{(0)}} V
\sum_{m \ne n}
\frac{
\ket{m^{(0)}}
}
{
E_n^{(0)} - E_m^{(0)}
}
V_{m n}.
\end{dmath}
%
That is
%
\begin{dmath}\label{eqn:qmLecture20:900}
\Delta_{n_2}
=
\sum_{m \ne n} \frac{V_{n m} V_{m n} }{E_n^{(0)} - E_m^{(0)}}.
\end{dmath}
%
\paragraph{to \( O(\lambda^3) \) }

Similarly, it can be shown that
%
\begin{dmath}\label{eqn:qmLecture20:640}
\Delta_{n_3} =
\sum_{l, m \ne n} \frac{V_{n m} V_{m l} V_{l n} }{
\lr{ E_n^{(0)} - E_m^{(0)} }
\lr{ E_n^{(0)} - E_l^{(0)} }
}
-
\sum_{ m \ne n} \frac{V_{n m} V_{n n} V_{m n} }{
\lr{ E_n^{(0)} - E_m^{(0)} }^2
}.
\end{dmath}
%
In general, the energy perturbation is given by
%
\begin{dmath}\label{eqn:qmLecture20:660}
\Delta_n^{(l)} = \bra{n^{(0)}} V \ket{n_{l-1}}.
\end{dmath}
%
%\EndArticle
