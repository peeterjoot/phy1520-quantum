%
% Copyright � 2015 Peeter Joot.  All Rights Reserved.
% Licenced as described in the file LICENSE under the root directory of this GIT repository.
%
%{
%\input{../blogpost.tex}
%\renewcommand{\basename}{hyperfineElectronAndBosonNucleus}
%\renewcommand{\dirname}{notes/phy1520/}
%%\newcommand{\dateintitle}{}
%%\newcommand{\keywords}{}
%
%\input{../peeter_prologue_print2.tex}
%
%\usepackage{peeters_layout_exercise}
%\usepackage{peeters_braket}
%\usepackage{peeters_figures}
%
%\beginArtNoToc
%
%\generatetitle{Electron and Boson hyperfine spin interaction}
%%\chapter{Electron and Boson hyperfine spin interaction}
%%\label{chap:hyperfineElectronAndBosonNucleus}
%
\makeoproblem{Electron and Boson hyperfine spin interaction.}{problem:hyperfineElectronAndBosonNucleus:1}{2015 final}{
%
This problem is a variation of problem set 8 problem 4, but instead of the spin interaction of a Fermionic nucleus with an electron, this problem was to look at the spin interaction of an electron with a Bosonic nucleus.

The interaction Hamiltonian in this problem includes a Zeeman field
%
\begin{equation}\label{eqn:hyperfineElectronAndBosonNucleus:20}
H = J \BS_e \cdot \BS_n + B S_e^z,
\end{equation}
%
\makesubproblem{}{problem:hyperfineElectronAndBosonNucleus:1:a}
Solve the problem exactly.
%
\makesubproblem{}{problem:hyperfineElectronAndBosonNucleus:1:b}
Do a first order perturbative solution.
%
} % problem
%
\makeanswer{problem:hyperfineElectronAndBosonNucleus:1}{
\withproblemsetsParagraph{
%
\makeSubAnswer{}{problem:hyperfineElectronAndBosonNucleus:1:a}
With two states for the electron spin and three for the nuclear spin, the state space for the spin system is six dimensional.  Without the Zeeman contribution the action of Hamiltonian is
%
\begin{equation}\label{eqn:hyperfineElectronAndBosonNucleus:40}
\begin{aligned}
H \ket{\ncap; + } \ket{1} &= \frac{J \Hbar^2}{2} \ket{\ncap; + } \ket{1}, \\
H \ket{\ncap; - } \ket{1} &= -\frac{J \Hbar^2}{2} \ket{\ncap; - } \ket{1}, \\
H \ket{\ncap; + } \ket{0} &= 0 \ket{\ncap; + } \ket{0}, \\
H \ket{\ncap; - } \ket{0} &= 0 \ket{\ncap; - } \ket{0}, \\
H \ket{\ncap; + } \ket{-1} &= -\frac{J \Hbar^2}{2} \ket{\ncap; + } \ket{-1}, \\
H \ket{\ncap; - } \ket{-1} &= \frac{J \Hbar^2}{2} \ket{\ncap; - } \ket{-1}.
\end{aligned}
\end{equation}
This can be put into block matrix form
%
\begin{equation}\label{eqn:hyperfineElectronAndBosonNucleus:60}
J \BS_e \cdot \BS_n
=
\frac{J \Hbar^2}{2}
\begin{bmatrix}
\sigma_z & 0 & 0 \\
0        & 0 & 0 \\
0        & 0 & -\sigma_z \\
\end{bmatrix}.
\end{equation}
%
The Zeeman field in the basis above is
%
\begin{equation}\label{eqn:hyperfineElectronAndBosonNucleus:80}
B S_e^z
=
\frac{B \Hbar}{2}
\begin{bmatrix}
R & 0 & 0 \\
0 & R & 0 \\
0 & 0 & R \\
\end{bmatrix},
\end{equation}
%
where \( R \) is the representation of \( \sigma_z \) in a \( \ket{\ncap; + }, \ket{\ncap; - } \) basis.  Representing the electron spin states as
%
\begin{dmath}\label{eqn:hyperfineElectronAndBosonNucleus:100}
\begin{aligned}
\ket{\ncap; +} &=
\begin{bmatrix}
\cos(\theta/2) e^{-i\phi} \\
\sin(\theta/2)
\end{bmatrix}
=
\cos(\theta/2) e^{-i\phi} \ket{\zcap;+}
+
\sin(\theta/2) \ket{\zcap;-}
\\
\ket{\ncap; -} &=
\begin{bmatrix}
-\sin(\theta/2) e^{-i\phi} \\
\cos(\theta/2)
\end{bmatrix}
=
-\sin(\theta/2) e^{-i\phi} \ket{\zcap;+}
+ \cos(\theta/2) \ket{\zcap;-},
\end{aligned}
\end{dmath}

so the \( S_e^z \) action on the spin states is
\begin{dmath}\label{eqn:hyperfineElectronAndBosonNucleus:120}
S_e^z \ket{\ncap;+}
=
\frac{\Hbar}{2} \lr{
\cos(\theta/2) e^{-i\phi} \ket{\zcap;+}
-
\sin(\theta/2) \ket{\zcap;-}
}
=
\frac{\Hbar}{2}
\begin{bmatrix}
\cos(\theta/2) e^{-i\phi} \\
-\sin(\theta/2)
\end{bmatrix},
\end{dmath}
%
and
\begin{dmath}\label{eqn:hyperfineElectronAndBosonNucleus:140}
S_e^z \ket{\ncap;-}
=
\frac{\Hbar}{2} \lr{
-\sin(\theta/2) e^{-i\phi} \ket{\zcap;+}
-
\cos(\theta/2) \ket{\zcap;-}
}
=
\frac{\Hbar}{2}
\begin{bmatrix}
-\sin(\theta/2) e^{-i\phi} \\
-\cos(\theta/2)
\end{bmatrix}.
\end{dmath}
%
The matrix elements for \( S_e^z \) are
%
\begin{equation}\label{eqn:hyperfineElectronAndBosonNucleus:200}
\begin{aligned}
\bra{\ncap;+} S_e^z \ket{\ncap;+}
&=
\frac{\Hbar}{2}
\begin{bmatrix}
\cos(\theta/2) e^{i\phi} &
\sin(\theta/2)
\end{bmatrix}
\begin{bmatrix}
\cos(\theta/2) e^{-i\phi} \\
-\sin(\theta/2)
\end{bmatrix} \\
&=
\frac{\Hbar}{2} \cos\theta,
\end{aligned}
\end{equation}
%
\begin{equation}\label{eqn:hyperfineElectronAndBosonNucleus:160}
\begin{aligned}
\bra{\ncap;+} S_e^z \ket{\ncap;-}
&=
\frac{\Hbar}{2}
\begin{bmatrix}
\cos(\theta/2) e^{i\phi} &
\sin(\theta/2)
\end{bmatrix}
\begin{bmatrix}
-\sin(\theta/2) e^{-i\phi} \\
-\cos(\theta/2)
\end{bmatrix} \\
&=
-\frac{\Hbar}{2} \sin\theta,
\end{aligned}
\end{equation}
%
\begin{equation}\label{eqn:hyperfineElectronAndBosonNucleus:220}
\begin{aligned}
\bra{\ncap;-} S_e^z \ket{\ncap;+}
&=
\frac{\Hbar}{2}
\begin{bmatrix}
-\sin(\theta/2) e^{i\phi} &
\cos(\theta/2)
\end{bmatrix}
\begin{bmatrix}
\cos(\theta/2) e^{-i\phi} \\
-\sin(\theta/2)
\end{bmatrix} \\
&=
-\frac{\Hbar}{2} \sin\theta,
\end{aligned}
\end{equation}
%
\begin{equation}\label{eqn:hyperfineElectronAndBosonNucleus:180}
\begin{aligned}
\bra{\ncap;-} S_e^z \ket{\ncap;-}
&=
\frac{\Hbar}{2}
\begin{bmatrix}
-\sin(\theta/2) e^{i\phi} &
\cos(\theta/2)
\end{bmatrix}
\begin{bmatrix}
-\sin(\theta/2) e^{-i\phi} \\
-\cos(\theta/2)
\end{bmatrix} \\
&=
-\frac{\Hbar}{2} \cos\theta,
\end{aligned}
\end{equation}
%
or
\begin{equation}\label{eqn:hyperfineElectronAndBosonNucleus:240}
R =
\begin{bmatrix}
\cos\theta & -\sin\theta \\
- \sin\theta & -\cos\theta
\end{bmatrix}.
\end{equation}
%
%\paragraph{Exact solution, and strong and weak field approximations.}
\makeSubAnswer{}{problem:hyperfineElectronAndBosonNucleus:1:b}
%

In \nbref{finalExamProblem6HyperfineInteraction.nb} the exact solution to the system is found to be
%
\begin{equation}\label{eqn:hyperfineElectronAndBosonNucleus:260}
E \in \setlr{ \pm \frac{B\Hbar}{2}, \pm \frac{\Hbar}{2} \sqrt{ B^2 \pm 2 B J \Hbar \cos\theta + J^2 \Hbar^2 } },
\end{dmath}
%
or with \( m \in \setlr{-1, 0, 1} \)
%
\begin{equation}\label{eqn:hyperfineElectronAndBosonNucleus:280}
E \in \pm \frac{\Hbar}{2} \sqrt{ B^2 + 2 B J m \Hbar \cos\theta + m^2 J^2 \Hbar^2 }.
\end{dmath}
%
In the strong field case where \( B \gg \Hbar J \), we have
\begin{equation}\label{eqn:hyperfineElectronAndBosonNucleus:300}
E \approx \pm \frac{\Hbar B}{2} \lr{ 1 + \frac{J}{B} m \Hbar \cos\theta + \inv{2} m^2 \frac{J^2}{B^2} \Hbar^2 },
\end{dmath}
%
whereas in the weak field case where \( B \ll \Hbar J \), we have
\begin{dmath}\label{eqn:hyperfineElectronAndBosonNucleus:320}
E \approx
\left\{
\begin{array}{l l}
\pm \frac{\Hbar B}{2} & \quad \mbox{if \( m = 0\),} \\
\pm \frac{\Hbar^2 J}{2} \lr{
1
+ \frac{B}{\Hbar J} m \cos\theta
+ \inv{2} \frac{B^2}{J^2 \Hbar^2} }
 & \quad \mbox{if \( m \ne 0\).}
\end{array}
\right.
\end{dmath}
%
Note that the result above is exact for the \( m = 0 \) case.
\paragraph{Perturbative solutions}
The degenerate perturbation requires diagonalizing \( B S_e^z \).  Since \( R \) is a representation of \( S_e^z \), the eigenvalues of \( \frac{ B \Hbar R}{2} \) are \( \pm B \Hbar /2 \).
This means that the first order shifts to the energy eigenvalues are
%
\begin{equation}\label{eqn:hyperfineElectronAndBosonNucleus:340}
\begin{aligned}
\frac{J \Hbar^2}{2} &\rightarrow \frac{J \Hbar^2}{2} + \frac{B \Hbar}{2}, \\
-\frac{J \Hbar^2}{2} &\rightarrow -\frac{J \Hbar^2}{2} - \frac{B \Hbar}{2}, \\
0 &\rightarrow + \frac{B \Hbar}{2}, \\
0 &\rightarrow - \frac{B \Hbar}{2}, \\
-\frac{J \Hbar^2}{2} &\rightarrow -\frac{J \Hbar^2}{2} + \frac{B \Hbar}{2}, \\
\frac{J \Hbar^2}{2} &\rightarrow \frac{J \Hbar^2}{2} - \frac{B \Hbar}{2}.
\end{aligned}
\end{equation}
}
} % answer
%}
%\EndNoBibArticle
