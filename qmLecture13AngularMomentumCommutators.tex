%
% Copyright © 2015 Peeter Joot.  All Rights Reserved.
% Licenced as described in the file LICENSE under the root directory of this GIT repository.
%
\makeproblem{Angular momentum commutators.}{problem:qmLecture13:1}{
\index{angular momentum!commutators}
Using \( \hatL_i = \epsilon_{ijk} \hatr_j \hatp_k \), show that
%
\begin{dmath}\label{eqn:qmLecture13:1620}
\antisymmetric{\hatL_i}{\hatL_j} = i \Hbar \epsilon_{ijk} \hatL_k.
\end{dmath}
%
} % problem
%
\makeanswer{problem:qmLecture13:1}{
%
Let's start without using abstract index expressions, computing the commutator for \( \hatL_1, \hatL_2 \), which should show the basic steps required
%
\begin{dmath}\label{eqn:qmLecture13:1640}
\antisymmetric{\hatL_1}{\hatL_2}
=
\antisymmetric{\hatr_2 \hatp_3 - \hatr_3 \hatp_2}{\hatr_3 \hatp_1 - \hatr_1 \hatp_3}
=
\antisymmetric{\hatr_2 \hatp_3}{\hatr_3 \hatp_1}
-\antisymmetric{\hatr_2 \hatp_3}{\hatr_1 \hatp_3}
-\antisymmetric{\hatr_3 \hatp_2}{\hatr_3 \hatp_1}
+\antisymmetric{\hatr_3 \hatp_2}{\hatr_1 \hatp_3}.
\end{dmath}
%
The first of these commutators is
%
\begin{dmath}\label{eqn:qmLecture13:1660}
\antisymmetric{\hatr_2 \hatp_3}{\hatr_3 \hatp_1}
=
{\hatr_2 \hatp_3}{\hatr_3 \hatp_1}
-
{\hatr_3 \hatp_1}
{\hatr_2 \hatp_3}
=
\hatr_2 \hatp_1 \antisymmetric{\hatp_3}{\hatr_3}
=
-i \Hbar \hatr_2 \hatp_1.
\end{dmath}
%
We see that any factors in the commutator don't have like indexes (i.e. \( \hatr_k, \hatp_k \)) on both position and momentum terms, can be pulled out of the commutator.  This leaves
%
\begin{dmath}\label{eqn:qmLecture13:1680}
\antisymmetric{\hatL_1}{\hatL_2}
=
\hatr_2 \hatp_1 \antisymmetric{\hatp_3}{\hatr_3}
-\cancel{\antisymmetric{\hatr_2 \hatp_3}{\hatr_1 \hatp_3}}
-\cancel{\antisymmetric{\hatr_3 \hatp_2}{\hatr_3 \hatp_1}}
+\hatr_1 \hatp_2 \antisymmetric{\hatr_3}{\hatp_3}
=
i \Hbar \lr{ \hatr_1 \hatp_2 - \hatr_2 \hatp_1 }
=
i \Hbar \hatL_3.
\end{dmath}
%
With cyclic permutation this is really enough to consider \cref{eqn:qmLecture13:1620} proven.  However, can we do this in the general case with the abstract index expression?  The quantity to simplify looks forbidding
%
\begin{dmath}\label{eqn:qmLecture13:1700}
\antisymmetric{\hatL_i}{\hatL_j}
=
\epsilon_{i a b }
\epsilon_{j s t }
\antisymmetric{ \hatr_a \hatp_b }{ \hatr_s \hatp_t }.
\end{dmath}
Because there are no repeated indexes, this doesn't submit to any of the normal reduction identities.  Note however, since we only care about the \( i \ne j \) case, that one of the indexes \( a, b \) must be \( j \) for this quantity to be non-zero.  Therefore (for \( i \ne j \))
%
\begin{dmath}\label{eqn:qmLecture13:1720}
\antisymmetric{\hatL_i}{\hatL_j}
=
\epsilon_{i j b }
\epsilon_{j s t }
\antisymmetric{ \hatr_j \hatp_b }{ \hatr_s \hatp_t   }
+
\epsilon_{i a j }
\epsilon_{j s t }
\antisymmetric{ \hatr_a \hatp_j }{ \hatr_s \hatp_t   }
=
\epsilon_{i j b }
\epsilon_{j s t }
\lr{
\antisymmetric{ \hatr_j \hatp_b }{ \hatr_s \hatp_t   }
-
\antisymmetric{ \hatr_b \hatp_j }{ \hatr_s \hatp_t   }
}
=
-\delta^{s t}_{[i b]}
\antisymmetric{ \hatr_j \hatp_b - \hatr_b \hatp_j }{ \hatr_s \hatp_t }
=
\antisymmetric{ \hatr_j \hatp_b - \hatr_b \hatp_j }{ \hatr_b \hatp_i - \hatr_i \hatp_b }
=
  \antisymmetric{ \hatr_j \hatp_b }{ \hatr_b \hatp_i }
- \cancel{\antisymmetric{ \hatr_j \hatp_b }{ \hatr_i \hatp_b }}
- \cancel{\antisymmetric{ \hatr_b \hatp_j }{ \hatr_b \hatp_i }}
+ \antisymmetric{ \hatr_b \hatp_j }{ \hatr_i \hatp_b }
=
\hatr_j \hatp_i  \antisymmetric{ \hatp_b }{ \hatr_b }
+ \hatr_i \hatp_j \antisymmetric{ \hatr_b }{ \hatp_b }
=
 i \Hbar \lr{ \hatr_i \hatp_j - \hatr_j \hatp_i }
=
 i \Hbar \epsilon_{i j k} \hatr_i \hatp_j .
\end{dmath}
%
} % answer

%\EndArticle
