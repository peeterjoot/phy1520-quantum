%
% Copyright � 2015 Peeter Joot.  All Rights Reserved.
% Licenced as described in the file LICENSE under the root directory of this GIT repository.
%
%\input{../blogpost.tex}
%\renewcommand{\basename}{qmLecture22}
%\renewcommand{\dirname}{notes/phy1520/}
%\newcommand{\keywords}{PHY1520H}
%\input{../peeter_prologue_print2.tex}
%
%%\usepackage{phy1520}
%\usepackage{peeters_braket}
%\usepackage{macros_cal}
%%\usepackage{peeters_layout_exercise}
%\usepackage{peeters_figures}
%\usepackage{mathtools}
%
%\beginArtNoToc
%\generatetitle{PHY1520H Graduate Quantum Mechanics.  Lecture 22: van der Walls potential and Stark effect.  Taught by Prof.\ Arun Paramekanti}
%%\chapter{van der Walls potential and Stark effect}
%\label{chap:qmLecture22}
%
%\paragraph{Disclaimer}
%
%Peeter's lecture notes from class.  These may be incoherent and rough.
%
%These are notes for the UofT course PHY1520, Graduate Quantum Mechanics, taught by Prof. Paramekanti, covering \textchapref{{5}} \citep{sakurai2014modern} content.
%
\paragraph{Another approach (for last time?)}

Imagine we perturb a potential, say a harmonic oscillator with an electric field
%
\begin{equation}\label{eqn:qmLecture22:20}
V_0(x) = \inv{2} k x^2,
\end{equation}
\begin{equation}\label{eqn:qmLecture22:40}
V(x) = \calE e x.
\end{equation}
After minimizing the energy, using \( \PDi{x}{V} = 0 \), we get
%
\begin{equation}\label{eqn:qmLecture22:60}
\inv{2} k x^2 + \calE e x \rightarrow k x^\conj = - e \calE,
\end{equation}
%
\begin{equation}\label{eqn:qmLecture22:80}
p^\conj = -e x^\conj = - \frac{e^2 \calE}{k}.
\end{equation}
For such a system the polarizability is
%
\begin{equation}\label{eqn:qmLecture22:100}
\alpha = \frac{e^2 }{k},
\end{equation}
%
\begin{equation}\label{eqn:qmLecture22:120}
\begin{aligned}
\inv{2} k \lr{ -\frac{ e \calE}{k} }^2 + \calE e \lr{ - \frac{e \calE}{k} }
&= - \inv{2} \lr{ \frac{e^2}{k} } \calE^2
\\ &= - \inv{2} \alpha \calE^2
\end{aligned}
\end{equation}
%\cref{fig:lecture22:lecture22Fig1}.
%\imageFigure{../figures/phy1520-quantum/lecture22Fig1}{CAPTION: lecture22Fig1}{fig:lecture22:lecture22Fig1}{0.2}
\section{van der Walls potential.}
\index{van der Walls}
%
\begin{equation}\label{eqn:qmLecture22:140}
H_0 =
H_{0 1} + H_{0 2},
\end{equation}
%
where
%
\begin{equation}\label{eqn:qmLecture22:160}
H_{0 \alpha} = \frac{p_\alpha^2}{2m} - \frac{e^2}{4 \pi \epsilon_0 \Abs{ \Br_\alpha - \BR_\alpha} }, \qquad \alpha = 1,2.
\end{equation}
%
The full interaction potential is
%
\begin{equation}\label{eqn:qmLecture22:180}
V =
\frac{e^2}{4 \pi \epsilon_0} \lr{
\inv{\Abs{\BR_1 - \BR_2}}
+
\inv{\Abs{\Br_1 - \Br_2}}
-
\inv{\Abs{\Br_1 - \BR_2}}
-
\inv{\Abs{\Br_2 - \BR_1}}
}.
\end{equation}
Let
%
\begin{equation}\label{eqn:qmLecture22:200}
\Bx_\alpha = \Br_\alpha - \BR_\alpha,
\end{equation}
%
\begin{equation}\label{eqn:qmLecture22:220}
\BR = \BR_1 - \BR_2,
\end{equation}
%
as sketched in \cref{fig:lecture22:lecture22Fig2}.
\imageFigure{../figures/phy1520-quantum/lecture22Fig2}{Two atom interaction.}{fig:lecture22:lecture22Fig2}{0.2}
%
\begin{equation}\label{eqn:qmLecture22:240}
H_{0 \alpha}
=
\frac{\Bp^2}{2m}
-\frac{e^2}{4 \pi \epsilon_0 \Abs{\Bx_\alpha}},
\end{equation}
which allows the total interaction potential to be written
\begin{equation}\label{eqn:qmLecture22:260}
V =
\frac{e^2}{4 \pi \epsilon_0 R}
\lr{
1
+
\frac{R}{\Abs{\Bx_1 - \Bx_2 + \BR}}
-
\frac{R}{\Abs{\Bx_1 + \BR}}
-
\frac{R}{\Abs{-\Bx_2 + \BR}}
}.
\end{equation}
For \( R \gg x_1, x_2 \), this interaction potential, after a multipole expansion, is approximately
%
\begin{equation}\label{eqn:qmLecture22:280}
V =
\frac{e^2}{4 \pi \epsilon_0} \lr{
\frac{\Bx_1 \cdot \Bx_2}{\Abs{\BR}^3}
-3 \frac{
(\Bx_1 \cdot \BR)
(\Bx_2 \cdot \BR)
}{\Abs{\BR}^5}
}
\end{equation}
Showing this is left as a exercise.
\paragraph{1. \( O(\lambda) \) }.
With
%
\begin{equation}\label{eqn:qmLecture22:300}
\psi_0 = \ket{ 1s, 1s },
\end{equation}
%
\begin{equation}\label{eqn:qmLecture22:320}
\Delta E^{(1)} = \bra{\psi_0} V \ket{\psi_0},
\end{equation}
the two particle wave functions are of the form
%
\begin{equation}\label{eqn:qmLecture22:340}
\braket{ \Bx_1, \Bx_2 }{\psi_0} =
\psi_{1s}(\Bx_1)
\psi_{1s}(\Bx_2),
\end{equation}
%
so braket integrals must be evaluated over a six-fold space.  Recall that
%
\begin{equation}\label{eqn:qmLecture22:740}
\psi_{1s} = \inv{\sqrt{\pi} a_0^{3/2} } e^{-r/a_0},
\end{equation}
%
so
%
\begin{equation}\label{eqn:qmLecture22:760}
\bra{\psi_{1s}} x_i \ket{\psi_{1s}}
\propto
\int_0^\pi \sin\theta d\theta \int_0^{2\pi} d\phi x_i,
\end{equation}
where
\begin{equation}\label{eqn:qmLecture22:780}
x_i \in \setlr{ r \sin\theta \cos\phi, r \sin\theta \sin\phi, r \cos\theta }.
\end{equation}
%
The \( x, y \) integrals are zero because of the \( \phi \) integral, and the \( z \) integral is proportional to \( \int_0^\pi \sin(2 \theta) d\theta \), which is also zero.  This leads to zero averages
%
\begin{equation}\label{eqn:qmLecture22:360}
\expectation{\Bx_1} = 0 = \expectation{\Bx_2},
\end{equation}
so
%
\begin{equation}\label{eqn:qmLecture22:380}
\Delta E^{(1)} = 0.
\end{equation}
%
\paragraph{2. \( O(\lambda^2) \)}.
%
\begin{equation}\label{eqn:qmLecture22:400}
\begin{aligned}
\Delta E^{(2)}
&= \sum_{n \ne 0} \frac{ \Abs{ \bra{\psi_n } V \ket{\psi_0} }^2 }{E_0 - E_n}
\\ &= \sum_{n \ne 0} \frac{ \bra{\psi_0 } V \ket{\psi_n}  \bra{\psi_n } V \ket{\psi_0} }{E_0 - E_n}.
\end{aligned}
\end{equation}
%
This is a sum over all excited states.
%Here the \( \sum' \) indicates the sum over all excited states.
We expect that this will be of the form
%
\begin{equation}\label{eqn:qmLecture22:420}
\Delta E^{(2)} = - \lr{ \frac{e^2}{4 \pi \epsilon_0} }^2 \frac{C_6}{R^6},
\end{equation}
where
\( \Bx_1 \) and \( \Bx_2 \) are dipole operators.  The first time this has a non-zero expectation is when we go from the 1s to the 2p states (both 1s and 2s states are spherically symmetric).
Noting that \( E_n = -e^2/2 n^2 a_0 \), we can compute a minimum bound for the energy denominator
%
\begin{equation}\label{eqn:qmLecture22:440}
\begin{aligned}
\lr{E_n - E_0}^{\mathrm{min}}
&= 2 \lr{ E_{2p} - E_{1s} }
\\ &= 2 E_{1s} \lr{ \inv{4} - 1 }
\\ &= 2 \frac{3}{4} \Abs{E_{1s}}
\\ &= \frac{3}{2} \Abs{E_{1s}}.
\end{aligned}
\end{equation}
%
Note that the factor of two above comes from summing over the energies for both electrons.  This gives us
%
\begin{equation}\label{eqn:qmLecture22:460}
C_6
=
\frac{3}{2} \Abs{E_{1s}}
\bra{\psi_0 } \tilde{V} \ket{\psi_0},
\end{equation}
%
where
%
\begin{equation}\label{eqn:qmLecture22:480}
\tilde{V} =
\lr{
\Bx_1 \cdot \Bx_2
-3
(\Bx_1 \cdot \Rcap)
(\Bx_2 \cdot \Rcap)
}.
\end{equation}
\paragraph{What about degeneracy?}
\index{degeneracy}
%
\begin{equation}\label{eqn:qmLecture22:500}
\Delta E^{(2)}_n
= \sum_{m \ne n} \frac{ \Abs{ \bra{\psi_n } V \ket{\psi_0} }^2 }{E_0 - E_n}.
\end{equation}
If \( \bra{\psi_n} V \ket{\psi_m} \propto \delta_{n m} \) then it's okay.
In general the we can't expect the matrix element will be anything but fully populated, say
%
\begin{equation}\label{eqn:qmLecture22:520}
V =
\begin{bmatrix}
V_{11} & V_{12} & V_{13} & V_{14} \\
V_{21} & V_{22} & V_{23} & V_{24} \\
V_{31} & V_{32} & V_{33} & V_{34} \\
V_{41} & V_{42} & V_{43} & V_{44} \\
\end{bmatrix}.
\end{equation}
%
If we choose a basis so that
%
\begin{equation}\label{eqn:qmLecture22:540}
V =
\begin{bmatrix}
V_{11} &        &        &        \\
       & V_{22} &        &        \\
       &        & V_{33} &        \\
       &        &        & V_{44} \\
\end{bmatrix}.
\end{equation}
%
When this is the case, we have no mixing of elements in the sum of \cref{eqn:qmLecture22:500}

\paragraph{Degeneracy in the Stark effect}
\index{degeneracy!Stark effect}
%
\begin{equation}\label{eqn:qmLecture22:560}
H = H_0 + e \calE z,
\end{equation}
%
where
%
\begin{equation}\label{eqn:qmLecture22:580}
H_0 = \frac{\Bp^2}{2m} - \frac{e}{4 \pi \epsilon_0} \inv{\Abs{\Bx}}.
\end{equation}
Consider the states \( 2s, 2 p_x, 2p_y, 2p_z \), for which \( E_n^{(0)} \equiv E_{2 s} \), as sketched in \cref{fig:lecture22:lecture22Fig3}.
\imageFigure{../figures/phy1520-quantum/lecture22Fig3}{2s 2p degeneracy.}{fig:lecture22:lecture22Fig3}{0.2}
Because of spherical symmetry
%
\begin{equation}\label{eqn:qmLecture22:600}
\begin{aligned}
\bra{2 s} e \calE z \ket{ 2 s}      &= 0, \\
\bra{2 p_x} e \calE z \ket{ 2 p_x}  &= 0, \\
\bra{2 p_y} e \calE z \ket{ 2 p_y}  &= 0, \\
\bra{2 p_z} e \calE z \ket{ 2 p_z}  &= 0.
\end{aligned}
\end{equation}
Looking at odd and even properties, it turns out that the only off-diagonal matrix element is
%
\begin{equation}\label{eqn:qmLecture22:620}
\bra{2 s} e \calE z \ket{ 2 p_z } = V_1 = -3 e \calE a_0.
\end{equation}
%
With a \( \setlr{ 2s, 2p_x, 2p_y, 2p_z } \) basis the potential matrix is
%
\begin{equation}\label{eqn:qmLecture22:640}
\begin{bmatrix}
0 & 0 & 0 & V_1 \\
0 & 0 & 0 & 0 \\
0 & 0 & 0 & 0 \\
V_1^\conj & 0 & 0 & 0 \\
\end{bmatrix}
\end{equation}
which has the block structure
%\cref{fig:lecture22:lecture22Fig4}.
%\imageFigure{../figures/phy1520-quantum/lecture22Fig4}{CAPTION: lecture22Fig4}{fig:lecture22:lecture22Fig4}{0.2}
% 2s, 2p_z
\begin{equation}\label{eqn:qmLecture22:660}
\begin{bmatrix}
0 & -\Abs{V_1} \\
-\Abs{V_1} & 0 \\
\end{bmatrix}.
\end{equation}
This implies that the energy splitting goes as
%
\begin{equation}\label{eqn:qmLecture22:680}
E_{2s} \rightarrow
E_{2s} \pm \Abs{V_1},
\end{equation}
%
as sketched in \cref{fig:lecture22:lecture22Fig5}.
\imageFigure{../figures/phy1520-quantum/lecture22Fig5}{Stark effect energy level splitting.}{fig:lecture22:lecture22Fig5}{0.2}
The diagonalizing states corresponding to eigenvalues \( \pm 3 a_0 \calE \), are \( (\ket{2s} \mp \ket{2p_z})/\sqrt{2} \).
The matrix element above is calculated explicitly in \nbref{lecture22Integrals.nb}.
The degeneracy that is left unsplit here, and has to be accounted for should we attempt higher order perturbation calculations.
%\EndArticle
