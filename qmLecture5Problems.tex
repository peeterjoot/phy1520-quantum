%
% Copyright © 2015 Peeter Joot.  All Rights Reserved.
% Licenced as described in the file LICENSE under the root directory of this GIT repository.
%
\makeproblem{Lorentz force from classical electrodynamic Hamiltonian.}{problem:qmLecture5:1}{
\index{classical Hamiltonian!Lorentz force}

Given the classical Hamiltonian
%
\begin{dmath}\label{eqn:qmLecture5:381}
H = \inv{2 m} \lr{ \Bp - q \BA }^2 + q \phi.
\end{dmath}
%
apply the Hamiltonian equations of motion
%
\begin{equation}\label{eqn:qmLecture5:480}
\begin{aligned}
\ddt{\Bp} &= - \PD{\Bq}{H} \\
\ddt{\Bq} &= \PD{\Bp}{H},
\end{aligned}
\end{equation}

to show that this is the Hamiltonian that describes the Lorentz force equation, and to find the velocity in terms of the canonical momentum and vector potential.
} % problem

\makeanswer{problem:qmLecture5:1}{

The particle velocity follows easily
%
\begin{dmath}\label{eqn:qmLecture5:500}
\Bv
= \ddt{\Br}
= \PD{\Bp}{H}
= \inv{m} \lr{ \Bp - q \BA }.
\end{dmath}
%
For the Lorentz force we can proceed in the coordinate representation
%
\begin{dmath}\label{eqn:qmLecture5:520}
\ddt{p_k}
= - \PD{x_k}{H}
= - \frac{2}{2m} \lr{ p_m - q A_m } \PD{x_k}{}\lr{ p_m - q A_m } - q \PD{x_k}{\phi}
= q v_m \PD{x_k}{A_m} - q \PD{x_k}{\phi},
\end{dmath}
%
We also have
%
\begin{dmath}\label{eqn:qmLecture5:540}
\ddt{p_k}
=
\ddt{} \lr{m x_k + q A_k }
=
m \frac{d^2 x_k}{dt^2} + q \PD{x_m}{A_k} \frac{d x_m}{dt} + q \PD{t}{A_k}.
\end{dmath}
%
Putting these together we've got
%
\begin{dmath}\label{eqn:qmLecture5:560}
m \frac{d^2 x_k}{dt^2}
= q v_m \PD{x_k}{A_m} - q \PD{x_k}{\phi},
- q \PD{x_m}{A_k} \frac{d x_m}{dt} - q \PD{t}{A_k}
=
q v_m \lr{ \PD{x_k}{A_m} - \PD{x_m}{A_k} } + q E_k
=
q v_m \epsilon_{k m s} B_s + q E_k,
\end{dmath}
%
or
%
\begin{dmath}\label{eqn:qmLecture5:580}
m \frac{d^2 \Bx}{dt^2}
=
q \Be_k v_m \epsilon_{k m s} B_s + q E_k
= q \Bv \cross \BB + q \BE.
\end{dmath}
%
} % answer

\makeproblem{Show gauge invariance of the magnetic and electric fields.}{problem:qmLecture5:2}{
\index{gauge invariance}

After the gauge transformation of \cref{eqn:qmLecture5:420a} show that the electric and magnetic fields are unaltered.
} % problem

\makeanswer{problem:qmLecture5:2}{

For the magnetic field the transformed field is
%
\begin{dmath}\label{eqn:qmLecture5:600}
\BB'
= \spacegrad \cross \lr{ \BA + \spacegrad \chi }
= \spacegrad \cross \BA + \spacegrad \cross \lr{ \spacegrad \chi }
= \spacegrad \cross \BA
= \BB.
\end{dmath}
%
\begin{dmath}\label{eqn:qmLecture5:620}
\BE'
=
- \PD{t}{\BA'} - \spacegrad \phi'
=
- \PD{t}{}\lr{\BA + \spacegrad \chi} - \spacegrad \lr{ \phi - \PD{t}{\chi}}
=
- \PD{t}{\BA} - \spacegrad \phi
=
\BE.
\end{dmath}
%
} % answer
