%
% Copyright � 2015 Peeter Joot.  All Rights Reserved.
% Licenced as described in the file LICENSE under the root directory of this GIT repository.
%
%{
%\input{../blogpost.tex}
%\renewcommand{\basename}{timeReversalPlaneWaveAndSpinor}
%\renewcommand{\dirname}{notes/phy1520/}
%%\newcommand{\dateintitle}{}
%%\newcommand{\keywords}{}
%
%\input{../peeter_prologue_print2.tex}
%
%\usepackage{peeters_layout_exercise}
%\usepackage{peeters_braket}
%\usepackage{peeters_figures}
%\usepackage{enumerate}
%\usepackage{macros_qed}
%
%\beginArtNoToc
%
%\generatetitle{Plane wave and spinor under time reversal}
%\chapter{Plane wave and spinor under time reversal}
%\label{chap:timeReversalPlaneWaveAndSpinor}
%\paragraph{Q: \citep{sakurai2014modern} pr 4.7}

\makeoproblem{Plane wave and spinor under time reversal.}{problem:timeReversalPlaneWaveAndSpinor:1}{\citep{sakurai2014modern} pr. 4.7}{
\index{time reversal!plane wave}
\index{time reversal!spinor}

\makesubproblem{}{problem:timeReversalPlaneWaveAndSpinor:1:a}
Find the time reversed form of a spinless plane wave state in three dimensions.

\makesubproblem{}{problem:timeReversalPlaneWaveAndSpinor:1:b}
For the eigenspinor of \( \Bsigma \cdot \ncap \) expressed in terms of polar and azimuthal angles \( \beta\) and \( \gamma \), show that \( -i \sigma_y \chi^\conj(\ncap) \) has the reversed spin direction.

} % problem

\makeanswer{problem:timeReversalPlaneWaveAndSpinor:1}{

\makeSubAnswer{}{problem:timeReversalPlaneWaveAndSpinor:1:a}

The Hamiltonian for a plane wave is
%
\begin{equation}\label{eqn:timeReversalPlaneWaveAndSpinor:20}
H = \frac{\Bp^2}{2m} = i \PD{t}.
\end{equation}

Under time reversal the momentum side transforms as
%
\begin{dmath}\label{eqn:timeReversalPlaneWaveAndSpinor:40}
\Theta \frac{\Bp^2}{2m} \Theta^{-1}
=
\frac{\lr{ \Theta \Bp \Theta^{-1}} \cdot \lr{ \Theta \Bp \Theta^{-1}} }{2m}
=
\frac{(-\Bp) \cdot (-\Bp)}{2m}
=
\frac{\Bp^2}{2m}.
\end{dmath}

The time derivative side of the equation is also time reversal invariant
\begin{dmath}\label{eqn:timeReversalPlaneWaveAndSpinor:60}
\Theta i \PD{t}{} \Theta^{-1}
=
\Theta i \Theta^{-1} \Theta \PD{t}{} \Theta^{-1}
=
-i \PD{(-t)}{}
=
i \PD{t}{}.
\end{dmath}

Solutions to this equation are linear combinations of
%
\begin{dmath}\label{eqn:timeReversalPlaneWaveAndSpinor:80}
\psi(\Bx, t) = e^{i \Bk \cdot \Bx - i E t/\Hbar},
\end{dmath}

where \( \Hbar^2 \Bk^2/2m = E \), the energy of the particle.  Under time reversal we have
%
\begin{dmath}\label{eqn:timeReversalPlaneWaveAndSpinor:100}
\psi(\Bx, t)
\rightarrow e^{-i \Bk \cdot \Bx + i E (-t)/\Hbar}
= \lr{ e^{i \Bk \cdot \Bx - i E (-t)/\Hbar} }^\conj
=
\psi^\conj(\Bx, -t)
\end{dmath}

\makeSubAnswer{}{problem:timeReversalPlaneWaveAndSpinor:1:b}

The text uses a requirement for time reversal of spin states to show that the Pauli matrix form of the time reversal operator is
%
\begin{dmath}\label{eqn:timeReversalPlaneWaveAndSpinor:120}
\Theta = -i \sigma_y \eta K,
\end{dmath}

where \( K \) is a complex conjugating operator, and \( \eta \) is a phase factor with \( \Abs{\eta}^2 = 1 \).  The form of the spin up state used in that demonstration was
%
\begin{dmath}\label{eqn:timeReversalPlaneWaveAndSpinor:140}
\ket{\ncap ; +}
= e^{-i S_z \beta/\Hbar} e^{-i S_y \gamma/\Hbar} \ket{+}
= e^{-i \sigma_z \beta/2} e^{-i \sigma_y \gamma/2} \ket{+}
= \lr{ \cos(\beta/2) - i \sigma_z \sin(\beta/2) }
 \lr{ \cos(\gamma/2) - i \sigma_y \sin(\gamma/2) } \ket{+}
= \lr{ \cos(\beta/2) - i \PauliZ \sin(\beta/2) }
 \lr{ \cos(\gamma/2) - i \PauliY \sin(\gamma/2) } \ket{+}
=
\begin{bmatrix}
e^{-i\beta/2} & 0 \\
0 & e^{i \beta/2}
\end{bmatrix}
\begin{bmatrix}
\cos(\gamma/2) & -\sin(\gamma/2)  \\
\sin(\gamma/2) & \cos(\gamma/2)
\end{bmatrix}
\begin{bmatrix}
1 \\
0
\end{bmatrix}
=
\begin{bmatrix}
e^{-i\beta/2} & 0 \\
0 & e^{i \beta/2}
\end{bmatrix}
\begin{bmatrix}
\cos(\gamma/2) \\
\sin(\gamma/2) \\
\end{bmatrix}
=
\begin{bmatrix}
\cos(\gamma/2)
e^{-i\beta/2}
\\
\sin(\gamma/2)
e^{i \beta/2}
\end{bmatrix}.
\end{dmath}

The state orthogonal to this one is claimed to be
%
\begin{dmath}\label{eqn:timeReversalPlaneWaveAndSpinor:180}
\ket{\ncap ; -}
= e^{-i S_z \beta/\Hbar} e^{-i S_y (\gamma + \pi)/\Hbar} \ket{+}
= e^{-i \sigma_z \beta/2} e^{-i \sigma_y (\gamma + \pi)/2} \ket{+}.
\end{dmath}

We have
%
\begin{dmath}\label{eqn:timeReversalPlaneWaveAndSpinor:200}
\cos((\gamma + \pi)/2)
=
\Real e^{i(\gamma + \pi)/2}
=
\Real i e^{i\gamma/2}
=
-\sin(\gamma/2),
\end{dmath}

and
\begin{dmath}\label{eqn:timeReversalPlaneWaveAndSpinor:220}
\sin((\gamma + \pi)/2)
=
\Imag e^{i(\gamma + \pi)/2}
=
\Imag i e^{i\gamma/2}
=
\cos(\gamma/2),
\end{dmath}

so we should have
%
\begin{dmath}\label{eqn:timeReversalPlaneWaveAndSpinor:240}
\ket{\ncap ; -}
=
\begin{bmatrix}
-\sin(\gamma/2)
e^{-i\beta/2}
\\
\cos(\gamma/2)
e^{i \beta/2}
\end{bmatrix}.
\end{dmath}

This looks right, but we can sanity check orthogonality
%
\begin{dmath}\label{eqn:timeReversalPlaneWaveAndSpinor:260}
\braket{\ncap ; -}{\ncap ; +}
=
\begin{bmatrix}
-\sin(\gamma/2)
e^{i\beta/2}
&
\cos(\gamma/2)
e^{-i \beta/2}
\end{bmatrix}
\begin{bmatrix}
\cos(\gamma/2)
e^{-i\beta/2}
\\
\sin(\gamma/2)
e^{i \beta/2}
\end{bmatrix}
=
0,
\end{dmath}

as expected.

The task at hand appears to be the operation on the column representation of \( \ket{\ncap; +} \) using the Pauli representation of the time reversal operator.  With the phase factor \( \eta = 1 \) the time reversal action on the spin up state is
%
\begin{dmath}\label{eqn:timeReversalPlaneWaveAndSpinor:160}
\Theta \ket{\ncap ; +}
=
-i \sigma_y K
\begin{bmatrix}
e^{-i\beta/2} \cos(\gamma/2) \\
e^{i \beta/2} \sin(\gamma/2)
\end{bmatrix}
=
-i \PauliY
\begin{bmatrix}
e^{i\beta/2} \cos(\gamma/2) \\
e^{-i \beta/2} \sin(\gamma/2)
\end{bmatrix}
=
\begin{bmatrix}
0 & -1 \\
1 & 0
\end{bmatrix}
\begin{bmatrix}
e^{i\beta/2} \cos(\gamma/2) \\
e^{-i \beta/2} \sin(\gamma/2)
\end{bmatrix}
=
\begin{bmatrix}
-e^{-i \beta/2} \sin(\gamma/2) \\
e^{i\beta/2} \cos(\gamma/2) \\
\end{bmatrix}
= \ket{\ncap ; -}. \qedmarker
\end{dmath}
} % answer

Observe that we need \( \eta = i \) to have this match eq. (4.79) in the text where \( \Theta = i^{2m} \ket{j, -m} \).

%}
%\EndArticle
