%
% Copyright � 2015 Peeter Joot.  All Rights Reserved.
% Licenced as described in the file LICENSE under the root directory of this GIT repository.
%
%\input{../blogpost.tex}
%\renewcommand{\basename}{pauliMatrixXYgeometry}
%\renewcommand{\dirname}{notes/phy1540/}
%%\newcommand{\dateintitle}{}
%%\newcommand{\keywords}{}
%
%\input{../peeter_prologue_print2.tex}
%
%
%\beginArtNoToc
%
%\generatetitle{An observation about the geometry of Pauli x,y matrices}
%\chapter{An observation about the geometry of Pauli x,y matrices}
%\label{chap:pauliMatrixXYgeometry}

\paragraph{Motivation}

\index{Pauli matrix!orthonality}
The conventional form for the Pauli matrices is
%
\begin{equation}\label{eqn:pauliMatrixXYgeometry:20}
\begin{aligned}
\sigma_x &= \PauliX, \\
\sigma_y &= \PauliY, \\
\sigma_z &= \PauliZ.
\end{aligned}
\end{equation}
%
In \citep{desai2009quantum} these forms are derived based on the commutation relations
%
\begin{equation}\label{eqn:pauliMatrixXYgeometry:40}
\antisymmetric{\sigma_r}{\sigma_s} = 2 i \epsilon_{r s t} \sigma_t,
\end{equation}
%
by defining raising and lowering operators \( \sigma_{\pm} = \sigma_x \pm i \sigma_y \) and figuring out what form the matrix must take.  I noticed an interesting geometrical relation hiding in that derivation if \( \sigma_{+} \) is not assumed to be real.

\paragraph{Derivation}

For completeness, I'll repeat the argument of \citep{desai2009quantum}, which builds on the commutation relations of the raising and lowering operators.  Those are
%
\begin{dmath}\label{eqn:pauliMatrixXYgeometry:60}
\antisymmetric{\sigma_z}{\sigma_{\pm}}
=
\sigma_z \lr{ \sigma_x \pm i \sigma_y }
-\lr{ \sigma_x \pm i \sigma_y } \sigma_z
=
\antisymmetric{\sigma_z}{\sigma_x} \pm i \antisymmetric{\sigma_z}{\sigma_y}
=
2 i \sigma_y \pm i (-2 i) \sigma_x
= \pm 2 \lr{ \sigma_x \pm i \sigma_y }
= \pm 2 \sigma_{\pm},
\end{dmath}
%
and
%
\begin{dmath}\label{eqn:pauliMatrixXYgeometry:80}
\antisymmetric{\sigma_{+}}{\sigma_{-}}
=
\lr{ \sigma_x + i \sigma_y } \lr{ \sigma_x - i \sigma_y }
-\lr{ \sigma_x - i \sigma_y } \lr{ \sigma_x + i \sigma_y }
=
-i \sigma_x \sigma_y + i \sigma_y \sigma_x
- i \sigma_x \sigma_y + i \sigma_y \sigma_x
= 2 i \antisymmetric{ \sigma_y }{\sigma_x}
= 2 i (-2i) \sigma_z
= 4 \sigma_z.
\end{dmath}
From these a matrix representation containing unknown values can be assumed.  Let
%
\begin{dmath}\label{eqn:pauliMatrixXYgeometry:100}
\sigma_{+} =
\begin{bmatrix}
a & b \\
c & d
\end{bmatrix}.
\end{dmath}
%
The commutator with \( \sigma_z \) can be computed
%
\begin{dmath}\label{eqn:pauliMatrixXYgeometry:120}
\antisymmetric{\sigma_z}{\sigma_{+}}
=
\PauliZ
\begin{bmatrix}
a & b \\
c & d
\end{bmatrix}
-
\begin{bmatrix}
a & b \\
c & d
\end{bmatrix}
\PauliZ
=
\begin{bmatrix}
a & b \\
-c & -d
\end{bmatrix}
-
\begin{bmatrix}
a & -b \\
c & -d
\end{bmatrix}
=
2
\begin{bmatrix}
0 & b \\
-c & 0
\end{bmatrix}.
\end{dmath}
Now compare this with \cref{eqn:pauliMatrixXYgeometry:60}
%
\begin{dmath}\label{eqn:pauliMatrixXYgeometry:140}
2
\begin{bmatrix}
0 & b \\
-c & 0
\end{bmatrix}
=
2 \sigma_{+}
=
2
\begin{bmatrix}
a & b \\
d & d
\end{bmatrix}.
\end{dmath}
%
This shows that \( a = 0 \), and \( d = 0 \).  Similarly the \( \sigma_z \) commutator with the lowering operator is
%
\begin{dmath}\label{eqn:pauliMatrixXYgeometry:160}
\antisymmetric{\sigma_z}{\sigma_{-}}
=
\PauliZ
\begin{bmatrix}
0 & -c^\conj \\
b^\conj & 0
\end{bmatrix}
-
\begin{bmatrix}
0 & -c^\conj \\
b^\conj & 0
\end{bmatrix}
\PauliZ
=
\begin{bmatrix}
0 & -c^\conj \\
-b^\conj & 0
\end{bmatrix}
-
\begin{bmatrix}
0 & c^\conj \\
b^\conj & 0
\end{bmatrix}
=
-2
\begin{bmatrix}
0 & c^\conj \\
b^\conj & 0
\end{bmatrix}.
\end{dmath}
Again comparing to \cref{eqn:pauliMatrixXYgeometry:60}, we have
\begin{dmath}\label{eqn:pauliMatrixXYgeometry:180}
-2
\begin{bmatrix}
0 & c^\conj \\
b^\conj & 0
\end{bmatrix}
= - 2 \sigma_{-}
= - 2
\begin{bmatrix}
0 & -c^\conj \\
b^\conj & 0
\end{bmatrix},
\end{dmath}
%
so \( c = 0 \).  Computing the commutator of the raising and lowering operators fixes \( b \)
%
\begin{dmath}\label{eqn:pauliMatrixXYgeometry:200}
\antisymmetric{\sigma_{+}}{\sigma_{-}}
=
\begin{bmatrix}
0 & b \\
0 & 0 \\
\end{bmatrix}
\begin{bmatrix}
0 & 0 \\
b^\conj & 0 \\
\end{bmatrix}
-
\begin{bmatrix}
0 & 0 \\
b^\conj & 0 \\
\end{bmatrix}
\begin{bmatrix}
0 & b \\
0 & 0 \\
\end{bmatrix}
=
\begin{bmatrix}
\Abs{b}^2 & 0 \\
0 & 0
\end{bmatrix}
-
\begin{bmatrix}
0 & 0
0 & -\Abs{b}^2 \\
\end{bmatrix}
=
\Abs{b}^2 \PauliZ
=
\Abs{b}^2 \sigma_z.
\end{dmath}
%
From \cref{eqn:pauliMatrixXYgeometry:80} it must be that \( \Abs{b}^2 = 4\), so the most general form of the raising operator is
%
\begin{dmath}\label{eqn:pauliMatrixXYgeometry:220}
\sigma_{+}
=
2
\begin{bmatrix}
0 & e^{i \phi}  \\
0 & 0
\end{bmatrix}.
\end{dmath}
%
\paragraph{Observation}
The conventional choice is to set \( \phi = 0 \), but I found it interesting to see the form of \( \sigma_x, \sigma_y \) without that choice.  That is
%
\begin{dmath}\label{eqn:pauliMatrixXYgeometry:240}
\sigma_x = \inv{2} \lr{ \sigma_{+} + \sigma_{-} }
=
\begin{bmatrix}
0 & e^{i \phi}  \\
e^{-i \phi} & 0 \\
\end{bmatrix},
\end{dmath}
%
\begin{dmath}\label{eqn:pauliMatrixXYgeometry:260}
\sigma_y = \inv{2 i} \lr{ \sigma_{+} - \sigma_{-} }
=
\begin{bmatrix}
0 & -i e^{i \phi}  \\
-i e^{-i \phi} & 0 \\
\end{bmatrix}
=
\begin{bmatrix}
0 & e^{i (\phi - \pi/2) }  \\
e^{-i (\phi - \pi/2)} & 0 \\
\end{bmatrix}.
\end{dmath}
%
Notice that the Pauli matrices \( \sigma_x \) and \( \sigma_y \) actually both have the same form as \( \sigma_x \), but the phase of the complex argument of each differs by \ang{90}.  That \ang{90} separation isn't obvious in the standard form \cref{eqn:pauliMatrixXYgeometry:20}.

It's a small detail, but I thought it was kind of cool that the orthogonality of these matrix unit vector representations is built directly into the structure of their matrix representations.

%\EndArticle
