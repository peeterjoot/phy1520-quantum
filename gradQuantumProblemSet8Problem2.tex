%
% Copyright � 2015 Peeter Joot.  All Rights Reserved.
% Licenced as described in the file LICENSE under the root directory of this GIT repository.
%
\makeoproblem{Quadrupolar potential.}{gradQuantum:problemSet8:2}{2015 ps8 p2}{
\index{quadrupolar potential}

Consider a p-orbital electron of hydrogen with \( \ket{ n,l = 1, m } \), with \( m = 0, \pm 1 \), subject to an external potential
%
\begin{dmath}\label{eqn:gradQuantumProblemSet8Problem2:20}
V(x, y, z) = \lambda(x^2 - y^2),
\end{dmath}
%
with \( \lambda \) being a constant. For fixed \( n \), obtain the correct eigenstates which diagonalize
the perturbation, without worrying about doing radial integrals explicitly. Show that the three-fold degeneracy of the
p-orbital is completely broken by the perturbation to linear order in \( \lambda \).
%
%\makesubproblem{}{gradQuantum:problemSet8:2a}
} % makeproblem
%
\makeanswer{gradQuantum:problemSet8:2}{
\withproblemsetsParagraph{
%\makeSubAnswer{}{gradQuantum:problemSet8:2a}
%
The potential in spherical coordinates is
%
\begin{dmath}\label{eqn:gradQuantumProblemSet8Problem2:40}
V = \lambda r^2 \sin^2\theta \lr{ \cos^2\phi - \sin^2\phi } = \lambda r^2 \sin^2\theta \cos(2 \phi).
\end{dmath}
%
The p-orbital wave functions are
%
\begin{dmath}\label{eqn:gradQuantumProblemSet8Problem2:60}
\psi_{n1m}(r, \theta, \phi) = R_n(r) Y_{1,m}(\theta, \phi),
\end{dmath}
%
where
\begin{equation}\label{eqn:gradQuantumProblemSet8Problem2:80}
\begin{aligned}
Y_{1,1}(\theta, \phi) &= -\frac{1}{2} \sqrt{\frac{3}{2 \pi }} e^{i \phi } \sin  \theta, \\
Y_{1,0}(\theta, \phi) &= \frac{1}{2} \sqrt{\frac{3}{\pi }} \cos  \theta, \\
Y_{1,-1}(\theta, \phi) &= \frac{1}{2} \sqrt{\frac{3}{2 \pi }} e^{-i \phi } \sin  \theta.
\end{aligned}
\end{equation}
%
That is enough information to construct the matrix element of the perturbing potential with respect to these states.  Those are
%
\begin{equation}\label{eqn:gradQuantumProblemSet8Problem2:100}
\begin{aligned}
&\bra{n' 1 m'} V \ket{n 1 m} \\
&=
\int_0^\infty r^2 dr \int_0^\pi \sin\theta d\theta 
\,\times \\ &\qquad
\int_0^{2 \pi} d\phi R_n(r) Y^\conj_{1, m'}(\theta, \phi) \lambda r^2 \sin^2\theta \cos(2 \phi) R_n(r) Y_{1, m}(\theta, \phi) \\
&=
\lambda \int_0^\infty r^4 dr R^2_n(r)
\int_0^\pi \sin^3\theta \cos(2\phi) d\theta \int_0^{2 \pi} d\phi Y^\conj_{1, m'}(\theta, \phi) Y_{1, m}(\theta, \phi) \\
&=
\lambda \int_0^\infty r^4 dr R^2_n(r)
\begin{bmatrix}
0 & 0 & -\frac{2}{5} \\
0 & 0 & 0 \\
-\frac{2}{5} & 0 & 0 \\
\end{bmatrix}.
\end{aligned}
\end{equation}
%
See
\nbref{ps8:quadrupolarPotentialPorbitalSplitting.nb}
for a computation of the matrix.
It has eigenvalues
%
\begin{dmath}\label{eqn:gradQuantumProblemSet8Problem2:120}
\lambda \int_0^\infty r^4 dr R^2_n(r)
\setlr{
 -\frac{2}{5},
 \frac{2}{5},
 0
}.
\end{dmath}
%
We see that the radial factor \( R_n(r) \) of these wave function provides only a constant adjustment to the energy levels splitting that breaks the degeneracy.   That degeneracy is completely broken by this perturbation.

The respective eigenvectors for this matrix are
\begin{dmath}\label{eqn:gradQuantumProblemSet8Problem2:140}
\setlr{
\inv{\sqrt{2}}
\begin{bmatrix}
1 \\
0 \\
1
\end{bmatrix},
\inv{\sqrt{2}}
\begin{bmatrix}
1 \\
0 \\
-1
\end{bmatrix},
\begin{bmatrix}
0 \\
1 \\
0
\end{bmatrix}
},
\end{dmath}
%
so the wave functions, say \( \setlr{\psi_{n,-1},\psi_{n,1}, \psi_{n,0}} \), that diagonalize this perturbation potential are
%
\begin{dmath}\label{eqn:gradQuantumProblemSet8Problem2:160}
\begin{aligned}
\psi_{n,-1}(r, \theta, \phi) &= \frac{R_n(r)}{\sqrt{2}} \lr{ Y_{-1,1}(\theta, \phi) + Y_{1,1}(\theta, \phi) } = \frac{R_n(r)}{2 i} \sqrt{\frac{3}{\pi }} \sin\phi \sin\theta, \\
\psi_{n,1}(r, \theta, \phi) &= \frac{R_n(r)}{\sqrt{2}} \lr{ Y_{-1,1}(\theta, \phi) - Y_{1,1}(\theta, \phi) } = -\frac{R_n(r)}{2} \sqrt{\frac{3}{\pi }} \cos\phi \sin\theta, \\
\psi_{n,0}(r, \theta, \phi) &= R_n(r) Y_{1,0}(\theta, \phi) = \frac{R_n(r)}{2} \sqrt{\frac{3}{\pi }} \cos\theta.
\end{aligned}
\end{dmath}
%
It is natural to adjust the phases above, forming an alternate basis
%
\begin{dmath}\label{eqn:gradQuantumProblemSet8Problem2:180}
\setlr{-\psi_{n,1}, i\psi_{n,-1}, \psi_{n,0}}
=
\frac{R_n(r)}{2} \sqrt{\frac{3}{\pi }} \rcap,
\end{dmath}
%
where \( \rcap = \setlr{ \sin\theta \cos\phi, \sin\theta \sin\phi, \cos\theta } \), the set of components of the unit vector parameterized by \( \theta, \phi \).
In this basis that level splitting is \( \lambda \int_0^\infty r^4 dr R^2_n(r) \setlr{ \frac{2}{5}, -\frac{2}{5}, 0 } \) respectively.
}
}
