%
% Copyright � 2015 Peeter Joot.  All Rights Reserved.
% Licenced as described in the file LICENSE under the root directory of this GIT repository.
%
\input{../assignment.tex}
\renewcommand{\basename}{gradQuantumProblemSet8}
\renewcommand{\dirname}{notes/phy1520-quantum/}
\newcommand{\keywords}{Graduate Quantum Mechanics, PHY1520H}
\newcommand{\dateintitle}{}
\input{../peeter_prologue_print2.tex}

\usepackage{peeters_layout_exercise}
\usepackage{peeters_braket}
%\usepackage{phy1520}
%\usepackage{siunitx}
%\usepackage{esint} % \oiint

\renewcommand{\QuestionNB}{\alph{Question}.\ }
\renewcommand{\theQuestion}{\alph{Question}}

\newcommand{\nbref}[1]{%
\itemRef{phy1520}{#1}%
}

\beginArtNoToc
\generatetitle{PHY1520H Graduate Quantum Mechanics.  Problem Set 8: Perturbation theory}
%\chapter{Pertubation theory}
\label{chap:gradQuantumProblemSet8}

%\section{Disclaimer}
%
%This is an ungraded set of answers to the problems posed.

%
% Copyright � 2015 Peeter Joot.  All Rights Reserved.
% Licenced as described in the file LICENSE under the root directory of this GIT repository.
%
\makeoproblem{Anharmonic oscillator.}{gradQuantum:problemSet8:1}{2015 ps8 p1}{
\index{anharmonic oscillator}

Consider a quantum particle in the ground state of a 1D anharmonic oscillator potential
%
\begin{equation}\label{eqn:gradQuantumProblemSet8Problem1:20}
V(x) = \inv{2} m \omega^2 x^2 + \lambda x^4 = V_0 + \lambda V'.
\end{equation}
%
Compute the first and second order energy shift of this oscillator perturbatively in \( \lambda \).
%
%\makesubproblem{}{gradQuantum:problemSet8:1a}
} % makeproblem
%
\makeanswer{gradQuantum:problemSet8:1}{
\withproblemsetsParagraph{

Using \nbref{ps8:harmonicOscillatorRaiseAndLoweringOperators.nb} the action of the potential on the ground state is
%
\begin{equation}\label{eqn:gradQuantumProblemSet8Problem1:40}
\begin{aligned}
V' \ket{0}
&= x^4 \ket{0}
\\ &=
x_0^4 \lr{ \frac{3}{4} \ket{0}
+ \frac{3}{\sqrt{2}} \ket{2}
+ \sqrt{\frac{3}{2}} \ket{4}
}.
\end{aligned}
\end{equation}
%
That allows us to compute the first order energy shift
%
\begin{equation}\label{eqn:gradQuantumProblemSet8Problem1:60}
\begin{aligned}
\Delta^{(1)}
&= \bra{0} V' \ket{0}
\\ &= \frac{3}{4} x_0^4.
\end{aligned}
\end{equation}
%
Writing the perturbed state as
%
\begin{equation}\label{eqn:gradQuantumProblemSet8Problem1:80}
\ket{n} = \ket{0} + \lambda \ket{0}' + \lambda^2 \ket{0}'' + \cdots,
\end{equation}
%
the first order perturbation \( \ket{0}' \) of the ground state is
%
\begin{equation}\label{eqn:gradQuantumProblemSet8Problem1:100}
\begin{aligned}
\ket{0}'
&= \sum_{m \ne 0} \frac{\ket{m}\bra{m} x^4 \ket{0} }{\Hbar \omega/2 - \Hbar \omega( m + 1/2 ) }
\\ &=
- \frac{ x_0^4}{\Hbar \omega} \sum_{m \ne 0} \frac{\ket{m}\bra{m} }{m}
\lr{ \frac{3}{4} \ket{0}
+ \frac{3}{\sqrt{2}} \ket{2}
+ \sqrt{\frac{3}{2}} \ket{4}
}
\\ &=
- \frac{ x_0^4}{\Hbar \omega}
\lr{
\inv{2}
  \frac{3}{\sqrt{2}} \ket{2}
+
\inv{4}\sqrt{\frac{3}{2}} \ket{4}
}.
\end{aligned}
\end{equation}
%
The second order energy shift can now be calculated, and is
%
\begin{equation}\label{eqn:gradQuantumProblemSet8Problem1:120}
\begin{aligned}
\Delta^{(2)}
&=
\bra{0} V' \ket{0}'
\\ &=
- \frac{ x_0^8}{\Hbar \omega}
\lr{ \frac{3}{4} \bra{0}
+ \frac{3}{\sqrt{2}} \bra{2}
+ \sqrt{\frac{3}{2}} \bra{4}
}
\lr{
  \inv{2}\frac{3}{\sqrt{2}} \ket{2}
+ \inv{4}\sqrt{\frac{3}{2}} \ket{4}
}
\\ &=
- \frac{ x_0^8}{\Hbar \omega} \frac{21}{8}.
\end{aligned}
\end{equation}
%
To second order the total energy shift is
%\begin{dmath}\label{eqn:gradQuantumProblemSet8Problem1:140}
\boxedEquation{eqn:gradQuantumProblemSet8Problem1:160}{
\Delta
= \frac{3}{4} \lambda x_0^4
- \frac{21 x_0^8 \lambda^2}{8 \Hbar \omega}.
}
%\end{dmath}
%
%\makeSubAnswer{}{gradQuantum:problemSet8:1a}
}
}

%
% Copyright � 2015 Peeter Joot.  All Rights Reserved.
% Licenced as described in the file LICENSE under the root directory of this GIT repository.
%
\makeoproblem{Quadrupolar potential.}{gradQuantum:problemSet8:2}{2015 ps8 p2}{
\index{quadrupolar potential}

Consider a p-orbital electron of hydrogen with \( \ket{ n,l = 1, m } \), with \( m = 0, \pm 1 \), subject to an external potential
%
\begin{equation}\label{eqn:gradQuantumProblemSet8Problem2:20}
V(x, y, z) = \lambda(x^2 - y^2),
\end{dmath}
%
with \( \lambda \) being a constant. For fixed \( n \), obtain the correct eigenstates which diagonalize
the perturbation, without worrying about doing radial integrals explicitly. Show that the three-fold degeneracy of the
p-orbital is completely broken by the perturbation to linear order in \( \lambda \).
%
%\makesubproblem{}{gradQuantum:problemSet8:2a}
} % makeproblem
%
\makeanswer{gradQuantum:problemSet8:2}{
\withproblemsetsParagraph{
%\makeSubAnswer{}{gradQuantum:problemSet8:2a}
%
The potential in spherical coordinates is
%
\begin{equation}\label{eqn:gradQuantumProblemSet8Problem2:40}
V = \lambda r^2 \sin^2\theta \lr{ \cos^2\phi - \sin^2\phi } = \lambda r^2 \sin^2\theta \cos(2 \phi).
\end{dmath}
%
The p-orbital wave functions are
%
\begin{equation}\label{eqn:gradQuantumProblemSet8Problem2:60}
\psi_{n1m}(r, \theta, \phi) = R_n(r) Y_{1,m}(\theta, \phi),
\end{dmath}
%
where
\begin{equation}\label{eqn:gradQuantumProblemSet8Problem2:80}
\begin{aligned}
Y_{1,1}(\theta, \phi) &= -\frac{1}{2} \sqrt{\frac{3}{2 \pi }} e^{i \phi } \sin  \theta, \\
Y_{1,0}(\theta, \phi) &= \frac{1}{2} \sqrt{\frac{3}{\pi }} \cos  \theta, \\
Y_{1,-1}(\theta, \phi) &= \frac{1}{2} \sqrt{\frac{3}{2 \pi }} e^{-i \phi } \sin  \theta.
\end{aligned}
\end{equation}
%
That is enough information to construct the matrix element of the perturbing potential with respect to these states.  Those are
%
\begin{equation}\label{eqn:gradQuantumProblemSet8Problem2:100}
\begin{aligned}
&\bra{n' 1 m'} V \ket{n 1 m} \\
&=
\int_0^\infty r^2 dr \int_0^\pi \sin\theta d\theta 
\,\times \\ &\qquad
\int_0^{2 \pi} d\phi R_n(r) Y^\conj_{1, m'}(\theta, \phi) \lambda r^2 \sin^2\theta \cos(2 \phi) R_n(r) Y_{1, m}(\theta, \phi) \\
&=
\lambda \int_0^\infty r^4 dr R^2_n(r)
\int_0^\pi \sin^3\theta \cos(2\phi) d\theta \int_0^{2 \pi} d\phi Y^\conj_{1, m'}(\theta, \phi) Y_{1, m}(\theta, \phi) \\
&=
\lambda \int_0^\infty r^4 dr R^2_n(r)
\begin{bmatrix}
0 & 0 & -\frac{2}{5} \\
0 & 0 & 0 \\
-\frac{2}{5} & 0 & 0 \\
\end{bmatrix}.
\end{aligned}
\end{equation}
%
See
\nbref{ps8:quadrupolarPotentialPorbitalSplitting.nb}
for a computation of the matrix.
It has eigenvalues
%
\begin{dmath}\label{eqn:gradQuantumProblemSet8Problem2:120}
\lambda \int_0^\infty r^4 dr R^2_n(r)
\setlr{
 -\frac{2}{5},
 \frac{2}{5},
 0
}.
\end{dmath}
%
We see that the radial factor \( R_n(r) \) of these wave function provides only a constant adjustment to the energy levels splitting that breaks the degeneracy.   That degeneracy is completely broken by this perturbation.

The respective eigenvectors for this matrix are
\begin{dmath}\label{eqn:gradQuantumProblemSet8Problem2:140}
\setlr{
\inv{\sqrt{2}}
\begin{bmatrix}
1 \\
0 \\
1
\end{bmatrix},
\inv{\sqrt{2}}
\begin{bmatrix}
1 \\
0 \\
-1
\end{bmatrix},
\begin{bmatrix}
0 \\
1 \\
0
\end{bmatrix}
},
\end{dmath}
%
so the wave functions, say \( \setlr{\psi_{n,-1},\psi_{n,1}, \psi_{n,0}} \), that diagonalize this perturbation potential are
%
\begin{dmath}\label{eqn:gradQuantumProblemSet8Problem2:160}
\begin{aligned}
\psi_{n,-1}(r, \theta, \phi) &= \frac{R_n(r)}{\sqrt{2}} \lr{ Y_{-1,1}(\theta, \phi) + Y_{1,1}(\theta, \phi) } = \frac{R_n(r)}{2 i} \sqrt{\frac{3}{\pi }} \sin\phi \sin\theta, \\
\psi_{n,1}(r, \theta, \phi) &= \frac{R_n(r)}{\sqrt{2}} \lr{ Y_{-1,1}(\theta, \phi) - Y_{1,1}(\theta, \phi) } = -\frac{R_n(r)}{2} \sqrt{\frac{3}{\pi }} \cos\phi \sin\theta, \\
\psi_{n,0}(r, \theta, \phi) &= R_n(r) Y_{1,0}(\theta, \phi) = \frac{R_n(r)}{2} \sqrt{\frac{3}{\pi }} \cos\theta.
\end{aligned}
\end{dmath}
%
It is natural to adjust the phases above, forming an alternate basis
%
\begin{dmath}\label{eqn:gradQuantumProblemSet8Problem2:180}
\setlr{-\psi_{n,1}, i\psi_{n,-1}, \psi_{n,0}}
=
\frac{R_n(r)}{2} \sqrt{\frac{3}{\pi }} \rcap,
\end{dmath}
%
where \( \rcap = \setlr{ \sin\theta \cos\phi, \sin\theta \sin\phi, \cos\theta } \), the set of components of the unit vector parameterized by \( \theta, \phi \).
In this basis that level splitting is \( \lambda \int_0^\infty r^4 dr R^2_n(r) \setlr{ \frac{2}{5}, -\frac{2}{5}, 0 } \) respectively.
}
}

%
% Copyright � 2015 Peeter Joot.  All Rights Reserved.
% Licenced as described in the file LICENSE under the root directory of this GIT repository.
%
\makeoproblem{Harmonic oscillator.}{gradQuantum:problemSet8:3}{2015 ps8 p3}{
\index{harmonic oscillator!anharmonic perturbation}

Consider a 2D harmonic oscillator with
%
\begin{dmath}\label{eqn:gradQuantumProblemSet8Problem3:20}
H =
\frac{p_x^2}{2m}
+\frac{p_y^2}{2m}
+ \inv{2} m \omega^2 \lr{ x^2 + y^2 }
\end{dmath}

Turn on an anharmonic perturbation
%
\begin{dmath}\label{eqn:gradQuantumProblemSet8Problem3:40}
V = \lambda g_1 \frac{m^2 \omega^3}{\Hbar} \lr{ x^4 + y^4 } + \lambda^2 g_2 m \omega^2 x y,
\end{dmath}
%
Note that the potentials have been altered from the original problem statement to have dimensions of energy with dimensionless scale factors \( g_1, g_2, \lambda \).

\makesubproblem{}{gradQuantum:problemSet8:3a}
%
Derive the equations for the energy shifts and perturbed states for a second order perturbing potential of the form above.

\makesubproblem{}{gradQuantum:problemSet8:3b}
%
Find the perturbed eigenstate and the corresponding energy shifts up to \( O(\lambda^2) \) for the ground state.  Ignore terms of \( O(\lambda^3) \).

\makesubproblem{}{gradQuantum:problemSet8:3c}
%
Do the same for the first two degenerate states.

} % makeproblem
%
\makeanswer{gradQuantum:problemSet8:3}{
\withproblemsetsParagraph{
\makeSubAnswer{}{gradQuantum:problemSet8:3a}
%
With a \( \lambda^2 \) perturbation we have to step back and revisit the derivation of the energy level and perturbed state formulas.  Given
%
\begin{dmath}\label{eqn:gradQuantumProblemSet8Problem3:60}
H = H_0 + \lambda V_1 + \lambda^2 V_2,
\end{dmath}
%
with known solution \( H_0 \ket{n^{(0)}} = E^{(0)} \ket{n^{(0)}} \), we seek the a power series representation of the perturbed ket and an energy shift \( \Delta \)
%
\begin{dmath}\label{eqn:gradQuantumProblemSet8Problem3:80}
\ket{n} = \ket{n_0} + \lambda \ket{n_1} + \lambda^2 \ket{n_2} + \cdots
\end{dmath}
\begin{dmath}\label{eqn:gradQuantumProblemSet8Problem3:100}
\Delta = \lambda \Delta^{(1)} + \lambda^2 \Delta^{(2)} + \cdots
\end{dmath}

where
%
\begin{dmath}\label{eqn:gradQuantumProblemSet8Problem3:120}
H \ket{n} = \lr{ E^{(0)} + \Delta } \ket{n}.
\end{dmath}
%
We can assume that the we have the same sort of representation of the perturbed state
%
\begin{dmath}\label{eqn:gradQuantumProblemSet8Problem3:140}
\ket{n} = \ket{n^{(0)}} + \frac{\overbar{P}_n}{E^{(0)} - H_0} \lr{ \lambda_1 V_1 + \lambda^2 V_2 - \Delta } \ket{n},
\end{dmath}
%
where
%
\begin{dmath}\label{eqn:gradQuantumProblemSet8Problem3:160}
\overbar{P}_n = 1 - \ket{n^{(0)}}\bra{n^{(0)}} = \sum_{m \ne n} \ket{m^{(0)}}\bra{m^{(0)}}.
\end{dmath}
%
To check this, operating with \( E^{(0)} - H_0 \), we have
%
\begin{dmath}\label{eqn:gradQuantumProblemSet8Problem3:180}
\lr{ E^{(0)} - H_0 } \ket{n}
=
\lr{ E^{(0)} - H_0 } \ket{n^{(0)}} +
\overbar{P}_n \lr{ \lambda_1 V_1 + \lambda^2 V_2 - \Delta } \ket{n}
=
\lr{ 1 - \ket{n^{(0)}}\bra{n^{(0)}} } \lr{ H - H_0 - \Delta } \ket{n},
\end{dmath}
%
or
\begin{dmath}\label{eqn:gradQuantumProblemSet8Problem3:200}
\lr{ E^{(0)} - H + \Delta} \ket{n}
=
-\ket{n^{(0)}}\bra{n^{(0)}} \lr{ H - H_0 - \Delta } \ket{n}
=
-\ket{n^{(0)}} \lr{
\lr{ E^{(0)} + \Delta} - E^{(0)} -\Delta } \braket{n^{(0)}}{n}
= 0.
\end{dmath}
%
The LHS is also zero as desired, showing that \cref{eqn:gradQuantumProblemSet8Problem3:140} is the desired perturbation relationship.
For the perturbed state we are looking for just the \( \lambda^1 \) terms of \cref{eqn:gradQuantumProblemSet8Problem3:140}, which after dropping all second order and higher terms is
%
\begin{equation}\label{eqn:gradQuantumProblemSet8Problem3:420}
\ket{n_0} + \lambda \ket{n_1} = \ket{n_0} + \sum_{m \ne n} \frac{\ket{m^{(0)}} \bra{m^{(0)}}}{E^{(0)} - E_m} \lr{ \lambda V_1 - \lambda \Delta^{(1)} } \lr{ \ket{n_0} + \lambda \ket{n_1} },
\end{equation}
%
so the first order state perturbation is
%
\begin{equation}\label{eqn:gradQuantumProblemSet8Problem3:440}
\ket{n_1} = \sum_{m \ne n} \frac{\ket{m^{(0)}} \bra{m^{(0)}}}{E^{(0)} - E_m} \lr{ V_1 - \Delta^{(1)} } \ket{n_0}.
\end{equation}
%
The \( \Delta^{(1)} \) contribution drops out, leaving
%
\boxedEquation{eqn:gradQuantumProblemSet8Problem3:460}{
\ket{n_1} = \sum_{m \ne n} \frac{\ket{m^{(0)}} \bra{m^{(0)}}}{E^{(0)} - E_m} V_1 \ket{n_0},
}

just as we had for a strictly first order perturbing potential.

For the energy shifts consider the braket
%
\begin{dmath}\label{eqn:gradQuantumProblemSet8Problem3:220}
\bra{n^{(0)}} H -H_0 \ket{n}
=
\bra{n^{(0)}} { V_1 \Delta^{(1)} + V_2 \Delta^{(2)} } \ket{n}
=
\lr{ \lr{ E^{(0)} + \Delta } -E^{(0)} } \braket{n^{(0)}}{n}
=
\Delta \braket{n^{(0)}}{n},
\end{dmath}
%
or
%As with the a first order \( \lambda \) perturbation, we can impose a requirement that \( \braket{0^{(0)}}{n} = 1 \), so
%
\begin{dmath}\label{eqn:gradQuantumProblemSet8Problem3:240}
\Delta \braket{n^{(0)}}{n} = \bra{n^{(0)}} { V_1 \Delta^{(1)} + V_2 \Delta^{(2)} } \ket{n}.
\end{dmath}
%
Expanding both sides in powers of \( \lambda \) we have
%
\begin{dmath}\label{eqn:gradQuantumProblemSet8Problem3:260}
\sum_{r = 1, s = 0} \lambda^{r+s} \Delta^{(r)} \braket{n^{(0)}}{n_s}
=
\sum_{m = 0} \lambda^{m+1} \bra{n^{(0)}} V_1 \ket{n_m} + \lambda^{m+2} \bra{n^{(0)}} V_2 \ket{n_m}
\end{dmath}

With \( \ket{n^{(0)}} = \ket{n_0} \) as required in the \( \lambda \rightarrow 0 \) limit, the \( \lambda = 1 \) contribution to these sums is
%
%\begin{dmath}\label{eqn:gradQuantumProblemSet8Problem3:280}
\boxedEquation{eqn:gradQuantumProblemSet8Problem3:300}{
\Delta^{(1)} = \bra{n_0} V_1 \ket{n_0}.
}
%\end{dmath}
%
The second order contribution is
%
\begin{dmath}\label{eqn:gradQuantumProblemSet8Problem3:320}
\Delta^{(1)} \braket{n^{(0)}}{n_1} + \Delta^{(2)} \braket{n^{(0)}}{n_0}
=
\bra{n^{(0)}} V_1 \ket{n_1} + \bra{n^{(0)}} V_2 \ket{n_0},
\end{dmath}
%
or
\begin{equation}\label{eqn:gradQuantumProblemSet8Problem3:380}
\Delta^{(2)} = \bra{n_0} V_1 \ket{n_1} + \bra{n_0} V_2 \ket{n_0} - \bra{n_0} V_1 \ket{n_0} \braket{n_0}{n_1}.
\end{equation}
%
%We can write this as
%\begin{equation}\label{eqn:gradQuantumProblemSet8Problem3:360}
%%\boxedEquation{eqn:gradQuantumProblemSet8Problem3:400}{
%\begin{aligned}
%\Delta^{(2)} &= \bra{n_0} V_1 \ket{n_1}_\perp + \bra{n_0} V_2 \ket{n_0} \\
%\ket{n_1}_\perp &= \biglr{ 1 - \ket{n_0}\bra{n_0} } \ket{n_1},
%\end{aligned}
%%}
%\end{equation}
%
%where \( \ket{n_1}_\perp \) is the rejection of \( \ket{n_0} \) from the first order perturbed state \( \ket{n_1} \), the portions of \( \ket{n_1} \) that are orthogonal to \( \ket{n_0} \).
However, from \cref{eqn:gradQuantumProblemSet8Problem3:460} we see that \( \ket{n_1} \) has no \( \ket{n_0} \) component, this means the second order shift is just
%
\boxedEquation{eqn:gradQuantumProblemSet8Problem3:500}{
\Delta^{(2)} = \bra{n_0} V_1 \ket{n_1} + \bra{n_0} V_2 \ket{n_0}.
}

\makeSubAnswer{}{gradQuantum:problemSet8:3b}
%
In \nbref{ps8:2dHarmonicOscillatorOperators.nb}, for
an initial state \( \ket{n_0} = \ket{0,0} \), it is calculated that
%
\begin{dmath}\label{eqn:gradQuantumProblemSet8Problem3:480}
V_1 \ket{0,0}
=
g_1 \frac{\Hbar \omega}{x_0^4} \lr{ x^4 + y^4 } \ket{0,0}
=
g_1 \Hbar \omega
\lr{
\frac{3 }{2} \ket{0,0}
+
\frac{3 }{\sqrt{2}} \lr{ \ket{2,0} + \ket{0,2} }
+
\sqrt{\frac{3}{2}} \lr{ \ket{4,0} + \ket{0,4} }
}.
\end{dmath}
%
The first energy shift is
%
\begin{dmath}\label{eqn:gradQuantumProblemSet8Problem3:520}
\Delta^{(1)} = \bra{0,0} V_1 \ket{0,0} = \frac{3}{2} g_1 \Hbar \omega,
\end{dmath}
%
and the first order perturbation of the state is
%
\begin{dmath}\label{eqn:gradQuantumProblemSet8Problem3:540}
\ket{n_1} = g_1 \Hbar \omega
\lr{
\frac{3}{\sqrt{2}} \frac{\ket{2,0} + \ket{0,2} }{ \Hbar \omega \lr{ 1 + 0 + 0 } - \Hbar \omega \lr{ 1 + 2 + 0} }
+
\sqrt{\frac{3}{2}} \frac{ \ket{4,0} + \ket{0,4} }{ \Hbar \omega \lr{ 1 + 0 + 0 } - \Hbar \omega \lr{ 1 + 4 + 0} }
},
\end{dmath}
%
or
\begin{dmath}\label{eqn:gradQuantumProblemSet8Problem3:560}
\ket{n_1} = -g_1
\lr{
\frac{3}{2 \sqrt{2}} \lr{\ket{2,0} + \ket{0,2} }
+
\inv{4} \sqrt{\frac{3}{2}} \lr{ \ket{4,0} + \ket{0,4} }
},
\end{dmath}
%
We can calculate
\begin{dmath}\label{eqn:gradQuantumProblemSet8Problem3:580}
\begin{aligned}
\bra{0,0} V_2 \ket{0,0} &= 0 \\
\bra{0,0} V_1 \ket{n_1} &= -\frac{21}{4} \Hbar \omega g_1^2,
\end{aligned}
\end{dmath}

so the second order energy shift is
\begin{dmath}\label{eqn:gradQuantumProblemSet8Problem3:600}
\Delta^{(2)} = -\frac{21}{4} \Hbar \omega g_1^2,
\end{dmath}
%
so the ground state energy shift, to second order in \( \lambda \), is
%
%\begin{dmath}\label{eqn:gradQuantumProblemSet8Problem3:620}
\boxedEquation{eqn:gradQuantumProblemSet8Problem3:680}{
\Hbar \omega \rightarrow \Hbar \omega + \frac{3}{2} \Hbar \omega g_1 \lambda - \frac{21}{4} \Hbar \omega g_1^2 \lambda^2.
}
%\end{dmath}
%
For the ground state, there is no contribution from the second order potential \( \lambda^2 V_2 \).

\makeSubAnswer{}{gradQuantum:problemSet8:3c}
%
The next two highest states are \( \ket{1,0}, \ket{0,1} \) both with unperturbed energy eigenvalues \( 2 \Hbar \omega \).  For a basis spanning the \( \setlr{ \ket{1,0}, \ket{0,1} } \) subspace, the matrix element of the perturbed Hamiltonian is
%
\begin{dmath}\label{eqn:gradQuantumProblemSet8Problem3:640}
H_0 + \lambda V_1 + \lambda^2 V_2
=
2 \Hbar \omega I
+ \frac{9}{2} g_1 \lambda \Hbar \omega I
+ \frac{1}{2} g_2 \lambda^2 \Hbar \omega \sigma_1
=
\Hbar \omega
\begin{bmatrix}
2 + \frac{9}{2} g_1 \lambda & \frac{1}{2} g_2 \lambda^2 \\
\frac{1}{2} g_2 \lambda^2 & 2 + \frac{9}{2} g_1 \lambda
\end{bmatrix}.
\end{dmath}
%
Since the eigenvalues of \(
\begin{bmatrix}
a & b \\
b & a
\end{bmatrix} \) are just \( a \pm b \), the energy splitting to first order for these first two degenerate states is
%
%\begin{equation}\label{eqn:gradQuantumProblemSet8Problem3:660}
\boxedEquation{eqn:gradQuantumProblemSet8Problem3:700}{
2 \Hbar \omega \rightarrow \Hbar \omega \lr{ 2 + \frac{9}{2} g_1 \lambda \pm \frac{g_2 \lambda^2 }{2} }.
}
%\end{equation}
%
While the \( x y \) perturbation potential left the ground state untouched, it is responsible for the energy level splitting for the degenerate states \( \ket{1,0}, \ket{1,0} \).
}
}

%
% Copyright � 2015 Peeter Joot.  All Rights Reserved.
% Licenced as described in the file LICENSE under the root directory of this GIT repository.
%
\makeoproblem{Hyperfine levels.}{gradQuantum:problemSet8:4}{2015 ps8 p4}{
\index{hyperfine levels}

We can schematically model the hyperfine interaction between the electron and proton spins as \( A \BS_\txte \cdot \BS_\txtp \) where \( A \) is the hyperfine interaction energy.

\makesubproblem{}{gradQuantum:problemSet8:4a}
Consider the spin-1/2 proton interacting with a spin-1/2 electron.
What are the spin eigenstates and eigenvalues?

\makesubproblem{}{gradQuantum:problemSet8:4b}
Now consider applying a magnetic field which leads to an extra term
%
\begin{dmath}\label{eqn:gradQuantumProblemSet8Problem4:20}
-B \lr{ g_\txte \mu_\txte S_\txte^z + g_\txtp \mu_\txtp S_\txtp^z }
\end{dmath}

with gyromagnetic ratios \( g_\txte \approx -2 \) and \( g_\txtp \approx 5.5 \), with magnetic moments \( \mu_\txte = e/2m_\txte \) and
\( \mu_\txtp = e/2m_\txtp \). The large nuclear mass ensures \( \mu_\txte/\mu_\txtp \sim 2000 \), so let us simply set \( \mu_\txtp = 0\). For convenience, set \( B g_\txte \mu_\txte \rightarrow B_{\textrm{eff}} \) so the Hamiltonian becomes
%
\begin{dmath}\label{eqn:gradQuantumProblemSet8Problem4:40}
H = A \BS_\txte \cdot \BS_\txtp - B_{\textrm{eff}} S_\txte^z,
\end{dmath}
%
so the only dimensionless parameter is \( B_{\textrm{eff}}/A \).

Using perturbation theory (degenerate or non-degenerate as appropriate) find how the coupled hyperfine levels split
for weak field \( B_{\textrm{eff}}/A \ll 1 \).
Also consider the strong field limit \( B_{\textrm{eff}}/A \gg 1 \).

\makesubproblem{}{gradQuantum:problemSet8:4c}
Compute the full field evolution of the levels and compare with the perturbative low field regime result and the high field regime result.

} % makeproblem

\makeanswer{gradQuantum:problemSet8:4}{
\withproblemsetsParagraph{
\makeSubAnswer{}{gradQuantum:problemSet8:4a}
%What are the spin eigenstates and eigenvalues?

With respect to the basis \( \beta = \ket{++}, \ket{-+}, \ket{+-}, \ket{--} \), where \( \ket{\pm, \pm} = \ket{\pm}_\txte \otimes \ket{\pm}_\txtp \) are the direct products of the eigenkets of the \( \BS_\txte \) and \( \BS_\txtp \) operators (not of the respective \( S^z \) operators), the unperturbed interaction Hamiltonian is
%
\begin{dmath}\label{eqn:gradQuantumProblemSet8Problem4:60}
A \BS_\txte \cdot \BS_\txtp
=
A
\begin{bmatrix}
\bra{++} \BS_\txte \cdot \BS_\txtp \ket{++} & \bra{++} \BS_\txte \cdot \BS_\txtp \ket{-+} & \bra{++} \BS_\txte \cdot \BS_\txtp \ket{+-} & \bra{++} \BS_\txte \cdot \BS_\txtp \ket{--} \\
\bra{-+} \BS_\txte \cdot \BS_\txtp \ket{++} & \bra{-+} \BS_\txte \cdot \BS_\txtp \ket{-+} & \bra{-+} \BS_\txte \cdot \BS_\txtp \ket{+-} & \bra{-+} \BS_\txte \cdot \BS_\txtp \ket{--} \\
\bra{+-} \BS_\txte \cdot \BS_\txtp \ket{++} & \bra{+-} \BS_\txte \cdot \BS_\txtp \ket{-+} & \bra{+-} \BS_\txte \cdot \BS_\txtp \ket{+-} & \bra{+-} \BS_\txte \cdot \BS_\txtp \ket{--} \\
\bra{--} \BS_\txte \cdot \BS_\txtp \ket{++} & \bra{--} \BS_\txte \cdot \BS_\txtp \ket{-+} & \bra{--} \BS_\txte \cdot \BS_\txtp \ket{+-} & \bra{--} \BS_\txte \cdot \BS_\txtp \ket{--} \\
\end{bmatrix}
=
\frac{A \Hbar^2}{4}
\begin{bmatrix}
\bra{+} \sigma_\txte \ket{+} \bra{+} \sigma_\txtp \ket{+} & \bra{+} \sigma_\txte \ket{-} \bra{+} \sigma_\txtp \ket{+} & \bra{+} \sigma_\txte \ket{+} \bra{+} \sigma_\txtp \ket{-} & \bra{+} \sigma_\txte \ket{-} \bra{+} \sigma_\txtp \ket{-} \\
\bra{-} \sigma_\txte \ket{+} \bra{+} \sigma_\txtp \ket{+} & \bra{-} \sigma_\txte \ket{-} \bra{+} \sigma_\txtp \ket{+} & \bra{-} \sigma_\txte \ket{+} \bra{+} \sigma_\txtp \ket{-} & \bra{-} \sigma_\txte \ket{-} \bra{+} \sigma_\txtp \ket{-} \\
\bra{+} \sigma_\txte \ket{+} \bra{-} \sigma_\txtp \ket{+} & \bra{+} \sigma_\txte \ket{-} \bra{-} \sigma_\txtp \ket{+} & \bra{+} \sigma_\txte \ket{+} \bra{-} \sigma_\txtp \ket{-} & \bra{+} \sigma_\txte \ket{-} \bra{-} \sigma_\txtp \ket{-} \\
\bra{-} \sigma_\txte \ket{+} \bra{-} \sigma_\txtp \ket{+} & \bra{-} \sigma_\txte \ket{-} \bra{-} \sigma_\txtp \ket{+} & \bra{-} \sigma_\txte \ket{+} \bra{-} \sigma_\txtp \ket{-} & \bra{-} \sigma_\txte \ket{-} \bra{-} \sigma_\txtp \ket{-} \\
\end{bmatrix}
=
\frac{A \Hbar^2}{4}
\begin{bmatrix}
(1) (1) & (0) (1) & (1) (0) & (0) (0) \\
(0) (1) & (-1) (1) & (0) (0) & (-1) (0) \\
(1) (0) & (0) (0) & (1) (-1) & (0) (-1) \\
(0) (0) & (-1) (0) & (0) (-1) & (-1) (-1) \\
\end{bmatrix}
=
\frac{A \Hbar^2}{4}
\begin{bmatrix}
\sigma_3 & 0 \\
0 & -\sigma_3
\end{bmatrix}.
\end{dmath}

The spin eigenstates are the basis elements of \( \beta \) above, with respective eigenvalues
%
\begin{dmath}\label{eqn:gradQuantumProblemSet8Problem4:80}
\setlr{ A \Hbar^2/4, -A \Hbar^2/4, -A \Hbar^2/4, A \Hbar^2/4}
\end{dmath}
%
\makeSubAnswer{}{gradQuantum:problemSet8:4b}

The matrix representation of the perturbation potential is
%
\begin{dmath}\label{eqn:gradQuantumProblemSet8Problem4:100}
-B_{\textrm{eff}} S^z_\txte
=
-\frac{B_{\textrm{eff}} \Hbar}{2}
\begin{bmatrix}
\bra{+} \sigma^z_\txte \ket{+} \braket{+}{+} & \bra{+} \sigma^z_\txte \ket{-} \braket{+}{+} & \bra{+} \sigma^z_\txte \ket{+} \braket{+}{-} & \bra{+} \sigma^z_\txte \ket{-} \braket{+}{-} \\
\bra{-} \sigma^z_\txte \ket{+} \braket{+}{+} & \bra{-} \sigma^z_\txte \ket{-} \braket{+}{+} & \bra{-} \sigma^z_\txte \ket{+} \braket{+}{-} & \bra{-} \sigma^z_\txte \ket{-} \braket{+}{-} \\
\bra{+} \sigma^z_\txte \ket{+} \braket{-}{+} & \bra{+} \sigma^z_\txte \ket{-} \braket{-}{+} & \bra{+} \sigma^z_\txte \ket{+} \braket{-}{-} & \bra{+} \sigma^z_\txte \ket{-} \braket{-}{-} \\
\bra{-} \sigma^z_\txte \ket{+} \braket{-}{+} & \bra{-} \sigma^z_\txte \ket{-} \braket{-}{+} & \bra{-} \sigma^z_\txte \ket{+} \braket{-}{-} & \bra{-} \sigma^z_\txte \ket{-} \braket{-}{-} \\
\end{bmatrix}
=
-\frac{B_{\textrm{eff}} \Hbar}{2}
\begin{bmatrix}
\sigma^z_\txte & 0 \\
0 & \sigma^z_\txte
\end{bmatrix},
\end{dmath}
%
Assuming the \( \BS_\txte \) operator is directed along \( \ncap = (\sin\theta \cos\phi, \sin\theta \sin\phi, \cos\theta) \) with eigenkets
\begin{dmath}\label{eqn:gradQuantumProblemSet8Problem4:120}
\ket{+} =
\begin{bmatrix}
e^{-i\phi} \cos(\theta/2) \\
\sin(\theta/2) \\
\end{bmatrix}
\end{dmath}
\begin{dmath}\label{eqn:gradQuantumProblemSet8Problem4:140}
\ket{-} =
\begin{bmatrix}
-e^{-i\phi} \sin(\theta/2) \\
\cos(\theta/2) \\
\end{bmatrix},
\end{dmath}
%
the representation of the \( \sigma^z_\txte \) operator is
%
\begin{dmath}\label{eqn:gradQuantumProblemSet8Problem4:160}
\sigma^z_\txte
=
\begin{bmatrix}
\cos\theta & -\sin\theta \\
-\sin\theta & -\cos\theta \\
\end{bmatrix}
=
U \PauliZ U^{-1},
\end{dmath}
%
where
\begin{equation}\label{eqn:gradQuantumProblemSet8Problem4:180}
U =
\begin{bmatrix}
-\cos(\theta/2) & \sin(\theta/2) \\
\sin(\theta/2) & \cos(\theta/2)
\end{bmatrix}.
\end{equation}

This is demonstrated in \nbref{ps8:PauliMatrixSpinOperators.nb}.

The full Hamiltonian can now be written in block matrix form
%
\begin{dmath}\label{eqn:gradQuantumProblemSet8Problem4:200}
H
=
\frac{A \Hbar^2}{4}
\begin{bmatrix}
\sigma_z & 0 \\
0 & -\sigma_z
\end{bmatrix}
-\frac{B_{\textrm{eff}} \Hbar}{2}
\begin{bmatrix}
U \sigma_z U^{-1} & 0 \\
0 & U \sigma_z U^{-1}
\end{bmatrix}
\end{dmath}

Transforming the Hamiltonian to the \( S^z_\txte \) basis we have
%
\begin{dmath}\label{eqn:gradQuantumProblemSet8Problem4:220}
H' =
\frac{A \Hbar^2}{4}
\begin{bmatrix}
U^{-1} \sigma_z U & 0 \\
0 & -U^{-1} \sigma_z U
\end{bmatrix}
-\frac{B_{\textrm{eff}} \Hbar}{2}
\begin{bmatrix}
\sigma_z & 0 \\
0 & \sigma_z
\end{bmatrix}
\end{dmath}

%With \( C = \cos(\theta/2), S = \sin(\theta/2) \) these \( U^{-1} \sigma_z U \) block matrices are
%
%\begin{dmath}\label{eqn:gradQuantumProblemSet8Problem4:240}
%U^{-1} \sigma_z U
%=
%\inv{-C^2 - S^2}
%\begin{bmatrix}
%C & -S \\
%-S & -C
%\end{bmatrix}
%\begin{bmatrix}
%1 & 0 \\
%0 & -1
%\end{bmatrix}
%\begin{bmatrix}
%-C & S \\
%S & C
%\end{bmatrix}
%=
%\begin{bmatrix}
%-C & S \\
%S & C
%\end{bmatrix}
%\begin{bmatrix}
%-C & S \\
%-S & -C
%\end{bmatrix}
%=
%\begin{bmatrix}
%C^2 - S^2 & -2 S C \\
%-2 C S & S^2 - C^2
%\end{bmatrix}
%=
%\begin{bmatrix}
%\cos\theta & - \sin\theta \\
%-\sin\theta & -\cos\theta
%\end{bmatrix}.
%\end{dmath}
%
%We don't need this for the zeroth order energy split.

For \( B_{\textrm{eff}} \ll A \), the first order energy splitting can be read off by inspection
%
\begin{equation}\label{eqn:gradQuantumProblemSet8Problem4:260}
\begin{aligned}
\frac{\Hbar^2 A}{4} &\rightarrow \frac{\Hbar^2 A}{4} -\frac{B_{\textrm{eff}} \Hbar}{2} \\
-\frac{\Hbar^2 A}{4} &\rightarrow -\frac{\Hbar^2 A}{4} +\frac{B_{\textrm{eff}} \Hbar}{2} \\
-\frac{\Hbar^2 A}{4} &\rightarrow -\frac{\Hbar^2 A}{4} -\frac{B_{\textrm{eff}} \Hbar}{2} \\
\frac{\Hbar^2 A}{4} &\rightarrow \frac{\Hbar^2 A}{4} +\frac{B_{\textrm{eff}} \Hbar}{2} \\
\end{aligned}
\end{equation}

For the strong field limit, we can flip the problem, and consider \( A \BS_\txte \cdot \BS_\txtp \) to be a perturbation of an initial Hamiltonian \( H_0 = -B_{\textrm{eff}} S^z_\txte \).  The diagonalization of that perturbation is just \cref{eqn:gradQuantumProblemSet8Problem4:200} so the first order energy shifts are
%
\begin{equation}\label{eqn:gradQuantumProblemSet8Problem4:280}
\begin{aligned}
-\frac{\Hbar B_{\textrm{eff}}}{2} &\rightarrow -\frac{B_{\textrm{eff}} \Hbar}{2} + \frac{\Hbar^2 A}{4} \\
\frac{\Hbar B_{\textrm{eff}}}{2} &\rightarrow \frac{B_{\textrm{eff}} \Hbar}{2} - \frac{\Hbar^2 A}{4} \\
-\frac{\Hbar B_{\textrm{eff}}}{2} &\rightarrow -\frac{B_{\textrm{eff}} \Hbar}{2} - \frac{\Hbar^2 A}{4} \\
\frac{\Hbar B_{\textrm{eff}}}{2} &\rightarrow \frac{B_{\textrm{eff}} \Hbar}{2} + \frac{\Hbar^2 A}{4}.
\end{aligned}
\end{equation}

The splitting is the same to first order, but the starting energies are different.

\makeSubAnswer{}{gradQuantum:problemSet8:4a}

The full field solutions (as found in \nbref{ps8:PauliMatrixSpinOperators.nb}) are
%
\begin{equation}\label{eqn:gradQuantumProblemSet8Problem4:300}
\begin{array}{c}
 -\frac{\Hbar}{4} \sqrt{4 B_{\textrm{eff}}^2 - 4 A \Hbar \cos\theta B_{\textrm{eff}} + A^2 \Hbar^2}, \\
  \frac{\Hbar}{4} \sqrt{4 B_{\textrm{eff}}^2 - 4 A \Hbar \cos\theta B_{\textrm{eff}} + A^2 \Hbar^2}, \\
 -\frac{\Hbar}{4} \sqrt{4 B_{\textrm{eff}}^2 + 4 A \Hbar \cos\theta B_{\textrm{eff}} + A^2 \Hbar^2}, \\
  \frac{\Hbar}{4} \sqrt{4 B_{\textrm{eff}}^2 + 4 A \Hbar \cos\theta B_{\textrm{eff}} + A^2 \Hbar^2}.
\end{array}
\end{equation}

For the weak field \( B_{\textrm{eff}} \ll A \) the respective approximations of these energies are
%
\begin{equation}\label{eqn:gradQuantumProblemSet8Problem4:320}
\begin{array}{c}
 -\frac{A \Hbar^2}{4} - \frac{B_{\textrm{eff}} \Hbar \cos\theta}{2}, \\
  \frac{A \Hbar^2}{4} - \frac{B_{\textrm{eff}} \Hbar \cos\theta}{2}, \\
 -\frac{A \Hbar^2}{4} + \frac{B_{\textrm{eff}} \Hbar \cos\theta}{2}, \\
  \frac{A \Hbar^2}{4} + \frac{B_{\textrm{eff}} \Hbar \cos\theta}{2},
\end{array}
\end{equation}

whereas for the strong field \( B_{\textrm{eff}} \gg A \) the respective approximations of these energies are
%
\begin{equation}\label{eqn:gradQuantumProblemSet8Problem4:340}
\begin{array}{c}
 -\frac{\Hbar B_{\textrm{eff}}}{2} - \frac{A \Hbar^2 \cos\theta}{4}, \\
  \frac{\Hbar B_{\textrm{eff}}}{2} - \frac{A \Hbar^2 \cos\theta}{4}, \\
 -\frac{\Hbar B_{\textrm{eff}}}{2} + \frac{A \Hbar^2 \cos\theta}{4}, \\
  \frac{\Hbar B_{\textrm{eff}}}{2} + \frac{A \Hbar^2 \cos\theta}{4}.
\end{array}
\end{equation}

The full field solution has an orientation specific coupling that the first order perturbative solution does not find, so the perturbation is most accurate when the electron spin orientation is close to the z-axis (\( \ncap \cdot \zcap = \cos\theta \approx 1 \)).
}
}


\clearpage
\paragraph{Mathematica Sources}

Mathematica code associated with these notes is available under
\href{https://github.com/peeterjoot/mathematica/tree/master/phy1520-quantum/ps8}{phy1520/ps8/}
within the github repository:

\begin{itemize}
\item git@github.com:peeterjoot/mathematica.git
\end{itemize}

Notebooks created for this problem set

\input{ps8mathematica.tex}

The notebooks referenced in these notes were generated with versions not greater than:

\begin{itemize}
\item commit 891d35b80ba67a96fe5c34619c0018e01b1d8f09
\end{itemize}

%\EndArticle
\EndNoBibArticle
