%
% Copyright � 2015 Peeter Joot.  All Rights Reserved.
% Licenced as described in the file LICENSE under the root directory of this GIT repository.
%

\makeoproblem{Correlation function.}{gradQuantum:problemSet2:3}{phy1520 2015 ps2.3}{
\index{correlation function}
Consider \( \expectation{x(0)x(t)} \) and \( \expectation{p(0)p(t)} \) where operators are in the Heisenberg representation. These are called correlation functions. Evaluate this for the 1D harmonic oscillator in an energy eigenstate \( \ket{n} \).
} % makeproblem
%
%
% ground state version of this in ../phy1520/correlationSHO.tex
%
\makeanswer{gradQuantum:problemSet2:3}{
\withproblemsetsParagraph{

In the Heisenberg representation the position and momentum operators evolve as
%
\begin{equation}\label{eqn:gradQuantumProblemSet2Problem3:20}
\begin{aligned}
x(t) &= x(0) \cos(\omega t) + \frac{p(0)}{m \omega} \sin(\omega t) \\
p(t) &= p(0) \cos(\omega t) - m \omega x(0) \sin(\omega t).
\end{aligned}
\end{equation}
%
To evaluate the expectation operations, we'll want the ladder operator representations of the position and momentum operators
%
\begin{equation}\label{eqn:gradQuantumProblemSet2Problem3:40}
\begin{aligned}
x(0) &= \frac{x_0}{\sqrt{2}} \evalbar{\lr{ a + a^\dagger }}{t = 0} \\
p(0) &= \frac{i \Hbar}{x_0 \sqrt{2}} \evalbar{\lr{ a^\dagger - a}}{t = 0},
\end{aligned}
\end{equation}

where
%
\begin{equation}\label{eqn:gradQuantumProblemSet2Problem3:60}
x_0 = \sqrt{\frac{\Hbar}{m \omega}}.
\end{equation}
%
The expectations of interest, with the raising and lowering operators evaluated at \( t = 0 \), are
%
\begin{dmath}\label{eqn:gradQuantumProblemSet2Problem3:80}
\bra{n} x(0) x(0) \ket{n}
=
\frac{x_0^2}{2} \bra{n} \lr{ a + a^\dagger }^2 \ket{n}
=
\frac{x_0^2}{2}
\lr{ \sqrt{n+1} \bra{n+1} + \sqrt{n} \bra{n-1} }
\lr{ \sqrt{n+1} \ket{n+1} + \sqrt{n} \ket{n-1} }
=
\frac{x_0^2}{2} \lr{ 2 n + 1 },
\end{dmath}
%
and
\begin{dmath}\label{eqn:gradQuantumProblemSet2Problem3:100}
\bra{n} x(0) p(0) \ket{n}
=
\frac{i \Hbar}{2}
\bra{n} \lr{ a + a^\dagger } \lr{ a^\dagger - a }  \ket{n}
=
\frac{i \Hbar}{2}
\lr{ \sqrt{n+1} \bra{n+1} + \sqrt{n} \bra{n-1} }
\lr{ \sqrt{n+1} \ket{n+1} - \sqrt{n} \ket{n-1} }
=
\frac{i \Hbar}{2},
\end{dmath}
%
\begin{dmath}\label{eqn:gradQuantumProblemSet2Problem3:101}
\bra{n} p(0) p(0) \ket{n}
=
\frac{-\Hbar^2 }{ 2 x_0^2}
\bra{n} \lr{ a^\dagger - a } \lr{ a^\dagger - a }  \ket{n}
=
\frac{-\Hbar^2 }{ 2 x_0^2}
\lr{ \sqrt{n+1} \bra{n+1} - \sqrt{n} \bra{n-1} }
\lr{ \sqrt{n+1} \ket{n+1} - \sqrt{n} \ket{n-1} }
=
\frac{(-\Hbar^2) }{ 2 x_0^2}
\lr{
2 n + 1
},
\end{dmath}
%
and finally
%
\begin{dmath}\label{eqn:gradQuantumProblemSet2Problem3:120}
\bra{n} p(0) x(0) \ket{n}
=
\bra{n}
\lr{
\antisymmetric{p(0)}{x(0)} + x(0) p(0)
}
\ket{n}
=
-i \Hbar + \frac{i\Hbar}{2}
=
\frac{i \Hbar}{2}.
\end{dmath}
%
Now we are ready to compute the correlations.  The position correlation is
%
\begin{dmath}\label{eqn:gradQuantumProblemSet2Problem3:140}
\bra{n} x(0) x(t) \ket{n}
=
\bra{n} x(0) \lr{
x(0) \cos(\omega t) + \frac{p(0)}{m \omega} \sin(\omega t)
} \ket{n}
=
\cos(\omega t) \bra{n} x(0) x(0) \ket{n}
+
\frac{1}{m \omega} \sin(\omega t) \bra{n} x(0) p(0) \ket{n}
=
\cos(\omega t)
\frac{x_0^2}{2} \lr{ 2 n + 1 }
+ \frac{1}{m \omega} \sin(\omega t) \frac{i \Hbar}{2},
\end{dmath}
%
which is
%
%\begin{equation}\label{eqn:gradQuantumProblemSet2Problem3:160}
\boxedEquation{eqn:gradQuantumProblemSet2Problem3:180}{
\bra{n} x(0) x(t) \ket{n}
=
\frac{x_0^2}{2}
\lr{
\lr{ 2 n + 1 }
\cos(\omega t)
+ i \sin(\omega t)
}.
}
%\end{equation}
%
The momentum correlation is
%
\begin{dmath}\label{eqn:gradQuantumProblemSet2Problem3:200}
\bra{n} p(0) p(t) \ket{n}
=
\bra{n} p(0) \lr{
p(0) \cos(\omega t) - m \omega x(0) \sin(\omega t)
} \ket{n}
=
\cos(\omega t)
\lr{ 2 n + 1 }
\frac{(-\Hbar^2) }{ 2 x_0^2}
- m \omega \sin(\omega t) \frac{i \Hbar}{2}.
\end{dmath}
%
With \( p_0^2 = m \omega \Hbar \), this is
%
\boxedEquation{eqn:gradQuantumProblemSet2Problem3:n}{
\bra{n} p(0) p(t) \ket{n}
=
-\frac{p_0^2}{2} \lr{
\lr{ 2 n + 1 }
\cos(\omega t)
+ i \sin(\omega t)
}.
}

}
}
