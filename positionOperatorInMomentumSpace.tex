%
% Copyright � 2015 Peeter Joot.  All Rights Reserved.
% Licenced as described in the file LICENSE under the root directory of this GIT repository.
%
%\input{../blogpost.tex}
%\renewcommand{\basename}{positionOperatorInMomentumSpace}
%\renewcommand{\dirname}{notes/phy1520/}
%%\newcommand{\dateintitle}{}
%%\newcommand{\keywords}{}
%
%\input{../peeter_prologue_print2.tex}
%
%\usepackage{peeters_layout_exercise}
%\usepackage{peeters_braket}
%\usepackage{peeters_figures}
%
%\beginArtNoToc
%
%\generatetitle{Position operator in momentum space representation}
%%\label{chap:positionOperatorInMomentumSpace}

A derivation of the position space representation of the momentum operator \( -i \Hbar \partial_x \) is made in \citep{desai2009quantum}, starting with the position-momentum commutator.  Here I'll repeat that argument for the momentum space representation of the position operator.

What we want to do is expand the matrix element of the commutator.  First using the definition of the commutator
%
\begin{dmath}\label{eqn:positionOperatorInMomentumSpace:20}
\bra{p'} X P - P X \ket{p''}
=
i \Hbar \braket{p'}{p''}
=
i \Hbar \delta{p' - p''},
\end{dmath}
%
and then by inserting an identity operation in a momentum space basis
%
\begin{dmath}\label{eqn:positionOperatorInMomentumSpace:40}
\bra{p'} X P - P X \ket{p''}
=
\int dp
\bra{p'} X \ket{p}\bra{p} P \ket{p''}
-\int dp
\bra{p'} P \ket{p}\bra{p} X \ket{p''}
=
\int dp
\bra{p'} X \ket{p} p \delta(p - p'')
-\int dp
p  \delta(p' - p)
\bra{p} X \ket{p''}
=
\bra{p'} X \ket{p''} p''
-
p' \bra{p'} X \ket{p''}.
\end{dmath}
%
So we have
%
\begin{dmath}\label{eqn:positionOperatorInMomentumSpace:60}
\bra{p'} X \ket{p''} p''
-
p' \bra{p'} X \ket{p''}
=
i \Hbar \delta{p' - p''}.
\end{dmath}
%
Because the RHS is zero whenever \( p' \ne p'' \), the matrix element \( \bra{p'} X \ket{p''} \) must also include a delta function.  Let
%
\begin{equation}\label{eqn:positionOperatorInMomentumSpace:80}
\bra{p'} X \ket{p''} = \delta(p' - p'') X(p'').
\end{equation}
%
Because \cref{eqn:positionOperatorInMomentumSpace:60} is an operator equation that really only takes on meaning when applied to a wave function and integrated, we do that
%
\begin{dmath}\label{eqn:positionOperatorInMomentumSpace:100}
\int dp'' \delta(p' - p'') X(p'') p'' \psi(p'')
-
\int dp'' p' \delta(p' - p'') X(p'') \psi(p'')
=
\int dp'' i \Hbar \delta{p' - p''} \psi(p''),
\end{dmath}
%
or
\begin{dmath}\label{eqn:positionOperatorInMomentumSpace:120}
i \Hbar \psi(p')
=
X(p') p' \psi(p')
-
p'
X(p') \psi(p').
\end{dmath}
%
Provided \( X(p') \) operates on everything to its right, this equation is solved by setting
%
%\begin{dmath}\label{eqn:positionOperatorInMomentumSpace:140}
\boxedEquation{eqn:positionOperatorInMomentumSpace:140}{
X(p') = i \Hbar \PD{p'}{}.
}
%\end{dmath}
%
%\EndArticle
