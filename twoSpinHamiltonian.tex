%
% Copyright � 2015 Peeter Joot.  All Rights Reserved.
% Licenced as described in the file LICENSE under the root directory of this GIT repository.
%
%\input{../blogpost.tex}
%\renewcommand{\basename}{twoSpinHamiltonian}
%\renewcommand{\dirname}{notes/phy1520/}
%%\newcommand{\dateintitle}{}
%%\newcommand{\keywords}{}
%
%\input{../peeter_prologue_print2.tex}
%
%\usepackage{peeters_layout_exercise}
%\usepackage{peeters_braket}
%\usepackage{peeters_figures}
%
%\beginArtNoToc
%
%\generatetitle{Two spin time evolution}
%%\chapter{Two spin time evolution}
%%\label{chap:twoSpinHamiltonian}
%\section{Motivation}

%Our midterm posed a (low mark ``quick question'') that I didn't complete (or at least not properly).  This shouldn't have been a difficult question, but I spend way too much time on it, costing me time that I needed for other questions.
%
%It turns out that there isn't anything fancy required for this question, just perseverance and careful work.
%
%\section{Guts}
%
\makeoproblem{Two spin time evolution.}{problem:twoSpinHamiltonian:1}{midterm pr. 1.iii}{
Compute the time evolution of a two particle state
%
\begin{equation}\label{eqn:twoSpinHamiltonian:20}
\psi = \inv{\sqrt{2}} \lr{ \ket{\uparrow \downarrow} - \ket{\downarrow \uparrow} },
\end{equation}
under the action of the Hamiltonian
%
\begin{equation}\label{eqn:twoSpinHamiltonian:40}
H = - B S_{z,1} + 2 B S_{x,2} = \frac{\Hbar B}{2}\lr{  -\sigma_{z,1} + 2 \sigma_{x,2} } .
\end{equation}
%
} % problem
%
\makeanswer{problem:twoSpinHamiltonian:1}{
We have to know the action of the Hamiltonian on all the states
%
\begin{equation}\label{eqn:twoSpinHamiltonian:60}
\begin{aligned}
H \ket{\uparrow \uparrow} &= \frac{B \Hbar}{2} \lr{ -\ket{\uparrow \uparrow} + 2 \ket{\uparrow \downarrow} } \\
H \ket{\uparrow \downarrow} &= \frac{B \Hbar}{2} \lr{ -\ket{\uparrow \downarrow} + 2 \ket{\uparrow \uparrow} } \\
H \ket{\downarrow \uparrow} &= \frac{B \Hbar}{2} \lr{ \ket{\downarrow \uparrow} + 2 \ket{\downarrow \downarrow} } \\
H \ket{\downarrow \downarrow} &= \frac{B \Hbar}{2} \lr{ \ket{\downarrow \downarrow} + 2 \ket{\downarrow \uparrow} }.
\end{aligned}
\end{equation}
With respect to the basis \( \setlr{ \ket{\uparrow \uparrow}, \ket{\uparrow \downarrow}, \ket{\downarrow \uparrow}, \ket{\downarrow \downarrow} } \), the matrix of the Hamiltonian is
%
\begin{dmath}\label{eqn:twoSpinHamiltonian:80}
H =
% upup updown downup downdown
\frac{ \Hbar B }{2}
\begin{bmatrix}
-1 &  2 & 0 & 0 \\
 2 & -1 & 0 & 0 \\
 0 &  0 & 1 & 2 \\
 0 &  0 & 2 & 1 \\
\end{bmatrix}.
\end{dmath}
Utilizing the block diagonal form (and ignoring the \( \Hbar B/2 \) factor for now), the characteristic equation is
%
\begin{dmath}\label{eqn:twoSpinHamiltonian:100}
0
=
\begin{vmatrix}
-1 -\lambda &  2 \\
 2 & -1 - \lambda
\end{vmatrix}
\begin{vmatrix}
1 -\lambda &  2 \\
 2 & 1 - \lambda
\end{vmatrix}
=
\lr{(1 + \lambda)^2 - 4}
\lr{(1 - \lambda)^2 - 4}.
\end{dmath}
%
This has solutions
%
\begin{equation}\label{eqn:twoSpinHamiltonian:120}
1 \pm \lambda = \pm 2,
\end{equation}
%
or, with the \( \Hbar B/2 \) factors put back in
%
\begin{equation}\label{eqn:twoSpinHamiltonian:140}
\lambda = \pm \Hbar B/2 , \pm 3 \Hbar B/2.
\end{equation}
%
I was thinking that we needed to compute the time evolution operator
%
\begin{equation}\label{eqn:twoSpinHamiltonian:160}
U = e^{-i H t/\Hbar},
\end{equation}
%
but we actually only need the eigenvectors, and the inverse relations.  We can find the eigenvectors by inspection in each case from
%
\begin{dmath}\label{eqn:twoSpinHamiltonian:180}
\begin{aligned}
H - (1) \frac{ \Hbar B }{2}
&=
\frac{ \Hbar B }{2}
\begin{bmatrix}
-2 &  2 & 0 & 0 \\
 2 & -2 & 0 & 0 \\
 0 &  0 & 0 & 2 \\
 0 &  0 & 2 & 0 \\
\end{bmatrix} \\
H - (-1) \frac{ \Hbar B }{2}
&=
\frac{ \Hbar B }{2}
\begin{bmatrix}
 0 &  2 & 0 & 0 \\
 2 &  0 & 0 & 0 \\
 0 &  0 & 2 & 2 \\
 0 &  0 & 2 & 2 \\
\end{bmatrix} \\
H - (3) \frac{ \Hbar B }{2}
&=
\frac{ \Hbar B }{2}
\begin{bmatrix}
-4 &  2 & 0 & 0 \\
 2 & -4 & 0 & 0 \\
 0 &  0 &-2 & 2 \\
 0 &  0 & 2 &-2 \\
\end{bmatrix} \\
H - (-3) \frac{ \Hbar B }{2}
&=
\frac{ \Hbar B }{2}
\begin{bmatrix}
 2 &  2 & 0 & 0 \\
 2 &  2 & 0 & 0 \\
 0 &  0 & 4 & 2 \\
 0 &  0 & 2 & 1 \\
\end{bmatrix}.
\end{aligned}
\end{dmath}
%
The eigenkets are
%
\begin{equation}\label{eqn:twoSpinHamiltonian:280}
\begin{aligned}
\ket{1} &=
\inv{\sqrt{2}}
\begin{bmatrix}
1 \\
1 \\
0 \\
0 \\
\end{bmatrix},\qquad
\ket{-1} =
\inv{\sqrt{2}}
\begin{bmatrix}
0 \\
0 \\
1 \\
-1 \\
\end{bmatrix}, \\
\ket{3} &=
\inv{\sqrt{2}}
\begin{bmatrix}
0 \\
0 \\
1 \\
1 \\
\end{bmatrix},\qquad
\ket{-3} =
\inv{\sqrt{2}}
\begin{bmatrix}
1 \\
-1 \\
0 \\
0 \\
\end{bmatrix},
\end{aligned}
\end{equation}
or
%
\begin{equation}\label{eqn:twoSpinHamiltonian:300}
\begin{aligned}
\sqrt{2} \ket{1} &= \ket{\uparrow \uparrow} + \ket{\uparrow \downarrow} \\
\sqrt{2} \ket{-1} &= \ket{\downarrow \uparrow} - \ket{\downarrow \downarrow} \\
\sqrt{2} \ket{3} &= \ket{\downarrow \uparrow} + \ket{\downarrow \downarrow} \\
\sqrt{2} \ket{-3} &= \ket{\uparrow \uparrow} - \ket{\uparrow \downarrow}.
\end{aligned}
\end{equation}
%
We can invert these
%
\begin{dmath}\label{eqn:twoSpinHamiltonian:220}
\begin{aligned}
\ket{\uparrow \uparrow} &= \inv{\sqrt{2}} \lr{ \ket{1} + \ket{-3} } \\
\ket{\uparrow \downarrow} &= \inv{\sqrt{2}} \lr{ \ket{1} - \ket{-3} } \\
\ket{\downarrow \uparrow} &= \inv{\sqrt{2}} \lr{ \ket{3} + \ket{-1} } \\
\ket{\downarrow \downarrow} &= \inv{\sqrt{2}} \lr{ \ket{3} - \ket{-1} }.
\end{aligned}
\end{dmath}
The original state of interest can now be expressed in terms of the eigenkets
%
\begin{dmath}\label{eqn:twoSpinHamiltonian:240}
\psi
=
\inv{2} \lr{
\ket{1} - \ket{-3} -
\ket{3} - \ket{-1}
}.
\end{dmath}
The time evolution of this ket is
%
\begin{equation}\label{eqn:twoSpinHamiltonian:260}
\begin{aligned}
\psi(t)
&=
\inv{2}
\lr{
e^{-i B t/2} \ket{1}
- e^{3 i B t/2} \ket{-3}
- e^{-3 i B t/2} \ket{3}
- e^{i B t/2} \ket{-1}
} \\
&=
\inv{2 \sqrt{2}}
\Biglr{
e^{-i B t/2} \lr{ \ket{\uparrow \uparrow} + \ket{\uparrow \downarrow} }
- e^{3 i B t/2} \lr{ \ket{\uparrow \uparrow} - \ket{\uparrow \downarrow} } \\
&\qquad - e^{-3 i B t/2} \lr{ \ket{\downarrow \uparrow} + \ket{\downarrow \downarrow} }
- e^{i B t/2} \lr{ \ket{\downarrow \uparrow} - \ket{\downarrow \downarrow} }
} \\
&=
\inv{2 \sqrt{2}}
\Biglr{
  \lr{ e^{-i B t/2} - e^{3 i B t/2} } \ket{\uparrow \uparrow}
+ \lr{ e^{-i B t/2} + e^{3 i B t/2} } \ket{\uparrow \downarrow}  \\
&\qquad - \lr{ e^{-3 i B t/2} + e^{i B t/2} } \ket{\downarrow \uparrow}
+ \lr{ e^{i B t/2} - e^{-3 i B t/2} } \ket{\downarrow \downarrow}
} \\
&=
\inv{2 \sqrt{2}}
\Biglr{
  e^{i B t/2} \lr{ e^{-2 i B t/2} - e^{2 i B t/2} } \ket{\uparrow \uparrow}
+ e^{i B t/2}  \lr{ e^{-2 i B t/2} + e^{2 i B t/2} } \ket{\uparrow \downarrow}  \\
&\qquad - e^{- i B t/2} \lr{ e^{-2 i B t/2} + e^{2 i B t/2} } \ket{\downarrow \uparrow}
+ e^{- i B t/2} \lr{ e^{2 i B t/2} - e^{-2 i B t/2} } \ket{\downarrow \downarrow}
} \\
&=
\inv{\sqrt{2}}
\biglr{
i \sin( B t )
\lr{
 e^{- i B t/2} \ket{\downarrow \downarrow} - e^{i B t/2} \ket{\uparrow \uparrow}
} \\
&\qquad
+ \cos( B t ) \lr{
e^{i B t/2} \ket{\uparrow \downarrow}
- e^{- i B t/2} \ket{\downarrow \uparrow}
}
}.
\end{aligned}
\end{equation}
Note that this returns to the original state when \( t = \frac{2 \pi n}{B}, n \in \bbZ \).  I think I've got it right this time (although I got a slightly different answer on paper before typing it up.)

%This doesn't exactly seem like a quick answer question, at least to me.  Is there some easier way to do it?
} % answer

%\EndNoBibArticle
