%
% Copyright � 2015 Peeter Joot.  All Rights Reserved.
% Licenced as described in the file LICENSE under the root directory of this GIT repository.
%
%\input{../blogpost.tex}
%\renewcommand{\basename}{quadraticZeeman}
%\renewcommand{\dirname}{notes/phy1520/}
%%\newcommand{\dateintitle}{}
%%\newcommand{\keywords}{}
%
%\input{../peeter_prologue_print2.tex}
%
%\usepackage{peeters_layout_exercise}
%\usepackage{peeters_braket}
%\usepackage{peeters_figures}
%
%\beginArtNoToc
%
%\generatetitle{Quadratic Zeeman effect}
%%\chapter{Quadradic Zeeman effect}
%%\label{chap:quadraticZeeman}
%
\makeoproblem{Quadratic Zeeman effect.}{problem:quadraticZeeman:1}{\citep{sakurai2014modern} pr. 5.18}{
\index{Zeeman effect!quadratic}

Work out the quadratic Zeeman effect for the ground state hydrogen atom due to the usually neglected \( e^2 \BA^2/2 m_e c^2 \) term in the Hamiltonian.
%
} % problem
%
\makeanswer{problem:quadraticZeeman:1}{
%
%The first order energy shift 
For a z-oriented magnetic field we can use
%
\begin{equation}\label{eqn:quadraticZeeman:20}
\BA = \frac{B}{2} \lr{ -y, x, 0 },
\end{equation}
%
so the perturbation potential is
%
\begin{equation}\label{eqn:quadraticZeeman:40}
\begin{aligned}
V
&= \frac{e^2 \BA^2}{2 m_e c^2}
\\ &= \frac{e^2 \BB^2 (x^2 + y^2)}{8 m_e c^2}
\\ &= \frac{ e^2 \BB^2 r^2 \sin^2\theta }{8 m_e c^2}.
\end{aligned}
\end{equation}
The ground state wave function is
%
\begin{equation}\label{eqn:quadraticZeeman:60}
\begin{aligned}
\psi_0
&= \braket{\Bx}{0}
\\ &= \inv{\sqrt{\pi a_0^3}} e^{-r/a_0},
\end{aligned}
\end{equation}
%
so the energy shift is
%
\begin{equation}\label{eqn:quadraticZeeman:80}
\begin{aligned}
\Delta
&= \bra{0} V \ket{0}
\\ &= \inv{ \pi a_0^3 } 2 \pi \frac{ e^2 \BB^2 }{8 m_e c^2} \int_0^\infty r^2 \sin\theta e^{-2r/a_0} r^2 \sin^2\theta dr d\theta
\\ &=
\frac{ e^2 \BB^2 }{4 a_0^3 m_e c^2}
\int_0^\infty r^4 e^{-2r/a_0} dr \int_0^\pi \sin^3\theta d\theta
\\ &= -
\frac{ e^2 \BB^2 }{4 a_0^3 m_e c^2}
\frac{4!}{(2/a_0)^{4+1} } \evalrange{\lr{u - \frac{u^3}{3}}}{1}{-1}
\\ &=
\frac{ e^2 a_0^2 \BB^2 }{4 m_e c^2}.
\end{aligned}
\end{equation}
%
If this energy shift is written in terms of a diamagnetic susceptibility \( \chi \) defined by
%
\begin{equation}\label{eqn:quadraticZeeman:100}
\Delta = -\inv{2} \chi \BB^2,
\end{equation}
%
the diamagnetic susceptibility is
%
\begin{equation}\label{eqn:quadraticZeeman:120}
\chi = -\frac{ e^2 a_0^2 }{2 m_e c^2}.
\end{equation}
} % answer

%\EndArticle
