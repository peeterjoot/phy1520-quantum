%
% Copyright � 2015 Peeter Joot.  All Rights Reserved.
% Licenced as described in the file LICENSE under the root directory of this GIT repository.
%
\paragraph{Trig}
%
\begin{dmath}\label{eqn:usefulFormulas:20}
1 + \cos x = 2 \cos^2 \frac{x}{2}
\end{dmath}
\begin{dmath}\label{eqn:usefulFormulas:40}
1 - \cos x = 2 \sin^2 \frac{x}{2}
\end{dmath}
\begin{dmath}\label{eqn:usefulFormulas:60}
\sin x = 2 \sin \frac{x}{2} \cos \frac{x}{2}
\end{dmath}
%
\begin{dmath}\label{eqn:usefulFormulas:560}
2 \cos a \cos b = \cos(a + b) + \cos(a-b)
\end{dmath}
\begin{dmath}\label{eqn:usefulFormulas:580}
2 \sin a \sin b = \cos(a - b) - \cos(a-b)
\end{dmath}
\begin{dmath}\label{eqn:usefulFormulas:561}
2 \cos a \sin b = \sin(a + b) - \sin(a-b)
\end{dmath}
\begin{dmath}\label{eqn:usefulFormulas:581}
2 \sin a \cos b = \sin(a - b) + \sin(a+b)
\end{dmath}

%
\begin{dmath}\label{eqn:usefulFormulas:600}
\cos(a \pm b) = \cos a\cos b \mp \sin a \sin b
\end{dmath}
\begin{dmath}\label{eqn:usefulFormulas:620}
\sin(a \pm b) = \sin a\cos b \pm \cos a \sin b
\end{dmath}

\paragraph{Basics}
%
\begin{dmath}\label{eqn:usefulFormulas:640}
\braket{x}{p} \propto e^{i p x/\Hbar}
\end{dmath}
%
\begin{dmath}\label{eqn:usefulFormulas:660}
\ket{\Psi(t)} = U \ket{\Psi(0)}
\end{dmath}
%
\begin{dmath}\label{eqn:usefulFormulas:680}
U = e^{-i H t/\Hbar}, \qquad \mbox{for time independent \( H \)}
\end{dmath}
%
\begin{dmath}\label{eqn:usefulFormulas:700}
p = -i \Hbar \PD{x}{}
\end{dmath}
%
\begin{dmath}\label{eqn:usefulFormulas:720}
x = i \Hbar \PD{p}{}
\end{dmath}
%
\begin{dmath}\label{eqn:usefulFormulas:721}
H = i \Hbar \PD{t}{}
\end{dmath}
%
\begin{dmath}\label{eqn:usefulFormulas:740}
A \sim ( \bra{a_{\textrm{row}}} A \ket{a_{\textrm{column}}} )
\end{dmath}
%
\begin{equation}\label{eqn:usefulFormulas:540}
\begin{aligned}
\rho &= \psi^\conj \psi \\
\Bj &= \frac{\Hbar}{m} \Imag\lr{ \psi^\conj \spacegrad \psi} = -\frac{i \Hbar}{2 m} \lr{ \psi^\conj \spacegrad \psi - \psi \spacegrad \psi^\conj }
\end{aligned}
\end{equation}
%
\begin{dmath}\label{eqn:usefulFormulas:780}
\PD{t}{\rho} + \spacegrad \cdot \Bj = 0
\end{dmath}

\paragraph{Commutators}
%
\begin{dmath}\label{eqn:usefulFormulas:800}
\antisymmetric{x}{p} = i \Hbar
\end{dmath}
%
\begin{equation}\label{eqn:usefulFormulas:460}
\begin{aligned}
\antisymmetric{x_j}{F(\Bp)} &= i \Hbar \PD{p_j}{F} \\
\antisymmetric{p_j}{G(\Bx)} &= -i \Hbar \PD{x_j}{G}
\end{aligned}
\end{equation}
%
\begin{equation}\label{eqn:usefulFormulas:860}
e^{A \mu} B e^{-A \mu} = B + \mu \antisymmetric{A}{B} + \frac{\mu^2}{2!} \antisymmetric{A}{ \antisymmetric{A}{B} } + \cdots
\end{equation}
%
\begin{equation}\label{eqn:usefulFormulas:1100}
\antisymmetric{A}{B}_{\textrm{classical}} =
\PD{q}{A} \PD{p}{B} - \PD{p}{A} \PD{q}{B}
\end{equation}
\index{commutator!classical}
\index{Poisson bracket}

\paragraph{Heisenberg picture}
%
\begin{dmath}\label{eqn:usefulFormulas:880}
A_{\textrm{H}} = U^\dagger A U
\end{dmath}
%
\begin{dmath}\label{eqn:usefulFormulas:900}
\frac{d A_{\textrm{H}}}{dt } = \inv{i \Hbar} \antisymmetric{ A_{\textrm{H}} }{H }
\end{dmath}
%
\begin{dmath}\label{eqn:usefulFormulas:1460}
\ddt{} \expectation{ \Bx \cdot \Bp } = \expectation{ \frac{\Bp^2 }{m} } - \expectation{ \Bx \cdot \spacegrad V }
\end{dmath}

\paragraph{Density operator}
%
\begin{equation}\label{eqn:usefulFormulas:760}
\rho = \sum_i w_i \ket{\alpha^i}\bra{\alpha^i}, \qquad \sum_i w_i = 1
\end{equation}
%
\begin{equation}\label{eqn:usefulFormulas:1540}
[A] = \sum_i w_i \bra{\alpha^i} A \ket{\alpha^i} = \tr( \rho A )
\end{equation}
%
%\begin{dmath}\label{eqn:usefulFormulas:920}
%\rho = \ket{\Psi}\bra{\Psi}
%\end{dmath}
%
%\begin{dmath}\label{eqn:usefulFormulas:940}
%\bra{\phi} A \ket{\phi} = \tr( \rho A )
%\end{dmath}
%
\begin{equation}\label{eqn:usefulFormulas:960}
S = - \tr( \rho \ln \rho ) = - \tr( \rho_{kk} \ln \rho_{kk} )
\end{equation}
%
\begin{equation}\label{eqn:usefulFormulas:1320}
\begin{aligned}
\ket{\psi_{mn}} &= \ket{a_m}_1 \otimes \ket{a_n} \\
\ket{\psi} &= \sum_{m n} c_{mn} \ket{\psi_{mn}} \\
\rho &= \ket{\psi} \bra{\psi} \\
\tr_2(\rho) &= \rho_{2} = \sum_a \prescript{}{2}{\bra{a}} \rho \ket{a}_2 \\
\sum_a \prescript{}{1}{\bra{a}} \tr_2(\rho) \ket{a}_1 &= \sum_n \Abs{c_{mn}}^2
\end{aligned}
\end{equation}

\paragraph{Pauli matrices}
%
\begin{dmath}\label{eqn:usefulFormulas:80}
\begin{aligned}
\sigma_x &= \PauliX \\
\sigma_y &= \PauliY \\
\sigma_z &= \PauliZ
\end{aligned}
\end{dmath}
%
\begin{equation}\label{eqn:usefulFormulas:1120}
\antisymmetric{\sigma_a}{\sigma_b} = 2 i \epsilon_{a b c} \sigma_c
\end{equation}

\begin{subequations}
\label{eqn:usefulFormulas:100}
\begin{dmath}\label{eqn:usefulFormulas:120}
\ket{S_x ; \pm } =
\inv{\sqrt{2}}
\begin{bmatrix}
1 \\
\pm 1
\end{bmatrix}
\end{dmath}
\begin{dmath}\label{eqn:usefulFormulas:140}
\ket{S_y ; \pm } =
\inv{\sqrt{2}}
\begin{bmatrix}
1 \\
\pm i
\end{bmatrix}
\end{dmath}
\begin{dmath}\label{eqn:usefulFormulas:160}
\ket{S_z ; \pm } =
\begin{bmatrix}
1 \\
0
\end{bmatrix},
\begin{bmatrix}
0 \\
1 \\
\end{bmatrix}
\end{dmath}
\end{subequations}

For \( \ncap = \lr{\sin\theta \cos\phi, \sin\theta\sin\phi, \cos\theta} \), the eigenkets of \( \BS \cdot \ncap \) are
%
\begin{equation}\label{eqn:usefulFormulas:180}
\ket{\ncap ; +}
=
\begin{bmatrix}
\cos\lr{\theta/2} e^{-i\phi/2} \\
\sin\lr{\theta/2} e^{i\phi/2} \\
\end{bmatrix}
= e^{-i\sigma_z \phi/2} e^{-i\sigma_y \theta/2} \ket{\zcap ; +}
\end{equation}
%
\begin{equation}\label{eqn:usefulFormulas:200}
\ket{\ncap ; -}
=
\begin{bmatrix}
-\sin\lr{\theta/2} e^{-i\phi/2} \\
\cos\lr{\theta/2} e^{i\phi/2} \\
\end{bmatrix}
= e^{-i\sigma_z \phi/2} e^{-i\sigma_y (\theta+\pi)/2} \ket{\zcap ; +}
\end{equation}
%
\begin{equation}\label{eqn:usefulFormulas:980}
H = -\frac{e}{m c} \BS \cdot \BB = -\frac{e B}{m c} S_z
\end{equation}

\paragraph{Harmonic oscillator}
%
\begin{dmath}\label{eqn:usefulFormulas:220}
x_0^2 = \frac{\Hbar}{m \omega}
\end{dmath}
\begin{dmath}\label{eqn:usefulFormulas:1560}
p_0^2 = m \omega \Hbar
\end{dmath}
\begin{dmath}\label{eqn:usefulFormulas:240}
x(t) = \frac{x_0}{\sqrt{2}} \lr{ a e^{-i\omega t} + a^\dagger e^{i \omega t} }
\end{dmath}
\begin{dmath}\label{eqn:usefulFormulas:260}
p(t) = \frac{i \Hbar}{\sqrt{2} x_0} \lr{
a^\dagger e^{i \omega t}
-
a e^{-i\omega t}
}
\end{dmath}
\begin{equation}\label{eqn:usefulFormulas:1640}
a,a^\dagger = \inv{\sqrt{2} x_0} \lr{ x \mp x_0^2 \frac{d}{dx} }
\end{equation}
%
\begin{dmath}\label{eqn:usefulFormulas:280}
a \ket{n} = \sqrt{n} \ket{n-1}
\end{dmath}
\begin{dmath}\label{eqn:usefulFormulas:300}
a^\dagger \ket{n} = \sqrt{n+1} \ket{n+1}
\end{dmath}
\begin{dmath}\label{eqn:usefulFormulas:320}
x(t) = x(0) \cos \omega t + \frac{p(0)}{m \omega} \sin \omega t
\end{dmath}
\begin{dmath}\label{eqn:usefulFormulas:340}
p(t) = p(0) \cos \omega t - m \omega x(0) \sin \omega t
\end{dmath}
\begin{dmath}\label{eqn:usefulFormulas:360}
x(t)^2 = \frac{\Hbar \omega}{2} \lr{ a e^{-i\omega t} + a^\dagger e^{i \omega t} }^2
\end{dmath}
\begin{dmath}\label{eqn:usefulFormulas:380}
p(t)^2 = \frac{\Hbar \omega}{2}
\lr{ a e^{-i\omega t} - a^\dagger e^{i \omega t} }
\lr{ a^\dagger e^{i \omega t} -a e^{-i\omega t} }
\end{dmath}
%
\begin{equation}\label{eqn:usefulFormulas:1000}
N = a^\dagger a
\end{equation}
\begin{equation}\label{eqn:usefulFormulas:1020}
N \ket{n} = n \ket{n}
\end{equation}
\begin{equation}\label{eqn:usefulFormulas:1040}
\antisymmetric{N}{a} = -a
\end{equation}
\begin{equation}\label{eqn:usefulFormulas:1060}
\antisymmetric{N}{a^\dagger} = a^\dagger
\end{equation}
%
\begin{dmath}\label{eqn:usefulFormulas:400}
H = \Hbar \omega \lr{ N + \inv{2} }
\end{dmath}
%
\begin{dmath}\label{eqn:usefulFormulas:420}
\antisymmetric{a}{a^\dagger} = 1
\end{dmath}
%
\begin{dmath}\label{eqn:usefulFormulas:440}
a = \inv{x_0 \sqrt{2}} \lr{ x + \frac{ i p }{m\omega} }
\end{dmath}
%
\begin{equation}\label{eqn:usefulFormulas:1080}
\braket{x}{n} = \inv{ \pi^{1/4} \sqrt{ 2^n n! } } x_0^{-(n+1/2)} \lr{ x - x_0^2 \frac{d}{dx} }^n e^{-(x/x_0)^2/2}
\end{equation}
%
\begin{equation}\label{eqn:usefulFormulas:1660}
\ket{n} = \frac{\lr{a^\dagger}^n}{\sqrt{n!}} \ket{0}
\end{equation}

\paragraph{Coherent states}
%
\begin{equation}\label{eqn:usefulFormulas:1140}
a \ket{z} = z \ket{z}
\end{equation}
%
\begin{equation}\label{eqn:usefulFormulas:1160}
\ket{z} = e^{-\Abs{z}^2/2 + z a^\dagger} \ket{0}
\end{equation}
%
\begin{equation}\label{eqn:usefulFormulas:1180}
\bra{z} a a^\dagger \ket{z} = \bra{z} 1 + a^\dagger a \ket{z}
\end{equation}
%
\begin{dmath}\label{eqn:usefulFormulas:1360}
\begin{aligned}
\expectation{x}(0) &= x_0 \equiv \sqrt{\frac{2 \Hbar}{m \omega}} \Real z = \sqrt{\frac{\Hbar}{2 m \omega}} \lr{ z + z^\conj } \\
\expectation{p}(0) &= p_0 \equiv \sqrt{2 m \Hbar \omega} \Imag z = -i \sqrt{\frac{m \Hbar \omega}{2}} \lr{ z - z^\conj }
\end{aligned}
\end{dmath}

\paragraph{Electromagnetism}
%
\begin{equation}\label{eqn:usefulFormulas:1200}
(\BA', \phi') = ( \BA + \spacegrad \chi, \phi - \partial_t \chi )
\end{equation}
%
\begin{equation}\label{eqn:usefulFormulas:1580}
\begin{aligned}
\BE &= - \PD{t}{\BA} - \spacegrad \phi \\
\BB &= \spacegrad \cross \BA
\end{aligned}
\end{equation}
%
\begin{dmath}\label{eqn:usefulFormulas:1340}
\BA =
\left\{
\begin{array}{l l}
\frac{B \rho_a^2}{2 \rho} \phicap & \quad \mbox{if \( \rho \ge \rho_a \) } \implies \BB = 0 \\
\frac{B \rho}{2} \phicap & \quad \mbox{if \( \rho < \rho_a \)} \implies \BB = B \zcap
\end{array}
\right.
\end{dmath}
%
\begin{equation}\label{eqn:usefulFormulas:1400}
\Phi \equiv \oint \BA \cdot d\Bl
\end{equation}

\paragraph{Aharonov-Bohm and Magnetic fields}
%
\begin{equation}\label{eqn:usefulFormulas:1380}
H = \inv{2m} \lr{ \Bp - \frac{q}{c} \BA } + q \phi
\end{equation}
%
\begin{equation}\label{eqn:usefulFormulas:1480}
\begin{aligned}
\BPi &= \Bp - \frac{e}{c} \BA \\
\antisymmetric{\Pi_x}{\Pi_y} &= \frac{i e \Hbar}{c} B_z \\
H &= \inv{2 m} \BPi^2 = \Hbar \omega \lr{ b^\dagger b + \inv{2} } \\
\omega &= \frac{e B_0}{m c} \\
b &= \inv{ \sqrt{ 2 m \omega \Hbar }} \lr{ \Pi_x + i \Pi_y }
\end{aligned}
\end{equation}
%
\begin{equation}\label{eqn:usefulFormulas:1500}
\BA = \frac{B_0}{2} \lr{ -y, x, 0}, \BA = B_0 \lr{ 0, x, 0 } \implies \BB = B_0 \zcap
\end{equation}

\paragraph{Dirac equation}
%
\begin{equation}\label{eqn:usefulFormulas:1240}
H =
\begin{bmatrix}
\hatp c + V & m c^2 \\
m c^2 & - \hatp c + V
\end{bmatrix}
=
\hatp c \sigma_z + m c^2 \sigma_x + V
\end{equation}
%
\begin{equation}\label{eqn:usefulFormulas:1260}
\begin{aligned}
\psi &= e^{\pm i k x - i E t/\Hbar} \\
\ket{ +k ; \pm \epsilon } &=
\begin{bmatrix}
\cos\theta \\
\sin\theta
\end{bmatrix},
\begin{bmatrix}
-\sin\theta \\
\cos\theta \\
\end{bmatrix} \\
\ket{ -k ; \pm \epsilon } &=
\begin{bmatrix}
\sin\theta \\
\cos\theta \\
\end{bmatrix},
\begin{bmatrix}
-\cos\theta \\
\sin\theta \\
\end{bmatrix} \\
\tan( 2 \theta ) &= \ifrac{m c}{ \Hbar \Abs{k} } \\
\epsilon^2 &= \lr{ m c^2 }^2 + \lr{ \Hbar k c }^2
\end{aligned}
\end{equation}
%
\begin{equation}\label{eqn:usefulFormulas:1280}
\begin{aligned}
\psi &= e^{\mp k x - i E t/\Hbar} \\
\ket{\pm} &= \inv{\sqrt{2}}
\begin{bmatrix}
\pm e^{\pm i \phi/2} \\
e^{\mp i \phi/2}
\end{bmatrix} \\
\epsilon^2 &= \lr{ m c^2 }^2 - \lr{ \Hbar k c }^2 \\
e^{i \phi} &= \frac{ \epsilon \pm i k \Hbar c }{ m c^2 }
\end{aligned}
\end{equation}
%
\begin{equation}\label{eqn:usefulFormulas:1300}
j = c \Psi^\dagger \sigma_z \Psi
\end{equation}

\paragraph{Generators}
%
\begin{equation}\label{eqn:usefulFormulas:1420}
T_\Ba = e^{-i \Bp \cdot \Ba /\Hbar }
\end{equation}
%
\begin{equation}\label{eqn:usefulFormulas:1440}
T_{\tilde{\Bp}} = e^{i \tilde{\Bp} \cdot \Bx /\Hbar }
\end{equation}
%
\begin{dmath}\label{eqn:usefulFormulas:2500}
\calD(\ncap, \phi) = \exp\lr{ -i \BJ \cdot \ncap \phi/\Hbar }
\end{dmath}

\paragraph{Calculus}
%
\begin{equation}\label{eqn:usefulFormulas:1520}
\spacegrad^2 \psi = \inv{\rho} \partial_\rho \lr{ \rho \partial_\rho \psi } + \inv{\rho^2} \partial_{\phi\phi} \psi + \partial_{z z} \psi
\end{equation}
%
\begin{equation}\label{eqn:usefulFormulas:2520}
\spacegrad^2 \psi = \inv{r^2} \partial_r \lr{ r^2 \partial_r \psi } + \inv{r^2 \sin\theta} \partial_{\theta} \lr{ \sin\theta \partial_\theta \psi } + \inv{r^2 \sin^2\theta} \partial_{\phi \phi}
\end{equation}
%
\begin{equation}\label{eqn:usefulFormulas:1600}
\int_{-\infty}^\infty \exp( a x^2 ) dx = \sqrt{\frac{-\pi}{a}}
\end{equation}
%
\begin{equation}\label{eqn:usefulFormulas:1620}
\Gamma(t) = \int_0^\infty x^{t-1} e^{-x} dx = (t-1)!
\end{equation}
%
\begin{equation}\label{eqn:usefulFormulas:1940}
\int_0^\infty e^{-\alpha r} r^n dr = \frac{n!}{\alpha^{n+1}}
\end{equation}
%
\begin{dmath}\label{eqn:usefulFormulas:2540}
\lim_{N\rightarrow \infty} \lr{ 1 + \frac{x}{N} }^N = e^x
\end{dmath}

\paragraph{Symmetries}
%
\begin{dmath}\label{eqn:usefulFormulas:2560}
\pi^\dagger = \pi = \pi^{-1}
\end{dmath}
\begin{dmath}\label{eqn:usefulFormulas:2260}
\pi^\dagger \Bx \pi = -\Bx
\end{dmath}
\begin{dmath}\label{eqn:usefulFormulas:2280}
\pi^\dagger \Bp \pi = -\Bp
\end{dmath}
\begin{equation}\label{eqn:usefulFormulas:2301}
\pi \ket{\Bx} = \ket{-\Bx}
\end{equation}
\begin{equation}\label{eqn:usefulFormulas:2300}
\bra{\Bx} \pi \ket{\psi} = \braket{-\Bx}{\psi} = \psi(-\Bx)
\end{equation}
\begin{dmath}\label{eqn:usefulFormulas:2320}
\antisymmetric{\pi}{\BJ} = 0
\end{dmath}
%
\begin{dmath}\label{eqn:usefulFormulas:2340}
\Theta H = H \Theta
\end{dmath}
\begin{dmath}\label{eqn:usefulFormulas:2360}
\Theta \Bp \Theta^{-1} = - \Bp
\end{dmath}
\begin{dmath}\label{eqn:usefulFormulas:2380}
\Theta \BJ \Theta^{-1} = - \BJ
\end{dmath}
\begin{dmath}\label{eqn:usefulFormulas:2400}
\Theta \Bx \Theta^{-1} = \Bx
\end{dmath}
\begin{dmath}\label{eqn:usefulFormulas:2480}
\bra{\Bx}\Theta\ket{\alpha} = \braket{\alpha}{\Bx}
\end{dmath}
\begin{dmath}\label{eqn:usefulFormulas:2420}
\Theta \ket{j, m} = i^{2m} \ket{j,-m}
\end{dmath}
\begin{dmath}\label{eqn:usefulFormulas:2440}
\Theta^2 = (-1)^{2j}
\end{dmath}
\begin{dmath}\label{eqn:usefulFormulas:2460}
\begin{aligned}
\Theta &= -i \sigma_y \eta K, \qquad \eta = i \\
K i &= -i \\
\Theta &\ket{\ncap ; +} = \eta \ket{\ncap ; -} \\
\Theta &\ket{\ncap ; -} = -\eta \ket{\ncap ; +}
\end{aligned}
\end{dmath}

\paragraph{Spin}
%
\begin{dmath}\label{eqn:usefulFormulas:2000}
\antisymmetric{J_x}{J_y} = i \Hbar J_z
\end{dmath}
%
\begin{dmath}\label{eqn:usefulFormulas:2020}
\antisymmetric{J_r}{J_s} = i \Hbar \epsilon_{r s t} J_t
\end{dmath}
%
\begin{dmath}\label{eqn:usefulFormulas:2040}
\antisymmetric{\BJ^2}{J_{\pm}} = 0
\end{dmath}
%
\begin{dmath}\label{eqn:usefulFormulas:2060}
\antisymmetric{J_z}{J_{\pm}} = \pm \Hbar J_{\pm}
\end{dmath}
%
\begin{dmath}\label{eqn:usefulFormulas:2080}
\antisymmetric{J_{+}}{J_{-}} = 2 \Hbar J_z
\end{dmath}
%
\begin{dmath}\label{eqn:usefulFormulas:2100}
\BJ = \BL \otimes 1 + 1 \otimes \BS = \BL + \BS
\end{dmath}
%
\begin{dmath}\label{eqn:usefulFormulas:2200}
J_z \ket{j,m} = m \Hbar \ket{j, m}
\end{dmath}
%
\begin{dmath}\label{eqn:usefulFormulas:2220}
\BJ^2 \ket{j,m} = j(j+1) \Hbar^2 \ket{j, m}
\end{dmath}
%
\begin{dmath}\label{eqn:usefulFormulas:2240}
J_{\pm} \ket{j,m} = \Hbar \sqrt{ (j \mp m)(j \pm m + 1)} \ket{j, m}
\end{dmath}
%
\begin{dmath}\label{eqn:usefulFormulas:2120}
\calD(R) =
\exp\lr{ -i \BL \cdot \ncap \phi/\Hbar }
\otimes
\exp\lr{ -i \BS \cdot \ncap \phi/\Hbar }
\end{dmath}

\paragraph{Spin one representation}
%
\begin{dmath}\label{eqn:usefulFormulas:2140}
J_x
=
\frac{\Hbar}{\sqrt{2}}
\begin{bmatrix}
0 & 1 & 0 \\
1 & 0 & 1 \\
0 & 1 & 0 \\
\end{bmatrix}
\end{dmath}
%
\begin{dmath}\label{eqn:usefulFormulas:2160}
J_y
=
\frac{\Hbar i}{\sqrt{2}}
\begin{bmatrix}
0 & -1 &  0 \\
1 &  0 & -1 \\
0 &  1 &  0 \\
\end{bmatrix}
\end{dmath}
%
\begin{dmath}\label{eqn:usefulFormulas:2180}
J_z
=\Hbar
\begin{bmatrix}
1 & 0 & 0 \\
0 & 0 & 0 \\
0 & 0 & -1 \\
\end{bmatrix}
\end{dmath}
\paragraph{Angular momentum}
%
\begin{dmath}\label{eqn:usefulFormulas:1840}
\BL^2
= L_z^2 + \inv{2}\lr{ L_{+}L_{-} + L_{-}L_{+}}
=
-\Hbar^2 \lr{ \inv{\sin^2\theta} \partial_{\phi\phi} + \inv{\sin\theta} \partial_\theta \lr{ \sin\theta \partial_\theta} }
= \Bx^2 \Bp^2 - (\Bx \cdot \Bp)^2 + i \Hbar \Bx \cdot \Bp
= r^2 \Bp^2 + \Hbar^2 \lr{ r^2 \partial_{rr} + 2 r \partial_r }
\end{dmath}
%
\begin{equation}\label{eqn:usefulFormulas:1860}
\begin{aligned}
L_x &= - i\Hbar \lr{ -\sin\phi \partial_\theta - \cot\theta \cos\phi \partial_\phi} \\
L_y &= - i\Hbar \lr{ \cos\phi \partial_\theta - \cot\theta \sin\phi \partial_\phi} \\
L_z &= -i \Hbar \partial_\phi
\end{aligned}
\end{equation}
%
\begin{dmath}\label{eqn:usefulFormulas:1980}
\antisymmetric{L_x}{L_y} = i \Hbar L_z
\end{dmath}
%
\begin{equation}\label{eqn:usefulFormulas:1981}
\symmetric{L_{+}}{L_{-}} = 2(L_x^2 + L_y^2) = 2( \BL^2 - L_z^2 )
\end{equation}
%
\begin{equation}\label{eqn:usefulFormulas:1880}
L_{\pm} = -i \Hbar e^{\pm i \phi} \lr{ \pm i \partial_\theta - \cot\theta \partial_\phi } = L_x \pm i L_y
\end{equation}
%
\begin{equation}\label{eqn:usefulFormulas:1881}
L_{\pm} \ket{l,m} =\Hbar \sqrt{(l \mp m)(l \pm m + 1)} \ket{l, m \pm 1}
\end{equation}
%
\begin{equation}\label{eqn:usefulFormulas:1900}
\BL^2 \ket{l, m} = l (l+1) \Hbar^2 \ket{l,m}
\end{equation}
\begin{equation}\label{eqn:usefulFormulas:1920}
L_z \ket{l, m} = m \Hbar \ket{l,m}
\end{equation}

\paragraph{Spherical harmonics}
%
\begin{equation}\label{eqn:qmLecture21:760}
\begin{aligned}
Y_{00} &= \frac{1}{2 \sqrt{\pi}} \\
Y_{10} &= \frac{1}{2} \sqrt{\frac{3}{\pi }} \cos(\theta) \\
Y_{20} &= \frac{1}{4} \sqrt{\frac{5}{\pi }} \lr{3 \cos^2(\theta)-1} \\
Y_{30} &= \frac{1}{4} \sqrt{\frac{7}{\pi }} \lr{5 \cos^3(\theta)-3 \cos(\theta)} \\
Y_{40} &= \frac{3 \lr{(35 \cos^4(\theta)-30 \cos^2(\theta)+3}}{16 \sqrt{\pi }} \\
Y_{50} &= \frac{1}{16} \sqrt{\frac{11}{\pi }} \lr{63 \cos^5(\theta)-70 \cos^3(\theta)+15 \cos(\theta)} \\
Y_{60} &= \frac{1}{32} \sqrt{\frac{13}{\pi }} \lr{231 \cos^6(\theta)-315 \cos^4(\theta)+105 \cos^2(\theta)-5} \\
Y_{70} &= \frac{1}{32} \sqrt{\frac{15}{\pi }} \lr{429 \cos^7(\theta)-693 \cos^5(\theta)+315 \cos^3(\theta)-35 \cos(\theta)},
%Y_{80} &= \frac{1}{256} \sqrt{\frac{17}{\pi }} \biglr{6435 \cos^8(\theta)-12012 \cos^6(\theta) \\
%&\qquad +6930 \cos^4(\theta)-1260 \cos^2(\theta)+35 } \\
\end{aligned}
\end{equation}
%
\begin{dmath}\label{eqn:usefulFormulas:2580}
Y_{l,-m} = (-1)^m Y_{lm}^\conj
\end{dmath}
%FIXME: complete this table.
\paragraph{Hydrogen wavefunctions}

% URL referenced is gone.  New versions from mit physical chemistry 2017 seem to be different.
%From \citep{hydrogenWavefunctionsMIT}, with the \( a_0 \) factors added in.
\begin{subequations}
\label{eqn:qmLecture21:640}
\begin{equation}\label{eqn:qmLecture21:660}
\psi_{1 s} = \psi_{100} = \inv{\sqrt{\pi}} \lr{ \frac{Z}{a_0} }^{3/2} e^{- Z r/a_0}
\end{equation}
\begin{equation}\label{eqn:qmLecture21:680}
\psi_{2 s} = \psi_{200} = \inv{4 \sqrt{2 \pi}} \lr{ \frac{Z}{a_0} }^{3/2} \lr{ 2 - \frac{r Z}{a_0} } e^{-Z r/2a_0}
\end{equation}
\begin{equation}\label{eqn:qmLecture21:700}
\psi_{2 p_x} = \frac{1}{\sqrt{2}} \lr{ \psi_{2,1,-1} - \psi_{2,1,1} }
= \inv{4 \sqrt{2 \pi}} \lr{ \frac{Z}{a_0} }^{3/2} \frac{rZ}{a_0} e^{-rZ/2a_0} \sin\theta\cos\phi
\end{equation}
\begin{equation}\label{eqn:qmLecture21:720}
\psi_{2 p_y} = \frac{i}{\sqrt{2}} \lr{ \psi_{2,1,-1} + \psi_{2,1,1} }
= \inv{4 \sqrt{2 \pi}} \lr{ \frac{Z}{a_0} }^{3/2} \frac{r Z }{a_0} e^{-rZ/2a_0} \sin\theta\sin\phi
\end{equation}
\begin{equation}\label{eqn:qmLecture21:740}
\psi_{2 p_z} = \psi_{210} = \inv{4 \sqrt{2 \pi}} \lr{ \frac{Z}{a_0} }^{3/2} \frac{r Z}{a_0} e^{-r Z/2a_0} \cos\theta
\end{equation}
\end{subequations}

I looked to \citep{hydrogenWavefunctionsHyperPhysics} to see where to add in the \( a_0 \) factors.

Energy levels are \( n \) dependent only
%
\begin{dmath}\label{eqn:usefulFormulas:1820}
E_n = - \frac{Z^2 e^2}{2 n^2 a_0}
\end{dmath}
%
\begin{equation}\label{eqn:usefulFormulas:1960}
a_0 = \frac{4 \pi \epsilon_0 \Hbar^2}{m e^2}
\end{equation}

\paragraph{Perturbation}

\begin{subequations}
\label{eqn:usefulFormulas:1680}
\begin{equation}\label{eqn:usefulFormulas:1700}
H = H_0 + \lambda V
\end{equation}
\begin{equation}\label{eqn:usefulFormulas:1720}
\ket{n} = \sum_{j=0} \lambda^j \ket{n_j}
\end{equation}
\begin{equation}\label{eqn:usefulFormulas:1740}
\Delta_{n} = \sum_{j=1} \lambda^j \Delta_{n_j}
\end{equation}
\begin{equation}\label{eqn:usefulFormulas:1760}
\overbar{P}_n = \sum_{m \ne n} \ket{m}\bra{m}
\end{equation}
\end{subequations}
%
\begin{equation}\label{eqn:usefulFormulas:1780}
\begin{aligned}
\ket{n_0} &= \ket{n^{(0)}} \\
\ket{n_1} &= \frac{\overbar{P}_n}{E_n^{(0)} - H_0} V \ket{n_0} \\
\ket{n_2} &= \frac{\overbar{P}_n}{E_n^{(0)} - H_0} \lr{ V - \Delta_{n_1} } \ket{n_1} \\
\ket{n_3} &= \frac{\overbar{P}_n}{E_n^{(0)} - H_0} \lr{ V \ket{n_2} - \Delta_{n_1} \ket{n_2} - \Delta_{n_2} \ket{n_1} } \\
%\ket{n_4} &= \frac{\overbar{P}_n}{E_n^{(0)} - H_0} \lr{ V \ket{n_3} - \Delta_{n_1} \ket{n_3} - \Delta_{n_2} \ket{n_2} - \Delta_{n_3} \ket{n_1} } \\
\ket{n_j} &= \frac{\overbar{P}_n}{E_n^{(0)} - H_0} \lr{ V \ket{n_{j-1}} - \sum_{k = 1}^{j-1} \Delta_{n_k} \ket{n_{j-k}} } \\
\end{aligned}
\end{equation}
%
\begin{equation}\label{eqn:usefulFormulas:1800}
\Delta_{n_j} = \bra{n_0} V \ket{n_{j-1}}
\end{equation}
%
\begin{dmath}\label{eqn:usefulFormulas:2600}
f(\Abs{x}) = \Theta(x) f(x) + \Theta(-x) f(-x)
\end{dmath}
%
\begin{dmath}\label{eqn:usefulFormulas:2620}
\frac{d}{dx} \delta(x) = -\frac{\delta(x)}{x}
\end{dmath}

\paragraph{Clebsch Ex. \( l_1 = 2 \), \( l_2 = 1 \)}
%
\begin{equation}\label{eqn:usefulFormulas:2640}
\begin{aligned}
\ket{3,3} &= \ket{2,2} \otimes \ket{1,1} \\
\ket{3,2} &= \inv{\bra{3,2} L_{-} \ket{3,3} } \lr{ (L_{-} \ket{2,2}) \otimes \ket{1,1} + \ket{2,2} \otimes L_{-} \ket{1,1} } \\
\ket{2,2} &: a\ket{1} + b\ket{2} \rightarrow -b\ket{1} + a\ket{2} \\
\ket{1,1} &= \ket{3,1} \cross \ket{2,1}
\end{aligned}
\end{equation}
